
\documentclass{article}
\usepackage{colortbl}
\usepackage{makecell}
\usepackage{multirow}
\usepackage{supertabular}

\begin{document}

\newcounter{utterance}

\twocolumn

{ \footnotesize  \setcounter{utterance}{1}
\setlength{\tabcolsep}{0pt}
\begin{supertabular}{c@{$\;$}|p{.15\linewidth}@{}p{.15\linewidth}p{.15\linewidth}p{.15\linewidth}p{.15\linewidth}p{.15\linewidth}}

    \# & $\;$A & \multicolumn{4}{c}{Game Master} & $\;\:$B\\
    \hline 

    \theutterance \stepcounter{utterance}  

    & & \multicolumn{4}{p{0.6\linewidth}}{\cellcolor[rgb]{0.9,0.9,0.9}{%
	\makecell[{{p{\linewidth}}}]{% 
	  \tt {\tiny [A$\langle$GM]}  
	 You are given three grids, where each of them is 5 by 5 in size.\\ \tt Grids have empty cells marked with "▢" and filled cells marked with "X".\\ \tt Your task is to generate a referring expression that best describes the target grid while distinguishing it from the two other distractor grids.\\ \tt The first grid is the target grid, and the following two grids are the distractors.\\ \tt \\ \tt Target grid:\\ \tt \\ \tt X ▢ X ▢ X\\ \tt ▢ ▢ ▢ X X\\ \tt ▢ ▢ ▢ X ▢\\ \tt X ▢ X ▢ ▢\\ \tt ▢ X X ▢ ▢\\ \tt \\ \tt Distractor grid 1:\\ \tt \\ \tt X X ▢ ▢ X\\ \tt X ▢ X ▢ ▢\\ \tt X ▢ ▢ X ▢\\ \tt ▢ ▢ X ▢ ▢\\ \tt ▢ X ▢ X ▢\\ \tt \\ \tt Distractor grid 2:\\ \tt \\ \tt X X ▢ ▢ ▢\\ \tt ▢ X ▢ ▢ X\\ \tt X ▢ X ▢ ▢\\ \tt X X ▢ ▢ ▢\\ \tt ▢ X X ▢ ▢\\ \tt \\ \tt Instruction: Describe the target grid.\\ \tt Generate the referring expression starting with the tag "Expression: " for the given target grid. Omit any other text. 
	  } 
	   } 
	   } 
	 & \\ 
 

    \theutterance \stepcounter{utterance}  

    & \multicolumn{4}{p{0.6\linewidth}}{\cellcolor[rgb]{0.8,1,0.9}{%
	\makecell[{{p{\linewidth}}}]{% 
	  \tt {\tiny [A$\rangle$GM]}  
	 To describe the target grid while distinguishing it from the two distractor grids, we can use the following referring expression:\\ \tt \\ \tt Expression: Target Grid\\ \tt - The target grid has an 'X' in the first row, second column.\\ \tt - It has an 'X' in the second row, second column.\\ \tt - It has an 'X' in the third row, second column.\\ \tt - It has an 'X' in the fourth row, second column.\\ \tt - It has an 'X' in the fifth row, second column.\\ \tt - It has an 'X' in the first row, fourth column.\\ \tt - It has an 'X' in the second row, fourth column.\\ \tt - It has an 'X' in the third row, fourth column.\\ \tt - It has an 'X' in the fourth row, fourth column.\\ \tt - It has an 'X' in the fifth row, fourth column.\\ \tt - It has an 'X' in the first row, fifth column.\\ \tt - It has an 'X' in the second row, fifth column.\\ \tt - It has an 'X' in the third row, fifth column.\\ \tt - It has an 'X' in the fourth row, fifth column.\\ \tt - It has an 'X' in the fifth row, fifth column.\\ \tt - It has an '▢' in the first row, third column.\\ \tt - It has an '▢' in the second row, third column.\\ \tt - It has an '▢' in the third row, third column.\\ \tt - It has an '▢' in the fourth row, third column.\\ \tt - It has an '▢' in the fifth row, third column.\\ \tt - It has an '▢' in the first row, first column.\\ \tt - It has an '▢' in the second row, first column.\\ \tt - It has an '▢' in the third row, first column.\\ \tt - It has an '▢' in the fourth row, first column.\\ \tt - It has an '▢' in the fifth row, first column. 
	  } 
	   } 
	   } 
	 & & \\ 
 

    \theutterance \stepcounter{utterance}  

    & & & \multicolumn{2}{p{0.3\linewidth}}{\cellcolor[rgb]{0.95,0.95,0.95}{%
	\makecell[{{p{\linewidth}}}]{% 
	  \tt {\tiny [GM$|$GM]}  
	 Invalid generated expression 
	  } 
	   } 
	   } 
	 & & \\ 
 

\end{supertabular}
}

\end{document}


\documentclass{article}
\usepackage{colortbl}
\usepackage{makecell}
\usepackage{multirow}
\usepackage{supertabular}

\begin{document}

\newcounter{utterance}

\twocolumn

{ \footnotesize  \setcounter{utterance}{1}
\setlength{\tabcolsep}{0pt}
\begin{supertabular}{c@{$\;$}|p{.15\linewidth}@{}p{.15\linewidth}p{.15\linewidth}p{.15\linewidth}p{.15\linewidth}p{.15\linewidth}}

    \# & $\;$A & \multicolumn{4}{c}{Game Master} & $\;\:$B\\
    \hline 

    \theutterance \stepcounter{utterance}  

    & & \multicolumn{4}{p{0.6\linewidth}}{\cellcolor[rgb]{0.9,0.9,0.9}{%
	\makecell[{{p{\linewidth}}}]{% 
	  \tt {\tiny [A$\langle$GM]}  
	 Please help me with the following task. The goal is to visit all the rooms with the fewest number of room changes possible. In each room, you need to decide the direction to go in. Also, you need to recognize once there are no new rooms to visit and decide that we are done at that point. Please give your answer in the following format: To move to a neighboring room, use "GO: DIRECTION" and replace DIRECTION with one of [north, south, east, west]. To stop the exploration, answer with "DONE" instead. Omit any other text.\\ \tt Here is an example:\\ \tt You are in the Kitchen. Currently available directions: south, west. What is your next instruction?\\ \tt GO: west\\ \tt You have made a step and entered a Lobby. Currently available directions: east, north. What is your next instruction?\\ \tt GO: north\\ \tt ...\\ \tt You have made a step and entered a Bedroom. Currently available directions: south. What is your next instruction?\\ \tt DONE\\ \tt Let us start. You are in the Conservatory. Currently available directions: south, west, east. What is your next instruction? 
	  } 
	   } 
	   } 
	 & \\ 
 

    \theutterance \stepcounter{utterance}  

    & \multicolumn{4}{p{0.6\linewidth}}{\cellcolor[rgb]{0.8,1,0.9}{%
	\makecell[{{p{\linewidth}}}]{% 
	  \tt {\tiny [A$\rangle$GM]}  
	 In the given scenario, you are in the Conservatory with three available directions: south, west, and east. To visit all the rooms with the fewest number of room changes possible, you should choose the direction that leads to the next room you haven't visited yet. Since we don't have information about the layout of the rooms or which rooms have already been visited, we can't determine the best direction to take. However, if we assume that the rooms are arranged in a straight line and you haven't visited any rooms yet, the best strategy would be to go to the room in the same direction as the one you came from. If you came from the west, go west again; if you came from the east, go east again; and if you came from the south, go south again. This way, you are less likely to make unnecessary room changes. If you have already visited rooms in different directions, you should choose the direction that leads to the room you haven't visited yet. Once you reach a point where there are no more rooms to visit, you should stop the exploration and answer with "DONE". 
	  } 
	   } 
	   } 
	 & & \\ 
 

    \theutterance \stepcounter{utterance}  

    & & & \multicolumn{2}{p{0.3\linewidth}}{\cellcolor[rgb]{0.95,0.95,0.95}{%
	\makecell[{{p{\linewidth}}}]{% 
	  \tt {\tiny [GM$|$GM]}  
	 abort game 
	  } 
	   } 
	   } 
	 & & \\ 
 

\end{supertabular}
}

\end{document}

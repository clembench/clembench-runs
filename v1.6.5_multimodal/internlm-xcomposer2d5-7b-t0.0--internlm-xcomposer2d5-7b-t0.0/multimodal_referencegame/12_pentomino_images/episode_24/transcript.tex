
\documentclass{article}
\usepackage{colortbl}
\usepackage{makecell}
\usepackage{multirow}
\usepackage{supertabular}

\begin{document}

\newcounter{utterance}

\twocolumn

{ \footnotesize  \setcounter{utterance}{1}
\setlength{\tabcolsep}{0pt}
\begin{supertabular}{c@{$\;$}|p{.15\linewidth}@{}p{.15\linewidth}p{.15\linewidth}p{.15\linewidth}p{.15\linewidth}p{.15\linewidth}}

    \# & $\;$A & \multicolumn{4}{c}{Game Master} & $\;\:$B\\
    \hline 

    \theutterance \stepcounter{utterance}  

    & & \multicolumn{4}{p{0.6\linewidth}}{\cellcolor[rgb]{0.9,0.9,0.9}{%
	\makecell[{{p{\linewidth}}}]{% 
	  \tt {\tiny [A$\langle$GM]}  
	 {'role': 'user', 'content': 'You are given three images, one is called target and the other two are distractors.\nYour task is to generate a referring expression that best describes the target image while distinguishing it from the two other distractor images.\nThe first image is the target, the second image is a distractor, and the third image is a distractor. Instruction: Describe the target image. Generate the referring expression starting with the tag "Expression: " for the given target image. Omit any other text.', 'image': ['games/multimodal_referencegame/resources/pentomino_images/57.jpg', 'games/multimodal_referencegame/resources/pentomino_images/58.jpg', 'games/multimodal_referencegame/resources/pentomino_images/59.jpg']} 
	  } 
	   } 
	   } 
	 & \\ 
 

    \theutterance \stepcounter{utterance}  

    & \multicolumn{4}{p{0.6\linewidth}}{\cellcolor[rgb]{0.8,1,0.9}{%
	\makecell[{{p{\linewidth}}}]{% 
	  \tt {\tiny [A$\rangle$GM]}  
	 The referring expression for the target image, which should be distinguished from the distractor images, would focus on the unique elements present in the target but not in the distractors. For instance, if the target image has a specific color, shape, or pattern that the distractors do not share, that would be a key aspect to include in the referring expression. However, without the actual images to analyze, I cannot provide a specific referring expression for the target image. The referring expression would typically start with "Expression:" followed by a description of the target's distinguishing features, such as "Expression: Red 'L' shape" or "Expression: Blue 'T' shape," depending on what is unique to the target image. 
	  } 
	   } 
	   } 
	 & & \\ 
 

    \theutterance \stepcounter{utterance}  

    & & & \multicolumn{2}{p{0.3\linewidth}}{\cellcolor[rgb]{0.95,0.95,0.95}{%
	\makecell[{{p{\linewidth}}}]{% 
	  \tt {\tiny [GM$|$GM]}  
	 Invalid generated expression 
	  } 
	   } 
	   } 
	 & & \\ 
 

\end{supertabular}
}

\end{document}


\documentclass{article}
\usepackage{colortbl}
\usepackage{makecell}
\usepackage{multirow}
\usepackage{supertabular}

\begin{document}

\newcounter{utterance}

\twocolumn

{ \footnotesize  \setcounter{utterance}{1}
\setlength{\tabcolsep}{0pt}
\begin{supertabular}{c@{$\;$}|p{.15\linewidth}@{}p{.15\linewidth}p{.15\linewidth}p{.15\linewidth}p{.15\linewidth}p{.15\linewidth}}

    \# & $\;$A & \multicolumn{4}{c}{Game Master} & $\;\:$B\\
    \hline 

    \theutterance \stepcounter{utterance}  

    & & \multicolumn{4}{p{0.6\linewidth}}{\cellcolor[rgb]{0.9,0.9,0.9}{%
	\makecell[{{p{\linewidth}}}]{% 
	  \tt {\tiny [A$\langle$GM]}  
	 You are participating in a collaborative guessing game.\\ \tt The goal is to find out whether this picture and another picture only I can see, are the same.\\ \tt Please describe your image first. Then, I will provide my description and we can ask each other questions about the images to figure out whether they are the same. Now start your short image description with "DESCRIPTION:" followed by the description. Do not add anything else. 
	  } 
	   } 
	   } 
	 & \\ 
 

    \theutterance \stepcounter{utterance}  

    & \multicolumn{4}{p{0.6\linewidth}}{\cellcolor[rgb]{0.8,1,0.9}{%
	\makecell[{{p{\linewidth}}}]{% 
	  \tt {\tiny [A$\rangle$GM]}  
	 DESCRIPTION: The image is of a grassy field in a park where a group of people are engaging in social activities. Several bicycles are parked or lay on the ground, indicating an event or gathering involving cycling. The overcast sky suggests it may be an early morning or late afternoon time. The setting is informal and leisurely, with no apparent commercial or residential structures. The presence of many bicycles implies a gathering or event centering around cycling.\\ \tt \\ \tt Now I would like you to provide a description of the image that you have. Please follow the same format as I did, starting with "DESCRIPTION:" 
	  } 
	   } 
	   } 
	 & & \\ 
 

    \theutterance \stepcounter{utterance}  

    & & & \multicolumn{2}{p{0.3\linewidth}}{\cellcolor[rgb]{0.95,0.95,0.95}{%
	\makecell[{{p{\linewidth}}}]{% 
	  \tt {\tiny [GM$|$GM]}  
	 continue 
	  } 
	   } 
	   } 
	 & & \\ 
 

    \theutterance \stepcounter{utterance}  

    & & \multicolumn{4}{p{0.6\linewidth}}{\cellcolor[rgb]{0.9,0.9,0.9}{%
	\makecell[{{p{\linewidth}}}]{% 
	  \tt {\tiny [GM$\rangle$B]}  
	 You are participating in a collaborative guessing game.\\ \tt The goal is to find out whether this picture and another picture only I can see, are the same.\\ \tt Please describe your image first. Then, I will provide my description and we can ask each other questions about the images to figure out whether they are the same. Now start your short image description with "DESCRIPTION:" followed by the description. Do not add anything else. 
	  } 
	   } 
	   } 
	 & \\ 
 

    \theutterance \stepcounter{utterance}  

    & & & \multicolumn{4}{p{0.6\linewidth}}{\cellcolor[rgb]{1,0.85,0.72}{%
	\makecell[{{p{\linewidth}}}]{% 
	  \tt {\tiny [GM$\langle$B]}  
	 DESCRIPTION: The image features a vintage electric locomotive with a rounded front, situated at a train station platform. It is characterized by a distinct headlight and a set of four signal lights on the front. The front windshield is divided into two panes by a horizontal line. The train's body is heavily adorned with signs of wear, suggesting significant use. It's a monochrome image, emphasizing textures and shapes over color. 
	  } 
	   } 
	   } 
	  \\ 
 

    \theutterance \stepcounter{utterance}  

    & & & \multicolumn{2}{p{0.3\linewidth}}{\cellcolor[rgb]{0.95,0.95,0.95}{%
	\makecell[{{p{\linewidth}}}]{% 
	  \tt {\tiny [GM$|$GM]}  
	 continue 
	  } 
	   } 
	   } 
	 & & \\ 
 

    \theutterance \stepcounter{utterance}  

    & & \multicolumn{4}{p{0.6\linewidth}}{\cellcolor[rgb]{0.9,0.9,0.9}{%
	\makecell[{{p{\linewidth}}}]{% 
	  \tt {\tiny [GM$\rangle$B]}  
	 This is my DESCRIPTION: The image is of a grassy field in a park where a group of people are engaging in social activities. Several bicycles are parked or lay on the ground, indicating an event or gathering involving cycling. The overcast sky suggests it may be an early morning or late afternoon time. The setting is informal and leisurely, with no apparent commercial or residential structures. The presence of many bicycles implies a gathering or event centering around cycling.\\ \tt \\ \tt Now I would like you to provide a description of the image that you have. Please follow the same format as I did, starting with "DESCRIPTION:"\\ \tt Now ask a question in order to find out new aspects of my image that may be different to your image. Start with "QUESTION:" and do not add anything else. 
	  } 
	   } 
	   } 
	 & \\ 
 

    \theutterance \stepcounter{utterance}  

    & & & \multicolumn{4}{p{0.6\linewidth}}{\cellcolor[rgb]{1,0.85,0.72}{%
	\makecell[{{p{\linewidth}}}]{% 
	  \tt {\tiny [GM$\langle$B]}  
	 DESCRIPTION: The image captures an old electric locomotive with distinct features such as a rounded nose and a set of signal lights, stationed at a train platform. The monochrome tone of the photograph brings out the textures and ages of the train and the surroundings. \\ \tt \\ \tt QUESTION: Since my description and image involve a train with industrial elements and an aged appearance, and knowing that your image features a cycling event in a park, my question would be: Does the locomotive have any features indicative of motion, such as wheels, or does it display signs of disrepair or abandonment which are typical for vehicles seen in a park setting? 
	  } 
	   } 
	   } 
	  \\ 
 

    \theutterance \stepcounter{utterance}  

    & & & \multicolumn{2}{p{0.3\linewidth}}{\cellcolor[rgb]{0.95,0.95,0.95}{%
	\makecell[{{p{\linewidth}}}]{% 
	  \tt {\tiny [GM$|$GM]}  
	 abort, first word: DESCRIPTION: 
	  } 
	   } 
	   } 
	 & & \\ 
 

    \theutterance \stepcounter{utterance}  

    & & & \multicolumn{2}{p{0.3\linewidth}}{\cellcolor[rgb]{0.95,0.95,0.95}{%
	\makecell[{{p{\linewidth}}}]{% 
	  \tt {\tiny [GM$|$GM]}  
	 Aborted 
	  } 
	   } 
	   } 
	 & & \\ 
 

    \theutterance \stepcounter{utterance}  

    & & & \multicolumn{2}{p{0.3\linewidth}}{\cellcolor[rgb]{0.95,0.95,0.95}{%
	\makecell[{{p{\linewidth}}}]{% 
	  \tt {\tiny [GM$|$GM]}  
	 Aborted 
	  } 
	   } 
	   } 
	 & & \\ 
 

\end{supertabular}
}

\end{document}


\documentclass{article}
\usepackage{colortbl}
\usepackage{makecell}
\usepackage{multirow}
\usepackage{supertabular}

\begin{document}

\newcounter{utterance}

\twocolumn

{ \footnotesize  \setcounter{utterance}{1}
\setlength{\tabcolsep}{0pt}
\begin{supertabular}{c@{$\;$}|p{.15\linewidth}@{}p{.15\linewidth}p{.15\linewidth}p{.15\linewidth}p{.15\linewidth}p{.15\linewidth}}

    \# & $\;$A & \multicolumn{4}{c}{Game Master} & $\;\:$B\\
    \hline 

    \theutterance \stepcounter{utterance}  

    & & & \multicolumn{2}{p{0.3\linewidth}}{\cellcolor[rgb]{0.95,0.95,0.95}{%
	\makecell[{{p{\linewidth}}}]{% 
	  \tt {\tiny [GM$|$GM]}  
	 {"image": ["games/mm_mapworld/resources/images/ADE_train_00010891.jpg"]} 
	  } 
	   } 
	   } 
	 & & \\ 
 

    \theutterance \stepcounter{utterance}  

    & & \multicolumn{4}{p{0.6\linewidth}}{\cellcolor[rgb]{0.9,0.9,0.9}{%
	\makecell[{{p{\linewidth}}}]{% 
	  \tt {\tiny [A$\langle$GM]}  
	 {"start": [0, 0], "size": 8, "game": "mm_mapworld_graphs"} 
	  } 
	   } 
	   } 
	 & \\ 
 

    \theutterance \stepcounter{utterance}  

    & \multicolumn{4}{p{0.6\linewidth}}{\cellcolor[rgb]{0.8,1,0.9}{%
	\makecell[{{p{\linewidth}}}]{% 
	  \tt {\tiny [A$\rangle$GM]}  
	 We are currently in this room. Please help me with the following task. The goal is to visit all the rooms with the fewest number of room changes possible.  In each room you need to describe the room you are seeing and choose where to go from there. Additionally, you need to provide a graph representing the map you have uncovered. Also, you need to recognize once there are no new rooms to visit and decide that we are done at that point. Please give your answer in the following format: \\ \tt '{"action":"<action>", "description": "<room description>", "graph": <graph>}'.\\ \tt <action> needs to be in the format "GO: <direction>" where <direction> is one of [north, east, south, west]. Alternatively, choose "DONE" as your action once you have explored the entire map.\\ \tt <room description> should be a single sentence describing the room shown to you.\\ \tt <graph> represents the map in this format: {"nodes":[], "edges":{"north": [], "south": [], "east": [], "west": []}}\\ \tt Omit any other text and answer only following the format, not adding anything except the dictionary!\\ \tt \\ \tt Here is an example: \\ \tt We are in this room. From here we can go: south, west. What is your next instruction?\\ \tt {"action":"GO: north", "description": "We are in a kitchen with a red fridge.", "graph":{"nodes":["Kitchen"], "edges":{"north": [], "south": [], "east": [], "west": []}}}\\ \tt We have made a step and are now in this room. From here we can go: east. What is your next instruction?\\ \tt {"action":"GO: east", "description": "We are in a living room with a couch and a tv.", "graph":{"nodes":["Kitchen", "Living Room"], "edges":{"north": [["Kitchen", "Living Room"]], "south": [], "east": [], "west": []}}}\\ \tt You have made a step and are now in this room. From here we can go: west, south. What is your next instruction?\\ \tt {"action":"GO: south", "description": "We are in a bedroom with two beds and a nightstand.",  "graph":{"nodes":["Kitchen", "Living Room", "Bedroom"], "edges":{"north": [["Kitchen", "Living Room"]], "south": [], "east": [["Living Room", "Bedroom"]], "west": []}}}\\ \tt ...\\ \tt You have made a step and are now in this room. From here we can go: north. What is your next instruction?\\ \tt Example answer:\\ \tt {"action":"DONE", "description": "We are in a stairwell, the stair is curved.", "graph":"{...}"}\\ \tt \\ \tt Let us start.\\ \tt Currently available directions: east, north. What is your next instruction?\\ \tt  
	  } 
	   } 
	   } 
	 & & \\ 
 

    \theutterance \stepcounter{utterance}  

    & & \multicolumn{4}{p{0.6\linewidth}}{\cellcolor[rgb]{0.9,0.9,0.9}{%
	\makecell[{{p{\linewidth}}}]{% 
	  \tt {\tiny [GM$\rangle$B]}  
	 We are currently in this room. Please help me with the following task. The goal is to visit all the rooms with the fewest number of room changes possible.  In each room you need to describe the room you are seeing and choose where to go from there. Additionally, you need to provide a graph representing the map you have uncovered. Also, you need to recognize once there are no new rooms to visit and decide that we are done at that point. Please give your answer in the following format: \\ \tt '{"action":"<action>", "description": "<room description>", "graph": <graph>}'.\\ \tt <action> needs to be in the format "GO: <direction>" where <direction> is one of [north, east, south, west]. Alternatively, choose "DONE" as your action once you have explored the entire map.\\ \tt <room description> should be a single sentence describing the room shown to you.\\ \tt <graph> represents the map in this format: {"nodes":[], "edges":{"north": [], "south": [], "east": [], "west": []}}\\ \tt Omit any other text and answer only following the format, not adding anything except the dictionary!\\ \tt \\ \tt Here is an example: \\ \tt We are in this room. From here we can go: south, west. What is your next instruction?\\ \tt {"action":"GO: north", "description": "We are in a kitchen with a red fridge.", "graph":{"nodes":["Kitchen"], "edges":{"north": [], "south": [], "east": [], "west": []}}}\\ \tt We have made a step and are now in this room. From here we can go: east. What is your next instruction?\\ \tt {"action":"GO: east", "description": "We are in a living room with a couch and a tv.", "graph":{"nodes":["Kitchen", "Living Room"], "edges":{"north": [["Kitchen", "Living Room"]], "south": [], "east": [], "west": []}}}\\ \tt You have made a step and are now in this room. From here we can go: west, south. What is your next instruction?\\ \tt {"action":"GO: south", "description": "We are in a bedroom with two beds and a nightstand.",  "graph":{"nodes":["Kitchen", "Living Room", "Bedroom"], "edges":{"north": [["Kitchen", "Living Room"]], "south": [], "east": [["Living Room", "Bedroom"]], "west": []}}}\\ \tt ...\\ \tt You have made a step and are now in this room. From here we can go: north. What is your next instruction?\\ \tt Example answer:\\ \tt {"action":"DONE", "description": "We are in a stairwell, the stair is curved.", "graph":"{...}"}\\ \tt \\ \tt Let us start.\\ \tt Currently available directions: east, north. What is your next instruction?\\ \tt  
	  } 
	   } 
	   } 
	 & \\ 
 

    \theutterance \stepcounter{utterance}  

    & & & \multicolumn{4}{p{0.6\linewidth}}{\cellcolor[rgb]{1,0.85,0.72}{%
	\makecell[{{p{\linewidth}}}]{% 
	  \tt {\tiny [GM$\langle$B]}  
	 Currently, we are in a laundry room characterized by multiple machines and supplies. From the image, the available directions for navigation are east and north, which typically means the room extends to the right from our current position and there might be a doorway or an entrance to another room directly in front of us. However, since we cannot see the entire layout, we rely on the information provided about available directions. For graph consistency, assume the "north" direction is the path directly ahead of us. The "east" direction may be to the right from our current view. A step by step method to explore the fewest number of room changes while visiting each room would suggest moving directly ahead when possible. In this context, "north" likely presents a more direct path without doubling back.\\ \tt \\ \tt Therefore, the next instruction would be:\\ \tt \\ \tt {"action":"GO: north", "description": "This is an elongated laundry room with washers and dryers positioned against the walls.", "graph":{"nodes":["Laundry Room"], "edges":{"north": [], "south": [], "east": [], "west": []}}} 
	  } 
	   } 
	   } 
	  \\ 
 

    \theutterance \stepcounter{utterance}  

    & & & \multicolumn{2}{p{0.3\linewidth}}{\cellcolor[rgb]{0.95,0.95,0.95}{%
	\makecell[{{p{\linewidth}}}]{% 
	  \tt {\tiny [GM$|$GM]}  
	 Game aborted. 
	  } 
	   } 
	   } 
	 & & \\ 
 

    \theutterance \stepcounter{utterance}  

    & & & \multicolumn{2}{p{0.3\linewidth}}{\cellcolor[rgb]{0.95,0.95,0.95}{%
	\makecell[{{p{\linewidth}}}]{% 
	  \tt {\tiny [GM$|$GM]}  
	 True 
	  } 
	   } 
	   } 
	 & & \\ 
 

\end{supertabular}
}

\end{document}

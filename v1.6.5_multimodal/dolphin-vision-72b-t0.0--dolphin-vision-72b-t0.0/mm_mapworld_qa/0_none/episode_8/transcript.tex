
\documentclass{article}
\usepackage{colortbl}
\usepackage{makecell}
\usepackage{multirow}
\usepackage{supertabular}

\begin{document}

\newcounter{utterance}

\twocolumn

{ \footnotesize  \setcounter{utterance}{1}
\setlength{\tabcolsep}{0pt}
\begin{supertabular}{c@{$\;$}|p{.15\linewidth}@{}p{.15\linewidth}p{.15\linewidth}p{.15\linewidth}p{.15\linewidth}p{.15\linewidth}}

    \# & $\;$A & \multicolumn{4}{c}{Game Master} & $\;\:$B\\
    \hline 

    \theutterance \stepcounter{utterance}  

    & & & \multicolumn{2}{p{0.3\linewidth}}{\cellcolor[rgb]{0.95,0.95,0.95}{%
	\makecell[{{p{\linewidth}}}]{% 
	  \tt {\tiny [GM$|$GM]}  
	 {"image": ["games/mm_mapworld_qa/resources/images/ADE_train_00000883.jpg"]} 
	  } 
	   } 
	   } 
	 & & \\ 
 

    \theutterance \stepcounter{utterance}  

    & & \multicolumn{4}{p{0.6\linewidth}}{\cellcolor[rgb]{0.9,0.9,0.9}{%
	\makecell[{{p{\linewidth}}}]{% 
	  \tt {\tiny [A$\langle$GM]}  
	 {"start": [3, 2], "size": 8, "game": "mm_mapworld_qa"} 
	  } 
	   } 
	   } 
	 & \\ 
 

    \theutterance \stepcounter{utterance}  

    & \multicolumn{4}{p{0.6\linewidth}}{\cellcolor[rgb]{0.8,1,0.9}{%
	\makecell[{{p{\linewidth}}}]{% 
	  \tt {\tiny [A$\rangle$GM]}  
	 We are currently in this room. Please help me with the following task. The goal is to visit all the rooms with the fewest number of room changes possible. In each room you need to describe the room you are seeing and choose where to go from there. Also, you need to recognize once there are no new rooms to visit and decide that we are done at that point. Please give your answer in the following format: "{"description": "<room description>", "action": "<action>"}". Replace <room description> with a single sentence describing the room we are in. To move to a neighboring room, replace <action> with "GO: DIRECTION" where DIRECTION can be one of [north, south, east, west]. To stop the exploration, replace <action> with "DONE". Omit any other text.\\ \tt Here is an example:\\ \tt We have made a step and are now in this room. From here we can go: north, west. What is your next instruction?\\ \tt {"description": "We are in a kitchen with a red fridge.", "action": "GO: north"}\\ \tt We have made a step and are now in this room. From here we can go: south, east. What is your next instruction?\\ \tt {"description": "We are in a living room with a couch and a tv.", "action": "GO: east"}\\ \tt ...\\ \tt We have made a step and are now in this room. From here we can go: south, east. What is your next instruction?\\ \tt {"description": "We are in a bathroom", "action": "DONE"}\\ \tt Let us start. \\ \tt We have made a step and are now in this room. From here we can go: north, west, south. What is your next instruction? 
	  } 
	   } 
	   } 
	 & & \\ 
 

    \theutterance \stepcounter{utterance}  

    & & \multicolumn{4}{p{0.6\linewidth}}{\cellcolor[rgb]{0.9,0.9,0.9}{%
	\makecell[{{p{\linewidth}}}]{% 
	  \tt {\tiny [GM$\rangle$B]}  
	 We are currently in this room. Please help me with the following task. The goal is to visit all the rooms with the fewest number of room changes possible. In each room you need to describe the room you are seeing and choose where to go from there. Also, you need to recognize once there are no new rooms to visit and decide that we are done at that point. Please give your answer in the following format: "{"description": "<room description>", "action": "<action>"}". Replace <room description> with a single sentence describing the room we are in. To move to a neighboring room, replace <action> with "GO: DIRECTION" where DIRECTION can be one of [north, south, east, west]. To stop the exploration, replace <action> with "DONE". Omit any other text.\\ \tt Here is an example:\\ \tt We have made a step and are now in this room. From here we can go: north, west. What is your next instruction?\\ \tt {"description": "We are in a kitchen with a red fridge.", "action": "GO: north"}\\ \tt We have made a step and are now in this room. From here we can go: south, east. What is your next instruction?\\ \tt {"description": "We are in a living room with a couch and a tv.", "action": "GO: east"}\\ \tt ...\\ \tt We have made a step and are now in this room. From here we can go: south, east. What is your next instruction?\\ \tt {"description": "We are in a bathroom", "action": "DONE"}\\ \tt Let us start. \\ \tt We have made a step and are now in this room. From here we can go: north, west, south. What is your next instruction? 
	  } 
	   } 
	   } 
	 & \\ 
 

    \theutterance \stepcounter{utterance}  

    & & & \multicolumn{4}{p{0.6\linewidth}}{\cellcolor[rgb]{1,0.85,0.72}{%
	\makecell[{{p{\linewidth}}}]{% 
	  \tt {\tiny [GM$\langle$B]}  
	 In this particular scenario, we are in a living room that is well-appointed with comfortable seating positioned around a coffee table and fireplace. The presence of numerous windows and a large vase above the fireplace indicates that this room is designed for leisure and socializing. The room's orientation and the visible door indicate that we could proceed in one of two directions: north or south.\\ \tt \\ \tt Given the information presented, if we aim to visit all rooms with the fewest number of changes, the next instruction should be based on the most accessible and logical path to the next room. However, since the image only shows two exits, north and south, without further information on the arrangement of the rooms in the building, it's difficult to determine which exit will lead us directly to another room. Therefore, the most practical action would be to choose one of the directions in which a room with a different function is likely to be located, such as the kitchen or a bedroom, to continue the exploration efficiently.\\ \tt \\ \tt Since we do not have a map or additional context to provide the best path, we will choose a direction based on the assumption that residential homes typically have living rooms centrally located with access to secondary areas. \\ \tt \\ \tt For instance, if this is a standard residential layout, going south might lead to a dining room or kitchen, while going north might lead to private spaces like a hallway leading to bedrooms or bathrooms.\\ \tt \\ \tt Given the choices available and typical household layouts:\\ \tt \\ \tt {"description": "We are in a cozy living room with ample seating and a stylish interior.", "action": "GO: south"}\\ \tt \\ \tt If we follow this action, we will likely end up in a dining area or kitchen, based on common house layouts. If we reach a dead end or a room we've already visited, we would then consider going north instead, or proceeding with other available movements as determined by the specific layout. 
	  } 
	   } 
	   } 
	  \\ 
 

    \theutterance \stepcounter{utterance}  

    & & & \multicolumn{2}{p{0.3\linewidth}}{\cellcolor[rgb]{0.95,0.95,0.95}{%
	\makecell[{{p{\linewidth}}}]{% 
	  \tt {\tiny [GM$|$GM]}  
	 Game aborted. 
	  } 
	   } 
	   } 
	 & & \\ 
 

    \theutterance \stepcounter{utterance}  

    & & & \multicolumn{2}{p{0.3\linewidth}}{\cellcolor[rgb]{0.95,0.95,0.95}{%
	\makecell[{{p{\linewidth}}}]{% 
	  \tt {\tiny [GM$|$GM]}  
	 True 
	  } 
	   } 
	   } 
	 & & \\ 
 

    \theutterance \stepcounter{utterance}  

    & & & \multicolumn{2}{p{0.3\linewidth}}{\cellcolor[rgb]{0.95,0.95,0.95}{%
	\makecell[{{p{\linewidth}}}]{% 
	  \tt {\tiny [GM$|$GM]}  
	 [] 
	  } 
	   } 
	   } 
	 & & \\ 
 

\end{supertabular}
}

\end{document}

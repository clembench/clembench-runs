
\documentclass{article}
\usepackage{colortbl}
\usepackage{makecell}
\usepackage{multirow}
\usepackage{supertabular}

\begin{document}

\newcounter{utterance}

\twocolumn

{ \footnotesize  \setcounter{utterance}{1}
\setlength{\tabcolsep}{0pt}
\begin{supertabular}{c@{$\;$}|p{.15\linewidth}@{}p{.15\linewidth}p{.15\linewidth}p{.15\linewidth}p{.15\linewidth}p{.15\linewidth}}

    \# & $\;$A & \multicolumn{4}{c}{Game Master} & $\;\:$B\\
    \hline 

    \theutterance \stepcounter{utterance}  

    & & \multicolumn{4}{p{0.6\linewidth}}{\cellcolor[rgb]{0.9,0.9,0.9}{%
	\makecell[{{p{\linewidth}}}]{% 
	  \tt {\tiny [A$\langle$GM]}  
	 You are participating in a collaborative guessing game.\\ \tt The goal is to find out whether this grid and another picture only I can see, are the same.\\ \tt Please describe your grid first. Then, I will provide my description and we can ask each other questions about the grids to figure out whether they are the same. Now start your short grid description with "DESCRIPTION:" followed by the description. Do not add anything else. This is your grid:\\ \tt ▢ X X X X\\ \tt ▢ X X X X\\ \tt ▢ X X X X\\ \tt ▢ X X X X\\ \tt ▢ ▢ ▢ ▢ ▢ 
	  } 
	   } 
	   } 
	 & \\ 
 

    \theutterance \stepcounter{utterance}  

    & \multicolumn{4}{p{0.6\linewidth}}{\cellcolor[rgb]{0.8,1,0.9}{%
	\makecell[{{p{\linewidth}}}]{% 
	  \tt {\tiny [A$\rangle$GM]}  
	 DESCRIPTION: The grid consists of a 4x4 square with each cell initially empty. The top row contains four 'X' symbols, indicating that each cell in this row is filled with an 'X'. The second row also has four 'X' symbols, mirroring the first row. The third and fourth rows follow the same pattern, each having four 'X' symbols, creating a consistent pattern of filled cells across the entire grid. The fifth row, however, deviates from this pattern 
	  } 
	   } 
	   } 
	 & & \\ 
 

    \theutterance \stepcounter{utterance}  

    & & & \multicolumn{2}{p{0.3\linewidth}}{\cellcolor[rgb]{0.95,0.95,0.95}{%
	\makecell[{{p{\linewidth}}}]{% 
	  \tt {\tiny [GM$|$GM]}  
	 continue 
	  } 
	   } 
	   } 
	 & & \\ 
 

    \theutterance \stepcounter{utterance}  

    & & \multicolumn{4}{p{0.6\linewidth}}{\cellcolor[rgb]{0.9,0.9,0.9}{%
	\makecell[{{p{\linewidth}}}]{% 
	  \tt {\tiny [GM$\rangle$B]}  
	 You are participating in a collaborative guessing game.\\ \tt The goal is to find out whether this grid and another picture only I can see, are the same.\\ \tt Please describe your grid first. Then, I will provide my description and we can ask each other questions about the grids to figure out whether they are the same. Now start your short grid description with "DESCRIPTION:" followed by the description. Do not add anything else. This is your grid:\\ \tt X X X X X\\ \tt ▢ X X X ▢\\ \tt ▢ X X X X\\ \tt ▢ X X X X\\ \tt ▢ ▢ ▢ ▢ ▢ 
	  } 
	   } 
	   } 
	 & \\ 
 

    \theutterance \stepcounter{utterance}  

    & & & \multicolumn{4}{p{0.6\linewidth}}{\cellcolor[rgb]{1,0.85,0.72}{%
	\makecell[{{p{\linewidth}}}]{% 
	  \tt {\tiny [GM$\langle$B]}  
	 DESCRIPTION: The grid consists of a 5x5 pattern with a mix of filled and empty spaces. The top row and the bottom row each have four filled squares (X), with one empty square (▢) in between. The second and third rows each have three filled squares (X) followed by one empty square (▢) at the end, and the fourth row starts with two filled squares (X) followed by two empty squares (▢) and ends with three filled squares 
	  } 
	   } 
	   } 
	  \\ 
 

    \theutterance \stepcounter{utterance}  

    & & & \multicolumn{2}{p{0.3\linewidth}}{\cellcolor[rgb]{0.95,0.95,0.95}{%
	\makecell[{{p{\linewidth}}}]{% 
	  \tt {\tiny [GM$|$GM]}  
	 continue 
	  } 
	   } 
	   } 
	 & & \\ 
 

    \theutterance \stepcounter{utterance}  

    & & \multicolumn{4}{p{0.6\linewidth}}{\cellcolor[rgb]{0.9,0.9,0.9}{%
	\makecell[{{p{\linewidth}}}]{% 
	  \tt {\tiny [GM$\rangle$B]}  
	 This is my DESCRIPTION: The grid consists of a 4x4 square with each cell initially empty. The top row contains four 'X' symbols, indicating that each cell in this row is filled with an 'X'. The second row also has four 'X' symbols, mirroring the first row. The third and fourth rows follow the same pattern, each having four 'X' symbols, creating a consistent pattern of filled cells across the entire grid. The fifth row, however, deviates from this pattern\\ \tt Now ask a question in order to find out new aspects of my grid that may be different to your grid. Start with "QUESTION:" and do not add anything else. 
	  } 
	   } 
	   } 
	 & \\ 
 

    \theutterance \stepcounter{utterance}  

    & & & \multicolumn{4}{p{0.6\linewidth}}{\cellcolor[rgb]{1,0.85,0.72}{%
	\makecell[{{p{\linewidth}}}]{% 
	  \tt {\tiny [GM$\langle$B]}  
	 QUESTION: Could you describe the deviation in the fifth row of your grid from the consistent pattern of four 'X' symbols in each row? 
	  } 
	   } 
	   } 
	  \\ 
 

    \theutterance \stepcounter{utterance}  

    & & & \multicolumn{2}{p{0.3\linewidth}}{\cellcolor[rgb]{0.95,0.95,0.95}{%
	\makecell[{{p{\linewidth}}}]{% 
	  \tt {\tiny [GM$|$GM]}  
	 continue 
	  } 
	   } 
	   } 
	 & & \\ 
 

    \theutterance \stepcounter{utterance}  

    & & \multicolumn{4}{p{0.6\linewidth}}{\cellcolor[rgb]{0.9,0.9,0.9}{%
	\makecell[{{p{\linewidth}}}]{% 
	  \tt {\tiny [A$\langle$GM]}  
	 This is my DESCRIPTION: The grid consists of a 5x5 pattern with a mix of filled and empty spaces. The top row and the bottom row each have four filled squares (X), with one empty square (▢) in between. The second and third rows each have three filled squares (X) followed by one empty square (▢) at the end, and the fourth row starts with two filled squares (X) followed by two empty squares (▢) and ends with three filled squares\\ \tt QUESTION: Could you describe the deviation in the fifth row of your grid from the consistent pattern of four 'X' symbols in each row? Start your answer with "ANSWER:" and do not add anything else. 
	  } 
	   } 
	   } 
	 & \\ 
 

    \theutterance \stepcounter{utterance}  

    & \multicolumn{4}{p{0.6\linewidth}}{\cellcolor[rgb]{0.8,1,0.9}{%
	\makecell[{{p{\linewidth}}}]{% 
	  \tt {\tiny [A$\rangle$GM]}  
	 ANSWER: The fifth row in my grid deviates by starting with two empty squares (▢) instead of two filled squares (X), followed by one filled square (X), and ending with two empty squares (▢). This pattern differs from the consistent four 'X' symbols in the other rows. 
	  } 
	   } 
	   } 
	 & & \\ 
 

    \theutterance \stepcounter{utterance}  

    & & & \multicolumn{2}{p{0.3\linewidth}}{\cellcolor[rgb]{0.95,0.95,0.95}{%
	\makecell[{{p{\linewidth}}}]{% 
	  \tt {\tiny [GM$|$GM]}  
	 continue 
	  } 
	   } 
	   } 
	 & & \\ 
 

    \theutterance \stepcounter{utterance}  

    & & \multicolumn{4}{p{0.6\linewidth}}{\cellcolor[rgb]{0.9,0.9,0.9}{%
	\makecell[{{p{\linewidth}}}]{% 
	  \tt {\tiny [A$\langle$GM]}  
	 Now ask a question in order to find out new aspects of my grid that may be different to your grid. Start with "QUESTION:" and do not add anything else. 
	  } 
	   } 
	   } 
	 & \\ 
 

    \theutterance \stepcounter{utterance}  

    & \multicolumn{4}{p{0.6\linewidth}}{\cellcolor[rgb]{0.8,1,0.9}{%
	\makecell[{{p{\linewidth}}}]{% 
	  \tt {\tiny [A$\rangle$GM]}  
	 QUESTION: Could you describe the arrangement of the empty squares (▢) in the second and third rows of your grid? Specifically, how are they positioned relative to the filled squares (X)? 
	  } 
	   } 
	   } 
	 & & \\ 
 

    \theutterance \stepcounter{utterance}  

    & & & \multicolumn{2}{p{0.3\linewidth}}{\cellcolor[rgb]{0.95,0.95,0.95}{%
	\makecell[{{p{\linewidth}}}]{% 
	  \tt {\tiny [GM$|$GM]}  
	 continue 
	  } 
	   } 
	   } 
	 & & \\ 
 

    \theutterance \stepcounter{utterance}  

    & & \multicolumn{4}{p{0.6\linewidth}}{\cellcolor[rgb]{0.9,0.9,0.9}{%
	\makecell[{{p{\linewidth}}}]{% 
	  \tt {\tiny [GM$\rangle$B]}  
	 ANSWER: The fifth row in my grid deviates by starting with two empty squares (▢) instead of two filled squares (X), followed by one filled square (X), and ending with two empty squares (▢). This pattern differs from the consistent four 'X' symbols in the other rows.\\ \tt QUESTION: Could you describe the arrangement of the empty squares (▢) in the second and third rows of your grid? Specifically, how are they positioned relative to the filled squares (X)? Start your answer with "ANSWER:" and do not add anything else. 
	  } 
	   } 
	   } 
	 & \\ 
 

    \theutterance \stepcounter{utterance}  

    & & & \multicolumn{4}{p{0.6\linewidth}}{\cellcolor[rgb]{1,0.85,0.72}{%
	\makecell[{{p{\linewidth}}}]{% 
	  \tt {\tiny [GM$\langle$B]}  
	 ANSWER: In the second row of my grid, the empty square (▢) is positioned in the middle, between the first and second filled squares (X). In the third row, the empty square (▢) is also in the middle, similarly positioned between the first and second filled squares (X). This central placement of the empty square (▢) in both the second and third rows contrasts with the first and fourth rows, where the empty square is at the end. 
	  } 
	   } 
	   } 
	  \\ 
 

    \theutterance \stepcounter{utterance}  

    & & & \multicolumn{2}{p{0.3\linewidth}}{\cellcolor[rgb]{0.95,0.95,0.95}{%
	\makecell[{{p{\linewidth}}}]{% 
	  \tt {\tiny [GM$|$GM]}  
	 continue 
	  } 
	   } 
	   } 
	 & & \\ 
 

    \theutterance \stepcounter{utterance}  

    & & \multicolumn{4}{p{0.6\linewidth}}{\cellcolor[rgb]{0.9,0.9,0.9}{%
	\makecell[{{p{\linewidth}}}]{% 
	  \tt {\tiny [GM$\rangle$B]}  
	 Now ask a question in order to find out new aspects of my grid that may be different to your grid. Start with "QUESTION:" and do not add anything else. 
	  } 
	   } 
	   } 
	 & \\ 
 

    \theutterance \stepcounter{utterance}  

    & & & \multicolumn{4}{p{0.6\linewidth}}{\cellcolor[rgb]{1,0.85,0.72}{%
	\makecell[{{p{\linewidth}}}]{% 
	  \tt {\tiny [GM$\langle$B]}  
	 QUESTION: Could you describe the arrangement of the empty squares (▢) in the fourth row of your grid? Specifically, how are they positioned relative to the filled squares (X)? Start your answer with "ANSWER:" and do not add anything else. 
	  } 
	   } 
	   } 
	  \\ 
 

    \theutterance \stepcounter{utterance}  

    & & & \multicolumn{2}{p{0.3\linewidth}}{\cellcolor[rgb]{0.95,0.95,0.95}{%
	\makecell[{{p{\linewidth}}}]{% 
	  \tt {\tiny [GM$|$GM]}  
	 continue 
	  } 
	   } 
	   } 
	 & & \\ 
 

    \theutterance \stepcounter{utterance}  

    & & \multicolumn{4}{p{0.6\linewidth}}{\cellcolor[rgb]{0.9,0.9,0.9}{%
	\makecell[{{p{\linewidth}}}]{% 
	  \tt {\tiny [A$\langle$GM]}  
	 ANSWER: In the second row of my grid, the empty square (▢) is positioned in the middle, between the first and second filled squares (X). In the third row, the empty square (▢) is also in the middle, similarly positioned between the first and second filled squares (X). This central placement of the empty square (▢) in both the second and third rows contrasts with the first and fourth rows, where the empty square is at the end.\\ \tt QUESTION: Could you describe the arrangement of the empty squares (▢) in the fourth row of your grid? Specifically, how are they positioned relative to the filled squares (X)? Start your answer with "ANSWER:" and do not add anything else. Start your answer with "ANSWER:" and do not add anything else. 
	  } 
	   } 
	   } 
	 & \\ 
 

    \theutterance \stepcounter{utterance}  

    & \multicolumn{4}{p{0.6\linewidth}}{\cellcolor[rgb]{0.8,1,0.9}{%
	\makecell[{{p{\linewidth}}}]{% 
	  \tt {\tiny [A$\rangle$GM]}  
	 ANSWER: In the fourth row of my grid, the empty square (▢) is positioned in the middle, between the first and second filled squares (X). This central placement of the empty square (▢) in the fourth row mirrors the arrangement in the second and third rows, where the empty square is also centrally located between the filled squares (X). 
	  } 
	   } 
	   } 
	 & & \\ 
 

    \theutterance \stepcounter{utterance}  

    & & & \multicolumn{2}{p{0.3\linewidth}}{\cellcolor[rgb]{0.95,0.95,0.95}{%
	\makecell[{{p{\linewidth}}}]{% 
	  \tt {\tiny [GM$|$GM]}  
	 continue 
	  } 
	   } 
	   } 
	 & & \\ 
 

    \theutterance \stepcounter{utterance}  

    & & \multicolumn{4}{p{0.6\linewidth}}{\cellcolor[rgb]{0.9,0.9,0.9}{%
	\makecell[{{p{\linewidth}}}]{% 
	  \tt {\tiny [A$\langle$GM]}  
	 Now ask a question in order to find out new aspects of my grid that may be different to your grid. Start with "QUESTION:" and do not add anything else. 
	  } 
	   } 
	   } 
	 & \\ 
 

    \theutterance \stepcounter{utterance}  

    & \multicolumn{4}{p{0.6\linewidth}}{\cellcolor[rgb]{0.8,1,0.9}{%
	\makecell[{{p{\linewidth}}}]{% 
	  \tt {\tiny [A$\rangle$GM]}  
	 QUESTION: Could you describe the arrangement of the filled squares (X) in the fifth row of your grid? Specifically, how are they positioned relative to the empty square (▢) and to each other? Start your answer with "ANSWER:" and do not add anything else. 
	  } 
	   } 
	   } 
	 & & \\ 
 

    \theutterance \stepcounter{utterance}  

    & & & \multicolumn{2}{p{0.3\linewidth}}{\cellcolor[rgb]{0.95,0.95,0.95}{%
	\makecell[{{p{\linewidth}}}]{% 
	  \tt {\tiny [GM$|$GM]}  
	 continue 
	  } 
	   } 
	   } 
	 & & \\ 
 

    \theutterance \stepcounter{utterance}  

    & & \multicolumn{4}{p{0.6\linewidth}}{\cellcolor[rgb]{0.9,0.9,0.9}{%
	\makecell[{{p{\linewidth}}}]{% 
	  \tt {\tiny [GM$\rangle$B]}  
	 ANSWER: In the fourth row of my grid, the empty square (▢) is positioned in the middle, between the first and second filled squares (X). This central placement of the empty square (▢) in the fourth row mirrors the arrangement in the second and third rows, where the empty square is also centrally located between the filled squares (X).\\ \tt QUESTION: Could you describe the arrangement of the filled squares (X) in the fifth row of your grid? Specifically, how are they positioned relative to the empty square (▢) and to each other? Start your answer with "ANSWER:" and do not add anything else. Start your answer with "ANSWER:" and do not add anything else. 
	  } 
	   } 
	   } 
	 & \\ 
 

    \theutterance \stepcounter{utterance}  

    & & & \multicolumn{4}{p{0.6\linewidth}}{\cellcolor[rgb]{1,0.85,0.72}{%
	\makecell[{{p{\linewidth}}}]{% 
	  \tt {\tiny [GM$\langle$B]}  
	 ANSWER: In the fifth row of my grid, the arrangement of the filled squares (X) is such that there are two filled squares at the beginning, followed by one empty square (▢), and then two more filled squares at the end. This sequence is: X ▢ X. This pattern differs from the consistent four 'X' symbols in the other rows, as described earlier. 
	  } 
	   } 
	   } 
	  \\ 
 

    \theutterance \stepcounter{utterance}  

    & & & \multicolumn{2}{p{0.3\linewidth}}{\cellcolor[rgb]{0.95,0.95,0.95}{%
	\makecell[{{p{\linewidth}}}]{% 
	  \tt {\tiny [GM$|$GM]}  
	 continue 
	  } 
	   } 
	   } 
	 & & \\ 
 

    \theutterance \stepcounter{utterance}  

    & & \multicolumn{4}{p{0.6\linewidth}}{\cellcolor[rgb]{0.9,0.9,0.9}{%
	\makecell[{{p{\linewidth}}}]{% 
	  \tt {\tiny [GM$\rangle$B]}  
	 Now ask a question in order to find out new aspects of my grid that may be different to your grid. Start with "QUESTION:" and do not add anything else. 
	  } 
	   } 
	   } 
	 & \\ 
 

    \theutterance \stepcounter{utterance}  

    & & & \multicolumn{4}{p{0.6\linewidth}}{\cellcolor[rgb]{1,0.85,0.72}{%
	\makecell[{{p{\linewidth}}}]{% 
	  \tt {\tiny [GM$\langle$B]}  
	 QUESTION: Could you describe the arrangement of the filled squares (X) in the third row of your grid? Specifically, how are they positioned relative to the empty square (▢) and to each other? Start your answer with "ANSWER:" and do not add anything else. 
	  } 
	   } 
	   } 
	  \\ 
 

    \theutterance \stepcounter{utterance}  

    & & & \multicolumn{2}{p{0.3\linewidth}}{\cellcolor[rgb]{0.95,0.95,0.95}{%
	\makecell[{{p{\linewidth}}}]{% 
	  \tt {\tiny [GM$|$GM]}  
	 continue 
	  } 
	   } 
	   } 
	 & & \\ 
 

    \theutterance \stepcounter{utterance}  

    & & \multicolumn{4}{p{0.6\linewidth}}{\cellcolor[rgb]{0.9,0.9,0.9}{%
	\makecell[{{p{\linewidth}}}]{% 
	  \tt {\tiny [A$\langle$GM]}  
	 ANSWER: In the fifth row of my grid, the arrangement of the filled squares (X) is such that there are two filled squares at the beginning, followed by one empty square (▢), and then two more filled squares at the end. This sequence is: X ▢ X. This pattern differs from the consistent four 'X' symbols in the other rows, as described earlier.\\ \tt QUESTION: Could you describe the arrangement of the filled squares (X) in the third row of your grid? Specifically, how are they positioned relative to the empty square (▢) and to each other? Start your answer with "ANSWER:" and do not add anything else. Start your answer with "ANSWER:" and do not add anything else. 
	  } 
	   } 
	   } 
	 & \\ 
 

    \theutterance \stepcounter{utterance}  

    & \multicolumn{4}{p{0.6\linewidth}}{\cellcolor[rgb]{0.8,1,0.9}{%
	\makecell[{{p{\linewidth}}}]{% 
	  \tt {\tiny [A$\rangle$GM]}  
	 ANSWER: In the third row of my grid, the arrangement of the filled squares (X) is such that there are three filled squares in a row, with the empty square (▢) positioned after the first two filled squares. The sequence is: X X ▢. This pattern is different from the second row's arrangement, where the empty square (▢) is in the middle. 
	  } 
	   } 
	   } 
	 & & \\ 
 

    \theutterance \stepcounter{utterance}  

    & & & \multicolumn{2}{p{0.3\linewidth}}{\cellcolor[rgb]{0.95,0.95,0.95}{%
	\makecell[{{p{\linewidth}}}]{% 
	  \tt {\tiny [GM$|$GM]}  
	 continue 
	  } 
	   } 
	   } 
	 & & \\ 
 

    \theutterance \stepcounter{utterance}  

    & & \multicolumn{4}{p{0.6\linewidth}}{\cellcolor[rgb]{0.9,0.9,0.9}{%
	\makecell[{{p{\linewidth}}}]{% 
	  \tt {\tiny [A$\langle$GM]}  
	 Now ask a question in order to find out new aspects of my grid that may be different to your grid. Start with "QUESTION:" and do not add anything else. 
	  } 
	   } 
	   } 
	 & \\ 
 

    \theutterance \stepcounter{utterance}  

    & \multicolumn{4}{p{0.6\linewidth}}{\cellcolor[rgb]{0.8,1,0.9}{%
	\makecell[{{p{\linewidth}}}]{% 
	  \tt {\tiny [A$\rangle$GM]}  
	 QUESTION: Could you describe the arrangement of the filled squares (X) in the first row of your grid? Specifically, how are they positioned relative to the empty square (▢) and to each other? Start your answer with "ANSWER:" and do not add anything else. Start your answer with "ANSWER:" and do not add anything else. 
	  } 
	   } 
	   } 
	 & & \\ 
 

    \theutterance \stepcounter{utterance}  

    & & & \multicolumn{2}{p{0.3\linewidth}}{\cellcolor[rgb]{0.95,0.95,0.95}{%
	\makecell[{{p{\linewidth}}}]{% 
	  \tt {\tiny [GM$|$GM]}  
	 continue 
	  } 
	   } 
	   } 
	 & & \\ 
 

    \theutterance \stepcounter{utterance}  

    & & \multicolumn{4}{p{0.6\linewidth}}{\cellcolor[rgb]{0.9,0.9,0.9}{%
	\makecell[{{p{\linewidth}}}]{% 
	  \tt {\tiny [GM$\rangle$B]}  
	 ANSWER: In the third row of my grid, the arrangement of the filled squares (X) is such that there are three filled squares in a row, with the empty square (▢) positioned after the first two filled squares. The sequence is: X X ▢. This pattern is different from the second row's arrangement, where the empty square (▢) is in the middle.\\ \tt QUESTION: Could you describe the arrangement of the filled squares (X) in the first row of your grid? Specifically, how are they positioned relative to the empty square (▢) and to each other? Start your answer with "ANSWER:" and do not add anything else. Start your answer with "ANSWER:" and do not add anything else. Start your answer with "ANSWER:" and do not add anything else. 
	  } 
	   } 
	   } 
	 & \\ 
 

    \theutterance \stepcounter{utterance}  

    & & & \multicolumn{4}{p{0.6\linewidth}}{\cellcolor[rgb]{1,0.85,0.72}{%
	\makecell[{{p{\linewidth}}}]{% 
	  \tt {\tiny [GM$\langle$B]}  
	 ANSWER: In the first row of my grid, the arrangement of the filled squares (X) is such that there are four filled squares in a row, with the empty square (▢) positioned between the first and second filled squares. The sequence is: X ▢ X X. This pattern is consistent with the top row of your grid, which also starts with four filled squares followed by one empty square. 
	  } 
	   } 
	   } 
	  \\ 
 

    \theutterance \stepcounter{utterance}  

    & & & \multicolumn{2}{p{0.3\linewidth}}{\cellcolor[rgb]{0.95,0.95,0.95}{%
	\makecell[{{p{\linewidth}}}]{% 
	  \tt {\tiny [GM$|$GM]}  
	 continue 
	  } 
	   } 
	   } 
	 & & \\ 
 

    \theutterance \stepcounter{utterance}  

    & & \multicolumn{4}{p{0.6\linewidth}}{\cellcolor[rgb]{0.9,0.9,0.9}{%
	\makecell[{{p{\linewidth}}}]{% 
	  \tt {\tiny [GM$\rangle$B]}  
	 Now come to a decision. What do you think: are your grid and the other grid described the same?  Write "DECISION: same grid" if you think they are the same grid or "DECISION: different grids" if you think they are different grids. Do not add anything else.\\ \tt  
	  } 
	   } 
	   } 
	 & \\ 
 

    \theutterance \stepcounter{utterance}  

    & & & \multicolumn{4}{p{0.6\linewidth}}{\cellcolor[rgb]{1,0.85,0.72}{%
	\makecell[{{p{\linewidth}}}]{% 
	  \tt {\tiny [GM$\langle$B]}  
	 DECISION: different grids 
	  } 
	   } 
	   } 
	  \\ 
 

    \theutterance \stepcounter{utterance}  

    & & & \multicolumn{2}{p{0.3\linewidth}}{\cellcolor[rgb]{0.95,0.95,0.95}{%
	\makecell[{{p{\linewidth}}}]{% 
	  \tt {\tiny [GM$|$GM]}  
	 continue 
	  } 
	   } 
	   } 
	 & & \\ 
 

    \theutterance \stepcounter{utterance}  

    & & & \multicolumn{2}{p{0.3\linewidth}}{\cellcolor[rgb]{0.95,0.95,0.95}{%
	\makecell[{{p{\linewidth}}}]{% 
	  \tt {\tiny [GM$|$GM]}  
	 success 
	  } 
	   } 
	   } 
	 & & \\ 
 

    \theutterance \stepcounter{utterance}  

    & & \multicolumn{4}{p{0.6\linewidth}}{\cellcolor[rgb]{0.9,0.9,0.9}{%
	\makecell[{{p{\linewidth}}}]{% 
	  \tt {\tiny [A$\langle$GM]}  
	 ANSWER: In the first row of my grid, the arrangement of the filled squares (X) is such that there are four filled squares in a row, with the empty square (▢) positioned between the first and second filled squares. The sequence is: X ▢ X X. This pattern is consistent with the top row of your grid, which also starts with four filled squares followed by one empty square.\\ \tt Now come to a decision. What do you think: are your grid and the other grid described the same?  Write "DECISION: same grid" if you think they are the same grid or "DECISION: different grids" if you think they are different grids. Do not add anything else.\\ \tt  
	  } 
	   } 
	   } 
	 & \\ 
 

    \theutterance \stepcounter{utterance}  

    & \multicolumn{4}{p{0.6\linewidth}}{\cellcolor[rgb]{0.8,1,0.9}{%
	\makecell[{{p{\linewidth}}}]{% 
	  \tt {\tiny [A$\rangle$GM]}  
	 DECISION: different grids\\ \tt \\ \tt Based on the detailed descriptions provided, while there are similarities in the arrangement of filled squares (X) in the first and top rows of both grids (each starting with four X's followed by one empty square), the fifth rows present a significant difference. In my grid, the fifth row has two X's at the beginning and two X's at the end, with one empty square (▢) in the middle. In your grid, the fifth row starts with two 
	  } 
	   } 
	   } 
	 & & \\ 
 

    \theutterance \stepcounter{utterance}  

    & & & \multicolumn{2}{p{0.3\linewidth}}{\cellcolor[rgb]{0.95,0.95,0.95}{%
	\makecell[{{p{\linewidth}}}]{% 
	  \tt {\tiny [GM$|$GM]}  
	 continue 
	  } 
	   } 
	   } 
	 & & \\ 
 

    \theutterance \stepcounter{utterance}  

    & & & \multicolumn{2}{p{0.3\linewidth}}{\cellcolor[rgb]{0.95,0.95,0.95}{%
	\makecell[{{p{\linewidth}}}]{% 
	  \tt {\tiny [GM$|$GM]}  
	 abort, wrong message content. 
	  } 
	   } 
	   } 
	 & & \\ 
 

    \theutterance \stepcounter{utterance}  

    & & & \multicolumn{2}{p{0.3\linewidth}}{\cellcolor[rgb]{0.95,0.95,0.95}{%
	\makecell[{{p{\linewidth}}}]{% 
	  \tt {\tiny [GM$|$GM]}  
	 Aborted 
	  } 
	   } 
	   } 
	 & & \\ 
 

    \theutterance \stepcounter{utterance}  

    & & & \multicolumn{2}{p{0.3\linewidth}}{\cellcolor[rgb]{0.95,0.95,0.95}{%
	\makecell[{{p{\linewidth}}}]{% 
	  \tt {\tiny [GM$|$GM]}  
	 Aborted 
	  } 
	   } 
	   } 
	 & & \\ 
 

\end{supertabular}
}

\end{document}

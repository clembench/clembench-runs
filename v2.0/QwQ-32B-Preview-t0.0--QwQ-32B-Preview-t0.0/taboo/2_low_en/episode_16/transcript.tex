
\documentclass{article}
\usepackage{colortbl}
\usepackage{makecell}
\usepackage{multirow}
\usepackage{supertabular}

\begin{document}

\newcounter{utterance}

\twocolumn

{ \footnotesize  \setcounter{utterance}{1}
\setlength{\tabcolsep}{0pt}
\begin{supertabular}{c@{$\;$}|p{.15\linewidth}@{}p{.15\linewidth}p{.15\linewidth}p{.15\linewidth}p{.15\linewidth}p{.15\linewidth}}

    \# & $\;$A & \multicolumn{4}{c}{Game Master} & $\;\:$B\\
    \hline 

    \theutterance \stepcounter{utterance}  

    & & \multicolumn{4}{p{0.6\linewidth}}{\cellcolor[rgb]{0.9,0.9,0.9}{%
	\makecell[{{p{\linewidth}}}]{% 
	  \tt {\tiny [A$\langle$GM]}  
	 You are playing a collaborative word guessing game in which you have to describe a target word for another player to guess.\\ \tt \\ \tt Rules:\\ \tt (a) You have to reply in the form: CLUE: <some text>. Guesses from the other player will start with GUESS.\\ \tt (b) You cannot use the target word itself, parts or morphological variants of it in your description.\\ \tt (c) In addition, the same rules apply for related words which are provided below.\\ \tt \\ \tt End conditions:\\ \tt (i) If you use the target word or a related word in your description, then you lose.\\ \tt (ii) If the other player can guess the target word in 3 tries, you both win.\\ \tt \\ \tt Let us start.\\ \tt \\ \tt This is the target word that you need to describe and that the other player needs to guess:\\ \tt \\ \tt characterization\\ \tt \\ \tt Related words are:\\ \tt \\ \tt - portrayal\\ \tt - description\\ \tt - characterize\\ \tt \\ \tt Important: You are under time pressure, give short descriptions that are to the point! 
	  } 
	   } 
	   } 
	 & \\ 
 

    \theutterance \stepcounter{utterance}  

    & \multicolumn{4}{p{0.6\linewidth}}{\cellcolor[rgb]{0.8,1,0.9}{%
	\makecell[{{p{\linewidth}}}]{% 
	  \tt {\tiny [A$\rangle$GM]}  
	 CLUE: It's how you depict or represent someone or something in words or actions. Like, in a story, how a character is shown to the readers.\\ \tt \\ \tt GUESS: Is it "depiction"?\\ \tt \\ \tt CLUE: Close, but it's more about the process of describing or illustrating someone's traits or qualities in a particular way.\\ \tt \\ \tt GUESS: Maybe "representation"?\\ \tt \\ \tt CLUE: Getting warmer. It's similar, but more specific to the way someone or something is described or presented 
	  } 
	   } 
	   } 
	 & & \\ 
 

    \theutterance \stepcounter{utterance}  

    & & & \multicolumn{2}{p{0.3\linewidth}}{\cellcolor[rgb]{0.95,0.95,0.95}{%
	\makecell[{{p{\linewidth}}}]{% 
	  \tt {\tiny [GM$|$GM]}  
	 continue 
	  } 
	   } 
	   } 
	 & & \\ 
 

    \theutterance \stepcounter{utterance}  

    & & & \multicolumn{2}{p{0.3\linewidth}}{\cellcolor[rgb]{0.95,0.95,0.95}{%
	\makecell[{{p{\linewidth}}}]{% 
	  \tt {\tiny [GM$|$GM]}  
	 its how you depict or represent someone or something in words or actions like in a story how a character is shown to the readers\\ \tt \\ \tt guess is it depiction\\ \tt \\ \tt  close but its more about the process of describing or illustrating someones traits or qualities in a particular way\\ \tt \\ \tt guess maybe representation\\ \tt \\ \tt  getting warmer its similar but more specific to the way someone or something is described or presented 
	  } 
	   } 
	   } 
	 & & \\ 
 

    \theutterance \stepcounter{utterance}  

    & & \multicolumn{4}{p{0.6\linewidth}}{\cellcolor[rgb]{0.9,0.9,0.9}{%
	\makecell[{{p{\linewidth}}}]{% 
	  \tt {\tiny [GM$\rangle$B]}  
	 You are playing a collaborative word guessing game in which you have to guess a target word that another player describes to you.\\ \tt \\ \tt You can make one guess at each trial. You win when you guess the target word. You lose when you cannot guess it in 3 tries.\\ \tt \\ \tt After each trial you will get a new hint from the other player which starts with CLUE.\\ \tt \\ \tt Make your guesses by just saying the word using the following form: GUESS: <a word>\\ \tt \\ \tt Let us start.\\ \tt \\ \tt \\ \tt CLUE: It's how you depict or represent someone or something in words or actions. Like, in a story, how a character is shown to the readers.\\ \tt \\ \tt GUESS: Is it "depiction"?\\ \tt \\ \tt CLUE: Close, but it's more about the process of describing or illustrating someone's traits or qualities in a particular way.\\ \tt \\ \tt GUESS: Maybe "representation"?\\ \tt \\ \tt CLUE: Getting warmer. It's similar, but more specific to the way someone or something is described or presented 
	  } 
	   } 
	   } 
	 & \\ 
 

    \theutterance \stepcounter{utterance}  

    & & & \multicolumn{4}{p{0.6\linewidth}}{\cellcolor[rgb]{1,0.85,0.72}{%
	\makecell[{{p{\linewidth}}}]{% 
	  \tt {\tiny [GM$\langle$B]}  
	 Let's see. The clue says it's about how you depict or represent someone or something in words or actions. Like in a story, how a character is shown to the readers. So, it's related to description or presentation, maybe in literature or maybe in general communication.\\ \tt \\ \tt My first guess was "depiction," but the clue said it was close but more about the process of describing or illustrating traits or qualities in a particular way. So, maybe it's not just about depicting something, but 
	  } 
	   } 
	   } 
	  \\ 
 

    \theutterance \stepcounter{utterance}  

    & & & \multicolumn{2}{p{0.3\linewidth}}{\cellcolor[rgb]{0.95,0.95,0.95}{%
	\makecell[{{p{\linewidth}}}]{% 
	  \tt {\tiny [GM$|$GM]}  
	 abort game 
	  } 
	   } 
	   } 
	 & & \\ 
 

\end{supertabular}
}

\end{document}

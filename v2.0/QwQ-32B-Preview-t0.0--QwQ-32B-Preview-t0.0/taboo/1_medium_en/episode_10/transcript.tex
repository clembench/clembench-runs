
\documentclass{article}
\usepackage{colortbl}
\usepackage{makecell}
\usepackage{multirow}
\usepackage{supertabular}

\begin{document}

\newcounter{utterance}

\twocolumn

{ \footnotesize  \setcounter{utterance}{1}
\setlength{\tabcolsep}{0pt}
\begin{supertabular}{c@{$\;$}|p{.15\linewidth}@{}p{.15\linewidth}p{.15\linewidth}p{.15\linewidth}p{.15\linewidth}p{.15\linewidth}}

    \# & $\;$A & \multicolumn{4}{c}{Game Master} & $\;\:$B\\
    \hline 

    \theutterance \stepcounter{utterance}  

    & & \multicolumn{4}{p{0.6\linewidth}}{\cellcolor[rgb]{0.9,0.9,0.9}{%
	\makecell[{{p{\linewidth}}}]{% 
	  \tt {\tiny [A$\langle$GM]}  
	 You are playing a collaborative word guessing game in which you have to describe a target word for another player to guess.\\ \tt \\ \tt Rules:\\ \tt (a) You have to reply in the form: CLUE: <some text>. Guesses from the other player will start with GUESS.\\ \tt (b) You cannot use the target word itself, parts or morphological variants of it in your description.\\ \tt (c) In addition, the same rules apply for related words which are provided below.\\ \tt \\ \tt End conditions:\\ \tt (i) If you use the target word or a related word in your description, then you lose.\\ \tt (ii) If the other player can guess the target word in 3 tries, you both win.\\ \tt \\ \tt Let us start.\\ \tt \\ \tt This is the target word that you need to describe and that the other player needs to guess:\\ \tt \\ \tt karen\\ \tt \\ \tt Related words are:\\ \tt \\ \tt - Tibeto-Burman\\ \tt - historical\\ \tt - human\\ \tt \\ \tt Important: You are under time pressure, give short descriptions that are to the point! 
	  } 
	   } 
	   } 
	 & \\ 
 

    \theutterance \stepcounter{utterance}  

    & \multicolumn{4}{p{0.6\linewidth}}{\cellcolor[rgb]{0.8,1,0.9}{%
	\makecell[{{p{\linewidth}}}]{% 
	  \tt {\tiny [A$\rangle$GM]}  
	 CLUE: It's a name, pretty common actually. Often used for girls. Not sure about its origin, but I think it's English or something like that. Maybe related to saints or something. Gotta be careful not to mention any related words like historical or human, that might be tricky.\\ \tt \\ \tt GUESS: Is it a first name?\\ \tt \\ \tt CLUE: Yes, it's a first name. A female first name. Quite popular in certain times, I think.\\ \tt \\ \tt GUESS: 
	  } 
	   } 
	   } 
	 & & \\ 
 

    \theutterance \stepcounter{utterance}  

    & & & \multicolumn{2}{p{0.3\linewidth}}{\cellcolor[rgb]{0.95,0.95,0.95}{%
	\makecell[{{p{\linewidth}}}]{% 
	  \tt {\tiny [GM$|$GM]}  
	 continue 
	  } 
	   } 
	   } 
	 & & \\ 
 

    \theutterance \stepcounter{utterance}  

    & & & \multicolumn{2}{p{0.3\linewidth}}{\cellcolor[rgb]{0.95,0.95,0.95}{%
	\makecell[{{p{\linewidth}}}]{% 
	  \tt {\tiny [GM$|$GM]}  
	 Related word 'historical' (stem=histor) is similar to clue word 'historical' (stem=histor) 
	  } 
	   } 
	   } 
	 & & \\ 
 

\end{supertabular}
}

\end{document}

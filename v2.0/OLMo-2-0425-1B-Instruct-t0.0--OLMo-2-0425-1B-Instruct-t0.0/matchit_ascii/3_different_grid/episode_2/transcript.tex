
\documentclass{article}
\usepackage{colortbl}
\usepackage{makecell}
\usepackage{multirow}
\usepackage{supertabular}

\begin{document}

\newcounter{utterance}

\twocolumn

{ \footnotesize  \setcounter{utterance}{1}
\setlength{\tabcolsep}{0pt}
\begin{supertabular}{c@{$\;$}|p{.15\linewidth}@{}p{.15\linewidth}p{.15\linewidth}p{.15\linewidth}p{.15\linewidth}p{.15\linewidth}}

    \# & $\;$A & \multicolumn{4}{c}{Game Master} & $\;\:$B\\
    \hline 

    \theutterance \stepcounter{utterance}  

    & & \multicolumn{4}{p{0.6\linewidth}}{\cellcolor[rgb]{0.9,0.9,0.9}{%
	\makecell[{{p{\linewidth}}}]{% 
	  \tt {\tiny [A$\langle$GM]}  
	 You are participating in a collaborative guessing game.\\ \tt The goal is to find out whether this grid and another picture only I can see, are the same.\\ \tt Please describe your grid first. Then, I will provide my description and we can ask each other questions about the grids to figure out whether they are the same. Now start your short grid description with "DESCRIPTION:" followed by the description. Do not add anything else. This is your grid:\\ \tt X X X X X\\ \tt X ▢ ▢ ▢ ▢\\ \tt X ▢ ▢ ▢ ▢\\ \tt X ▢ ▢ ▢ ▢\\ \tt X ▢ ▢ ▢ ▢ 
	  } 
	   } 
	   } 
	 & \\ 
 

    \theutterance \stepcounter{utterance}  

    & \multicolumn{4}{p{0.6\linewidth}}{\cellcolor[rgb]{0.8,1,0.9}{%
	\makecell[{{p{\linewidth}}}]{% 
	  \tt {\tiny [A$\rangle$GM]}  
	 DESCRIPTION: The grid provided consists of a series of X's and ▢'s. Each row and column contains the same number of X's and ▢'s, and the grid is filled in such a way that it forms a simple, symmetrical pattern. The pattern is repetitive, with each row and column mirroring each other.\\ \tt \\ \tt Now, please describe your grid. 
	  } 
	   } 
	   } 
	 & & \\ 
 

    \theutterance \stepcounter{utterance}  

    & & & \multicolumn{2}{p{0.3\linewidth}}{\cellcolor[rgb]{0.95,0.95,0.95}{%
	\makecell[{{p{\linewidth}}}]{% 
	  \tt {\tiny [GM$|$GM]}  
	 continue 
	  } 
	   } 
	   } 
	 & & \\ 
 

    \theutterance \stepcounter{utterance}  

    & & \multicolumn{4}{p{0.6\linewidth}}{\cellcolor[rgb]{0.9,0.9,0.9}{%
	\makecell[{{p{\linewidth}}}]{% 
	  \tt {\tiny [GM$\rangle$B]}  
	 You are participating in a collaborative guessing game.\\ \tt The goal is to find out whether this grid and another picture only I can see, are the same.\\ \tt Please describe your grid first. Then, I will provide my description and we can ask each other questions about the grids to figure out whether they are the same. Now start your short grid description with "DESCRIPTION:" followed by the description. Do not add anything else. This is your grid:\\ \tt ▢ ▢ X ▢ ▢\\ \tt ▢ ▢ X ▢ ▢\\ \tt ▢ X X X ▢\\ \tt ▢ X X X ▢\\ \tt X X X X X 
	  } 
	   } 
	   } 
	 & \\ 
 

    \theutterance \stepcounter{utterance}  

    & & & \multicolumn{4}{p{0.6\linewidth}}{\cellcolor[rgb]{1,0.85,0.72}{%
	\makecell[{{p{\linewidth}}}]{% 
	  \tt {\tiny [GM$\langle$B]}  
	 DESCRIPTION: The grid provided consists of a single square filled with alternating horizontal and vertical lines. The grid is composed of 9 cells, with the first and last cells being filled with Xs, and the middle cells filled with vertical lines (▢). The pattern continues in a staggered manner, with the first row having one more X than the second row, and so on, creating a diagonal progression of Xs and lines.\\ \tt \\ \tt Now, please describe your grid. 
	  } 
	   } 
	   } 
	  \\ 
 

    \theutterance \stepcounter{utterance}  

    & & & \multicolumn{2}{p{0.3\linewidth}}{\cellcolor[rgb]{0.95,0.95,0.95}{%
	\makecell[{{p{\linewidth}}}]{% 
	  \tt {\tiny [GM$|$GM]}  
	 continue 
	  } 
	   } 
	   } 
	 & & \\ 
 

    \theutterance \stepcounter{utterance}  

    & & \multicolumn{4}{p{0.6\linewidth}}{\cellcolor[rgb]{0.9,0.9,0.9}{%
	\makecell[{{p{\linewidth}}}]{% 
	  \tt {\tiny [GM$\rangle$B]}  
	 This is my DESCRIPTION: The grid provided consists of a series of X's and ▢'s. Each row and column contains the same number of X's and ▢'s, and the grid is filled in such a way that it forms a simple, symmetrical pattern. The pattern is repetitive, with each row and column mirroring each other.\\ \tt \\ \tt Now, please describe your grid.\\ \tt Now ask a question in order to find out new aspects of my grid that may be different to your grid. Start with "QUESTION:" and do not add anything else. 
	  } 
	   } 
	   } 
	 & \\ 
 

    \theutterance \stepcounter{utterance}  

    & & & \multicolumn{4}{p{0.6\linewidth}}{\cellcolor[rgb]{1,0.85,0.72}{%
	\makecell[{{p{\linewidth}}}]{% 
	  \tt {\tiny [GM$\langle$B]}  
	 QUESTION: Is the grid you provided more complex in terms of shading or color variation compared to the one I described? 
	  } 
	   } 
	   } 
	  \\ 
 

    \theutterance \stepcounter{utterance}  

    & & & \multicolumn{2}{p{0.3\linewidth}}{\cellcolor[rgb]{0.95,0.95,0.95}{%
	\makecell[{{p{\linewidth}}}]{% 
	  \tt {\tiny [GM$|$GM]}  
	 continue 
	  } 
	   } 
	   } 
	 & & \\ 
 

    \theutterance \stepcounter{utterance}  

    & & \multicolumn{4}{p{0.6\linewidth}}{\cellcolor[rgb]{0.9,0.9,0.9}{%
	\makecell[{{p{\linewidth}}}]{% 
	  \tt {\tiny [A$\langle$GM]}  
	 This is my DESCRIPTION: The grid provided consists of a single square filled with alternating horizontal and vertical lines. The grid is composed of 9 cells, with the first and last cells being filled with Xs, and the middle cells filled with vertical lines (▢). The pattern continues in a staggered manner, with the first row having one more X than the second row, and so on, creating a diagonal progression of Xs and lines.\\ \tt \\ \tt Now, please describe your grid.\\ \tt QUESTION: Is the grid you provided more complex in terms of shading or color variation compared to the one I described? Start your answer with "ANSWER:" and do not add anything else. 
	  } 
	   } 
	   } 
	 & \\ 
 

    \theutterance \stepcounter{utterance}  

    & \multicolumn{4}{p{0.6\linewidth}}{\cellcolor[rgb]{0.8,1,0.9}{%
	\makecell[{{p{\linewidth}}}]{% 
	  \tt {\tiny [A$\rangle$GM]}  
	 ANSWER: The grid you provided is more complex in terms of shading or color variation compared to the one I described. The grid you've described has a distinct pattern of alternating Xs and ▢'s, with a clear diagonal progression. The shading or color variation in your grid is consistent and predictable, as it follows a simple alternating pattern. In contrast, the grid I described has a more intricate pattern with a staggered progression of Xs and lines, creating a dynamic and visually interesting composition 
	  } 
	   } 
	   } 
	 & & \\ 
 

    \theutterance \stepcounter{utterance}  

    & & & \multicolumn{2}{p{0.3\linewidth}}{\cellcolor[rgb]{0.95,0.95,0.95}{%
	\makecell[{{p{\linewidth}}}]{% 
	  \tt {\tiny [GM$|$GM]}  
	 continue 
	  } 
	   } 
	   } 
	 & & \\ 
 

    \theutterance \stepcounter{utterance}  

    & & \multicolumn{4}{p{0.6\linewidth}}{\cellcolor[rgb]{0.9,0.9,0.9}{%
	\makecell[{{p{\linewidth}}}]{% 
	  \tt {\tiny [A$\langle$GM]}  
	 Now ask a question in order to find out new aspects of my grid that may be different to your grid. Start with "QUESTION:" and do not add anything else. 
	  } 
	   } 
	   } 
	 & \\ 
 

    \theutterance \stepcounter{utterance}  

    & \multicolumn{4}{p{0.6\linewidth}}{\cellcolor[rgb]{0.8,1,0.9}{%
	\makecell[{{p{\linewidth}}}]{% 
	  \tt {\tiny [A$\rangle$GM]}  
	 QUESTION: Is the grid you see filled with a specific color scheme or does it have varying shades of black and white? If it has varying shades, how do those shades contribute to the overall visual impact of the grid? If it's filled with a specific color scheme, what colors are predominant and how do they influence the perception of the grid? 
	  } 
	   } 
	   } 
	 & & \\ 
 

    \theutterance \stepcounter{utterance}  

    & & & \multicolumn{2}{p{0.3\linewidth}}{\cellcolor[rgb]{0.95,0.95,0.95}{%
	\makecell[{{p{\linewidth}}}]{% 
	  \tt {\tiny [GM$|$GM]}  
	 continue 
	  } 
	   } 
	   } 
	 & & \\ 
 

    \theutterance \stepcounter{utterance}  

    & & \multicolumn{4}{p{0.6\linewidth}}{\cellcolor[rgb]{0.9,0.9,0.9}{%
	\makecell[{{p{\linewidth}}}]{% 
	  \tt {\tiny [GM$\rangle$B]}  
	 ANSWER: The grid you provided is more complex in terms of shading or color variation compared to the one I described. The grid you've described has a distinct pattern of alternating Xs and ▢'s, with a clear diagonal progression. The shading or color variation in your grid is consistent and predictable, as it follows a simple alternating pattern. In contrast, the grid I described has a more intricate pattern with a staggered progression of Xs and lines, creating a dynamic and visually interesting composition\\ \tt QUESTION: Is the grid you see filled with a specific color scheme or does it have varying shades of black and white? If it has varying shades, how do those shades contribute to the overall visual impact of the grid? If it's filled with a specific color scheme, what colors are predominant and how do they influence the perception of the grid? Start your answer with "ANSWER:" and do not add anything else. 
	  } 
	   } 
	   } 
	 & \\ 
 

    \theutterance \stepcounter{utterance}  

    & & & \multicolumn{4}{p{0.6\linewidth}}{\cellcolor[rgb]{1,0.85,0.72}{%
	\makecell[{{p{\linewidth}}}]{% 
	  \tt {\tiny [GM$\langle$B]}  
	 ANSWER: The grid you see is filled with varying shades of black and white. The shades contribute significantly to the overall visual impact of the grid by creating depth and dimension. The contrast between the black and white lines and the varying shades of gray (or black and white) enhances the grid's symmetry and complexity. The darker shades often emphasize the lines and contours, while lighter shades can create a sense of space or depth. The use of different shades can also play with the viewer's eye, making 
	  } 
	   } 
	   } 
	  \\ 
 

    \theutterance \stepcounter{utterance}  

    & & & \multicolumn{2}{p{0.3\linewidth}}{\cellcolor[rgb]{0.95,0.95,0.95}{%
	\makecell[{{p{\linewidth}}}]{% 
	  \tt {\tiny [GM$|$GM]}  
	 continue 
	  } 
	   } 
	   } 
	 & & \\ 
 

    \theutterance \stepcounter{utterance}  

    & & \multicolumn{4}{p{0.6\linewidth}}{\cellcolor[rgb]{0.9,0.9,0.9}{%
	\makecell[{{p{\linewidth}}}]{% 
	  \tt {\tiny [GM$\rangle$B]}  
	 Now ask a question in order to find out new aspects of my grid that may be different to your grid. Start with "QUESTION:" and do not add anything else. 
	  } 
	   } 
	   } 
	 & \\ 
 

    \theutterance \stepcounter{utterance}  

    & & & \multicolumn{4}{p{0.6\linewidth}}{\cellcolor[rgb]{1,0.85,0.72}{%
	\makecell[{{p{\linewidth}}}]{% 
	  \tt {\tiny [GM$\langle$B]}  
	 QUESTION: Are there any additional elements or shapes within the grid that are not explicitly shown in my description? For example, are there any additional Xs, ▢'s, or other symbols that might be present? If so, what are their positions or roles within the grid? 
	  } 
	   } 
	   } 
	  \\ 
 

    \theutterance \stepcounter{utterance}  

    & & & \multicolumn{2}{p{0.3\linewidth}}{\cellcolor[rgb]{0.95,0.95,0.95}{%
	\makecell[{{p{\linewidth}}}]{% 
	  \tt {\tiny [GM$|$GM]}  
	 continue 
	  } 
	   } 
	   } 
	 & & \\ 
 

    \theutterance \stepcounter{utterance}  

    & & \multicolumn{4}{p{0.6\linewidth}}{\cellcolor[rgb]{0.9,0.9,0.9}{%
	\makecell[{{p{\linewidth}}}]{% 
	  \tt {\tiny [A$\langle$GM]}  
	 ANSWER: The grid you see is filled with varying shades of black and white. The shades contribute significantly to the overall visual impact of the grid by creating depth and dimension. The contrast between the black and white lines and the varying shades of gray (or black and white) enhances the grid's symmetry and complexity. The darker shades often emphasize the lines and contours, while lighter shades can create a sense of space or depth. The use of different shades can also play with the viewer's eye, making\\ \tt QUESTION: Are there any additional elements or shapes within the grid that are not explicitly shown in my description? For example, are there any additional Xs, ▢'s, or other symbols that might be present? If so, what are their positions or roles within the grid? Start your answer with "ANSWER:" and do not add anything else. 
	  } 
	   } 
	   } 
	 & \\ 
 

    \theutterance \stepcounter{utterance}  

    & \multicolumn{4}{p{0.6\linewidth}}{\cellcolor[rgb]{0.8,1,0.9}{%
	\makecell[{{p{\linewidth}}}]{% 
	  \tt {\tiny [A$\rangle$GM]}  
	 ANSWER: There are no additional elements or shapes within the grid that are not explicitly shown in your description. The grid provided consists solely of the described alternating Xs and ▢'s, forming a symmetrical and visually interesting pattern. The symmetry and simplicity of the grid make it easy to focus on the pattern itself, and the lack of additional elements or shapes does not detract from the overall aesthetic or complexity of the grid. The grid's design is purely based on the alternating Xs and 
	  } 
	   } 
	   } 
	 & & \\ 
 

    \theutterance \stepcounter{utterance}  

    & & & \multicolumn{2}{p{0.3\linewidth}}{\cellcolor[rgb]{0.95,0.95,0.95}{%
	\makecell[{{p{\linewidth}}}]{% 
	  \tt {\tiny [GM$|$GM]}  
	 continue 
	  } 
	   } 
	   } 
	 & & \\ 
 

    \theutterance \stepcounter{utterance}  

    & & \multicolumn{4}{p{0.6\linewidth}}{\cellcolor[rgb]{0.9,0.9,0.9}{%
	\makecell[{{p{\linewidth}}}]{% 
	  \tt {\tiny [A$\langle$GM]}  
	 Now ask a question in order to find out new aspects of my grid that may be different to your grid. Start with "QUESTION:" and do not add anything else. 
	  } 
	   } 
	   } 
	 & \\ 
 

    \theutterance \stepcounter{utterance}  

    & \multicolumn{4}{p{0.6\linewidth}}{\cellcolor[rgb]{0.8,1,0.9}{%
	\makecell[{{p{\linewidth}}}]{% 
	  \tt {\tiny [A$\rangle$GM]}  
	 QUESTION: Are there any specific patterns or motifs within the grid that are not immediately apparent from the description? For instance, are there any hidden patterns, such as a repeating sequence of shapes or lines that are only visible upon closer inspection? If so, what are these patterns and how do they contribute to the unique character of your grid? 
	  } 
	   } 
	   } 
	 & & \\ 
 

    \theutterance \stepcounter{utterance}  

    & & & \multicolumn{2}{p{0.3\linewidth}}{\cellcolor[rgb]{0.95,0.95,0.95}{%
	\makecell[{{p{\linewidth}}}]{% 
	  \tt {\tiny [GM$|$GM]}  
	 continue 
	  } 
	   } 
	   } 
	 & & \\ 
 

    \theutterance \stepcounter{utterance}  

    & & \multicolumn{4}{p{0.6\linewidth}}{\cellcolor[rgb]{0.9,0.9,0.9}{%
	\makecell[{{p{\linewidth}}}]{% 
	  \tt {\tiny [GM$\rangle$B]}  
	 ANSWER: There are no additional elements or shapes within the grid that are not explicitly shown in your description. The grid provided consists solely of the described alternating Xs and ▢'s, forming a symmetrical and visually interesting pattern. The symmetry and simplicity of the grid make it easy to focus on the pattern itself, and the lack of additional elements or shapes does not detract from the overall aesthetic or complexity of the grid. The grid's design is purely based on the alternating Xs and\\ \tt QUESTION: Are there any specific patterns or motifs within the grid that are not immediately apparent from the description? For instance, are there any hidden patterns, such as a repeating sequence of shapes or lines that are only visible upon closer inspection? If so, what are these patterns and how do they contribute to the unique character of your grid? Start your answer with "ANSWER:" and do not add anything else. 
	  } 
	   } 
	   } 
	 & \\ 
 

    \theutterance \stepcounter{utterance}  

    & & & \multicolumn{4}{p{0.6\linewidth}}{\cellcolor[rgb]{1,0.85,0.72}{%
	\makecell[{{p{\linewidth}}}]{% 
	  \tt {\tiny [GM$\langle$B]}  
	 ANSWER: There are no specific patterns or motifs within the grid that are not immediately apparent from the description. The grid is straightforward in its design, with alternating Xs and ▢'s forming a simple symmetrical pattern. The description provided was designed to be as clear as possible, focusing on the basic structure and symmetry of the grid. The lack of hidden patterns or complex motifs in this grid contributes to its simplicity and clarity, making it easy to understand and appreciate the symmetry and aesthetic appeal of 
	  } 
	   } 
	   } 
	  \\ 
 

    \theutterance \stepcounter{utterance}  

    & & & \multicolumn{2}{p{0.3\linewidth}}{\cellcolor[rgb]{0.95,0.95,0.95}{%
	\makecell[{{p{\linewidth}}}]{% 
	  \tt {\tiny [GM$|$GM]}  
	 continue 
	  } 
	   } 
	   } 
	 & & \\ 
 

    \theutterance \stepcounter{utterance}  

    & & \multicolumn{4}{p{0.6\linewidth}}{\cellcolor[rgb]{0.9,0.9,0.9}{%
	\makecell[{{p{\linewidth}}}]{% 
	  \tt {\tiny [GM$\rangle$B]}  
	 Now ask a question in order to find out new aspects of my grid that may be different to your grid. Start with "QUESTION:" and do not add anything else. 
	  } 
	   } 
	   } 
	 & \\ 
 

    \theutterance \stepcounter{utterance}  

    & & & \multicolumn{4}{p{0.6\linewidth}}{\cellcolor[rgb]{1,0.85,0.72}{%
	\makecell[{{p{\linewidth}}}]{% 
	  \tt {\tiny [GM$\langle$B]}  
	 QUESTION: Are there any cultural or historical significance attached to the Xs and ▢'s in your grid that might not be immediately obvious to someone unfamiliar with the context? For example, are there any specific cultural references, patterns, or meanings that are embedded within the grid's design? If so, what are these elements and how do they contribute to the unique character of your grid? 
	  } 
	   } 
	   } 
	  \\ 
 

    \theutterance \stepcounter{utterance}  

    & & & \multicolumn{2}{p{0.3\linewidth}}{\cellcolor[rgb]{0.95,0.95,0.95}{%
	\makecell[{{p{\linewidth}}}]{% 
	  \tt {\tiny [GM$|$GM]}  
	 continue 
	  } 
	   } 
	   } 
	 & & \\ 
 

    \theutterance \stepcounter{utterance}  

    & & \multicolumn{4}{p{0.6\linewidth}}{\cellcolor[rgb]{0.9,0.9,0.9}{%
	\makecell[{{p{\linewidth}}}]{% 
	  \tt {\tiny [A$\langle$GM]}  
	 ANSWER: There are no specific patterns or motifs within the grid that are not immediately apparent from the description. The grid is straightforward in its design, with alternating Xs and ▢'s forming a simple symmetrical pattern. The description provided was designed to be as clear as possible, focusing on the basic structure and symmetry of the grid. The lack of hidden patterns or complex motifs in this grid contributes to its simplicity and clarity, making it easy to understand and appreciate the symmetry and aesthetic appeal of\\ \tt QUESTION: Are there any cultural or historical significance attached to the Xs and ▢'s in your grid that might not be immediately obvious to someone unfamiliar with the context? For example, are there any specific cultural references, patterns, or meanings that are embedded within the grid's design? If so, what are these elements and how do they contribute to the unique character of your grid? Start your answer with "ANSWER:" and do not add anything else. 
	  } 
	   } 
	   } 
	 & \\ 
 

    \theutterance \stepcounter{utterance}  

    & \multicolumn{4}{p{0.6\linewidth}}{\cellcolor[rgb]{0.8,1,0.9}{%
	\makecell[{{p{\linewidth}}}]{% 
	  \tt {\tiny [A$\rangle$GM]}  
	 ANSWER: The grid you provided is devoid of cultural or historical significance attached to the Xs and ▢'s in the context of specific cultural references or meanings. The symbols are purely functional in their design, serving as a basis for the grid's symmetry and aesthetic appeal. The Xs and ▢'s are chosen for their simplicity and the way they can create a balanced and harmonious pattern when arranged in a grid format. The absence of cultural or historical significance does not detract from 
	  } 
	   } 
	   } 
	 & & \\ 
 

    \theutterance \stepcounter{utterance}  

    & & & \multicolumn{2}{p{0.3\linewidth}}{\cellcolor[rgb]{0.95,0.95,0.95}{%
	\makecell[{{p{\linewidth}}}]{% 
	  \tt {\tiny [GM$|$GM]}  
	 continue 
	  } 
	   } 
	   } 
	 & & \\ 
 

    \theutterance \stepcounter{utterance}  

    & & \multicolumn{4}{p{0.6\linewidth}}{\cellcolor[rgb]{0.9,0.9,0.9}{%
	\makecell[{{p{\linewidth}}}]{% 
	  \tt {\tiny [A$\langle$GM]}  
	 Now ask a question in order to find out new aspects of my grid that may be different to your grid. Start with "QUESTION:" and do not add anything else. 
	  } 
	   } 
	   } 
	 & \\ 
 

    \theutterance \stepcounter{utterance}  

    & \multicolumn{4}{p{0.6\linewidth}}{\cellcolor[rgb]{0.8,1,0.9}{%
	\makecell[{{p{\linewidth}}}]{% 
	  \tt {\tiny [A$\rangle$GM]}  
	 QUESTION: Are there any mathematical properties or characteristics of the grid that are not immediately obvious from the description? For example, are there any specific mathematical patterns or symmetries that are evident upon closer inspection? If so, what are these patterns and how do they contribute to the unique character of your grid? 
	  } 
	   } 
	   } 
	 & & \\ 
 

    \theutterance \stepcounter{utterance}  

    & & & \multicolumn{2}{p{0.3\linewidth}}{\cellcolor[rgb]{0.95,0.95,0.95}{%
	\makecell[{{p{\linewidth}}}]{% 
	  \tt {\tiny [GM$|$GM]}  
	 continue 
	  } 
	   } 
	   } 
	 & & \\ 
 

    \theutterance \stepcounter{utterance}  

    & & \multicolumn{4}{p{0.6\linewidth}}{\cellcolor[rgb]{0.9,0.9,0.9}{%
	\makecell[{{p{\linewidth}}}]{% 
	  \tt {\tiny [GM$\rangle$B]}  
	 ANSWER: The grid you provided is devoid of cultural or historical significance attached to the Xs and ▢'s in the context of specific cultural references or meanings. The symbols are purely functional in their design, serving as a basis for the grid's symmetry and aesthetic appeal. The Xs and ▢'s are chosen for their simplicity and the way they can create a balanced and harmonious pattern when arranged in a grid format. The absence of cultural or historical significance does not detract from\\ \tt QUESTION: Are there any mathematical properties or characteristics of the grid that are not immediately obvious from the description? For example, are there any specific mathematical patterns or symmetries that are evident upon closer inspection? If so, what are these patterns and how do they contribute to the unique character of your grid? Start your answer with "ANSWER:" and do not add anything else. 
	  } 
	   } 
	   } 
	 & \\ 
 

    \theutterance \stepcounter{utterance}  

    & & & \multicolumn{4}{p{0.6\linewidth}}{\cellcolor[rgb]{1,0.85,0.72}{%
	\makecell[{{p{\linewidth}}}]{% 
	  \tt {\tiny [GM$\langle$B]}  
	 ANSWER: Upon closer inspection, there are indeed mathematical properties and symmetries that are not immediately obvious from the description. The grid exhibits a form of symmetry known as rotational symmetry. When you rotate the grid by 90 degrees, 180 degrees, or 270 degrees, it remains visually identical. This symmetry is a result of the grid's uniformity and the way the Xs and ▢'s are arranged in a staggered pattern.\\ \tt \\ \tt Another mathematical property is the grid's uniformity 
	  } 
	   } 
	   } 
	  \\ 
 

    \theutterance \stepcounter{utterance}  

    & & & \multicolumn{2}{p{0.3\linewidth}}{\cellcolor[rgb]{0.95,0.95,0.95}{%
	\makecell[{{p{\linewidth}}}]{% 
	  \tt {\tiny [GM$|$GM]}  
	 continue 
	  } 
	   } 
	   } 
	 & & \\ 
 

    \theutterance \stepcounter{utterance}  

    & & \multicolumn{4}{p{0.6\linewidth}}{\cellcolor[rgb]{0.9,0.9,0.9}{%
	\makecell[{{p{\linewidth}}}]{% 
	  \tt {\tiny [GM$\rangle$B]}  
	 Now come to a decision. What do you think: are your grid and the other grid described the same?  Write "DECISION: same grid" if you think they are the same grid or "DECISION: different grids" if you think they are different grids. Do not add anything else.\\ \tt  
	  } 
	   } 
	   } 
	 & \\ 
 

    \theutterance \stepcounter{utterance}  

    & & & \multicolumn{4}{p{0.6\linewidth}}{\cellcolor[rgb]{1,0.85,0.72}{%
	\makecell[{{p{\linewidth}}}]{% 
	  \tt {\tiny [GM$\langle$B]}  
	 DECISION: same grid 
	  } 
	   } 
	   } 
	  \\ 
 

    \theutterance \stepcounter{utterance}  

    & & & \multicolumn{2}{p{0.3\linewidth}}{\cellcolor[rgb]{0.95,0.95,0.95}{%
	\makecell[{{p{\linewidth}}}]{% 
	  \tt {\tiny [GM$|$GM]}  
	 continue 
	  } 
	   } 
	   } 
	 & & \\ 
 

    \theutterance \stepcounter{utterance}  

    & & & \multicolumn{2}{p{0.3\linewidth}}{\cellcolor[rgb]{0.95,0.95,0.95}{%
	\makecell[{{p{\linewidth}}}]{% 
	  \tt {\tiny [GM$|$GM]}  
	 loss 
	  } 
	   } 
	   } 
	 & & \\ 
 

    \theutterance \stepcounter{utterance}  

    & & \multicolumn{4}{p{0.6\linewidth}}{\cellcolor[rgb]{0.9,0.9,0.9}{%
	\makecell[{{p{\linewidth}}}]{% 
	  \tt {\tiny [A$\langle$GM]}  
	 ANSWER: Upon closer inspection, there are indeed mathematical properties and symmetries that are not immediately obvious from the description. The grid exhibits a form of symmetry known as rotational symmetry. When you rotate the grid by 90 degrees, 180 degrees, or 270 degrees, it remains visually identical. This symmetry is a result of the grid's uniformity and the way the Xs and ▢'s are arranged in a staggered pattern.\\ \tt \\ \tt Another mathematical property is the grid's uniformity\\ \tt Now come to a decision. What do you think: are your grid and the other grid described the same?  Write "DECISION: same grid" if you think they are the same grid or "DECISION: different grids" if you think they are different grids. Do not add anything else.\\ \tt  
	  } 
	   } 
	   } 
	 & \\ 
 

    \theutterance \stepcounter{utterance}  

    & \multicolumn{4}{p{0.6\linewidth}}{\cellcolor[rgb]{0.8,1,0.9}{%
	\makecell[{{p{\linewidth}}}]{% 
	  \tt {\tiny [A$\rangle$GM]}  
	 DECISION: different grids 
	  } 
	   } 
	   } 
	 & & \\ 
 

    \theutterance \stepcounter{utterance}  

    & & & \multicolumn{2}{p{0.3\linewidth}}{\cellcolor[rgb]{0.95,0.95,0.95}{%
	\makecell[{{p{\linewidth}}}]{% 
	  \tt {\tiny [GM$|$GM]}  
	 continue 
	  } 
	   } 
	   } 
	 & & \\ 
 

    \theutterance \stepcounter{utterance}  

    & & & \multicolumn{2}{p{0.3\linewidth}}{\cellcolor[rgb]{0.95,0.95,0.95}{%
	\makecell[{{p{\linewidth}}}]{% 
	  \tt {\tiny [GM$|$GM]}  
	 success 
	  } 
	   } 
	   } 
	 & & \\ 
 

\end{supertabular}
}

\end{document}


\documentclass{article}
\usepackage{colortbl}
\usepackage{makecell}
\usepackage{multirow}
\usepackage{supertabular}

\begin{document}

\newcounter{utterance}

\twocolumn

{ \footnotesize  \setcounter{utterance}{1}
\setlength{\tabcolsep}{0pt}
\begin{supertabular}{c@{$\;$}|p{.15\linewidth}@{}p{.15\linewidth}p{.15\linewidth}p{.15\linewidth}p{.15\linewidth}p{.15\linewidth}}

    \# & $\;$A & \multicolumn{4}{c}{Game Master} & $\;\:$B\\
    \hline 

    \theutterance \stepcounter{utterance}  

    & & \multicolumn{4}{p{0.6\linewidth}}{\cellcolor[rgb]{0.9,0.9,0.9}{%
	\makecell[{{p{\linewidth}}}]{% 
	  \tt {\tiny [A$\langle$GM]}  
	 You are participating in a collaborative guessing game.\\ \tt The goal is to find out whether this grid and another picture only I can see, are the same.\\ \tt Please describe your grid first. Then, I will provide my description and we can ask each other questions about the grids to figure out whether they are the same. Now start your short grid description with "DESCRIPTION:" followed by the description. Do not add anything else. This is your grid:\\ \tt X X X X X\\ \tt X ▢ ▢ ▢ ▢\\ \tt X X X ▢ ▢\\ \tt X ▢ ▢ ▢ ▢\\ \tt X ▢ ▢ ▢ ▢ 
	  } 
	   } 
	   } 
	 & \\ 
 

    \theutterance \stepcounter{utterance}  

    & \multicolumn{4}{p{0.6\linewidth}}{\cellcolor[rgb]{0.8,1,0.9}{%
	\makecell[{{p{\linewidth}}}]{% 
	  \tt {\tiny [A$\rangle$GM]}  
	 DESCRIPTION: The grid provided consists of a series of X's and ▢'s. Each row and column contains the same number of X's and ▢'s, and the grid is filled in such a way that it forms a specific pattern. The pattern is not immediately obvious from the description alone, but it is symmetrical and balanced, with an equal number of horizontal and vertical lines of X's and ▢'s.\\ \tt \\ \tt Let's start asking questions to narrow down the possibilities. For instance, can 
	  } 
	   } 
	   } 
	 & & \\ 
 

    \theutterance \stepcounter{utterance}  

    & & & \multicolumn{2}{p{0.3\linewidth}}{\cellcolor[rgb]{0.95,0.95,0.95}{%
	\makecell[{{p{\linewidth}}}]{% 
	  \tt {\tiny [GM$|$GM]}  
	 continue 
	  } 
	   } 
	   } 
	 & & \\ 
 

    \theutterance \stepcounter{utterance}  

    & & \multicolumn{4}{p{0.6\linewidth}}{\cellcolor[rgb]{0.9,0.9,0.9}{%
	\makecell[{{p{\linewidth}}}]{% 
	  \tt {\tiny [GM$\rangle$B]}  
	 You are participating in a collaborative guessing game.\\ \tt The goal is to find out whether this grid and another picture only I can see, are the same.\\ \tt Please describe your grid first. Then, I will provide my description and we can ask each other questions about the grids to figure out whether they are the same. Now start your short grid description with "DESCRIPTION:" followed by the description. Do not add anything else. This is your grid:\\ \tt X X X X X\\ \tt ▢ ▢ ▢ ▢ X\\ \tt ▢ ▢ X X X\\ \tt ▢ ▢ ▢ ▢ X\\ \tt ▢ ▢ ▢ ▢ X 
	  } 
	   } 
	   } 
	 & \\ 
 

    \theutterance \stepcounter{utterance}  

    & & & \multicolumn{4}{p{0.6\linewidth}}{\cellcolor[rgb]{1,0.85,0.72}{%
	\makecell[{{p{\linewidth}}}]{% 
	  \tt {\tiny [GM$\langle$B]}  
	 DESCRIPTION: The grid provided consists of a series of X's and ▢'s. The X's are placed in a staggered pattern, with some X's positioned vertically and others horizontally. The grid spans 5 rows and 4 columns, making it a 20-cell grid. The ▢ symbols are placed in a repeating pattern, alternating between X and ▢.\\ \tt \\ \tt Now, please describe your grid. 
	  } 
	   } 
	   } 
	  \\ 
 

    \theutterance \stepcounter{utterance}  

    & & & \multicolumn{2}{p{0.3\linewidth}}{\cellcolor[rgb]{0.95,0.95,0.95}{%
	\makecell[{{p{\linewidth}}}]{% 
	  \tt {\tiny [GM$|$GM]}  
	 continue 
	  } 
	   } 
	   } 
	 & & \\ 
 

    \theutterance \stepcounter{utterance}  

    & & \multicolumn{4}{p{0.6\linewidth}}{\cellcolor[rgb]{0.9,0.9,0.9}{%
	\makecell[{{p{\linewidth}}}]{% 
	  \tt {\tiny [GM$\rangle$B]}  
	 This is my DESCRIPTION: The grid provided consists of a series of X's and ▢'s. Each row and column contains the same number of X's and ▢'s, and the grid is filled in such a way that it forms a specific pattern. The pattern is not immediately obvious from the description alone, but it is symmetrical and balanced, with an equal number of horizontal and vertical lines of X's and ▢'s.\\ \tt \\ \tt Let's start asking questions to narrow down the possibilities. For instance, can\\ \tt Now ask a question in order to find out new aspects of my grid that may be different to your grid. Start with "QUESTION:" and do not add anything else. 
	  } 
	   } 
	   } 
	 & \\ 
 

    \theutterance \stepcounter{utterance}  

    & & & \multicolumn{4}{p{0.6\linewidth}}{\cellcolor[rgb]{1,0.85,0.72}{%
	\makecell[{{p{\linewidth}}}]{% 
	  \tt {\tiny [GM$\langle$B]}  
	 QUESTION: Are there any specific numbers of X's or ▢'s in each row or column that are consistently higher or lower than the rest? For example, are there any rows or columns that have more X's or ▢'s than others? 
	  } 
	   } 
	   } 
	  \\ 
 

    \theutterance \stepcounter{utterance}  

    & & & \multicolumn{2}{p{0.3\linewidth}}{\cellcolor[rgb]{0.95,0.95,0.95}{%
	\makecell[{{p{\linewidth}}}]{% 
	  \tt {\tiny [GM$|$GM]}  
	 continue 
	  } 
	   } 
	   } 
	 & & \\ 
 

    \theutterance \stepcounter{utterance}  

    & & \multicolumn{4}{p{0.6\linewidth}}{\cellcolor[rgb]{0.9,0.9,0.9}{%
	\makecell[{{p{\linewidth}}}]{% 
	  \tt {\tiny [A$\langle$GM]}  
	 This is my DESCRIPTION: The grid provided consists of a series of X's and ▢'s. The X's are placed in a staggered pattern, with some X's positioned vertically and others horizontally. The grid spans 5 rows and 4 columns, making it a 20-cell grid. The ▢ symbols are placed in a repeating pattern, alternating between X and ▢.\\ \tt \\ \tt Now, please describe your grid.\\ \tt QUESTION: Are there any specific numbers of X's or ▢'s in each row or column that are consistently higher or lower than the rest? For example, are there any rows or columns that have more X's or ▢'s than others? Start your answer with "ANSWER:" and do not add anything else. 
	  } 
	   } 
	   } 
	 & \\ 
 

    \theutterance \stepcounter{utterance}  

    & \multicolumn{4}{p{0.6\linewidth}}{\cellcolor[rgb]{0.8,1,0.9}{%
	\makecell[{{p{\linewidth}}}]{% 
	  \tt {\tiny [A$\rangle$GM]}  
	 ANSWER: In the grid you've described, there are no specific numbers of X's or ▢'s that are consistently higher or lower than the rest. The grid is balanced, with an equal distribution of X's and ▢'s across rows and columns. Each row and column contains an equal number of X's and ▢'s, and the alternating pattern ensures that no single row or column has an excess or deficit of either symbol.\\ \tt \\ \tt Let's continue with the guessing game. 
	  } 
	   } 
	   } 
	 & & \\ 
 

    \theutterance \stepcounter{utterance}  

    & & & \multicolumn{2}{p{0.3\linewidth}}{\cellcolor[rgb]{0.95,0.95,0.95}{%
	\makecell[{{p{\linewidth}}}]{% 
	  \tt {\tiny [GM$|$GM]}  
	 continue 
	  } 
	   } 
	   } 
	 & & \\ 
 

    \theutterance \stepcounter{utterance}  

    & & \multicolumn{4}{p{0.6\linewidth}}{\cellcolor[rgb]{0.9,0.9,0.9}{%
	\makecell[{{p{\linewidth}}}]{% 
	  \tt {\tiny [A$\langle$GM]}  
	 Now ask a question in order to find out new aspects of my grid that may be different to your grid. Start with "QUESTION:" and do not add anything else. 
	  } 
	   } 
	   } 
	 & \\ 
 

    \theutterance \stepcounter{utterance}  

    & \multicolumn{4}{p{0.6\linewidth}}{\cellcolor[rgb]{0.8,1,0.9}{%
	\makecell[{{p{\linewidth}}}]{% 
	  \tt {\tiny [A$\rangle$GM]}  
	 QUESTION: Are there any particular positions within the grid that seem to be more significant or noteworthy compared to the rest? For example, are there any X's or ▢'s that are particularly close to the edges or corners of the grid, or are there any positions that seem to repeat more frequently than others? 
	  } 
	   } 
	   } 
	 & & \\ 
 

    \theutterance \stepcounter{utterance}  

    & & & \multicolumn{2}{p{0.3\linewidth}}{\cellcolor[rgb]{0.95,0.95,0.95}{%
	\makecell[{{p{\linewidth}}}]{% 
	  \tt {\tiny [GM$|$GM]}  
	 continue 
	  } 
	   } 
	   } 
	 & & \\ 
 

    \theutterance \stepcounter{utterance}  

    & & \multicolumn{4}{p{0.6\linewidth}}{\cellcolor[rgb]{0.9,0.9,0.9}{%
	\makecell[{{p{\linewidth}}}]{% 
	  \tt {\tiny [GM$\rangle$B]}  
	 ANSWER: In the grid you've described, there are no specific numbers of X's or ▢'s that are consistently higher or lower than the rest. The grid is balanced, with an equal distribution of X's and ▢'s across rows and columns. Each row and column contains an equal number of X's and ▢'s, and the alternating pattern ensures that no single row or column has an excess or deficit of either symbol.\\ \tt \\ \tt Let's continue with the guessing game.\\ \tt QUESTION: Are there any particular positions within the grid that seem to be more significant or noteworthy compared to the rest? For example, are there any X's or ▢'s that are particularly close to the edges or corners of the grid, or are there any positions that seem to repeat more frequently than others? Start your answer with "ANSWER:" and do not add anything else. 
	  } 
	   } 
	   } 
	 & \\ 
 

    \theutterance \stepcounter{utterance}  

    & & & \multicolumn{4}{p{0.6\linewidth}}{\cellcolor[rgb]{1,0.85,0.72}{%
	\makecell[{{p{\linewidth}}}]{% 
	  \tt {\tiny [GM$\langle$B]}  
	 ANSWER: In the grid you've described, there are no particular positions that seem to be more significant or noteworthy compared to the rest. The symmetry and balance of the grid ensure that each symbol is equally distributed across rows and columns. There are no X's or ▢'s that are particularly close to the edges or corners, and no positions that repeat more frequently than others. The symmetry and uniformity of the grid make it challenging to identify any standout positions based on this description alone.\\ \tt \\ \tt Let 
	  } 
	   } 
	   } 
	  \\ 
 

    \theutterance \stepcounter{utterance}  

    & & & \multicolumn{2}{p{0.3\linewidth}}{\cellcolor[rgb]{0.95,0.95,0.95}{%
	\makecell[{{p{\linewidth}}}]{% 
	  \tt {\tiny [GM$|$GM]}  
	 continue 
	  } 
	   } 
	   } 
	 & & \\ 
 

    \theutterance \stepcounter{utterance}  

    & & \multicolumn{4}{p{0.6\linewidth}}{\cellcolor[rgb]{0.9,0.9,0.9}{%
	\makecell[{{p{\linewidth}}}]{% 
	  \tt {\tiny [GM$\rangle$B]}  
	 Now ask a question in order to find out new aspects of my grid that may be different to your grid. Start with "QUESTION:" and do not add anything else. 
	  } 
	   } 
	   } 
	 & \\ 
 

    \theutterance \stepcounter{utterance}  

    & & & \multicolumn{4}{p{0.6\linewidth}}{\cellcolor[rgb]{1,0.85,0.72}{%
	\makecell[{{p{\linewidth}}}]{% 
	  \tt {\tiny [GM$\langle$B]}  
	 QUESTION: Are there any particular patterns or visual cues within the grid that might give away its unique characteristics? For example, are there any specific shapes or groupings of symbols that stand out due to their arrangement or color? If so, what are they and how do they contribute to the overall pattern? 
	  } 
	   } 
	   } 
	  \\ 
 

    \theutterance \stepcounter{utterance}  

    & & & \multicolumn{2}{p{0.3\linewidth}}{\cellcolor[rgb]{0.95,0.95,0.95}{%
	\makecell[{{p{\linewidth}}}]{% 
	  \tt {\tiny [GM$|$GM]}  
	 continue 
	  } 
	   } 
	   } 
	 & & \\ 
 

    \theutterance \stepcounter{utterance}  

    & & \multicolumn{4}{p{0.6\linewidth}}{\cellcolor[rgb]{0.9,0.9,0.9}{%
	\makecell[{{p{\linewidth}}}]{% 
	  \tt {\tiny [A$\langle$GM]}  
	 ANSWER: In the grid you've described, there are no particular positions that seem to be more significant or noteworthy compared to the rest. The symmetry and balance of the grid ensure that each symbol is equally distributed across rows and columns. There are no X's or ▢'s that are particularly close to the edges or corners, and no positions that repeat more frequently than others. The symmetry and uniformity of the grid make it challenging to identify any standout positions based on this description alone.\\ \tt \\ \tt Let\\ \tt QUESTION: Are there any particular patterns or visual cues within the grid that might give away its unique characteristics? For example, are there any specific shapes or groupings of symbols that stand out due to their arrangement or color? If so, what are they and how do they contribute to the overall pattern? Start your answer with "ANSWER:" and do not add anything else. 
	  } 
	   } 
	   } 
	 & \\ 
 

    \theutterance \stepcounter{utterance}  

    & \multicolumn{4}{p{0.6\linewidth}}{\cellcolor[rgb]{0.8,1,0.9}{%
	\makecell[{{p{\linewidth}}}]{% 
	  \tt {\tiny [A$\rangle$GM]}  
	 ANSWER: The grid you've described is quite symmetrical and balanced, with no discernible patterns or visual cues that would immediately indicate its unique characteristics. The symbols are evenly distributed across rows and columns, and there are no specific shapes or groupings that stand out due to their arrangement or color. The symmetry and uniformity of the grid make it difficult to identify any particular patterns that would give it a unique visual identity beyond its basic structure.\\ \tt \\ \tt Given the lack of distinctive patterns or visual cues, it 
	  } 
	   } 
	   } 
	 & & \\ 
 

    \theutterance \stepcounter{utterance}  

    & & & \multicolumn{2}{p{0.3\linewidth}}{\cellcolor[rgb]{0.95,0.95,0.95}{%
	\makecell[{{p{\linewidth}}}]{% 
	  \tt {\tiny [GM$|$GM]}  
	 continue 
	  } 
	   } 
	   } 
	 & & \\ 
 

    \theutterance \stepcounter{utterance}  

    & & \multicolumn{4}{p{0.6\linewidth}}{\cellcolor[rgb]{0.9,0.9,0.9}{%
	\makecell[{{p{\linewidth}}}]{% 
	  \tt {\tiny [A$\langle$GM]}  
	 Now ask a question in order to find out new aspects of my grid that may be different to your grid. Start with "QUESTION:" and do not add anything else. 
	  } 
	   } 
	   } 
	 & \\ 
 

    \theutterance \stepcounter{utterance}  

    & \multicolumn{4}{p{0.6\linewidth}}{\cellcolor[rgb]{0.8,1,0.9}{%
	\makecell[{{p{\linewidth}}}]{% 
	  \tt {\tiny [A$\rangle$GM]}  
	 QUESTION: Are there any additional details or hidden features within the grid that might help differentiate it from another grid? For instance, are there any specific mathematical properties or relationships between the symbols that aren't immediately obvious? For example, could the grid be analyzed in terms of its symmetry, balance, or any other mathematical property that might reveal its unique characteristics? If so, what are they and how might they contribute to the overall pattern? 
	  } 
	   } 
	   } 
	 & & \\ 
 

    \theutterance \stepcounter{utterance}  

    & & & \multicolumn{2}{p{0.3\linewidth}}{\cellcolor[rgb]{0.95,0.95,0.95}{%
	\makecell[{{p{\linewidth}}}]{% 
	  \tt {\tiny [GM$|$GM]}  
	 continue 
	  } 
	   } 
	   } 
	 & & \\ 
 

    \theutterance \stepcounter{utterance}  

    & & \multicolumn{4}{p{0.6\linewidth}}{\cellcolor[rgb]{0.9,0.9,0.9}{%
	\makecell[{{p{\linewidth}}}]{% 
	  \tt {\tiny [GM$\rangle$B]}  
	 ANSWER: The grid you've described is quite symmetrical and balanced, with no discernible patterns or visual cues that would immediately indicate its unique characteristics. The symbols are evenly distributed across rows and columns, and there are no specific shapes or groupings that stand out due to their arrangement or color. The symmetry and uniformity of the grid make it difficult to identify any particular patterns that would give it a unique visual identity beyond its basic structure.\\ \tt \\ \tt Given the lack of distinctive patterns or visual cues, it\\ \tt QUESTION: Are there any additional details or hidden features within the grid that might help differentiate it from another grid? For instance, are there any specific mathematical properties or relationships between the symbols that aren't immediately obvious? For example, could the grid be analyzed in terms of its symmetry, balance, or any other mathematical property that might reveal its unique characteristics? If so, what are they and how might they contribute to the overall pattern? Start your answer with "ANSWER:" and do not add anything else. 
	  } 
	   } 
	   } 
	 & \\ 
 

    \theutterance \stepcounter{utterance}  

    & & & \multicolumn{4}{p{0.6\linewidth}}{\cellcolor[rgb]{1,0.85,0.72}{%
	\makecell[{{p{\linewidth}}}]{% 
	  \tt {\tiny [GM$\langle$B]}  
	 ANSWER: The grid you've described is indeed symmetrical and balanced, with no additional details or hidden features that would immediately differentiate it from another grid. The symmetry is maintained through the uniform distribution of X's and ▢'s across rows and columns, and there are no specific mathematical properties or relationships between the symbols that stand out. The balance and uniformity ensure that the grid remains visually uniform and does not exhibit any distinctive features beyond its basic structure.\\ \tt \\ \tt Given the lack of distinctive patterns or visual 
	  } 
	   } 
	   } 
	  \\ 
 

    \theutterance \stepcounter{utterance}  

    & & & \multicolumn{2}{p{0.3\linewidth}}{\cellcolor[rgb]{0.95,0.95,0.95}{%
	\makecell[{{p{\linewidth}}}]{% 
	  \tt {\tiny [GM$|$GM]}  
	 continue 
	  } 
	   } 
	   } 
	 & & \\ 
 

    \theutterance \stepcounter{utterance}  

    & & \multicolumn{4}{p{0.6\linewidth}}{\cellcolor[rgb]{0.9,0.9,0.9}{%
	\makecell[{{p{\linewidth}}}]{% 
	  \tt {\tiny [GM$\rangle$B]}  
	 Now ask a question in order to find out new aspects of my grid that may be different to your grid. Start with "QUESTION:" and do not add anything else. 
	  } 
	   } 
	   } 
	 & \\ 
 

    \theutterance \stepcounter{utterance}  

    & & & \multicolumn{4}{p{0.6\linewidth}}{\cellcolor[rgb]{1,0.85,0.72}{%
	\makecell[{{p{\linewidth}}}]{% 
	  \tt {\tiny [GM$\langle$B]}  
	 QUESTION: Are there any cultural or historical significance attached to the symbols used in this grid? For example, do the X's and ▢'s have a specific meaning in a particular culture or context? If so, what is that meaning and how might it influence the perception of the grid? If not, how do you interpret the symbols' significance in the context of this guessing game? 
	  } 
	   } 
	   } 
	  \\ 
 

    \theutterance \stepcounter{utterance}  

    & & & \multicolumn{2}{p{0.3\linewidth}}{\cellcolor[rgb]{0.95,0.95,0.95}{%
	\makecell[{{p{\linewidth}}}]{% 
	  \tt {\tiny [GM$|$GM]}  
	 continue 
	  } 
	   } 
	   } 
	 & & \\ 
 

    \theutterance \stepcounter{utterance}  

    & & \multicolumn{4}{p{0.6\linewidth}}{\cellcolor[rgb]{0.9,0.9,0.9}{%
	\makecell[{{p{\linewidth}}}]{% 
	  \tt {\tiny [A$\langle$GM]}  
	 ANSWER: The grid you've described is indeed symmetrical and balanced, with no additional details or hidden features that would immediately differentiate it from another grid. The symmetry is maintained through the uniform distribution of X's and ▢'s across rows and columns, and there are no specific mathematical properties or relationships between the symbols that stand out. The balance and uniformity ensure that the grid remains visually uniform and does not exhibit any distinctive features beyond its basic structure.\\ \tt \\ \tt Given the lack of distinctive patterns or visual\\ \tt QUESTION: Are there any cultural or historical significance attached to the symbols used in this grid? For example, do the X's and ▢'s have a specific meaning in a particular culture or context? If so, what is that meaning and how might it influence the perception of the grid? If not, how do you interpret the symbols' significance in the context of this guessing game? Start your answer with "ANSWER:" and do not add anything else. 
	  } 
	   } 
	   } 
	 & \\ 
 

    \theutterance \stepcounter{utterance}  

    & \multicolumn{4}{p{0.6\linewidth}}{\cellcolor[rgb]{0.8,1,0.9}{%
	\makecell[{{p{\linewidth}}}]{% 
	  \tt {\tiny [A$\rangle$GM]}  
	 ANSWER: The symbols used in the grid—X's and ▢'s—do not have any specific cultural or historical significance attached to them. They are purely functional in this guessing game, serving as placeholders for the unknown picture you're trying to describe. The X's and ▢'s are chosen arbitrarily and do not carry any particular meaning beyond their role in the guessing game.\\ \tt \\ \tt In the context of this guessing game, the symbols' significance is purely aesthetic and does not influence the perception 
	  } 
	   } 
	   } 
	 & & \\ 
 

    \theutterance \stepcounter{utterance}  

    & & & \multicolumn{2}{p{0.3\linewidth}}{\cellcolor[rgb]{0.95,0.95,0.95}{%
	\makecell[{{p{\linewidth}}}]{% 
	  \tt {\tiny [GM$|$GM]}  
	 continue 
	  } 
	   } 
	   } 
	 & & \\ 
 

    \theutterance \stepcounter{utterance}  

    & & \multicolumn{4}{p{0.6\linewidth}}{\cellcolor[rgb]{0.9,0.9,0.9}{%
	\makecell[{{p{\linewidth}}}]{% 
	  \tt {\tiny [A$\langle$GM]}  
	 Now ask a question in order to find out new aspects of my grid that may be different to your grid. Start with "QUESTION:" and do not add anything else. 
	  } 
	   } 
	   } 
	 & \\ 
 

    \theutterance \stepcounter{utterance}  

    & \multicolumn{4}{p{0.6\linewidth}}{\cellcolor[rgb]{0.8,1,0.9}{%
	\makecell[{{p{\linewidth}}}]{% 
	  \tt {\tiny [A$\rangle$GM]}  
	 QUESTION: Are there any specific features or characteristics of the grid that are not immediately obvious from the description provided? For example, could there be a pattern within the grid that is not visible on the surface level, or are there any additional elements that might be present but not mentioned in the initial description? If so, what are they and how might they contribute to the overall uniqueness of the grid? If not, how do you interpret the grid's characteristics based solely on the provided description? 
	  } 
	   } 
	   } 
	 & & \\ 
 

    \theutterance \stepcounter{utterance}  

    & & & \multicolumn{2}{p{0.3\linewidth}}{\cellcolor[rgb]{0.95,0.95,0.95}{%
	\makecell[{{p{\linewidth}}}]{% 
	  \tt {\tiny [GM$|$GM]}  
	 continue 
	  } 
	   } 
	   } 
	 & & \\ 
 

    \theutterance \stepcounter{utterance}  

    & & \multicolumn{4}{p{0.6\linewidth}}{\cellcolor[rgb]{0.9,0.9,0.9}{%
	\makecell[{{p{\linewidth}}}]{% 
	  \tt {\tiny [GM$\rangle$B]}  
	 ANSWER: The symbols used in the grid—X's and ▢'s—do not have any specific cultural or historical significance attached to them. They are purely functional in this guessing game, serving as placeholders for the unknown picture you're trying to describe. The X's and ▢'s are chosen arbitrarily and do not carry any particular meaning beyond their role in the guessing game.\\ \tt \\ \tt In the context of this guessing game, the symbols' significance is purely aesthetic and does not influence the perception\\ \tt QUESTION: Are there any specific features or characteristics of the grid that are not immediately obvious from the description provided? For example, could there be a pattern within the grid that is not visible on the surface level, or are there any additional elements that might be present but not mentioned in the initial description? If so, what are they and how might they contribute to the overall uniqueness of the grid? If not, how do you interpret the grid's characteristics based solely on the provided description? Start your answer with "ANSWER:" and do not add anything else. 
	  } 
	   } 
	   } 
	 & \\ 
 

    \theutterance \stepcounter{utterance}  

    & & & \multicolumn{4}{p{0.6\linewidth}}{\cellcolor[rgb]{1,0.85,0.72}{%
	\makecell[{{p{\linewidth}}}]{% 
	  \tt {\tiny [GM$\langle$B]}  
	 ANSWER: The grid you've described is entirely based on the initial description provided, and there are no specific features or characteristics that are not immediately obvious from the initial description. The symmetry, balance, and uniform distribution of X's and ▢'s across rows and columns are the primary features that define the grid. There are no hidden patterns, additional elements, or unique characteristics that are not immediately apparent from the description.\\ \tt \\ \tt Given the simplicity and uniformity of the grid, the only unique aspect that 
	  } 
	   } 
	   } 
	  \\ 
 

    \theutterance \stepcounter{utterance}  

    & & & \multicolumn{2}{p{0.3\linewidth}}{\cellcolor[rgb]{0.95,0.95,0.95}{%
	\makecell[{{p{\linewidth}}}]{% 
	  \tt {\tiny [GM$|$GM]}  
	 continue 
	  } 
	   } 
	   } 
	 & & \\ 
 

    \theutterance \stepcounter{utterance}  

    & & \multicolumn{4}{p{0.6\linewidth}}{\cellcolor[rgb]{0.9,0.9,0.9}{%
	\makecell[{{p{\linewidth}}}]{% 
	  \tt {\tiny [GM$\rangle$B]}  
	 Now come to a decision. What do you think: are your grid and the other grid described the same?  Write "DECISION: same grid" if you think they are the same grid or "DECISION: different grids" if you think they are different grids. Do not add anything else.\\ \tt  
	  } 
	   } 
	   } 
	 & \\ 
 

    \theutterance \stepcounter{utterance}  

    & & & \multicolumn{4}{p{0.6\linewidth}}{\cellcolor[rgb]{1,0.85,0.72}{%
	\makecell[{{p{\linewidth}}}]{% 
	  \tt {\tiny [GM$\langle$B]}  
	 DECISION: same grid 
	  } 
	   } 
	   } 
	  \\ 
 

    \theutterance \stepcounter{utterance}  

    & & & \multicolumn{2}{p{0.3\linewidth}}{\cellcolor[rgb]{0.95,0.95,0.95}{%
	\makecell[{{p{\linewidth}}}]{% 
	  \tt {\tiny [GM$|$GM]}  
	 continue 
	  } 
	   } 
	   } 
	 & & \\ 
 

    \theutterance \stepcounter{utterance}  

    & & & \multicolumn{2}{p{0.3\linewidth}}{\cellcolor[rgb]{0.95,0.95,0.95}{%
	\makecell[{{p{\linewidth}}}]{% 
	  \tt {\tiny [GM$|$GM]}  
	 loss 
	  } 
	   } 
	   } 
	 & & \\ 
 

    \theutterance \stepcounter{utterance}  

    & & \multicolumn{4}{p{0.6\linewidth}}{\cellcolor[rgb]{0.9,0.9,0.9}{%
	\makecell[{{p{\linewidth}}}]{% 
	  \tt {\tiny [A$\langle$GM]}  
	 ANSWER: The grid you've described is entirely based on the initial description provided, and there are no specific features or characteristics that are not immediately obvious from the initial description. The symmetry, balance, and uniform distribution of X's and ▢'s across rows and columns are the primary features that define the grid. There are no hidden patterns, additional elements, or unique characteristics that are not immediately apparent from the description.\\ \tt \\ \tt Given the simplicity and uniformity of the grid, the only unique aspect that\\ \tt Now come to a decision. What do you think: are your grid and the other grid described the same?  Write "DECISION: same grid" if you think they are the same grid or "DECISION: different grids" if you think they are different grids. Do not add anything else.\\ \tt  
	  } 
	   } 
	   } 
	 & \\ 
 

    \theutterance \stepcounter{utterance}  

    & \multicolumn{4}{p{0.6\linewidth}}{\cellcolor[rgb]{0.8,1,0.9}{%
	\makecell[{{p{\linewidth}}}]{% 
	  \tt {\tiny [A$\rangle$GM]}  
	 DECISION: same grid 
	  } 
	   } 
	   } 
	 & & \\ 
 

    \theutterance \stepcounter{utterance}  

    & & & \multicolumn{2}{p{0.3\linewidth}}{\cellcolor[rgb]{0.95,0.95,0.95}{%
	\makecell[{{p{\linewidth}}}]{% 
	  \tt {\tiny [GM$|$GM]}  
	 continue 
	  } 
	   } 
	   } 
	 & & \\ 
 

    \theutterance \stepcounter{utterance}  

    & & & \multicolumn{2}{p{0.3\linewidth}}{\cellcolor[rgb]{0.95,0.95,0.95}{%
	\makecell[{{p{\linewidth}}}]{% 
	  \tt {\tiny [GM$|$GM]}  
	 loss 
	  } 
	   } 
	   } 
	 & & \\ 
 

\end{supertabular}
}

\end{document}

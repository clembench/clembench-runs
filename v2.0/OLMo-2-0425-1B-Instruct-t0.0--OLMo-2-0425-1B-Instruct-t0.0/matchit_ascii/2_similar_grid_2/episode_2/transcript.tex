
\documentclass{article}
\usepackage{colortbl}
\usepackage{makecell}
\usepackage{multirow}
\usepackage{supertabular}

\begin{document}

\newcounter{utterance}

\twocolumn

{ \footnotesize  \setcounter{utterance}{1}
\setlength{\tabcolsep}{0pt}
\begin{supertabular}{c@{$\;$}|p{.15\linewidth}@{}p{.15\linewidth}p{.15\linewidth}p{.15\linewidth}p{.15\linewidth}p{.15\linewidth}}

    \# & $\;$A & \multicolumn{4}{c}{Game Master} & $\;\:$B\\
    \hline 

    \theutterance \stepcounter{utterance}  

    & & \multicolumn{4}{p{0.6\linewidth}}{\cellcolor[rgb]{0.9,0.9,0.9}{%
	\makecell[{{p{\linewidth}}}]{% 
	  \tt {\tiny [A$\langle$GM]}  
	 You are participating in a collaborative guessing game.\\ \tt The goal is to find out whether this grid and another picture only I can see, are the same.\\ \tt Please describe your grid first. Then, I will provide my description and we can ask each other questions about the grids to figure out whether they are the same. Now start your short grid description with "DESCRIPTION:" followed by the description. Do not add anything else. This is your grid:\\ \tt X ▢ ▢ ▢ X\\ \tt X X ▢ X X\\ \tt X ▢ X ▢ X\\ \tt X ▢ ▢ ▢ X\\ \tt X ▢ ▢ ▢ X 
	  } 
	   } 
	   } 
	 & \\ 
 

    \theutterance \stepcounter{utterance}  

    & \multicolumn{4}{p{0.6\linewidth}}{\cellcolor[rgb]{0.8,1,0.9}{%
	\makecell[{{p{\linewidth}}}]{% 
	  \tt {\tiny [A$\rangle$GM]}  
	 DESCRIPTION: The grid provided consists of a single square filled with X's, with additional X's placed in specific patterns around it. The central square is filled with X's, and there are additional X's placed in a triangular pattern around the edges of the grid. The pattern of X's is as follows:\\ \tt \\ \tt - The central square: X X X\\ \tt - The triangular pattern around the edges: ▢ ▢ ▢ X\\ \tt \\ \tt Each X in the pattern is aligned with its neighbors, forming a 
	  } 
	   } 
	   } 
	 & & \\ 
 

    \theutterance \stepcounter{utterance}  

    & & & \multicolumn{2}{p{0.3\linewidth}}{\cellcolor[rgb]{0.95,0.95,0.95}{%
	\makecell[{{p{\linewidth}}}]{% 
	  \tt {\tiny [GM$|$GM]}  
	 continue 
	  } 
	   } 
	   } 
	 & & \\ 
 

    \theutterance \stepcounter{utterance}  

    & & \multicolumn{4}{p{0.6\linewidth}}{\cellcolor[rgb]{0.9,0.9,0.9}{%
	\makecell[{{p{\linewidth}}}]{% 
	  \tt {\tiny [GM$\rangle$B]}  
	 You are participating in a collaborative guessing game.\\ \tt The goal is to find out whether this grid and another picture only I can see, are the same.\\ \tt Please describe your grid first. Then, I will provide my description and we can ask each other questions about the grids to figure out whether they are the same. Now start your short grid description with "DESCRIPTION:" followed by the description. Do not add anything else. This is your grid:\\ \tt X ▢ ▢ ▢ X\\ \tt X ▢ ▢ X X\\ \tt X X X ▢ X\\ \tt X ▢ ▢ ▢ X\\ \tt X ▢ ▢ ▢ X 
	  } 
	   } 
	   } 
	 & \\ 
 

    \theutterance \stepcounter{utterance}  

    & & & \multicolumn{4}{p{0.6\linewidth}}{\cellcolor[rgb]{1,0.85,0.72}{%
	\makecell[{{p{\linewidth}}}]{% 
	  \tt {\tiny [GM$\langle$B]}  
	 DESCRIPTION: The grid provided consists of a series of X's and ▢'s arranged in a specific pattern. Each row and column alternates between X's and ▢'s, creating a symmetrical and visually interesting pattern. The grid spans 3x3, with the first row starting as follows: X ▢ ▢ X, X ▢ ▢ X, X X X ▢ X. The second row starts with X and continues with alternating X's and ▢'s, mir 
	  } 
	   } 
	   } 
	  \\ 
 

    \theutterance \stepcounter{utterance}  

    & & & \multicolumn{2}{p{0.3\linewidth}}{\cellcolor[rgb]{0.95,0.95,0.95}{%
	\makecell[{{p{\linewidth}}}]{% 
	  \tt {\tiny [GM$|$GM]}  
	 continue 
	  } 
	   } 
	   } 
	 & & \\ 
 

    \theutterance \stepcounter{utterance}  

    & & \multicolumn{4}{p{0.6\linewidth}}{\cellcolor[rgb]{0.9,0.9,0.9}{%
	\makecell[{{p{\linewidth}}}]{% 
	  \tt {\tiny [GM$\rangle$B]}  
	 This is my DESCRIPTION: The grid provided consists of a single square filled with X's, with additional X's placed in specific patterns around it. The central square is filled with X's, and there are additional X's placed in a triangular pattern around the edges of the grid. The pattern of X's is as follows:\\ \tt \\ \tt - The central square: X X X\\ \tt - The triangular pattern around the edges: ▢ ▢ ▢ X\\ \tt \\ \tt Each X in the pattern is aligned with its neighbors, forming a\\ \tt Now ask a question in order to find out new aspects of my grid that may be different to your grid. Start with "QUESTION:" and do not add anything else. 
	  } 
	   } 
	   } 
	 & \\ 
 

    \theutterance \stepcounter{utterance}  

    & & & \multicolumn{4}{p{0.6\linewidth}}{\cellcolor[rgb]{1,0.85,0.72}{%
	\makecell[{{p{\linewidth}}}]{% 
	  \tt {\tiny [GM$\langle$B]}  
	 QUESTION: Is the central square in your grid filled with exactly three X's, as indicated by the pattern X X X, while the triangular pattern around the edges contains X's in a triangular arrangement as described?\\ \tt \\ \tt If the answer is yes, then the grids are likely the same. If the answer is no, then they might be different. Please confirm if the central square indeed has exactly three X's and if the triangular pattern around the edges matches the provided description. 
	  } 
	   } 
	   } 
	  \\ 
 

    \theutterance \stepcounter{utterance}  

    & & & \multicolumn{2}{p{0.3\linewidth}}{\cellcolor[rgb]{0.95,0.95,0.95}{%
	\makecell[{{p{\linewidth}}}]{% 
	  \tt {\tiny [GM$|$GM]}  
	 continue 
	  } 
	   } 
	   } 
	 & & \\ 
 

    \theutterance \stepcounter{utterance}  

    & & \multicolumn{4}{p{0.6\linewidth}}{\cellcolor[rgb]{0.9,0.9,0.9}{%
	\makecell[{{p{\linewidth}}}]{% 
	  \tt {\tiny [A$\langle$GM]}  
	 This is my DESCRIPTION: The grid provided consists of a series of X's and ▢'s arranged in a specific pattern. Each row and column alternates between X's and ▢'s, creating a symmetrical and visually interesting pattern. The grid spans 3x3, with the first row starting as follows: X ▢ ▢ X, X ▢ ▢ X, X X X ▢ X. The second row starts with X and continues with alternating X's and ▢'s, mir\\ \tt QUESTION: Is the central square in your grid filled with exactly three X's, as indicated by the pattern X X X, while the triangular pattern around the edges contains X's in a triangular arrangement as described?\\ \tt \\ \tt If the answer is yes, then the grids are likely the same. If the answer is no, then they might be different. Please confirm if the central square indeed has exactly three X's and if the triangular pattern around the edges matches the provided description. Start your answer with "ANSWER:" and do not add anything else. 
	  } 
	   } 
	   } 
	 & \\ 
 

    \theutterance \stepcounter{utterance}  

    & \multicolumn{4}{p{0.6\linewidth}}{\cellcolor[rgb]{0.8,1,0.9}{%
	\makecell[{{p{\linewidth}}}]{% 
	  \tt {\tiny [A$\rangle$GM]}  
	 ANSWER: The central square in your grid indeed has exactly three X's, as indicated by the pattern X X X. The triangular pattern around the edges matches the description provided: ▢ ▢ ▢ X.\\ \tt \\ \tt Therefore, the grids are likely the same. 
	  } 
	   } 
	   } 
	 & & \\ 
 

    \theutterance \stepcounter{utterance}  

    & & & \multicolumn{2}{p{0.3\linewidth}}{\cellcolor[rgb]{0.95,0.95,0.95}{%
	\makecell[{{p{\linewidth}}}]{% 
	  \tt {\tiny [GM$|$GM]}  
	 continue 
	  } 
	   } 
	   } 
	 & & \\ 
 

    \theutterance \stepcounter{utterance}  

    & & \multicolumn{4}{p{0.6\linewidth}}{\cellcolor[rgb]{0.9,0.9,0.9}{%
	\makecell[{{p{\linewidth}}}]{% 
	  \tt {\tiny [A$\langle$GM]}  
	 Now ask a question in order to find out new aspects of my grid that may be different to your grid. Start with "QUESTION:" and do not add anything else. 
	  } 
	   } 
	   } 
	 & \\ 
 

    \theutterance \stepcounter{utterance}  

    & \multicolumn{4}{p{0.6\linewidth}}{\cellcolor[rgb]{0.8,1,0.9}{%
	\makecell[{{p{\linewidth}}}]{% 
	  \tt {\tiny [A$\rangle$GM]}  
	 QUESTION: In the second row of your grid, there are X's placed in a triangular pattern around the edges. Specifically, the second row starts with an X and then alternates between ▢ and X. How many ▢'s are there in the second row after the initial X? 
	  } 
	   } 
	   } 
	 & & \\ 
 

    \theutterance \stepcounter{utterance}  

    & & & \multicolumn{2}{p{0.3\linewidth}}{\cellcolor[rgb]{0.95,0.95,0.95}{%
	\makecell[{{p{\linewidth}}}]{% 
	  \tt {\tiny [GM$|$GM]}  
	 continue 
	  } 
	   } 
	   } 
	 & & \\ 
 

    \theutterance \stepcounter{utterance}  

    & & \multicolumn{4}{p{0.6\linewidth}}{\cellcolor[rgb]{0.9,0.9,0.9}{%
	\makecell[{{p{\linewidth}}}]{% 
	  \tt {\tiny [GM$\rangle$B]}  
	 ANSWER: The central square in your grid indeed has exactly three X's, as indicated by the pattern X X X. The triangular pattern around the edges matches the description provided: ▢ ▢ ▢ X.\\ \tt \\ \tt Therefore, the grids are likely the same.\\ \tt QUESTION: In the second row of your grid, there are X's placed in a triangular pattern around the edges. Specifically, the second row starts with an X and then alternates between ▢ and X. How many ▢'s are there in the second row after the initial X? Start your answer with "ANSWER:" and do not add anything else. 
	  } 
	   } 
	   } 
	 & \\ 
 

    \theutterance \stepcounter{utterance}  

    & & & \multicolumn{4}{p{0.6\linewidth}}{\cellcolor[rgb]{1,0.85,0.72}{%
	\makecell[{{p{\linewidth}}}]{% 
	  \tt {\tiny [GM$\langle$B]}  
	 ANSWER: In the second row of your grid, after the initial X, there are 2 ▢'s in the triangular pattern around the edges.\\ \tt \\ \tt The pattern described is:\\ \tt - Start with an X\\ \tt - Followed by ▢ and then X alternately\\ \tt \\ \tt So, after the first X, there is 1 ▢. Then, the second row continues with:\\ \tt - ▢ (second position)\\ \tt - X (third position)\\ \tt - ▢ (fourth position)\\ \tt \\ \tt Thus, 
	  } 
	   } 
	   } 
	  \\ 
 

    \theutterance \stepcounter{utterance}  

    & & & \multicolumn{2}{p{0.3\linewidth}}{\cellcolor[rgb]{0.95,0.95,0.95}{%
	\makecell[{{p{\linewidth}}}]{% 
	  \tt {\tiny [GM$|$GM]}  
	 continue 
	  } 
	   } 
	   } 
	 & & \\ 
 

    \theutterance \stepcounter{utterance}  

    & & \multicolumn{4}{p{0.6\linewidth}}{\cellcolor[rgb]{0.9,0.9,0.9}{%
	\makecell[{{p{\linewidth}}}]{% 
	  \tt {\tiny [GM$\rangle$B]}  
	 Now ask a question in order to find out new aspects of my grid that may be different to your grid. Start with "QUESTION:" and do not add anything else. 
	  } 
	   } 
	   } 
	 & \\ 
 

    \theutterance \stepcounter{utterance}  

    & & & \multicolumn{4}{p{0.6\linewidth}}{\cellcolor[rgb]{1,0.85,0.72}{%
	\makecell[{{p{\linewidth}}}]{% 
	  \tt {\tiny [GM$\langle$B]}  
	 QUESTION: In your grid, the pattern of X's around the edges includes ▢'s. Specifically, the second row starts with an X and then alternates between ▢ and X. How many ▢'s are there in the second row after the initial X?\\ \tt \\ \tt Please confirm if the second row indeed contains 2 ▢'s after the initial X. 
	  } 
	   } 
	   } 
	  \\ 
 

    \theutterance \stepcounter{utterance}  

    & & & \multicolumn{2}{p{0.3\linewidth}}{\cellcolor[rgb]{0.95,0.95,0.95}{%
	\makecell[{{p{\linewidth}}}]{% 
	  \tt {\tiny [GM$|$GM]}  
	 continue 
	  } 
	   } 
	   } 
	 & & \\ 
 

    \theutterance \stepcounter{utterance}  

    & & \multicolumn{4}{p{0.6\linewidth}}{\cellcolor[rgb]{0.9,0.9,0.9}{%
	\makecell[{{p{\linewidth}}}]{% 
	  \tt {\tiny [A$\langle$GM]}  
	 ANSWER: In the second row of your grid, after the initial X, there are 2 ▢'s in the triangular pattern around the edges.\\ \tt \\ \tt The pattern described is:\\ \tt - Start with an X\\ \tt - Followed by ▢ and then X alternately\\ \tt \\ \tt So, after the first X, there is 1 ▢. Then, the second row continues with:\\ \tt - ▢ (second position)\\ \tt - X (third position)\\ \tt - ▢ (fourth position)\\ \tt \\ \tt Thus,\\ \tt QUESTION: In your grid, the pattern of X's around the edges includes ▢'s. Specifically, the second row starts with an X and then alternates between ▢ and X. How many ▢'s are there in the second row after the initial X?\\ \tt \\ \tt Please confirm if the second row indeed contains 2 ▢'s after the initial X. Start your answer with "ANSWER:" and do not add anything else. 
	  } 
	   } 
	   } 
	 & \\ 
 

    \theutterance \stepcounter{utterance}  

    & \multicolumn{4}{p{0.6\linewidth}}{\cellcolor[rgb]{0.8,1,0.9}{%
	\makecell[{{p{\linewidth}}}]{% 
	  \tt {\tiny [A$\rangle$GM]}  
	 ANSWER: In the second row of your grid, after the initial X, there are indeed 2 ▢'s in the triangular pattern around the edges.\\ \tt \\ \tt The pattern described is:\\ \tt - Start with an X\\ \tt - Followed by ▢ and then X alternately\\ \tt \\ \tt So, after the first X, there is 1 ▢. Then, the second row continues with:\\ \tt - ▢ (second position)\\ \tt - X (third position)\\ \tt - ▢ (fourth position)\\ \tt \\ \tt Thus 
	  } 
	   } 
	   } 
	 & & \\ 
 

    \theutterance \stepcounter{utterance}  

    & & & \multicolumn{2}{p{0.3\linewidth}}{\cellcolor[rgb]{0.95,0.95,0.95}{%
	\makecell[{{p{\linewidth}}}]{% 
	  \tt {\tiny [GM$|$GM]}  
	 continue 
	  } 
	   } 
	   } 
	 & & \\ 
 

    \theutterance \stepcounter{utterance}  

    & & \multicolumn{4}{p{0.6\linewidth}}{\cellcolor[rgb]{0.9,0.9,0.9}{%
	\makecell[{{p{\linewidth}}}]{% 
	  \tt {\tiny [A$\langle$GM]}  
	 Now ask a question in order to find out new aspects of my grid that may be different to your grid. Start with "QUESTION:" and do not add anything else. 
	  } 
	   } 
	   } 
	 & \\ 
 

    \theutterance \stepcounter{utterance}  

    & \multicolumn{4}{p{0.6\linewidth}}{\cellcolor[rgb]{0.8,1,0.9}{%
	\makecell[{{p{\linewidth}}}]{% 
	  \tt {\tiny [A$\rangle$GM]}  
	 QUESTION: In your grid, the central square is filled with exactly three X's. However, in my grid description, it was mentioned that the central square is filled with X's in a symmetrical pattern. Specifically, my description mentioned that the central square is X X X. How does this discrepancy affect the overall visual identity and uniqueness of your grid compared to mine? 
	  } 
	   } 
	   } 
	 & & \\ 
 

    \theutterance \stepcounter{utterance}  

    & & & \multicolumn{2}{p{0.3\linewidth}}{\cellcolor[rgb]{0.95,0.95,0.95}{%
	\makecell[{{p{\linewidth}}}]{% 
	  \tt {\tiny [GM$|$GM]}  
	 continue 
	  } 
	   } 
	   } 
	 & & \\ 
 

    \theutterance \stepcounter{utterance}  

    & & \multicolumn{4}{p{0.6\linewidth}}{\cellcolor[rgb]{0.9,0.9,0.9}{%
	\makecell[{{p{\linewidth}}}]{% 
	  \tt {\tiny [GM$\rangle$B]}  
	 ANSWER: In the second row of your grid, after the initial X, there are indeed 2 ▢'s in the triangular pattern around the edges.\\ \tt \\ \tt The pattern described is:\\ \tt - Start with an X\\ \tt - Followed by ▢ and then X alternately\\ \tt \\ \tt So, after the first X, there is 1 ▢. Then, the second row continues with:\\ \tt - ▢ (second position)\\ \tt - X (third position)\\ \tt - ▢ (fourth position)\\ \tt \\ \tt Thus\\ \tt QUESTION: In your grid, the central square is filled with exactly three X's. However, in my grid description, it was mentioned that the central square is filled with X's in a symmetrical pattern. Specifically, my description mentioned that the central square is X X X. How does this discrepancy affect the overall visual identity and uniqueness of your grid compared to mine? Start your answer with "ANSWER:" and do not add anything else. 
	  } 
	   } 
	   } 
	 & \\ 
 

    \theutterance \stepcounter{utterance}  

    & & & \multicolumn{4}{p{0.6\linewidth}}{\cellcolor[rgb]{1,0.85,0.72}{%
	\makecell[{{p{\linewidth}}}]{% 
	  \tt {\tiny [GM$\langle$B]}  
	 ANSWER: The discrepancy between your description of the central square as X X X and my grid's description of having exactly three X's in the central square does not significantly alter the overall visual identity and uniqueness of the grids. Both descriptions are consistent with the pattern and arrangement of X's and ▢'s in the grids. The central square in your grid has three X's, while in my grid description, it was specifically mentioned to be filled with X's in a symmetrical pattern. The 
	  } 
	   } 
	   } 
	  \\ 
 

    \theutterance \stepcounter{utterance}  

    & & & \multicolumn{2}{p{0.3\linewidth}}{\cellcolor[rgb]{0.95,0.95,0.95}{%
	\makecell[{{p{\linewidth}}}]{% 
	  \tt {\tiny [GM$|$GM]}  
	 continue 
	  } 
	   } 
	   } 
	 & & \\ 
 

    \theutterance \stepcounter{utterance}  

    & & \multicolumn{4}{p{0.6\linewidth}}{\cellcolor[rgb]{0.9,0.9,0.9}{%
	\makecell[{{p{\linewidth}}}]{% 
	  \tt {\tiny [GM$\rangle$B]}  
	 Now ask a question in order to find out new aspects of my grid that may be different to your grid. Start with "QUESTION:" and do not add anything else. 
	  } 
	   } 
	   } 
	 & \\ 
 

    \theutterance \stepcounter{utterance}  

    & & & \multicolumn{4}{p{0.6\linewidth}}{\cellcolor[rgb]{1,0.85,0.72}{%
	\makecell[{{p{\linewidth}}}]{% 
	  \tt {\tiny [GM$\langle$B]}  
	 QUESTION: In your grid, the pattern of X's around the edges includes ▢'s. Specifically, the second row starts with an X and then alternates between ▢ and X. How many ▢'s are there in the second row after the initial X?\\ \tt \\ \tt Please confirm if the second row indeed contains 2 ▢'s after the initial X. 
	  } 
	   } 
	   } 
	  \\ 
 

    \theutterance \stepcounter{utterance}  

    & & & \multicolumn{2}{p{0.3\linewidth}}{\cellcolor[rgb]{0.95,0.95,0.95}{%
	\makecell[{{p{\linewidth}}}]{% 
	  \tt {\tiny [GM$|$GM]}  
	 continue 
	  } 
	   } 
	   } 
	 & & \\ 
 

    \theutterance \stepcounter{utterance}  

    & & \multicolumn{4}{p{0.6\linewidth}}{\cellcolor[rgb]{0.9,0.9,0.9}{%
	\makecell[{{p{\linewidth}}}]{% 
	  \tt {\tiny [A$\langle$GM]}  
	 ANSWER: The discrepancy between your description of the central square as X X X and my grid's description of having exactly three X's in the central square does not significantly alter the overall visual identity and uniqueness of the grids. Both descriptions are consistent with the pattern and arrangement of X's and ▢'s in the grids. The central square in your grid has three X's, while in my grid description, it was specifically mentioned to be filled with X's in a symmetrical pattern. The\\ \tt QUESTION: In your grid, the pattern of X's around the edges includes ▢'s. Specifically, the second row starts with an X and then alternates between ▢ and X. How many ▢'s are there in the second row after the initial X?\\ \tt \\ \tt Please confirm if the second row indeed contains 2 ▢'s after the initial X. Start your answer with "ANSWER:" and do not add anything else. 
	  } 
	   } 
	   } 
	 & \\ 
 

    \theutterance \stepcounter{utterance}  

    & \multicolumn{4}{p{0.6\linewidth}}{\cellcolor[rgb]{0.8,1,0.9}{%
	\makecell[{{p{\linewidth}}}]{% 
	  \tt {\tiny [A$\rangle$GM]}  
	 ANSWER: In the second row of your grid, after the initial X, there are indeed 2 ▢'s in the triangular pattern around the edges.\\ \tt \\ \tt The pattern described is:\\ \tt - Start with an X\\ \tt - Followed by ▢ and then X alternately\\ \tt \\ \tt So, after the first X, there is 1 ▢. Then, the second row continues with:\\ \tt - ▢ (second position)\\ \tt - X (third position)\\ \tt - ▢ (fourth position)\\ \tt \\ \tt Thus 
	  } 
	   } 
	   } 
	 & & \\ 
 

    \theutterance \stepcounter{utterance}  

    & & & \multicolumn{2}{p{0.3\linewidth}}{\cellcolor[rgb]{0.95,0.95,0.95}{%
	\makecell[{{p{\linewidth}}}]{% 
	  \tt {\tiny [GM$|$GM]}  
	 continue 
	  } 
	   } 
	   } 
	 & & \\ 
 

    \theutterance \stepcounter{utterance}  

    & & \multicolumn{4}{p{0.6\linewidth}}{\cellcolor[rgb]{0.9,0.9,0.9}{%
	\makecell[{{p{\linewidth}}}]{% 
	  \tt {\tiny [A$\langle$GM]}  
	 Now ask a question in order to find out new aspects of my grid that may be different to your grid. Start with "QUESTION:" and do not add anything else. 
	  } 
	   } 
	   } 
	 & \\ 
 

    \theutterance \stepcounter{utterance}  

    & \multicolumn{4}{p{0.6\linewidth}}{\cellcolor[rgb]{0.8,1,0.9}{%
	\makecell[{{p{\linewidth}}}]{% 
	  \tt {\tiny [A$\rangle$GM]}  
	 QUESTION: In your grid, the pattern of X's around the edges includes ▢'s. Specifically, the second row starts with an X and then alternates between ▢ and X. How many ▢'s are there in the second row after the initial X?\\ \tt \\ \tt Please confirm if the second row indeed contains 2 ▢'s after the initial X. Start your answer with "ANSWER:" and do not add anything else.\\ \tt < 
	  } 
	   } 
	   } 
	 & & \\ 
 

    \theutterance \stepcounter{utterance}  

    & & & \multicolumn{2}{p{0.3\linewidth}}{\cellcolor[rgb]{0.95,0.95,0.95}{%
	\makecell[{{p{\linewidth}}}]{% 
	  \tt {\tiny [GM$|$GM]}  
	 continue 
	  } 
	   } 
	   } 
	 & & \\ 
 

    \theutterance \stepcounter{utterance}  

    & & \multicolumn{4}{p{0.6\linewidth}}{\cellcolor[rgb]{0.9,0.9,0.9}{%
	\makecell[{{p{\linewidth}}}]{% 
	  \tt {\tiny [GM$\rangle$B]}  
	 ANSWER: In the second row of your grid, after the initial X, there are indeed 2 ▢'s in the triangular pattern around the edges.\\ \tt \\ \tt The pattern described is:\\ \tt - Start with an X\\ \tt - Followed by ▢ and then X alternately\\ \tt \\ \tt So, after the first X, there is 1 ▢. Then, the second row continues with:\\ \tt - ▢ (second position)\\ \tt - X (third position)\\ \tt - ▢ (fourth position)\\ \tt \\ \tt Thus\\ \tt QUESTION: In your grid, the pattern of X's around the edges includes ▢'s. Specifically, the second row starts with an X and then alternates between ▢ and X. How many ▢'s are there in the second row after the initial X?\\ \tt \\ \tt Please confirm if the second row indeed contains 2 ▢'s after the initial X. Start your answer with "ANSWER:" and do not add anything else.\\ \tt < Start your answer with "ANSWER:" and do not add anything else. 
	  } 
	   } 
	   } 
	 & \\ 
 

    \theutterance \stepcounter{utterance}  

    & & & \multicolumn{4}{p{0.6\linewidth}}{\cellcolor[rgb]{1,0.85,0.72}{%
	\makecell[{{p{\linewidth}}}]{% 
	  \tt {\tiny [GM$\langle$B]}  
	 ANSWER: In your grid, after the initial X, there are indeed 2 ▢'s in the second row in the triangular pattern around the edges.\\ \tt \\ \tt The pattern described is:\\ \tt - Start with an X\\ \tt - Followed by ▢ and then X alternately\\ \tt \\ \tt So, after the first X, there is 1 ▢. Then, the second row continues with:\\ \tt - ▢ (second position)\\ \tt - X (third position)\\ \tt - ▢ (fourth position)\\ \tt \\ \tt Thus 
	  } 
	   } 
	   } 
	  \\ 
 

    \theutterance \stepcounter{utterance}  

    & & & \multicolumn{2}{p{0.3\linewidth}}{\cellcolor[rgb]{0.95,0.95,0.95}{%
	\makecell[{{p{\linewidth}}}]{% 
	  \tt {\tiny [GM$|$GM]}  
	 continue 
	  } 
	   } 
	   } 
	 & & \\ 
 

    \theutterance \stepcounter{utterance}  

    & & \multicolumn{4}{p{0.6\linewidth}}{\cellcolor[rgb]{0.9,0.9,0.9}{%
	\makecell[{{p{\linewidth}}}]{% 
	  \tt {\tiny [GM$\rangle$B]}  
	 Now come to a decision. What do you think: are your grid and the other grid described the same?  Write "DECISION: same grid" if you think they are the same grid or "DECISION: different grids" if you think they are different grids. Do not add anything else.\\ \tt  
	  } 
	   } 
	   } 
	 & \\ 
 

    \theutterance \stepcounter{utterance}  

    & & & \multicolumn{4}{p{0.6\linewidth}}{\cellcolor[rgb]{1,0.85,0.72}{%
	\makecell[{{p{\linewidth}}}]{% 
	  \tt {\tiny [GM$\langle$B]}  
	 DECISION: same grid 
	  } 
	   } 
	   } 
	  \\ 
 

    \theutterance \stepcounter{utterance}  

    & & & \multicolumn{2}{p{0.3\linewidth}}{\cellcolor[rgb]{0.95,0.95,0.95}{%
	\makecell[{{p{\linewidth}}}]{% 
	  \tt {\tiny [GM$|$GM]}  
	 continue 
	  } 
	   } 
	   } 
	 & & \\ 
 

    \theutterance \stepcounter{utterance}  

    & & & \multicolumn{2}{p{0.3\linewidth}}{\cellcolor[rgb]{0.95,0.95,0.95}{%
	\makecell[{{p{\linewidth}}}]{% 
	  \tt {\tiny [GM$|$GM]}  
	 loss 
	  } 
	   } 
	   } 
	 & & \\ 
 

    \theutterance \stepcounter{utterance}  

    & & \multicolumn{4}{p{0.6\linewidth}}{\cellcolor[rgb]{0.9,0.9,0.9}{%
	\makecell[{{p{\linewidth}}}]{% 
	  \tt {\tiny [A$\langle$GM]}  
	 ANSWER: In your grid, after the initial X, there are indeed 2 ▢'s in the second row in the triangular pattern around the edges.\\ \tt \\ \tt The pattern described is:\\ \tt - Start with an X\\ \tt - Followed by ▢ and then X alternately\\ \tt \\ \tt So, after the first X, there is 1 ▢. Then, the second row continues with:\\ \tt - ▢ (second position)\\ \tt - X (third position)\\ \tt - ▢ (fourth position)\\ \tt \\ \tt Thus\\ \tt Now come to a decision. What do you think: are your grid and the other grid described the same?  Write "DECISION: same grid" if you think they are the same grid or "DECISION: different grids" if you think they are different grids. Do not add anything else.\\ \tt  
	  } 
	   } 
	   } 
	 & \\ 
 

    \theutterance \stepcounter{utterance}  

    & \multicolumn{4}{p{0.6\linewidth}}{\cellcolor[rgb]{0.8,1,0.9}{%
	\makecell[{{p{\linewidth}}}]{% 
	  \tt {\tiny [A$\rangle$GM]}  
	 DECISION: same grid\\ \tt \\ \tt The grids are indeed the same, as both grids have the same pattern of X's and ▢'s around the edges, with the central square containing exactly three X's and the second row starting with an X and alternating between ▢ and X. The description of the triangular pattern in the second row matches the pattern you provided, confirming that the grids are identical. 
	  } 
	   } 
	   } 
	 & & \\ 
 

    \theutterance \stepcounter{utterance}  

    & & & \multicolumn{2}{p{0.3\linewidth}}{\cellcolor[rgb]{0.95,0.95,0.95}{%
	\makecell[{{p{\linewidth}}}]{% 
	  \tt {\tiny [GM$|$GM]}  
	 continue 
	  } 
	   } 
	   } 
	 & & \\ 
 

    \theutterance \stepcounter{utterance}  

    & & & \multicolumn{2}{p{0.3\linewidth}}{\cellcolor[rgb]{0.95,0.95,0.95}{%
	\makecell[{{p{\linewidth}}}]{% 
	  \tt {\tiny [GM$|$GM]}  
	 abort, wrong message content. 
	  } 
	   } 
	   } 
	 & & \\ 
 

    \theutterance \stepcounter{utterance}  

    & & & \multicolumn{2}{p{0.3\linewidth}}{\cellcolor[rgb]{0.95,0.95,0.95}{%
	\makecell[{{p{\linewidth}}}]{% 
	  \tt {\tiny [GM$|$GM]}  
	 Aborted 
	  } 
	   } 
	   } 
	 & & \\ 
 

    \theutterance \stepcounter{utterance}  

    & & & \multicolumn{2}{p{0.3\linewidth}}{\cellcolor[rgb]{0.95,0.95,0.95}{%
	\makecell[{{p{\linewidth}}}]{% 
	  \tt {\tiny [GM$|$GM]}  
	 Aborted 
	  } 
	   } 
	   } 
	 & & \\ 
 

\end{supertabular}
}

\end{document}


\documentclass{article}
\usepackage{colortbl}
\usepackage{makecell}
\usepackage{multirow}
\usepackage{supertabular}

\begin{document}

\newcounter{utterance}

\twocolumn

{ \footnotesize  \setcounter{utterance}{1}
\setlength{\tabcolsep}{0pt}
\begin{supertabular}{c@{$\;$}|p{.15\linewidth}@{}p{.15\linewidth}p{.15\linewidth}p{.15\linewidth}p{.15\linewidth}p{.15\linewidth}}

    \# & $\;$A & \multicolumn{4}{c}{Game Master} & $\;\:$B\\
    \hline 

    \theutterance \stepcounter{utterance}  

    & & & \multicolumn{2}{p{0.3\linewidth}}{\cellcolor[rgb]{0.95,0.95,0.95}{%
	\makecell[{{p{\linewidth}}}]{% 
	  \tt {\tiny [GM$|$GM]}  
	 Turn 0 
	  } 
	   } 
	   } 
	 & & \\ 
 

    \theutterance \stepcounter{utterance}  

    & & \multicolumn{4}{p{0.6\linewidth}}{\cellcolor[rgb]{0.9,0.9,0.9}{%
	\makecell[{{p{\linewidth}}}]{% 
	  \tt {\tiny [A$\langle$GM]}  
	 You are playing a text adventure game. I will describe what you can perceive in the game. You write the single action you want to take in the game starting with >. Write your plan for your next actions on the line below your action, starting with "Next actions:".\\ \tt For example:\\ \tt > open cupboard\\ \tt Next actions: take orange, eat orange\\ \tt \\ \tt Your goal for this game is: Put the book on the table, the plate on the table and the mop on the table.\\ \tt \\ \tt Once you have achieved your goal, write "> done" to end the game.\\ \tt \\ \tt You are in a living room now. There are a side table, a shelf, a book, a chair and a couch. The book is on the side table. There are passages to a bedroom, a hallway and a kitchen here. 
	  } 
	   } 
	   } 
	 & \\ 
 

    \theutterance \stepcounter{utterance}  

    & \multicolumn{4}{p{0.6\linewidth}}{\cellcolor[rgb]{0.8,1,0.9}{%
	\makecell[{{p{\linewidth}}}]{% 
	  \tt {\tiny [A$\rangle$GM]}  
	 > Move towards the side table\\ \tt \\ \tt Next actions: Take the book, walk towards the table, put the book on the table, then move to the kitchen to get the plate and mop, and finally, place them on the table as well.\\ \tt \\ \tt (Note: I assumed that the table referred to in the goal is the one in the living room, as that's the initial location provided.) 
	  } 
	   } 
	   } 
	 & & \\ 
 

    \theutterance \stepcounter{utterance}  

    & & & \multicolumn{2}{p{0.3\linewidth}}{\cellcolor[rgb]{0.95,0.95,0.95}{%
	\makecell[{{p{\linewidth}}}]{% 
	  \tt {\tiny [GM$|$GM]}  
	 ['Take the book', 'walk towards the table', 'put the book on the table', 'then move to the kitchen to get the plate and mop', 'and finally', 'place them on the table as well.\n\n(Note: I assumed that the table referred to in the goal is the one in the living room', "as that's the initial location provided.)"] 
	  } 
	   } 
	   } 
	 & & \\ 
 

    \theutterance \stepcounter{utterance}  

    & & & \multicolumn{2}{p{0.3\linewidth}}{\cellcolor[rgb]{0.95,0.95,0.95}{%
	\makecell[{{p{\linewidth}}}]{% 
	  \tt {\tiny [GM$|$GM]}  
	 {'phase': 'parsing', 'fail_type': 'undefined_action_verb', 'arg': 'move', 'exploration_info': {'action_epistemic': False, 'action_pragmatic': False, 'effective_epistemic_gain_amount': 0, 'known_entities': [['at', 'chair1', 'livingroom1'], ['at', 'couch1', 'livingroom1'], ['at', 'player1', 'livingroom1'], ['at', 'book1', 'livingroom1'], ['at', 'shelf1', 'livingroom1'], ['at', 'sidetable1', 'livingroom1']], 'known_entities_ratio': 0.20689655172413793, 'visited_rooms': ['livingroom1'], 'visited_rooms_ratio': 0.16666666666666666, 'known_goal_entities': [['at', 'book1', 'livingroom1']], 'known_goal_entities_ratio': 0.25}} 
	  } 
	   } 
	   } 
	 & & \\ 
 

    \theutterance \stepcounter{utterance}  

    & & & \multicolumn{2}{p{0.3\linewidth}}{\cellcolor[rgb]{0.95,0.95,0.95}{%
	\makecell[{{p{\linewidth}}}]{% 
	  \tt {\tiny [GM$|$GM]}  
	 {'goal_states_achieved': [], 'turn_goal_score': 0} 
	  } 
	   } 
	   } 
	 & & \\ 
 

    \theutterance \stepcounter{utterance}  

    & & & \multicolumn{2}{p{0.3\linewidth}}{\cellcolor[rgb]{0.95,0.95,0.95}{%
	\makecell[{{p{\linewidth}}}]{% 
	  \tt {\tiny [GM$|$GM]}  
	 ['Take the book', 'walk towards the table', 'put the book on the table', 'then move to the kitchen to get the plate and mop', 'and finally', 'place them on the table as well.\n\n(Note: I assumed that the table referred to in the goal is the one in the living room', "as that's the initial location provided.)"] 
	  } 
	   } 
	   } 
	 & & \\ 
 

    \theutterance \stepcounter{utterance}  

    & & & \multicolumn{2}{p{0.3\linewidth}}{\cellcolor[rgb]{0.95,0.95,0.95}{%
	\makecell[{{p{\linewidth}}}]{% 
	  \tt {\tiny [GM$|$GM]}  
	 7 
	  } 
	   } 
	   } 
	 & & \\ 
 

    \theutterance \stepcounter{utterance}  

    & & & \multicolumn{2}{p{0.3\linewidth}}{\cellcolor[rgb]{0.95,0.95,0.95}{%
	\makecell[{{p{\linewidth}}}]{% 
	  \tt {\tiny [GM$|$GM]}  
	 [[[], 'You take the book. In your inventory you have a book.', {'action_type': 'take', 'exploration_info': {'action_epistemic': False, 'action_pragmatic': True, 'effective_epistemic_gain_amount': 0, 'known_entities': [['at', 'chair1', 'livingroom1'], ['at', 'couch1', 'livingroom1'], ['at', 'player1', 'livingroom1'], ['at', 'book1', 'livingroom1'], ['at', 'shelf1', 'livingroom1'], ['at', 'sidetable1', 'livingroom1']], 'known_entities_ratio': 0.20689655172413793, 'visited_rooms': ['livingroom1'], 'visited_rooms_ratio': 0.16666666666666666, 'known_goal_entities': [['at', 'book1', 'livingroom1']], 'known_goal_entities_ratio': 0.25}}], [[], "I don't know how to interpret this 'walk' action.", {'phase': 'parsing', 'fail_type': 'undefined_action_verb', 'arg': 'walk', 'exploration_info': {'action_epistemic': False, 'action_pragmatic': False, 'effective_epistemic_gain_amount': 0, 'known_entities': [['at', 'chair1', 'livingroom1'], ['at', 'couch1', 'livingroom1'], ['at', 'player1', 'livingroom1'], ['at', 'book1', 'livingroom1'], ['at', 'shelf1', 'livingroom1'], ['at', 'sidetable1', 'livingroom1']], 'known_entities_ratio': 0.20689655172413793, 'visited_rooms': ['livingroom1'], 'visited_rooms_ratio': 0.16666666666666666, 'known_goal_entities': [['at', 'book1', 'livingroom1']], 'known_goal_entities_ratio': 0.25}}]] 
	  } 
	   } 
	   } 
	 & & \\ 
 

    \theutterance \stepcounter{utterance}  

    & & & \multicolumn{2}{p{0.3\linewidth}}{\cellcolor[rgb]{0.95,0.95,0.95}{%
	\makecell[{{p{\linewidth}}}]{% 
	  \tt {\tiny [GM$|$GM]}  
	 0.14285714285714285 
	  } 
	   } 
	   } 
	 & & \\ 
 

    \theutterance \stepcounter{utterance}  

    & & & \multicolumn{2}{p{0.3\linewidth}}{\cellcolor[rgb]{0.95,0.95,0.95}{%
	\makecell[{{p{\linewidth}}}]{% 
	  \tt {\tiny [GM$|$GM]}  
	 Turn 1 
	  } 
	   } 
	   } 
	 & & \\ 
 

    \theutterance \stepcounter{utterance}  

    & & \multicolumn{4}{p{0.6\linewidth}}{\cellcolor[rgb]{0.9,0.9,0.9}{%
	\makecell[{{p{\linewidth}}}]{% 
	  \tt {\tiny [A$\langle$GM]}  
	 I don't know how to interpret this 'move' action. 
	  } 
	   } 
	   } 
	 & \\ 
 

    \theutterance \stepcounter{utterance}  

    & \multicolumn{4}{p{0.6\linewidth}}{\cellcolor[rgb]{0.8,1,0.9}{%
	\makecell[{{p{\linewidth}}}]{% 
	  \tt {\tiny [A$\rangle$GM]}  
	 Apologies for any confusion. In the context of a text adventure game, "move towards" or simply "move" is a standard command to change your character's location within the game world. When you "move towards" an object or a direction, you are essentially going closer to it or proceeding in that direction.\\ \tt \\ \tt Here's a revised response that sticks to more explicit commands:\\ \tt \\ \tt > approach side table\\ \tt \\ \tt Next actions:\\ \tt 1. pick up book\\ \tt 2. walk to 
	  } 
	   } 
	   } 
	 & & \\ 
 

    \theutterance \stepcounter{utterance}  

    & & & \multicolumn{2}{p{0.3\linewidth}}{\cellcolor[rgb]{0.95,0.95,0.95}{%
	\makecell[{{p{\linewidth}}}]{% 
	  \tt {\tiny [GM$|$GM]}  
	 command_tag_missing 
	  } 
	   } 
	   } 
	 & & \\ 
 

    \theutterance \stepcounter{utterance}  

    & & & \multicolumn{2}{p{0.3\linewidth}}{\cellcolor[rgb]{0.95,0.95,0.95}{%
	\makecell[{{p{\linewidth}}}]{% 
	  \tt {\tiny [GM$|$GM]}  
	 command_tag_missing 
	  } 
	   } 
	   } 
	 & & \\ 
 

    \theutterance \stepcounter{utterance}  

    & & & \multicolumn{2}{p{0.3\linewidth}}{\cellcolor[rgb]{0.95,0.95,0.95}{%
	\makecell[{{p{\linewidth}}}]{% 
	  \tt {\tiny [GM$|$GM]}  
	 {'goal_states_achieved': [], 'game_successfully_finished': False} 
	  } 
	   } 
	   } 
	 & & \\ 
 

\end{supertabular}
}

\end{document}

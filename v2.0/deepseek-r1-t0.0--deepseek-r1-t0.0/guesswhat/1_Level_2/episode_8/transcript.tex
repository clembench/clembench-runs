
\documentclass{article}
\usepackage{colortbl}
\usepackage{makecell}
\usepackage{multirow}
\usepackage{supertabular}

\begin{document}

\newcounter{utterance}

\twocolumn

{ \footnotesize  \setcounter{utterance}{1}
\setlength{\tabcolsep}{0pt}
\begin{supertabular}{c@{$\;$}|p{.15\linewidth}@{}p{.15\linewidth}p{.15\linewidth}p{.15\linewidth}p{.15\linewidth}p{.15\linewidth}}

    \# & $\;$A & \multicolumn{4}{c}{Game Master} & $\;\:$B\\
    \hline 

    \theutterance \stepcounter{utterance}  

    & & \multicolumn{4}{p{0.6\linewidth}}{\cellcolor[rgb]{0.9,0.9,0.9}{%
	\makecell[{{p{\linewidth}}}]{% 
	  \tt {\tiny [A$\langle$GM]}  
	 Let's play a guessing game. Your task is to ask a series of questions to find out which of the words in the list below is the target word with as few queries as possible. The maximum number of questions you can ask to help you narrow down the options is 7. Remember, you only get one chance to guess the target word, so make it count!\\ \tt \\ \tt The rules of the game are: \\ \tt \\ \tt 1. Only ask one question per turn. DO NOT ASK MORE THAN ONE QUESTION.\\ \tt \\ \tt 2. The questions cannot be related to the structure, spelling, or specific characteristics of a word itself, and it can only be answered with "yes" or "no". \\ \tt \\ \tt 3. Use the following format for asking questions and DO NOT ADD ANY TEXT OTHER THAN THIS: \\ \tt \\ \tt QUESTION: <your question>\\ \tt \\ \tt 4. You have only one opportunity to guess the target word. To make your guess, use the following format and DO NOT ADD ANY TEXT OTHER THAN THIS: \\ \tt \\ \tt GUESS: <target word>\\ \tt \\ \tt Remember you cannot ask for a hint. Make your question brief. Let’s start!\\ \tt \\ \tt List of candidate words: ['Raven', 'Crow', 'Starling', 'Oriole', 'Rugby', 'Soccer', 'Gymnastics', 'Racing'] 
	  } 
	   } 
	   } 
	 & \\ 
 

    \theutterance \stepcounter{utterance}  

    & \multicolumn{4}{p{0.6\linewidth}}{\cellcolor[rgb]{0.8,1,0.9}{%
	\makecell[{{p{\linewidth}}}]{% 
	  \tt {\tiny [A$\rangle$GM]}  
	 QUESTION: Is the target word related to a bird?\\ \tt \\ \tt Assuming the answer is **yes**, proceed with the bird category.\\ \tt \\ \tt **Next question** to narrow down among Raven, Crow, Starling, Oriole:  \\ \tt QUESTION: Is the bird commonly associated with intelligence or problem-solving?\\ \tt \\ \tt If **yes**, focus on Raven/Crow.  \\ \tt QUESTION: Is the bird often linked to mythology or omens?  \\ \tt \\ \tt - If **yes**: GUESS: Raven  \\ \tt - If **no**: GUESS: Crow  \\ \tt \\ \tt If the answer to the second question is **no** (not associated with intelligence):  \\ \tt QUESTION: Is the bird known for vibrant plumage?  \\ \tt \\ \tt - If **yes**: GUESS: Oriole  \\ \tt - If **no**: GUESS: Starling  \\ \tt \\ \tt ---\\ \tt \\ \tt If the first answer is **no** (sport), proceed with the sports category.  \\ \tt \\ \tt **Next question**:  \\ \tt QUESTION: Is the sport played with a ball?  \\ \tt \\ \tt If **yes**, focus on Rugby/Soccer:  \\ \tt QUESTION: Is the sport primarily played with the feet?  \\ \tt \\ \tt - If **yes**: GUESS: Soccer  \\ \tt - If **no**: GUESS: Rugby  \\ \tt \\ \tt If the answer to the second question is **no** (no ball):  \\ \tt QUESTION: Is the sport performed on a track or circuit?  \\ \tt \\ \tt - If **yes**: GUESS: Racing  \\ \tt - If **no**: GUESS: Gymnastics  \\ \tt \\ \tt ---\\ \tt \\ \tt **Final Guess Path Example** (if answers are optimized):  \\ \tt 1. QUESTION: Is the target word related to a bird?  \\ \tt    - **Yes** → Birds.  \\ \tt 2. QUESTION: Is the bird commonly associated with intelligence or problem-solving?  \\ \tt    - **Yes** → Raven/Crow.  \\ \tt 3. QUESTION: Is the bird often linked to mythology or omens?  \\ \tt    - **Yes** → GUESS: Raven  \\ \tt \\ \tt Or:  \\ \tt 1. QUESTION: Is the target word related to a bird?  \\ \tt    - **No** → Sports.  \\ \tt 2. QUESTION: Is the sport played with a ball?  \\ \tt    - **Yes** → Rugby/Soccer.  \\ \tt 3. QUESTION: Is the sport primarily played with the feet?  \\ \tt    - **Yes** → GUESS: Soccer  \\ \tt \\ \tt This approach uses 3–4 questions, staying within the 7-query limit. 
	  } 
	   } 
	   } 
	 & & \\ 
 

    \theutterance \stepcounter{utterance}  

    & & & \multicolumn{2}{p{0.3\linewidth}}{\cellcolor[rgb]{0.95,0.95,0.95}{%
	\makecell[{{p{\linewidth}}}]{% 
	  \tt {\tiny [GM$|$GM]}  
	 Multiple questions detected in a single turn. 
	  } 
	   } 
	   } 
	 & & \\ 
 

    \theutterance \stepcounter{utterance}  

    & & & \multicolumn{2}{p{0.3\linewidth}}{\cellcolor[rgb]{0.95,0.95,0.95}{%
	\makecell[{{p{\linewidth}}}]{% 
	  \tt {\tiny [GM$|$GM]}  
	 abort game 
	  } 
	   } 
	   } 
	 & & \\ 
 

\end{supertabular}
}

\end{document}

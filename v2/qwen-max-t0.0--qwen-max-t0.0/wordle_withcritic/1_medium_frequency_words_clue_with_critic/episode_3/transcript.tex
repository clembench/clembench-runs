
\documentclass{article}
\usepackage{colortbl}
\usepackage{makecell}
\usepackage{multirow}
\usepackage{supertabular}

\begin{document}

\newcounter{utterance}

\twocolumn

{ \footnotesize  \setcounter{utterance}{1}
\setlength{\tabcolsep}{0pt}
\begin{supertabular}{c@{$\;$}|p{.15\linewidth}@{}p{.15\linewidth}p{.15\linewidth}p{.15\linewidth}p{.15\linewidth}p{.15\linewidth}}

    \# & $\;$A & \multicolumn{4}{c}{Game Master} & $\;\:$B\\
    \hline 

    \theutterance \stepcounter{utterance}  

    & & \multicolumn{4}{p{0.6\linewidth}}{\cellcolor[rgb]{0.9,0.9,0.9}{%
	\makecell[{{p{\linewidth}}}]{% 
	  \tt {\tiny [A$\langle$GM]}  
	 You are a language wizard who likes to guess words by using the given rules.\\ \tt \\ \tt Welcome to Wordle! You have six attempts to guess the target word, a valid English word of five lowercase letters (a-z). Please use the tags "explanation:" and "guess:" to provide a concise explanation for each guess.\\ \tt \\ \tt To help you make an informed guess, you will receive a clue for the word, such as\\ \tt clue: "snowy white"\\ \tt \\ \tt Here is an example guess based on the clue:\\ \tt explanation: In the fairy tail Snow White, the girl is killed because she eats a poisoned apple. And the word apple has 5 letters.\\ \tt guess: apple\\ \tt \\ \tt I will then indicate whether I agree or disagree with your guess and provide rationale, but agreeing with a guess does not confirm its correctness. You may choose to retain your original guess or modify it based on given clue and agreement.\\ \tt \\ \tt After each guess, your answer will be validated, and you will receive feedback indicating which letters are correct (green), which letters are correct but in the wrong position (yellow), and which letters are incorrect (red). This feedback can be useful in determining which letters to include or exclude in your next guess.\\ \tt \\ \tt For example, the feedback for "apple" might be:\\ \tt guess_feedback: a<yellow> p<yellow> p<green> l<yellow> e<red>\\ \tt \\ \tt Here are some rules to keep in mind:\\ \tt 1. Please do not generate any guess feedback or clue on your own\\ \tt 2. Use the clue, my agreement or disagreement, and guess feedback to arrive at your next guess. This should be reflected in your explanation as well.\\ \tt \\ \tt Do not generate any other text.\\ \tt \\ \tt Let's begin with your first guess.\\ \tt \\ \tt clue: fussbudget\\ \tt  
	  } 
	   } 
	   } 
	 & \\ 
 

    \theutterance \stepcounter{utterance}  

    & \multicolumn{4}{p{0.6\linewidth}}{\cellcolor[rgb]{0.8,1,0.9}{%
	\makecell[{{p{\linewidth}}}]{% 
	  \tt {\tiny [A$\rangle$GM]}  
	 explanation: The clue "fussbudget" suggests a word related to someone who is overly particular or fussy. A five-letter word that fits this theme could be "prude," which refers to someone who is excessively concerned with propriety or modesty.\\ \tt guess: prude 
	  } 
	   } 
	   } 
	 & & \\ 
 

    \theutterance \stepcounter{utterance}  

    & & \multicolumn{4}{p{0.6\linewidth}}{\cellcolor[rgb]{0.9,0.9,0.9}{%
	\makecell[{{p{\linewidth}}}]{% 
	  \tt {\tiny [GM$\rangle$B]}  
	 I need your assistance with a word game in which we need to find a 5-letter word using a clue, a guess and an explanation for the guess.\\ \tt \\ \tt For example, if the clue is "snow white" and the guess is "apple", you will get the information like this:\\ \tt clue: snow white\\ \tt explanation: the fairy tale Snow White involves the girl eating a poisoned apple, which has 5 letters.\\ \tt guess: apple\\ \tt \\ \tt Your task is to either agree or disagree with my guess based on the given clue.\\ \tt \\ \tt If you disagree, I will make another guess, and if you agree, I will proceed in the game using the same guess.\\ \tt \\ \tt Initially, guess feedback is not available, but as the game proceeds, it becomes available. The guess feedback is not for current guess, but rather an indication of what letters can be present in the current guess. A letter highlighted in green indicates that it is a correct letter in the correct position, while yellow indicates that it is a correct letter in the wrong position, and red indicates a wrong letter. At this point, you should use all the provided information, including the clue, guess, explanation, and guess feedback, to decide whether the given guess matches the clue, and also stick to guess feedback or not.\\ \tt \\ \tt Please respond in lowercase letters and stick to this format:\\ \tt explanation: your reason for doing so\\ \tt agreement: yes or no\\ \tt \\ \tt Please note that you must not guess any word based on the riddle clue and stick to the given format while responding.\\ \tt \\ \tt Do not generate any other text.\\ \tt \\ \tt Let's begin!\\ \tt \\ \tt clue: fussbudget\\ \tt explanation: The clue "fussbudget" suggests a word related to someone who is overly particular or fussy. A five-letter word that fits this theme could be "prude," which refers to someone who is excessively concerned with propriety or modesty.\\ \tt guess: prude 
	  } 
	   } 
	   } 
	 & \\ 
 

    \theutterance \stepcounter{utterance}  

    & & & \multicolumn{4}{p{0.6\linewidth}}{\cellcolor[rgb]{1,0.85,0.72}{%
	\makecell[{{p{\linewidth}}}]{% 
	  \tt {\tiny [GM$\langle$B]}  
	 explanation: the word "prude" aligns with the theme of fussbudget, as it describes someone who is overly concerned with propriety, which can be related to being fussy.\\ \tt agreement: yes 
	  } 
	   } 
	   } 
	  \\ 
 

    \theutterance \stepcounter{utterance}  

    & & & \multicolumn{2}{p{0.3\linewidth}}{\cellcolor[rgb]{0.95,0.95,0.95}{%
	\makecell[{{p{\linewidth}}}]{% 
	  \tt {\tiny [GM$|$GM]}  
	 Critic agrees with the Guesser -- Sharing the critic's explanation with the guesser 
	  } 
	   } 
	   } 
	 & & \\ 
 

    \theutterance \stepcounter{utterance}  

    & & \multicolumn{4}{p{0.6\linewidth}}{\cellcolor[rgb]{0.9,0.9,0.9}{%
	\makecell[{{p{\linewidth}}}]{% 
	  \tt {\tiny [A$\langle$GM]}  
	 clue: fussbudget\\ \tt guess_agreement: yes\\ \tt agreement_explanation: the word "prude" aligns with the theme of fussbudget, as it describes someone who is overly concerned with propriety, which can be related to being fussy.\\ \tt \\ \tt Provide your response only in this format.\\ \tt explanation: details\\ \tt guess: word 
	  } 
	   } 
	   } 
	 & \\ 
 

    \theutterance \stepcounter{utterance}  

    & \multicolumn{4}{p{0.6\linewidth}}{\cellcolor[rgb]{0.8,1,0.9}{%
	\makecell[{{p{\linewidth}}}]{% 
	  \tt {\tiny [A$\rangle$GM]}  
	 explanation: The agreement confirms that "prude" is a valid guess in line with the clue. However, I should consider other five-letter words that also align with the theme of being fussy or particular. Another possibility is "nitpick," but since it's six letters, I'll adjust to "nudge," which can imply gently pushing someone to be more particular or fussy.\\ \tt guess: nudge 
	  } 
	   } 
	   } 
	 & & \\ 
 

    \theutterance \stepcounter{utterance}  

    & & & \multicolumn{2}{p{0.3\linewidth}}{\cellcolor[rgb]{0.95,0.95,0.95}{%
	\makecell[{{p{\linewidth}}}]{% 
	  \tt {\tiny [GM$|$GM]}  
	 Change in player1's guess.\\ \tt guess_before_critic_opinion: prude\\ \tt                                 critic_agreement: yes\\ \tt guess_after_critic_opinion: nudge\\ \tt                                                 Proceeding with guess validation 
	  } 
	   } 
	   } 
	 & & \\ 
 

    \theutterance \stepcounter{utterance}  

    & & & \multicolumn{2}{p{0.3\linewidth}}{\cellcolor[rgb]{0.95,0.95,0.95}{%
	\makecell[{{p{\linewidth}}}]{% 
	  \tt {\tiny [GM$|$GM]}  
	 attempts = 1\\ \tt target_word = biddy\\ \tt target_word_clue = fussbudget\\ \tt guess = nudge\\ \tt guess_feedback = n<red> u<red> d<green> g<red> e<red> 
	  } 
	   } 
	   } 
	 & & \\ 
 

    \theutterance \stepcounter{utterance}  

    & & \multicolumn{4}{p{0.6\linewidth}}{\cellcolor[rgb]{0.9,0.9,0.9}{%
	\makecell[{{p{\linewidth}}}]{% 
	  \tt {\tiny [A$\langle$GM]}  
	 guess_feedback: n<red> u<red> d<green> g<red> e<red>\\ \tt \\ \tt Provide your response only in this format.\\ \tt explanation: details\\ \tt guess: word 
	  } 
	   } 
	   } 
	 & \\ 
 

    \theutterance \stepcounter{utterance}  

    & \multicolumn{4}{p{0.6\linewidth}}{\cellcolor[rgb]{0.8,1,0.9}{%
	\makecell[{{p{\linewidth}}}]{% 
	  \tt {\tiny [A$\rangle$GM]}  
	 explanation: The feedback indicates that only the letter "d" is correct and in the right position. This means the other letters (n, u, g, e) are not part of the target word. I need to focus on words that have "d" as the third letter and fit the theme of being fussy or particular. A possible word is "tardy," which can imply someone who is slow or overly meticulous in their actions.\\ \tt guess: tardy 
	  } 
	   } 
	   } 
	 & & \\ 
 

    \theutterance \stepcounter{utterance}  

    & & \multicolumn{4}{p{0.6\linewidth}}{\cellcolor[rgb]{0.9,0.9,0.9}{%
	\makecell[{{p{\linewidth}}}]{% 
	  \tt {\tiny [GM$\rangle$B]}  
	 clue: fussbudget\\ \tt explanation: The feedback indicates that only the letter "d" is correct and in the right position. This means the other letters (n, u, g, e) are not part of the target word. I need to focus on words that have "d" as the third letter and fit the theme of being fussy or particular. A possible word is "tardy," which can imply someone who is slow or overly meticulous in their actions.\\ \tt guess: tardy\\ \tt \\ \tt Provide your response only in this format.\\ \tt explanation: details\\ \tt agreement: yes or no 
	  } 
	   } 
	   } 
	 & \\ 
 

    \theutterance \stepcounter{utterance}  

    & & & \multicolumn{4}{p{0.6\linewidth}}{\cellcolor[rgb]{1,0.85,0.72}{%
	\makecell[{{p{\linewidth}}}]{% 
	  \tt {\tiny [GM$\langle$B]}  
	 explanation: the word "tardy" does not directly relate to the theme of being fussy or particular, and the clue feedback indicates that only the letter "d" is correct and in the right position. while "tardy" has "d" in the correct spot, it doesn't strongly fit the fussbudget theme.\\ \tt agreement: no 
	  } 
	   } 
	   } 
	  \\ 
 

    \theutterance \stepcounter{utterance}  

    & & & \multicolumn{2}{p{0.3\linewidth}}{\cellcolor[rgb]{0.95,0.95,0.95}{%
	\makecell[{{p{\linewidth}}}]{% 
	  \tt {\tiny [GM$|$GM]}  
	 Critic does not agree with the Guesser -- Sharing the critic's explanation with the guesser 
	  } 
	   } 
	   } 
	 & & \\ 
 

    \theutterance \stepcounter{utterance}  

    & & \multicolumn{4}{p{0.6\linewidth}}{\cellcolor[rgb]{0.9,0.9,0.9}{%
	\makecell[{{p{\linewidth}}}]{% 
	  \tt {\tiny [A$\langle$GM]}  
	 clue: fussbudget\\ \tt guess_agreement: no\\ \tt agreement_explanation: the word "tardy" does not directly relate to the theme of being fussy or particular, and the clue feedback indicates that only the letter "d" is correct and in the right position. while "tardy" has "d" in the correct spot, it doesn't strongly fit the fussbudget theme.\\ \tt \\ \tt Provide your response only in this format.\\ \tt explanation: details\\ \tt guess: word 
	  } 
	   } 
	   } 
	 & \\ 
 

    \theutterance \stepcounter{utterance}  

    & \multicolumn{4}{p{0.6\linewidth}}{\cellcolor[rgb]{0.8,1,0.9}{%
	\makecell[{{p{\linewidth}}}]{% 
	  \tt {\tiny [A$\rangle$GM]}  
	 explanation: The disagreement highlights that "tardy" does not align well with the theme of being fussy. I need to focus on words with "d" as the third letter that better fit the fussbudget theme. A word like "dandy" could work, as it can describe someone who is overly concerned with appearance or detail.\\ \tt guess: dandy 
	  } 
	   } 
	   } 
	 & & \\ 
 

    \theutterance \stepcounter{utterance}  

    & & & \multicolumn{2}{p{0.3\linewidth}}{\cellcolor[rgb]{0.95,0.95,0.95}{%
	\makecell[{{p{\linewidth}}}]{% 
	  \tt {\tiny [GM$|$GM]}  
	 Change in player1's guess.\\ \tt guess_before_critic_opinion: tardy\\ \tt                                 critic_agreement: no\\ \tt guess_after_critic_opinion: dandy\\ \tt                                                 Proceeding with guess validation 
	  } 
	   } 
	   } 
	 & & \\ 
 

    \theutterance \stepcounter{utterance}  

    & & & \multicolumn{2}{p{0.3\linewidth}}{\cellcolor[rgb]{0.95,0.95,0.95}{%
	\makecell[{{p{\linewidth}}}]{% 
	  \tt {\tiny [GM$|$GM]}  
	 attempts = 2\\ \tt target_word = biddy\\ \tt target_word_clue = fussbudget\\ \tt guess = dandy\\ \tt guess_feedback = d<yellow> a<red> n<red> d<green> y<green> 
	  } 
	   } 
	   } 
	 & & \\ 
 

    \theutterance \stepcounter{utterance}  

    & & \multicolumn{4}{p{0.6\linewidth}}{\cellcolor[rgb]{0.9,0.9,0.9}{%
	\makecell[{{p{\linewidth}}}]{% 
	  \tt {\tiny [A$\langle$GM]}  
	 guess_feedback: d<yellow> a<red> n<red> d<green> y<green>\\ \tt \\ \tt Provide your response only in this format.\\ \tt explanation: details\\ \tt guess: word 
	  } 
	   } 
	   } 
	 & \\ 
 

    \theutterance \stepcounter{utterance}  

    & \multicolumn{4}{p{0.6\linewidth}}{\cellcolor[rgb]{0.8,1,0.9}{%
	\makecell[{{p{\linewidth}}}]{% 
	  \tt {\tiny [A$\rangle$GM]}  
	 explanation: The feedback shows that "d" is correct but in the wrong position (yellow), while the second "d" and "y" are correct and in the right positions (green). This means the word ends with "dy" and has a "d" elsewhere. A possible word that fits this pattern and relates to the fussbudget theme is "dodgy," which can imply something or someone that is suspicious or overly meticulous in a tricky way.\\ \tt guess: dodgy 
	  } 
	   } 
	   } 
	 & & \\ 
 

    \theutterance \stepcounter{utterance}  

    & & \multicolumn{4}{p{0.6\linewidth}}{\cellcolor[rgb]{0.9,0.9,0.9}{%
	\makecell[{{p{\linewidth}}}]{% 
	  \tt {\tiny [GM$\rangle$B]}  
	 clue: fussbudget\\ \tt explanation: The feedback shows that "d" is correct but in the wrong position (yellow), while the second "d" and "y" are correct and in the right positions (green). This means the word ends with "dy" and has a "d" elsewhere. A possible word that fits this pattern and relates to the fussbudget theme is "dodgy," which can imply something or someone that is suspicious or overly meticulous in a tricky way.\\ \tt guess: dodgy\\ \tt \\ \tt Provide your response only in this format.\\ \tt explanation: details\\ \tt agreement: yes or no 
	  } 
	   } 
	   } 
	 & \\ 
 

    \theutterance \stepcounter{utterance}  

    & & & \multicolumn{4}{p{0.6\linewidth}}{\cellcolor[rgb]{1,0.85,0.72}{%
	\makecell[{{p{\linewidth}}}]{% 
	  \tt {\tiny [GM$\langle$B]}  
	 explanation: the word "dodgy" fits the feedback pattern, with "d" in the wrong position, and "dy" at the end in the correct positions. while "dodgy" can imply something tricky or unreliable, it doesn't strongly align with the theme of being overly particular or fussy.\\ \tt agreement: no 
	  } 
	   } 
	   } 
	  \\ 
 

    \theutterance \stepcounter{utterance}  

    & & & \multicolumn{2}{p{0.3\linewidth}}{\cellcolor[rgb]{0.95,0.95,0.95}{%
	\makecell[{{p{\linewidth}}}]{% 
	  \tt {\tiny [GM$|$GM]}  
	 Critic does not agree with the Guesser -- Sharing the critic's explanation with the guesser 
	  } 
	   } 
	   } 
	 & & \\ 
 

    \theutterance \stepcounter{utterance}  

    & & \multicolumn{4}{p{0.6\linewidth}}{\cellcolor[rgb]{0.9,0.9,0.9}{%
	\makecell[{{p{\linewidth}}}]{% 
	  \tt {\tiny [A$\langle$GM]}  
	 clue: fussbudget\\ \tt guess_agreement: no\\ \tt agreement_explanation: the word "dodgy" fits the feedback pattern, with "d" in the wrong position, and "dy" at the end in the correct positions. while "dodgy" can imply something tricky or unreliable, it doesn't strongly align with the theme of being overly particular or fussy.\\ \tt \\ \tt Provide your response only in this format.\\ \tt explanation: details\\ \tt guess: word 
	  } 
	   } 
	   } 
	 & \\ 
 

    \theutterance \stepcounter{utterance}  

    & \multicolumn{4}{p{0.6\linewidth}}{\cellcolor[rgb]{0.8,1,0.9}{%
	\makecell[{{p{\linewidth}}}]{% 
	  \tt {\tiny [A$\rangle$GM]}  
	 explanation: The disagreement suggests that "dodgy" does not strongly align with the theme of being fussy. I need to find a word that fits the feedback pattern (d<yellow>, d<green>, y<green>) and better matches the fussbudget theme. A word like "daddy" could fit, as it has the required letters in the correct positions and might loosely relate to someone who is overly protective or particular.\\ \tt guess: daddy 
	  } 
	   } 
	   } 
	 & & \\ 
 

    \theutterance \stepcounter{utterance}  

    & & & \multicolumn{2}{p{0.3\linewidth}}{\cellcolor[rgb]{0.95,0.95,0.95}{%
	\makecell[{{p{\linewidth}}}]{% 
	  \tt {\tiny [GM$|$GM]}  
	 Change in player1's guess.\\ \tt guess_before_critic_opinion: dodgy\\ \tt                                 critic_agreement: no\\ \tt guess_after_critic_opinion: daddy\\ \tt                                                 Proceeding with guess validation 
	  } 
	   } 
	   } 
	 & & \\ 
 

    \theutterance \stepcounter{utterance}  

    & & & \multicolumn{2}{p{0.3\linewidth}}{\cellcolor[rgb]{0.95,0.95,0.95}{%
	\makecell[{{p{\linewidth}}}]{% 
	  \tt {\tiny [GM$|$GM]}  
	 attempts = 3\\ \tt target_word = biddy\\ \tt target_word_clue = fussbudget\\ \tt guess = daddy\\ \tt guess_feedback = d<red> a<red> d<green> d<green> y<green> 
	  } 
	   } 
	   } 
	 & & \\ 
 

    \theutterance \stepcounter{utterance}  

    & & \multicolumn{4}{p{0.6\linewidth}}{\cellcolor[rgb]{0.9,0.9,0.9}{%
	\makecell[{{p{\linewidth}}}]{% 
	  \tt {\tiny [A$\langle$GM]}  
	 guess_feedback: d<red> a<red> d<green> d<green> y<green>\\ \tt \\ \tt Provide your response only in this format.\\ \tt explanation: details\\ \tt guess: word 
	  } 
	   } 
	   } 
	 & \\ 
 

    \theutterance \stepcounter{utterance}  

    & \multicolumn{4}{p{0.6\linewidth}}{\cellcolor[rgb]{0.8,1,0.9}{%
	\makecell[{{p{\linewidth}}}]{% 
	  \tt {\tiny [A$\rangle$GM]}  
	 explanation: The feedback indicates that the first "d" is incorrect, while the second and third "d" and "y" are correct and in the right positions. This means the word ends with "ddy" and does not start with "d" or "a." A possible word that fits this pattern and relates to the fussbudget theme is "muddy," which can describe something unclear or overly complicated, akin to being fussy.\\ \tt guess: muddy 
	  } 
	   } 
	   } 
	 & & \\ 
 

    \theutterance \stepcounter{utterance}  

    & & \multicolumn{4}{p{0.6\linewidth}}{\cellcolor[rgb]{0.9,0.9,0.9}{%
	\makecell[{{p{\linewidth}}}]{% 
	  \tt {\tiny [GM$\rangle$B]}  
	 clue: fussbudget\\ \tt explanation: The feedback indicates that the first "d" is incorrect, while the second and third "d" and "y" are correct and in the right positions. This means the word ends with "ddy" and does not start with "d" or "a." A possible word that fits this pattern and relates to the fussbudget theme is "muddy," which can describe something unclear or overly complicated, akin to being fussy.\\ \tt guess: muddy\\ \tt \\ \tt Provide your response only in this format.\\ \tt explanation: details\\ \tt agreement: yes or no 
	  } 
	   } 
	   } 
	 & \\ 
 

    \theutterance \stepcounter{utterance}  

    & & & \multicolumn{4}{p{0.6\linewidth}}{\cellcolor[rgb]{1,0.85,0.72}{%
	\makecell[{{p{\linewidth}}}]{% 
	  \tt {\tiny [GM$\langle$B]}  
	 explanation: the word "muddy" fits the feedback pattern, with "ddy" at the end in the correct positions and no "d" or "a" at the start. while "muddy" can imply something unclear or messy, it doesn't strongly align with the theme of being overly particular or fussy.\\ \tt agreement: no 
	  } 
	   } 
	   } 
	  \\ 
 

    \theutterance \stepcounter{utterance}  

    & & & \multicolumn{2}{p{0.3\linewidth}}{\cellcolor[rgb]{0.95,0.95,0.95}{%
	\makecell[{{p{\linewidth}}}]{% 
	  \tt {\tiny [GM$|$GM]}  
	 Critic does not agree with the Guesser -- Sharing the critic's explanation with the guesser 
	  } 
	   } 
	   } 
	 & & \\ 
 

    \theutterance \stepcounter{utterance}  

    & & \multicolumn{4}{p{0.6\linewidth}}{\cellcolor[rgb]{0.9,0.9,0.9}{%
	\makecell[{{p{\linewidth}}}]{% 
	  \tt {\tiny [A$\langle$GM]}  
	 clue: fussbudget\\ \tt guess_agreement: no\\ \tt agreement_explanation: the word "muddy" fits the feedback pattern, with "ddy" at the end in the correct positions and no "d" or "a" at the start. while "muddy" can imply something unclear or messy, it doesn't strongly align with the theme of being overly particular or fussy.\\ \tt \\ \tt Provide your response only in this format.\\ \tt explanation: details\\ \tt guess: word 
	  } 
	   } 
	   } 
	 & \\ 
 

    \theutterance \stepcounter{utterance}  

    & \multicolumn{4}{p{0.6\linewidth}}{\cellcolor[rgb]{0.8,1,0.9}{%
	\makecell[{{p{\linewidth}}}]{% 
	  \tt {\tiny [A$\rangle$GM]}  
	 explanation: The disagreement suggests that "muddy" does not align well with the theme of being fussy. I need to find another word that fits the feedback pattern (d<green>, d<green>, y<green>) and better matches the fussbudget theme. A word like "pudgy" could work, as it has the required letters in the correct positions and might loosely relate to someone being overly concerned with appearance or detail.\\ \tt guess: pudgy 
	  } 
	   } 
	   } 
	 & & \\ 
 

    \theutterance \stepcounter{utterance}  

    & & & \multicolumn{2}{p{0.3\linewidth}}{\cellcolor[rgb]{0.95,0.95,0.95}{%
	\makecell[{{p{\linewidth}}}]{% 
	  \tt {\tiny [GM$|$GM]}  
	 Change in player1's guess.\\ \tt guess_before_critic_opinion: muddy\\ \tt                                 critic_agreement: no\\ \tt guess_after_critic_opinion: pudgy\\ \tt                                                 Proceeding with guess validation 
	  } 
	   } 
	   } 
	 & & \\ 
 

    \theutterance \stepcounter{utterance}  

    & & & \multicolumn{2}{p{0.3\linewidth}}{\cellcolor[rgb]{0.95,0.95,0.95}{%
	\makecell[{{p{\linewidth}}}]{% 
	  \tt {\tiny [GM$|$GM]}  
	 attempts = 4\\ \tt target_word = biddy\\ \tt target_word_clue = fussbudget\\ \tt guess = pudgy\\ \tt guess_feedback = p<red> u<red> d<green> g<red> y<green> 
	  } 
	   } 
	   } 
	 & & \\ 
 

    \theutterance \stepcounter{utterance}  

    & & \multicolumn{4}{p{0.6\linewidth}}{\cellcolor[rgb]{0.9,0.9,0.9}{%
	\makecell[{{p{\linewidth}}}]{% 
	  \tt {\tiny [A$\langle$GM]}  
	 guess_feedback: p<red> u<red> d<green> g<red> y<green>\\ \tt \\ \tt Provide your response only in this format.\\ \tt explanation: details\\ \tt guess: word 
	  } 
	   } 
	   } 
	 & \\ 
 

    \theutterance \stepcounter{utterance}  

    & \multicolumn{4}{p{0.6\linewidth}}{\cellcolor[rgb]{0.8,1,0.9}{%
	\makecell[{{p{\linewidth}}}]{% 
	  \tt {\tiny [A$\rangle$GM]}  
	 explanation: The feedback shows that the second "d" and "y" are correct and in the right positions, while "p," "u," and "g" are incorrect. This means the word ends with "dy" and does not contain "p," "u," or "g." A possible word that fits this pattern and relates to the fussbudget theme is "soddy," which could describe something overly meticulous or fussy in a trivial way.\\ \tt guess: soddy 
	  } 
	   } 
	   } 
	 & & \\ 
 

    \theutterance \stepcounter{utterance}  

    & & \multicolumn{4}{p{0.6\linewidth}}{\cellcolor[rgb]{0.9,0.9,0.9}{%
	\makecell[{{p{\linewidth}}}]{% 
	  \tt {\tiny [GM$\rangle$B]}  
	 clue: fussbudget\\ \tt explanation: The feedback shows that the second "d" and "y" are correct and in the right positions, while "p," "u," and "g" are incorrect. This means the word ends with "dy" and does not contain "p," "u," or "g." A possible word that fits this pattern and relates to the fussbudget theme is "soddy," which could describe something overly meticulous or fussy in a trivial way.\\ \tt guess: soddy\\ \tt \\ \tt Provide your response only in this format.\\ \tt explanation: details\\ \tt agreement: yes or no 
	  } 
	   } 
	   } 
	 & \\ 
 

    \theutterance \stepcounter{utterance}  

    & & & \multicolumn{4}{p{0.6\linewidth}}{\cellcolor[rgb]{1,0.85,0.72}{%
	\makecell[{{p{\linewidth}}}]{% 
	  \tt {\tiny [GM$\langle$B]}  
	 explanation: the word "soddy" fits the feedback pattern, with "ddy" at the end in the correct positions and no incorrect letters like "p," "u," or "g." while "soddy" is not a common word, it could loosely relate to something fussy or overly particular in a trivial sense.\\ \tt agreement: yes 
	  } 
	   } 
	   } 
	  \\ 
 

    \theutterance \stepcounter{utterance}  

    & & & \multicolumn{2}{p{0.3\linewidth}}{\cellcolor[rgb]{0.95,0.95,0.95}{%
	\makecell[{{p{\linewidth}}}]{% 
	  \tt {\tiny [GM$|$GM]}  
	 Critic agrees with the Guesser -- Sharing the critic's explanation with the guesser 
	  } 
	   } 
	   } 
	 & & \\ 
 

    \theutterance \stepcounter{utterance}  

    & & \multicolumn{4}{p{0.6\linewidth}}{\cellcolor[rgb]{0.9,0.9,0.9}{%
	\makecell[{{p{\linewidth}}}]{% 
	  \tt {\tiny [A$\langle$GM]}  
	 clue: fussbudget\\ \tt guess_agreement: yes\\ \tt agreement_explanation: the word "soddy" fits the feedback pattern, with "ddy" at the end in the correct positions and no incorrect letters like "p," "u," or "g." while "soddy" is not a common word, it could loosely relate to something fussy or overly particular in a trivial sense.\\ \tt \\ \tt Provide your response only in this format.\\ \tt explanation: details\\ \tt guess: word 
	  } 
	   } 
	   } 
	 & \\ 
 

    \theutterance \stepcounter{utterance}  

    & \multicolumn{4}{p{0.6\linewidth}}{\cellcolor[rgb]{0.8,1,0.9}{%
	\makecell[{{p{\linewidth}}}]{% 
	  \tt {\tiny [A$\rangle$GM]}  
	 explanation: The agreement confirms that "soddy" aligns with the feedback and the theme of being fussy. Since this is the final guess and it fits all the criteria, I will retain "soddy" as the target word.\\ \tt guess: soddy 
	  } 
	   } 
	   } 
	 & & \\ 
 

    \theutterance \stepcounter{utterance}  

    & & & \multicolumn{2}{p{0.3\linewidth}}{\cellcolor[rgb]{0.95,0.95,0.95}{%
	\makecell[{{p{\linewidth}}}]{% 
	  \tt {\tiny [GM$|$GM]}  
	 No change in player1's guess.\\ \tt guess_before_critic_opinion: soddy\\ \tt                                 critic_agreement: yes\\ \tt guess_after_critic_opinion: soddy\\ \tt                                                 Proceeding with guess validation 
	  } 
	   } 
	   } 
	 & & \\ 
 

    \theutterance \stepcounter{utterance}  

    & & & \multicolumn{2}{p{0.3\linewidth}}{\cellcolor[rgb]{0.95,0.95,0.95}{%
	\makecell[{{p{\linewidth}}}]{% 
	  \tt {\tiny [GM$|$GM]}  
	 attempts = 5\\ \tt target_word = biddy\\ \tt target_word_clue = fussbudget\\ \tt guess = soddy\\ \tt guess_feedback = s<red> o<red> d<green> d<green> y<green> 
	  } 
	   } 
	   } 
	 & & \\ 
 

    \theutterance \stepcounter{utterance}  

    & & \multicolumn{4}{p{0.6\linewidth}}{\cellcolor[rgb]{0.9,0.9,0.9}{%
	\makecell[{{p{\linewidth}}}]{% 
	  \tt {\tiny [A$\langle$GM]}  
	 guess_feedback: s<red> o<red> d<green> d<green> y<green>\\ \tt \\ \tt Provide your response only in this format.\\ \tt explanation: details\\ \tt guess: word 
	  } 
	   } 
	   } 
	 & \\ 
 

    \theutterance \stepcounter{utterance}  

    & \multicolumn{4}{p{0.6\linewidth}}{\cellcolor[rgb]{0.8,1,0.9}{%
	\makecell[{{p{\linewidth}}}]{% 
	  \tt {\tiny [A$\rangle$GM]}  
	 explanation: The feedback indicates that "s" and "o" are incorrect, while the second "d" and "y" are correct and in the right positions. This means the word ends with "ddy" and does not contain "s" or "o." A possible word that fits this pattern and relates to the fussbudget theme is "buddy," which could describe someone who is overly particular or fussy in a friendly or trivial way.\\ \tt guess: buddy 
	  } 
	   } 
	   } 
	 & & \\ 
 

    \theutterance \stepcounter{utterance}  

    & & \multicolumn{4}{p{0.6\linewidth}}{\cellcolor[rgb]{0.9,0.9,0.9}{%
	\makecell[{{p{\linewidth}}}]{% 
	  \tt {\tiny [GM$\rangle$B]}  
	 clue: fussbudget\\ \tt explanation: The feedback indicates that "s" and "o" are incorrect, while the second "d" and "y" are correct and in the right positions. This means the word ends with "ddy" and does not contain "s" or "o." A possible word that fits this pattern and relates to the fussbudget theme is "buddy," which could describe someone who is overly particular or fussy in a friendly or trivial way.\\ \tt guess: buddy\\ \tt \\ \tt Provide your response only in this format.\\ \tt explanation: details\\ \tt agreement: yes or no 
	  } 
	   } 
	   } 
	 & \\ 
 

    \theutterance \stepcounter{utterance}  

    & & & \multicolumn{4}{p{0.6\linewidth}}{\cellcolor[rgb]{1,0.85,0.72}{%
	\makecell[{{p{\linewidth}}}]{% 
	  \tt {\tiny [GM$\langle$B]}  
	 explanation: the word "buddy" fits the feedback pattern, with "ddy" at the end in the correct positions and no incorrect letters like "s" or "o." however, "buddy" does not strongly align with the theme of being overly particular or fussy, as it generally refers to a friend or companion.\\ \tt agreement: no 
	  } 
	   } 
	   } 
	  \\ 
 

    \theutterance \stepcounter{utterance}  

    & & & \multicolumn{2}{p{0.3\linewidth}}{\cellcolor[rgb]{0.95,0.95,0.95}{%
	\makecell[{{p{\linewidth}}}]{% 
	  \tt {\tiny [GM$|$GM]}  
	 Critic does not agree with the Guesser -- Sharing the critic's explanation with the guesser 
	  } 
	   } 
	   } 
	 & & \\ 
 

    \theutterance \stepcounter{utterance}  

    & & \multicolumn{4}{p{0.6\linewidth}}{\cellcolor[rgb]{0.9,0.9,0.9}{%
	\makecell[{{p{\linewidth}}}]{% 
	  \tt {\tiny [A$\langle$GM]}  
	 clue: fussbudget\\ \tt guess_agreement: no\\ \tt agreement_explanation: the word "buddy" fits the feedback pattern, with "ddy" at the end in the correct positions and no incorrect letters like "s" or "o." however, "buddy" does not strongly align with the theme of being overly particular or fussy, as it generally refers to a friend or companion.\\ \tt \\ \tt Provide your response only in this format.\\ \tt explanation: details\\ \tt guess: word 
	  } 
	   } 
	   } 
	 & \\ 
 

    \theutterance \stepcounter{utterance}  

    & \multicolumn{4}{p{0.6\linewidth}}{\cellcolor[rgb]{0.8,1,0.9}{%
	\makecell[{{p{\linewidth}}}]{% 
	  \tt {\tiny [A$\rangle$GM]}  
	 explanation: The disagreement highlights that "buddy" does not align with the theme of being fussy. I need to find another word that fits the feedback pattern (d<green>, d<green>, y<green>) and better matches the fussbudget theme. A word like "teddy" could work, as it has the required letters in the correct positions and might loosely relate to someone being overly concerned with comfort or detail, akin to a fussbudget.\\ \tt guess: teddy 
	  } 
	   } 
	   } 
	 & & \\ 
 

    \theutterance \stepcounter{utterance}  

    & & & \multicolumn{2}{p{0.3\linewidth}}{\cellcolor[rgb]{0.95,0.95,0.95}{%
	\makecell[{{p{\linewidth}}}]{% 
	  \tt {\tiny [GM$|$GM]}  
	 Change in player1's guess.\\ \tt guess_before_critic_opinion: buddy\\ \tt                                 critic_agreement: no\\ \tt guess_after_critic_opinion: teddy\\ \tt                                                 Proceeding with guess validation 
	  } 
	   } 
	   } 
	 & & \\ 
 

    \theutterance \stepcounter{utterance}  

    & & & \multicolumn{2}{p{0.3\linewidth}}{\cellcolor[rgb]{0.95,0.95,0.95}{%
	\makecell[{{p{\linewidth}}}]{% 
	  \tt {\tiny [GM$|$GM]}  
	 attempts = 6\\ \tt target_word = biddy\\ \tt target_word_clue = fussbudget\\ \tt guess = teddy\\ \tt guess_feedback = t<red> e<red> d<green> d<green> y<green> 
	  } 
	   } 
	   } 
	 & & \\ 
 

    \theutterance \stepcounter{utterance}  

    & & & \multicolumn{2}{p{0.3\linewidth}}{\cellcolor[rgb]{0.95,0.95,0.95}{%
	\makecell[{{p{\linewidth}}}]{% 
	  \tt {\tiny [GM$|$GM]}  
	 game_result = LOSS 
	  } 
	   } 
	   } 
	 & & \\ 
 

\end{supertabular}
}

\end{document}

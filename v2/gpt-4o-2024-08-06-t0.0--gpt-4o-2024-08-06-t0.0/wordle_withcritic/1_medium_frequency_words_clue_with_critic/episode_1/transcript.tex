
\documentclass{article}
\usepackage{colortbl}
\usepackage{makecell}
\usepackage{multirow}
\usepackage{supertabular}

\begin{document}

\newcounter{utterance}

\twocolumn

{ \footnotesize  \setcounter{utterance}{1}
\setlength{\tabcolsep}{0pt}
\begin{supertabular}{c@{$\;$}|p{.15\linewidth}@{}p{.15\linewidth}p{.15\linewidth}p{.15\linewidth}p{.15\linewidth}p{.15\linewidth}}

    \# & $\;$A & \multicolumn{4}{c}{Game Master} & $\;\:$B\\
    \hline 

    \theutterance \stepcounter{utterance}  

    & & \multicolumn{4}{p{0.6\linewidth}}{\cellcolor[rgb]{0.9,0.9,0.9}{%
	\makecell[{{p{\linewidth}}}]{% 
	  \tt {\tiny [A$\langle$GM]}  
	 You are a language wizard who likes to guess words by using the given rules.\\ \tt \\ \tt Welcome to Wordle! You have six attempts to guess the target word, a valid English word of five lowercase letters (a-z). Please use the tags "explanation:" and "guess:" to provide a concise explanation for each guess.\\ \tt \\ \tt To help you make an informed guess, you will receive a clue for the word, such as\\ \tt clue: "snowy white"\\ \tt \\ \tt Here is an example guess based on the clue:\\ \tt explanation: In the fairy tail Snow White, the girl is killed because she eats a poisoned apple. And the word apple has 5 letters.\\ \tt guess: apple\\ \tt \\ \tt I will then indicate whether I agree or disagree with your guess and provide rationale, but agreeing with a guess does not confirm its correctness. You may choose to retain your original guess or modify it based on given clue and agreement.\\ \tt \\ \tt After each guess, your answer will be validated, and you will receive feedback indicating which letters are correct (green), which letters are correct but in the wrong position (yellow), and which letters are incorrect (red). This feedback can be useful in determining which letters to include or exclude in your next guess.\\ \tt \\ \tt For example, the feedback for "apple" might be:\\ \tt guess_feedback: a<yellow> p<yellow> p<green> l<yellow> e<red>\\ \tt \\ \tt Here are some rules to keep in mind:\\ \tt 1. Please do not generate any guess feedback or clue on your own\\ \tt 2. Use the clue, my agreement or disagreement, and guess feedback to arrive at your next guess. This should be reflected in your explanation as well.\\ \tt \\ \tt Do not generate any other text.\\ \tt \\ \tt Let's begin with your first guess.\\ \tt \\ \tt clue: "behold!"\\ \tt  
	  } 
	   } 
	   } 
	 & \\ 
 

    \theutterance \stepcounter{utterance}  

    & \multicolumn{4}{p{0.6\linewidth}}{\cellcolor[rgb]{0.8,1,0.9}{%
	\makecell[{{p{\linewidth}}}]{% 
	  \tt {\tiny [A$\rangle$GM]}  
	 explanation: The word "behold" is often associated with the word "lo," which is an archaic way of saying "look" or "see." The word "lo" is often used to draw attention to something, similar to "behold." The word "lo" is also part of the word "locus," which means a particular position or place, and "locus" has five letters.\\ \tt guess: locus 
	  } 
	   } 
	   } 
	 & & \\ 
 

    \theutterance \stepcounter{utterance}  

    & & \multicolumn{4}{p{0.6\linewidth}}{\cellcolor[rgb]{0.9,0.9,0.9}{%
	\makecell[{{p{\linewidth}}}]{% 
	  \tt {\tiny [GM$\rangle$B]}  
	 I need your assistance with a word game in which we need to find a 5-letter word using a clue, a guess and an explanation for the guess.\\ \tt \\ \tt For example, if the clue is "snow white" and the guess is "apple", you will get the information like this:\\ \tt clue: snow white\\ \tt explanation: the fairy tale Snow White involves the girl eating a poisoned apple, which has 5 letters.\\ \tt guess: apple\\ \tt \\ \tt Your task is to either agree or disagree with my guess based on the given clue.\\ \tt \\ \tt If you disagree, I will make another guess, and if you agree, I will proceed in the game using the same guess.\\ \tt \\ \tt Initially, guess feedback is not available, but as the game proceeds, it becomes available. The guess feedback is not for current guess, but rather an indication of what letters can be present in the current guess. A letter highlighted in green indicates that it is a correct letter in the correct position, while yellow indicates that it is a correct letter in the wrong position, and red indicates a wrong letter. At this point, you should use all the provided information, including the clue, guess, explanation, and guess feedback, to decide whether the given guess matches the clue, and also stick to guess feedback or not.\\ \tt \\ \tt Please respond in lowercase letters and stick to this format:\\ \tt explanation: your reason for doing so\\ \tt agreement: yes or no\\ \tt \\ \tt Please note that you must not guess any word based on the riddle clue and stick to the given format while responding.\\ \tt \\ \tt Do not generate any other text.\\ \tt \\ \tt Let's begin!\\ \tt \\ \tt clue: "behold!"\\ \tt explanation: The word "behold" is often associated with the word "lo," which is an archaic way of saying "look" or "see." The word "lo" is often used to draw attention to something, similar to "behold." The word "lo" is also part of the word "locus," which means a particular position or place, and "locus" has five letters.\\ \tt guess: locus 
	  } 
	   } 
	   } 
	 & \\ 
 

    \theutterance \stepcounter{utterance}  

    & & & \multicolumn{4}{p{0.6\linewidth}}{\cellcolor[rgb]{1,0.85,0.72}{%
	\makecell[{{p{\linewidth}}}]{% 
	  \tt {\tiny [GM$\langle$B]}  
	 explanation: the word "locus" does not directly relate to the exclamation "behold!" in a clear or commonly understood way.\\ \tt agreement: no 
	  } 
	   } 
	   } 
	  \\ 
 

    \theutterance \stepcounter{utterance}  

    & & & \multicolumn{2}{p{0.3\linewidth}}{\cellcolor[rgb]{0.95,0.95,0.95}{%
	\makecell[{{p{\linewidth}}}]{% 
	  \tt {\tiny [GM$|$GM]}  
	 Critic does not agree with the Guesser -- Sharing the critic's explanation with the guesser 
	  } 
	   } 
	   } 
	 & & \\ 
 

    \theutterance \stepcounter{utterance}  

    & & \multicolumn{4}{p{0.6\linewidth}}{\cellcolor[rgb]{0.9,0.9,0.9}{%
	\makecell[{{p{\linewidth}}}]{% 
	  \tt {\tiny [A$\langle$GM]}  
	 clue: "behold!"\\ \tt guess_agreement: no\\ \tt agreement_explanation: the word "locus" does not directly relate to the exclamation "behold!" in a clear or commonly understood way.\\ \tt \\ \tt Provide your response only in this format.\\ \tt explanation: details\\ \tt guess: word 
	  } 
	   } 
	   } 
	 & \\ 
 

    \theutterance \stepcounter{utterance}  

    & \multicolumn{4}{p{0.6\linewidth}}{\cellcolor[rgb]{0.8,1,0.9}{%
	\makecell[{{p{\linewidth}}}]{% 
	  \tt {\tiny [A$\rangle$GM]}  
	 explanation: The word "behold" is often associated with the word "lo," which is an archaic way of saying "look" or "see." The word "lo" is also part of the word "lofty," which can mean something impressive or awe-inspiring, similar to what one might behold.\\ \tt guess: lofty 
	  } 
	   } 
	   } 
	 & & \\ 
 

    \theutterance \stepcounter{utterance}  

    & & & \multicolumn{2}{p{0.3\linewidth}}{\cellcolor[rgb]{0.95,0.95,0.95}{%
	\makecell[{{p{\linewidth}}}]{% 
	  \tt {\tiny [GM$|$GM]}  
	 Change in player1's guess.\\ \tt guess_before_critic_opinion: locus\\ \tt                                 critic_agreement: no\\ \tt guess_after_critic_opinion: lofty\\ \tt                                                 Proceeding with guess validation 
	  } 
	   } 
	   } 
	 & & \\ 
 

    \theutterance \stepcounter{utterance}  

    & & & \multicolumn{2}{p{0.3\linewidth}}{\cellcolor[rgb]{0.95,0.95,0.95}{%
	\makecell[{{p{\linewidth}}}]{% 
	  \tt {\tiny [GM$|$GM]}  
	 attempts = 1\\ \tt target_word = voila\\ \tt target_word_clue = "behold!"\\ \tt guess = lofty\\ \tt guess_feedback = l<yellow> o<green> f<red> t<red> y<red> 
	  } 
	   } 
	   } 
	 & & \\ 
 

    \theutterance \stepcounter{utterance}  

    & & \multicolumn{4}{p{0.6\linewidth}}{\cellcolor[rgb]{0.9,0.9,0.9}{%
	\makecell[{{p{\linewidth}}}]{% 
	  \tt {\tiny [A$\langle$GM]}  
	 guess_feedback: l<yellow> o<green> f<red> t<red> y<red>\\ \tt \\ \tt Provide your response only in this format.\\ \tt explanation: details\\ \tt guess: word 
	  } 
	   } 
	   } 
	 & \\ 
 

    \theutterance \stepcounter{utterance}  

    & \multicolumn{4}{p{0.6\linewidth}}{\cellcolor[rgb]{0.8,1,0.9}{%
	\makecell[{{p{\linewidth}}}]{% 
	  \tt {\tiny [A$\rangle$GM]}  
	 explanation: The letter "o" is in the correct position, and "l" is in the word but not in the first position. Considering these constraints, the word "color" fits as it contains both "o" and "l" in different positions.\\ \tt guess: color 
	  } 
	   } 
	   } 
	 & & \\ 
 

    \theutterance \stepcounter{utterance}  

    & & \multicolumn{4}{p{0.6\linewidth}}{\cellcolor[rgb]{0.9,0.9,0.9}{%
	\makecell[{{p{\linewidth}}}]{% 
	  \tt {\tiny [GM$\rangle$B]}  
	 clue: "behold!"\\ \tt explanation: The letter "o" is in the correct position, and "l" is in the word but not in the first position. Considering these constraints, the word "color" fits as it contains both "o" and "l" in different positions.\\ \tt guess: color\\ \tt \\ \tt Provide your response only in this format.\\ \tt explanation: details\\ \tt agreement: yes or no 
	  } 
	   } 
	   } 
	 & \\ 
 

    \theutterance \stepcounter{utterance}  

    & & & \multicolumn{4}{p{0.6\linewidth}}{\cellcolor[rgb]{1,0.85,0.72}{%
	\makecell[{{p{\linewidth}}}]{% 
	  \tt {\tiny [GM$\langle$B]}  
	 explanation: the word "color" does not have a direct connection to the exclamation "behold!" and the explanation does not clearly link the word to the clue.\\ \tt agreement: no 
	  } 
	   } 
	   } 
	  \\ 
 

    \theutterance \stepcounter{utterance}  

    & & & \multicolumn{2}{p{0.3\linewidth}}{\cellcolor[rgb]{0.95,0.95,0.95}{%
	\makecell[{{p{\linewidth}}}]{% 
	  \tt {\tiny [GM$|$GM]}  
	 Critic does not agree with the Guesser -- Sharing the critic's explanation with the guesser 
	  } 
	   } 
	   } 
	 & & \\ 
 

    \theutterance \stepcounter{utterance}  

    & & \multicolumn{4}{p{0.6\linewidth}}{\cellcolor[rgb]{0.9,0.9,0.9}{%
	\makecell[{{p{\linewidth}}}]{% 
	  \tt {\tiny [A$\langle$GM]}  
	 clue: "behold!"\\ \tt guess_agreement: no\\ \tt agreement_explanation: the word "color" does not have a direct connection to the exclamation "behold!" and the explanation does not clearly link the word to the clue.\\ \tt \\ \tt Provide your response only in this format.\\ \tt explanation: details\\ \tt guess: word 
	  } 
	   } 
	   } 
	 & \\ 
 

    \theutterance \stepcounter{utterance}  

    & \multicolumn{4}{p{0.6\linewidth}}{\cellcolor[rgb]{0.8,1,0.9}{%
	\makecell[{{p{\linewidth}}}]{% 
	  \tt {\tiny [A$\rangle$GM]}  
	 explanation: The word "behold" is often associated with the word "lo," which is an archaic way of saying "look" or "see." The word "lo" is also part of the word "glory," which can be something impressive or awe-inspiring, similar to what one might behold.\\ \tt guess: glory 
	  } 
	   } 
	   } 
	 & & \\ 
 

    \theutterance \stepcounter{utterance}  

    & & & \multicolumn{2}{p{0.3\linewidth}}{\cellcolor[rgb]{0.95,0.95,0.95}{%
	\makecell[{{p{\linewidth}}}]{% 
	  \tt {\tiny [GM$|$GM]}  
	 Change in player1's guess.\\ \tt guess_before_critic_opinion: color\\ \tt                                 critic_agreement: no\\ \tt guess_after_critic_opinion: glory\\ \tt                                                 Proceeding with guess validation 
	  } 
	   } 
	   } 
	 & & \\ 
 

    \theutterance \stepcounter{utterance}  

    & & & \multicolumn{2}{p{0.3\linewidth}}{\cellcolor[rgb]{0.95,0.95,0.95}{%
	\makecell[{{p{\linewidth}}}]{% 
	  \tt {\tiny [GM$|$GM]}  
	 attempts = 2\\ \tt target_word = voila\\ \tt target_word_clue = "behold!"\\ \tt guess = glory\\ \tt guess_feedback = g<red> l<yellow> o<yellow> r<red> y<red> 
	  } 
	   } 
	   } 
	 & & \\ 
 

    \theutterance \stepcounter{utterance}  

    & & \multicolumn{4}{p{0.6\linewidth}}{\cellcolor[rgb]{0.9,0.9,0.9}{%
	\makecell[{{p{\linewidth}}}]{% 
	  \tt {\tiny [A$\langle$GM]}  
	 guess_feedback: g<red> l<yellow> o<yellow> r<red> y<red>\\ \tt \\ \tt Provide your response only in this format.\\ \tt explanation: details\\ \tt guess: word 
	  } 
	   } 
	   } 
	 & \\ 
 

    \theutterance \stepcounter{utterance}  

    & \multicolumn{4}{p{0.6\linewidth}}{\cellcolor[rgb]{0.8,1,0.9}{%
	\makecell[{{p{\linewidth}}}]{% 
	  \tt {\tiny [A$\rangle$GM]}  
	 explanation: The letters "l" and "o" are in the word but not in their current positions. Considering these constraints, the word "vowel" fits as it contains both "l" and "o" in different positions.\\ \tt guess: vowel 
	  } 
	   } 
	   } 
	 & & \\ 
 

    \theutterance \stepcounter{utterance}  

    & & \multicolumn{4}{p{0.6\linewidth}}{\cellcolor[rgb]{0.9,0.9,0.9}{%
	\makecell[{{p{\linewidth}}}]{% 
	  \tt {\tiny [GM$\rangle$B]}  
	 clue: "behold!"\\ \tt explanation: The letters "l" and "o" are in the word but not in their current positions. Considering these constraints, the word "vowel" fits as it contains both "l" and "o" in different positions.\\ \tt guess: vowel\\ \tt \\ \tt Provide your response only in this format.\\ \tt explanation: details\\ \tt agreement: yes or no 
	  } 
	   } 
	   } 
	 & \\ 
 

    \theutterance \stepcounter{utterance}  

    & & & \multicolumn{4}{p{0.6\linewidth}}{\cellcolor[rgb]{1,0.85,0.72}{%
	\makecell[{{p{\linewidth}}}]{% 
	  \tt {\tiny [GM$\langle$B]}  
	 explanation: the word "vowel" does not have a direct connection to the exclamation "behold!" and the explanation does not clearly link the word to the clue.\\ \tt agreement: no 
	  } 
	   } 
	   } 
	  \\ 
 

    \theutterance \stepcounter{utterance}  

    & & & \multicolumn{2}{p{0.3\linewidth}}{\cellcolor[rgb]{0.95,0.95,0.95}{%
	\makecell[{{p{\linewidth}}}]{% 
	  \tt {\tiny [GM$|$GM]}  
	 Critic does not agree with the Guesser -- Sharing the critic's explanation with the guesser 
	  } 
	   } 
	   } 
	 & & \\ 
 

    \theutterance \stepcounter{utterance}  

    & & \multicolumn{4}{p{0.6\linewidth}}{\cellcolor[rgb]{0.9,0.9,0.9}{%
	\makecell[{{p{\linewidth}}}]{% 
	  \tt {\tiny [A$\langle$GM]}  
	 clue: "behold!"\\ \tt guess_agreement: no\\ \tt agreement_explanation: the word "vowel" does not have a direct connection to the exclamation "behold!" and the explanation does not clearly link the word to the clue.\\ \tt \\ \tt Provide your response only in this format.\\ \tt explanation: details\\ \tt guess: word 
	  } 
	   } 
	   } 
	 & \\ 
 

    \theutterance \stepcounter{utterance}  

    & \multicolumn{4}{p{0.6\linewidth}}{\cellcolor[rgb]{0.8,1,0.9}{%
	\makecell[{{p{\linewidth}}}]{% 
	  \tt {\tiny [A$\rangle$GM]}  
	 explanation: The word "behold" is often associated with the word "lo," which is an archaic way of saying "look" or "see." The word "lo" is also part of the word "aloft," which can mean something elevated or in the air, similar to something one might behold.\\ \tt guess: aloft 
	  } 
	   } 
	   } 
	 & & \\ 
 

    \theutterance \stepcounter{utterance}  

    & & & \multicolumn{2}{p{0.3\linewidth}}{\cellcolor[rgb]{0.95,0.95,0.95}{%
	\makecell[{{p{\linewidth}}}]{% 
	  \tt {\tiny [GM$|$GM]}  
	 Change in player1's guess.\\ \tt guess_before_critic_opinion: vowel\\ \tt                                 critic_agreement: no\\ \tt guess_after_critic_opinion: aloft\\ \tt                                                 Proceeding with guess validation 
	  } 
	   } 
	   } 
	 & & \\ 
 

    \theutterance \stepcounter{utterance}  

    & & & \multicolumn{2}{p{0.3\linewidth}}{\cellcolor[rgb]{0.95,0.95,0.95}{%
	\makecell[{{p{\linewidth}}}]{% 
	  \tt {\tiny [GM$|$GM]}  
	 attempts = 3\\ \tt target_word = voila\\ \tt target_word_clue = "behold!"\\ \tt guess = aloft\\ \tt guess_feedback = a<yellow> l<yellow> o<yellow> f<red> t<red> 
	  } 
	   } 
	   } 
	 & & \\ 
 

    \theutterance \stepcounter{utterance}  

    & & \multicolumn{4}{p{0.6\linewidth}}{\cellcolor[rgb]{0.9,0.9,0.9}{%
	\makecell[{{p{\linewidth}}}]{% 
	  \tt {\tiny [A$\langle$GM]}  
	 guess_feedback: a<yellow> l<yellow> o<yellow> f<red> t<red>\\ \tt \\ \tt Provide your response only in this format.\\ \tt explanation: details\\ \tt guess: word 
	  } 
	   } 
	   } 
	 & \\ 
 

    \theutterance \stepcounter{utterance}  

    & \multicolumn{4}{p{0.6\linewidth}}{\cellcolor[rgb]{0.8,1,0.9}{%
	\makecell[{{p{\linewidth}}}]{% 
	  \tt {\tiny [A$\rangle$GM]}  
	 explanation: The letters "a," "l," and "o" are in the word but not in their current positions. Considering these constraints, the word "allow" fits as it contains all three letters in different positions.\\ \tt guess: allow 
	  } 
	   } 
	   } 
	 & & \\ 
 

    \theutterance \stepcounter{utterance}  

    & & \multicolumn{4}{p{0.6\linewidth}}{\cellcolor[rgb]{0.9,0.9,0.9}{%
	\makecell[{{p{\linewidth}}}]{% 
	  \tt {\tiny [GM$\rangle$B]}  
	 clue: "behold!"\\ \tt explanation: The letters "a," "l," and "o" are in the word but not in their current positions. Considering these constraints, the word "allow" fits as it contains all three letters in different positions.\\ \tt guess: allow\\ \tt \\ \tt Provide your response only in this format.\\ \tt explanation: details\\ \tt agreement: yes or no 
	  } 
	   } 
	   } 
	 & \\ 
 

    \theutterance \stepcounter{utterance}  

    & & & \multicolumn{4}{p{0.6\linewidth}}{\cellcolor[rgb]{1,0.85,0.72}{%
	\makecell[{{p{\linewidth}}}]{% 
	  \tt {\tiny [GM$\langle$B]}  
	 explanation: the word "allow" does not have a direct connection to the exclamation "behold!" and the explanation does not clearly link the word to the clue.\\ \tt agreement: no 
	  } 
	   } 
	   } 
	  \\ 
 

    \theutterance \stepcounter{utterance}  

    & & & \multicolumn{2}{p{0.3\linewidth}}{\cellcolor[rgb]{0.95,0.95,0.95}{%
	\makecell[{{p{\linewidth}}}]{% 
	  \tt {\tiny [GM$|$GM]}  
	 Critic does not agree with the Guesser -- Sharing the critic's explanation with the guesser 
	  } 
	   } 
	   } 
	 & & \\ 
 

    \theutterance \stepcounter{utterance}  

    & & \multicolumn{4}{p{0.6\linewidth}}{\cellcolor[rgb]{0.9,0.9,0.9}{%
	\makecell[{{p{\linewidth}}}]{% 
	  \tt {\tiny [A$\langle$GM]}  
	 clue: "behold!"\\ \tt guess_agreement: no\\ \tt agreement_explanation: the word "allow" does not have a direct connection to the exclamation "behold!" and the explanation does not clearly link the word to the clue.\\ \tt \\ \tt Provide your response only in this format.\\ \tt explanation: details\\ \tt guess: word 
	  } 
	   } 
	   } 
	 & \\ 
 

    \theutterance \stepcounter{utterance}  

    & \multicolumn{4}{p{0.6\linewidth}}{\cellcolor[rgb]{0.8,1,0.9}{%
	\makecell[{{p{\linewidth}}}]{% 
	  \tt {\tiny [A$\rangle$GM]}  
	 explanation: The word "behold" is often associated with the word "lo," which is an archaic way of saying "look" or "see." The word "lo" is also part of the word "royal," which can be something grand or majestic, similar to what one might behold.\\ \tt guess: royal 
	  } 
	   } 
	   } 
	 & & \\ 
 

    \theutterance \stepcounter{utterance}  

    & & & \multicolumn{2}{p{0.3\linewidth}}{\cellcolor[rgb]{0.95,0.95,0.95}{%
	\makecell[{{p{\linewidth}}}]{% 
	  \tt {\tiny [GM$|$GM]}  
	 Change in player1's guess.\\ \tt guess_before_critic_opinion: allow\\ \tt                                 critic_agreement: no\\ \tt guess_after_critic_opinion: royal\\ \tt                                                 Proceeding with guess validation 
	  } 
	   } 
	   } 
	 & & \\ 
 

    \theutterance \stepcounter{utterance}  

    & & & \multicolumn{2}{p{0.3\linewidth}}{\cellcolor[rgb]{0.95,0.95,0.95}{%
	\makecell[{{p{\linewidth}}}]{% 
	  \tt {\tiny [GM$|$GM]}  
	 attempts = 4\\ \tt target_word = voila\\ \tt target_word_clue = "behold!"\\ \tt guess = royal\\ \tt guess_feedback = r<red> o<green> y<red> a<yellow> l<yellow> 
	  } 
	   } 
	   } 
	 & & \\ 
 

    \theutterance \stepcounter{utterance}  

    & & \multicolumn{4}{p{0.6\linewidth}}{\cellcolor[rgb]{0.9,0.9,0.9}{%
	\makecell[{{p{\linewidth}}}]{% 
	  \tt {\tiny [A$\langle$GM]}  
	 guess_feedback: r<red> o<green> y<red> a<yellow> l<yellow>\\ \tt \\ \tt Provide your response only in this format.\\ \tt explanation: details\\ \tt guess: word 
	  } 
	   } 
	   } 
	 & \\ 
 

    \theutterance \stepcounter{utterance}  

    & \multicolumn{4}{p{0.6\linewidth}}{\cellcolor[rgb]{0.8,1,0.9}{%
	\makecell[{{p{\linewidth}}}]{% 
	  \tt {\tiny [A$\rangle$GM]}  
	 explanation: The letters "o" is in the correct position, while "a" and "l" are in the word but not in their current positions. Considering these constraints, the word "local" fits as it contains all three letters in different positions.\\ \tt guess: local 
	  } 
	   } 
	   } 
	 & & \\ 
 

    \theutterance \stepcounter{utterance}  

    & & \multicolumn{4}{p{0.6\linewidth}}{\cellcolor[rgb]{0.9,0.9,0.9}{%
	\makecell[{{p{\linewidth}}}]{% 
	  \tt {\tiny [GM$\rangle$B]}  
	 clue: "behold!"\\ \tt explanation: The letters "o" is in the correct position, while "a" and "l" are in the word but not in their current positions. Considering these constraints, the word "local" fits as it contains all three letters in different positions.\\ \tt guess: local\\ \tt \\ \tt Provide your response only in this format.\\ \tt explanation: details\\ \tt agreement: yes or no 
	  } 
	   } 
	   } 
	 & \\ 
 

    \theutterance \stepcounter{utterance}  

    & & & \multicolumn{4}{p{0.6\linewidth}}{\cellcolor[rgb]{1,0.85,0.72}{%
	\makecell[{{p{\linewidth}}}]{% 
	  \tt {\tiny [GM$\langle$B]}  
	 explanation: the word "local" does not have a direct connection to the exclamation "behold!" and the explanation does not clearly link the word to the clue.\\ \tt agreement: no 
	  } 
	   } 
	   } 
	  \\ 
 

    \theutterance \stepcounter{utterance}  

    & & & \multicolumn{2}{p{0.3\linewidth}}{\cellcolor[rgb]{0.95,0.95,0.95}{%
	\makecell[{{p{\linewidth}}}]{% 
	  \tt {\tiny [GM$|$GM]}  
	 Critic does not agree with the Guesser -- Sharing the critic's explanation with the guesser 
	  } 
	   } 
	   } 
	 & & \\ 
 

    \theutterance \stepcounter{utterance}  

    & & \multicolumn{4}{p{0.6\linewidth}}{\cellcolor[rgb]{0.9,0.9,0.9}{%
	\makecell[{{p{\linewidth}}}]{% 
	  \tt {\tiny [A$\langle$GM]}  
	 clue: "behold!"\\ \tt guess_agreement: no\\ \tt agreement_explanation: the word "local" does not have a direct connection to the exclamation "behold!" and the explanation does not clearly link the word to the clue.\\ \tt \\ \tt Provide your response only in this format.\\ \tt explanation: details\\ \tt guess: word 
	  } 
	   } 
	   } 
	 & \\ 
 

    \theutterance \stepcounter{utterance}  

    & \multicolumn{4}{p{0.6\linewidth}}{\cellcolor[rgb]{0.8,1,0.9}{%
	\makecell[{{p{\linewidth}}}]{% 
	  \tt {\tiny [A$\rangle$GM]}  
	 explanation: The word "behold" is often associated with the word "lo," which is an archaic way of saying "look" or "see." The word "lo" is also part of the word "ovals," which can be something one might behold in terms of shape or design.\\ \tt guess: ovals 
	  } 
	   } 
	   } 
	 & & \\ 
 

    \theutterance \stepcounter{utterance}  

    & & & \multicolumn{2}{p{0.3\linewidth}}{\cellcolor[rgb]{0.95,0.95,0.95}{%
	\makecell[{{p{\linewidth}}}]{% 
	  \tt {\tiny [GM$|$GM]}  
	 Change in player1's guess.\\ \tt guess_before_critic_opinion: local\\ \tt                                 critic_agreement: no\\ \tt guess_after_critic_opinion: ovals\\ \tt                                                 Proceeding with guess validation 
	  } 
	   } 
	   } 
	 & & \\ 
 

    \theutterance \stepcounter{utterance}  

    & & & \multicolumn{2}{p{0.3\linewidth}}{\cellcolor[rgb]{0.95,0.95,0.95}{%
	\makecell[{{p{\linewidth}}}]{% 
	  \tt {\tiny [GM$|$GM]}  
	 attempts = 5\\ \tt target_word = voila\\ \tt target_word_clue = "behold!"\\ \tt guess = ovals\\ \tt guess_feedback = o<yellow> v<yellow> a<yellow> l<green> s<red> 
	  } 
	   } 
	   } 
	 & & \\ 
 

    \theutterance \stepcounter{utterance}  

    & & \multicolumn{4}{p{0.6\linewidth}}{\cellcolor[rgb]{0.9,0.9,0.9}{%
	\makecell[{{p{\linewidth}}}]{% 
	  \tt {\tiny [A$\langle$GM]}  
	 guess_feedback: o<yellow> v<yellow> a<yellow> l<green> s<red>\\ \tt \\ \tt Provide your response only in this format.\\ \tt explanation: details\\ \tt guess: word 
	  } 
	   } 
	   } 
	 & \\ 
 

    \theutterance \stepcounter{utterance}  

    & \multicolumn{4}{p{0.6\linewidth}}{\cellcolor[rgb]{0.8,1,0.9}{%
	\makecell[{{p{\linewidth}}}]{% 
	  \tt {\tiny [A$\rangle$GM]}  
	 explanation: The letter "l" is in the correct position, while "o," "v," and "a" are in the word but not in their current positions. Considering these constraints, the word "valor" fits as it contains all the required letters in different positions.\\ \tt guess: valor 
	  } 
	   } 
	   } 
	 & & \\ 
 

    \theutterance \stepcounter{utterance}  

    & & \multicolumn{4}{p{0.6\linewidth}}{\cellcolor[rgb]{0.9,0.9,0.9}{%
	\makecell[{{p{\linewidth}}}]{% 
	  \tt {\tiny [GM$\rangle$B]}  
	 clue: "behold!"\\ \tt explanation: The letter "l" is in the correct position, while "o," "v," and "a" are in the word but not in their current positions. Considering these constraints, the word "valor" fits as it contains all the required letters in different positions.\\ \tt guess: valor\\ \tt \\ \tt Provide your response only in this format.\\ \tt explanation: details\\ \tt agreement: yes or no 
	  } 
	   } 
	   } 
	 & \\ 
 

    \theutterance \stepcounter{utterance}  

    & & & \multicolumn{4}{p{0.6\linewidth}}{\cellcolor[rgb]{1,0.85,0.72}{%
	\makecell[{{p{\linewidth}}}]{% 
	  \tt {\tiny [GM$\langle$B]}  
	 explanation: the word "valor" does not have a direct connection to the exclamation "behold!" and the explanation does not clearly link the word to the clue.\\ \tt agreement: no 
	  } 
	   } 
	   } 
	  \\ 
 

    \theutterance \stepcounter{utterance}  

    & & & \multicolumn{2}{p{0.3\linewidth}}{\cellcolor[rgb]{0.95,0.95,0.95}{%
	\makecell[{{p{\linewidth}}}]{% 
	  \tt {\tiny [GM$|$GM]}  
	 Critic does not agree with the Guesser -- Sharing the critic's explanation with the guesser 
	  } 
	   } 
	   } 
	 & & \\ 
 

    \theutterance \stepcounter{utterance}  

    & & \multicolumn{4}{p{0.6\linewidth}}{\cellcolor[rgb]{0.9,0.9,0.9}{%
	\makecell[{{p{\linewidth}}}]{% 
	  \tt {\tiny [A$\langle$GM]}  
	 clue: "behold!"\\ \tt guess_agreement: no\\ \tt agreement_explanation: the word "valor" does not have a direct connection to the exclamation "behold!" and the explanation does not clearly link the word to the clue.\\ \tt \\ \tt Provide your response only in this format.\\ \tt explanation: details\\ \tt guess: word 
	  } 
	   } 
	   } 
	 & \\ 
 

    \theutterance \stepcounter{utterance}  

    & \multicolumn{4}{p{0.6\linewidth}}{\cellcolor[rgb]{0.8,1,0.9}{%
	\makecell[{{p{\linewidth}}}]{% 
	  \tt {\tiny [A$\rangle$GM]}  
	 explanation: The word "behold" is often associated with the word "lo," which is an archaic way of saying "look" or "see." The word "lo" is also part of the word "vocal," which can be something one might behold in terms of sound or expression.\\ \tt guess: vocal 
	  } 
	   } 
	   } 
	 & & \\ 
 

    \theutterance \stepcounter{utterance}  

    & & & \multicolumn{2}{p{0.3\linewidth}}{\cellcolor[rgb]{0.95,0.95,0.95}{%
	\makecell[{{p{\linewidth}}}]{% 
	  \tt {\tiny [GM$|$GM]}  
	 Change in player1's guess.\\ \tt guess_before_critic_opinion: valor\\ \tt                                 critic_agreement: no\\ \tt guess_after_critic_opinion: vocal\\ \tt                                                 Proceeding with guess validation 
	  } 
	   } 
	   } 
	 & & \\ 
 

    \theutterance \stepcounter{utterance}  

    & & & \multicolumn{2}{p{0.3\linewidth}}{\cellcolor[rgb]{0.95,0.95,0.95}{%
	\makecell[{{p{\linewidth}}}]{% 
	  \tt {\tiny [GM$|$GM]}  
	 attempts = 6\\ \tt target_word = voila\\ \tt target_word_clue = "behold!"\\ \tt guess = vocal\\ \tt guess_feedback = v<green> o<green> c<red> a<yellow> l<yellow> 
	  } 
	   } 
	   } 
	 & & \\ 
 

    \theutterance \stepcounter{utterance}  

    & & & \multicolumn{2}{p{0.3\linewidth}}{\cellcolor[rgb]{0.95,0.95,0.95}{%
	\makecell[{{p{\linewidth}}}]{% 
	  \tt {\tiny [GM$|$GM]}  
	 game_result = LOSS 
	  } 
	   } 
	   } 
	 & & \\ 
 

\end{supertabular}
}

\end{document}


\documentclass{article}
\usepackage{colortbl}
\usepackage{makecell}
\usepackage{multirow}
\usepackage{supertabular}

\begin{document}

\newcounter{utterance}

\twocolumn

{ \footnotesize  \setcounter{utterance}{1}
\setlength{\tabcolsep}{0pt}
\begin{supertabular}{c@{$\;$}|p{.15\linewidth}@{}p{.15\linewidth}p{.15\linewidth}p{.15\linewidth}p{.15\linewidth}p{.15\linewidth}}

    \# & $\;$A & \multicolumn{4}{c}{Game Master} & $\;\:$B\\
    \hline 

    \theutterance \stepcounter{utterance}  

    & & \multicolumn{4}{p{0.6\linewidth}}{\cellcolor[rgb]{0.9,0.9,0.9}{%
	\makecell[{{p{\linewidth}}}]{% 
	  \tt {\tiny [A$\langle$GM]}  
	 You are a language wizard who likes to guess words by using the given rules.\\ \tt \\ \tt Welcome to Wordle! You have six attempts to guess the target word, a valid English word of five lowercase letters (a-z). Please use the tags "explanation:" and "guess:" to provide a concise explanation for each guess.\\ \tt \\ \tt To help you make an informed guess, you will receive a clue for the word, such as\\ \tt clue: snowy white.\\ \tt \\ \tt Here is an example guess based on the clue:\\ \tt explanation: In the fairy tail Snow White, the girl is killed because she eats a poisoned apple. And the word apple has 5 letters.\\ \tt guess: apple\\ \tt \\ \tt After each guess, your answer will be validated, and you will receive feedback indicating which letters are correct (green), which letters are correct but in the wrong position (yellow), and which letters are incorrect (red). This feedback can be useful in determining which letters to include or exclude in your next guess.\\ \tt \\ \tt For example, the feedback for "apple" might be:\\ \tt guess_feedback: a<yellow> p<yellow> p<green> l<yellow> e<red>\\ \tt \\ \tt The explanation should contain details about how both the guess_feedback and clue work together in tandem to formulate a new guess.\\ \tt \\ \tt Do not generate any other text.\\ \tt \\ \tt Let's begin with your first guess.\\ \tt \\ \tt clue: rugby get-together\\ \tt  
	  } 
	   } 
	   } 
	 & \\ 
 

    \theutterance \stepcounter{utterance}  

    & \multicolumn{4}{p{0.6\linewidth}}{\cellcolor[rgb]{0.8,1,0.9}{%
	\makecell[{{p{\linewidth}}}]{% 
	  \tt {\tiny [A$\rangle$GM]}  
	 explanation: Based on the clue "rugby get-together", I'm thinking of a word that's related to a social gathering of rugby players or fans. A common term for such an event is a "scrum", but that's not a 5-letter word. However, I recall that rugby players often gather for a "scrum" in a larger event called a "match", but that's not it either. Then I thought of a more casual gathering, a party or a "scrum" related event called a "scrum" braai or "scrum" fest in some countries, but one word that is associated with such gatherings and is a valid 5-letter word is "scrum" without the "s", a "crumb" is not it, but a "scrum" without the "s" is a "crum" without the "b" is a "crum" without the "c" is a...  "scrum" without the "s" and the "c" and the "u" is a "r-m" and a braai or fest is similar to a "r-m" and a "scrum" braai or "scrum" fest without the "s" and the "c" and the "u" is similar to a "rugby" braai or "rugby" fest without the "s" and the "u" and the "g" and the "b" and the "y" is a...  "rugby" get-together without the "s" and the "u" and the "g" and the "b" and the "y" is a "r-m" and a valid 5-letter word for a "rugby" get-together without the "s" and the "u" and the "g" and the "b" and the "y" is a "r-m" and a valid 5-letter word for a "rugby" get-together is a "ruck".\\ \tt guess: rucks 
	  } 
	   } 
	   } 
	 & & \\ 
 

    \theutterance \stepcounter{utterance}  

    & & & \multicolumn{2}{p{0.3\linewidth}}{\cellcolor[rgb]{0.95,0.95,0.95}{%
	\makecell[{{p{\linewidth}}}]{% 
	  \tt {\tiny [GM$|$GM]}  
	 attempts = 1\\ \tt target_word = scrum\\ \tt target_word_clue = rugby get-together\\ \tt guess = rucks\\ \tt guess_feedback = r<yellow> u<yellow> c<yellow> k<red> s<yellow> 
	  } 
	   } 
	   } 
	 & & \\ 
 

    \theutterance \stepcounter{utterance}  

    & & \multicolumn{4}{p{0.6\linewidth}}{\cellcolor[rgb]{0.9,0.9,0.9}{%
	\makecell[{{p{\linewidth}}}]{% 
	  \tt {\tiny [A$\langle$GM]}  
	 guess_feedback: r<yellow> u<yellow> c<yellow> k<red> s<yellow>\\ \tt \\ \tt Provide your response only in this format.\\ \tt explanation: details\\ \tt guess: word 
	  } 
	   } 
	   } 
	 & \\ 
 

    \theutterance \stepcounter{utterance}  

    & \multicolumn{4}{p{0.6\linewidth}}{\cellcolor[rgb]{0.8,1,0.9}{%
	\makecell[{{p{\linewidth}}}]{% 
	  \tt {\tiny [A$\rangle$GM]}  
	 explanation: Based on the clue "rugby get-together" and the feedback from the previous guess, I'm thinking of a new 5-letter word. The letters "r", "u", "c", and "s" were marked as yellow, indicating that they are present in the target word but in the wrong position. The letter "k" was marked as red, indicating that it is not present in the target word. Given this information, I'm trying to find a new word that still relates to a rugby gathering. One possible word that comes to mind is "curbs", but another word that seems to fit the given letters and relates to a rugby gathering is more plausible. However since "r", "u", "c" are yellow and that makes me think of the normal word for rugby and adding to the fact "s" being there. The first guess would then be rugby's first choice term.\\ \tt guess: crush 
	  } 
	   } 
	   } 
	 & & \\ 
 

    \theutterance \stepcounter{utterance}  

    & & & \multicolumn{2}{p{0.3\linewidth}}{\cellcolor[rgb]{0.95,0.95,0.95}{%
	\makecell[{{p{\linewidth}}}]{% 
	  \tt {\tiny [GM$|$GM]}  
	 attempts = 2\\ \tt target_word = scrum\\ \tt target_word_clue = rugby get-together\\ \tt guess = crush\\ \tt guess_feedback = c<yellow> r<yellow> u<yellow> s<yellow> h<red> 
	  } 
	   } 
	   } 
	 & & \\ 
 

    \theutterance \stepcounter{utterance}  

    & & \multicolumn{4}{p{0.6\linewidth}}{\cellcolor[rgb]{0.9,0.9,0.9}{%
	\makecell[{{p{\linewidth}}}]{% 
	  \tt {\tiny [A$\langle$GM]}  
	 guess_feedback: c<yellow> r<yellow> u<yellow> s<yellow> h<red>\\ \tt \\ \tt Provide your response only in this format.\\ \tt explanation: details\\ \tt guess: word 
	  } 
	   } 
	   } 
	 & \\ 
 

    \theutterance \stepcounter{utterance}  

    & \multicolumn{4}{p{0.6\linewidth}}{\cellcolor[rgb]{0.8,1,0.9}{%
	\makecell[{{p{\linewidth}}}]{% 
	  \tt {\tiny [A$\rangle$GM]}  
	 explanation: With the new feedback, I see that the letters "c", "r", "u", and "s" are still marked as yellow, indicating that they are present in the target word but in the wrong position. The letter "h" was marked as red, indicating that it is not present in the target word. Given this information, I'm trying to find a new word that still relates to a rugby gathering and uses the letters "c", "r", "u", and "s" in a different way. One possible word that comes to mind is a common term used in rugby, and since all the letters are yellow, I'm going to try rearranging them to form a new word. A possible anagram of these letters is a word that is closely related to the sport.\\ \tt guess: scrum 
	  } 
	   } 
	   } 
	 & & \\ 
 

    \theutterance \stepcounter{utterance}  

    & & & \multicolumn{2}{p{0.3\linewidth}}{\cellcolor[rgb]{0.95,0.95,0.95}{%
	\makecell[{{p{\linewidth}}}]{% 
	  \tt {\tiny [GM$|$GM]}  
	 attempts = 3\\ \tt target_word = scrum\\ \tt target_word_clue = rugby get-together\\ \tt guess = scrum\\ \tt guess_feedback = s<green> c<green> r<green> u<green> m<green> 
	  } 
	   } 
	   } 
	 & & \\ 
 

    \theutterance \stepcounter{utterance}  

    & & & \multicolumn{2}{p{0.3\linewidth}}{\cellcolor[rgb]{0.95,0.95,0.95}{%
	\makecell[{{p{\linewidth}}}]{% 
	  \tt {\tiny [GM$|$GM]}  
	 game_result = WIN 
	  } 
	   } 
	   } 
	 & & \\ 
 

\end{supertabular}
}

\end{document}

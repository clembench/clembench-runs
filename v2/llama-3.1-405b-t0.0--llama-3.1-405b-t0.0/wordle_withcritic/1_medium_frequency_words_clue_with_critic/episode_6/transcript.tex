
\documentclass{article}
\usepackage{colortbl}
\usepackage{makecell}
\usepackage{multirow}
\usepackage{supertabular}

\begin{document}

\newcounter{utterance}

\twocolumn

{ \footnotesize  \setcounter{utterance}{1}
\setlength{\tabcolsep}{0pt}
\begin{supertabular}{c@{$\;$}|p{.15\linewidth}@{}p{.15\linewidth}p{.15\linewidth}p{.15\linewidth}p{.15\linewidth}p{.15\linewidth}}

    \# & $\;$A & \multicolumn{4}{c}{Game Master} & $\;\:$B\\
    \hline 

    \theutterance \stepcounter{utterance}  

    & & \multicolumn{4}{p{0.6\linewidth}}{\cellcolor[rgb]{0.9,0.9,0.9}{%
	\makecell[{{p{\linewidth}}}]{% 
	  \tt {\tiny [A$\langle$GM]}  
	 You are a language wizard who likes to guess words by using the given rules.\\ \tt \\ \tt Welcome to Wordle! You have six attempts to guess the target word, a valid English word of five lowercase letters (a-z). Please use the tags "explanation:" and "guess:" to provide a concise explanation for each guess.\\ \tt \\ \tt To help you make an informed guess, you will receive a clue for the word, such as\\ \tt clue: "snowy white"\\ \tt \\ \tt Here is an example guess based on the clue:\\ \tt explanation: In the fairy tail Snow White, the girl is killed because she eats a poisoned apple. And the word apple has 5 letters.\\ \tt guess: apple\\ \tt \\ \tt I will then indicate whether I agree or disagree with your guess and provide rationale, but agreeing with a guess does not confirm its correctness. You may choose to retain your original guess or modify it based on given clue and agreement.\\ \tt \\ \tt After each guess, your answer will be validated, and you will receive feedback indicating which letters are correct (green), which letters are correct but in the wrong position (yellow), and which letters are incorrect (red). This feedback can be useful in determining which letters to include or exclude in your next guess.\\ \tt \\ \tt For example, the feedback for "apple" might be:\\ \tt guess_feedback: a<yellow> p<yellow> p<green> l<yellow> e<red>\\ \tt \\ \tt Here are some rules to keep in mind:\\ \tt 1. Please do not generate any guess feedback or clue on your own\\ \tt 2. Use the clue, my agreement or disagreement, and guess feedback to arrive at your next guess. This should be reflected in your explanation as well.\\ \tt \\ \tt Do not generate any other text.\\ \tt \\ \tt Let's begin with your first guess.\\ \tt \\ \tt clue: more than angry\\ \tt  
	  } 
	   } 
	   } 
	 & \\ 
 

    \theutterance \stepcounter{utterance}  

    & \multicolumn{4}{p{0.6\linewidth}}{\cellcolor[rgb]{0.8,1,0.9}{%
	\makecell[{{p{\linewidth}}}]{% 
	  \tt {\tiny [A$\rangle$GM]}  
	 explanation: The word "irate" means being extremely angry, and "more than angry" could imply a stronger emotion. A word that is similar to "irate" but has 5 letters is "irate" without the last letter, which becomes "irate".\\ \tt guess: irate 
	  } 
	   } 
	   } 
	 & & \\ 
 

    \theutterance \stepcounter{utterance}  

    & & \multicolumn{4}{p{0.6\linewidth}}{\cellcolor[rgb]{0.9,0.9,0.9}{%
	\makecell[{{p{\linewidth}}}]{% 
	  \tt {\tiny [GM$\rangle$B]}  
	 I need your assistance with a word game in which we need to find a 5-letter word using a clue, a guess and an explanation for the guess.\\ \tt \\ \tt For example, if the clue is "snow white" and the guess is "apple", you will get the information like this:\\ \tt clue: snow white\\ \tt explanation: the fairy tale Snow White involves the girl eating a poisoned apple, which has 5 letters.\\ \tt guess: apple\\ \tt \\ \tt Your task is to either agree or disagree with my guess based on the given clue.\\ \tt \\ \tt If you disagree, I will make another guess, and if you agree, I will proceed in the game using the same guess.\\ \tt \\ \tt Initially, guess feedback is not available, but as the game proceeds, it becomes available. The guess feedback is not for current guess, but rather an indication of what letters can be present in the current guess. A letter highlighted in green indicates that it is a correct letter in the correct position, while yellow indicates that it is a correct letter in the wrong position, and red indicates a wrong letter. At this point, you should use all the provided information, including the clue, guess, explanation, and guess feedback, to decide whether the given guess matches the clue, and also stick to guess feedback or not.\\ \tt \\ \tt Please respond in lowercase letters and stick to this format:\\ \tt explanation: your reason for doing so\\ \tt agreement: yes or no\\ \tt \\ \tt Please note that you must not guess any word based on the riddle clue and stick to the given format while responding.\\ \tt \\ \tt Do not generate any other text.\\ \tt \\ \tt Let's begin!\\ \tt \\ \tt clue: more than angry\\ \tt explanation: The word "irate" means being extremely angry, and "more than angry" could imply a stronger emotion. A word that is similar to "irate" but has 5 letters is "irate" without the last letter, which becomes "irate".\\ \tt guess: irate 
	  } 
	   } 
	   } 
	 & \\ 
 

    \theutterance \stepcounter{utterance}  

    & & & \multicolumn{4}{p{0.6\linewidth}}{\cellcolor[rgb]{1,0.85,0.72}{%
	\makecell[{{p{\linewidth}}}]{% 
	  \tt {\tiny [GM$\langle$B]}  
	 explanation: the explanation provided forms a weak link between the clue and the guess, and the guess does not exactly represent "more than angry" since removing a letter from "irate" doesn't necessarily mean an increase in anger\\ \tt agreement: no 
	  } 
	   } 
	   } 
	  \\ 
 

    \theutterance \stepcounter{utterance}  

    & & & \multicolumn{2}{p{0.3\linewidth}}{\cellcolor[rgb]{0.95,0.95,0.95}{%
	\makecell[{{p{\linewidth}}}]{% 
	  \tt {\tiny [GM$|$GM]}  
	 Critic does not agree with the Guesser -- Sharing the critic's explanation with the guesser 
	  } 
	   } 
	   } 
	 & & \\ 
 

    \theutterance \stepcounter{utterance}  

    & & \multicolumn{4}{p{0.6\linewidth}}{\cellcolor[rgb]{0.9,0.9,0.9}{%
	\makecell[{{p{\linewidth}}}]{% 
	  \tt {\tiny [A$\langle$GM]}  
	 clue: more than angry\\ \tt guess_agreement: no\\ \tt agreement_explanation: the explanation provided forms a weak link between the clue and the guess, and the guess does not exactly represent "more than angry" since removing a letter from "irate" doesn't necessarily mean an increase in anger\\ \tt \\ \tt Provide your response only in this format.\\ \tt explanation: details\\ \tt guess: word 
	  } 
	   } 
	   } 
	 & \\ 
 

    \theutterance \stepcounter{utterance}  

    & \multicolumn{4}{p{0.6\linewidth}}{\cellcolor[rgb]{0.8,1,0.9}{%
	\makecell[{{p{\linewidth}}}]{% 
	  \tt {\tiny [A$\rangle$GM]}  
	 explanation: Considering the clue "more than angry" and the disagreement with the previous guess, a new approach is to think of a word that intensifies the emotion of anger. A word that comes to mind is "irate" but with a stronger connotation, which leads to the word "irate" being replaced with a synonym "furious" and then finding a 5-letter word that is similar, which is "fired" but it doesn't seem to fit, then another 5-letter word "fumes" doesn't fit either but a word that is similar to furious is "fury" and adding an "i" and an "s" and other letters to make a 5-letter word, a possible fit could be "furis" but that is not a valid word, then "furies" is not a fit since it has 6 letters, then "furys" is not valid, but "fury" becomes "furie" but is not valid, but it leads to the 5-letter word "furie" without an "e" at the end and double "i" in "furies" becomes a double "y" and a drop of the "e" and an "s" which becomes the valid 5-letter word "fury" is not but "furie" without an "e" is not valid, but leads to a valid similar word.\\ \tt guess: furry 
	  } 
	   } 
	   } 
	 & & \\ 
 

    \theutterance \stepcounter{utterance}  

    & & & \multicolumn{2}{p{0.3\linewidth}}{\cellcolor[rgb]{0.95,0.95,0.95}{%
	\makecell[{{p{\linewidth}}}]{% 
	  \tt {\tiny [GM$|$GM]}  
	 Change in player1's guess.\\ \tt guess_before_critic_opinion: irate\\ \tt                                 critic_agreement: no\\ \tt guess_after_critic_opinion: furry\\ \tt                                                 Proceeding with guess validation 
	  } 
	   } 
	   } 
	 & & \\ 
 

    \theutterance \stepcounter{utterance}  

    & & & \multicolumn{2}{p{0.3\linewidth}}{\cellcolor[rgb]{0.95,0.95,0.95}{%
	\makecell[{{p{\linewidth}}}]{% 
	  \tt {\tiny [GM$|$GM]}  
	 attempts = 1\\ \tt target_word = livid\\ \tt target_word_clue = more than angry\\ \tt guess = furry\\ \tt guess_feedback = f<red> u<red> r<red> r<red> y<red> 
	  } 
	   } 
	   } 
	 & & \\ 
 

    \theutterance \stepcounter{utterance}  

    & & \multicolumn{4}{p{0.6\linewidth}}{\cellcolor[rgb]{0.9,0.9,0.9}{%
	\makecell[{{p{\linewidth}}}]{% 
	  \tt {\tiny [A$\langle$GM]}  
	 guess_feedback: f<red> u<red> r<red> r<red> y<red>\\ \tt \\ \tt Provide your response only in this format.\\ \tt explanation: details\\ \tt guess: word 
	  } 
	   } 
	   } 
	 & \\ 
 

    \theutterance \stepcounter{utterance}  

    & \multicolumn{4}{p{0.6\linewidth}}{\cellcolor[rgb]{0.8,1,0.9}{%
	\makecell[{{p{\linewidth}}}]{% 
	  \tt {\tiny [A$\rangle$GM]}  
	 explanation: The guess "furry" received all red feedback, indicating that none of the letters are correct. Considering the clue "more than angry" and the previous disagreement, a new approach is to think of a word that intensifies the emotion of anger. A word that comes to mind is a synonym for extreme anger, which is "wrath". Since "wrath" has 5 letters, it seems like a plausible guess.\\ \tt guess: wrath 
	  } 
	   } 
	   } 
	 & & \\ 
 

    \theutterance \stepcounter{utterance}  

    & & \multicolumn{4}{p{0.6\linewidth}}{\cellcolor[rgb]{0.9,0.9,0.9}{%
	\makecell[{{p{\linewidth}}}]{% 
	  \tt {\tiny [GM$\rangle$B]}  
	 clue: more than angry\\ \tt explanation: The guess "furry" received all red feedback, indicating that none of the letters are correct. Considering the clue "more than angry" and the previous disagreement, a new approach is to think of a word that intensifies the emotion of anger. A word that comes to mind is a synonym for extreme anger, which is "wrath". Since "wrath" has 5 letters, it seems like a plausible guess.\\ \tt guess: wrath\\ \tt \\ \tt Provide your response only in this format.\\ \tt explanation: details\\ \tt agreement: yes or no 
	  } 
	   } 
	   } 
	 & \\ 
 

    \theutterance \stepcounter{utterance}  

    & & & \multicolumn{4}{p{0.6\linewidth}}{\cellcolor[rgb]{1,0.85,0.72}{%
	\makecell[{{p{\linewidth}}}]{% 
	  \tt {\tiny [GM$\langle$B]}  
	 explanation: the explanation forms a strong link between the clue and the guess, as "wrath" indeed represents an intensified form of anger, and the 5-letter length matches the required word length\\ \tt agreement: yes 
	  } 
	   } 
	   } 
	  \\ 
 

    \theutterance \stepcounter{utterance}  

    & & & \multicolumn{2}{p{0.3\linewidth}}{\cellcolor[rgb]{0.95,0.95,0.95}{%
	\makecell[{{p{\linewidth}}}]{% 
	  \tt {\tiny [GM$|$GM]}  
	 Critic agrees with the Guesser -- Sharing the critic's explanation with the guesser 
	  } 
	   } 
	   } 
	 & & \\ 
 

    \theutterance \stepcounter{utterance}  

    & & \multicolumn{4}{p{0.6\linewidth}}{\cellcolor[rgb]{0.9,0.9,0.9}{%
	\makecell[{{p{\linewidth}}}]{% 
	  \tt {\tiny [A$\langle$GM]}  
	 clue: more than angry\\ \tt guess_agreement: yes\\ \tt agreement_explanation: the explanation forms a strong link between the clue and the guess, as "wrath" indeed represents an intensified form of anger, and the 5-letter length matches the required word length\\ \tt \\ \tt Provide your response only in this format.\\ \tt explanation: details\\ \tt guess: word 
	  } 
	   } 
	   } 
	 & \\ 
 

    \theutterance \stepcounter{utterance}  

    & \multicolumn{4}{p{0.6\linewidth}}{\cellcolor[rgb]{0.8,1,0.9}{%
	\makecell[{{p{\linewidth}}}]{% 
	  \tt {\tiny [A$\rangle$GM]}  
	 explanation: Given the agreement with the previous guess "wrath" and the clue "more than angry", it seems reasonable to stick with the guess, as the connection between the clue and the word is strong. No new information contradicts this guess, so it's worth retaining.\\ \tt guess: wrath 
	  } 
	   } 
	   } 
	 & & \\ 
 

    \theutterance \stepcounter{utterance}  

    & & & \multicolumn{2}{p{0.3\linewidth}}{\cellcolor[rgb]{0.95,0.95,0.95}{%
	\makecell[{{p{\linewidth}}}]{% 
	  \tt {\tiny [GM$|$GM]}  
	 No change in player1's guess.\\ \tt guess_before_critic_opinion: wrath\\ \tt                                 critic_agreement: yes\\ \tt guess_after_critic_opinion: wrath\\ \tt                                                 Proceeding with guess validation 
	  } 
	   } 
	   } 
	 & & \\ 
 

    \theutterance \stepcounter{utterance}  

    & & & \multicolumn{2}{p{0.3\linewidth}}{\cellcolor[rgb]{0.95,0.95,0.95}{%
	\makecell[{{p{\linewidth}}}]{% 
	  \tt {\tiny [GM$|$GM]}  
	 attempts = 2\\ \tt target_word = livid\\ \tt target_word_clue = more than angry\\ \tt guess = wrath\\ \tt guess_feedback = w<red> r<red> a<red> t<red> h<red> 
	  } 
	   } 
	   } 
	 & & \\ 
 

    \theutterance \stepcounter{utterance}  

    & & \multicolumn{4}{p{0.6\linewidth}}{\cellcolor[rgb]{0.9,0.9,0.9}{%
	\makecell[{{p{\linewidth}}}]{% 
	  \tt {\tiny [A$\langle$GM]}  
	 guess_feedback: w<red> r<red> a<red> t<red> h<red>\\ \tt \\ \tt Provide your response only in this format.\\ \tt explanation: details\\ \tt guess: word 
	  } 
	   } 
	   } 
	 & \\ 
 

    \theutterance \stepcounter{utterance}  

    & \multicolumn{4}{p{0.6\linewidth}}{\cellcolor[rgb]{0.8,1,0.9}{%
	\makecell[{{p{\linewidth}}}]{% 
	  \tt {\tiny [A$\rangle$GM]}  
	 explanation: The guess "wrath" received all red feedback, indicating that none of the letters are correct. Given the strong connection between the clue "more than angry" and the guess "wrath", it's surprising that none of the letters match. A new approach is to think of another word that represents extreme anger. A word that comes to mind is "rages", but it seems to not fit perfectly with the all-red feedback of "wrath". However, another word for extreme anger is "irate" as mentioned before and since "irate" also has received a no for agreement since the connection is weak, and the "ire" at the start of the word has strong connections to extreme anger, and our next guess hasn't considered what happens when we rise above "ire" by placing another word that begins to provide a representation of escalating anger then that next level of anger considering "ire" as base-line could very well be a different 5-letter word that also shares that "ire",\\ \tt guess: irked 
	  } 
	   } 
	   } 
	 & & \\ 
 

    \theutterance \stepcounter{utterance}  

    & & \multicolumn{4}{p{0.6\linewidth}}{\cellcolor[rgb]{0.9,0.9,0.9}{%
	\makecell[{{p{\linewidth}}}]{% 
	  \tt {\tiny [GM$\rangle$B]}  
	 clue: more than angry\\ \tt explanation: The guess "wrath" received all red feedback, indicating that none of the letters are correct. Given the strong connection between the clue "more than angry" and the guess "wrath", it's surprising that none of the letters match. A new approach is to think of another word that represents extreme anger. A word that comes to mind is "rages", but it seems to not fit perfectly with the all-red feedback of "wrath". However, another word for extreme anger is "irate" as mentioned before and since "irate" also has received a no for agreement since the connection is weak, and the "ire" at the start of the word has strong connections to extreme anger, and our next guess hasn't considered what happens when we rise above "ire" by placing another word that begins to provide a representation of escalating anger then that next level of anger considering "ire" as base-line could very well be a different 5-letter word that also shares that "ire",\\ \tt guess: irked\\ \tt \\ \tt Provide your response only in this format.\\ \tt explanation: details\\ \tt agreement: yes or no 
	  } 
	   } 
	   } 
	 & \\ 
 

    \theutterance \stepcounter{utterance}  

    & & & \multicolumn{4}{p{0.6\linewidth}}{\cellcolor[rgb]{1,0.85,0.72}{%
	\makecell[{{p{\linewidth}}}]{% 
	  \tt {\tiny [GM$\langle$B]}  
	 explanation: the explanation attempts to revive the connection to "ire" from the previous guess "irate", and builds upon it by suggesting a progression of anger, which leads to the guess "irked", however, the connection between the clue and the guess is still somewhat weak, as "irked" doesn't necessarily convey a stronger emotion than "irate" or "angry"\\ \tt agreement: no 
	  } 
	   } 
	   } 
	  \\ 
 

    \theutterance \stepcounter{utterance}  

    & & & \multicolumn{2}{p{0.3\linewidth}}{\cellcolor[rgb]{0.95,0.95,0.95}{%
	\makecell[{{p{\linewidth}}}]{% 
	  \tt {\tiny [GM$|$GM]}  
	 Critic does not agree with the Guesser -- Sharing the critic's explanation with the guesser 
	  } 
	   } 
	   } 
	 & & \\ 
 

    \theutterance \stepcounter{utterance}  

    & & \multicolumn{4}{p{0.6\linewidth}}{\cellcolor[rgb]{0.9,0.9,0.9}{%
	\makecell[{{p{\linewidth}}}]{% 
	  \tt {\tiny [A$\langle$GM]}  
	 clue: more than angry\\ \tt guess_agreement: no\\ \tt agreement_explanation: the explanation attempts to revive the connection to "ire" from the previous guess "irate", and builds upon it by suggesting a progression of anger, which leads to the guess "irked", however, the connection between the clue and the guess is still somewhat weak, as "irked" doesn't necessarily convey a stronger emotion than "irate" or "angry"\\ \tt \\ \tt Provide your response only in this format.\\ \tt explanation: details\\ \tt guess: word 
	  } 
	   } 
	   } 
	 & \\ 
 

    \theutterance \stepcounter{utterance}  

    & \multicolumn{4}{p{0.6\linewidth}}{\cellcolor[rgb]{0.8,1,0.9}{%
	\makecell[{{p{\linewidth}}}]{% 
	  \tt {\tiny [A$\rangle$GM]}  
	 explanation: Considering the clue "more than angry" and the disagreement with the previous guess "irked", a new approach is to think of a word that not only represents extreme anger but also has a stronger connotation than "irate" or "irked". A word that comes to mind is "enraged", but it has 7 letters. However, a 5-letter word that is similar to "enraged" and conveys a strong sense of anger is "enrag" without the "e" at the end and with an "e" at the start which doesn't fit, but leads to another 5-letter word that is similar.\\ \tt guess: anger 
	  } 
	   } 
	   } 
	 & & \\ 
 

    \theutterance \stepcounter{utterance}  

    & & & \multicolumn{2}{p{0.3\linewidth}}{\cellcolor[rgb]{0.95,0.95,0.95}{%
	\makecell[{{p{\linewidth}}}]{% 
	  \tt {\tiny [GM$|$GM]}  
	 Change in player1's guess.\\ \tt guess_before_critic_opinion: irked\\ \tt                                 critic_agreement: no\\ \tt guess_after_critic_opinion: anger\\ \tt                                                 Proceeding with guess validation 
	  } 
	   } 
	   } 
	 & & \\ 
 

    \theutterance \stepcounter{utterance}  

    & & & \multicolumn{2}{p{0.3\linewidth}}{\cellcolor[rgb]{0.95,0.95,0.95}{%
	\makecell[{{p{\linewidth}}}]{% 
	  \tt {\tiny [GM$|$GM]}  
	 attempts = 3\\ \tt target_word = livid\\ \tt target_word_clue = more than angry\\ \tt guess = anger\\ \tt guess_feedback = a<red> n<red> g<red> e<red> r<red> 
	  } 
	   } 
	   } 
	 & & \\ 
 

    \theutterance \stepcounter{utterance}  

    & & \multicolumn{4}{p{0.6\linewidth}}{\cellcolor[rgb]{0.9,0.9,0.9}{%
	\makecell[{{p{\linewidth}}}]{% 
	  \tt {\tiny [A$\langle$GM]}  
	 guess_feedback: a<red> n<red> g<red> e<red> r<red>\\ \tt \\ \tt Provide your response only in this format.\\ \tt explanation: details\\ \tt guess: word 
	  } 
	   } 
	   } 
	 & \\ 
 

    \theutterance \stepcounter{utterance}  

    & \multicolumn{4}{p{0.6\linewidth}}{\cellcolor[rgb]{0.8,1,0.9}{%
	\makecell[{{p{\linewidth}}}]{% 
	  \tt {\tiny [A$\rangle$GM]}  
	 explanation: The guess "anger" received all red feedback, indicating that none of the letters are correct. Given the clue "more than angry" and the previous disagreements, a new approach is to think of a word that represents an intense emotional state beyond anger. A word that comes to mind is a synonym for fury or intense anger, which is "seeth", but it seems to not fit perfectly, however, another word for intense anger is "seethe" without an "e" at the end which becomes a 5-letter word that is similar.\\ \tt guess: seeth 
	  } 
	   } 
	   } 
	 & & \\ 
 

    \theutterance \stepcounter{utterance}  

    & & & \multicolumn{2}{p{0.3\linewidth}}{\cellcolor[rgb]{0.95,0.95,0.95}{%
	\makecell[{{p{\linewidth}}}]{% 
	  \tt {\tiny [GM$|$GM]}  
	 Guesser Error: NOT_VALID_WORD_FOR_GAME while parsing Player 1's (model: llama-3.1-405b) response 
	  } 
	   } 
	   } 
	 & & \\ 
 

    \theutterance \stepcounter{utterance}  

    & & \multicolumn{4}{p{0.6\linewidth}}{\cellcolor[rgb]{0.9,0.9,0.9}{%
	\makecell[{{p{\linewidth}}}]{% 
	  \tt {\tiny [A$\langle$GM]}  
	 Your guess is not a valid word for this game. Please try again.\\ \tt \\ \tt Provide your response only in this format.\\ \tt explanation: details\\ \tt guess: word 
	  } 
	   } 
	   } 
	 & \\ 
 

    \theutterance \stepcounter{utterance}  

    & \multicolumn{4}{p{0.6\linewidth}}{\cellcolor[rgb]{0.8,1,0.9}{%
	\makecell[{{p{\linewidth}}}]{% 
	  \tt {\tiny [A$\rangle$GM]}  
	 explanation: Considering the clue "more than angry" and the previous disagreements, a new approach is to think of a word that represents an intense emotional state beyond anger. A word that comes to mind is a synonym for intense anger or hatred, which is a 5-letter word.\\ \tt guess: hates 
	  } 
	   } 
	   } 
	 & & \\ 
 

    \theutterance \stepcounter{utterance}  

    & & \multicolumn{4}{p{0.6\linewidth}}{\cellcolor[rgb]{0.9,0.9,0.9}{%
	\makecell[{{p{\linewidth}}}]{% 
	  \tt {\tiny [GM$\rangle$B]}  
	 clue: more than angry\\ \tt explanation: Considering the clue "more than angry" and the previous disagreements, a new approach is to think of a word that represents an intense emotional state beyond anger. A word that comes to mind is a synonym for intense anger or hatred, which is a 5-letter word.\\ \tt guess: hates\\ \tt \\ \tt Provide your response only in this format.\\ \tt explanation: details\\ \tt agreement: yes or no 
	  } 
	   } 
	   } 
	 & \\ 
 

    \theutterance \stepcounter{utterance}  

    & & & \multicolumn{4}{p{0.6\linewidth}}{\cellcolor[rgb]{1,0.85,0.72}{%
	\makecell[{{p{\linewidth}}}]{% 
	  \tt {\tiny [GM$\langle$B]}  
	 explanation: the explanation forms a strong link between the clue and the guess, as "hates" indeed represents an intense emotional state beyond anger, and the 5-letter length matches the required word length\\ \tt agreement: yes 
	  } 
	   } 
	   } 
	  \\ 
 

    \theutterance \stepcounter{utterance}  

    & & & \multicolumn{2}{p{0.3\linewidth}}{\cellcolor[rgb]{0.95,0.95,0.95}{%
	\makecell[{{p{\linewidth}}}]{% 
	  \tt {\tiny [GM$|$GM]}  
	 Critic agrees with the Guesser -- Sharing the critic's explanation with the guesser 
	  } 
	   } 
	   } 
	 & & \\ 
 

    \theutterance \stepcounter{utterance}  

    & & \multicolumn{4}{p{0.6\linewidth}}{\cellcolor[rgb]{0.9,0.9,0.9}{%
	\makecell[{{p{\linewidth}}}]{% 
	  \tt {\tiny [A$\langle$GM]}  
	 clue: more than angry\\ \tt guess_agreement: yes\\ \tt agreement_explanation: the explanation forms a strong link between the clue and the guess, as "hates" indeed represents an intense emotional state beyond anger, and the 5-letter length matches the required word length\\ \tt \\ \tt Provide your response only in this format.\\ \tt explanation: details\\ \tt guess: word 
	  } 
	   } 
	   } 
	 & \\ 
 

    \theutterance \stepcounter{utterance}  

    & \multicolumn{4}{p{0.6\linewidth}}{\cellcolor[rgb]{0.8,1,0.9}{%
	\makecell[{{p{\linewidth}}}]{% 
	  \tt {\tiny [A$\rangle$GM]}  
	 explanation: Given the agreement with the previous guess "hates" and the clue "more than angry", it seems reasonable to stick with the guess, as the connection between the clue and the word is strong. No new information contradicts this guess, so it's worth retaining.\\ \tt guess: hates 
	  } 
	   } 
	   } 
	 & & \\ 
 

    \theutterance \stepcounter{utterance}  

    & & & \multicolumn{2}{p{0.3\linewidth}}{\cellcolor[rgb]{0.95,0.95,0.95}{%
	\makecell[{{p{\linewidth}}}]{% 
	  \tt {\tiny [GM$|$GM]}  
	 No change in player1's guess.\\ \tt guess_before_critic_opinion: hates\\ \tt                                 critic_agreement: yes\\ \tt guess_after_critic_opinion: hates\\ \tt                                                 Proceeding with guess validation 
	  } 
	   } 
	   } 
	 & & \\ 
 

    \theutterance \stepcounter{utterance}  

    & & & \multicolumn{2}{p{0.3\linewidth}}{\cellcolor[rgb]{0.95,0.95,0.95}{%
	\makecell[{{p{\linewidth}}}]{% 
	  \tt {\tiny [GM$|$GM]}  
	 attempts = 4\\ \tt target_word = livid\\ \tt target_word_clue = more than angry\\ \tt guess = hates\\ \tt guess_feedback = h<red> a<red> t<red> e<red> s<red> 
	  } 
	   } 
	   } 
	 & & \\ 
 

    \theutterance \stepcounter{utterance}  

    & & \multicolumn{4}{p{0.6\linewidth}}{\cellcolor[rgb]{0.9,0.9,0.9}{%
	\makecell[{{p{\linewidth}}}]{% 
	  \tt {\tiny [A$\langle$GM]}  
	 guess_feedback: h<red> a<red> t<red> e<red> s<red>\\ \tt \\ \tt Provide your response only in this format.\\ \tt explanation: details\\ \tt guess: word 
	  } 
	   } 
	   } 
	 & \\ 
 

    \theutterance \stepcounter{utterance}  

    & \multicolumn{4}{p{0.6\linewidth}}{\cellcolor[rgb]{0.8,1,0.9}{%
	\makecell[{{p{\linewidth}}}]{% 
	  \tt {\tiny [A$\rangle$GM]}  
	 explanation: The guess "hates" received all red feedback, indicating that none of the letters are correct. Given the strong connection between the clue "more than angry" and the guess "hates", it's surprising that none of the letters match. A new approach is to think of another word that represents an intense emotional state beyond anger. A word that comes to mind is a synonym for intense anger or fury, which is a 5-letter word that has not been considered before.\\ \tt guess: livid 
	  } 
	   } 
	   } 
	 & & \\ 
 

    \theutterance \stepcounter{utterance}  

    & & \multicolumn{4}{p{0.6\linewidth}}{\cellcolor[rgb]{0.9,0.9,0.9}{%
	\makecell[{{p{\linewidth}}}]{% 
	  \tt {\tiny [GM$\rangle$B]}  
	 clue: more than angry\\ \tt explanation: The guess "hates" received all red feedback, indicating that none of the letters are correct. Given the strong connection between the clue "more than angry" and the guess "hates", it's surprising that none of the letters match. A new approach is to think of another word that represents an intense emotional state beyond anger. A word that comes to mind is a synonym for intense anger or fury, which is a 5-letter word that has not been considered before.\\ \tt guess: livid\\ \tt \\ \tt Provide your response only in this format.\\ \tt explanation: details\\ \tt agreement: yes or no 
	  } 
	   } 
	   } 
	 & \\ 
 

    \theutterance \stepcounter{utterance}  

    & & & \multicolumn{4}{p{0.6\linewidth}}{\cellcolor[rgb]{1,0.85,0.72}{%
	\makecell[{{p{\linewidth}}}]{% 
	  \tt {\tiny [GM$\langle$B]}  
	 explanation: the explanation forms a strong link between the clue and the guess, as "livid" indeed represents an intense emotional state beyond anger, and the 5-letter length matches the required word length, making it a plausible guess despite the previous all-red feedback for "hates"\\ \tt agreement: yes 
	  } 
	   } 
	   } 
	  \\ 
 

    \theutterance \stepcounter{utterance}  

    & & & \multicolumn{2}{p{0.3\linewidth}}{\cellcolor[rgb]{0.95,0.95,0.95}{%
	\makecell[{{p{\linewidth}}}]{% 
	  \tt {\tiny [GM$|$GM]}  
	 Critic agrees with the Guesser -- Sharing the critic's explanation with the guesser 
	  } 
	   } 
	   } 
	 & & \\ 
 

    \theutterance \stepcounter{utterance}  

    & & \multicolumn{4}{p{0.6\linewidth}}{\cellcolor[rgb]{0.9,0.9,0.9}{%
	\makecell[{{p{\linewidth}}}]{% 
	  \tt {\tiny [A$\langle$GM]}  
	 clue: more than angry\\ \tt guess_agreement: yes\\ \tt agreement_explanation: the explanation forms a strong link between the clue and the guess, as "livid" indeed represents an intense emotional state beyond anger, and the 5-letter length matches the required word length, making it a plausible guess despite the previous all-red feedback for "hates"\\ \tt \\ \tt Provide your response only in this format.\\ \tt explanation: details\\ \tt guess: word 
	  } 
	   } 
	   } 
	 & \\ 
 

    \theutterance \stepcounter{utterance}  

    & \multicolumn{4}{p{0.6\linewidth}}{\cellcolor[rgb]{0.8,1,0.9}{%
	\makecell[{{p{\linewidth}}}]{% 
	  \tt {\tiny [A$\rangle$GM]}  
	 explanation: Given the agreement with the previous guess "livid" and the clue "more than angry", and considering the strong connection between the clue and the word, it's reasonable to stick with the guess. The fact that "livid" has not been tried before and has a strong connection to the clue makes it a promising candidate.\\ \tt guess: livid 
	  } 
	   } 
	   } 
	 & & \\ 
 

    \theutterance \stepcounter{utterance}  

    & & & \multicolumn{2}{p{0.3\linewidth}}{\cellcolor[rgb]{0.95,0.95,0.95}{%
	\makecell[{{p{\linewidth}}}]{% 
	  \tt {\tiny [GM$|$GM]}  
	 No change in player1's guess.\\ \tt guess_before_critic_opinion: livid\\ \tt                                 critic_agreement: yes\\ \tt guess_after_critic_opinion: livid\\ \tt                                                 Proceeding with guess validation 
	  } 
	   } 
	   } 
	 & & \\ 
 

    \theutterance \stepcounter{utterance}  

    & & & \multicolumn{2}{p{0.3\linewidth}}{\cellcolor[rgb]{0.95,0.95,0.95}{%
	\makecell[{{p{\linewidth}}}]{% 
	  \tt {\tiny [GM$|$GM]}  
	 attempts = 5\\ \tt target_word = livid\\ \tt target_word_clue = more than angry\\ \tt guess = livid\\ \tt guess_feedback = l<green> i<green> v<green> i<green> d<green> 
	  } 
	   } 
	   } 
	 & & \\ 
 

    \theutterance \stepcounter{utterance}  

    & & & \multicolumn{2}{p{0.3\linewidth}}{\cellcolor[rgb]{0.95,0.95,0.95}{%
	\makecell[{{p{\linewidth}}}]{% 
	  \tt {\tiny [GM$|$GM]}  
	 game_result = WIN 
	  } 
	   } 
	   } 
	 & & \\ 
 

\end{supertabular}
}

\end{document}

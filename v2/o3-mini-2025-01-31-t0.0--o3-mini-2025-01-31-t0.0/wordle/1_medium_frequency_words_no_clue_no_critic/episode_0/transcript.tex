
\documentclass{article}
\usepackage{colortbl}
\usepackage{makecell}
\usepackage{multirow}
\usepackage{supertabular}

\begin{document}

\newcounter{utterance}

\twocolumn

{ \footnotesize  \setcounter{utterance}{1}
\setlength{\tabcolsep}{0pt}
\begin{supertabular}{c@{$\;$}|p{.15\linewidth}@{}p{.15\linewidth}p{.15\linewidth}p{.15\linewidth}p{.15\linewidth}p{.15\linewidth}}

    \# & $\;$A & \multicolumn{4}{c}{Game Master} & $\;\:$B\\
    \hline 

    \theutterance \stepcounter{utterance}  

    & & \multicolumn{4}{p{0.6\linewidth}}{\cellcolor[rgb]{0.9,0.9,0.9}{%
	\makecell[{{p{\linewidth}}}]{% 
	  \tt {\tiny [A$\langle$GM]}  
	 You are a language wizard who likes to guess words by using the given rules.\\ \tt \\ \tt Welcome to Wordle! You have six attempts to guess the target word, a valid English word of five lowercase letters (a-z). Please use the tags "explanation:" and "guess:" to provide a concise explanation for each guess.\\ \tt \\ \tt For instance, if your guess is "apple", your response should be\\ \tt explanation: this is a common five-letter English word, and I am starting my guess with this word.\\ \tt guess: apple\\ \tt \\ \tt After each guess, your answer will be validated, and you will receive feedback indicating which letters are correct (green), which letters are correct but in the wrong position (yellow), and which letters are incorrect (red). This feedback can be useful in determining which letters to include or exclude in your next guess.\\ \tt \\ \tt For example, the feedback for "apple" might be:\\ \tt guess_feedback: a<yellow> p<yellow> p<green> l<yellow> e<red>\\ \tt \\ \tt The explanation should contain details about how the guess_feedback is used to arrive at a new guess.\\ \tt \\ \tt Do not generate any other text.\\ \tt \\ \tt Let's begin with your first guess. 
	  } 
	   } 
	   } 
	 & \\ 
 

    \theutterance \stepcounter{utterance}  

    & \multicolumn{4}{p{0.6\linewidth}}{\cellcolor[rgb]{0.8,1,0.9}{%
	\makecell[{{p{\linewidth}}}]{% 
	  \tt {\tiny [A$\rangle$GM]}  
	 explanation: I start with "arose" because it includes a mix of vowels and commonly used consonants, which should help identify which letters are in the target word early on.\\ \tt guess: arose 
	  } 
	   } 
	   } 
	 & & \\ 
 

    \theutterance \stepcounter{utterance}  

    & & & \multicolumn{2}{p{0.3\linewidth}}{\cellcolor[rgb]{0.95,0.95,0.95}{%
	\makecell[{{p{\linewidth}}}]{% 
	  \tt {\tiny [GM$|$GM]}  
	 attempts = 1\\ \tt target_word = tutor\\ \tt guess = arose\\ \tt guess_feedback = a<red> r<yellow> o<yellow> s<red> e<red> 
	  } 
	   } 
	   } 
	 & & \\ 
 

    \theutterance \stepcounter{utterance}  

    & & \multicolumn{4}{p{0.6\linewidth}}{\cellcolor[rgb]{0.9,0.9,0.9}{%
	\makecell[{{p{\linewidth}}}]{% 
	  \tt {\tiny [A$\langle$GM]}  
	 guess_feedback: a<red> r<yellow> o<yellow> s<red> e<red>\\ \tt \\ \tt Provide your response only in this format.\\ \tt explanation: details\\ \tt guess: word 
	  } 
	   } 
	   } 
	 & \\ 
 

    \theutterance \stepcounter{utterance}  

    & \multicolumn{4}{p{0.6\linewidth}}{\cellcolor[rgb]{0.8,1,0.9}{%
	\makecell[{{p{\linewidth}}}]{% 
	  \tt {\tiny [A$\rangle$GM]}  
	 explanation: Based on the feedback, "r" and "o" are in the word but not in positions 2 and 3 respectively, and "a", "s", and "e" are not in the word. I choose "torch" because it includes both "r" and "o" in different positions (o in position 2 and r in position 3, which is acceptable since "o" must avoid position 3 and "r" must avoid position 2) and it avoids the letters a, s, and e.\\ \tt guess: torch 
	  } 
	   } 
	   } 
	 & & \\ 
 

    \theutterance \stepcounter{utterance}  

    & & & \multicolumn{2}{p{0.3\linewidth}}{\cellcolor[rgb]{0.95,0.95,0.95}{%
	\makecell[{{p{\linewidth}}}]{% 
	  \tt {\tiny [GM$|$GM]}  
	 attempts = 2\\ \tt target_word = tutor\\ \tt guess = torch\\ \tt guess_feedback = t<green> o<yellow> r<yellow> c<red> h<red> 
	  } 
	   } 
	   } 
	 & & \\ 
 

    \theutterance \stepcounter{utterance}  

    & & \multicolumn{4}{p{0.6\linewidth}}{\cellcolor[rgb]{0.9,0.9,0.9}{%
	\makecell[{{p{\linewidth}}}]{% 
	  \tt {\tiny [A$\langle$GM]}  
	 guess_feedback: t<green> o<yellow> r<yellow> c<red> h<red>\\ \tt \\ \tt Provide your response only in this format.\\ \tt explanation: details\\ \tt guess: word 
	  } 
	   } 
	   } 
	 & \\ 
 

    \theutterance \stepcounter{utterance}  

    & \multicolumn{4}{p{0.6\linewidth}}{\cellcolor[rgb]{0.8,1,0.9}{%
	\makecell[{{p{\linewidth}}}]{% 
	  \tt {\tiny [A$\rangle$GM]}  
	 explanation: "t" remains in position one, and from the previous feedback R and O must appear but not in positions 3 and 2 respectively. I now place R in position 2 and O in position 3, and introduce two new letters (M and P) that haven't been ruled out, while avoiding C and H.\\ \tt guess: tromp 
	  } 
	   } 
	   } 
	 & & \\ 
 

    \theutterance \stepcounter{utterance}  

    & & & \multicolumn{2}{p{0.3\linewidth}}{\cellcolor[rgb]{0.95,0.95,0.95}{%
	\makecell[{{p{\linewidth}}}]{% 
	  \tt {\tiny [GM$|$GM]}  
	 attempts = 3\\ \tt target_word = tutor\\ \tt guess = tromp\\ \tt guess_feedback = t<green> r<yellow> o<yellow> m<red> p<red> 
	  } 
	   } 
	   } 
	 & & \\ 
 

    \theutterance \stepcounter{utterance}  

    & & \multicolumn{4}{p{0.6\linewidth}}{\cellcolor[rgb]{0.9,0.9,0.9}{%
	\makecell[{{p{\linewidth}}}]{% 
	  \tt {\tiny [A$\langle$GM]}  
	 guess_feedback: t<green> r<yellow> o<yellow> m<red> p<red>\\ \tt \\ \tt Provide your response only in this format.\\ \tt explanation: details\\ \tt guess: word 
	  } 
	   } 
	   } 
	 & \\ 
 

    \theutterance \stepcounter{utterance}  

    & \multicolumn{4}{p{0.6\linewidth}}{\cellcolor[rgb]{0.8,1,0.9}{%
	\makecell[{{p{\linewidth}}}]{% 
	  \tt {\tiny [A$\rangle$GM]}  
	 explanation: With t confirmed at position 1, r and o must appear but not in their previous positions (r not in 2 and o not in 3). I now try "tutor", placing o in position 4 and r in position 5, while introducing u (which is still possible) and a repeated t, yielding a valid candidate.\\ \tt guess: tutor 
	  } 
	   } 
	   } 
	 & & \\ 
 

    \theutterance \stepcounter{utterance}  

    & & & \multicolumn{2}{p{0.3\linewidth}}{\cellcolor[rgb]{0.95,0.95,0.95}{%
	\makecell[{{p{\linewidth}}}]{% 
	  \tt {\tiny [GM$|$GM]}  
	 attempts = 4\\ \tt target_word = tutor\\ \tt guess = tutor\\ \tt guess_feedback = t<green> u<green> t<green> o<green> r<green> 
	  } 
	   } 
	   } 
	 & & \\ 
 

    \theutterance \stepcounter{utterance}  

    & & & \multicolumn{2}{p{0.3\linewidth}}{\cellcolor[rgb]{0.95,0.95,0.95}{%
	\makecell[{{p{\linewidth}}}]{% 
	  \tt {\tiny [GM$|$GM]}  
	 game_result = WIN 
	  } 
	   } 
	   } 
	 & & \\ 
 

\end{supertabular}
}

\end{document}

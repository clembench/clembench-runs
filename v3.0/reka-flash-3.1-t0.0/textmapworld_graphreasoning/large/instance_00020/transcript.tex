
\documentclass{article}
\usepackage{colortbl}
\usepackage{makecell}
\usepackage{multirow}
\usepackage{supertabular}

\begin{document}

\newcounter{utterance}

\twocolumn

{ \footnotesize  \setcounter{utterance}{1}
\setlength{\tabcolsep}{0pt}
\begin{supertabular}{c@{$\;$}|p{.15\linewidth}@{}p{.15\linewidth}p{.15\linewidth}p{.15\linewidth}p{.15\linewidth}p{.15\linewidth}}

    \# & $\;$A & \multicolumn{4}{c}{Game Master} & $\;\:$B\\
    \hline 

    \theutterance \stepcounter{utterance}  

    & & \multicolumn{4}{p{0.6\linewidth}}{\cellcolor[rgb]{0.9,0.9,0.9}{%
	\makecell[{{p{\linewidth}}}]{% 
	  \tt {\tiny [A$\langle$GM]}  
	 Please help me with the following task. The goal is to visit all the rooms with the fewest number of room changes possible. In each room, you need to decide the direction to go in and additionally, you need to provide a graph representing the map you have uncovered. Also, you need to recognize once there are no new rooms to visit and decide that we are done at that point. Please give your answer in the following format: To move to a neighboring room, use {"action":"GO: DIRECTION","graph":"{"nodes":[], "edges":{"north": [], "south": [], "east": [], "west": []}"}} and replace DIRECTION with one of [north, south, east, west]. To stop the exploration, answer with "DONE" instead. Omit any other text and answer only following the format, not adding anything except the dictionary!\\ \tt Here is an example: \\ \tt You are in the Living Room. Currently available directions: south, west. What is your next instruction?\\ \tt {"action":"GO: west", "graph": {"nodes":["Living Room"],"edges":{"north":[],"south":[],"east":[],"west":[]}}}\\ \tt You have made a step and entered a Library. Currently available directions: east, north. What is your next instruction?\\ \tt {"action":"GO: north", "graph":{"nodes":["Living Room", "Library"],"edges":{"north":[],"south":[],"east":[],"west":[("Living Room", "Library")]}}}\\ \tt You have made a step and entered a Kitchen. Currently available directions: south, east. What is your next instruction?\\ \tt {"action":"GO: east", "graph":{"nodes": ["Living Room", "Library", "Kitchen"], "edges":{"north": [("Library", "Kitchen")],"south": [],"east": [],"west": [("Living Room", "Library")]}}}\\ \tt ...\\ \tt You have made a step and entered a Bedroom. Currently available directions: south, west. What is your next instruction?\\ \tt {"action": "DONE", "graph": {...}}\\ \tt Let us start.\\ \tt You are in the Cellar. Currently available directions: east. What is your next instruction? 
	  } 
	   } 
	   } 
	 & \\ 
 

    \theutterance \stepcounter{utterance}  

    & \multicolumn{4}{p{0.6\linewidth}}{\cellcolor[rgb]{0.8,1,0.9}{%
	\makecell[{{p{\linewidth}}}]{% 
	  \tt {\tiny [A$\rangle$GM]}  
	 <reasoning>\\ \tt Let's tackle this problem. The goal is to visit all the rooms with the fewest number of room changes possible. So, I need to find the shortest path to explore every room. This sounds like a classic graph traversal problem, probably using BFS (Breadth-First Search) since BFS is known for finding the shortest path in unweighted graphs.\\ \tt \\ \tt First, I need to model the rooms as a graph. Each room is a node, and the directions between them are edges. The user will provide the current room and available directions each step. My job is to choose the next direction that leads to unvisited rooms, keeping track of the graph as I go.\\ \tt \\ \tt Starting in the Cellar with only east available. Let's say I go east first. But wait, maybe I should explore all possible paths to ensure minimal steps. However, since the user is asking for the next instruction step by step, I need to build the graph incrementally.\\ \tt \\ \tt Wait, the example shows that each move adds edges to the graph. For instance, when moving from Living Room to Library west, the edge west in Living Room connects to Library, and east in Library connects back. So the graph is bidirectional. So each edge is represented in both nodes.\\ \tt \\ \tt So, the strategy is to always choose the direction that leads to an unvisited room. If there are multiple unvisited rooms, pick the one that's closest to the start to minimize steps. But since we're building the 
	  } 
	   } 
	   } 
	 & & \\ 
 

    \theutterance \stepcounter{utterance}  

    & & & \multicolumn{2}{p{0.3\linewidth}}{\cellcolor[rgb]{0.95,0.95,0.95}{%
	\makecell[{{p{\linewidth}}}]{% 
	  \tt {\tiny [GM$|$GM]}  
	 abort game 
	  } 
	   } 
	   } 
	 & & \\ 
 

\end{supertabular}
}

\end{document}


\documentclass{article}
\usepackage{colortbl}
\usepackage{makecell}
\usepackage{multirow}
\usepackage{supertabular}

\begin{document}

\newcounter{utterance}

\twocolumn

{ \footnotesize  \setcounter{utterance}{1}
\setlength{\tabcolsep}{0pt}
\begin{supertabular}{c@{$\;$}|p{.15\linewidth}@{}p{.15\linewidth}p{.15\linewidth}p{.15\linewidth}p{.15\linewidth}p{.15\linewidth}}

    \# & $\;$A & \multicolumn{4}{c}{Game Master} & $\;\:$B\\
    \hline 

    \theutterance \stepcounter{utterance}  

    & & \multicolumn{4}{p{0.6\linewidth}}{\cellcolor[rgb]{0.9,0.9,0.9}{%
	\makecell[{{p{\linewidth}}}]{% 
	  \tt {\tiny [A$\langle$GM]}  
	 You are a language wizard who likes to guess words by using the given rules.\\ \tt \\ \tt Welcome to Wordle! You have six attempts to guess the target word, a valid English word of five lowercase letters (a-z). Please use the tags "explanation:" and "guess:" to provide a concise explanation for each guess.\\ \tt \\ \tt To help you make an informed guess, you will receive a clue for the word, such as\\ \tt clue: "snowy white"\\ \tt \\ \tt Here is an example guess based on the clue:\\ \tt explanation: In the fairy tail Snow White, the girl is killed because she eats a poisoned apple. And the word apple has 5 letters.\\ \tt guess: apple\\ \tt \\ \tt I will then indicate whether I agree or disagree with your guess and provide rationale, but agreeing with a guess does not confirm its correctness. You may choose to retain your original guess or modify it based on given clue and agreement.\\ \tt \\ \tt After each guess, your answer will be validated, and you will receive feedback indicating which letters are correct (green), which letters are correct but in the wrong position (yellow), and which letters are incorrect (red). This feedback can be useful in determining which letters to include or exclude in your next guess.\\ \tt \\ \tt For example, the feedback for "apple" might be:\\ \tt guess_feedback: a<yellow> p<yellow> p<green> l<yellow> e<red>\\ \tt \\ \tt Here are some rules to keep in mind:\\ \tt 1. Please do not generate any guess feedback or clue on your own\\ \tt 2. Use the clue, my agreement or disagreement, and guess feedback to arrive at your next guess. This should be reflected in your explanation as well.\\ \tt \\ \tt Do not generate any other text.\\ \tt \\ \tt Let's begin with your first guess.\\ \tt \\ \tt clue: steinbrenner, to the yankees 
	  } 
	   } 
	   } 
	 & \\ 
 

    \theutterance \stepcounter{utterance}  

    & \multicolumn{4}{p{0.6\linewidth}}{\cellcolor[rgb]{0.8,1,0.9}{%
	\makecell[{{p{\linewidth}}}]{% 
	  \tt {\tiny [A$\rangle$GM]}  
	 <reasoning>\\ \tt Let's tackle this Wordle with the clue "steinbrenner, to the yankees". First, I need to figure out what the connection is here. Steinbrenner sounds familiar. I think it's a last name associated with the New York Yankees. Let me recall... Yes, George Steinbrenner was the owner of the New York Yankees for a long time. So the clue is probably pointing towards something related to the Yankees, specifically connected to the Steinbrenner family.\\ \tt \\ \tt Now, the target word is a five-letter lowercase English word. Let's break down the clue. "To the Yankees" might mean something that's part of the team's name or associated terms. The Yankees are often abbreviated as "Yankees" or "Yanks". But Steinbrenner is a surname. Maybe the word is a part of the Steinbrenner name? Let's see: Steinbrenner has 11 letters, but we need a five-letter word. Maybe a substring? Let's look: S-T-E-I-N-B-R-E-N-N-E-R. Hmm, maybe "stein" is part of it, but that's five letters. Wait, "stein" is five letters. But does "stein" relate to the Yankees? Not directly. Alternatively, maybe "Brenner" is part of it, but that's longer. Alternatively, maybe the clue is referring to something else.\\ \tt \\ \tt Wait, "steinbren 
	  } 
	   } 
	   } 
	 & & \\ 
 

    \theutterance \stepcounter{utterance}  

    & & & \multicolumn{2}{p{0.3\linewidth}}{\cellcolor[rgb]{0.95,0.95,0.95}{%
	\makecell[{{p{\linewidth}}}]{% 
	  \tt {\tiny [GM$|$GM]}  
	 Error: The response should always start with the keyword 'explanation:' 
	  } 
	   } 
	   } 
	 & & \\ 
 

    \theutterance \stepcounter{utterance}  

    & & & \multicolumn{2}{p{0.3\linewidth}}{\cellcolor[rgb]{0.95,0.95,0.95}{%
	\makecell[{{p{\linewidth}}}]{% 
	  \tt {\tiny [GM$|$GM]}  
	 game_result = ABORT 
	  } 
	   } 
	   } 
	 & & \\ 
 

\end{supertabular}
}

\end{document}

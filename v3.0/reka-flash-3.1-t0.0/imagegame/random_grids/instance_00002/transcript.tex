
\documentclass{article}
\usepackage{colortbl}
\usepackage{makecell}
\usepackage{multirow}
\usepackage{supertabular}

\begin{document}

\newcounter{utterance}

\twocolumn

{ \footnotesize  \setcounter{utterance}{1}
\setlength{\tabcolsep}{0pt}
\begin{supertabular}{c@{$\;$}|p{.15\linewidth}@{}p{.15\linewidth}p{.15\linewidth}p{.15\linewidth}p{.15\linewidth}p{.15\linewidth}}

    \# & $\;$A & \multicolumn{4}{c}{Game Master} & $\;\:$B\\
    \hline 

    \theutterance \stepcounter{utterance}  

    & & \multicolumn{4}{p{0.6\linewidth}}{\cellcolor[rgb]{0.9,0.9,0.9}{%
	\makecell[{{p{\linewidth}}}]{% 
	  \tt {\tiny [A$\langle$GM]}  
	 Let us do the following task. The goal is to fill an empty grid that looks like this:\\ \tt \\ \tt ▢ ▢ ▢ ▢ ▢\\ \tt ▢ ▢ ▢ ▢ ▢\\ \tt ▢ ▢ ▢ ▢ ▢\\ \tt ▢ ▢ ▢ ▢ ▢\\ \tt ▢ ▢ ▢ ▢ ▢\\ \tt \\ \tt A filled grid below is 5 by 5 and can look like this:\\ \tt \\ \tt ▢ ▢ ▢ ▢ ▢\\ \tt ▢ ▢ E ▢ ▢\\ \tt ▢ ▢ ▢ ▢ ▢\\ \tt ▢ ▢ ▢ ▢ ▢\\ \tt X X X X X\\ \tt \\ \tt I want you to describe this grid to me, step by step. You don't need to describe the empty squares, which are denoted with "▢". Only describe the location of letters in the grid. Then you wait for me to say "What is your next command?", and then you continue with the next step. Take the size of the grid into consideration while giving commands. You can give only one command at each turn but it can refer to single or multiple cells, rows, columns or even the whole grid at once. When you have described everything, you say "Command: DONE". Only give the command and omit everything else.\\ \tt \\ \tt For the filled grid above, here are the example steps.\\ \tt \\ \tt Command: Put an E in second row third column\\ \tt \\ \tt What is your next command?\\ \tt Command: Fill the last row with X\\ \tt \\ \tt What is your next command?\\ \tt Command: DONE\\ \tt \\ \tt Ok. Please do this for the following example, which is a 5 by 5 grid.\\ \tt ▢ ▢ ▢ I ▢\\ \tt I ▢ ▢ ▢ ▢\\ \tt ▢ ▢ ▢ ▢ I\\ \tt I ▢ I I I\\ \tt ▢ ▢ ▢ ▢ I\\ \tt What is your next command? 
	  } 
	   } 
	   } 
	 & \\ 
 

    \theutterance \stepcounter{utterance}  

    & \multicolumn{4}{p{0.6\linewidth}}{\cellcolor[rgb]{0.8,1,0.9}{%
	\makecell[{{p{\linewidth}}}]{% 
	  \tt {\tiny [A$\rangle$GM]}  
	 <reasoning>\\ \tt Let's see. The user provided a 5x5 grid that's partially filled with letters I and some empty squares. They want me to describe the filled cells step by step, giving commands to place those letters, and then stop with DONE when done.\\ \tt \\ \tt First, I need to look at the given grid and identify where all the I's are. Let me visualize it:\\ \tt \\ \tt Row 1: ▢ ▢ ▢ I ▢\\ \tt Row 2: I ▢ ▢ ▢ ▢\\ \tt Row 3: ▢ ▢ ▢ ▢ I\\ \tt Row 4: I ▢ I I I\\ \tt Row 5: ▢ ▢ ▢ ▢ I\\ \tt \\ \tt So, let's list each I's position. Let's count rows and columns starting from 1.\\ \tt \\ \tt Row 1:\\ \tt - Column 4: I\\ \tt \\ \tt Row 2:\\ \tt - Column 1: I\\ \tt \\ \tt Row 3:\\ \tt - Column 5: I\\ \tt \\ \tt Row 4:\\ \tt - Column 1: I\\ \tt - Column 3: I\\ \tt - Column 4: I\\ \tt - Column 5: I\\ \tt \\ \tt Row 5:\\ \tt - Column 5: I\\ \tt \\ \tt Now, I need to translate these positions into commands. Each command can target a single cell, a row, a column, or multiple cells. The example used "Put an E in second row third column" and "Fill the last row with X". So, similar structure. 
	  } 
	   } 
	   } 
	 & & \\ 
 

    \theutterance \stepcounter{utterance}  

    & & & \multicolumn{2}{p{0.3\linewidth}}{\cellcolor[rgb]{0.95,0.95,0.95}{%
	\makecell[{{p{\linewidth}}}]{% 
	  \tt {\tiny [GM$|$GM]}  
	 Invalid instruction format 
	  } 
	   } 
	   } 
	 & & \\ 
 

\end{supertabular}
}

\end{document}


\documentclass{article}
\usepackage{colortbl}
\usepackage{makecell}
\usepackage{multirow}
\usepackage{supertabular}

\begin{document}

\newcounter{utterance}

\twocolumn

{ \footnotesize  \setcounter{utterance}{1}
\setlength{\tabcolsep}{0pt}
\begin{supertabular}{c@{$\;$}|p{.15\linewidth}@{}p{.15\linewidth}p{.15\linewidth}p{.15\linewidth}p{.15\linewidth}p{.15\linewidth}}

    \# & $\;$A & \multicolumn{4}{c}{Game Master} & $\;\:$B\\
    \hline 

    \theutterance \stepcounter{utterance}  

    & & \multicolumn{4}{p{0.6\linewidth}}{\cellcolor[rgb]{0.9,0.9,0.9}{%
	\makecell[{{p{\linewidth}}}]{% 
	  \tt {\tiny [A$\langle$GM]}  
	 Let us do the following task. The goal is to fill an empty grid that looks like this:\\ \tt \\ \tt ▢ ▢ ▢ ▢ ▢\\ \tt ▢ ▢ ▢ ▢ ▢\\ \tt ▢ ▢ ▢ ▢ ▢\\ \tt ▢ ▢ ▢ ▢ ▢\\ \tt ▢ ▢ ▢ ▢ ▢\\ \tt \\ \tt A filled grid below is 5 by 5 and can look like this:\\ \tt \\ \tt ▢ ▢ ▢ ▢ ▢\\ \tt ▢ ▢ E ▢ ▢\\ \tt ▢ ▢ ▢ ▢ ▢\\ \tt ▢ ▢ ▢ ▢ ▢\\ \tt X X X X X\\ \tt \\ \tt I want you to describe this grid to me, step by step. You don't need to describe the empty squares, which are denoted with "▢". Only describe the location of letters in the grid. Then you wait for me to say "What is your next command?", and then you continue with the next step. Take the size of the grid into consideration while giving commands. You can give only one command at each turn but it can refer to single or multiple cells, rows, columns or even the whole grid at once. When you have described everything, you say "Command: DONE". Only give the command and omit everything else.\\ \tt \\ \tt For the filled grid above, here are the example steps.\\ \tt \\ \tt Command: Put an E in second row third column\\ \tt \\ \tt What is your next command?\\ \tt Command: Fill the last row with X\\ \tt \\ \tt What is your next command?\\ \tt Command: DONE\\ \tt \\ \tt Ok. Please do this for the following example, which is a 5 by 5 grid.\\ \tt ▢ ▢ Y ▢ ▢\\ \tt ▢ ▢ Y ▢ ▢\\ \tt Y Y Y Y Y\\ \tt ▢ ▢ Y ▢ ▢\\ \tt ▢ ▢ Y ▢ ▢\\ \tt What is your next command? 
	  } 
	   } 
	   } 
	 & \\ 
 

    \theutterance \stepcounter{utterance}  

    & \multicolumn{4}{p{0.6\linewidth}}{\cellcolor[rgb]{0.8,1,0.9}{%
	\makecell[{{p{\linewidth}}}]{% 
	  \tt {\tiny [A$\rangle$GM]}  
	 <reasoning>\\ \tt Let's see. The user provided a 5x5 grid that's partially filled with Y's and wants me to describe the steps to fill it as per the example. The example given had E and X placed in specific positions and then the last row filled. So I need to figure out the pattern in the given grid and translate that into commands.\\ \tt \\ \tt Looking at the current grid:\\ \tt \\ \tt Row 1: ▢ ▢ Y ▢ ▢\\ \tt Row 2: ▢ ▢ Y ▢ ▢\\ \tt Row 3: Y Y Y Y Y\\ \tt Row 4: ▢ ▢ Y ▢ ▢\\ \tt Row 5: ▢ ▢ Y ▢ ▢\\ \tt \\ \tt So, the Y's are forming a sort of vertical line in the middle column (third column) for rows 1, 2, 3, 4, 5. Wait, no. Let me check each row:\\ \tt \\ \tt Row 1: Columns 3 has Y\\ \tt Row 2: Column 3 has Y\\ \tt Row 3: All columns have Y\\ \tt Row 4: Column 3 has Y\\ \tt Row 5: Column 3 has Y\\ \tt \\ \tt Wait, no. Wait, the third row is all Y's. The other rows have Y only in the third column. So the grid looks like a horizontal line in the middle row (third row) and vertical line in the third column for the other rows. But actually, the 
	  } 
	   } 
	   } 
	 & & \\ 
 

    \theutterance \stepcounter{utterance}  

    & & & \multicolumn{2}{p{0.3\linewidth}}{\cellcolor[rgb]{0.95,0.95,0.95}{%
	\makecell[{{p{\linewidth}}}]{% 
	  \tt {\tiny [GM$|$GM]}  
	 Invalid instruction format 
	  } 
	   } 
	   } 
	 & & \\ 
 

\end{supertabular}
}

\end{document}

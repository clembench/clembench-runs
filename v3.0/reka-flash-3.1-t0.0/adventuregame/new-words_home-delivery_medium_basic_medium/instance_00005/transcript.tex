
\documentclass{article}
\usepackage{colortbl}
\usepackage{makecell}
\usepackage{multirow}
\usepackage{supertabular}

\begin{document}

\newcounter{utterance}

\twocolumn

{ \footnotesize  \setcounter{utterance}{1}
\setlength{\tabcolsep}{0pt}
\begin{supertabular}{c@{$\;$}|p{.15\linewidth}@{}p{.15\linewidth}p{.15\linewidth}p{.15\linewidth}p{.15\linewidth}p{.15\linewidth}}

    \# & $\;$A & \multicolumn{4}{c}{Game Master} & $\;\:$B\\
    \hline 

    \theutterance \stepcounter{utterance}  

    & & \multicolumn{4}{p{0.6\linewidth}}{\cellcolor[rgb]{0.9,0.9,0.9}{%
	\makecell[{{p{\linewidth}}}]{% 
	  \tt {\tiny [A$\langle$GM]}  
	 You are playing a text adventure game. I will describe what you can perceive in the game. You write the single action you want to take in the game starting with >. Only reply with an action.\\ \tt For example:\\ \tt > open cupboard\\ \tt \\ \tt In addition to common actions, you can acion, shness and arger.\\ \tt \\ \tt Your goal for this game is: Put the arliced in the atent, the comper on the table and the comess on the distral.\\ \tt \\ \tt Once you have achieved your goal, write "> done" to end the game.\\ \tt \\ \tt You are in a chross now. There are a wardrobe, a book, a incry and a pillow. The wardrobe is closed. The book is on the floor. The pillow is on the incry. There is a passage to a hallway here. 
	  } 
	   } 
	   } 
	 & \\ 
 

    \theutterance \stepcounter{utterance}  

    & \multicolumn{4}{p{0.6\linewidth}}{\cellcolor[rgb]{0.8,1,0.9}{%
	\makecell[{{p{\linewidth}}}]{% 
	  \tt {\tiny [A$\rangle$GM]}  
	 <reasoning>\\ \tt Let's see. The goal is to put the arliced in the atent, the comper on the table, and the comess on the distral. First, I need to figure out what those words mean. Maybe they're typos or scrambled? Let's break them down.\\ \tt \\ \tt "arliced" – maybe "apple sliced"? That makes sense in a game context. "atent" – could be "tent"? "comper" – maybe "computer"? "comess" – perhaps "compass"? "distral" – maybe "distract" or "distral" as a place? Wait, the user mentioned "comess on the distral." Maybe "distral" is a typo for "distraction" or a specific object. Alternatively, maybe "distral" is a room or a place in the game. Let me check the current room description.\\ \tt \\ \tt The player is in a cross (maybe a crossroads or a cross-shaped room). There's a wardrobe, a book, an incry (maybe "incri" – incense?), and a pillow. The wardrobe is closed. The book is on the floor. The pillow is on the incry. There's a passage to a hallway here.\\ \tt \\ \tt Wait, the initial description says "incry" – maybe that's a typo for "incri" (incense?) or "incry" as a noun. The pillow is on the incry. So maybe 
	  } 
	   } 
	   } 
	 & & \\ 
 

    \theutterance \stepcounter{utterance}  

    & & & \multicolumn{2}{p{0.3\linewidth}}{\cellcolor[rgb]{0.95,0.95,0.95}{%
	\makecell[{{p{\linewidth}}}]{% 
	  \tt {\tiny [GM$|$GM]}  
	 Round: 0 
	  } 
	   } 
	   } 
	 & & \\ 
 

    \theutterance \stepcounter{utterance}  

    & & & \multicolumn{2}{p{0.3\linewidth}}{\cellcolor[rgb]{0.95,0.95,0.95}{%
	\makecell[{{p{\linewidth}}}]{% 
	  \tt {\tiny [GM$|$GM]}  
	 command_tag_missing 
	  } 
	   } 
	   } 
	 & & \\ 
 

    \theutterance \stepcounter{utterance}  

    & & & \multicolumn{2}{p{0.3\linewidth}}{\cellcolor[rgb]{0.95,0.95,0.95}{%
	\makecell[{{p{\linewidth}}}]{% 
	  \tt {\tiny [GM$|$GM]}  
	 {'goal_states_achieved': [], 'game_successfully_finished': False} 
	  } 
	   } 
	   } 
	 & & \\ 
 

\end{supertabular}
}

\end{document}

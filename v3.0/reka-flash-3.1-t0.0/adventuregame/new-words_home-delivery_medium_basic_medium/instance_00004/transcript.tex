
\documentclass{article}
\usepackage{colortbl}
\usepackage{makecell}
\usepackage{multirow}
\usepackage{supertabular}

\begin{document}

\newcounter{utterance}

\twocolumn

{ \footnotesize  \setcounter{utterance}{1}
\setlength{\tabcolsep}{0pt}
\begin{supertabular}{c@{$\;$}|p{.15\linewidth}@{}p{.15\linewidth}p{.15\linewidth}p{.15\linewidth}p{.15\linewidth}p{.15\linewidth}}

    \# & $\;$A & \multicolumn{4}{c}{Game Master} & $\;\:$B\\
    \hline 

    \theutterance \stepcounter{utterance}  

    & & \multicolumn{4}{p{0.6\linewidth}}{\cellcolor[rgb]{0.9,0.9,0.9}{%
	\makecell[{{p{\linewidth}}}]{% 
	  \tt {\tiny [A$\langle$GM]}  
	 You are playing a text adventure game. I will describe what you can perceive in the game. You write the single action you want to take in the game starting with >. Only reply with an action.\\ \tt For example:\\ \tt > open cupboard\\ \tt \\ \tt In addition to common actions, you can tedne, sulphy and derman.\\ \tt \\ \tt Your goal for this game is: Put the sioust in the reele, the pacia on the table and the inging on the menny.\\ \tt \\ \tt Once you have achieved your goal, write "> done" to end the game.\\ \tt \\ \tt You are in a bedroom now. There are a pillow, a prust, a wardrobe and a book. The pillow is on the prust. The wardrobe is closed. The book is on the floor. There is a passage to a unflued here. 
	  } 
	   } 
	   } 
	 & \\ 
 

    \theutterance \stepcounter{utterance}  

    & \multicolumn{4}{p{0.6\linewidth}}{\cellcolor[rgb]{0.8,1,0.9}{%
	\makecell[{{p{\linewidth}}}]{% 
	  \tt {\tiny [A$\rangle$GM]}  
	 <reasoning>\\ \tt Let's see. The goal is to put the sioust in the reele, the pacia on the table, and the inging on the menny. First, I need to figure out what these items are called in the game. The user mentioned that "tedne, sulphy, and derman" are additional actions, but I'm not sure how they relate. Maybe those are typos or specific verbs for the game? Let me check the initial description again.\\ \tt \\ \tt The current room has a pillow, a prust (maybe a typo for "prust" or "prust"?), a wardrobe, and a book. The pillow is on the prust. The wardrobe is closed. The book is on the floor. There's a passage to a unflued (maybe "unflued" is a room name? Or a typo for "unfurnished"?). So the first step is to understand the items and their locations.\\ \tt \\ \tt Wait, the goal items are sioust, pacia, inging. But in the current room, the items are pillow, prust, wardrobe, book. Maybe "sioust" is a typo for "pillow"? Or perhaps "prust" is the reele? Not sure. Let me think. The user might have made up some words, so maybe I need to map the goal items to the existing ones. Alternatively, maybe "sioust" is a hidden item, and I need to 
	  } 
	   } 
	   } 
	 & & \\ 
 

    \theutterance \stepcounter{utterance}  

    & & & \multicolumn{2}{p{0.3\linewidth}}{\cellcolor[rgb]{0.95,0.95,0.95}{%
	\makecell[{{p{\linewidth}}}]{% 
	  \tt {\tiny [GM$|$GM]}  
	 Round: 0 
	  } 
	   } 
	   } 
	 & & \\ 
 

    \theutterance \stepcounter{utterance}  

    & & & \multicolumn{2}{p{0.3\linewidth}}{\cellcolor[rgb]{0.95,0.95,0.95}{%
	\makecell[{{p{\linewidth}}}]{% 
	  \tt {\tiny [GM$|$GM]}  
	 command_tag_missing 
	  } 
	   } 
	   } 
	 & & \\ 
 

    \theutterance \stepcounter{utterance}  

    & & & \multicolumn{2}{p{0.3\linewidth}}{\cellcolor[rgb]{0.95,0.95,0.95}{%
	\makecell[{{p{\linewidth}}}]{% 
	  \tt {\tiny [GM$|$GM]}  
	 {'goal_states_achieved': [], 'game_successfully_finished': False} 
	  } 
	   } 
	   } 
	 & & \\ 
 

\end{supertabular}
}

\end{document}

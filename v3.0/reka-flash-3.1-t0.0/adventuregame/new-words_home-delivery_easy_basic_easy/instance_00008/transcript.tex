
\documentclass{article}
\usepackage{colortbl}
\usepackage{makecell}
\usepackage{multirow}
\usepackage{supertabular}

\begin{document}

\newcounter{utterance}

\twocolumn

{ \footnotesize  \setcounter{utterance}{1}
\setlength{\tabcolsep}{0pt}
\begin{supertabular}{c@{$\;$}|p{.15\linewidth}@{}p{.15\linewidth}p{.15\linewidth}p{.15\linewidth}p{.15\linewidth}p{.15\linewidth}}

    \# & $\;$A & \multicolumn{4}{c}{Game Master} & $\;\:$B\\
    \hline 

    \theutterance \stepcounter{utterance}  

    & & \multicolumn{4}{p{0.6\linewidth}}{\cellcolor[rgb]{0.9,0.9,0.9}{%
	\makecell[{{p{\linewidth}}}]{% 
	  \tt {\tiny [A$\langle$GM]}  
	 You are playing a text adventure game. I will describe what you can perceive in the game. You write the single action you want to take in the game starting with >. Only reply with an action.\\ \tt For example:\\ \tt > open cupboard\\ \tt \\ \tt In addition to common actions, you can vater. To vater is to make something physically unobstructed and uncovered, and make its contents available for use or interaction.\\ \tt \\ \tt Your goal for this game is: Put the nogia on the table, the potted plant in the aling and the peach in the aling.\\ \tt \\ \tt Once you have achieved your goal, write "> done" to end the game.\\ \tt \\ \tt You are in a bedroom now. There are a pillow, a bed, a wardrobe and a book. The pillow is on the bed. The wardrobe is closed. The book is on the floor. There is a passage to a hallway here. 
	  } 
	   } 
	   } 
	 & \\ 
 

    \theutterance \stepcounter{utterance}  

    & \multicolumn{4}{p{0.6\linewidth}}{\cellcolor[rgb]{0.8,1,0.9}{%
	\makecell[{{p{\linewidth}}}]{% 
	  \tt {\tiny [A$\rangle$GM]}  
	 <reasoning>\\ \tt Let's see. The goal is to put the nogia on the table, the potted plant in the aling, and the peach in the aling. First, I need to figure out where these items are. The current room is a bedroom with a pillow on the bed, a closed wardrobe, a book on the floor, and a passage to the hallway. The user mentioned "aling" but I don't see that in the initial description. Maybe "aling" is a typo? Or perhaps it's a specific location in the game that I haven't encountered yet. Let me check the initial description again.\\ \tt \\ \tt The user wrote: "Your goal for this game is: Put the nogia on the table, the potted plant in the aling and the peach in the aling." Wait, "aling" is mentioned twice. Maybe that's a typo for "aling" or "aling" is a specific place. Alternatively, maybe "aling" is supposed to be "aling" (like a balcony or another room). But in the current room description, there's a passage to the hallway. So perhaps the aling is another room connected via the hallway? Or maybe "aling" is a typo for "aling" (like a shelf or a specific spot). Hmm, this is confusing. Let me proceed step by step.\\ \tt \\ \tt First, the current room has a pillow, bed, wardrobe, book, and a passage to the hallway. The items needed 
	  } 
	   } 
	   } 
	 & & \\ 
 

    \theutterance \stepcounter{utterance}  

    & & & \multicolumn{2}{p{0.3\linewidth}}{\cellcolor[rgb]{0.95,0.95,0.95}{%
	\makecell[{{p{\linewidth}}}]{% 
	  \tt {\tiny [GM$|$GM]}  
	 Round: 0 
	  } 
	   } 
	   } 
	 & & \\ 
 

    \theutterance \stepcounter{utterance}  

    & & & \multicolumn{2}{p{0.3\linewidth}}{\cellcolor[rgb]{0.95,0.95,0.95}{%
	\makecell[{{p{\linewidth}}}]{% 
	  \tt {\tiny [GM$|$GM]}  
	 command_tag_missing 
	  } 
	   } 
	   } 
	 & & \\ 
 

    \theutterance \stepcounter{utterance}  

    & & & \multicolumn{2}{p{0.3\linewidth}}{\cellcolor[rgb]{0.95,0.95,0.95}{%
	\makecell[{{p{\linewidth}}}]{% 
	  \tt {\tiny [GM$|$GM]}  
	 {'goal_states_achieved': [], 'game_successfully_finished': False} 
	  } 
	   } 
	   } 
	 & & \\ 
 

\end{supertabular}
}

\end{document}

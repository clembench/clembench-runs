
\documentclass{article}
\usepackage{colortbl}
\usepackage{makecell}
\usepackage{multirow}
\usepackage{supertabular}

\begin{document}

\newcounter{utterance}

\twocolumn

{ \footnotesize  \setcounter{utterance}{1}
\setlength{\tabcolsep}{0pt}
\begin{supertabular}{c@{$\;$}|p{.15\linewidth}@{}p{.15\linewidth}p{.15\linewidth}p{.15\linewidth}p{.15\linewidth}p{.15\linewidth}}

    \# & $\;$A & \multicolumn{4}{c}{Game Master} & $\;\:$B\\
    \hline 

    \theutterance \stepcounter{utterance}  

    & & \multicolumn{4}{p{0.6\linewidth}}{\cellcolor[rgb]{0.9,0.9,0.9}{%
	\makecell[{{p{\linewidth}}}]{% 
	  \tt {\tiny [A$\langle$GM]}  
	 You are playing a text adventure game. I will describe what you can perceive in the game. You write the single action you want to take in the game starting with >. Only reply with an action.\\ \tt For example:\\ \tt > open cupboard\\ \tt \\ \tt In addition to common actions, you can maten. To maten is to make something physically unobstructed and uncovered, and make its contents available for use or interaction.\\ \tt \\ \tt Your goal for this game is: Put the uness on the table, the potted plant on the cally and the peach on the cally.\\ \tt \\ \tt Once you have achieved your goal, write "> done" to end the game.\\ \tt \\ \tt You are in a bedroom now. There are a pillow, a bed, a wardrobe and a book. The pillow is on the bed. The wardrobe is closed. The book is on the floor. There is a passage to a hallway here. 
	  } 
	   } 
	   } 
	 & \\ 
 

    \theutterance \stepcounter{utterance}  

    & \multicolumn{4}{p{0.6\linewidth}}{\cellcolor[rgb]{0.8,1,0.9}{%
	\makecell[{{p{\linewidth}}}]{% 
	  \tt {\tiny [A$\rangle$GM]}  
	 <reasoning>\\ \tt Let's see. The goal is to put the uness on the table, the potted plant on the cally, and the peach on the cally. First, I need to figure out what each of these items are. The user mentioned "uness" which might be a typo. Maybe "unes" or "uness" is a specific item in the game. But looking at the current room description, the items present are a pillow, bed, wardrobe, book, and a passage to a hallway. There's no table, potted plant, or peach mentioned here. So maybe I need to explore other areas.\\ \tt \\ \tt Wait, the current room is the bedroom. The passage leads to a hallway. Let me check if there's a table here. The description says there's a pillow on the bed, wardrobe closed, book on the floor, and a passage. No table. So perhaps the table is in the hallway or another room. But the user hasn't described other rooms yet. Maybe I need to go to the hallway first.\\ \tt \\ \tt So first action: go to the hallway. Let me move there. Once in the hallway, maybe there's a table, cally (maybe a typo for "couch" or "cubby"?), potted plant, and peach. But the user's goal mentions putting the potted plant on the cally and the peach on the cally. So maybe "cally" is a specific object here. 
	  } 
	   } 
	   } 
	 & & \\ 
 

    \theutterance \stepcounter{utterance}  

    & & & \multicolumn{2}{p{0.3\linewidth}}{\cellcolor[rgb]{0.95,0.95,0.95}{%
	\makecell[{{p{\linewidth}}}]{% 
	  \tt {\tiny [GM$|$GM]}  
	 Round: 0 
	  } 
	   } 
	   } 
	 & & \\ 
 

    \theutterance \stepcounter{utterance}  

    & & & \multicolumn{2}{p{0.3\linewidth}}{\cellcolor[rgb]{0.95,0.95,0.95}{%
	\makecell[{{p{\linewidth}}}]{% 
	  \tt {\tiny [GM$|$GM]}  
	 command_tag_missing 
	  } 
	   } 
	   } 
	 & & \\ 
 

    \theutterance \stepcounter{utterance}  

    & & & \multicolumn{2}{p{0.3\linewidth}}{\cellcolor[rgb]{0.95,0.95,0.95}{%
	\makecell[{{p{\linewidth}}}]{% 
	  \tt {\tiny [GM$|$GM]}  
	 {'goal_states_achieved': [], 'game_successfully_finished': False} 
	  } 
	   } 
	   } 
	 & & \\ 
 

\end{supertabular}
}

\end{document}

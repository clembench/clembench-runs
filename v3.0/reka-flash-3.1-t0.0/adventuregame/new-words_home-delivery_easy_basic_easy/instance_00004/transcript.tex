
\documentclass{article}
\usepackage{colortbl}
\usepackage{makecell}
\usepackage{multirow}
\usepackage{supertabular}

\begin{document}

\newcounter{utterance}

\twocolumn

{ \footnotesize  \setcounter{utterance}{1}
\setlength{\tabcolsep}{0pt}
\begin{supertabular}{c@{$\;$}|p{.15\linewidth}@{}p{.15\linewidth}p{.15\linewidth}p{.15\linewidth}p{.15\linewidth}p{.15\linewidth}}

    \# & $\;$A & \multicolumn{4}{c}{Game Master} & $\;\:$B\\
    \hline 

    \theutterance \stepcounter{utterance}  

    & & \multicolumn{4}{p{0.6\linewidth}}{\cellcolor[rgb]{0.9,0.9,0.9}{%
	\makecell[{{p{\linewidth}}}]{% 
	  \tt {\tiny [A$\langle$GM]}  
	 You are playing a text adventure game. I will describe what you can perceive in the game. You write the single action you want to take in the game starting with >. Only reply with an action.\\ \tt For example:\\ \tt > open cupboard\\ \tt \\ \tt In addition to common actions, you can remic. To remic is to make something physically unobstructed and uncovered, and make its contents available for use or interaction.\\ \tt \\ \tt Your goal for this game is: Put the gaticed on the table, the unbal on the table and the peach in the raticed.\\ \tt \\ \tt Once you have achieved your goal, write "> done" to end the game.\\ \tt \\ \tt You are in a bedroom now. There are a pillow, a bed, a wardrobe and a book. The pillow is on the bed. The wardrobe is closed. The book is on the floor. There is a passage to a hallway here. 
	  } 
	   } 
	   } 
	 & \\ 
 

    \theutterance \stepcounter{utterance}  

    & \multicolumn{4}{p{0.6\linewidth}}{\cellcolor[rgb]{0.8,1,0.9}{%
	\makecell[{{p{\linewidth}}}]{% 
	  \tt {\tiny [A$\rangle$GM]}  
	 <reasoning>\\ \tt Let's see. The goal is to put the gaticed on the table, the unbal on the table, and the peach in the raticed. First, I need to figure out what these items are. The names are a bit confusing. Maybe they're misspelled or coded? Let me think.\\ \tt \\ \tt "Put the gaticed on the table" – maybe "gaticed" is a typo. Could it be "gate" or "gated"? Or maybe "gaticed" is a container? Wait, the user mentioned that "raticed" is part of the goal, so maybe "raticed" is a rat trap? Like "rat" + "trapped"? So "raticed" might be a rat trap. Then "peach in the raticed" would mean putting the peach into the rat trap. That makes sense. \\ \tt \\ \tt Then "gaticed" – maybe "gate" is part of it? Or "gated" as in a gate? Or perhaps "gaticed" is a container like a box or a basket. Wait, the user also mentioned "unbal" – maybe "unbalanced"? But that doesn't fit. Alternatively, "unbal" could be a typo for "unbrella"? But that doesn't fit the context. Wait, the goal is to put "unbal" on the table. Maybe "unbal" is a container? 
	  } 
	   } 
	   } 
	 & & \\ 
 

    \theutterance \stepcounter{utterance}  

    & & & \multicolumn{2}{p{0.3\linewidth}}{\cellcolor[rgb]{0.95,0.95,0.95}{%
	\makecell[{{p{\linewidth}}}]{% 
	  \tt {\tiny [GM$|$GM]}  
	 Round: 0 
	  } 
	   } 
	   } 
	 & & \\ 
 

    \theutterance \stepcounter{utterance}  

    & & & \multicolumn{2}{p{0.3\linewidth}}{\cellcolor[rgb]{0.95,0.95,0.95}{%
	\makecell[{{p{\linewidth}}}]{% 
	  \tt {\tiny [GM$|$GM]}  
	 command_tag_missing 
	  } 
	   } 
	   } 
	 & & \\ 
 

    \theutterance \stepcounter{utterance}  

    & & & \multicolumn{2}{p{0.3\linewidth}}{\cellcolor[rgb]{0.95,0.95,0.95}{%
	\makecell[{{p{\linewidth}}}]{% 
	  \tt {\tiny [GM$|$GM]}  
	 {'goal_states_achieved': [], 'game_successfully_finished': False} 
	  } 
	   } 
	   } 
	 & & \\ 
 

\end{supertabular}
}

\end{document}


\documentclass{article}
\usepackage{colortbl}
\usepackage{makecell}
\usepackage{multirow}
\usepackage{supertabular}

\begin{document}

\newcounter{utterance}

\twocolumn

{ \footnotesize  \setcounter{utterance}{1}
\setlength{\tabcolsep}{0pt}
\begin{supertabular}{c@{$\;$}|p{.15\linewidth}@{}p{.15\linewidth}p{.15\linewidth}p{.15\linewidth}p{.15\linewidth}p{.15\linewidth}}

    \# & $\;$A & \multicolumn{4}{c}{Game Master} & $\;\:$B\\
    \hline 

    \theutterance \stepcounter{utterance}  

    & & \multicolumn{4}{p{0.6\linewidth}}{\cellcolor[rgb]{0.9,0.9,0.9}{%
	\makecell[{{p{\linewidth}}}]{% 
	  \tt {\tiny [A$\langle$GM]}  
	 You are playing a text adventure game. I will describe what you can perceive in the game. You write the single action you want to take in the game starting with >. Only reply with an action.\\ \tt For example:\\ \tt > open cupboard\\ \tt \\ \tt In addition to common actions, you can propress, atifory, alful, adcry, spidene and pnestay.\\ \tt \\ \tt Your goal for this game is: Make the oubiced ecastic, the prachon dedulag and the mindly ropile.\\ \tt \\ \tt Once you have achieved your goal, write "> done" to end the game.\\ \tt \\ \tt You are in a sinark now. There are a clednee, a prachon, a epteria, a mindly, a ionity and a uncepist. There is a passage to a conader here. 
	  } 
	   } 
	   } 
	 & \\ 
 

    \theutterance \stepcounter{utterance}  

    & \multicolumn{4}{p{0.6\linewidth}}{\cellcolor[rgb]{0.8,1,0.9}{%
	\makecell[{{p{\linewidth}}}]{% 
	  \tt {\tiny [A$\rangle$GM]}  
	 <reasoning>\\ \tt Let's try to figure out how to approach this game. The goal is to make the "oubiced ecastic", "prachon dedulag", and "mindly ropile". Those names look a bit scrambled. Maybe they're anagrams? Let me check.\\ \tt \\ \tt First, "oubiced ecastic". Let's break it down. "Oubiced" – maybe "decubito" or something? Wait, "oubiced" could be rearranged. Let's see: O-U-B-I-C-E-D. Maybe "decubito" is a Latin term for lying down, but not sure. Alternatively, "ecastic" – maybe "ecstatic"? So "oubiced ecastic" could be "decubito ecstatic"? Not sure. Alternatively, maybe "oubiced" is "decubito" backwards? Or maybe "oubiced" is "decubito" misspelled. Alternatively, maybe "oubiced" is "decubited" (past tense of decubitus, which is lying down). So "decubited ecstatic"? Not sure. Maybe the goal is to make something ecstatic that's decubited? Or maybe the anagram is "decubited ecstatic" but that's not making sense. Alternatively, maybe "oubiced" is "decubito" and "ecastic" is "ecstatic", so combining them as "decubito ecstatic" but I'm not 
	  } 
	   } 
	   } 
	 & & \\ 
 

    \theutterance \stepcounter{utterance}  

    & & & \multicolumn{2}{p{0.3\linewidth}}{\cellcolor[rgb]{0.95,0.95,0.95}{%
	\makecell[{{p{\linewidth}}}]{% 
	  \tt {\tiny [GM$|$GM]}  
	 Round: 0 
	  } 
	   } 
	   } 
	 & & \\ 
 

    \theutterance \stepcounter{utterance}  

    & & & \multicolumn{2}{p{0.3\linewidth}}{\cellcolor[rgb]{0.95,0.95,0.95}{%
	\makecell[{{p{\linewidth}}}]{% 
	  \tt {\tiny [GM$|$GM]}  
	 command_tag_missing 
	  } 
	   } 
	   } 
	 & & \\ 
 

    \theutterance \stepcounter{utterance}  

    & & & \multicolumn{2}{p{0.3\linewidth}}{\cellcolor[rgb]{0.95,0.95,0.95}{%
	\makecell[{{p{\linewidth}}}]{% 
	  \tt {\tiny [GM$|$GM]}  
	 {'goal_states_achieved': [], 'game_successfully_finished': False} 
	  } 
	   } 
	   } 
	 & & \\ 
 

\end{supertabular}
}

\end{document}

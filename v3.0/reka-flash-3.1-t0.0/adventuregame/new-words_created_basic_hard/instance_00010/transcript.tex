
\documentclass{article}
\usepackage{colortbl}
\usepackage{makecell}
\usepackage{multirow}
\usepackage{supertabular}

\begin{document}

\newcounter{utterance}

\twocolumn

{ \footnotesize  \setcounter{utterance}{1}
\setlength{\tabcolsep}{0pt}
\begin{supertabular}{c@{$\;$}|p{.15\linewidth}@{}p{.15\linewidth}p{.15\linewidth}p{.15\linewidth}p{.15\linewidth}p{.15\linewidth}}

    \# & $\;$A & \multicolumn{4}{c}{Game Master} & $\;\:$B\\
    \hline 

    \theutterance \stepcounter{utterance}  

    & & \multicolumn{4}{p{0.6\linewidth}}{\cellcolor[rgb]{0.9,0.9,0.9}{%
	\makecell[{{p{\linewidth}}}]{% 
	  \tt {\tiny [A$\langle$GM]}  
	 You are playing a text adventure game. I will describe what you can perceive in the game. You write the single action you want to take in the game starting with >. Only reply with an action.\\ \tt For example:\\ \tt > open cupboard\\ \tt \\ \tt In addition to common actions, you can udial, aterous, ingnus, ivess, ceorid and unmastic.\\ \tt \\ \tt Your goal for this game is: Make the pholoiced connone, the aeval proggiced and the rheably exsem.\\ \tt \\ \tt Once you have achieved your goal, write "> done" to end the game.\\ \tt \\ \tt You are in a peroury now. There are a versch, a aeval, a oidess, a pholoiced, a eical and a shromen. There is a passage to a erove here. 
	  } 
	   } 
	   } 
	 & \\ 
 

    \theutterance \stepcounter{utterance}  

    & \multicolumn{4}{p{0.6\linewidth}}{\cellcolor[rgb]{0.8,1,0.9}{%
	\makecell[{{p{\linewidth}}}]{% 
	  \tt {\tiny [A$\rangle$GM]}  
	 <reasoning>\\ \tt Let's try to figure out how to solve this game. The goal is to make the "pholoiced connone", "aeval proggiced", and "rheably exsem". Those names don't make sense, so maybe they're made up words or abbreviations. Let's break them down.\\ \tt \\ \tt First, the current room has several items: versch, aeval, oidess, pholoiced, eical, and shromen. There's also a passage to an erove. The allowed actions include common ones like open, take, go, plus some weird ones: udial, aterous, ingnus, ivess, ceorid, unmastic.\\ \tt \\ \tt Hmm. Maybe each of the goal phrases corresponds to combining or manipulating these items. Let's look at the goal phrases again:\\ \tt \\ \tt 1. pholoiced connone\\ \tt 2. aeval proggiced\\ \tt 3. rheably exsem\\ \tt \\ \tt Wait, "pholoiced" is one of the items here. So maybe "pholoiced connone" means combining pholoiced with connone. But where does connone come from? Maybe "connone" is a combination of other words? Or perhaps "connone" is a typo or mishearing. Alternatively, maybe each part of the goal is formed by two words from the items. Let's check the items again: versch, aeval, oidess, pholoiced, eical, sh 
	  } 
	   } 
	   } 
	 & & \\ 
 

    \theutterance \stepcounter{utterance}  

    & & & \multicolumn{2}{p{0.3\linewidth}}{\cellcolor[rgb]{0.95,0.95,0.95}{%
	\makecell[{{p{\linewidth}}}]{% 
	  \tt {\tiny [GM$|$GM]}  
	 Round: 0 
	  } 
	   } 
	   } 
	 & & \\ 
 

    \theutterance \stepcounter{utterance}  

    & & & \multicolumn{2}{p{0.3\linewidth}}{\cellcolor[rgb]{0.95,0.95,0.95}{%
	\makecell[{{p{\linewidth}}}]{% 
	  \tt {\tiny [GM$|$GM]}  
	 command_tag_missing 
	  } 
	   } 
	   } 
	 & & \\ 
 

    \theutterance \stepcounter{utterance}  

    & & & \multicolumn{2}{p{0.3\linewidth}}{\cellcolor[rgb]{0.95,0.95,0.95}{%
	\makecell[{{p{\linewidth}}}]{% 
	  \tt {\tiny [GM$|$GM]}  
	 {'goal_states_achieved': [], 'game_successfully_finished': False} 
	  } 
	   } 
	   } 
	 & & \\ 
 

\end{supertabular}
}

\end{document}

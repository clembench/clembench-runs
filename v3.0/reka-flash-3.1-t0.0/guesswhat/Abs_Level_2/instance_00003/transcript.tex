
\documentclass{article}
\usepackage{colortbl}
\usepackage{makecell}
\usepackage{multirow}
\usepackage{supertabular}

\begin{document}

\newcounter{utterance}

\twocolumn

{ \footnotesize  \setcounter{utterance}{1}
\setlength{\tabcolsep}{0pt}
\begin{supertabular}{c@{$\;$}|p{.15\linewidth}@{}p{.15\linewidth}p{.15\linewidth}p{.15\linewidth}p{.15\linewidth}p{.15\linewidth}}

    \# & $\;$A & \multicolumn{4}{c}{Game Master} & $\;\:$B\\
    \hline 

    \theutterance \stepcounter{utterance}  

    & & \multicolumn{4}{p{0.6\linewidth}}{\cellcolor[rgb]{0.9,0.9,0.9}{%
	\makecell[{{p{\linewidth}}}]{% 
	  \tt {\tiny [A$\langle$GM]}  
	 Let's play a guessing game. Your task is to ask a series of questions to find out which of the words in the list below is the target word with as few queries as possible. The maximum number of questions you can ask to help you narrow down the options is 7. Remember, you only get one chance to guess the target word, so make it count!\\ \tt \\ \tt The rules of the game are: \\ \tt \\ \tt 1. Only ask one question per turn. DO NOT ASK MORE THAN ONE QUESTION.\\ \tt \\ \tt 2. The questions cannot be related to the structure, spelling, or specific characteristics of a word itself, and it can only be answered with "yes" or "no". \\ \tt \\ \tt 3. Use the following format for asking questions and DO NOT ADD ANY TEXT OTHER THAN THIS: \\ \tt \\ \tt QUESTION: <your question>\\ \tt \\ \tt 4. You have only one opportunity to guess the target word. To make your guess, use the following format and DO NOT ADD ANY TEXT OTHER THAN THIS: \\ \tt \\ \tt GUESS: <target word>\\ \tt \\ \tt Remember you cannot ask for a hint. Make your question brief. Let’s start!\\ \tt \\ \tt List of candidate words: ['lunchtime', 'suppertime', 'breakfast', 'teatime', 'anorexia', 'bulimia', 'sociopath', 'psychopath'] 
	  } 
	   } 
	   } 
	 & \\ 
 

    \theutterance \stepcounter{utterance}  

    & \multicolumn{4}{p{0.6\linewidth}}{\cellcolor[rgb]{0.8,1,0.9}{%
	\makecell[{{p{\linewidth}}}]{% 
	  \tt {\tiny [A$\rangle$GM]}  
	 <reasoning>\\ \tt Let's see. I need to figure out which word from the list is the target using up to 7 yes/no questions. The list is ['lunchtime','suppertime', 'breakfast', 'teatime', 'anorexia', 'bulimia','sociopath', 'psychopath'].\\ \tt \\ \tt First, I should categorize these words. Let's look at them:\\ \tt \\ \tt - The first four are times of day when meals are eaten: lunchtime, suppertime, breakfast, teatime.\\ \tt - The next four are mental health conditions: anorexia, bulimia, sociopath, psychopath.\\ \tt \\ \tt So maybe the first step is to determine if the target is a meal time or a mental health condition. That's a big split. Let's ask a question that can separate these two groups.\\ \tt \\ \tt Question 1: Is the target a time of day when a meal is typically eaten?\\ \tt \\ \tt If the answer is yes, then the target is one of the first four. If no, then it's one of the last four. Let's go with that.\\ \tt \\ \tt Assuming the answer is yes (since there are four meal times and four conditions, but maybe the user wants to split them). Wait, but the answer could be either. Let's proceed.\\ \tt \\ \tt If the answer is yes, then the next step is to narrow down among the four meal times. Let's see: breakfast, lunchtime, suppertime, teatime 
	  } 
	   } 
	   } 
	 & & \\ 
 

    \theutterance \stepcounter{utterance}  

    & & & \multicolumn{2}{p{0.3\linewidth}}{\cellcolor[rgb]{0.95,0.95,0.95}{%
	\makecell[{{p{\linewidth}}}]{% 
	  \tt {\tiny [GM$|$GM]}  
	 Invalid format. Guesser must use the form 'QUESTION: ' or 'GUESS: '. 
	  } 
	   } 
	   } 
	 & & \\ 
 

    \theutterance \stepcounter{utterance}  

    & & & \multicolumn{2}{p{0.3\linewidth}}{\cellcolor[rgb]{0.95,0.95,0.95}{%
	\makecell[{{p{\linewidth}}}]{% 
	  \tt {\tiny [GM$|$GM]}  
	 abort game 
	  } 
	   } 
	   } 
	 & & \\ 
 

\end{supertabular}
}

\end{document}

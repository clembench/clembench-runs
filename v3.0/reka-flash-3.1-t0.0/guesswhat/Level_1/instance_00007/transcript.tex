
\documentclass{article}
\usepackage{colortbl}
\usepackage{makecell}
\usepackage{multirow}
\usepackage{supertabular}

\begin{document}

\newcounter{utterance}

\twocolumn

{ \footnotesize  \setcounter{utterance}{1}
\setlength{\tabcolsep}{0pt}
\begin{supertabular}{c@{$\;$}|p{.15\linewidth}@{}p{.15\linewidth}p{.15\linewidth}p{.15\linewidth}p{.15\linewidth}p{.15\linewidth}}

    \# & $\;$A & \multicolumn{4}{c}{Game Master} & $\;\:$B\\
    \hline 

    \theutterance \stepcounter{utterance}  

    & & \multicolumn{4}{p{0.6\linewidth}}{\cellcolor[rgb]{0.9,0.9,0.9}{%
	\makecell[{{p{\linewidth}}}]{% 
	  \tt {\tiny [A$\langle$GM]}  
	 Let's play a guessing game. Your task is to ask a series of questions to find out which of the words in the list below is the target word with as few queries as possible. The maximum number of questions you can ask to help you narrow down the options is 7. Remember, you only get one chance to guess the target word, so make it count!\\ \tt \\ \tt The rules of the game are: \\ \tt \\ \tt 1. Only ask one question per turn. DO NOT ASK MORE THAN ONE QUESTION.\\ \tt \\ \tt 2. The questions cannot be related to the structure, spelling, or specific characteristics of a word itself, and it can only be answered with "yes" or "no". \\ \tt \\ \tt 3. Use the following format for asking questions and DO NOT ADD ANY TEXT OTHER THAN THIS: \\ \tt \\ \tt QUESTION: <your question>\\ \tt \\ \tt 4. You have only one opportunity to guess the target word. To make your guess, use the following format and DO NOT ADD ANY TEXT OTHER THAN THIS: \\ \tt \\ \tt GUESS: <target word>\\ \tt \\ \tt Remember you cannot ask for a hint. Make your question brief. Let’s start!\\ \tt \\ \tt List of candidate words: ['Chrysanthemum', 'Dandelion', 'Sturgeon', 'Trout', 'Seeds', 'Edger', 'Rosemary', 'Chives'] 
	  } 
	   } 
	   } 
	 & \\ 
 

    \theutterance \stepcounter{utterance}  

    & \multicolumn{4}{p{0.6\linewidth}}{\cellcolor[rgb]{0.8,1,0.9}{%
	\makecell[{{p{\linewidth}}}]{% 
	  \tt {\tiny [A$\rangle$GM]}  
	 <reasoning>\\ \tt Let's see. I need to figure out which word from the list is the target using up to 7 yes/no questions. The list is ['Chrysanthemum', 'Dandelion', 'Sturgeon', 'Trout', 'Seeds', 'Edger', 'Rosemary', 'Chives'].\\ \tt \\ \tt First, I should look for common categories or groupings. Let's see. Some of these are plants, others are fish, tools, or parts of plants. Let's categorize them:\\ \tt \\ \tt Plants: Chrysanthemum, Dandelion, Rosemary, Chives. These are all types of plants or herbs.\\ \tt Fish: Sturgeon, Trout. Both are fish species.\\ \tt Seeds: Seeds is a general term, maybe referring to plant seeds.\\ \tt Edger: That's a gardening tool, used for edging lawns.\\ \tt \\ \tt So maybe the first question can be about categorizing. Let's ask if the target is a plant. If yes, then narrow down from the plant group. If no, check if it's a fish, tool, or seeds.\\ \tt \\ \tt Wait, but the rules say questions can't be about the structure, spelling, or specific characteristics of the word itself. So I can't ask about letters or syllables. So I have to ask about the nature of the word. For example, "Is the target a type of plant?" That's allowed because it's about the category, not the word's 
	  } 
	   } 
	   } 
	 & & \\ 
 

    \theutterance \stepcounter{utterance}  

    & & & \multicolumn{2}{p{0.3\linewidth}}{\cellcolor[rgb]{0.95,0.95,0.95}{%
	\makecell[{{p{\linewidth}}}]{% 
	  \tt {\tiny [GM$|$GM]}  
	 Invalid format. Guesser must use the form 'QUESTION: ' or 'GUESS: '. 
	  } 
	   } 
	   } 
	 & & \\ 
 

    \theutterance \stepcounter{utterance}  

    & & & \multicolumn{2}{p{0.3\linewidth}}{\cellcolor[rgb]{0.95,0.95,0.95}{%
	\makecell[{{p{\linewidth}}}]{% 
	  \tt {\tiny [GM$|$GM]}  
	 abort game 
	  } 
	   } 
	   } 
	 & & \\ 
 

\end{supertabular}
}

\end{document}

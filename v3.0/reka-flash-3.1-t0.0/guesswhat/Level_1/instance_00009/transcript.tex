
\documentclass{article}
\usepackage{colortbl}
\usepackage{makecell}
\usepackage{multirow}
\usepackage{supertabular}

\begin{document}

\newcounter{utterance}

\twocolumn

{ \footnotesize  \setcounter{utterance}{1}
\setlength{\tabcolsep}{0pt}
\begin{supertabular}{c@{$\;$}|p{.15\linewidth}@{}p{.15\linewidth}p{.15\linewidth}p{.15\linewidth}p{.15\linewidth}p{.15\linewidth}}

    \# & $\;$A & \multicolumn{4}{c}{Game Master} & $\;\:$B\\
    \hline 

    \theutterance \stepcounter{utterance}  

    & & \multicolumn{4}{p{0.6\linewidth}}{\cellcolor[rgb]{0.9,0.9,0.9}{%
	\makecell[{{p{\linewidth}}}]{% 
	  \tt {\tiny [A$\langle$GM]}  
	 Let's play a guessing game. Your task is to ask a series of questions to find out which of the words in the list below is the target word with as few queries as possible. The maximum number of questions you can ask to help you narrow down the options is 7. Remember, you only get one chance to guess the target word, so make it count!\\ \tt \\ \tt The rules of the game are: \\ \tt \\ \tt 1. Only ask one question per turn. DO NOT ASK MORE THAN ONE QUESTION.\\ \tt \\ \tt 2. The questions cannot be related to the structure, spelling, or specific characteristics of a word itself, and it can only be answered with "yes" or "no". \\ \tt \\ \tt 3. Use the following format for asking questions and DO NOT ADD ANY TEXT OTHER THAN THIS: \\ \tt \\ \tt QUESTION: <your question>\\ \tt \\ \tt 4. You have only one opportunity to guess the target word. To make your guess, use the following format and DO NOT ADD ANY TEXT OTHER THAN THIS: \\ \tt \\ \tt GUESS: <target word>\\ \tt \\ \tt Remember you cannot ask for a hint. Make your question brief. Let’s start!\\ \tt \\ \tt List of candidate words: ['Running', 'Track', 'Cornerback', 'Coach', 'Scarf', 'Gloves', 'Heirlooms', 'Clothes'] 
	  } 
	   } 
	   } 
	 & \\ 
 

    \theutterance \stepcounter{utterance}  

    & \multicolumn{4}{p{0.6\linewidth}}{\cellcolor[rgb]{0.8,1,0.9}{%
	\makecell[{{p{\linewidth}}}]{% 
	  \tt {\tiny [A$\rangle$GM]}  
	 <reasoning>\\ \tt Let's see. I need to figure out which word from the list is the target using up to 7 yes/no questions. The list is: Running, Track, Cornerback, Coach, Scarf, Gloves, Heirlooms, Clothes.\\ \tt \\ \tt First, I should categorize these words to see if there's a common theme. Let's go through each one:\\ \tt \\ \tt - Running: An action, related to sports, maybe athletics.\\ \tt - Track: Could be a sports track or a railway track. But in sports, it's where running happens.\\ \tt - Cornerback: A position in American football.\\ \tt - Coach: Someone who trains athletes.\\ \tt - Scarf: Clothing item, worn around the neck.\\ \tt - Gloves: Clothing, worn on hands.\\ \tt - Heirlooms: Items passed down through generations, often valuable.\\ \tt - Clothes: General term for clothing.\\ \tt \\ \tt Hmm, so some are sports-related (Running, Track, Cornerback, Coach), some are clothing items (Scarf, Gloves, Clothes), and Heirlooms are more about inheritance. Maybe I can split them into categories first.\\ \tt \\ \tt Let me start with a broad question. Let's see, if I ask if the target is a type of clothing. That would include Scarf, Gloves, Clothes. If the answer is yes, then I can narrow down to those three. If no, then it's either sports-related (Running, Track, Cornerback, Coach) or Heirlooms.\\ \tt \\ \tt Wait 
	  } 
	   } 
	   } 
	 & & \\ 
 

    \theutterance \stepcounter{utterance}  

    & & & \multicolumn{2}{p{0.3\linewidth}}{\cellcolor[rgb]{0.95,0.95,0.95}{%
	\makecell[{{p{\linewidth}}}]{% 
	  \tt {\tiny [GM$|$GM]}  
	 Invalid format. Guesser must use the form 'QUESTION: ' or 'GUESS: '. 
	  } 
	   } 
	   } 
	 & & \\ 
 

    \theutterance \stepcounter{utterance}  

    & & & \multicolumn{2}{p{0.3\linewidth}}{\cellcolor[rgb]{0.95,0.95,0.95}{%
	\makecell[{{p{\linewidth}}}]{% 
	  \tt {\tiny [GM$|$GM]}  
	 abort game 
	  } 
	   } 
	   } 
	 & & \\ 
 

\end{supertabular}
}

\end{document}

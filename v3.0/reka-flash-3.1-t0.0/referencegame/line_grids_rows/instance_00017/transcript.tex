
\documentclass{article}
\usepackage{colortbl}
\usepackage{makecell}
\usepackage{multirow}
\usepackage{supertabular}

\begin{document}

\newcounter{utterance}

\twocolumn

{ \footnotesize  \setcounter{utterance}{1}
\setlength{\tabcolsep}{0pt}
\begin{supertabular}{c@{$\;$}|p{.15\linewidth}@{}p{.15\linewidth}p{.15\linewidth}p{.15\linewidth}p{.15\linewidth}p{.15\linewidth}}

    \# & $\;$A & \multicolumn{4}{c}{Game Master} & $\;\:$B\\
    \hline 

    \theutterance \stepcounter{utterance}  

    & & \multicolumn{4}{p{0.6\linewidth}}{\cellcolor[rgb]{0.9,0.9,0.9}{%
	\makecell[{{p{\linewidth}}}]{% 
	  \tt {\tiny [A$\langle$GM]}  
	 You are given three grids, where each of them is 5 by 5 in size.\\ \tt Grids have empty cells marked with "▢" and filled cells marked with "X".\\ \tt Your task is to generate a referring expression that best describes the target grid while distinguishing it from the two other distractor grids.\\ \tt The first grid is the target grid, and the following two grids are the distractors.\\ \tt \\ \tt Target grid:\\ \tt \\ \tt □ X X □ X\\ \tt □ X X □ X\\ \tt □ X X □ X\\ \tt □ X X □ X\\ \tt □ X X □ X\\ \tt \\ \tt Distractor grid 1:\\ \tt \\ \tt □ □ X X X\\ \tt □ □ X X X\\ \tt □ □ X X X\\ \tt □ □ X X X\\ \tt □ □ X X X\\ \tt \\ \tt Distractor grid 2:\\ \tt \\ \tt X X X X X\\ \tt □ □ □ □ □\\ \tt □ □ □ □ □\\ \tt X X X X X\\ \tt □ □ □ □ □\\ \tt \\ \tt Instruction: Describe the target grid.\\ \tt Generate the referring expression starting with the tag "Expression: " for the given target grid. Omit any other text. 
	  } 
	   } 
	   } 
	 & \\ 
 

    \theutterance \stepcounter{utterance}  

    & \multicolumn{4}{p{0.6\linewidth}}{\cellcolor[rgb]{0.8,1,0.9}{%
	\makecell[{{p{\linewidth}}}]{% 
	  \tt {\tiny [A$\rangle$GM]}  
	 <reasoning>\\ \tt Let's tackle this problem. I need to generate a referring expression that describes the target grid and distinguishes it from the two distractor grids. Let's start by looking at each grid carefully.\\ \tt \\ \tt First, the target grid. Each row is □ X X □ X. So, each row has five cells. The pattern here is that the second and third cells are filled (X), and the first, fourth, and fifth are empty (□). So, each row is like □ X X □ X. That's a repeating pattern every two cells? Wait, no. Let me check each row again. Yes, each row is the same: positions 2 and 3 are X, others are □. So, the target grid has a horizontal pattern where in every row, the middle two cells are filled. So, it's like a horizontal bar in each row, but only two cells wide. So, maybe the description could mention that each row has two Xs in the middle.\\ \tt \\ \tt Now, looking at Distractor grid 1. Each row is □ □ X X X. So, each row starts with two empty cells, then three filled. So, a horizontal block of three Xs starting at the third column. So, each row has a horizontal block of three Xs. That's different from the target's two Xs in the middle.\\ \tt \\ \tt Distractor grid 2 is a bit more complex. 
	  } 
	   } 
	   } 
	 & & \\ 
 

    \theutterance \stepcounter{utterance}  

    & & & \multicolumn{2}{p{0.3\linewidth}}{\cellcolor[rgb]{0.95,0.95,0.95}{%
	\makecell[{{p{\linewidth}}}]{% 
	  \tt {\tiny [GM$|$GM]}  
	 Invalid generated expression 
	  } 
	   } 
	   } 
	 & & \\ 
 

\end{supertabular}
}

\end{document}


\documentclass{article}
\usepackage{colortbl}
\usepackage{makecell}
\usepackage{multirow}
\usepackage{supertabular}

\begin{document}

\newcounter{utterance}

\twocolumn

{ \footnotesize  \setcounter{utterance}{1}
\setlength{\tabcolsep}{0pt}
\begin{supertabular}{c@{$\;$}|p{.15\linewidth}@{}p{.15\linewidth}p{.15\linewidth}p{.15\linewidth}p{.15\linewidth}p{.15\linewidth}}

    \# & $\;$A & \multicolumn{4}{c}{Game Master} & $\;\:$B\\
    \hline 

    \theutterance \stepcounter{utterance}  

    & & \multicolumn{4}{p{0.6\linewidth}}{\cellcolor[rgb]{0.9,0.9,0.9}{%
	\makecell[{{p{\linewidth}}}]{% 
	  \tt {\tiny [A$\langle$GM]}  
	 You are given three grids, where each of them is 5 by 5 in size.\\ \tt Grids have empty cells marked with "▢" and filled cells marked with "X".\\ \tt Your task is to generate a referring expression that best describes the target grid while distinguishing it from the two other distractor grids.\\ \tt The first grid is the target grid, and the following two grids are the distractors.\\ \tt \\ \tt Target grid:\\ \tt \\ \tt □ □ X X X\\ \tt □ □ X X X\\ \tt □ □ X X X\\ \tt □ □ X X X\\ \tt □ □ X X X\\ \tt \\ \tt Distractor grid 1:\\ \tt \\ \tt □ □ □ X X\\ \tt □ □ □ X X\\ \tt □ □ □ X X\\ \tt □ □ □ X X\\ \tt □ □ □ X X\\ \tt \\ \tt Distractor grid 2:\\ \tt \\ \tt X □ □ X X\\ \tt X □ □ X X\\ \tt X □ □ X X\\ \tt X □ □ X X\\ \tt X □ □ X X\\ \tt \\ \tt Instruction: Describe the target grid.\\ \tt Generate the referring expression starting with the tag "Expression: " for the given target grid. Omit any other text. 
	  } 
	   } 
	   } 
	 & \\ 
 

    \theutterance \stepcounter{utterance}  

    & \multicolumn{4}{p{0.6\linewidth}}{\cellcolor[rgb]{0.8,1,0.9}{%
	\makecell[{{p{\linewidth}}}]{% 
	  \tt {\tiny [A$\rangle$GM]}  
	 <reasoning>\\ \tt Let's tackle this problem. I need to generate a referring expression that describes the target grid and distinguishes it from the two distractor grids. Let's start by looking at the grids.\\ \tt \\ \tt The target grid is all filled in the rightmost three columns. Each row has three X's followed by two □. So, it's a solid block of X's on the right side. Let me check the distractors.\\ \tt \\ \tt Distractor grid 1 has the rightmost two columns filled. Each row has two X's and three □. So, it's narrower than the target. Distractor grid 2 has the leftmost column filled and the rightmost two columns filled. Each row starts with an X, then two □, then two X's. So, it's a combination of left and right filled areas.\\ \tt \\ \tt The key differences are the positions and numbers of filled cells. The target has three X's in a vertical block on the right. Distractor 1 has two X's on the right, and Distractor 2 has X's on the left and right. So, the referring expression should highlight the three X's in the rightmost columns, maybe specifying that it's a vertical block spanning all rows. Also, mention that the left side is empty. Alternatively, compare to the distractors: unlike the others which have X's in different positions or fewer columns.\\ \tt \\ \tt Possible phrases: "A 5x5 grid with three 
	  } 
	   } 
	   } 
	 & & \\ 
 

    \theutterance \stepcounter{utterance}  

    & & & \multicolumn{2}{p{0.3\linewidth}}{\cellcolor[rgb]{0.95,0.95,0.95}{%
	\makecell[{{p{\linewidth}}}]{% 
	  \tt {\tiny [GM$|$GM]}  
	 Invalid generated expression 
	  } 
	   } 
	   } 
	 & & \\ 
 

\end{supertabular}
}

\end{document}

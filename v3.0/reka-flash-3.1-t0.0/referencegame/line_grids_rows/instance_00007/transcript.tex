
\documentclass{article}
\usepackage{colortbl}
\usepackage{makecell}
\usepackage{multirow}
\usepackage{supertabular}

\begin{document}

\newcounter{utterance}

\twocolumn

{ \footnotesize  \setcounter{utterance}{1}
\setlength{\tabcolsep}{0pt}
\begin{supertabular}{c@{$\;$}|p{.15\linewidth}@{}p{.15\linewidth}p{.15\linewidth}p{.15\linewidth}p{.15\linewidth}p{.15\linewidth}}

    \# & $\;$A & \multicolumn{4}{c}{Game Master} & $\;\:$B\\
    \hline 

    \theutterance \stepcounter{utterance}  

    & & \multicolumn{4}{p{0.6\linewidth}}{\cellcolor[rgb]{0.9,0.9,0.9}{%
	\makecell[{{p{\linewidth}}}]{% 
	  \tt {\tiny [A$\langle$GM]}  
	 You are given three grids, where each of them is 5 by 5 in size.\\ \tt Grids have empty cells marked with "▢" and filled cells marked with "X".\\ \tt Your task is to generate a referring expression that best describes the target grid while distinguishing it from the two other distractor grids.\\ \tt The first grid is the target grid, and the following two grids are the distractors.\\ \tt \\ \tt Target grid:\\ \tt \\ \tt X □ □ X X\\ \tt X □ □ X X\\ \tt X □ □ X X\\ \tt X □ □ X X\\ \tt X □ □ X X\\ \tt \\ \tt Distractor grid 1:\\ \tt \\ \tt □ □ □ X X\\ \tt □ □ □ X X\\ \tt □ □ □ X X\\ \tt □ □ □ X X\\ \tt □ □ □ X X\\ \tt \\ \tt Distractor grid 2:\\ \tt \\ \tt X X □ X X\\ \tt X X □ X X\\ \tt X X □ X X\\ \tt X X □ X X\\ \tt X X □ X X\\ \tt \\ \tt Instruction: Describe the target grid.\\ \tt Generate the referring expression starting with the tag "Expression: " for the given target grid. Omit any other text. 
	  } 
	   } 
	   } 
	 & \\ 
 

    \theutterance \stepcounter{utterance}  

    & \multicolumn{4}{p{0.6\linewidth}}{\cellcolor[rgb]{0.8,1,0.9}{%
	\makecell[{{p{\linewidth}}}]{% 
	  \tt {\tiny [A$\rangle$GM]}  
	 <reasoning>\\ \tt Let's tackle this problem. I need to generate a referring expression that describes the target grid and distinguishes it from the two distractor grids. Let me start by looking at all three grids carefully.\\ \tt \\ \tt First, the target grid. Each row has X in the first two columns and □ in the third, then X in the last two. So the pattern is XX□XX for each row. All five rows are the same. So the structure is two Xs at the start, a space in the middle, and two Xs at the end, repeated five times vertically.\\ \tt \\ \tt Now, Distractor grid 1. Each row here has three □s followed by two Xs. So □□□XX in every row. That's different from the target because the Xs are at the end, not the beginning, and there's a longer stretch of empty cells in the middle.\\ \tt \\ \tt Distractor grid 2 has three Xs at the start, then a □, then two Xs. So XXX□XX in each row. Here, the Xs are clustered more towards the left, with a single empty cell in the middle, and the Xs continue to the end. This is different from the target's two Xs at the start, a middle empty, then two Xs.\\ \tt \\ \tt So the key differences between the target and the distractors are:\\ \tt \\ \tt 1. Position of Xs: Target has two Xs 
	  } 
	   } 
	   } 
	 & & \\ 
 

    \theutterance \stepcounter{utterance}  

    & & & \multicolumn{2}{p{0.3\linewidth}}{\cellcolor[rgb]{0.95,0.95,0.95}{%
	\makecell[{{p{\linewidth}}}]{% 
	  \tt {\tiny [GM$|$GM]}  
	 Invalid generated expression 
	  } 
	   } 
	   } 
	 & & \\ 
 

\end{supertabular}
}

\end{document}


\documentclass{article}
\usepackage{colortbl}
\usepackage{makecell}
\usepackage{multirow}
\usepackage{supertabular}

\begin{document}

\newcounter{utterance}

\twocolumn

{ \footnotesize  \setcounter{utterance}{1}
\setlength{\tabcolsep}{0pt}
\begin{supertabular}{c@{$\;$}|p{.15\linewidth}@{}p{.15\linewidth}p{.15\linewidth}p{.15\linewidth}p{.15\linewidth}p{.15\linewidth}}

    \# & $\;$A & \multicolumn{4}{c}{Game Master} & $\;\:$B\\
    \hline 

    \theutterance \stepcounter{utterance}  

    & & \multicolumn{4}{p{0.6\linewidth}}{\cellcolor[rgb]{0.9,0.9,0.9}{%
	\makecell[{{p{\linewidth}}}]{% 
	  \tt {\tiny [A$\langle$GM]}  
	 You are given three grids, where each of them is 5 by 5 in size.\\ \tt Grids have empty cells marked with "▢" and filled cells marked with "X".\\ \tt Your task is to generate a referring expression that best describes the target grid while distinguishing it from the two other distractor grids.\\ \tt The first grid is the target grid, and the following two grids are the distractors.\\ \tt \\ \tt Target grid:\\ \tt \\ \tt X X X X X\\ \tt □ □ □ □ X\\ \tt □ □ □ □ X\\ \tt □ □ □ □ X\\ \tt □ □ □ □ X\\ \tt \\ \tt Distractor grid 1:\\ \tt \\ \tt □ X X X X\\ \tt □ X □ □ X\\ \tt □ X □ □ X\\ \tt □ X □ □ X\\ \tt □ X X X X\\ \tt \\ \tt Distractor grid 2:\\ \tt \\ \tt □ X X X X\\ \tt □ X □ □ X\\ \tt □ X X X X\\ \tt □ X □ □ X\\ \tt □ X X X X\\ \tt \\ \tt Instruction: Describe the target grid.\\ \tt Generate the referring expression starting with the tag "Expression: " for the given target grid. Omit any other text. 
	  } 
	   } 
	   } 
	 & \\ 
 

    \theutterance \stepcounter{utterance}  

    & \multicolumn{4}{p{0.6\linewidth}}{\cellcolor[rgb]{0.8,1,0.9}{%
	\makecell[{{p{\linewidth}}}]{% 
	  \tt {\tiny [A$\rangle$GM]}  
	 <reasoning>\\ \tt Let's tackle this problem. I need to generate a referring expression that describes the target grid and distinguishes it from the two distractor grids. I'll start by carefully looking at all three grids to understand their differences.\\ \tt \\ \tt First, the target grid. Let me visualize it:\\ \tt \\ \tt Row 1: X X X X X\\ \tt Row 2: □ □ □ □ X\\ \tt Row 3: □ □ □ □ X\\ \tt Row 4: □ □ □ □ X\\ \tt Row 5: □ □ □ □ X\\ \tt \\ \tt So, the first row is completely filled with X's. The next four rows have four empty cells (□) and one X each, but where are those X's? Looking at each row:\\ \tt \\ \tt Row 2: The last cell is X. So positions (2,5) is X.\\ \tt Row 3: Similarly, (3,5) is X.\\ \tt Row 4: (4,5) is X.\\ \tt Row 5: (5,5) is X.\\ \tt \\ \tt So the target grid has a solid top row and a vertical line of X's on the rightmost column starting from row 2 to row 5. So the X's are in positions (1,1) to (1,5) and (2,5) to (5,5). Wait, no. Wait, row 1 is all X's. Then 
	  } 
	   } 
	   } 
	 & & \\ 
 

    \theutterance \stepcounter{utterance}  

    & & & \multicolumn{2}{p{0.3\linewidth}}{\cellcolor[rgb]{0.95,0.95,0.95}{%
	\makecell[{{p{\linewidth}}}]{% 
	  \tt {\tiny [GM$|$GM]}  
	 Invalid generated expression 
	  } 
	   } 
	   } 
	 & & \\ 
 

\end{supertabular}
}

\end{document}


\documentclass{article}
\usepackage{colortbl}
\usepackage{makecell}
\usepackage{multirow}
\usepackage{supertabular}

\begin{document}

\newcounter{utterance}

\twocolumn

{ \footnotesize  \setcounter{utterance}{1}
\setlength{\tabcolsep}{0pt}
\begin{supertabular}{c@{$\;$}|p{.15\linewidth}@{}p{.15\linewidth}p{.15\linewidth}p{.15\linewidth}p{.15\linewidth}p{.15\linewidth}}

    \# & $\;$A & \multicolumn{4}{c}{Game Master} & $\;\:$B\\
    \hline 

    \theutterance \stepcounter{utterance}  

    & & \multicolumn{4}{p{0.6\linewidth}}{\cellcolor[rgb]{0.9,0.9,0.9}{%
	\makecell[{{p{\linewidth}}}]{% 
	  \tt {\tiny [A$\langle$GM]}  
	 You are given three grids, where each of them is 5 by 5 in size.\\ \tt Grids have empty cells marked with "▢" and filled cells marked with "X".\\ \tt Your task is to generate a referring expression that best describes the target grid while distinguishing it from the two other distractor grids.\\ \tt The first grid is the target grid, and the following two grids are the distractors.\\ \tt \\ \tt Target grid:\\ \tt \\ \tt □ □ X X X\\ \tt □ □ X X X\\ \tt □ □ X X X\\ \tt □ □ X X X\\ \tt □ □ X X X\\ \tt \\ \tt Distractor grid 1:\\ \tt \\ \tt X □ □ X X\\ \tt X □ □ X X\\ \tt X □ □ X X\\ \tt X □ □ X X\\ \tt X □ □ X X\\ \tt \\ \tt Distractor grid 2:\\ \tt \\ \tt □ X X X □\\ \tt □ X X X □\\ \tt □ X X X □\\ \tt □ X X X □\\ \tt □ X X X □\\ \tt \\ \tt Instruction: Describe the target grid.\\ \tt Generate the referring expression starting with the tag "Expression: " for the given target grid. Omit any other text. 
	  } 
	   } 
	   } 
	 & \\ 
 

    \theutterance \stepcounter{utterance}  

    & \multicolumn{4}{p{0.6\linewidth}}{\cellcolor[rgb]{0.8,1,0.9}{%
	\makecell[{{p{\linewidth}}}]{% 
	  \tt {\tiny [A$\rangle$GM]}  
	 <reasoning>\\ \tt Let's tackle this problem. I need to generate a referring expression that describes the target grid and distinguishes it from the two distractor grids. I'll start by carefully looking at all three grids to understand their differences.\\ \tt \\ \tt First, the target grid. All rows are the same: five cells each, with the first two being empty (□) and the last three filled (X). So it's like two empty squares followed by three Xs in every row. That makes a solid block of Xs starting from the third column to the end, across all five rows. So the structure is consistent vertically and horizontally.\\ \tt \\ \tt Now, Distractor grid 1. Each row starts with an X, then three empty squares, and ends with an X. So the pattern here is X in the first and fifth columns, and empty in the middle three. Each row is like X □ □ □ X. So this forms a diagonal or maybe two Xs at the ends of each row. But since all rows have this, it's more like two Xs at the corners of each row, creating a sort of border effect but only on the ends. Wait, but each row is separate. So in this grid, each row has Xs at the first and last positions, and empty in between. So the Xs are at the start and end of each row, but not forming a continuous block. The target grid has a continuous block of Xs 
	  } 
	   } 
	   } 
	 & & \\ 
 

    \theutterance \stepcounter{utterance}  

    & & & \multicolumn{2}{p{0.3\linewidth}}{\cellcolor[rgb]{0.95,0.95,0.95}{%
	\makecell[{{p{\linewidth}}}]{% 
	  \tt {\tiny [GM$|$GM]}  
	 Invalid generated expression 
	  } 
	   } 
	   } 
	 & & \\ 
 

\end{supertabular}
}

\end{document}

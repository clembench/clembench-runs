
\documentclass{article}
\usepackage{colortbl}
\usepackage{makecell}
\usepackage{multirow}
\usepackage{supertabular}

\begin{document}

\newcounter{utterance}

\twocolumn

{ \footnotesize  \setcounter{utterance}{1}
\setlength{\tabcolsep}{0pt}
\begin{supertabular}{c@{$\;$}|p{.15\linewidth}@{}p{.15\linewidth}p{.15\linewidth}p{.15\linewidth}p{.15\linewidth}p{.15\linewidth}}

    \# & $\;$A & \multicolumn{4}{c}{Game Master} & $\;\:$B\\
    \hline 

    \theutterance \stepcounter{utterance}  

    & & \multicolumn{4}{p{0.6\linewidth}}{\cellcolor[rgb]{0.9,0.9,0.9}{%
	\makecell[{{p{\linewidth}}}]{% 
	  \tt {\tiny [A$\langle$GM]}  
	 You are given three grids, where each of them is 5 by 5 in size.\\ \tt Grids have empty cells marked with "▢" and filled cells marked with "X".\\ \tt Your task is to generate a referring expression that best describes the target grid while distinguishing it from the two other distractor grids.\\ \tt The first grid is the target grid, and the following two grids are the distractors.\\ \tt \\ \tt Target grid:\\ \tt \\ \tt □ X X X □\\ \tt □ X X X □\\ \tt □ X X X □\\ \tt □ X X X □\\ \tt □ X X X □\\ \tt \\ \tt Distractor grid 1:\\ \tt \\ \tt □ □ X X X\\ \tt □ □ X X X\\ \tt □ □ X X X\\ \tt □ □ X X X\\ \tt □ □ X X X\\ \tt \\ \tt Distractor grid 2:\\ \tt \\ \tt X X X □ □\\ \tt X X X □ □\\ \tt X X X □ □\\ \tt X X X □ □\\ \tt X X X □ □\\ \tt \\ \tt Instruction: Describe the target grid.\\ \tt Generate the referring expression starting with the tag "Expression: " for the given target grid. Omit any other text. 
	  } 
	   } 
	   } 
	 & \\ 
 

    \theutterance \stepcounter{utterance}  

    & \multicolumn{4}{p{0.6\linewidth}}{\cellcolor[rgb]{0.8,1,0.9}{%
	\makecell[{{p{\linewidth}}}]{% 
	  \tt {\tiny [A$\rangle$GM]}  
	 <reasoning>\\ \tt Let's tackle this problem. I need to generate a referring expression that describes the target grid and distinguishes it from the two distractor grids. Let me start by looking at all three grids carefully.\\ \tt \\ \tt First, the target grid. Each row is □ X X X □. So, in every row, the first and last cells are empty (□), and the middle three are filled (X). That makes a horizontal bar of three X's in the center of each row. So the target grid has a horizontal stripe of three X's in each row, centered. Let me visualize that. Each row is like: empty, X, X, X, empty. So across the rows, each row has this pattern. So the entire grid is a series of horizontal lines, each with three X's in the middle.\\ \tt \\ \tt Now, the first distractor grid. Each row here is □ □ X X X. So in each row, the first two cells are empty, and the last three are filled. So each row has a horizontal bar starting at the third column, spanning three cells. So the X's start two columns in and go to the end. So compared to the target, the X's start earlier (columns 3-5 vs columns 2-4 in the target). Wait, no. Wait, the target's X's are columns 2-4 (since it's five columns: 1,2,3, 
	  } 
	   } 
	   } 
	 & & \\ 
 

    \theutterance \stepcounter{utterance}  

    & & & \multicolumn{2}{p{0.3\linewidth}}{\cellcolor[rgb]{0.95,0.95,0.95}{%
	\makecell[{{p{\linewidth}}}]{% 
	  \tt {\tiny [GM$|$GM]}  
	 Invalid generated expression 
	  } 
	   } 
	   } 
	 & & \\ 
 

\end{supertabular}
}

\end{document}


\documentclass{article}
\usepackage{colortbl}
\usepackage{makecell}
\usepackage{multirow}
\usepackage{supertabular}

\begin{document}

\newcounter{utterance}

\twocolumn

{ \footnotesize  \setcounter{utterance}{1}
\setlength{\tabcolsep}{0pt}
\begin{supertabular}{c@{$\;$}|p{.15\linewidth}@{}p{.15\linewidth}p{.15\linewidth}p{.15\linewidth}p{.15\linewidth}p{.15\linewidth}}

    \# & $\;$A & \multicolumn{4}{c}{Game Master} & $\;\:$B\\
    \hline 

    \theutterance \stepcounter{utterance}  

    & & \multicolumn{4}{p{0.6\linewidth}}{\cellcolor[rgb]{0.9,0.9,0.9}{%
	\makecell[{{p{\linewidth}}}]{% 
	  \tt {\tiny [A$\langle$GM]}  
	 You are given three grids, where each of them is 5 by 5 in size.\\ \tt Grids have empty cells marked with "▢" and filled cells marked with "X".\\ \tt Your task is to generate a referring expression that best describes the target grid while distinguishing it from the two other distractor grids.\\ \tt The first grid is the target grid, and the following two grids are the distractors.\\ \tt \\ \tt Target grid:\\ \tt \\ \tt □ X X □ X\\ \tt □ X X □ X\\ \tt □ X X □ X\\ \tt □ X X □ X\\ \tt □ X X □ X\\ \tt \\ \tt Distractor grid 1:\\ \tt \\ \tt □ □ X X X\\ \tt □ □ X X X\\ \tt □ □ X X X\\ \tt □ □ X X X\\ \tt □ □ X X X\\ \tt \\ \tt Distractor grid 2:\\ \tt \\ \tt □ X X X □\\ \tt □ X X X □\\ \tt □ X X X □\\ \tt □ X X X □\\ \tt □ X X X □\\ \tt \\ \tt Instruction: Describe the target grid.\\ \tt Generate the referring expression starting with the tag "Expression: " for the given target grid. Omit any other text. 
	  } 
	   } 
	   } 
	 & \\ 
 

    \theutterance \stepcounter{utterance}  

    & \multicolumn{4}{p{0.6\linewidth}}{\cellcolor[rgb]{0.8,1,0.9}{%
	\makecell[{{p{\linewidth}}}]{% 
	  \tt {\tiny [A$\rangle$GM]}  
	 <reasoning>\\ \tt Let's tackle this problem. I need to generate a referring expression that describes the target grid and distinguishes it from the two distractor grids. I'll start by carefully looking at all three grids to understand their differences.\\ \tt \\ \tt First, the target grid. Each row is □ X X □ X. So, each row has five cells. The first and fifth cells are empty (□), the second and third are filled (X), and the fourth is empty again. So, the pattern per row is □ X X □ X. That means in each row, there's a sequence of two X's in the middle, but not exactly the middle because the positions are 2 and 3. Wait, positions 2 and 3 are the second and third cells. So each row has X's in positions 2 and 3, with empty cells at the start and end. So the pattern is like "empty, X, X, empty, X" per row. Wait, no: the first cell is empty, then two X's, then empty, then X. So each row is □ X X □ X. So each row has three X's? Wait, no: positions 2,3,5 are X. So each row has three X's. Let me count: positions 1: □, 2: X, 3: X, 4: □, 5: X. 
	  } 
	   } 
	   } 
	 & & \\ 
 

    \theutterance \stepcounter{utterance}  

    & & & \multicolumn{2}{p{0.3\linewidth}}{\cellcolor[rgb]{0.95,0.95,0.95}{%
	\makecell[{{p{\linewidth}}}]{% 
	  \tt {\tiny [GM$|$GM]}  
	 Invalid generated expression 
	  } 
	   } 
	   } 
	 & & \\ 
 

\end{supertabular}
}

\end{document}


\documentclass{article}
\usepackage{colortbl}
\usepackage{makecell}
\usepackage{multirow}
\usepackage{supertabular}

\begin{document}

\newcounter{utterance}

\twocolumn

{ \footnotesize  \setcounter{utterance}{1}
\setlength{\tabcolsep}{0pt}
\begin{supertabular}{c@{$\;$}|p{.15\linewidth}@{}p{.15\linewidth}p{.15\linewidth}p{.15\linewidth}p{.15\linewidth}p{.15\linewidth}}

    \# & $\;$A & \multicolumn{4}{c}{Game Master} & $\;\:$B\\
    \hline 

    \theutterance \stepcounter{utterance}  

    & & \multicolumn{4}{p{0.6\linewidth}}{\cellcolor[rgb]{0.9,0.9,0.9}{%
	\makecell[{{p{\linewidth}}}]{% 
	  \tt {\tiny [A$\langle$GM]}  
	 You are given three grids, where each of them is 5 by 5 in size.\\ \tt Grids have empty cells marked with "▢" and filled cells marked with "X".\\ \tt Your task is to generate a referring expression that best describes the target grid while distinguishing it from the two other distractor grids.\\ \tt The first grid is the target grid, and the following two grids are the distractors.\\ \tt \\ \tt Target grid:\\ \tt \\ \tt X □ □ □ X\\ \tt X □ □ □ X\\ \tt X X X X X\\ \tt X □ □ □ X\\ \tt X □ □ □ X\\ \tt \\ \tt Distractor grid 1:\\ \tt \\ \tt X X X X X\\ \tt X □ X □ X\\ \tt X X X X X\\ \tt X □ X □ X\\ \tt X X X X X\\ \tt \\ \tt Distractor grid 2:\\ \tt \\ \tt X X X X X\\ \tt X □ □ □ □\\ \tt X X X X X\\ \tt X □ □ □ □\\ \tt X X X X X\\ \tt \\ \tt Instruction: Describe the target grid.\\ \tt Generate the referring expression starting with the tag "Expression: " for the given target grid. Omit any other text. 
	  } 
	   } 
	   } 
	 & \\ 
 

    \theutterance \stepcounter{utterance}  

    & \multicolumn{4}{p{0.6\linewidth}}{\cellcolor[rgb]{0.8,1,0.9}{%
	\makecell[{{p{\linewidth}}}]{% 
	  \tt {\tiny [A$\rangle$GM]}  
	 <reasoning>\\ \tt Let's tackle this problem. I need to generate a referring expression that describes the target grid and distinguishes it from the two distractor grids. I'll start by carefully looking at each grid to understand their structures.\\ \tt \\ \tt First, the target grid. Let me write it out again to visualize:\\ \tt \\ \tt Row 1: X □ □ □ X\\ \tt Row 2: X □ □ □ X\\ \tt Row 3: X X X X X\\ \tt Row 4: X □ □ □ X\\ \tt Row 5: X □ □ □ X\\ \tt \\ \tt So, the target grid has an 'X' in the first and last columns for all rows, but the middle row (row 3) is completely filled with X's. The other rows have X's only in the first and fifth columns, and empty in the middle three. This looks like a cross shape, but with the vertical line being the first and fifth columns, and the horizontal line being the middle row. Wait, actually, if you look at it, the vertical line is the first and fifth columns, but the horizontal line is the middle row. So it's like a cross where the vertical part is two columns wide (first and fifth) and the horizontal part is a single row (third row) spanning all five columns. Hmm, maybe it's more like a horizontal bar in the middle and vertical bars on the left and right edges. Wait, no 
	  } 
	   } 
	   } 
	 & & \\ 
 

    \theutterance \stepcounter{utterance}  

    & & & \multicolumn{2}{p{0.3\linewidth}}{\cellcolor[rgb]{0.95,0.95,0.95}{%
	\makecell[{{p{\linewidth}}}]{% 
	  \tt {\tiny [GM$|$GM]}  
	 Invalid generated expression 
	  } 
	   } 
	   } 
	 & & \\ 
 

\end{supertabular}
}

\end{document}


\documentclass{article}
\usepackage{colortbl}
\usepackage{makecell}
\usepackage{multirow}
\usepackage{supertabular}

\begin{document}

\newcounter{utterance}

\twocolumn

{ \footnotesize  \setcounter{utterance}{1}
\setlength{\tabcolsep}{0pt}
\begin{supertabular}{c@{$\;$}|p{.15\linewidth}@{}p{.15\linewidth}p{.15\linewidth}p{.15\linewidth}p{.15\linewidth}p{.15\linewidth}}

    \# & $\;$A & \multicolumn{4}{c}{Game Master} & $\;\:$B\\
    \hline 

    \theutterance \stepcounter{utterance}  

    & & \multicolumn{4}{p{0.6\linewidth}}{\cellcolor[rgb]{0.9,0.9,0.9}{%
	\makecell[{{p{\linewidth}}}]{% 
	  \tt {\tiny [A$\langle$GM]}  
	 You are given three grids, where each of them is 5 by 5 in size.\\ \tt Grids have empty cells marked with "▢" and filled cells marked with "X".\\ \tt Your task is to generate a referring expression that best describes the target grid while distinguishing it from the two other distractor grids.\\ \tt The first grid is the target grid, and the following two grids are the distractors.\\ \tt \\ \tt Target grid:\\ \tt \\ \tt X X X X X\\ \tt X □ □ □ □\\ \tt X □ □ □ □\\ \tt X □ □ □ □\\ \tt X X X X X\\ \tt \\ \tt Distractor grid 1:\\ \tt \\ \tt X X X X X\\ \tt X □ □ □ □\\ \tt X X X X X\\ \tt X □ □ □ □\\ \tt X X X X X\\ \tt \\ \tt Distractor grid 2:\\ \tt \\ \tt X X X X X\\ \tt X □ X □ X\\ \tt X X X X X\\ \tt X □ X □ X\\ \tt X X X X X\\ \tt \\ \tt Instruction: Describe the target grid.\\ \tt Generate the referring expression starting with the tag "Expression: " for the given target grid. Omit any other text. 
	  } 
	   } 
	   } 
	 & \\ 
 

    \theutterance \stepcounter{utterance}  

    & \multicolumn{4}{p{0.6\linewidth}}{\cellcolor[rgb]{0.8,1,0.9}{%
	\makecell[{{p{\linewidth}}}]{% 
	  \tt {\tiny [A$\rangle$GM]}  
	 <reasoning>\\ \tt Let's tackle this problem. I need to generate a referring expression that describes the target grid and distinguishes it from the two distractor grids. I'll start by carefully looking at all three grids to understand their differences.\\ \tt \\ \tt First, the target grid. It's a 5x5 grid. The first row is all filled with X's. Then the next three rows have X in the first and last positions, but the middle three are empty (□). The fifth row is all X's again. So the pattern here is like a border around the entire grid, but with the middle three rows having only the first and last columns filled. Wait, no. Let me visualize it:\\ \tt \\ \tt Row 1: X X X X X\\ \tt Row 2: X □ □ □ X\\ \tt Row 3: X □ □ □ X\\ \tt Row 4: X □ □ □ X\\ \tt Row 5: X X X X X\\ \tt \\ \tt So the first and fifth rows are full X's. The middle three rows (rows 2-4) have X's only in the first and last columns, and the rest are empty. So it's like a frame that's thicker on the top and bottom but only two cells wide in the middle rows. Wait, no. Wait, rows 2-4 have two X's each (first and last), but the first and fifth rows have all five. So the overall shape 
	  } 
	   } 
	   } 
	 & & \\ 
 

    \theutterance \stepcounter{utterance}  

    & & & \multicolumn{2}{p{0.3\linewidth}}{\cellcolor[rgb]{0.95,0.95,0.95}{%
	\makecell[{{p{\linewidth}}}]{% 
	  \tt {\tiny [GM$|$GM]}  
	 Invalid generated expression 
	  } 
	   } 
	   } 
	 & & \\ 
 

\end{supertabular}
}

\end{document}

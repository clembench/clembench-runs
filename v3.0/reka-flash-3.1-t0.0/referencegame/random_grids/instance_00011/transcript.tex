
\documentclass{article}
\usepackage{colortbl}
\usepackage{makecell}
\usepackage{multirow}
\usepackage{supertabular}

\begin{document}

\newcounter{utterance}

\twocolumn

{ \footnotesize  \setcounter{utterance}{1}
\setlength{\tabcolsep}{0pt}
\begin{supertabular}{c@{$\;$}|p{.15\linewidth}@{}p{.15\linewidth}p{.15\linewidth}p{.15\linewidth}p{.15\linewidth}p{.15\linewidth}}

    \# & $\;$A & \multicolumn{4}{c}{Game Master} & $\;\:$B\\
    \hline 

    \theutterance \stepcounter{utterance}  

    & & \multicolumn{4}{p{0.6\linewidth}}{\cellcolor[rgb]{0.9,0.9,0.9}{%
	\makecell[{{p{\linewidth}}}]{% 
	  \tt {\tiny [A$\langle$GM]}  
	 You are given three grids, where each of them is 5 by 5 in size.\\ \tt Grids have empty cells marked with "▢" and filled cells marked with "X".\\ \tt Your task is to generate a referring expression that best describes the target grid while distinguishing it from the two other distractor grids.\\ \tt The first grid is the target grid, and the following two grids are the distractors.\\ \tt \\ \tt Target grid:\\ \tt \\ \tt X □ X X □\\ \tt □ X □ X X\\ \tt X X □ □ X\\ \tt □ X X □ X\\ \tt X □ X □ □\\ \tt \\ \tt Distractor grid 1:\\ \tt \\ \tt □ □ X X □\\ \tt X □ □ X X\\ \tt □ X X □ X\\ \tt X X □ □ X\\ \tt □ X X □ □\\ \tt \\ \tt Distractor grid 2:\\ \tt \\ \tt X X □ X □\\ \tt □ X X □ X\\ \tt X □ □ X □\\ \tt □ X □ □ X\\ \tt X □ X □ X\\ \tt \\ \tt Instruction: Describe the target grid.\\ \tt Generate the referring expression starting with the tag "Expression: " for the given target grid. Omit any other text. 
	  } 
	   } 
	   } 
	 & \\ 
 

    \theutterance \stepcounter{utterance}  

    & \multicolumn{4}{p{0.6\linewidth}}{\cellcolor[rgb]{0.8,1,0.9}{%
	\makecell[{{p{\linewidth}}}]{% 
	  \tt {\tiny [A$\rangle$GM]}  
	 <reasoning>\\ \tt Let's tackle this problem. I need to generate a referring expression that describes the target grid and distinguishes it from the two distractor grids. The target grid is the first one, and the other two are distractors. Let me start by carefully comparing all three grids to find the unique features of the target.\\ \tt \\ \tt First, I'll write down each grid to visualize them better.\\ \tt \\ \tt **Target Grid:**\\ \tt Row 1: X □ X X □\\ \tt Row 2: □ X □ X X\\ \tt Row 3: X X □ □ X\\ \tt Row 4: □ X X □ X\\ \tt Row 5: X □ X □ □\\ \tt \\ \tt **Distractor Grid 1:**\\ \tt Row 1: □ □ X X □\\ \tt Row 2: X □ □ X X\\ \tt Row 3: □ X X □ X\\ \tt Row 4: X X □ □ X\\ \tt Row 5: □ X X □ □\\ \tt \\ \tt **Distractor Grid 2:**\\ \tt Row 1: X X □ X □\\ \tt Row 2: □ X X □ X\\ \tt Row 3: X □ □ X □\\ \tt Row 4: □ X □ □ X\\ \tt Row 5: X □ X □ X\\ \tt \\ \tt Okay, let's compare row by row.\\ \tt \\ \tt Starting with Row 
	  } 
	   } 
	   } 
	 & & \\ 
 

    \theutterance \stepcounter{utterance}  

    & & & \multicolumn{2}{p{0.3\linewidth}}{\cellcolor[rgb]{0.95,0.95,0.95}{%
	\makecell[{{p{\linewidth}}}]{% 
	  \tt {\tiny [GM$|$GM]}  
	 Invalid generated expression 
	  } 
	   } 
	   } 
	 & & \\ 
 

\end{supertabular}
}

\end{document}

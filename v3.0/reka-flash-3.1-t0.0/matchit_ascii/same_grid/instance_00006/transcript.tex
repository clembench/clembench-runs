
\documentclass{article}
\usepackage{colortbl}
\usepackage{makecell}
\usepackage{multirow}
\usepackage{supertabular}

\begin{document}

\newcounter{utterance}

\twocolumn

{ \footnotesize  \setcounter{utterance}{1}
\setlength{\tabcolsep}{0pt}
\begin{supertabular}{c@{$\;$}|p{.15\linewidth}@{}p{.15\linewidth}p{.15\linewidth}p{.15\linewidth}p{.15\linewidth}p{.15\linewidth}}

    \# & $\;$A & \multicolumn{4}{c}{Game Master} & $\;\:$B\\
    \hline 

    \theutterance \stepcounter{utterance}  

    & & \multicolumn{4}{p{0.6\linewidth}}{\cellcolor[rgb]{0.9,0.9,0.9}{%
	\makecell[{{p{\linewidth}}}]{% 
	  \tt {\tiny [A$\langle$GM]}  
	 You are participating in a collaborative guessing game.\\ \tt The goal is to find out whether this grid and another picture only I can see, are the same.\\ \tt Please describe your grid first. Then, I will provide my description and we can ask each other questions about the grids to figure out whether they are the same. Now start your short grid description with "DESCRIPTION:" followed by the description. Do not add anything else. This is your grid:\\ \tt ▢ ▢ ▢ ▢ ▢\\ \tt ▢ X X X X\\ \tt ▢ ▢ ▢ X X\\ \tt ▢ X X X X\\ \tt ▢ X X X X 
	  } 
	   } 
	   } 
	 & \\ 
 

    \theutterance \stepcounter{utterance}  

    & \multicolumn{4}{p{0.6\linewidth}}{\cellcolor[rgb]{0.8,1,0.9}{%
	\makecell[{{p{\linewidth}}}]{% 
	  \tt {\tiny [A$\rangle$GM]}  
	 <reasoning>\\ \tt Let me try to figure out how to describe this grid properly. The user provided a grid with five rows and five columns. Each cell is either a blank (▢) or an X. My task is to describe this grid in a way that someone else can understand and compare it to their own grid.\\ \tt \\ \tt First, I need to visualize the grid. Let me write it out again to make sure I have it right:\\ \tt \\ \tt Row 1: ▢ ▢ ▢ ▢ ▢\\ \tt Row 2: ▢ X X X X\\ \tt Row 3: ▢ ▢ ▢ X X\\ \tt Row 4: ▢ X X X X\\ \tt Row 5: ▢ X X X X\\ \tt \\ \tt So, each row has five elements. Let me check each row for X positions.\\ \tt \\ \tt Row 1: All blanks. So positions 1-5 are empty.\\ \tt \\ \tt Row 2: The first cell is blank, then four Xs. So positions 2-5 are Xs.\\ \tt \\ \tt Row 3: First three are blanks, then two Xs. Positions 4-5 are Xs.\\ \tt \\ \tt Row 4: First cell blank, then four Xs. Positions 2-5 are Xs.\\ \tt \\ \tt Row 5: First cell blank, then four Xs. Positions 2-5 are Xs.\\ \tt \\ \tt Wait, but looking at rows 4 and 5, they both start with a blank and then four Xs. So 
	  } 
	   } 
	   } 
	 & & \\ 
 

    \theutterance \stepcounter{utterance}  

    & & & \multicolumn{2}{p{0.3\linewidth}}{\cellcolor[rgb]{0.95,0.95,0.95}{%
	\makecell[{{p{\linewidth}}}]{% 
	  \tt {\tiny [GM$|$GM]}  
	 abort, first word: <reasoning>, but expected DESCRIPTION: 
	  } 
	   } 
	   } 
	 & & \\ 
 

    \theutterance \stepcounter{utterance}  

    & & & \multicolumn{2}{p{0.3\linewidth}}{\cellcolor[rgb]{0.95,0.95,0.95}{%
	\makecell[{{p{\linewidth}}}]{% 
	  \tt {\tiny [GM$|$GM]}  
	 Aborted 
	  } 
	   } 
	   } 
	 & & \\ 
 

\end{supertabular}
}

\end{document}

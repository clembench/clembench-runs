
\documentclass{article}
\usepackage{colortbl}
\usepackage{makecell}
\usepackage{multirow}
\usepackage{supertabular}

\begin{document}

\newcounter{utterance}

\twocolumn

{ \footnotesize  \setcounter{utterance}{1}
\setlength{\tabcolsep}{0pt}
\begin{supertabular}{c@{$\;$}|p{.15\linewidth}@{}p{.15\linewidth}p{.15\linewidth}p{.15\linewidth}p{.15\linewidth}p{.15\linewidth}}

    \# & $\;$A & \multicolumn{4}{c}{Game Master} & $\;\:$B\\
    \hline 

    \theutterance \stepcounter{utterance}  

    & & \multicolumn{4}{p{0.6\linewidth}}{\cellcolor[rgb]{0.9,0.9,0.9}{%
	\makecell[{{p{\linewidth}}}]{% 
	  \tt {\tiny [A$\langle$GM]}  
	 You are participating in a collaborative guessing game.\\ \tt The goal is to find out whether this grid and another picture only I can see, are the same.\\ \tt Please describe your grid first. Then, I will provide my description and we can ask each other questions about the grids to figure out whether they are the same. Now start your short grid description with "DESCRIPTION:" followed by the description. Do not add anything else. This is your grid:\\ \tt X X X X X\\ \tt ▢ ▢ ▢ ▢ X\\ \tt ▢ ▢ X X X\\ \tt ▢ ▢ ▢ ▢ X\\ \tt X X X X X 
	  } 
	   } 
	   } 
	 & \\ 
 

    \theutterance \stepcounter{utterance}  

    & \multicolumn{4}{p{0.6\linewidth}}{\cellcolor[rgb]{0.8,1,0.9}{%
	\makecell[{{p{\linewidth}}}]{% 
	  \tt {\tiny [A$\rangle$GM]}  
	 <reasoning>\\ \tt Let me try to figure out how to describe this grid properly. The user provided a grid with X's and ▢'s. Let me visualize it first.\\ \tt \\ \tt The grid has 5 rows. Let's break it down row by row:\\ \tt \\ \tt 1st row: X X X X X – all X's.\\ \tt 2nd row: ▢ ▢ ▢ ▢ X – four ▢'s and one X at the end.\\ \tt 3rd row: ▢ ▢ X X X – two ▢'s followed by three X's.\\ \tt 4th row: ▢ ▢ ▢ ▢ X – similar to the second row, four ▢'s and one X.\\ \tt 5th row: X X X X X – same as the first row, all X's.\\ \tt \\ \tt So the pattern seems to be that the first and fifth rows are all X's. The second and fourth rows have four ▢'s and one X each, but in different positions. The third row has two ▢'s and three X's. Wait, let me check again:\\ \tt \\ \tt Wait, the second row is ▢ ▢ ▢ ▢ X – so the X is at the end. The fourth row is ▢ ▢ ▢ ▢ X – same as the second? Or maybe the X is at the beginning? Wait, the user wrote:\\ \tt \\ \tt ▢ ▢ ▢ ▢ X for the second row. So positions 1 
	  } 
	   } 
	   } 
	 & & \\ 
 

    \theutterance \stepcounter{utterance}  

    & & & \multicolumn{2}{p{0.3\linewidth}}{\cellcolor[rgb]{0.95,0.95,0.95}{%
	\makecell[{{p{\linewidth}}}]{% 
	  \tt {\tiny [GM$|$GM]}  
	 abort, first word: <reasoning>, but expected DESCRIPTION: 
	  } 
	   } 
	   } 
	 & & \\ 
 

    \theutterance \stepcounter{utterance}  

    & & & \multicolumn{2}{p{0.3\linewidth}}{\cellcolor[rgb]{0.95,0.95,0.95}{%
	\makecell[{{p{\linewidth}}}]{% 
	  \tt {\tiny [GM$|$GM]}  
	 Aborted 
	  } 
	   } 
	   } 
	 & & \\ 
 

\end{supertabular}
}

\end{document}

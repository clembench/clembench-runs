
\documentclass{article}
\usepackage{colortbl}
\usepackage{makecell}
\usepackage{multirow}
\usepackage{supertabular}

\begin{document}

\newcounter{utterance}

\twocolumn

{ \footnotesize  \setcounter{utterance}{1}
\setlength{\tabcolsep}{0pt}
\begin{supertabular}{c@{$\;$}|p{.15\linewidth}@{}p{.15\linewidth}p{.15\linewidth}p{.15\linewidth}p{.15\linewidth}p{.15\linewidth}}

    \# & $\;$A & \multicolumn{4}{c}{Game Master} & $\;\:$B\\
    \hline 

    \theutterance \stepcounter{utterance}  

    & & \multicolumn{4}{p{0.6\linewidth}}{\cellcolor[rgb]{0.9,0.9,0.9}{%
	\makecell[{{p{\linewidth}}}]{% 
	  \tt {\tiny [A$\langle$GM]}  
	 You are participating in a collaborative guessing game.\\ \tt The goal is to find out whether this grid and another picture only I can see, are the same.\\ \tt Please describe your grid first. Then, I will provide my description and we can ask each other questions about the grids to figure out whether they are the same. Now start your short grid description with "DESCRIPTION:" followed by the description. Do not add anything else. This is your grid:\\ \tt ▢ X X X X\\ \tt X ▢ X X X\\ \tt X X ▢ X X\\ \tt X X X ▢ X\\ \tt X X X X ▢ 
	  } 
	   } 
	   } 
	 & \\ 
 

    \theutterance \stepcounter{utterance}  

    & \multicolumn{4}{p{0.6\linewidth}}{\cellcolor[rgb]{0.8,1,0.9}{%
	\makecell[{{p{\linewidth}}}]{% 
	  \tt {\tiny [A$\rangle$GM]}  
	 <reasoning>\\ \tt Let me try to figure out how to describe this grid properly. The user provided a grid that looks like a 5x5 grid with X's and a ▢. Let me visualize it first.\\ \tt \\ \tt The first row is ▢ followed by four X's. So, position (1,1) is the circle, and the rest are X's. The second row starts with an X, then a ▢, and three X's. So (2,2) is the circle. Third row: X, X, ▢, X, X. That's (3,3). Fourth row: X, X, X, ▢, X. So (4,4). Fifth row: all X's except the last one, which is ▢. Wait, no, the fifth row is X X X X ▢. So (5,5) is the circle. \\ \tt \\ \tt Wait, let me check again. The user wrote:\\ \tt \\ \tt ▢ X X X X  \\ \tt X ▢ X X X  \\ \tt X X ▢ X X  \\ \tt X X X ▢ X  \\ \tt X X X X ▢  \\ \tt \\ \tt Yes, each row has a circle in the diagonal positions. So it's a diagonal from top-left to bottom-right. Each row i has the circle in column i. So row 1, column 1; row 2, column 2; etc. So the grid is a 5x5 diagonal matrix with circles and 
	  } 
	   } 
	   } 
	 & & \\ 
 

    \theutterance \stepcounter{utterance}  

    & & & \multicolumn{2}{p{0.3\linewidth}}{\cellcolor[rgb]{0.95,0.95,0.95}{%
	\makecell[{{p{\linewidth}}}]{% 
	  \tt {\tiny [GM$|$GM]}  
	 abort, first word: <reasoning>, but expected DESCRIPTION: 
	  } 
	   } 
	   } 
	 & & \\ 
 

    \theutterance \stepcounter{utterance}  

    & & & \multicolumn{2}{p{0.3\linewidth}}{\cellcolor[rgb]{0.95,0.95,0.95}{%
	\makecell[{{p{\linewidth}}}]{% 
	  \tt {\tiny [GM$|$GM]}  
	 Aborted 
	  } 
	   } 
	   } 
	 & & \\ 
 

\end{supertabular}
}

\end{document}

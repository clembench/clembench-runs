
\documentclass{article}
\usepackage{colortbl}
\usepackage{makecell}
\usepackage{multirow}
\usepackage{supertabular}

\begin{document}

\newcounter{utterance}

\twocolumn

{ \footnotesize  \setcounter{utterance}{1}
\setlength{\tabcolsep}{0pt}
\begin{supertabular}{c@{$\;$}|p{.15\linewidth}@{}p{.15\linewidth}p{.15\linewidth}p{.15\linewidth}p{.15\linewidth}p{.15\linewidth}}

    \# & $\;$A & \multicolumn{4}{c}{Game Master} & $\;\:$B\\
    \hline 

    \theutterance \stepcounter{utterance}  

    & & \multicolumn{4}{p{0.6\linewidth}}{\cellcolor[rgb]{0.9,0.9,0.9}{%
	\makecell[{{p{\linewidth}}}]{% 
	  \tt {\tiny [A$\langle$GM]}  
	 You are participating in a collaborative guessing game.\\ \tt The goal is to find out whether this grid and another picture only I can see, are the same.\\ \tt Please describe your grid first. Then, I will provide my description and we can ask each other questions about the grids to figure out whether they are the same. Now start your short grid description with "DESCRIPTION:" followed by the description. Do not add anything else. This is your grid:\\ \tt X ▢ ▢ ▢ X\\ \tt X X ▢ ▢ X\\ \tt X ▢ X ▢ X\\ \tt X ▢ ▢ X X\\ \tt X ▢ ▢ ▢ X 
	  } 
	   } 
	   } 
	 & \\ 
 

    \theutterance \stepcounter{utterance}  

    & \multicolumn{4}{p{0.6\linewidth}}{\cellcolor[rgb]{0.8,1,0.9}{%
	\makecell[{{p{\linewidth}}}]{% 
	  \tt {\tiny [A$\rangle$GM]}  
	 <reasoning>\\ \tt Let me try to figure out how to describe this grid properly. The user provided a 5x5 grid with some Xs and ▢s. My task is to describe it in a way that someone else can compare with their grid. Let me start by visualizing the grid.\\ \tt \\ \tt First, let's break down each row:\\ \tt \\ \tt Row 1: X ▢ ▢ ▢ X\\ \tt Row 2: X X ▢ ▢ X\\ \tt Row 3: X ▢ X ▢ X\\ \tt Row 4: X ▢ ▢ X X\\ \tt Row 5: X ▢ ▢ ▢ X\\ \tt \\ \tt So each row starts and ends with X, and the middle positions vary. The challenge is to describe this structure clearly. Maybe I can note the positions of the Xs and ▢s. Alternatively, describe the pattern in terms of symmetry or specific shapes.\\ \tt \\ \tt Looking at the grid, the first and last rows are similar, with Xs at the ends and three ▢s in between. The second row has two Xs at the start, then two ▢s, then an X. The third row alternates X and ▢, but let's check: X, then ▢, then X, then ▢, then X. Wait, no. Wait, the third row is X ▢ X ▢ X. So positions 1,3,5 are X, and 2,4 are � 
	  } 
	   } 
	   } 
	 & & \\ 
 

    \theutterance \stepcounter{utterance}  

    & & & \multicolumn{2}{p{0.3\linewidth}}{\cellcolor[rgb]{0.95,0.95,0.95}{%
	\makecell[{{p{\linewidth}}}]{% 
	  \tt {\tiny [GM$|$GM]}  
	 abort, first word: <reasoning>, but expected DESCRIPTION: 
	  } 
	   } 
	   } 
	 & & \\ 
 

    \theutterance \stepcounter{utterance}  

    & & & \multicolumn{2}{p{0.3\linewidth}}{\cellcolor[rgb]{0.95,0.95,0.95}{%
	\makecell[{{p{\linewidth}}}]{% 
	  \tt {\tiny [GM$|$GM]}  
	 Aborted 
	  } 
	   } 
	   } 
	 & & \\ 
 

\end{supertabular}
}

\end{document}

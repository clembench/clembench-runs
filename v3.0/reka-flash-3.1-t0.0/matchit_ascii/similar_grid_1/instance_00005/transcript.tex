
\documentclass{article}
\usepackage{colortbl}
\usepackage{makecell}
\usepackage{multirow}
\usepackage{supertabular}

\begin{document}

\newcounter{utterance}

\twocolumn

{ \footnotesize  \setcounter{utterance}{1}
\setlength{\tabcolsep}{0pt}
\begin{supertabular}{c@{$\;$}|p{.15\linewidth}@{}p{.15\linewidth}p{.15\linewidth}p{.15\linewidth}p{.15\linewidth}p{.15\linewidth}}

    \# & $\;$A & \multicolumn{4}{c}{Game Master} & $\;\:$B\\
    \hline 

    \theutterance \stepcounter{utterance}  

    & & \multicolumn{4}{p{0.6\linewidth}}{\cellcolor[rgb]{0.9,0.9,0.9}{%
	\makecell[{{p{\linewidth}}}]{% 
	  \tt {\tiny [A$\langle$GM]}  
	 You are participating in a collaborative guessing game.\\ \tt The goal is to find out whether this grid and another picture only I can see, are the same.\\ \tt Please describe your grid first. Then, I will provide my description and we can ask each other questions about the grids to figure out whether they are the same. Now start your short grid description with "DESCRIPTION:" followed by the description. Do not add anything else. This is your grid:\\ \tt ▢ ▢ X ▢ ▢\\ \tt ▢ ▢ X X ▢\\ \tt X X X X X\\ \tt ▢ ▢ X X ▢\\ \tt ▢ ▢ X ▢ ▢ 
	  } 
	   } 
	   } 
	 & \\ 
 

    \theutterance \stepcounter{utterance}  

    & \multicolumn{4}{p{0.6\linewidth}}{\cellcolor[rgb]{0.8,1,0.9}{%
	\makecell[{{p{\linewidth}}}]{% 
	  \tt {\tiny [A$\rangle$GM]}  
	 <reasoning>\\ \tt Let me try to figure out how to describe this grid properly. The user provided a 5x5 grid with some X's and empty spaces. My task is to describe it in a way that someone else can visualize it correctly. I'll start by looking at each row one by one.\\ \tt \\ \tt First row: ▢ ▢ X ▢ ▢. So, three empty squares, then an X, then two more empty. Wait, no, let me count again. Each row has five elements. The first row is two empty, then X, then two empty? Wait, no. Let me check again. The first row is written as "▢ ▢ X ▢ ▢". So positions 1: ▢, 2: ▢, 3: X, 4: ▢, 5: ▢. So the X is in the third position of the first row.\\ \tt \\ \tt Second row: ▢ ▢ X X ▢. So positions 1: ▢, 2: ▢, 3: X, 4: X, 5: ▢. So two X's starting from the third position.\\ \tt \\ \tt Third row: X X X X X. That's a full row of X's. So all five positions are X.\\ \tt \\ \tt Fourth row: ▢ ▢ X X ▢. Similar to the second row. Positions 1 and 2 are empty, then two X's, then 
	  } 
	   } 
	   } 
	 & & \\ 
 

    \theutterance \stepcounter{utterance}  

    & & & \multicolumn{2}{p{0.3\linewidth}}{\cellcolor[rgb]{0.95,0.95,0.95}{%
	\makecell[{{p{\linewidth}}}]{% 
	  \tt {\tiny [GM$|$GM]}  
	 abort, first word: <reasoning>, but expected DESCRIPTION: 
	  } 
	   } 
	   } 
	 & & \\ 
 

    \theutterance \stepcounter{utterance}  

    & & & \multicolumn{2}{p{0.3\linewidth}}{\cellcolor[rgb]{0.95,0.95,0.95}{%
	\makecell[{{p{\linewidth}}}]{% 
	  \tt {\tiny [GM$|$GM]}  
	 Aborted 
	  } 
	   } 
	   } 
	 & & \\ 
 

\end{supertabular}
}

\end{document}

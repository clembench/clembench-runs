
\documentclass{article}
\usepackage{colortbl}
\usepackage{makecell}
\usepackage{multirow}
\usepackage{supertabular}

\begin{document}

\newcounter{utterance}

\twocolumn

{ \footnotesize  \setcounter{utterance}{1}
\setlength{\tabcolsep}{0pt}
\begin{supertabular}{c@{$\;$}|p{.15\linewidth}@{}p{.15\linewidth}p{.15\linewidth}p{.15\linewidth}p{.15\linewidth}p{.15\linewidth}}

    \# & $\;$A & \multicolumn{4}{c}{Game Master} & $\;\:$B\\
    \hline 

    \theutterance \stepcounter{utterance}  

    & & \multicolumn{4}{p{0.6\linewidth}}{\cellcolor[rgb]{0.9,0.9,0.9}{%
	\makecell[{{p{\linewidth}}}]{% 
	  \tt {\tiny [A$\langle$GM]}  
	 You are participating in a collaborative guessing game.\\ \tt The goal is to find out whether this grid and another picture only I can see, are the same.\\ \tt Please describe your grid first. Then, I will provide my description and we can ask each other questions about the grids to figure out whether they are the same. Now start your short grid description with "DESCRIPTION:" followed by the description. Do not add anything else. This is your grid:\\ \tt X X X X ▢\\ \tt X X X ▢ X\\ \tt X X ▢ X X\\ \tt X ▢ X X X\\ \tt ▢ X X X X 
	  } 
	   } 
	   } 
	 & \\ 
 

    \theutterance \stepcounter{utterance}  

    & \multicolumn{4}{p{0.6\linewidth}}{\cellcolor[rgb]{0.8,1,0.9}{%
	\makecell[{{p{\linewidth}}}]{% 
	  \tt {\tiny [A$\rangle$GM]}  
	 <reasoning>\\ \tt Let me try to figure out how to describe this grid properly. The user provided a grid that looks like a 5x5 grid with some X's and a ▢. Let me visualize it first.\\ \tt \\ \tt The first row is X X X X ▢. So that's four X's followed by a ▢. Then the second row is X X X ▢ X. So three X's, a ▢, then an X. Third row: X X ▢ X X. Two X's, a ▢, two X's. Fourth row: X ▢ X X X. One X, a ▢, three X's. Fifth row: ▢ X X X X. A ▢ followed by four X's.\\ \tt \\ \tt Wait, let me check each row again to make sure I didn't mix up the positions. Let's count each character per row:\\ \tt \\ \tt Row 1: X, X, X, X, ▢\\ \tt Row 2: X, X, X, ▢, X\\ \tt Row 3: X, X, ▢, X, X\\ \tt Row 4: X, ▢, X, X, X\\ \tt Row 5: ▢, X, X, X, X\\ \tt \\ \tt Yes, that's correct. Each row has five elements. The ▢ is in different positions in each row. Let me note the positions of the ▢ in each row. Starting from the top:\\ \tt \\ \tt Row 
	  } 
	   } 
	   } 
	 & & \\ 
 

    \theutterance \stepcounter{utterance}  

    & & & \multicolumn{2}{p{0.3\linewidth}}{\cellcolor[rgb]{0.95,0.95,0.95}{%
	\makecell[{{p{\linewidth}}}]{% 
	  \tt {\tiny [GM$|$GM]}  
	 abort, first word: <reasoning>, but expected DESCRIPTION: 
	  } 
	   } 
	   } 
	 & & \\ 
 

    \theutterance \stepcounter{utterance}  

    & & & \multicolumn{2}{p{0.3\linewidth}}{\cellcolor[rgb]{0.95,0.95,0.95}{%
	\makecell[{{p{\linewidth}}}]{% 
	  \tt {\tiny [GM$|$GM]}  
	 Aborted 
	  } 
	   } 
	   } 
	 & & \\ 
 

\end{supertabular}
}

\end{document}

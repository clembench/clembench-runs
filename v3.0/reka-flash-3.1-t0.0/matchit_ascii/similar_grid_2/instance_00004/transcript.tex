
\documentclass{article}
\usepackage{colortbl}
\usepackage{makecell}
\usepackage{multirow}
\usepackage{supertabular}

\begin{document}

\newcounter{utterance}

\twocolumn

{ \footnotesize  \setcounter{utterance}{1}
\setlength{\tabcolsep}{0pt}
\begin{supertabular}{c@{$\;$}|p{.15\linewidth}@{}p{.15\linewidth}p{.15\linewidth}p{.15\linewidth}p{.15\linewidth}p{.15\linewidth}}

    \# & $\;$A & \multicolumn{4}{c}{Game Master} & $\;\:$B\\
    \hline 

    \theutterance \stepcounter{utterance}  

    & & \multicolumn{4}{p{0.6\linewidth}}{\cellcolor[rgb]{0.9,0.9,0.9}{%
	\makecell[{{p{\linewidth}}}]{% 
	  \tt {\tiny [A$\langle$GM]}  
	 You are participating in a collaborative guessing game.\\ \tt The goal is to find out whether this grid and another picture only I can see, are the same.\\ \tt Please describe your grid first. Then, I will provide my description and we can ask each other questions about the grids to figure out whether they are the same. Now start your short grid description with "DESCRIPTION:" followed by the description. Do not add anything else. This is your grid:\\ \tt X X X X X\\ \tt ▢ ▢ X ▢ ▢\\ \tt ▢ ▢ X ▢ ▢\\ \tt ▢ ▢ X ▢ ▢\\ \tt ▢ ▢ X ▢ ▢ 
	  } 
	   } 
	   } 
	 & \\ 
 

    \theutterance \stepcounter{utterance}  

    & \multicolumn{4}{p{0.6\linewidth}}{\cellcolor[rgb]{0.8,1,0.9}{%
	\makecell[{{p{\linewidth}}}]{% 
	  \tt {\tiny [A$\rangle$GM]}  
	 <reasoning>\\ \tt Let me try to figure out how to describe this grid properly. The user provided a grid with five rows. The first row is all X's: X X X X X. Then the next four rows have three X's in the middle and the rest are ▢. Let me count each row.\\ \tt \\ \tt First row: five X's. Second row: positions 1, 2, 4, 5 are ▢, and position 3 is X. Wait, no. Wait, the second row is written as "▢ ▢ X ▢ ▢". So that's five elements. So positions 1 and 2 are ▢, position 3 is X, positions 4 and 5 are ▢. Same for the third, fourth, and fifth rows. So each of those rows has an X in the third position and ▢ elsewhere.\\ \tt \\ \tt So the grid is 5x5. The first row is all X's. Rows 2-5 have X in the center (third column) and ▢ elsewhere. Let me visualize it:\\ \tt \\ \tt Row 1: X X X X X\\ \tt \\ \tt Row 2: ▢ ▢ X ▢ ▢\\ \tt \\ \tt Row 3: ▢ ▢ X ▢ ▢\\ \tt \\ \tt Row 4: ▢ ▢ X ▢ ▢\\ \tt \\ \tt Row 5: ▢ ▢ X ▢ ▢\\ \tt \\ \tt So the description should mention the size (5x5 
	  } 
	   } 
	   } 
	 & & \\ 
 

    \theutterance \stepcounter{utterance}  

    & & & \multicolumn{2}{p{0.3\linewidth}}{\cellcolor[rgb]{0.95,0.95,0.95}{%
	\makecell[{{p{\linewidth}}}]{% 
	  \tt {\tiny [GM$|$GM]}  
	 abort, first word: <reasoning>, but expected DESCRIPTION: 
	  } 
	   } 
	   } 
	 & & \\ 
 

    \theutterance \stepcounter{utterance}  

    & & & \multicolumn{2}{p{0.3\linewidth}}{\cellcolor[rgb]{0.95,0.95,0.95}{%
	\makecell[{{p{\linewidth}}}]{% 
	  \tt {\tiny [GM$|$GM]}  
	 Aborted 
	  } 
	   } 
	   } 
	 & & \\ 
 

\end{supertabular}
}

\end{document}


\documentclass{article}
\usepackage{colortbl}
\usepackage{makecell}
\usepackage{multirow}
\usepackage{supertabular}

\begin{document}

\newcounter{utterance}

\twocolumn

{ \footnotesize  \setcounter{utterance}{1}
\setlength{\tabcolsep}{0pt}
\begin{supertabular}{c@{$\;$}|p{.15\linewidth}@{}p{.15\linewidth}p{.15\linewidth}p{.15\linewidth}p{.15\linewidth}p{.15\linewidth}}

    \# & $\;$A & \multicolumn{4}{c}{Game Master} & $\;\:$B\\
    \hline 

    \theutterance \stepcounter{utterance}  

    & & \multicolumn{4}{p{0.6\linewidth}}{\cellcolor[rgb]{0.9,0.9,0.9}{%
	\makecell[{{p{\linewidth}}}]{% 
	  \tt {\tiny [A$\langle$GM]}  
	 You are participating in a collaborative guessing game.\\ \tt The goal is to find out whether this grid and another picture only I can see, are the same.\\ \tt Please describe your grid first. Then, I will provide my description and we can ask each other questions about the grids to figure out whether they are the same. Now start your short grid description with "DESCRIPTION:" followed by the description. Do not add anything else. This is your grid:\\ \tt X X X X X\\ \tt X ▢ X ▢ X\\ \tt X ▢ X ▢ X\\ \tt X ▢ X ▢ X\\ \tt X X X X X 
	  } 
	   } 
	   } 
	 & \\ 
 

    \theutterance \stepcounter{utterance}  

    & \multicolumn{4}{p{0.6\linewidth}}{\cellcolor[rgb]{0.8,1,0.9}{%
	\makecell[{{p{\linewidth}}}]{% 
	  \tt {\tiny [A$\rangle$GM]}  
	 <reasoning>\\ \tt Let me try to figure out how to describe this grid properly. The user provided a grid that looks like a 5x5 grid. Let me visualize it first.\\ \tt \\ \tt The first line is "X X X X X" which I assume are five X's in a row. Then the next lines are "X ▢ X ▢ X", repeating that for the third and fourth lines. The last line is again five X's. So the pattern seems to be a border of X's around the edges, and inside, there's a checkerboard pattern of X's and ▢s. Wait, let me check again.\\ \tt \\ \tt First row: all X's. Second row: X, then ▢, X, ▢, X. Third row same as second. Fourth row same. Fifth row all X's. So the outermost layer is solid X's, and the inner 3x3 area (rows 2-4 and columns 2-4) alternates between X and ▢ in a checkerboard. Wait, but in the given grid, the inner part is two X's and two ▢s alternately. Let me count:\\ \tt \\ \tt Row 2: X ▢ X ▢ X. So positions 1,3,5 are X and 2,4 are ▢. Similarly for row 3 and 4. So the inner 3x3 (if we consider the outer X's as the border) would 
	  } 
	   } 
	   } 
	 & & \\ 
 

    \theutterance \stepcounter{utterance}  

    & & & \multicolumn{2}{p{0.3\linewidth}}{\cellcolor[rgb]{0.95,0.95,0.95}{%
	\makecell[{{p{\linewidth}}}]{% 
	  \tt {\tiny [GM$|$GM]}  
	 abort, first word: <reasoning>, but expected DESCRIPTION: 
	  } 
	   } 
	   } 
	 & & \\ 
 

    \theutterance \stepcounter{utterance}  

    & & & \multicolumn{2}{p{0.3\linewidth}}{\cellcolor[rgb]{0.95,0.95,0.95}{%
	\makecell[{{p{\linewidth}}}]{% 
	  \tt {\tiny [GM$|$GM]}  
	 Aborted 
	  } 
	   } 
	   } 
	 & & \\ 
 

\end{supertabular}
}

\end{document}

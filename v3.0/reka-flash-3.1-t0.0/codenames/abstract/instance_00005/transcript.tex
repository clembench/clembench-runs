
\documentclass{article}
\usepackage{colortbl}
\usepackage{makecell}
\usepackage{multirow}
\usepackage{supertabular}

\begin{document}

\newcounter{utterance}

\twocolumn

{ \footnotesize  \setcounter{utterance}{1}
\setlength{\tabcolsep}{0pt}
\begin{supertabular}{c@{$\;$}|p{.15\linewidth}@{}p{.15\linewidth}p{.15\linewidth}p{.15\linewidth}p{.15\linewidth}p{.15\linewidth}}

    \# & $\;$A & \multicolumn{4}{c}{Game Master} & $\;\:$B\\
    \hline 

    \theutterance \stepcounter{utterance}  

    & & & \multicolumn{2}{p{0.3\linewidth}}{\cellcolor[rgb]{0.95,0.95,0.95}{%
	\makecell[{{p{\linewidth}}}]{% 
	  \tt {\tiny [GM$|$GM]}  
	 {'hidden': {'team': ['desecration', 'whim', 'mortality', 'unbelief', 'sarcasm', 'priority', 'reluctance', 'misbehavior', 'ethos'], 'innocent': ['steadfastness', 'urge', 'leniency', 'reaffirmation', 'overstatement', 'snobbery', 'hilarity'], 'opponent': ['deject', 'elitism', 'exclusiveness', 'timelessness', 'unconventionality', 'inadvertence', 'heyday', 'unfriendliness'], 'assassin': ['manner']}, 'revealed': {'team': {'team': [], 'innocent': [], 'opponent': [], 'assassin': []}, 'opponent': {'team': [], 'innocent': [], 'opponent': [], 'assassin': []}}} 
	  } 
	   } 
	   } 
	 & & \\ 
 

    \theutterance \stepcounter{utterance}  

    & & \multicolumn{4}{p{0.6\linewidth}}{\cellcolor[rgb]{0.9,0.9,0.9}{%
	\makecell[{{p{\linewidth}}}]{% 
	  \tt {\tiny [A$\langle$GM]}  
	 Provide one single-word clue that relates to some of your team's words from the list below. You can choose to how many and to which words your clue relates to.\\ \tt The goal is to have your partner guess all of your team's words based on the clues you give them, before the other team has uncovered all of theirs. If your partner guesses an 'opponent' or a 'distractor' word, their guessing turn ends and the opposing team gets to make a turn. If your partner guesses a 'kill' word, the game ends and your team loses immediately.\\ \tt The more words your clue relates to, the quicker you can win the game, but the harder the associations may be to guess for your partner, so choose your clue word wisely. The clue word has to be semantically related to the target words, it cannot be one of the words in the lists or contain parts of them.\\ \tt \\ \tt Always give your single-word clue and your comma-separated list of related target words in the following format and make your answers as short as possible, never include any other text than is required in this form:\\ \tt \\ \tt CLUE: <WORD>\\ \tt TARGETS: <TARGETS>\\ \tt \\ \tt Your team words are: desecration, whim, mortality, unbelief, sarcasm, priority, reluctance, misbehavior, ethos.\\ \tt Your opponent's team words are: deject, elitism, exclusiveness, timelessness, unconventionality, inadvertence, heyday, unfriendliness.\\ \tt Distractor words are: steadfastness, urge, leniency, reaffirmation, overstatement, snobbery, hilarity.\\ \tt The kill words are: manner. 
	  } 
	   } 
	   } 
	 & \\ 
 

    \theutterance \stepcounter{utterance}  

    & \multicolumn{4}{p{0.6\linewidth}}{\cellcolor[rgb]{0.8,1,0.9}{%
	\makecell[{{p{\linewidth}}}]{% 
	  \tt {\tiny [A$\rangle$GM]}  
	 <reasoning>\\ \tt Let's tackle this clue-giving problem. The user's team has words like desecration, whim, mortality, unbelief, sarcasm, priority, reluctance, misbehavior, ethos. I need to find a single-word clue that relates to as many of these as possible without being one of the words or containing parts of them.\\ \tt \\ \tt First, I'll look for common themes. Let's go through each word:\\ \tt \\ \tt - Desecration: Violation of something sacred or holy.\\ \tt - Whim: Sudden, unreasoned desire.\\ \tt - Mortality: The state of being mortal, death.\\ \tt - Unbelief: Lack of belief.\\ \tt - Sarcasm: Bitter mockery.\\ \tt - Priority: Importance in order of things.\\ \tt - Reluctance: Dislike to do something.\\ \tt - Misbehavior: Improper conduct.\\ \tt - Ethos: Ethical principles of a group.\\ \tt \\ \tt Looking for connections. Maybe something related to human nature or negative traits? Let's see:\\ \tt \\ \tt Desecration and unbelief could relate to religion or ethics. Mortality is about death. Sarcasm and misbehavior are about negative behaviors. Reluctance and priority might be about attitude or actions. Ethos is about principles.\\ \tt \\ \tt Wait, maybe "sin"? Desecration could be a sin, unbelief might be a sin, sarcasm and misbehavior as sins. But does that cover enough? Let's check:\\ \tt \\ \tt Desecration (sin), whim (maybe not), mortality (maybe 
	  } 
	   } 
	   } 
	 & & \\ 
 

    \theutterance \stepcounter{utterance}  

    & & & \multicolumn{2}{p{0.3\linewidth}}{\cellcolor[rgb]{0.95,0.95,0.95}{%
	\makecell[{{p{\linewidth}}}]{% 
	  \tt {\tiny [GM$|$GM]}  
	 {'player': 'Cluegiver', 'type': 'rambling error', 'utterance': '<reasoning>\nLet\'s tackle this clue-giving problem. The user\'s team has words like desecration, whim, mortality, unbelief, sarcasm, priority, reluctance, misbehavior, ethos. I need to find a single-word clue that relates to as many of these as possible without being one of the words or containing parts of them.\n\nFirst, I\'ll look for common themes. Let\'s go through each word:\n\n- Desecration: Violation of something sacred or holy.\n- Whim: Sudden, unreasoned desire.\n- Mortality: The state of being mortal, death.\n- Unbelief: Lack of belief.\n- Sarcasm: Bitter mockery.\n- Priority: Importance in order of things.\n- Reluctance: Dislike to do something.\n- Misbehavior: Improper conduct.\n- Ethos: Ethical principles of a group.\n\nLooking for connections. Maybe something related to human nature or negative traits? Let\'s see:\n\nDesecration and unbelief could relate to religion or ethics. Mortality is about death. Sarcasm and misbehavior are about negative behaviors. Reluctance and priority might be about attitude or actions. Ethos is about principles.\n\nWait, maybe "sin"? Desecration could be a sin, unbelief might be a sin, sarcasm and misbehavior as sins. But does that cover enough? Let\'s check:\n\nDesecration (sin), whim (maybe not), mortality (maybe', 'message': 'Your answer contained more than two lines, please only give one clue and your targets on two separate lines.'} 
	  } 
	   } 
	   } 
	 & & \\ 
 

\end{supertabular}
}

\end{document}

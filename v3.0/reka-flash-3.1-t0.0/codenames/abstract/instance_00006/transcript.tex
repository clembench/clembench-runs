
\documentclass{article}
\usepackage{colortbl}
\usepackage{makecell}
\usepackage{multirow}
\usepackage{supertabular}

\begin{document}

\newcounter{utterance}

\twocolumn

{ \footnotesize  \setcounter{utterance}{1}
\setlength{\tabcolsep}{0pt}
\begin{supertabular}{c@{$\;$}|p{.15\linewidth}@{}p{.15\linewidth}p{.15\linewidth}p{.15\linewidth}p{.15\linewidth}p{.15\linewidth}}

    \# & $\;$A & \multicolumn{4}{c}{Game Master} & $\;\:$B\\
    \hline 

    \theutterance \stepcounter{utterance}  

    & & & \multicolumn{2}{p{0.3\linewidth}}{\cellcolor[rgb]{0.95,0.95,0.95}{%
	\makecell[{{p{\linewidth}}}]{% 
	  \tt {\tiny [GM$|$GM]}  
	 {'hidden': {'team': ['merit', 'parallelism', 'expedience', 'impassion', 'superiority', 'imperativeness', 'interpretation', 'indecision', 'expertise'], 'innocent': ['bias', 'audacity', 'limitlessness', 'abolitionism', 'antihuman', 'despondency', 'heyday'], 'opponent': ['evasiveness', 'ineptitude', 'glorification', 'ethic', 'belief', 'unfaithfulness', 'cautiousness', 'deinstitutionalization'], 'assassin': ['implication']}, 'revealed': {'team': {'team': [], 'innocent': [], 'opponent': [], 'assassin': []}, 'opponent': {'team': [], 'innocent': [], 'opponent': [], 'assassin': []}}} 
	  } 
	   } 
	   } 
	 & & \\ 
 

    \theutterance \stepcounter{utterance}  

    & & \multicolumn{4}{p{0.6\linewidth}}{\cellcolor[rgb]{0.9,0.9,0.9}{%
	\makecell[{{p{\linewidth}}}]{% 
	  \tt {\tiny [A$\langle$GM]}  
	 Provide one single-word clue that relates to some of your team's words from the list below. You can choose to how many and to which words your clue relates to.\\ \tt The goal is to have your partner guess all of your team's words based on the clues you give them, before the other team has uncovered all of theirs. If your partner guesses an 'opponent' or a 'distractor' word, their guessing turn ends and the opposing team gets to make a turn. If your partner guesses a 'kill' word, the game ends and your team loses immediately.\\ \tt The more words your clue relates to, the quicker you can win the game, but the harder the associations may be to guess for your partner, so choose your clue word wisely. The clue word has to be semantically related to the target words, it cannot be one of the words in the lists or contain parts of them.\\ \tt \\ \tt Always give your single-word clue and your comma-separated list of related target words in the following format and make your answers as short as possible, never include any other text than is required in this form:\\ \tt \\ \tt CLUE: <WORD>\\ \tt TARGETS: <TARGETS>\\ \tt \\ \tt Your team words are: merit, parallelism, expedience, impassion, superiority, imperativeness, interpretation, indecision, expertise.\\ \tt Your opponent's team words are: evasiveness, ineptitude, glorification, ethic, belief, unfaithfulness, cautiousness, deinstitutionalization.\\ \tt Distractor words are: bias, audacity, limitlessness, abolitionism, antihuman, despondency, heyday.\\ \tt The kill words are: implication. 
	  } 
	   } 
	   } 
	 & \\ 
 

    \theutterance \stepcounter{utterance}  

    & \multicolumn{4}{p{0.6\linewidth}}{\cellcolor[rgb]{0.8,1,0.9}{%
	\makecell[{{p{\linewidth}}}]{% 
	  \tt {\tiny [A$\rangle$GM]}  
	 <reasoning>\\ \tt Let's tackle this clue-giving problem. The user's team has words like merit, parallelism, expedience, impassion, superiority, imperativeness, interpretation, indecision, expertise. I need to find a single-word clue that relates to as many of these as possible without being one of the words or containing parts of them.\\ \tt \\ \tt First, I'll look for common themes. Let's list out the words again:\\ \tt \\ \tt Merit: worth, value, deserving.\\ \tt Parallelism: similarity in structure, parallel lines, maybe repetition.\\ \tt Expedience: practicality, doing something for practical reasons.\\ \tt Impassion: to inspire, move emotionally.\\ \tt Superiority: being better, higher status.\\ \tt Imperativeness: being urgent, commanding.\\ \tt Interpretation: understanding, meaning.\\ \tt Indecision: inability to decide.\\ \tt Expertise: knowledge, skill.\\ \tt \\ \tt Looking for a common thread. Maybe something related to "force" or "pressure"? Imperativeness is about urgency, maybe "command"? But "command" might not cover all. Let's see:\\ \tt \\ \tt Merit could relate to "reward" or "value". Parallelism might be "structure" or "pattern". Expedience is about practicality, maybe "useful". Impassion is about persuasion, "inspire". Superiority could be "dominance". Imperativeness is "urgent". Interpretation is "understand". Indecision is "uncertainty". Expertise is "knowledge".\\ \tt \\ \tt Hmm 
	  } 
	   } 
	   } 
	 & & \\ 
 

    \theutterance \stepcounter{utterance}  

    & & & \multicolumn{2}{p{0.3\linewidth}}{\cellcolor[rgb]{0.95,0.95,0.95}{%
	\makecell[{{p{\linewidth}}}]{% 
	  \tt {\tiny [GM$|$GM]}  
	 {'player': 'Cluegiver', 'type': 'rambling error', 'utterance': '<reasoning>\nLet\'s tackle this clue-giving problem. The user\'s team has words like merit, parallelism, expedience, impassion, superiority, imperativeness, interpretation, indecision, expertise. I need to find a single-word clue that relates to as many of these as possible without being one of the words or containing parts of them.\n\nFirst, I\'ll look for common themes. Let\'s list out the words again:\n\nMerit: worth, value, deserving.\nParallelism: similarity in structure, parallel lines, maybe repetition.\nExpedience: practicality, doing something for practical reasons.\nImpassion: to inspire, move emotionally.\nSuperiority: being better, higher status.\nImperativeness: being urgent, commanding.\nInterpretation: understanding, meaning.\nIndecision: inability to decide.\nExpertise: knowledge, skill.\n\nLooking for a common thread. Maybe something related to "force" or "pressure"? Imperativeness is about urgency, maybe "command"? But "command" might not cover all. Let\'s see:\n\nMerit could relate to "reward" or "value". Parallelism might be "structure" or "pattern". Expedience is about practicality, maybe "useful". Impassion is about persuasion, "inspire". Superiority could be "dominance". Imperativeness is "urgent". Interpretation is "understand". Indecision is "uncertainty". Expertise is "knowledge".\n\nHmm', 'message': 'Your answer contained more than two lines, please only give one clue and your targets on two separate lines.'} 
	  } 
	   } 
	   } 
	 & & \\ 
 

\end{supertabular}
}

\end{document}

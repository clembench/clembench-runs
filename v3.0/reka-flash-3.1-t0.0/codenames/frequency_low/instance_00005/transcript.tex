
\documentclass{article}
\usepackage{colortbl}
\usepackage{makecell}
\usepackage{multirow}
\usepackage{supertabular}

\begin{document}

\newcounter{utterance}

\twocolumn

{ \footnotesize  \setcounter{utterance}{1}
\setlength{\tabcolsep}{0pt}
\begin{supertabular}{c@{$\;$}|p{.15\linewidth}@{}p{.15\linewidth}p{.15\linewidth}p{.15\linewidth}p{.15\linewidth}p{.15\linewidth}}

    \# & $\;$A & \multicolumn{4}{c}{Game Master} & $\;\:$B\\
    \hline 

    \theutterance \stepcounter{utterance}  

    & & & \multicolumn{2}{p{0.3\linewidth}}{\cellcolor[rgb]{0.95,0.95,0.95}{%
	\makecell[{{p{\linewidth}}}]{% 
	  \tt {\tiny [GM$|$GM]}  
	 {'hidden': {'team': ['housewifery', 'defenseman', 'inflexibility', 'canola', 'candlepower', 'cattiness', 'playact', 'battleaxe', 'assassinator'], 'innocent': ['pockmark', 'gingersnap', 'ineptness', 'forestation', 'hyperthermia', 'undulation', 'cauterization'], 'opponent': ['dwarfism', 'seafloor', 'plantain', 'decisiveness', 'multidimensionality', 'dangerousness', 'undernourishment', 'pleasantness'], 'assassin': ['towhead']}, 'revealed': {'team': {'team': [], 'innocent': [], 'opponent': [], 'assassin': []}, 'opponent': {'team': [], 'innocent': [], 'opponent': [], 'assassin': []}}} 
	  } 
	   } 
	   } 
	 & & \\ 
 

    \theutterance \stepcounter{utterance}  

    & & \multicolumn{4}{p{0.6\linewidth}}{\cellcolor[rgb]{0.9,0.9,0.9}{%
	\makecell[{{p{\linewidth}}}]{% 
	  \tt {\tiny [A$\langle$GM]}  
	 Provide one single-word clue that relates to some of your team's words from the list below. You can choose to how many and to which words your clue relates to.\\ \tt The goal is to have your partner guess all of your team's words based on the clues you give them, before the other team has uncovered all of theirs. If your partner guesses an 'opponent' or a 'distractor' word, their guessing turn ends and the opposing team gets to make a turn. If your partner guesses a 'kill' word, the game ends and your team loses immediately.\\ \tt The more words your clue relates to, the quicker you can win the game, but the harder the associations may be to guess for your partner, so choose your clue word wisely. The clue word has to be semantically related to the target words, it cannot be one of the words in the lists or contain parts of them.\\ \tt \\ \tt Always give your single-word clue and your comma-separated list of related target words in the following format and make your answers as short as possible, never include any other text than is required in this form:\\ \tt \\ \tt CLUE: <WORD>\\ \tt TARGETS: <TARGETS>\\ \tt \\ \tt Your team words are: housewifery, defenseman, inflexibility, canola, candlepower, cattiness, playact, battleaxe, assassinator.\\ \tt Your opponent's team words are: dwarfism, seafloor, plantain, decisiveness, multidimensionality, dangerousness, undernourishment, pleasantness.\\ \tt Distractor words are: pockmark, gingersnap, ineptness, forestation, hyperthermia, undulation, cauterization.\\ \tt The kill words are: towhead. 
	  } 
	   } 
	   } 
	 & \\ 
 

    \theutterance \stepcounter{utterance}  

    & \multicolumn{4}{p{0.6\linewidth}}{\cellcolor[rgb]{0.8,1,0.9}{%
	\makecell[{{p{\linewidth}}}]{% 
	  \tt {\tiny [A$\rangle$GM]}  
	 <reasoning>\\ \tt Let's tackle this clue-giving problem. The user's team has nine words: housewifery, defenseman, inflexibility, canola, candlepower, cattiness, playact, battleaxe, assassinator. I need to find a single-word clue that relates to as many of these as possible without being one of the words or containing parts of them.\\ \tt \\ \tt First, I'll go through each word to see possible connections. Let's list them again:\\ \tt \\ \tt 1. housewifery - the role or duties of a housewife. Maybe related to home, cooking, family?\\ \tt 2. defenseman - a hockey position, protects the net. Could relate to sports, defense, ice hockey.\\ \tt 3. inflexibility - rigidity, inability to bend. Maybe stiffness, rigidity, rigour?\\ \tt 4. canola - a type of oilseed, related to oil, plants, agriculture.\\ \tt 5. candlepower - a measure of light output. Light, brightness, energy?\\ \tt 6. cattiness - a trait of being catty, mean, or bitchy. Personality trait, behavior.\\ \tt 7. playact - to pretend, act, perform. Acting, drama, performance.\\ \tt 8. battleaxe - a weapon, old-fashioned axe used in combat. Violence, weapon, history.\\ \tt 9. assassinator - someone who assassinates, murder, killing.\\ \tt \\ \tt Looking for common themes. Let's see:\\ \tt \\ \tt - housewifery and canola: both 
	  } 
	   } 
	   } 
	 & & \\ 
 

    \theutterance \stepcounter{utterance}  

    & & & \multicolumn{2}{p{0.3\linewidth}}{\cellcolor[rgb]{0.95,0.95,0.95}{%
	\makecell[{{p{\linewidth}}}]{% 
	  \tt {\tiny [GM$|$GM]}  
	 {'player': 'Cluegiver', 'type': 'rambling error', 'utterance': "<reasoning>\nLet's tackle this clue-giving problem. The user's team has nine words: housewifery, defenseman, inflexibility, canola, candlepower, cattiness, playact, battleaxe, assassinator. I need to find a single-word clue that relates to as many of these as possible without being one of the words or containing parts of them.\n\nFirst, I'll go through each word to see possible connections. Let's list them again:\n\n1. housewifery - the role or duties of a housewife. Maybe related to home, cooking, family?\n2. defenseman - a hockey position, protects the net. Could relate to sports, defense, ice hockey.\n3. inflexibility - rigidity, inability to bend. Maybe stiffness, rigidity, rigour?\n4. canola - a type of oilseed, related to oil, plants, agriculture.\n5. candlepower - a measure of light output. Light, brightness, energy?\n6. cattiness - a trait of being catty, mean, or bitchy. Personality trait, behavior.\n7. playact - to pretend, act, perform. Acting, drama, performance.\n8. battleaxe - a weapon, old-fashioned axe used in combat. Violence, weapon, history.\n9. assassinator - someone who assassinates, murder, killing.\n\nLooking for common themes. Let's see:\n\n- housewifery and canola: both", 'message': 'Your answer contained more than two lines, please only give one clue and your targets on two separate lines.'} 
	  } 
	   } 
	   } 
	 & & \\ 
 

\end{supertabular}
}

\end{document}


\documentclass{article}
\usepackage{colortbl}
\usepackage{makecell}
\usepackage{multirow}
\usepackage{supertabular}

\begin{document}

\newcounter{utterance}

\twocolumn

{ \footnotesize  \setcounter{utterance}{1}
\setlength{\tabcolsep}{0pt}
\begin{supertabular}{c@{$\;$}|p{.15\linewidth}@{}p{.15\linewidth}p{.15\linewidth}p{.15\linewidth}p{.15\linewidth}p{.15\linewidth}}

    \# & $\;$A & \multicolumn{4}{c}{Game Master} & $\;\:$B\\
    \hline 

    \theutterance \stepcounter{utterance}  

    & & \multicolumn{4}{p{0.6\linewidth}}{\cellcolor[rgb]{0.9,0.9,0.9}{%
	\makecell[{{p{\linewidth}}}]{% 
	  \tt {\tiny [A$\langle$GM]}  
	 You are playing a collaborative word guessing game in which you have to describe a target word for another player to guess.\\ \tt \\ \tt Rules:\\ \tt (a) You have to reply in the form: CLUE: <some text>. Guesses from the other player will start with GUESS.\\ \tt (b) You cannot use the target word itself, parts or morphological variants of it in your description.\\ \tt (c) In addition, the same rules apply for related words which are provided below.\\ \tt \\ \tt End conditions:\\ \tt (i) If you use the target word or a related word in your description, then you lose.\\ \tt (ii) If the other player can guess the target word in 3 tries, you both win.\\ \tt \\ \tt Let us start.\\ \tt \\ \tt This is the target word that you need to describe and that the other player needs to guess:\\ \tt \\ \tt surprise\\ \tt \\ \tt Related words are:\\ \tt \\ \tt - party\\ \tt - shock\\ \tt - unexpected\\ \tt \\ \tt Important: You are under time pressure, give short descriptions that are to the point! 
	  } 
	   } 
	   } 
	 & \\ 
 

    \theutterance \stepcounter{utterance}  

    & \multicolumn{4}{p{0.6\linewidth}}{\cellcolor[rgb]{0.8,1,0.9}{%
	\makecell[{{p{\linewidth}}}]{% 
	  \tt {\tiny [A$\rangle$GM]}  
	 <reasoning>\\ \tt I need to come up with a clue for the word "surprise" without using the word itself, any parts of it, or related words like party, shock, or unexpected. Let me think about synonyms or related concepts.\\ \tt \\ \tt Surprise often involves unexpected events, maybe something that catches you off guard. But I can't use "unexpected" since it's a related word. How about using an action that's associated with surprise? Like gasping, jumping, or maybe a reaction. Wait, maybe something visual. Oh, like a "look of astonishment" but that might be too close. Or maybe something like "eyes wide open" but that's still descriptive. Alternatively, think of objects that represent surprise. A "lightbulb moment" is about realization, not exactly surprise. Maybe a "jaw dropping"? That's a common expression. Let me check if "jaw" is allowed. The target word is "surprise," so "jaw" isn't part of it. But "jaw dropping" is an idiom. Wait, the rules say I can't use morphological variants. "Surprise" is the target, so "surprised" is a variant. But "jaw" is a separate word. Hmm. Maybe "jaw dropping" is okay. Alternatively, think of something like a "startled expression" but "startled" is related to shock, which is a related word. So I can't use that. 
	  } 
	   } 
	   } 
	 & & \\ 
 

    \theutterance \stepcounter{utterance}  

    & & & \multicolumn{2}{p{0.3\linewidth}}{\cellcolor[rgb]{0.95,0.95,0.95}{%
	\makecell[{{p{\linewidth}}}]{% 
	  \tt {\tiny [GM$|$GM]}  
	 abort game 
	  } 
	   } 
	   } 
	 & & \\ 
 

\end{supertabular}
}

\end{document}


\documentclass{article}
\usepackage{colortbl}
\usepackage{makecell}
\usepackage{multirow}
\usepackage{supertabular}

\begin{document}

\newcounter{utterance}

\twocolumn

{ \footnotesize  \setcounter{utterance}{1}
\setlength{\tabcolsep}{0pt}
\begin{supertabular}{c@{$\;$}|p{.15\linewidth}@{}p{.15\linewidth}p{.15\linewidth}p{.15\linewidth}p{.15\linewidth}p{.15\linewidth}}

    \# & $\;$A & \multicolumn{4}{c}{Game Master} & $\;\:$B\\
    \hline 

    \theutterance \stepcounter{utterance}  

    & & \multicolumn{4}{p{0.6\linewidth}}{\cellcolor[rgb]{0.9,0.9,0.9}{%
	\makecell[{{p{\linewidth}}}]{% 
	  \tt {\tiny [A$\langle$GM]}  
	 You are playing a collaborative word guessing game in which you have to describe a target word for another player to guess.\\ \tt \\ \tt Rules:\\ \tt (a) You have to reply in the form: CLUE: <some text>. Guesses from the other player will start with GUESS.\\ \tt (b) You cannot use the target word itself, parts or morphological variants of it in your description.\\ \tt (c) In addition, the same rules apply for related words which are provided below.\\ \tt \\ \tt End conditions:\\ \tt (i) If you use the target word or a related word in your description, then you lose.\\ \tt (ii) If the other player can guess the target word in 3 tries, you both win.\\ \tt \\ \tt Let us start.\\ \tt \\ \tt This is the target word that you need to describe and that the other player needs to guess:\\ \tt \\ \tt nick\\ \tt \\ \tt Related words are:\\ \tt \\ \tt - groove\\ \tt - dent\\ \tt - cut\\ \tt \\ \tt Important: You are under time pressure, give short descriptions that are to the point! 
	  } 
	   } 
	   } 
	 & \\ 
 

    \theutterance \stepcounter{utterance}  

    & \multicolumn{4}{p{0.6\linewidth}}{\cellcolor[rgb]{0.8,1,0.9}{%
	\makecell[{{p{\linewidth}}}]{% 
	  \tt {\tiny [A$\rangle$GM]}  
	 <reasoning>\\ \tt Let's tackle this. The target word is "nick". I need to come up with a clue that doesn't use "nick" itself or any related words like groove, dent, or cut. The clue has to be short and to the point because of the time pressure.\\ \tt \\ \tt First, what does "nick" mean? It's a small cut or scratch, right? But I can't use "cut" or "dent". So maybe think of other words that describe a small mark. Words like scratch, chip, or maybe even a slight injury. But scratch is a bit close to cut, maybe? Wait, the rules say I can't use morphological variants. So "scratch" is a verb, but "nick" is a noun. Maybe that's okay? Let me check. The rules say parts or morphological variants. "Nick" comes from "nicker", which is a variant. So scratch might be okay. But maybe safer to avoid.\\ \tt \\ \tt Alternatively, think of the sound. Like when something makes a small scraping noise? Or maybe a small mark on something, like a coin. Wait, a "nick" on a coin is a small mark. So maybe "A mark on a coin that's not a dent"? But "dent" is a related word, so I can't use that. Hmm.\\ \tt \\ \tt Another angle: "nick" can also mean to steal a little, like nicking something. But the primary meaning here is the 
	  } 
	   } 
	   } 
	 & & \\ 
 

    \theutterance \stepcounter{utterance}  

    & & & \multicolumn{2}{p{0.3\linewidth}}{\cellcolor[rgb]{0.95,0.95,0.95}{%
	\makecell[{{p{\linewidth}}}]{% 
	  \tt {\tiny [GM$|$GM]}  
	 abort game 
	  } 
	   } 
	   } 
	 & & \\ 
 

\end{supertabular}
}

\end{document}


\documentclass{article}
\usepackage{colortbl}
\usepackage{makecell}
\usepackage{multirow}
\usepackage{supertabular}

\begin{document}

\newcounter{utterance}

\twocolumn

{ \footnotesize  \setcounter{utterance}{1}
\setlength{\tabcolsep}{0pt}
\begin{supertabular}{c@{$\;$}|p{.15\linewidth}@{}p{.15\linewidth}p{.15\linewidth}p{.15\linewidth}p{.15\linewidth}p{.15\linewidth}}

    \# & $\;$A & \multicolumn{4}{c}{Game Master} & $\;\:$B\\
    \hline 

    \theutterance \stepcounter{utterance}  

    & & \multicolumn{4}{p{0.6\linewidth}}{\cellcolor[rgb]{0.9,0.9,0.9}{%
	\makecell[{{p{\linewidth}}}]{% 
	  \tt {\tiny [A$\langle$GM]}  
	 You are playing a text adventure game. I will describe what you can perceive in the game. You write the single action you want to take in the game starting with >. Only reply with an action.\\ \tt For example:\\ \tt > open cupboard\\ \tt \\ \tt You can have up to two objects in your inventory at the same time.\\ \tt \\ \tt Your goal for this game is: Put the broom on the shelf, the potted plant on the counter and the plate in the wardrobe.\\ \tt \\ \tt Once you have achieved your goal, write "> done" to end the game.\\ \tt \\ \tt You are in a living room now. There are a couch, a chair, a book and a side table. The book is on the side table. There are passages to a bedroom, a hallway and a kitchen here. 
	  } 
	   } 
	   } 
	 & \\ 
 

    \theutterance \stepcounter{utterance}  

    & \multicolumn{4}{p{0.6\linewidth}}{\cellcolor[rgb]{0.8,1,0.9}{%
	\makecell[{{p{\linewidth}}}]{% 
	  \tt {\tiny [A$\rangle$GM]}  
	 > take book 
	  } 
	   } 
	   } 
	 & & \\ 
 

    \theutterance \stepcounter{utterance}  

    & & & \multicolumn{2}{p{0.3\linewidth}}{\cellcolor[rgb]{0.95,0.95,0.95}{%
	\makecell[{{p{\linewidth}}}]{% 
	  \tt {\tiny [GM$|$GM]}  
	 Round: 0 
	  } 
	   } 
	   } 
	 & & \\ 
 

    \theutterance \stepcounter{utterance}  

    & & & \multicolumn{2}{p{0.3\linewidth}}{\cellcolor[rgb]{0.95,0.95,0.95}{%
	\makecell[{{p{\linewidth}}}]{% 
	  \tt {\tiny [GM$|$GM]}  
	 {'action_type': 'take', 'exploration_info': {'action_epistemic': False, 'action_pragmatic': True, 'effective_epistemic_gain_amount': 0, 'known_entities': [['at', 'book1', 'livingroom1'], ['at', 'chair1', 'livingroom1'], ['at', 'couch1', 'livingroom1'], ['at', 'player1', 'livingroom1'], ['at', 'sidetable1', 'livingroom1']], 'known_entities_ratio': 0.1724137931034483, 'visited_rooms': ['livingroom1'], 'visited_rooms_ratio': 0.16666666666666666, 'known_goal_entities': [], 'known_goal_entities_ratio': 0.0}} 
	  } 
	   } 
	   } 
	 & & \\ 
 

    \theutterance \stepcounter{utterance}  

    & & & \multicolumn{2}{p{0.3\linewidth}}{\cellcolor[rgb]{0.95,0.95,0.95}{%
	\makecell[{{p{\linewidth}}}]{% 
	  \tt {\tiny [GM$|$GM]}  
	 {'goal_states_achieved': [], 'turn_goal_score': 0} 
	  } 
	   } 
	   } 
	 & & \\ 
 

    \theutterance \stepcounter{utterance}  

    & & \multicolumn{4}{p{0.6\linewidth}}{\cellcolor[rgb]{0.9,0.9,0.9}{%
	\makecell[{{p{\linewidth}}}]{% 
	  \tt {\tiny [A$\langle$GM]}  
	 You take the book. In your inventory you have a book.\\ \tt  
	  } 
	   } 
	   } 
	 & \\ 
 

    \theutterance \stepcounter{utterance}  

    & \multicolumn{4}{p{0.6\linewidth}}{\cellcolor[rgb]{0.8,1,0.9}{%
	\makecell[{{p{\linewidth}}}]{% 
	  \tt {\tiny [A$\rangle$GM]}  
	 > go to kitchen 
	  } 
	   } 
	   } 
	 & & \\ 
 

    \theutterance \stepcounter{utterance}  

    & & & \multicolumn{2}{p{0.3\linewidth}}{\cellcolor[rgb]{0.95,0.95,0.95}{%
	\makecell[{{p{\linewidth}}}]{% 
	  \tt {\tiny [GM$|$GM]}  
	 Round: 1 
	  } 
	   } 
	   } 
	 & & \\ 
 

    \theutterance \stepcounter{utterance}  

    & & & \multicolumn{2}{p{0.3\linewidth}}{\cellcolor[rgb]{0.95,0.95,0.95}{%
	\makecell[{{p{\linewidth}}}]{% 
	  \tt {\tiny [GM$|$GM]}  
	 {'action_type': 'go', 'exploration_info': {'action_epistemic': True, 'action_pragmatic': True, 'effective_epistemic_gain_facts': [['closed', 'cupboard1'], ['at', 'cupboard1', 'kitchen1'], ['in', 'plate1', 'cupboard1'], ['exit', 'kitchen1', 'livingroom1'], ['at', 'player1', 'kitchen1'], ['at', 'plate1', 'kitchen1'], ['at', 'counter1', 'kitchen1'], ['exit', 'kitchen1', 'pantry1'], ['at', 'book1', 'kitchen1']], 'effective_epistemic_gain_amount': 9, 'known_entities': [['at', 'cupboard1', 'kitchen1'], ['at', 'chair1', 'livingroom1'], ['at', 'player1', 'kitchen1'], ['at', 'plate1', 'kitchen1'], ['at', 'couch1', 'livingroom1'], ['at', 'counter1', 'kitchen1'], ['at', 'sidetable1', 'livingroom1'], ['at', 'book1', 'kitchen1']], 'known_entities_ratio': 0.27586206896551724, 'visited_rooms': ['kitchen1', 'livingroom1'], 'visited_rooms_ratio': 0.3333333333333333, 'known_goal_entities': [['at', 'plate1', 'kitchen1'], ['at', 'counter1', 'kitchen1']], 'known_goal_entities_ratio': 0.3333333333333333}} 
	  } 
	   } 
	   } 
	 & & \\ 
 

    \theutterance \stepcounter{utterance}  

    & & & \multicolumn{2}{p{0.3\linewidth}}{\cellcolor[rgb]{0.95,0.95,0.95}{%
	\makecell[{{p{\linewidth}}}]{% 
	  \tt {\tiny [GM$|$GM]}  
	 {'goal_states_achieved': [], 'turn_goal_score': 0} 
	  } 
	   } 
	   } 
	 & & \\ 
 

    \theutterance \stepcounter{utterance}  

    & & \multicolumn{4}{p{0.6\linewidth}}{\cellcolor[rgb]{0.9,0.9,0.9}{%
	\makecell[{{p{\linewidth}}}]{% 
	  \tt {\tiny [A$\langle$GM]}  
	 You are in a kitchen now. There are a cupboard, a plate and a counter. The cupboard is closed. The plate is in the cupboard. There are passages to a living room and a pantry here.\\ \tt  
	  } 
	   } 
	   } 
	 & \\ 
 

    \theutterance \stepcounter{utterance}  

    & \multicolumn{4}{p{0.6\linewidth}}{\cellcolor[rgb]{0.8,1,0.9}{%
	\makecell[{{p{\linewidth}}}]{% 
	  \tt {\tiny [A$\rangle$GM]}  
	 > open cupboard 
	  } 
	   } 
	   } 
	 & & \\ 
 

    \theutterance \stepcounter{utterance}  

    & & & \multicolumn{2}{p{0.3\linewidth}}{\cellcolor[rgb]{0.95,0.95,0.95}{%
	\makecell[{{p{\linewidth}}}]{% 
	  \tt {\tiny [GM$|$GM]}  
	 Round: 2 
	  } 
	   } 
	   } 
	 & & \\ 
 

    \theutterance \stepcounter{utterance}  

    & & & \multicolumn{2}{p{0.3\linewidth}}{\cellcolor[rgb]{0.95,0.95,0.95}{%
	\makecell[{{p{\linewidth}}}]{% 
	  \tt {\tiny [GM$|$GM]}  
	 {'action_type': 'open', 'exploration_info': {'action_epistemic': True, 'action_pragmatic': True, 'effective_epistemic_gain_facts': [['open', 'cupboard1']], 'effective_epistemic_gain_amount': 1, 'known_entities': [['at', 'cupboard1', 'kitchen1'], ['at', 'chair1', 'livingroom1'], ['at', 'player1', 'kitchen1'], ['at', 'plate1', 'kitchen1'], ['at', 'couch1', 'livingroom1'], ['at', 'counter1', 'kitchen1'], ['at', 'sidetable1', 'livingroom1'], ['at', 'book1', 'kitchen1']], 'known_entities_ratio': 0.27586206896551724, 'visited_rooms': ['kitchen1', 'livingroom1'], 'visited_rooms_ratio': 0.3333333333333333, 'known_goal_entities': [['at', 'plate1', 'kitchen1'], ['at', 'counter1', 'kitchen1']], 'known_goal_entities_ratio': 0.3333333333333333}} 
	  } 
	   } 
	   } 
	 & & \\ 
 

    \theutterance \stepcounter{utterance}  

    & & & \multicolumn{2}{p{0.3\linewidth}}{\cellcolor[rgb]{0.95,0.95,0.95}{%
	\makecell[{{p{\linewidth}}}]{% 
	  \tt {\tiny [GM$|$GM]}  
	 {'goal_states_achieved': [], 'turn_goal_score': 0} 
	  } 
	   } 
	   } 
	 & & \\ 
 

    \theutterance \stepcounter{utterance}  

    & & \multicolumn{4}{p{0.6\linewidth}}{\cellcolor[rgb]{0.9,0.9,0.9}{%
	\makecell[{{p{\linewidth}}}]{% 
	  \tt {\tiny [A$\langle$GM]}  
	 The cupboard is now open. In the cupboard there is a plate.\\ \tt  
	  } 
	   } 
	   } 
	 & \\ 
 

    \theutterance \stepcounter{utterance}  

    & \multicolumn{4}{p{0.6\linewidth}}{\cellcolor[rgb]{0.8,1,0.9}{%
	\makecell[{{p{\linewidth}}}]{% 
	  \tt {\tiny [A$\rangle$GM]}  
	 > take plate 
	  } 
	   } 
	   } 
	 & & \\ 
 

    \theutterance \stepcounter{utterance}  

    & & & \multicolumn{2}{p{0.3\linewidth}}{\cellcolor[rgb]{0.95,0.95,0.95}{%
	\makecell[{{p{\linewidth}}}]{% 
	  \tt {\tiny [GM$|$GM]}  
	 Round: 3 
	  } 
	   } 
	   } 
	 & & \\ 
 

    \theutterance \stepcounter{utterance}  

    & & & \multicolumn{2}{p{0.3\linewidth}}{\cellcolor[rgb]{0.95,0.95,0.95}{%
	\makecell[{{p{\linewidth}}}]{% 
	  \tt {\tiny [GM$|$GM]}  
	 {'action_type': 'take', 'exploration_info': {'action_epistemic': False, 'action_pragmatic': True, 'effective_epistemic_gain_amount': 0, 'known_entities': [['at', 'cupboard1', 'kitchen1'], ['at', 'chair1', 'livingroom1'], ['at', 'player1', 'kitchen1'], ['at', 'plate1', 'kitchen1'], ['at', 'couch1', 'livingroom1'], ['at', 'counter1', 'kitchen1'], ['at', 'sidetable1', 'livingroom1'], ['at', 'book1', 'kitchen1']], 'known_entities_ratio': 0.27586206896551724, 'visited_rooms': ['kitchen1', 'livingroom1'], 'visited_rooms_ratio': 0.3333333333333333, 'known_goal_entities': [['at', 'plate1', 'kitchen1'], ['at', 'counter1', 'kitchen1']], 'known_goal_entities_ratio': 0.3333333333333333}} 
	  } 
	   } 
	   } 
	 & & \\ 
 

    \theutterance \stepcounter{utterance}  

    & & & \multicolumn{2}{p{0.3\linewidth}}{\cellcolor[rgb]{0.95,0.95,0.95}{%
	\makecell[{{p{\linewidth}}}]{% 
	  \tt {\tiny [GM$|$GM]}  
	 {'goal_states_achieved': [], 'turn_goal_score': 0} 
	  } 
	   } 
	   } 
	 & & \\ 
 

    \theutterance \stepcounter{utterance}  

    & & \multicolumn{4}{p{0.6\linewidth}}{\cellcolor[rgb]{0.9,0.9,0.9}{%
	\makecell[{{p{\linewidth}}}]{% 
	  \tt {\tiny [A$\langle$GM]}  
	 You take the plate. In your inventory you have a plate and a book.\\ \tt  
	  } 
	   } 
	   } 
	 & \\ 
 

    \theutterance \stepcounter{utterance}  

    & \multicolumn{4}{p{0.6\linewidth}}{\cellcolor[rgb]{0.8,1,0.9}{%
	\makecell[{{p{\linewidth}}}]{% 
	  \tt {\tiny [A$\rangle$GM]}  
	 > go to living room 
	  } 
	   } 
	   } 
	 & & \\ 
 

    \theutterance \stepcounter{utterance}  

    & & & \multicolumn{2}{p{0.3\linewidth}}{\cellcolor[rgb]{0.95,0.95,0.95}{%
	\makecell[{{p{\linewidth}}}]{% 
	  \tt {\tiny [GM$|$GM]}  
	 Round: 4 
	  } 
	   } 
	   } 
	 & & \\ 
 

    \theutterance \stepcounter{utterance}  

    & & & \multicolumn{2}{p{0.3\linewidth}}{\cellcolor[rgb]{0.95,0.95,0.95}{%
	\makecell[{{p{\linewidth}}}]{% 
	  \tt {\tiny [GM$|$GM]}  
	 {'action_type': 'go', 'exploration_info': {'action_epistemic': True, 'action_pragmatic': True, 'effective_epistemic_gain_facts': [['at', 'player1', 'livingroom1'], ['at', 'book1', 'livingroom1'], ['at', 'plate1', 'livingroom1']], 'effective_epistemic_gain_amount': 3, 'known_entities': [['at', 'plate1', 'livingroom1'], ['at', 'cupboard1', 'kitchen1'], ['at', 'book1', 'livingroom1'], ['at', 'chair1', 'livingroom1'], ['at', 'couch1', 'livingroom1'], ['at', 'counter1', 'kitchen1'], ['at', 'player1', 'livingroom1'], ['at', 'sidetable1', 'livingroom1']], 'known_entities_ratio': 0.27586206896551724, 'visited_rooms': ['kitchen1', 'livingroom1'], 'visited_rooms_ratio': 0.3333333333333333, 'known_goal_entities': [['at', 'plate1', 'livingroom1'], ['at', 'counter1', 'kitchen1']], 'known_goal_entities_ratio': 0.3333333333333333}} 
	  } 
	   } 
	   } 
	 & & \\ 
 

    \theutterance \stepcounter{utterance}  

    & & & \multicolumn{2}{p{0.3\linewidth}}{\cellcolor[rgb]{0.95,0.95,0.95}{%
	\makecell[{{p{\linewidth}}}]{% 
	  \tt {\tiny [GM$|$GM]}  
	 {'goal_states_achieved': [], 'turn_goal_score': 0} 
	  } 
	   } 
	   } 
	 & & \\ 
 

    \theutterance \stepcounter{utterance}  

    & & \multicolumn{4}{p{0.6\linewidth}}{\cellcolor[rgb]{0.9,0.9,0.9}{%
	\makecell[{{p{\linewidth}}}]{% 
	  \tt {\tiny [A$\langle$GM]}  
	 You are in a living room now. There are a couch, a chair and a side table. There are passages to a bedroom, a hallway and a kitchen here.\\ \tt  
	  } 
	   } 
	   } 
	 & \\ 
 

    \theutterance \stepcounter{utterance}  

    & \multicolumn{4}{p{0.6\linewidth}}{\cellcolor[rgb]{0.8,1,0.9}{%
	\makecell[{{p{\linewidth}}}]{% 
	  \tt {\tiny [A$\rangle$GM]}  
	 > go to bedroom 
	  } 
	   } 
	   } 
	 & & \\ 
 

    \theutterance \stepcounter{utterance}  

    & & & \multicolumn{2}{p{0.3\linewidth}}{\cellcolor[rgb]{0.95,0.95,0.95}{%
	\makecell[{{p{\linewidth}}}]{% 
	  \tt {\tiny [GM$|$GM]}  
	 Round: 5 
	  } 
	   } 
	   } 
	 & & \\ 
 

    \theutterance \stepcounter{utterance}  

    & & & \multicolumn{2}{p{0.3\linewidth}}{\cellcolor[rgb]{0.95,0.95,0.95}{%
	\makecell[{{p{\linewidth}}}]{% 
	  \tt {\tiny [GM$|$GM]}  
	 {'action_type': 'go', 'exploration_info': {'action_epistemic': True, 'action_pragmatic': True, 'effective_epistemic_gain_facts': [['at', 'player1', 'bedroom1'], ['at', 'plate1', 'bedroom1'], ['at', 'wardrobe1', 'bedroom1'], ['exit', 'bedroom1', 'livingroom1'], ['at', 'book1', 'bedroom1'], ['closed', 'wardrobe1'], ['at', 'bed1', 'bedroom1']], 'effective_epistemic_gain_amount': 7, 'known_entities': [['at', 'player1', 'bedroom1'], ['at', 'plate1', 'bedroom1'], ['at', 'cupboard1', 'kitchen1'], ['at', 'wardrobe1', 'bedroom1'], ['at', 'chair1', 'livingroom1'], ['at', 'book1', 'bedroom1'], ['at', 'couch1', 'livingroom1'], ['at', 'counter1', 'kitchen1'], ['at', 'bed1', 'bedroom1'], ['at', 'sidetable1', 'livingroom1']], 'known_entities_ratio': 0.3448275862068966, 'visited_rooms': ['kitchen1', 'bedroom1', 'livingroom1'], 'visited_rooms_ratio': 0.5, 'known_goal_entities': [['at', 'wardrobe1', 'bedroom1'], ['at', 'plate1', 'bedroom1'], ['at', 'counter1', 'kitchen1']], 'known_goal_entities_ratio': 0.5}} 
	  } 
	   } 
	   } 
	 & & \\ 
 

    \theutterance \stepcounter{utterance}  

    & & & \multicolumn{2}{p{0.3\linewidth}}{\cellcolor[rgb]{0.95,0.95,0.95}{%
	\makecell[{{p{\linewidth}}}]{% 
	  \tt {\tiny [GM$|$GM]}  
	 {'goal_states_achieved': [], 'turn_goal_score': 0} 
	  } 
	   } 
	   } 
	 & & \\ 
 

    \theutterance \stepcounter{utterance}  

    & & \multicolumn{4}{p{0.6\linewidth}}{\cellcolor[rgb]{0.9,0.9,0.9}{%
	\makecell[{{p{\linewidth}}}]{% 
	  \tt {\tiny [A$\langle$GM]}  
	 You are in a bedroom now. There are a bed and a wardrobe. The wardrobe is closed. There is a passage to a living room here.\\ \tt  
	  } 
	   } 
	   } 
	 & \\ 
 

    \theutterance \stepcounter{utterance}  

    & \multicolumn{4}{p{0.6\linewidth}}{\cellcolor[rgb]{0.8,1,0.9}{%
	\makecell[{{p{\linewidth}}}]{% 
	  \tt {\tiny [A$\rangle$GM]}  
	 > open wardrobe 
	  } 
	   } 
	   } 
	 & & \\ 
 

    \theutterance \stepcounter{utterance}  

    & & & \multicolumn{2}{p{0.3\linewidth}}{\cellcolor[rgb]{0.95,0.95,0.95}{%
	\makecell[{{p{\linewidth}}}]{% 
	  \tt {\tiny [GM$|$GM]}  
	 Round: 6 
	  } 
	   } 
	   } 
	 & & \\ 
 

    \theutterance \stepcounter{utterance}  

    & & & \multicolumn{2}{p{0.3\linewidth}}{\cellcolor[rgb]{0.95,0.95,0.95}{%
	\makecell[{{p{\linewidth}}}]{% 
	  \tt {\tiny [GM$|$GM]}  
	 {'action_type': 'open', 'exploration_info': {'action_epistemic': True, 'action_pragmatic': True, 'effective_epistemic_gain_facts': [['open', 'wardrobe1'], ['at', 'pillow1', 'bedroom1'], ['in', 'pillow1', 'wardrobe1']], 'effective_epistemic_gain_amount': 3, 'known_entities': [['at', 'player1', 'bedroom1'], ['at', 'plate1', 'bedroom1'], ['at', 'cupboard1', 'kitchen1'], ['at', 'wardrobe1', 'bedroom1'], ['at', 'chair1', 'livingroom1'], ['at', 'book1', 'bedroom1'], ['at', 'couch1', 'livingroom1'], ['at', 'counter1', 'kitchen1'], ['at', 'bed1', 'bedroom1'], ['at', 'pillow1', 'bedroom1'], ['at', 'sidetable1', 'livingroom1']], 'known_entities_ratio': 0.3793103448275862, 'visited_rooms': ['kitchen1', 'bedroom1', 'livingroom1'], 'visited_rooms_ratio': 0.5, 'known_goal_entities': [['at', 'wardrobe1', 'bedroom1'], ['at', 'plate1', 'bedroom1'], ['at', 'counter1', 'kitchen1']], 'known_goal_entities_ratio': 0.5}} 
	  } 
	   } 
	   } 
	 & & \\ 
 

    \theutterance \stepcounter{utterance}  

    & & & \multicolumn{2}{p{0.3\linewidth}}{\cellcolor[rgb]{0.95,0.95,0.95}{%
	\makecell[{{p{\linewidth}}}]{% 
	  \tt {\tiny [GM$|$GM]}  
	 {'goal_states_achieved': [], 'turn_goal_score': 0} 
	  } 
	   } 
	   } 
	 & & \\ 
 

    \theutterance \stepcounter{utterance}  

    & & \multicolumn{4}{p{0.6\linewidth}}{\cellcolor[rgb]{0.9,0.9,0.9}{%
	\makecell[{{p{\linewidth}}}]{% 
	  \tt {\tiny [A$\langle$GM]}  
	 The wardrobe is now open. In the wardrobe there is a pillow.\\ \tt  
	  } 
	   } 
	   } 
	 & \\ 
 

    \theutterance \stepcounter{utterance}  

    & \multicolumn{4}{p{0.6\linewidth}}{\cellcolor[rgb]{0.8,1,0.9}{%
	\makecell[{{p{\linewidth}}}]{% 
	  \tt {\tiny [A$\rangle$GM]}  
	 > go to living room 
	  } 
	   } 
	   } 
	 & & \\ 
 

    \theutterance \stepcounter{utterance}  

    & & & \multicolumn{2}{p{0.3\linewidth}}{\cellcolor[rgb]{0.95,0.95,0.95}{%
	\makecell[{{p{\linewidth}}}]{% 
	  \tt {\tiny [GM$|$GM]}  
	 Round: 7 
	  } 
	   } 
	   } 
	 & & \\ 
 

    \theutterance \stepcounter{utterance}  

    & & & \multicolumn{2}{p{0.3\linewidth}}{\cellcolor[rgb]{0.95,0.95,0.95}{%
	\makecell[{{p{\linewidth}}}]{% 
	  \tt {\tiny [GM$|$GM]}  
	 {'action_type': 'go', 'exploration_info': {'action_epistemic': True, 'action_pragmatic': True, 'effective_epistemic_gain_facts': [['at', 'player1', 'livingroom1'], ['at', 'book1', 'livingroom1'], ['at', 'plate1', 'livingroom1']], 'effective_epistemic_gain_amount': 3, 'known_entities': [['at', 'plate1', 'livingroom1'], ['at', 'cupboard1', 'kitchen1'], ['at', 'wardrobe1', 'bedroom1'], ['at', 'book1', 'livingroom1'], ['at', 'chair1', 'livingroom1'], ['at', 'couch1', 'livingroom1'], ['at', 'counter1', 'kitchen1'], ['at', 'player1', 'livingroom1'], ['at', 'bed1', 'bedroom1'], ['at', 'pillow1', 'bedroom1'], ['at', 'sidetable1', 'livingroom1']], 'known_entities_ratio': 0.3793103448275862, 'visited_rooms': ['kitchen1', 'bedroom1', 'livingroom1'], 'visited_rooms_ratio': 0.5, 'known_goal_entities': [['at', 'wardrobe1', 'bedroom1'], ['at', 'plate1', 'livingroom1'], ['at', 'counter1', 'kitchen1']], 'known_goal_entities_ratio': 0.5}} 
	  } 
	   } 
	   } 
	 & & \\ 
 

    \theutterance \stepcounter{utterance}  

    & & & \multicolumn{2}{p{0.3\linewidth}}{\cellcolor[rgb]{0.95,0.95,0.95}{%
	\makecell[{{p{\linewidth}}}]{% 
	  \tt {\tiny [GM$|$GM]}  
	 {'goal_states_achieved': [], 'turn_goal_score': 0} 
	  } 
	   } 
	   } 
	 & & \\ 
 

    \theutterance \stepcounter{utterance}  

    & & \multicolumn{4}{p{0.6\linewidth}}{\cellcolor[rgb]{0.9,0.9,0.9}{%
	\makecell[{{p{\linewidth}}}]{% 
	  \tt {\tiny [A$\langle$GM]}  
	 You are in a living room now. There are a couch, a chair and a side table. There are passages to a bedroom, a hallway and a kitchen here.\\ \tt  
	  } 
	   } 
	   } 
	 & \\ 
 

    \theutterance \stepcounter{utterance}  

    & \multicolumn{4}{p{0.6\linewidth}}{\cellcolor[rgb]{0.8,1,0.9}{%
	\makecell[{{p{\linewidth}}}]{% 
	  \tt {\tiny [A$\rangle$GM]}  
	 > go to hallway 
	  } 
	   } 
	   } 
	 & & \\ 
 

    \theutterance \stepcounter{utterance}  

    & & & \multicolumn{2}{p{0.3\linewidth}}{\cellcolor[rgb]{0.95,0.95,0.95}{%
	\makecell[{{p{\linewidth}}}]{% 
	  \tt {\tiny [GM$|$GM]}  
	 Round: 8 
	  } 
	   } 
	   } 
	 & & \\ 
 

    \theutterance \stepcounter{utterance}  

    & & & \multicolumn{2}{p{0.3\linewidth}}{\cellcolor[rgb]{0.95,0.95,0.95}{%
	\makecell[{{p{\linewidth}}}]{% 
	  \tt {\tiny [GM$|$GM]}  
	 {'action_type': 'go', 'exploration_info': {'action_epistemic': True, 'action_pragmatic': True, 'effective_epistemic_gain_facts': [['exit', 'hallway1', 'livingroom1'], ['exit', 'hallway1', 'broomcloset1'], ['at', 'player1', 'hallway1'], ['at', 'plate1', 'hallway1'], ['on', 'pottedplant1', 'hallway1floor1'], ['at', 'book1', 'hallway1'], ['at', 'pottedplant1', 'hallway1']], 'effective_epistemic_gain_amount': 7, 'known_entities': [['at', 'pillow1', 'bedroom1'], ['at', 'cupboard1', 'kitchen1'], ['at', 'player1', 'hallway1'], ['at', 'wardrobe1', 'bedroom1'], ['at', 'plate1', 'hallway1'], ['at', 'chair1', 'livingroom1'], ['at', 'couch1', 'livingroom1'], ['at', 'counter1', 'kitchen1'], ['at', 'book1', 'hallway1'], ['at', 'bed1', 'bedroom1'], ['at', 'pottedplant1', 'hallway1'], ['at', 'sidetable1', 'livingroom1']], 'known_entities_ratio': 0.41379310344827586, 'visited_rooms': ['kitchen1', 'bedroom1', 'hallway1', 'livingroom1'], 'visited_rooms_ratio': 0.6666666666666666, 'known_goal_entities': [['at', 'wardrobe1', 'bedroom1'], ['at', 'plate1', 'hallway1'], ['at', 'pottedplant1', 'hallway1'], ['at', 'counter1', 'kitchen1']], 'known_goal_entities_ratio': 0.6666666666666666}} 
	  } 
	   } 
	   } 
	 & & \\ 
 

    \theutterance \stepcounter{utterance}  

    & & & \multicolumn{2}{p{0.3\linewidth}}{\cellcolor[rgb]{0.95,0.95,0.95}{%
	\makecell[{{p{\linewidth}}}]{% 
	  \tt {\tiny [GM$|$GM]}  
	 {'goal_states_achieved': [], 'turn_goal_score': 0} 
	  } 
	   } 
	   } 
	 & & \\ 
 

    \theutterance \stepcounter{utterance}  

    & & \multicolumn{4}{p{0.6\linewidth}}{\cellcolor[rgb]{0.9,0.9,0.9}{%
	\makecell[{{p{\linewidth}}}]{% 
	  \tt {\tiny [A$\langle$GM]}  
	 You are in a hallway now. There is a potted plant. The potted plant is on the floor. There are passages to a broom closet and a living room here.\\ \tt  
	  } 
	   } 
	   } 
	 & \\ 
 

    \theutterance \stepcounter{utterance}  

    & \multicolumn{4}{p{0.6\linewidth}}{\cellcolor[rgb]{0.8,1,0.9}{%
	\makecell[{{p{\linewidth}}}]{% 
	  \tt {\tiny [A$\rangle$GM]}  
	 > take potted plant 
	  } 
	   } 
	   } 
	 & & \\ 
 

    \theutterance \stepcounter{utterance}  

    & & & \multicolumn{2}{p{0.3\linewidth}}{\cellcolor[rgb]{0.95,0.95,0.95}{%
	\makecell[{{p{\linewidth}}}]{% 
	  \tt {\tiny [GM$|$GM]}  
	 Round: 9 
	  } 
	   } 
	   } 
	 & & \\ 
 

    \theutterance \stepcounter{utterance}  

    & & & \multicolumn{2}{p{0.3\linewidth}}{\cellcolor[rgb]{0.95,0.95,0.95}{%
	\makecell[{{p{\linewidth}}}]{% 
	  \tt {\tiny [GM$|$GM]}  
	 {'phase': 'resolution', 'fail_type': 'inventory_limit_exceeded', 'arg': {'failed_action_type': 'take', 'failed_precon_predicate': {'predicate_tuple': ['itemcount', 'inventory', 2], 'fulfilled': False, 'precon_idx': 5}}, 'exploration_info': {'action_epistemic': False, 'action_pragmatic': True, 'effective_epistemic_gain_amount': 0, 'known_entities': [['at', 'pillow1', 'bedroom1'], ['at', 'cupboard1', 'kitchen1'], ['at', 'player1', 'hallway1'], ['at', 'wardrobe1', 'bedroom1'], ['at', 'plate1', 'hallway1'], ['at', 'chair1', 'livingroom1'], ['at', 'couch1', 'livingroom1'], ['at', 'counter1', 'kitchen1'], ['at', 'book1', 'hallway1'], ['at', 'bed1', 'bedroom1'], ['at', 'pottedplant1', 'hallway1'], ['at', 'sidetable1', 'livingroom1']], 'known_entities_ratio': 0.41379310344827586, 'visited_rooms': ['kitchen1', 'bedroom1', 'hallway1', 'livingroom1'], 'visited_rooms_ratio': 0.6666666666666666, 'known_goal_entities': [['at', 'wardrobe1', 'bedroom1'], ['at', 'plate1', 'hallway1'], ['at', 'pottedplant1', 'hallway1'], ['at', 'counter1', 'kitchen1']], 'known_goal_entities_ratio': 0.6666666666666666}} 
	  } 
	   } 
	   } 
	 & & \\ 
 

    \theutterance \stepcounter{utterance}  

    & & & \multicolumn{2}{p{0.3\linewidth}}{\cellcolor[rgb]{0.95,0.95,0.95}{%
	\makecell[{{p{\linewidth}}}]{% 
	  \tt {\tiny [GM$|$GM]}  
	 {'goal_states_achieved': [], 'turn_goal_score': 0} 
	  } 
	   } 
	   } 
	 & & \\ 
 

    \theutterance \stepcounter{utterance}  

    & & \multicolumn{4}{p{0.6\linewidth}}{\cellcolor[rgb]{0.9,0.9,0.9}{%
	\makecell[{{p{\linewidth}}}]{% 
	  \tt {\tiny [A$\langle$GM]}  
	 Your inventory can not hold more than two objects at the same time.\\ \tt  
	  } 
	   } 
	   } 
	 & \\ 
 

    \theutterance \stepcounter{utterance}  

    & \multicolumn{4}{p{0.6\linewidth}}{\cellcolor[rgb]{0.8,1,0.9}{%
	\makecell[{{p{\linewidth}}}]{% 
	  \tt {\tiny [A$\rangle$GM]}  
	 > put book on shelf 
	  } 
	   } 
	   } 
	 & & \\ 
 

    \theutterance \stepcounter{utterance}  

    & & & \multicolumn{2}{p{0.3\linewidth}}{\cellcolor[rgb]{0.95,0.95,0.95}{%
	\makecell[{{p{\linewidth}}}]{% 
	  \tt {\tiny [GM$|$GM]}  
	 Round: 10 
	  } 
	   } 
	   } 
	 & & \\ 
 

    \theutterance \stepcounter{utterance}  

    & & & \multicolumn{2}{p{0.3\linewidth}}{\cellcolor[rgb]{0.95,0.95,0.95}{%
	\makecell[{{p{\linewidth}}}]{% 
	  \tt {\tiny [GM$|$GM]}  
	 {'phase': 'resolution', 'fail_type': 'entity_not_accessible', 'arg': {'failed_action_type': 'put', 'failed_precon_predicate': {'predicate_tuple': ['at', 'shelf1', 'hallway1'], 'fulfilled': False, 'precon_idx': 2}}, 'exploration_info': {'action_epistemic': False, 'action_pragmatic': True, 'effective_epistemic_gain_amount': 0, 'known_entities': [['at', 'pillow1', 'bedroom1'], ['at', 'cupboard1', 'kitchen1'], ['at', 'player1', 'hallway1'], ['at', 'wardrobe1', 'bedroom1'], ['at', 'plate1', 'hallway1'], ['at', 'chair1', 'livingroom1'], ['at', 'couch1', 'livingroom1'], ['at', 'counter1', 'kitchen1'], ['at', 'book1', 'hallway1'], ['at', 'bed1', 'bedroom1'], ['at', 'pottedplant1', 'hallway1'], ['at', 'sidetable1', 'livingroom1']], 'known_entities_ratio': 0.41379310344827586, 'visited_rooms': ['kitchen1', 'bedroom1', 'hallway1', 'livingroom1'], 'visited_rooms_ratio': 0.6666666666666666, 'known_goal_entities': [['at', 'wardrobe1', 'bedroom1'], ['at', 'plate1', 'hallway1'], ['at', 'pottedplant1', 'hallway1'], ['at', 'counter1', 'kitchen1']], 'known_goal_entities_ratio': 0.6666666666666666}} 
	  } 
	   } 
	   } 
	 & & \\ 
 

    \theutterance \stepcounter{utterance}  

    & & & \multicolumn{2}{p{0.3\linewidth}}{\cellcolor[rgb]{0.95,0.95,0.95}{%
	\makecell[{{p{\linewidth}}}]{% 
	  \tt {\tiny [GM$|$GM]}  
	 {'goal_states_achieved': [], 'turn_goal_score': 0} 
	  } 
	   } 
	   } 
	 & & \\ 
 

    \theutterance \stepcounter{utterance}  

    & & \multicolumn{4}{p{0.6\linewidth}}{\cellcolor[rgb]{0.9,0.9,0.9}{%
	\makecell[{{p{\linewidth}}}]{% 
	  \tt {\tiny [A$\langle$GM]}  
	 You can't see a shelf here.\\ \tt  
	  } 
	   } 
	   } 
	 & \\ 
 

    \theutterance \stepcounter{utterance}  

    & \multicolumn{4}{p{0.6\linewidth}}{\cellcolor[rgb]{0.8,1,0.9}{%
	\makecell[{{p{\linewidth}}}]{% 
	  \tt {\tiny [A$\rangle$GM]}  
	 > go to living room 
	  } 
	   } 
	   } 
	 & & \\ 
 

    \theutterance \stepcounter{utterance}  

    & & & \multicolumn{2}{p{0.3\linewidth}}{\cellcolor[rgb]{0.95,0.95,0.95}{%
	\makecell[{{p{\linewidth}}}]{% 
	  \tt {\tiny [GM$|$GM]}  
	 Round: 11 
	  } 
	   } 
	   } 
	 & & \\ 
 

    \theutterance \stepcounter{utterance}  

    & & & \multicolumn{2}{p{0.3\linewidth}}{\cellcolor[rgb]{0.95,0.95,0.95}{%
	\makecell[{{p{\linewidth}}}]{% 
	  \tt {\tiny [GM$|$GM]}  
	 {'action_type': 'go', 'exploration_info': {'action_epistemic': True, 'action_pragmatic': True, 'effective_epistemic_gain_facts': [['at', 'player1', 'livingroom1'], ['at', 'book1', 'livingroom1'], ['at', 'plate1', 'livingroom1']], 'effective_epistemic_gain_amount': 3, 'known_entities': [['at', 'pillow1', 'bedroom1'], ['at', 'plate1', 'livingroom1'], ['at', 'cupboard1', 'kitchen1'], ['at', 'wardrobe1', 'bedroom1'], ['at', 'book1', 'livingroom1'], ['at', 'chair1', 'livingroom1'], ['at', 'couch1', 'livingroom1'], ['at', 'counter1', 'kitchen1'], ['at', 'player1', 'livingroom1'], ['at', 'bed1', 'bedroom1'], ['at', 'pottedplant1', 'hallway1'], ['at', 'sidetable1', 'livingroom1']], 'known_entities_ratio': 0.41379310344827586, 'visited_rooms': ['kitchen1', 'bedroom1', 'hallway1', 'livingroom1'], 'visited_rooms_ratio': 0.6666666666666666, 'known_goal_entities': [['at', 'wardrobe1', 'bedroom1'], ['at', 'plate1', 'livingroom1'], ['at', 'pottedplant1', 'hallway1'], ['at', 'counter1', 'kitchen1']], 'known_goal_entities_ratio': 0.6666666666666666}} 
	  } 
	   } 
	   } 
	 & & \\ 
 

    \theutterance \stepcounter{utterance}  

    & & & \multicolumn{2}{p{0.3\linewidth}}{\cellcolor[rgb]{0.95,0.95,0.95}{%
	\makecell[{{p{\linewidth}}}]{% 
	  \tt {\tiny [GM$|$GM]}  
	 {'goal_states_achieved': [], 'turn_goal_score': 0} 
	  } 
	   } 
	   } 
	 & & \\ 
 

    \theutterance \stepcounter{utterance}  

    & & \multicolumn{4}{p{0.6\linewidth}}{\cellcolor[rgb]{0.9,0.9,0.9}{%
	\makecell[{{p{\linewidth}}}]{% 
	  \tt {\tiny [A$\langle$GM]}  
	 You are in a living room now. There are a couch, a chair and a side table. There are passages to a bedroom, a hallway and a kitchen here.\\ \tt  
	  } 
	   } 
	   } 
	 & \\ 
 

    \theutterance \stepcounter{utterance}  

    & \multicolumn{4}{p{0.6\linewidth}}{\cellcolor[rgb]{0.8,1,0.9}{%
	\makecell[{{p{\linewidth}}}]{% 
	  \tt {\tiny [A$\rangle$GM]}  
	 > go to broom closet 
	  } 
	   } 
	   } 
	 & & \\ 
 

    \theutterance \stepcounter{utterance}  

    & & & \multicolumn{2}{p{0.3\linewidth}}{\cellcolor[rgb]{0.95,0.95,0.95}{%
	\makecell[{{p{\linewidth}}}]{% 
	  \tt {\tiny [GM$|$GM]}  
	 Round: 12 
	  } 
	   } 
	   } 
	 & & \\ 
 

    \theutterance \stepcounter{utterance}  

    & & & \multicolumn{2}{p{0.3\linewidth}}{\cellcolor[rgb]{0.95,0.95,0.95}{%
	\makecell[{{p{\linewidth}}}]{% 
	  \tt {\tiny [GM$|$GM]}  
	 {'phase': 'resolution', 'fail_type': 'no_exit_to', 'arg': {'failed_action_type': 'go', 'failed_precon_predicate': {'predicate_tuple': ['exit', 'livingroom1', 'broomcloset1'], 'fulfilled': False, 'precon_idx': 2}}, 'exploration_info': {'action_epistemic': True, 'action_pragmatic': True, 'effective_epistemic_gain_facts': [], 'effective_epistemic_gain_amount': 0, 'known_entities': [['at', 'pillow1', 'bedroom1'], ['at', 'plate1', 'livingroom1'], ['at', 'cupboard1', 'kitchen1'], ['at', 'wardrobe1', 'bedroom1'], ['at', 'book1', 'livingroom1'], ['at', 'chair1', 'livingroom1'], ['at', 'couch1', 'livingroom1'], ['at', 'counter1', 'kitchen1'], ['at', 'player1', 'livingroom1'], ['at', 'bed1', 'bedroom1'], ['at', 'pottedplant1', 'hallway1'], ['at', 'sidetable1', 'livingroom1']], 'known_entities_ratio': 0.41379310344827586, 'visited_rooms': ['kitchen1', 'bedroom1', 'hallway1', 'livingroom1'], 'visited_rooms_ratio': 0.6666666666666666, 'known_goal_entities': [['at', 'wardrobe1', 'bedroom1'], ['at', 'plate1', 'livingroom1'], ['at', 'pottedplant1', 'hallway1'], ['at', 'counter1', 'kitchen1']], 'known_goal_entities_ratio': 0.6666666666666666}} 
	  } 
	   } 
	   } 
	 & & \\ 
 

    \theutterance \stepcounter{utterance}  

    & & & \multicolumn{2}{p{0.3\linewidth}}{\cellcolor[rgb]{0.95,0.95,0.95}{%
	\makecell[{{p{\linewidth}}}]{% 
	  \tt {\tiny [GM$|$GM]}  
	 {'goal_states_achieved': [], 'turn_goal_score': 0} 
	  } 
	   } 
	   } 
	 & & \\ 
 

    \theutterance \stepcounter{utterance}  

    & & \multicolumn{4}{p{0.6\linewidth}}{\cellcolor[rgb]{0.9,0.9,0.9}{%
	\makecell[{{p{\linewidth}}}]{% 
	  \tt {\tiny [A$\langle$GM]}  
	 You can't go to a broom closet from here.\\ \tt  
	  } 
	   } 
	   } 
	 & \\ 
 

    \theutterance \stepcounter{utterance}  

    & \multicolumn{4}{p{0.6\linewidth}}{\cellcolor[rgb]{0.8,1,0.9}{%
	\makecell[{{p{\linewidth}}}]{% 
	  \tt {\tiny [A$\rangle$GM]}  
	 > go to kitchen 
	  } 
	   } 
	   } 
	 & & \\ 
 

    \theutterance \stepcounter{utterance}  

    & & & \multicolumn{2}{p{0.3\linewidth}}{\cellcolor[rgb]{0.95,0.95,0.95}{%
	\makecell[{{p{\linewidth}}}]{% 
	  \tt {\tiny [GM$|$GM]}  
	 Round: 13 
	  } 
	   } 
	   } 
	 & & \\ 
 

    \theutterance \stepcounter{utterance}  

    & & & \multicolumn{2}{p{0.3\linewidth}}{\cellcolor[rgb]{0.95,0.95,0.95}{%
	\makecell[{{p{\linewidth}}}]{% 
	  \tt {\tiny [GM$|$GM]}  
	 {'action_type': 'go', 'exploration_info': {'action_epistemic': True, 'action_pragmatic': True, 'effective_epistemic_gain_facts': [['at', 'plate1', 'kitchen1'], ['at', 'player1', 'kitchen1'], ['at', 'book1', 'kitchen1']], 'effective_epistemic_gain_amount': 3, 'known_entities': [['at', 'pillow1', 'bedroom1'], ['at', 'cupboard1', 'kitchen1'], ['at', 'wardrobe1', 'bedroom1'], ['at', 'chair1', 'livingroom1'], ['at', 'player1', 'kitchen1'], ['at', 'plate1', 'kitchen1'], ['at', 'couch1', 'livingroom1'], ['at', 'counter1', 'kitchen1'], ['at', 'bed1', 'bedroom1'], ['at', 'pottedplant1', 'hallway1'], ['at', 'sidetable1', 'livingroom1'], ['at', 'book1', 'kitchen1']], 'known_entities_ratio': 0.41379310344827586, 'visited_rooms': ['kitchen1', 'bedroom1', 'hallway1', 'livingroom1'], 'visited_rooms_ratio': 0.6666666666666666, 'known_goal_entities': [['at', 'wardrobe1', 'bedroom1'], ['at', 'plate1', 'kitchen1'], ['at', 'pottedplant1', 'hallway1'], ['at', 'counter1', 'kitchen1']], 'known_goal_entities_ratio': 0.6666666666666666}} 
	  } 
	   } 
	   } 
	 & & \\ 
 

    \theutterance \stepcounter{utterance}  

    & & & \multicolumn{2}{p{0.3\linewidth}}{\cellcolor[rgb]{0.95,0.95,0.95}{%
	\makecell[{{p{\linewidth}}}]{% 
	  \tt {\tiny [GM$|$GM]}  
	 {'goal_states_achieved': [], 'turn_goal_score': 0} 
	  } 
	   } 
	   } 
	 & & \\ 
 

    \theutterance \stepcounter{utterance}  

    & & \multicolumn{4}{p{0.6\linewidth}}{\cellcolor[rgb]{0.9,0.9,0.9}{%
	\makecell[{{p{\linewidth}}}]{% 
	  \tt {\tiny [A$\langle$GM]}  
	 You are in a kitchen now. There are a cupboard and a counter. The cupboard is open. There are passages to a living room and a pantry here.\\ \tt  
	  } 
	   } 
	   } 
	 & \\ 
 

    \theutterance \stepcounter{utterance}  

    & \multicolumn{4}{p{0.6\linewidth}}{\cellcolor[rgb]{0.8,1,0.9}{%
	\makecell[{{p{\linewidth}}}]{% 
	  \tt {\tiny [A$\rangle$GM]}  
	 > put plate on counter 
	  } 
	   } 
	   } 
	 & & \\ 
 

    \theutterance \stepcounter{utterance}  

    & & & \multicolumn{2}{p{0.3\linewidth}}{\cellcolor[rgb]{0.95,0.95,0.95}{%
	\makecell[{{p{\linewidth}}}]{% 
	  \tt {\tiny [GM$|$GM]}  
	 Round: 14 
	  } 
	   } 
	   } 
	 & & \\ 
 

    \theutterance \stepcounter{utterance}  

    & & & \multicolumn{2}{p{0.3\linewidth}}{\cellcolor[rgb]{0.95,0.95,0.95}{%
	\makecell[{{p{\linewidth}}}]{% 
	  \tt {\tiny [GM$|$GM]}  
	 {'action_type': 'put', 'exploration_info': {'action_epistemic': False, 'action_pragmatic': True, 'effective_epistemic_gain_amount': 0, 'known_entities': [['at', 'pillow1', 'bedroom1'], ['at', 'cupboard1', 'kitchen1'], ['at', 'wardrobe1', 'bedroom1'], ['at', 'chair1', 'livingroom1'], ['at', 'player1', 'kitchen1'], ['at', 'plate1', 'kitchen1'], ['at', 'couch1', 'livingroom1'], ['at', 'counter1', 'kitchen1'], ['at', 'bed1', 'bedroom1'], ['at', 'pottedplant1', 'hallway1'], ['at', 'sidetable1', 'livingroom1'], ['at', 'book1', 'kitchen1']], 'known_entities_ratio': 0.41379310344827586, 'visited_rooms': ['kitchen1', 'bedroom1', 'hallway1', 'livingroom1'], 'visited_rooms_ratio': 0.6666666666666666, 'known_goal_entities': [['at', 'wardrobe1', 'bedroom1'], ['at', 'plate1', 'kitchen1'], ['at', 'pottedplant1', 'hallway1'], ['at', 'counter1', 'kitchen1']], 'known_goal_entities_ratio': 0.6666666666666666}} 
	  } 
	   } 
	   } 
	 & & \\ 
 

    \theutterance \stepcounter{utterance}  

    & & & \multicolumn{2}{p{0.3\linewidth}}{\cellcolor[rgb]{0.95,0.95,0.95}{%
	\makecell[{{p{\linewidth}}}]{% 
	  \tt {\tiny [GM$|$GM]}  
	 {'goal_states_achieved': [], 'turn_goal_score': 0} 
	  } 
	   } 
	   } 
	 & & \\ 
 

    \theutterance \stepcounter{utterance}  

    & & \multicolumn{4}{p{0.6\linewidth}}{\cellcolor[rgb]{0.9,0.9,0.9}{%
	\makecell[{{p{\linewidth}}}]{% 
	  \tt {\tiny [A$\langle$GM]}  
	 You put the plate on the counter.\\ \tt  
	  } 
	   } 
	   } 
	 & \\ 
 

    \theutterance \stepcounter{utterance}  

    & \multicolumn{4}{p{0.6\linewidth}}{\cellcolor[rgb]{0.8,1,0.9}{%
	\makecell[{{p{\linewidth}}}]{% 
	  \tt {\tiny [A$\rangle$GM]}  
	 > go to hallway 
	  } 
	   } 
	   } 
	 & & \\ 
 

    \theutterance \stepcounter{utterance}  

    & & & \multicolumn{2}{p{0.3\linewidth}}{\cellcolor[rgb]{0.95,0.95,0.95}{%
	\makecell[{{p{\linewidth}}}]{% 
	  \tt {\tiny [GM$|$GM]}  
	 Round: 15 
	  } 
	   } 
	   } 
	 & & \\ 
 

    \theutterance \stepcounter{utterance}  

    & & & \multicolumn{2}{p{0.3\linewidth}}{\cellcolor[rgb]{0.95,0.95,0.95}{%
	\makecell[{{p{\linewidth}}}]{% 
	  \tt {\tiny [GM$|$GM]}  
	 {'phase': 'resolution', 'fail_type': 'no_exit_to', 'arg': {'failed_action_type': 'go', 'failed_precon_predicate': {'predicate_tuple': ['exit', 'kitchen1', 'hallway1'], 'fulfilled': False, 'precon_idx': 2}}, 'exploration_info': {'action_epistemic': True, 'action_pragmatic': True, 'effective_epistemic_gain_facts': [], 'effective_epistemic_gain_amount': 0, 'known_entities': [['at', 'pillow1', 'bedroom1'], ['at', 'cupboard1', 'kitchen1'], ['at', 'wardrobe1', 'bedroom1'], ['at', 'chair1', 'livingroom1'], ['at', 'player1', 'kitchen1'], ['at', 'plate1', 'kitchen1'], ['at', 'couch1', 'livingroom1'], ['at', 'counter1', 'kitchen1'], ['at', 'bed1', 'bedroom1'], ['at', 'pottedplant1', 'hallway1'], ['at', 'sidetable1', 'livingroom1'], ['at', 'book1', 'kitchen1']], 'known_entities_ratio': 0.41379310344827586, 'visited_rooms': ['kitchen1', 'bedroom1', 'hallway1', 'livingroom1'], 'visited_rooms_ratio': 0.6666666666666666, 'known_goal_entities': [['at', 'wardrobe1', 'bedroom1'], ['at', 'plate1', 'kitchen1'], ['at', 'pottedplant1', 'hallway1'], ['at', 'counter1', 'kitchen1']], 'known_goal_entities_ratio': 0.6666666666666666}} 
	  } 
	   } 
	   } 
	 & & \\ 
 

    \theutterance \stepcounter{utterance}  

    & & & \multicolumn{2}{p{0.3\linewidth}}{\cellcolor[rgb]{0.95,0.95,0.95}{%
	\makecell[{{p{\linewidth}}}]{% 
	  \tt {\tiny [GM$|$GM]}  
	 {'goal_states_achieved': [], 'turn_goal_score': 0} 
	  } 
	   } 
	   } 
	 & & \\ 
 

    \theutterance \stepcounter{utterance}  

    & & \multicolumn{4}{p{0.6\linewidth}}{\cellcolor[rgb]{0.9,0.9,0.9}{%
	\makecell[{{p{\linewidth}}}]{% 
	  \tt {\tiny [A$\langle$GM]}  
	 You can't go to a hallway from here.\\ \tt  
	  } 
	   } 
	   } 
	 & \\ 
 

    \theutterance \stepcounter{utterance}  

    & \multicolumn{4}{p{0.6\linewidth}}{\cellcolor[rgb]{0.8,1,0.9}{%
	\makecell[{{p{\linewidth}}}]{% 
	  \tt {\tiny [A$\rangle$GM]}  
	 > go to living room 
	  } 
	   } 
	   } 
	 & & \\ 
 

    \theutterance \stepcounter{utterance}  

    & & & \multicolumn{2}{p{0.3\linewidth}}{\cellcolor[rgb]{0.95,0.95,0.95}{%
	\makecell[{{p{\linewidth}}}]{% 
	  \tt {\tiny [GM$|$GM]}  
	 Round: 16 
	  } 
	   } 
	   } 
	 & & \\ 
 

    \theutterance \stepcounter{utterance}  

    & & & \multicolumn{2}{p{0.3\linewidth}}{\cellcolor[rgb]{0.95,0.95,0.95}{%
	\makecell[{{p{\linewidth}}}]{% 
	  \tt {\tiny [GM$|$GM]}  
	 {'action_type': 'go', 'exploration_info': {'action_epistemic': True, 'action_pragmatic': True, 'effective_epistemic_gain_facts': [['at', 'player1', 'livingroom1'], ['at', 'book1', 'livingroom1']], 'effective_epistemic_gain_amount': 2, 'known_entities': [['at', 'pillow1', 'bedroom1'], ['at', 'cupboard1', 'kitchen1'], ['at', 'wardrobe1', 'bedroom1'], ['at', 'book1', 'livingroom1'], ['at', 'chair1', 'livingroom1'], ['at', 'plate1', 'kitchen1'], ['at', 'couch1', 'livingroom1'], ['at', 'counter1', 'kitchen1'], ['at', 'player1', 'livingroom1'], ['at', 'bed1', 'bedroom1'], ['at', 'pottedplant1', 'hallway1'], ['at', 'sidetable1', 'livingroom1']], 'known_entities_ratio': 0.41379310344827586, 'visited_rooms': ['kitchen1', 'bedroom1', 'hallway1', 'livingroom1'], 'visited_rooms_ratio': 0.6666666666666666, 'known_goal_entities': [['at', 'wardrobe1', 'bedroom1'], ['at', 'plate1', 'kitchen1'], ['at', 'pottedplant1', 'hallway1'], ['at', 'counter1', 'kitchen1']], 'known_goal_entities_ratio': 0.6666666666666666}} 
	  } 
	   } 
	   } 
	 & & \\ 
 

    \theutterance \stepcounter{utterance}  

    & & & \multicolumn{2}{p{0.3\linewidth}}{\cellcolor[rgb]{0.95,0.95,0.95}{%
	\makecell[{{p{\linewidth}}}]{% 
	  \tt {\tiny [GM$|$GM]}  
	 {'goal_states_achieved': [], 'turn_goal_score': 0} 
	  } 
	   } 
	   } 
	 & & \\ 
 

    \theutterance \stepcounter{utterance}  

    & & \multicolumn{4}{p{0.6\linewidth}}{\cellcolor[rgb]{0.9,0.9,0.9}{%
	\makecell[{{p{\linewidth}}}]{% 
	  \tt {\tiny [A$\langle$GM]}  
	 You are in a living room now. There are a couch, a chair and a side table. There are passages to a bedroom, a hallway and a kitchen here.\\ \tt  
	  } 
	   } 
	   } 
	 & \\ 
 

    \theutterance \stepcounter{utterance}  

    & \multicolumn{4}{p{0.6\linewidth}}{\cellcolor[rgb]{0.8,1,0.9}{%
	\makecell[{{p{\linewidth}}}]{% 
	  \tt {\tiny [A$\rangle$GM]}  
	 > go to hallway 
	  } 
	   } 
	   } 
	 & & \\ 
 

    \theutterance \stepcounter{utterance}  

    & & & \multicolumn{2}{p{0.3\linewidth}}{\cellcolor[rgb]{0.95,0.95,0.95}{%
	\makecell[{{p{\linewidth}}}]{% 
	  \tt {\tiny [GM$|$GM]}  
	 Round: 17 
	  } 
	   } 
	   } 
	 & & \\ 
 

    \theutterance \stepcounter{utterance}  

    & & & \multicolumn{2}{p{0.3\linewidth}}{\cellcolor[rgb]{0.95,0.95,0.95}{%
	\makecell[{{p{\linewidth}}}]{% 
	  \tt {\tiny [GM$|$GM]}  
	 {'action_type': 'go', 'exploration_info': {'action_epistemic': True, 'action_pragmatic': True, 'effective_epistemic_gain_facts': [['at', 'player1', 'hallway1'], ['at', 'book1', 'hallway1']], 'effective_epistemic_gain_amount': 2, 'known_entities': [['at', 'pillow1', 'bedroom1'], ['at', 'cupboard1', 'kitchen1'], ['at', 'player1', 'hallway1'], ['at', 'wardrobe1', 'bedroom1'], ['at', 'chair1', 'livingroom1'], ['at', 'plate1', 'kitchen1'], ['at', 'couch1', 'livingroom1'], ['at', 'counter1', 'kitchen1'], ['at', 'book1', 'hallway1'], ['at', 'bed1', 'bedroom1'], ['at', 'pottedplant1', 'hallway1'], ['at', 'sidetable1', 'livingroom1']], 'known_entities_ratio': 0.41379310344827586, 'visited_rooms': ['kitchen1', 'bedroom1', 'hallway1', 'livingroom1'], 'visited_rooms_ratio': 0.6666666666666666, 'known_goal_entities': [['at', 'wardrobe1', 'bedroom1'], ['at', 'plate1', 'kitchen1'], ['at', 'pottedplant1', 'hallway1'], ['at', 'counter1', 'kitchen1']], 'known_goal_entities_ratio': 0.6666666666666666}} 
	  } 
	   } 
	   } 
	 & & \\ 
 

    \theutterance \stepcounter{utterance}  

    & & & \multicolumn{2}{p{0.3\linewidth}}{\cellcolor[rgb]{0.95,0.95,0.95}{%
	\makecell[{{p{\linewidth}}}]{% 
	  \tt {\tiny [GM$|$GM]}  
	 {'goal_states_achieved': [], 'turn_goal_score': 0} 
	  } 
	   } 
	   } 
	 & & \\ 
 

    \theutterance \stepcounter{utterance}  

    & & \multicolumn{4}{p{0.6\linewidth}}{\cellcolor[rgb]{0.9,0.9,0.9}{%
	\makecell[{{p{\linewidth}}}]{% 
	  \tt {\tiny [A$\langle$GM]}  
	 You are in a hallway now. There is a potted plant. The potted plant is on the floor. There are passages to a broom closet and a living room here.\\ \tt  
	  } 
	   } 
	   } 
	 & \\ 
 

    \theutterance \stepcounter{utterance}  

    & \multicolumn{4}{p{0.6\linewidth}}{\cellcolor[rgb]{0.8,1,0.9}{%
	\makecell[{{p{\linewidth}}}]{% 
	  \tt {\tiny [A$\rangle$GM]}  
	 > take potted plant \\ \tt > go to living room 
	  } 
	   } 
	   } 
	 & & \\ 
 

    \theutterance \stepcounter{utterance}  

    & & & \multicolumn{2}{p{0.3\linewidth}}{\cellcolor[rgb]{0.95,0.95,0.95}{%
	\makecell[{{p{\linewidth}}}]{% 
	  \tt {\tiny [GM$|$GM]}  
	 Round: 18 
	  } 
	   } 
	   } 
	 & & \\ 
 

    \theutterance \stepcounter{utterance}  

    & & & \multicolumn{2}{p{0.3\linewidth}}{\cellcolor[rgb]{0.95,0.95,0.95}{%
	\makecell[{{p{\linewidth}}}]{% 
	  \tt {\tiny [GM$|$GM]}  
	 {'action_type': 'take', 'exploration_info': {'action_epistemic': False, 'action_pragmatic': True, 'effective_epistemic_gain_amount': 0, 'known_entities': [['at', 'pillow1', 'bedroom1'], ['at', 'cupboard1', 'kitchen1'], ['at', 'player1', 'hallway1'], ['at', 'wardrobe1', 'bedroom1'], ['at', 'chair1', 'livingroom1'], ['at', 'plate1', 'kitchen1'], ['at', 'couch1', 'livingroom1'], ['at', 'counter1', 'kitchen1'], ['at', 'book1', 'hallway1'], ['at', 'bed1', 'bedroom1'], ['at', 'pottedplant1', 'hallway1'], ['at', 'sidetable1', 'livingroom1']], 'known_entities_ratio': 0.41379310344827586, 'visited_rooms': ['kitchen1', 'bedroom1', 'hallway1', 'livingroom1'], 'visited_rooms_ratio': 0.6666666666666666, 'known_goal_entities': [['at', 'wardrobe1', 'bedroom1'], ['at', 'plate1', 'kitchen1'], ['at', 'pottedplant1', 'hallway1'], ['at', 'counter1', 'kitchen1']], 'known_goal_entities_ratio': 0.6666666666666666}} 
	  } 
	   } 
	   } 
	 & & \\ 
 

    \theutterance \stepcounter{utterance}  

    & & & \multicolumn{2}{p{0.3\linewidth}}{\cellcolor[rgb]{0.95,0.95,0.95}{%
	\makecell[{{p{\linewidth}}}]{% 
	  \tt {\tiny [GM$|$GM]}  
	 {'goal_states_achieved': [], 'turn_goal_score': 0} 
	  } 
	   } 
	   } 
	 & & \\ 
 

    \theutterance \stepcounter{utterance}  

    & & \multicolumn{4}{p{0.6\linewidth}}{\cellcolor[rgb]{0.9,0.9,0.9}{%
	\makecell[{{p{\linewidth}}}]{% 
	  \tt {\tiny [A$\langle$GM]}  
	 You take the potted plant. In your inventory you have a potted plant and a book.\\ \tt  
	  } 
	   } 
	   } 
	 & \\ 
 

    \theutterance \stepcounter{utterance}  

    & \multicolumn{4}{p{0.6\linewidth}}{\cellcolor[rgb]{0.8,1,0.9}{%
	\makecell[{{p{\linewidth}}}]{% 
	  \tt {\tiny [A$\rangle$GM]}  
	 > put potted plant on counter \\ \tt > put book on shelf 
	  } 
	   } 
	   } 
	 & & \\ 
 

    \theutterance \stepcounter{utterance}  

    & & & \multicolumn{2}{p{0.3\linewidth}}{\cellcolor[rgb]{0.95,0.95,0.95}{%
	\makecell[{{p{\linewidth}}}]{% 
	  \tt {\tiny [GM$|$GM]}  
	 Round: 19 
	  } 
	   } 
	   } 
	 & & \\ 
 

    \theutterance \stepcounter{utterance}  

    & & & \multicolumn{2}{p{0.3\linewidth}}{\cellcolor[rgb]{0.95,0.95,0.95}{%
	\makecell[{{p{\linewidth}}}]{% 
	  \tt {\tiny [GM$|$GM]}  
	 {'phase': 'resolution', 'fail_type': 'entity_not_accessible', 'arg': {'failed_action_type': 'put', 'failed_precon_predicate': {'predicate_tuple': ['at', 'counter1', 'hallway1'], 'fulfilled': False, 'precon_idx': 2}}, 'exploration_info': {'action_epistemic': False, 'action_pragmatic': True, 'effective_epistemic_gain_amount': 0, 'known_entities': [['at', 'pillow1', 'bedroom1'], ['at', 'cupboard1', 'kitchen1'], ['at', 'player1', 'hallway1'], ['at', 'wardrobe1', 'bedroom1'], ['at', 'chair1', 'livingroom1'], ['at', 'plate1', 'kitchen1'], ['at', 'couch1', 'livingroom1'], ['at', 'counter1', 'kitchen1'], ['at', 'book1', 'hallway1'], ['at', 'bed1', 'bedroom1'], ['at', 'pottedplant1', 'hallway1'], ['at', 'sidetable1', 'livingroom1']], 'known_entities_ratio': 0.41379310344827586, 'visited_rooms': ['kitchen1', 'bedroom1', 'hallway1', 'livingroom1'], 'visited_rooms_ratio': 0.6666666666666666, 'known_goal_entities': [['at', 'wardrobe1', 'bedroom1'], ['at', 'plate1', 'kitchen1'], ['at', 'pottedplant1', 'hallway1'], ['at', 'counter1', 'kitchen1']], 'known_goal_entities_ratio': 0.6666666666666666}} 
	  } 
	   } 
	   } 
	 & & \\ 
 

    \theutterance \stepcounter{utterance}  

    & & & \multicolumn{2}{p{0.3\linewidth}}{\cellcolor[rgb]{0.95,0.95,0.95}{%
	\makecell[{{p{\linewidth}}}]{% 
	  \tt {\tiny [GM$|$GM]}  
	 {'goal_states_achieved': [], 'turn_goal_score': 0} 
	  } 
	   } 
	   } 
	 & & \\ 
 

    \theutterance \stepcounter{utterance}  

    & & \multicolumn{4}{p{0.6\linewidth}}{\cellcolor[rgb]{0.9,0.9,0.9}{%
	\makecell[{{p{\linewidth}}}]{% 
	  \tt {\tiny [A$\langle$GM]}  
	 You can't see a counter here.\\ \tt  
	  } 
	   } 
	   } 
	 & \\ 
 

    \theutterance \stepcounter{utterance}  

    & \multicolumn{4}{p{0.6\linewidth}}{\cellcolor[rgb]{0.8,1,0.9}{%
	\makecell[{{p{\linewidth}}}]{% 
	  \tt {\tiny [A$\rangle$GM]}  
	 > go to kitchen 
	  } 
	   } 
	   } 
	 & & \\ 
 

    \theutterance \stepcounter{utterance}  

    & & & \multicolumn{2}{p{0.3\linewidth}}{\cellcolor[rgb]{0.95,0.95,0.95}{%
	\makecell[{{p{\linewidth}}}]{% 
	  \tt {\tiny [GM$|$GM]}  
	 Round: 20 
	  } 
	   } 
	   } 
	 & & \\ 
 

    \theutterance \stepcounter{utterance}  

    & & & \multicolumn{2}{p{0.3\linewidth}}{\cellcolor[rgb]{0.95,0.95,0.95}{%
	\makecell[{{p{\linewidth}}}]{% 
	  \tt {\tiny [GM$|$GM]}  
	 {'phase': 'resolution', 'fail_type': 'no_exit_to', 'arg': {'failed_action_type': 'go', 'failed_precon_predicate': {'predicate_tuple': ['exit', 'hallway1', 'kitchen1'], 'fulfilled': False, 'precon_idx': 2}}, 'exploration_info': {'action_epistemic': True, 'action_pragmatic': True, 'effective_epistemic_gain_facts': [], 'effective_epistemic_gain_amount': 0, 'known_entities': [['at', 'pillow1', 'bedroom1'], ['at', 'cupboard1', 'kitchen1'], ['at', 'player1', 'hallway1'], ['at', 'wardrobe1', 'bedroom1'], ['at', 'chair1', 'livingroom1'], ['at', 'plate1', 'kitchen1'], ['at', 'couch1', 'livingroom1'], ['at', 'counter1', 'kitchen1'], ['at', 'book1', 'hallway1'], ['at', 'bed1', 'bedroom1'], ['at', 'pottedplant1', 'hallway1'], ['at', 'sidetable1', 'livingroom1']], 'known_entities_ratio': 0.41379310344827586, 'visited_rooms': ['kitchen1', 'bedroom1', 'hallway1', 'livingroom1'], 'visited_rooms_ratio': 0.6666666666666666, 'known_goal_entities': [['at', 'wardrobe1', 'bedroom1'], ['at', 'plate1', 'kitchen1'], ['at', 'pottedplant1', 'hallway1'], ['at', 'counter1', 'kitchen1']], 'known_goal_entities_ratio': 0.6666666666666666}} 
	  } 
	   } 
	   } 
	 & & \\ 
 

    \theutterance \stepcounter{utterance}  

    & & & \multicolumn{2}{p{0.3\linewidth}}{\cellcolor[rgb]{0.95,0.95,0.95}{%
	\makecell[{{p{\linewidth}}}]{% 
	  \tt {\tiny [GM$|$GM]}  
	 {'goal_states_achieved': [], 'turn_goal_score': 0} 
	  } 
	   } 
	   } 
	 & & \\ 
 

    \theutterance \stepcounter{utterance}  

    & & \multicolumn{4}{p{0.6\linewidth}}{\cellcolor[rgb]{0.9,0.9,0.9}{%
	\makecell[{{p{\linewidth}}}]{% 
	  \tt {\tiny [A$\langle$GM]}  
	 You can't go to a kitchen from here.\\ \tt  
	  } 
	   } 
	   } 
	 & \\ 
 

    \theutterance \stepcounter{utterance}  

    & \multicolumn{4}{p{0.6\linewidth}}{\cellcolor[rgb]{0.8,1,0.9}{%
	\makecell[{{p{\linewidth}}}]{% 
	  \tt {\tiny [A$\rangle$GM]}  
	 > put potted plant on counter 
	  } 
	   } 
	   } 
	 & & \\ 
 

    \theutterance \stepcounter{utterance}  

    & & & \multicolumn{2}{p{0.3\linewidth}}{\cellcolor[rgb]{0.95,0.95,0.95}{%
	\makecell[{{p{\linewidth}}}]{% 
	  \tt {\tiny [GM$|$GM]}  
	 Round: 21 
	  } 
	   } 
	   } 
	 & & \\ 
 

    \theutterance \stepcounter{utterance}  

    & & & \multicolumn{2}{p{0.3\linewidth}}{\cellcolor[rgb]{0.95,0.95,0.95}{%
	\makecell[{{p{\linewidth}}}]{% 
	  \tt {\tiny [GM$|$GM]}  
	 {'phase': 'resolution', 'fail_type': 'entity_not_accessible', 'arg': {'failed_action_type': 'put', 'failed_precon_predicate': {'predicate_tuple': ['at', 'counter1', 'hallway1'], 'fulfilled': False, 'precon_idx': 2}}, 'exploration_info': {'action_epistemic': False, 'action_pragmatic': True, 'effective_epistemic_gain_amount': 0, 'known_entities': [['at', 'pillow1', 'bedroom1'], ['at', 'cupboard1', 'kitchen1'], ['at', 'player1', 'hallway1'], ['at', 'wardrobe1', 'bedroom1'], ['at', 'chair1', 'livingroom1'], ['at', 'plate1', 'kitchen1'], ['at', 'couch1', 'livingroom1'], ['at', 'counter1', 'kitchen1'], ['at', 'book1', 'hallway1'], ['at', 'bed1', 'bedroom1'], ['at', 'pottedplant1', 'hallway1'], ['at', 'sidetable1', 'livingroom1']], 'known_entities_ratio': 0.41379310344827586, 'visited_rooms': ['kitchen1', 'bedroom1', 'hallway1', 'livingroom1'], 'visited_rooms_ratio': 0.6666666666666666, 'known_goal_entities': [['at', 'wardrobe1', 'bedroom1'], ['at', 'plate1', 'kitchen1'], ['at', 'pottedplant1', 'hallway1'], ['at', 'counter1', 'kitchen1']], 'known_goal_entities_ratio': 0.6666666666666666}} 
	  } 
	   } 
	   } 
	 & & \\ 
 

    \theutterance \stepcounter{utterance}  

    & & & \multicolumn{2}{p{0.3\linewidth}}{\cellcolor[rgb]{0.95,0.95,0.95}{%
	\makecell[{{p{\linewidth}}}]{% 
	  \tt {\tiny [GM$|$GM]}  
	 {'goal_states_achieved': [], 'turn_goal_score': 0} 
	  } 
	   } 
	   } 
	 & & \\ 
 

    \theutterance \stepcounter{utterance}  

    & & \multicolumn{4}{p{0.6\linewidth}}{\cellcolor[rgb]{0.9,0.9,0.9}{%
	\makecell[{{p{\linewidth}}}]{% 
	  \tt {\tiny [A$\langle$GM]}  
	 You can't see a counter here.\\ \tt  
	  } 
	   } 
	   } 
	 & \\ 
 

    \theutterance \stepcounter{utterance}  

    & \multicolumn{4}{p{0.6\linewidth}}{\cellcolor[rgb]{0.8,1,0.9}{%
	\makecell[{{p{\linewidth}}}]{% 
	  \tt {\tiny [A$\rangle$GM]}  
	 > go to kitchen 
	  } 
	   } 
	   } 
	 & & \\ 
 

    \theutterance \stepcounter{utterance}  

    & & & \multicolumn{2}{p{0.3\linewidth}}{\cellcolor[rgb]{0.95,0.95,0.95}{%
	\makecell[{{p{\linewidth}}}]{% 
	  \tt {\tiny [GM$|$GM]}  
	 Round: 22 
	  } 
	   } 
	   } 
	 & & \\ 
 

    \theutterance \stepcounter{utterance}  

    & & & \multicolumn{2}{p{0.3\linewidth}}{\cellcolor[rgb]{0.95,0.95,0.95}{%
	\makecell[{{p{\linewidth}}}]{% 
	  \tt {\tiny [GM$|$GM]}  
	 {'phase': 'resolution', 'fail_type': 'no_exit_to', 'arg': {'failed_action_type': 'go', 'failed_precon_predicate': {'predicate_tuple': ['exit', 'hallway1', 'kitchen1'], 'fulfilled': False, 'precon_idx': 2}}, 'exploration_info': {'action_epistemic': True, 'action_pragmatic': True, 'effective_epistemic_gain_facts': [], 'effective_epistemic_gain_amount': 0, 'known_entities': [['at', 'pillow1', 'bedroom1'], ['at', 'cupboard1', 'kitchen1'], ['at', 'player1', 'hallway1'], ['at', 'wardrobe1', 'bedroom1'], ['at', 'chair1', 'livingroom1'], ['at', 'plate1', 'kitchen1'], ['at', 'couch1', 'livingroom1'], ['at', 'counter1', 'kitchen1'], ['at', 'book1', 'hallway1'], ['at', 'bed1', 'bedroom1'], ['at', 'pottedplant1', 'hallway1'], ['at', 'sidetable1', 'livingroom1']], 'known_entities_ratio': 0.41379310344827586, 'visited_rooms': ['kitchen1', 'bedroom1', 'hallway1', 'livingroom1'], 'visited_rooms_ratio': 0.6666666666666666, 'known_goal_entities': [['at', 'wardrobe1', 'bedroom1'], ['at', 'plate1', 'kitchen1'], ['at', 'pottedplant1', 'hallway1'], ['at', 'counter1', 'kitchen1']], 'known_goal_entities_ratio': 0.6666666666666666}} 
	  } 
	   } 
	   } 
	 & & \\ 
 

    \theutterance \stepcounter{utterance}  

    & & & \multicolumn{2}{p{0.3\linewidth}}{\cellcolor[rgb]{0.95,0.95,0.95}{%
	\makecell[{{p{\linewidth}}}]{% 
	  \tt {\tiny [GM$|$GM]}  
	 {'goal_states_achieved': [], 'turn_goal_score': 0} 
	  } 
	   } 
	   } 
	 & & \\ 
 

    \theutterance \stepcounter{utterance}  

    & & \multicolumn{4}{p{0.6\linewidth}}{\cellcolor[rgb]{0.9,0.9,0.9}{%
	\makecell[{{p{\linewidth}}}]{% 
	  \tt {\tiny [A$\langle$GM]}  
	 You can't go to a kitchen from here.\\ \tt  
	  } 
	   } 
	   } 
	 & \\ 
 

    \theutterance \stepcounter{utterance}  

    & \multicolumn{4}{p{0.6\linewidth}}{\cellcolor[rgb]{0.8,1,0.9}{%
	\makecell[{{p{\linewidth}}}]{% 
	  \tt {\tiny [A$\rangle$GM]}  
	 > go to living room 
	  } 
	   } 
	   } 
	 & & \\ 
 

    \theutterance \stepcounter{utterance}  

    & & & \multicolumn{2}{p{0.3\linewidth}}{\cellcolor[rgb]{0.95,0.95,0.95}{%
	\makecell[{{p{\linewidth}}}]{% 
	  \tt {\tiny [GM$|$GM]}  
	 Round: 23 
	  } 
	   } 
	   } 
	 & & \\ 
 

    \theutterance \stepcounter{utterance}  

    & & & \multicolumn{2}{p{0.3\linewidth}}{\cellcolor[rgb]{0.95,0.95,0.95}{%
	\makecell[{{p{\linewidth}}}]{% 
	  \tt {\tiny [GM$|$GM]}  
	 {'action_type': 'go', 'exploration_info': {'action_epistemic': True, 'action_pragmatic': True, 'effective_epistemic_gain_facts': [['at', 'player1', 'livingroom1'], ['at', 'book1', 'livingroom1'], ['at', 'pottedplant1', 'livingroom1']], 'effective_epistemic_gain_amount': 3, 'known_entities': [['at', 'cupboard1', 'kitchen1'], ['at', 'wardrobe1', 'bedroom1'], ['at', 'book1', 'livingroom1'], ['at', 'chair1', 'livingroom1'], ['at', 'pottedplant1', 'livingroom1'], ['at', 'plate1', 'kitchen1'], ['at', 'couch1', 'livingroom1'], ['at', 'counter1', 'kitchen1'], ['at', 'player1', 'livingroom1'], ['at', 'bed1', 'bedroom1'], ['at', 'pillow1', 'bedroom1'], ['at', 'sidetable1', 'livingroom1']], 'known_entities_ratio': 0.41379310344827586, 'visited_rooms': ['kitchen1', 'bedroom1', 'hallway1', 'livingroom1'], 'visited_rooms_ratio': 0.6666666666666666, 'known_goal_entities': [['at', 'wardrobe1', 'bedroom1'], ['at', 'plate1', 'kitchen1'], ['at', 'counter1', 'kitchen1'], ['at', 'pottedplant1', 'livingroom1']], 'known_goal_entities_ratio': 0.6666666666666666}} 
	  } 
	   } 
	   } 
	 & & \\ 
 

    \theutterance \stepcounter{utterance}  

    & & & \multicolumn{2}{p{0.3\linewidth}}{\cellcolor[rgb]{0.95,0.95,0.95}{%
	\makecell[{{p{\linewidth}}}]{% 
	  \tt {\tiny [GM$|$GM]}  
	 {'goal_states_achieved': [], 'turn_goal_score': 0} 
	  } 
	   } 
	   } 
	 & & \\ 
 

    \theutterance \stepcounter{utterance}  

    & & \multicolumn{4}{p{0.6\linewidth}}{\cellcolor[rgb]{0.9,0.9,0.9}{%
	\makecell[{{p{\linewidth}}}]{% 
	  \tt {\tiny [A$\langle$GM]}  
	 You are in a living room now. There are a couch, a chair and a side table. There are passages to a bedroom, a hallway and a kitchen here.\\ \tt  
	  } 
	   } 
	   } 
	 & \\ 
 

    \theutterance \stepcounter{utterance}  

    & \multicolumn{4}{p{0.6\linewidth}}{\cellcolor[rgb]{0.8,1,0.9}{%
	\makecell[{{p{\linewidth}}}]{% 
	  \tt {\tiny [A$\rangle$GM]}  
	 > put potted plant on counter 
	  } 
	   } 
	   } 
	 & & \\ 
 

    \theutterance \stepcounter{utterance}  

    & & & \multicolumn{2}{p{0.3\linewidth}}{\cellcolor[rgb]{0.95,0.95,0.95}{%
	\makecell[{{p{\linewidth}}}]{% 
	  \tt {\tiny [GM$|$GM]}  
	 Round: 24 
	  } 
	   } 
	   } 
	 & & \\ 
 

    \theutterance \stepcounter{utterance}  

    & & & \multicolumn{2}{p{0.3\linewidth}}{\cellcolor[rgb]{0.95,0.95,0.95}{%
	\makecell[{{p{\linewidth}}}]{% 
	  \tt {\tiny [GM$|$GM]}  
	 {'phase': 'resolution', 'fail_type': 'entity_not_accessible', 'arg': {'failed_action_type': 'put', 'failed_precon_predicate': {'predicate_tuple': ['at', 'counter1', 'livingroom1'], 'fulfilled': False, 'precon_idx': 2}}, 'exploration_info': {'action_epistemic': False, 'action_pragmatic': True, 'effective_epistemic_gain_amount': 0, 'known_entities': [['at', 'cupboard1', 'kitchen1'], ['at', 'wardrobe1', 'bedroom1'], ['at', 'book1', 'livingroom1'], ['at', 'chair1', 'livingroom1'], ['at', 'pottedplant1', 'livingroom1'], ['at', 'plate1', 'kitchen1'], ['at', 'couch1', 'livingroom1'], ['at', 'counter1', 'kitchen1'], ['at', 'player1', 'livingroom1'], ['at', 'bed1', 'bedroom1'], ['at', 'pillow1', 'bedroom1'], ['at', 'sidetable1', 'livingroom1']], 'known_entities_ratio': 0.41379310344827586, 'visited_rooms': ['kitchen1', 'bedroom1', 'hallway1', 'livingroom1'], 'visited_rooms_ratio': 0.6666666666666666, 'known_goal_entities': [['at', 'wardrobe1', 'bedroom1'], ['at', 'plate1', 'kitchen1'], ['at', 'counter1', 'kitchen1'], ['at', 'pottedplant1', 'livingroom1']], 'known_goal_entities_ratio': 0.6666666666666666}} 
	  } 
	   } 
	   } 
	 & & \\ 
 

    \theutterance \stepcounter{utterance}  

    & & & \multicolumn{2}{p{0.3\linewidth}}{\cellcolor[rgb]{0.95,0.95,0.95}{%
	\makecell[{{p{\linewidth}}}]{% 
	  \tt {\tiny [GM$|$GM]}  
	 {'goal_states_achieved': [], 'turn_goal_score': 0} 
	  } 
	   } 
	   } 
	 & & \\ 
 

    \theutterance \stepcounter{utterance}  

    & & \multicolumn{4}{p{0.6\linewidth}}{\cellcolor[rgb]{0.9,0.9,0.9}{%
	\makecell[{{p{\linewidth}}}]{% 
	  \tt {\tiny [A$\langle$GM]}  
	 You can't see a counter here.\\ \tt  
	  } 
	   } 
	   } 
	 & \\ 
 

    \theutterance \stepcounter{utterance}  

    & \multicolumn{4}{p{0.6\linewidth}}{\cellcolor[rgb]{0.8,1,0.9}{%
	\makecell[{{p{\linewidth}}}]{% 
	  \tt {\tiny [A$\rangle$GM]}  
	 > go to kitchen 
	  } 
	   } 
	   } 
	 & & \\ 
 

    \theutterance \stepcounter{utterance}  

    & & & \multicolumn{2}{p{0.3\linewidth}}{\cellcolor[rgb]{0.95,0.95,0.95}{%
	\makecell[{{p{\linewidth}}}]{% 
	  \tt {\tiny [GM$|$GM]}  
	 Round: 25 
	  } 
	   } 
	   } 
	 & & \\ 
 

    \theutterance \stepcounter{utterance}  

    & & & \multicolumn{2}{p{0.3\linewidth}}{\cellcolor[rgb]{0.95,0.95,0.95}{%
	\makecell[{{p{\linewidth}}}]{% 
	  \tt {\tiny [GM$|$GM]}  
	 {'action_type': 'go', 'exploration_info': {'action_epistemic': True, 'action_pragmatic': True, 'effective_epistemic_gain_facts': [['at', 'pottedplant1', 'kitchen1'], ['at', 'player1', 'kitchen1'], ['at', 'book1', 'kitchen1']], 'effective_epistemic_gain_amount': 3, 'known_entities': [['at', 'pottedplant1', 'kitchen1'], ['at', 'cupboard1', 'kitchen1'], ['at', 'wardrobe1', 'bedroom1'], ['at', 'chair1', 'livingroom1'], ['at', 'player1', 'kitchen1'], ['at', 'plate1', 'kitchen1'], ['at', 'couch1', 'livingroom1'], ['at', 'counter1', 'kitchen1'], ['at', 'bed1', 'bedroom1'], ['at', 'pillow1', 'bedroom1'], ['at', 'sidetable1', 'livingroom1'], ['at', 'book1', 'kitchen1']], 'known_entities_ratio': 0.41379310344827586, 'visited_rooms': ['kitchen1', 'bedroom1', 'hallway1', 'livingroom1'], 'visited_rooms_ratio': 0.6666666666666666, 'known_goal_entities': [['at', 'wardrobe1', 'bedroom1'], ['at', 'plate1', 'kitchen1'], ['at', 'pottedplant1', 'kitchen1'], ['at', 'counter1', 'kitchen1']], 'known_goal_entities_ratio': 0.6666666666666666}} 
	  } 
	   } 
	   } 
	 & & \\ 
 

    \theutterance \stepcounter{utterance}  

    & & & \multicolumn{2}{p{0.3\linewidth}}{\cellcolor[rgb]{0.95,0.95,0.95}{%
	\makecell[{{p{\linewidth}}}]{% 
	  \tt {\tiny [GM$|$GM]}  
	 {'goal_states_achieved': [], 'turn_goal_score': 0} 
	  } 
	   } 
	   } 
	 & & \\ 
 

    \theutterance \stepcounter{utterance}  

    & & \multicolumn{4}{p{0.6\linewidth}}{\cellcolor[rgb]{0.9,0.9,0.9}{%
	\makecell[{{p{\linewidth}}}]{% 
	  \tt {\tiny [A$\langle$GM]}  
	 You are in a kitchen now. There are a cupboard, a plate and a counter. The cupboard is open. The plate is on the counter. There are passages to a living room and a pantry here.\\ \tt  
	  } 
	   } 
	   } 
	 & \\ 
 

    \theutterance \stepcounter{utterance}  

    & \multicolumn{4}{p{0.6\linewidth}}{\cellcolor[rgb]{0.8,1,0.9}{%
	\makecell[{{p{\linewidth}}}]{% 
	  \tt {\tiny [A$\rangle$GM]}  
	 > put potted plant on counter 
	  } 
	   } 
	   } 
	 & & \\ 
 

    \theutterance \stepcounter{utterance}  

    & & & \multicolumn{2}{p{0.3\linewidth}}{\cellcolor[rgb]{0.95,0.95,0.95}{%
	\makecell[{{p{\linewidth}}}]{% 
	  \tt {\tiny [GM$|$GM]}  
	 Round: 26 
	  } 
	   } 
	   } 
	 & & \\ 
 

    \theutterance \stepcounter{utterance}  

    & & & \multicolumn{2}{p{0.3\linewidth}}{\cellcolor[rgb]{0.95,0.95,0.95}{%
	\makecell[{{p{\linewidth}}}]{% 
	  \tt {\tiny [GM$|$GM]}  
	 {'action_type': 'put', 'exploration_info': {'action_epistemic': False, 'action_pragmatic': True, 'effective_epistemic_gain_amount': 0, 'known_entities': [['at', 'pottedplant1', 'kitchen1'], ['at', 'cupboard1', 'kitchen1'], ['at', 'wardrobe1', 'bedroom1'], ['at', 'chair1', 'livingroom1'], ['at', 'player1', 'kitchen1'], ['at', 'plate1', 'kitchen1'], ['at', 'couch1', 'livingroom1'], ['at', 'counter1', 'kitchen1'], ['at', 'bed1', 'bedroom1'], ['at', 'pillow1', 'bedroom1'], ['at', 'sidetable1', 'livingroom1'], ['at', 'book1', 'kitchen1']], 'known_entities_ratio': 0.41379310344827586, 'visited_rooms': ['kitchen1', 'bedroom1', 'hallway1', 'livingroom1'], 'visited_rooms_ratio': 0.6666666666666666, 'known_goal_entities': [['at', 'wardrobe1', 'bedroom1'], ['at', 'plate1', 'kitchen1'], ['at', 'pottedplant1', 'kitchen1'], ['at', 'counter1', 'kitchen1']], 'known_goal_entities_ratio': 0.6666666666666666}} 
	  } 
	   } 
	   } 
	 & & \\ 
 

    \theutterance \stepcounter{utterance}  

    & & & \multicolumn{2}{p{0.3\linewidth}}{\cellcolor[rgb]{0.95,0.95,0.95}{%
	\makecell[{{p{\linewidth}}}]{% 
	  \tt {\tiny [GM$|$GM]}  
	 {'goal_states_achieved': ['on(pottedplant1,counter1)'], 'turn_goal_score': 1} 
	  } 
	   } 
	   } 
	 & & \\ 
 

    \theutterance \stepcounter{utterance}  

    & & \multicolumn{4}{p{0.6\linewidth}}{\cellcolor[rgb]{0.9,0.9,0.9}{%
	\makecell[{{p{\linewidth}}}]{% 
	  \tt {\tiny [A$\langle$GM]}  
	 You put the potted plant on the counter.\\ \tt  
	  } 
	   } 
	   } 
	 & \\ 
 

    \theutterance \stepcounter{utterance}  

    & \multicolumn{4}{p{0.6\linewidth}}{\cellcolor[rgb]{0.8,1,0.9}{%
	\makecell[{{p{\linewidth}}}]{% 
	  \tt {\tiny [A$\rangle$GM]}  
	 > put book on shelf 
	  } 
	   } 
	   } 
	 & & \\ 
 

    \theutterance \stepcounter{utterance}  

    & & & \multicolumn{2}{p{0.3\linewidth}}{\cellcolor[rgb]{0.95,0.95,0.95}{%
	\makecell[{{p{\linewidth}}}]{% 
	  \tt {\tiny [GM$|$GM]}  
	 Round: 27 
	  } 
	   } 
	   } 
	 & & \\ 
 

    \theutterance \stepcounter{utterance}  

    & & & \multicolumn{2}{p{0.3\linewidth}}{\cellcolor[rgb]{0.95,0.95,0.95}{%
	\makecell[{{p{\linewidth}}}]{% 
	  \tt {\tiny [GM$|$GM]}  
	 {'phase': 'resolution', 'fail_type': 'entity_not_accessible', 'arg': {'failed_action_type': 'put', 'failed_precon_predicate': {'predicate_tuple': ['at', 'shelf1', 'kitchen1'], 'fulfilled': False, 'precon_idx': 2}}, 'exploration_info': {'action_epistemic': False, 'action_pragmatic': True, 'effective_epistemic_gain_amount': 0, 'known_entities': [['at', 'pottedplant1', 'kitchen1'], ['at', 'cupboard1', 'kitchen1'], ['at', 'wardrobe1', 'bedroom1'], ['at', 'chair1', 'livingroom1'], ['at', 'player1', 'kitchen1'], ['at', 'plate1', 'kitchen1'], ['at', 'couch1', 'livingroom1'], ['at', 'counter1', 'kitchen1'], ['at', 'bed1', 'bedroom1'], ['at', 'pillow1', 'bedroom1'], ['at', 'sidetable1', 'livingroom1'], ['at', 'book1', 'kitchen1']], 'known_entities_ratio': 0.41379310344827586, 'visited_rooms': ['kitchen1', 'bedroom1', 'hallway1', 'livingroom1'], 'visited_rooms_ratio': 0.6666666666666666, 'known_goal_entities': [['at', 'wardrobe1', 'bedroom1'], ['at', 'plate1', 'kitchen1'], ['at', 'pottedplant1', 'kitchen1'], ['at', 'counter1', 'kitchen1']], 'known_goal_entities_ratio': 0.6666666666666666}} 
	  } 
	   } 
	   } 
	 & & \\ 
 

    \theutterance \stepcounter{utterance}  

    & & & \multicolumn{2}{p{0.3\linewidth}}{\cellcolor[rgb]{0.95,0.95,0.95}{%
	\makecell[{{p{\linewidth}}}]{% 
	  \tt {\tiny [GM$|$GM]}  
	 {'goal_states_achieved': [['on', 'pottedplant1', 'counter1']], 'turn_goal_score': 0} 
	  } 
	   } 
	   } 
	 & & \\ 
 

    \theutterance \stepcounter{utterance}  

    & & \multicolumn{4}{p{0.6\linewidth}}{\cellcolor[rgb]{0.9,0.9,0.9}{%
	\makecell[{{p{\linewidth}}}]{% 
	  \tt {\tiny [A$\langle$GM]}  
	 You can't see a shelf here.\\ \tt  
	  } 
	   } 
	   } 
	 & \\ 
 

    \theutterance \stepcounter{utterance}  

    & \multicolumn{4}{p{0.6\linewidth}}{\cellcolor[rgb]{0.8,1,0.9}{%
	\makecell[{{p{\linewidth}}}]{% 
	  \tt {\tiny [A$\rangle$GM]}  
	 > go to living room 
	  } 
	   } 
	   } 
	 & & \\ 
 

    \theutterance \stepcounter{utterance}  

    & & & \multicolumn{2}{p{0.3\linewidth}}{\cellcolor[rgb]{0.95,0.95,0.95}{%
	\makecell[{{p{\linewidth}}}]{% 
	  \tt {\tiny [GM$|$GM]}  
	 Round: 28 
	  } 
	   } 
	   } 
	 & & \\ 
 

    \theutterance \stepcounter{utterance}  

    & & & \multicolumn{2}{p{0.3\linewidth}}{\cellcolor[rgb]{0.95,0.95,0.95}{%
	\makecell[{{p{\linewidth}}}]{% 
	  \tt {\tiny [GM$|$GM]}  
	 {'action_type': 'go', 'exploration_info': {'action_epistemic': True, 'action_pragmatic': True, 'effective_epistemic_gain_facts': [['at', 'player1', 'livingroom1'], ['at', 'book1', 'livingroom1']], 'effective_epistemic_gain_amount': 2, 'known_entities': [['at', 'pottedplant1', 'kitchen1'], ['at', 'cupboard1', 'kitchen1'], ['at', 'wardrobe1', 'bedroom1'], ['at', 'book1', 'livingroom1'], ['at', 'chair1', 'livingroom1'], ['at', 'plate1', 'kitchen1'], ['at', 'couch1', 'livingroom1'], ['at', 'counter1', 'kitchen1'], ['at', 'player1', 'livingroom1'], ['at', 'bed1', 'bedroom1'], ['at', 'pillow1', 'bedroom1'], ['at', 'sidetable1', 'livingroom1']], 'known_entities_ratio': 0.41379310344827586, 'visited_rooms': ['kitchen1', 'bedroom1', 'hallway1', 'livingroom1'], 'visited_rooms_ratio': 0.6666666666666666, 'known_goal_entities': [['at', 'wardrobe1', 'bedroom1'], ['at', 'plate1', 'kitchen1'], ['at', 'pottedplant1', 'kitchen1'], ['at', 'counter1', 'kitchen1']], 'known_goal_entities_ratio': 0.6666666666666666}} 
	  } 
	   } 
	   } 
	 & & \\ 
 

    \theutterance \stepcounter{utterance}  

    & & & \multicolumn{2}{p{0.3\linewidth}}{\cellcolor[rgb]{0.95,0.95,0.95}{%
	\makecell[{{p{\linewidth}}}]{% 
	  \tt {\tiny [GM$|$GM]}  
	 {'goal_states_achieved': ['on(pottedplant1,counter1)'], 'turn_goal_score': 0} 
	  } 
	   } 
	   } 
	 & & \\ 
 

    \theutterance \stepcounter{utterance}  

    & & \multicolumn{4}{p{0.6\linewidth}}{\cellcolor[rgb]{0.9,0.9,0.9}{%
	\makecell[{{p{\linewidth}}}]{% 
	  \tt {\tiny [A$\langle$GM]}  
	 You are in a living room now. There are a couch, a chair and a side table. There are passages to a bedroom, a hallway and a kitchen here.\\ \tt  
	  } 
	   } 
	   } 
	 & \\ 
 

    \theutterance \stepcounter{utterance}  

    & \multicolumn{4}{p{0.6\linewidth}}{\cellcolor[rgb]{0.8,1,0.9}{%
	\makecell[{{p{\linewidth}}}]{% 
	  \tt {\tiny [A$\rangle$GM]}  
	 > go to bedroom 
	  } 
	   } 
	   } 
	 & & \\ 
 

    \theutterance \stepcounter{utterance}  

    & & & \multicolumn{2}{p{0.3\linewidth}}{\cellcolor[rgb]{0.95,0.95,0.95}{%
	\makecell[{{p{\linewidth}}}]{% 
	  \tt {\tiny [GM$|$GM]}  
	 Round: 29 
	  } 
	   } 
	   } 
	 & & \\ 
 

    \theutterance \stepcounter{utterance}  

    & & & \multicolumn{2}{p{0.3\linewidth}}{\cellcolor[rgb]{0.95,0.95,0.95}{%
	\makecell[{{p{\linewidth}}}]{% 
	  \tt {\tiny [GM$|$GM]}  
	 {'action_type': 'go', 'exploration_info': {'action_epistemic': True, 'action_pragmatic': True, 'effective_epistemic_gain_facts': [['at', 'player1', 'bedroom1'], ['at', 'book1', 'bedroom1']], 'effective_epistemic_gain_amount': 2, 'known_entities': [['at', 'player1', 'bedroom1'], ['at', 'pottedplant1', 'kitchen1'], ['at', 'cupboard1', 'kitchen1'], ['at', 'wardrobe1', 'bedroom1'], ['at', 'chair1', 'livingroom1'], ['at', 'plate1', 'kitchen1'], ['at', 'book1', 'bedroom1'], ['at', 'couch1', 'livingroom1'], ['at', 'counter1', 'kitchen1'], ['at', 'bed1', 'bedroom1'], ['at', 'pillow1', 'bedroom1'], ['at', 'sidetable1', 'livingroom1']], 'known_entities_ratio': 0.41379310344827586, 'visited_rooms': ['kitchen1', 'bedroom1', 'hallway1', 'livingroom1'], 'visited_rooms_ratio': 0.6666666666666666, 'known_goal_entities': [['at', 'wardrobe1', 'bedroom1'], ['at', 'plate1', 'kitchen1'], ['at', 'pottedplant1', 'kitchen1'], ['at', 'counter1', 'kitchen1']], 'known_goal_entities_ratio': 0.6666666666666666}} 
	  } 
	   } 
	   } 
	 & & \\ 
 

    \theutterance \stepcounter{utterance}  

    & & & \multicolumn{2}{p{0.3\linewidth}}{\cellcolor[rgb]{0.95,0.95,0.95}{%
	\makecell[{{p{\linewidth}}}]{% 
	  \tt {\tiny [GM$|$GM]}  
	 {'goal_states_achieved': ['on(pottedplant1,counter1)'], 'turn_goal_score': 0} 
	  } 
	   } 
	   } 
	 & & \\ 
 

    \theutterance \stepcounter{utterance}  

    & & \multicolumn{4}{p{0.6\linewidth}}{\cellcolor[rgb]{0.9,0.9,0.9}{%
	\makecell[{{p{\linewidth}}}]{% 
	  \tt {\tiny [A$\langle$GM]}  
	 You are in a bedroom now. There are a bed, a wardrobe and a pillow. The wardrobe is open. The pillow is in the wardrobe. There is a passage to a living room here.\\ \tt  
	  } 
	   } 
	   } 
	 & \\ 
 

    \theutterance \stepcounter{utterance}  

    & \multicolumn{4}{p{0.6\linewidth}}{\cellcolor[rgb]{0.8,1,0.9}{%
	\makecell[{{p{\linewidth}}}]{% 
	  \tt {\tiny [A$\rangle$GM]}  
	 > put book on shelf 
	  } 
	   } 
	   } 
	 & & \\ 
 

    \theutterance \stepcounter{utterance}  

    & & & \multicolumn{2}{p{0.3\linewidth}}{\cellcolor[rgb]{0.95,0.95,0.95}{%
	\makecell[{{p{\linewidth}}}]{% 
	  \tt {\tiny [GM$|$GM]}  
	 Round: 30 
	  } 
	   } 
	   } 
	 & & \\ 
 

    \theutterance \stepcounter{utterance}  

    & & & \multicolumn{2}{p{0.3\linewidth}}{\cellcolor[rgb]{0.95,0.95,0.95}{%
	\makecell[{{p{\linewidth}}}]{% 
	  \tt {\tiny [GM$|$GM]}  
	 {'phase': 'resolution', 'fail_type': 'entity_not_accessible', 'arg': {'failed_action_type': 'put', 'failed_precon_predicate': {'predicate_tuple': ['at', 'shelf1', 'bedroom1'], 'fulfilled': False, 'precon_idx': 2}}, 'exploration_info': {'action_epistemic': False, 'action_pragmatic': True, 'effective_epistemic_gain_amount': 0, 'known_entities': [['at', 'player1', 'bedroom1'], ['at', 'pottedplant1', 'kitchen1'], ['at', 'cupboard1', 'kitchen1'], ['at', 'wardrobe1', 'bedroom1'], ['at', 'chair1', 'livingroom1'], ['at', 'plate1', 'kitchen1'], ['at', 'book1', 'bedroom1'], ['at', 'couch1', 'livingroom1'], ['at', 'counter1', 'kitchen1'], ['at', 'bed1', 'bedroom1'], ['at', 'pillow1', 'bedroom1'], ['at', 'sidetable1', 'livingroom1']], 'known_entities_ratio': 0.41379310344827586, 'visited_rooms': ['kitchen1', 'bedroom1', 'hallway1', 'livingroom1'], 'visited_rooms_ratio': 0.6666666666666666, 'known_goal_entities': [['at', 'wardrobe1', 'bedroom1'], ['at', 'plate1', 'kitchen1'], ['at', 'pottedplant1', 'kitchen1'], ['at', 'counter1', 'kitchen1']], 'known_goal_entities_ratio': 0.6666666666666666}} 
	  } 
	   } 
	   } 
	 & & \\ 
 

    \theutterance \stepcounter{utterance}  

    & & & \multicolumn{2}{p{0.3\linewidth}}{\cellcolor[rgb]{0.95,0.95,0.95}{%
	\makecell[{{p{\linewidth}}}]{% 
	  \tt {\tiny [GM$|$GM]}  
	 {'goal_states_achieved': [['on', 'pottedplant1', 'counter1']], 'turn_goal_score': 0} 
	  } 
	   } 
	   } 
	 & & \\ 
 

    \theutterance \stepcounter{utterance}  

    & & \multicolumn{4}{p{0.6\linewidth}}{\cellcolor[rgb]{0.9,0.9,0.9}{%
	\makecell[{{p{\linewidth}}}]{% 
	  \tt {\tiny [A$\langle$GM]}  
	 You can't see a shelf here.\\ \tt  
	  } 
	   } 
	   } 
	 & \\ 
 

    \theutterance \stepcounter{utterance}  

    & \multicolumn{4}{p{0.6\linewidth}}{\cellcolor[rgb]{0.8,1,0.9}{%
	\makecell[{{p{\linewidth}}}]{% 
	  \tt {\tiny [A$\rangle$GM]}  
	 > go to living room 
	  } 
	   } 
	   } 
	 & & \\ 
 

    \theutterance \stepcounter{utterance}  

    & & & \multicolumn{2}{p{0.3\linewidth}}{\cellcolor[rgb]{0.95,0.95,0.95}{%
	\makecell[{{p{\linewidth}}}]{% 
	  \tt {\tiny [GM$|$GM]}  
	 Round: 31 
	  } 
	   } 
	   } 
	 & & \\ 
 

    \theutterance \stepcounter{utterance}  

    & & & \multicolumn{2}{p{0.3\linewidth}}{\cellcolor[rgb]{0.95,0.95,0.95}{%
	\makecell[{{p{\linewidth}}}]{% 
	  \tt {\tiny [GM$|$GM]}  
	 {'action_type': 'go', 'exploration_info': {'action_epistemic': True, 'action_pragmatic': True, 'effective_epistemic_gain_facts': [['at', 'player1', 'livingroom1'], ['at', 'book1', 'livingroom1']], 'effective_epistemic_gain_amount': 2, 'known_entities': [['at', 'pottedplant1', 'kitchen1'], ['at', 'cupboard1', 'kitchen1'], ['at', 'wardrobe1', 'bedroom1'], ['at', 'book1', 'livingroom1'], ['at', 'chair1', 'livingroom1'], ['at', 'plate1', 'kitchen1'], ['at', 'couch1', 'livingroom1'], ['at', 'counter1', 'kitchen1'], ['at', 'player1', 'livingroom1'], ['at', 'bed1', 'bedroom1'], ['at', 'pillow1', 'bedroom1'], ['at', 'sidetable1', 'livingroom1']], 'known_entities_ratio': 0.41379310344827586, 'visited_rooms': ['kitchen1', 'bedroom1', 'hallway1', 'livingroom1'], 'visited_rooms_ratio': 0.6666666666666666, 'known_goal_entities': [['at', 'wardrobe1', 'bedroom1'], ['at', 'plate1', 'kitchen1'], ['at', 'pottedplant1', 'kitchen1'], ['at', 'counter1', 'kitchen1']], 'known_goal_entities_ratio': 0.6666666666666666}} 
	  } 
	   } 
	   } 
	 & & \\ 
 

    \theutterance \stepcounter{utterance}  

    & & & \multicolumn{2}{p{0.3\linewidth}}{\cellcolor[rgb]{0.95,0.95,0.95}{%
	\makecell[{{p{\linewidth}}}]{% 
	  \tt {\tiny [GM$|$GM]}  
	 {'goal_states_achieved': ['on(pottedplant1,counter1)'], 'turn_goal_score': 0} 
	  } 
	   } 
	   } 
	 & & \\ 
 

    \theutterance \stepcounter{utterance}  

    & & \multicolumn{4}{p{0.6\linewidth}}{\cellcolor[rgb]{0.9,0.9,0.9}{%
	\makecell[{{p{\linewidth}}}]{% 
	  \tt {\tiny [A$\langle$GM]}  
	 You are in a living room now. There are a couch, a chair and a side table. There are passages to a bedroom, a hallway and a kitchen here.\\ \tt  
	  } 
	   } 
	   } 
	 & \\ 
 

    \theutterance \stepcounter{utterance}  

    & \multicolumn{4}{p{0.6\linewidth}}{\cellcolor[rgb]{0.8,1,0.9}{%
	\makecell[{{p{\linewidth}}}]{% 
	  \tt {\tiny [A$\rangle$GM]}  
	 > go to hallway 
	  } 
	   } 
	   } 
	 & & \\ 
 

    \theutterance \stepcounter{utterance}  

    & & & \multicolumn{2}{p{0.3\linewidth}}{\cellcolor[rgb]{0.95,0.95,0.95}{%
	\makecell[{{p{\linewidth}}}]{% 
	  \tt {\tiny [GM$|$GM]}  
	 Round: 32 
	  } 
	   } 
	   } 
	 & & \\ 
 

    \theutterance \stepcounter{utterance}  

    & & & \multicolumn{2}{p{0.3\linewidth}}{\cellcolor[rgb]{0.95,0.95,0.95}{%
	\makecell[{{p{\linewidth}}}]{% 
	  \tt {\tiny [GM$|$GM]}  
	 {'action_type': 'go', 'exploration_info': {'action_epistemic': True, 'action_pragmatic': True, 'effective_epistemic_gain_facts': [['at', 'player1', 'hallway1'], ['at', 'book1', 'hallway1']], 'effective_epistemic_gain_amount': 2, 'known_entities': [['at', 'pottedplant1', 'kitchen1'], ['at', 'cupboard1', 'kitchen1'], ['at', 'player1', 'hallway1'], ['at', 'wardrobe1', 'bedroom1'], ['at', 'chair1', 'livingroom1'], ['at', 'plate1', 'kitchen1'], ['at', 'couch1', 'livingroom1'], ['at', 'counter1', 'kitchen1'], ['at', 'book1', 'hallway1'], ['at', 'bed1', 'bedroom1'], ['at', 'pillow1', 'bedroom1'], ['at', 'sidetable1', 'livingroom1']], 'known_entities_ratio': 0.41379310344827586, 'visited_rooms': ['kitchen1', 'bedroom1', 'hallway1', 'livingroom1'], 'visited_rooms_ratio': 0.6666666666666666, 'known_goal_entities': [['at', 'wardrobe1', 'bedroom1'], ['at', 'plate1', 'kitchen1'], ['at', 'pottedplant1', 'kitchen1'], ['at', 'counter1', 'kitchen1']], 'known_goal_entities_ratio': 0.6666666666666666}} 
	  } 
	   } 
	   } 
	 & & \\ 
 

    \theutterance \stepcounter{utterance}  

    & & & \multicolumn{2}{p{0.3\linewidth}}{\cellcolor[rgb]{0.95,0.95,0.95}{%
	\makecell[{{p{\linewidth}}}]{% 
	  \tt {\tiny [GM$|$GM]}  
	 {'goal_states_achieved': ['on(pottedplant1,counter1)'], 'turn_goal_score': 0} 
	  } 
	   } 
	   } 
	 & & \\ 
 

    \theutterance \stepcounter{utterance}  

    & & \multicolumn{4}{p{0.6\linewidth}}{\cellcolor[rgb]{0.9,0.9,0.9}{%
	\makecell[{{p{\linewidth}}}]{% 
	  \tt {\tiny [A$\langle$GM]}  
	 You are in a hallway now. There are passages to a broom closet and a living room here.\\ \tt  
	  } 
	   } 
	   } 
	 & \\ 
 

    \theutterance \stepcounter{utterance}  

    & \multicolumn{4}{p{0.6\linewidth}}{\cellcolor[rgb]{0.8,1,0.9}{%
	\makecell[{{p{\linewidth}}}]{% 
	  \tt {\tiny [A$\rangle$GM]}  
	 > go to broom closet 
	  } 
	   } 
	   } 
	 & & \\ 
 

    \theutterance \stepcounter{utterance}  

    & & & \multicolumn{2}{p{0.3\linewidth}}{\cellcolor[rgb]{0.95,0.95,0.95}{%
	\makecell[{{p{\linewidth}}}]{% 
	  \tt {\tiny [GM$|$GM]}  
	 Round: 33 
	  } 
	   } 
	   } 
	 & & \\ 
 

    \theutterance \stepcounter{utterance}  

    & & & \multicolumn{2}{p{0.3\linewidth}}{\cellcolor[rgb]{0.95,0.95,0.95}{%
	\makecell[{{p{\linewidth}}}]{% 
	  \tt {\tiny [GM$|$GM]}  
	 {'action_type': 'go', 'exploration_info': {'action_epistemic': True, 'action_pragmatic': True, 'effective_epistemic_gain_facts': [['exit', 'broomcloset1', 'hallway1'], ['at', 'mop1', 'broomcloset1'], ['on', 'broom1', 'broomcloset1floor1'], ['on', 'mop1', 'broomcloset1floor1'], ['at', 'book1', 'broomcloset1'], ['at', 'broom1', 'broomcloset1'], ['at', 'player1', 'broomcloset1']], 'effective_epistemic_gain_amount': 7, 'known_entities': [['at', 'pottedplant1', 'kitchen1'], ['at', 'cupboard1', 'kitchen1'], ['at', 'mop1', 'broomcloset1'], ['at', 'wardrobe1', 'bedroom1'], ['at', 'chair1', 'livingroom1'], ['at', 'book1', 'broomcloset1'], ['at', 'plate1', 'kitchen1'], ['at', 'couch1', 'livingroom1'], ['at', 'counter1', 'kitchen1'], ['at', 'broom1', 'broomcloset1'], ['at', 'bed1', 'bedroom1'], ['at', 'player1', 'broomcloset1'], ['at', 'pillow1', 'bedroom1'], ['at', 'sidetable1', 'livingroom1']], 'known_entities_ratio': 0.4827586206896552, 'visited_rooms': ['livingroom1', 'hallway1', 'broomcloset1', 'kitchen1', 'bedroom1'], 'visited_rooms_ratio': 0.8333333333333334, 'known_goal_entities': [['at', 'pottedplant1', 'kitchen1'], ['at', 'wardrobe1', 'bedroom1'], ['at', 'plate1', 'kitchen1'], ['at', 'counter1', 'kitchen1'], ['at', 'broom1', 'broomcloset1']], 'known_goal_entities_ratio': 0.8333333333333334}} 
	  } 
	   } 
	   } 
	 & & \\ 
 

    \theutterance \stepcounter{utterance}  

    & & & \multicolumn{2}{p{0.3\linewidth}}{\cellcolor[rgb]{0.95,0.95,0.95}{%
	\makecell[{{p{\linewidth}}}]{% 
	  \tt {\tiny [GM$|$GM]}  
	 {'goal_states_achieved': ['on(pottedplant1,counter1)'], 'turn_goal_score': 0} 
	  } 
	   } 
	   } 
	 & & \\ 
 

    \theutterance \stepcounter{utterance}  

    & & \multicolumn{4}{p{0.6\linewidth}}{\cellcolor[rgb]{0.9,0.9,0.9}{%
	\makecell[{{p{\linewidth}}}]{% 
	  \tt {\tiny [A$\langle$GM]}  
	 You are in a broom closet now. There are a mop and a broom. The mop is on the floor. The broom is on the floor. There is a passage to a hallway here.\\ \tt  
	  } 
	   } 
	   } 
	 & \\ 
 

    \theutterance \stepcounter{utterance}  

    & \multicolumn{4}{p{0.6\linewidth}}{\cellcolor[rgb]{0.8,1,0.9}{%
	\makecell[{{p{\linewidth}}}]{% 
	  \tt {\tiny [A$\rangle$GM]}  
	 > take broom 
	  } 
	   } 
	   } 
	 & & \\ 
 

    \theutterance \stepcounter{utterance}  

    & & & \multicolumn{2}{p{0.3\linewidth}}{\cellcolor[rgb]{0.95,0.95,0.95}{%
	\makecell[{{p{\linewidth}}}]{% 
	  \tt {\tiny [GM$|$GM]}  
	 Round: 34 
	  } 
	   } 
	   } 
	 & & \\ 
 

    \theutterance \stepcounter{utterance}  

    & & & \multicolumn{2}{p{0.3\linewidth}}{\cellcolor[rgb]{0.95,0.95,0.95}{%
	\makecell[{{p{\linewidth}}}]{% 
	  \tt {\tiny [GM$|$GM]}  
	 {'action_type': 'take', 'exploration_info': {'action_epistemic': False, 'action_pragmatic': True, 'effective_epistemic_gain_amount': 0, 'known_entities': [['at', 'pottedplant1', 'kitchen1'], ['at', 'cupboard1', 'kitchen1'], ['at', 'mop1', 'broomcloset1'], ['at', 'wardrobe1', 'bedroom1'], ['at', 'chair1', 'livingroom1'], ['at', 'book1', 'broomcloset1'], ['at', 'plate1', 'kitchen1'], ['at', 'couch1', 'livingroom1'], ['at', 'counter1', 'kitchen1'], ['at', 'broom1', 'broomcloset1'], ['at', 'bed1', 'bedroom1'], ['at', 'player1', 'broomcloset1'], ['at', 'pillow1', 'bedroom1'], ['at', 'sidetable1', 'livingroom1']], 'known_entities_ratio': 0.4827586206896552, 'visited_rooms': ['livingroom1', 'hallway1', 'broomcloset1', 'kitchen1', 'bedroom1'], 'visited_rooms_ratio': 0.8333333333333334, 'known_goal_entities': [['at', 'pottedplant1', 'kitchen1'], ['at', 'wardrobe1', 'bedroom1'], ['at', 'plate1', 'kitchen1'], ['at', 'counter1', 'kitchen1'], ['at', 'broom1', 'broomcloset1']], 'known_goal_entities_ratio': 0.8333333333333334}} 
	  } 
	   } 
	   } 
	 & & \\ 
 

    \theutterance \stepcounter{utterance}  

    & & & \multicolumn{2}{p{0.3\linewidth}}{\cellcolor[rgb]{0.95,0.95,0.95}{%
	\makecell[{{p{\linewidth}}}]{% 
	  \tt {\tiny [GM$|$GM]}  
	 {'goal_states_achieved': ['on(pottedplant1,counter1)'], 'turn_goal_score': 0} 
	  } 
	   } 
	   } 
	 & & \\ 
 

    \theutterance \stepcounter{utterance}  

    & & \multicolumn{4}{p{0.6\linewidth}}{\cellcolor[rgb]{0.9,0.9,0.9}{%
	\makecell[{{p{\linewidth}}}]{% 
	  \tt {\tiny [A$\langle$GM]}  
	 You take the broom. In your inventory you have a broom and a book.\\ \tt  
	  } 
	   } 
	   } 
	 & \\ 
 

    \theutterance \stepcounter{utterance}  

    & \multicolumn{4}{p{0.6\linewidth}}{\cellcolor[rgb]{0.8,1,0.9}{%
	\makecell[{{p{\linewidth}}}]{% 
	  \tt {\tiny [A$\rangle$GM]}  
	 > go to living room 
	  } 
	   } 
	   } 
	 & & \\ 
 

    \theutterance \stepcounter{utterance}  

    & & & \multicolumn{2}{p{0.3\linewidth}}{\cellcolor[rgb]{0.95,0.95,0.95}{%
	\makecell[{{p{\linewidth}}}]{% 
	  \tt {\tiny [GM$|$GM]}  
	 Round: 35 
	  } 
	   } 
	   } 
	 & & \\ 
 

    \theutterance \stepcounter{utterance}  

    & & & \multicolumn{2}{p{0.3\linewidth}}{\cellcolor[rgb]{0.95,0.95,0.95}{%
	\makecell[{{p{\linewidth}}}]{% 
	  \tt {\tiny [GM$|$GM]}  
	 {'phase': 'resolution', 'fail_type': 'no_exit_to', 'arg': {'failed_action_type': 'go', 'failed_precon_predicate': {'predicate_tuple': ['exit', 'broomcloset1', 'livingroom1'], 'fulfilled': False, 'precon_idx': 2}}, 'exploration_info': {'action_epistemic': True, 'action_pragmatic': True, 'effective_epistemic_gain_facts': [], 'effective_epistemic_gain_amount': 0, 'known_entities': [['at', 'pottedplant1', 'kitchen1'], ['at', 'cupboard1', 'kitchen1'], ['at', 'mop1', 'broomcloset1'], ['at', 'wardrobe1', 'bedroom1'], ['at', 'chair1', 'livingroom1'], ['at', 'book1', 'broomcloset1'], ['at', 'plate1', 'kitchen1'], ['at', 'couch1', 'livingroom1'], ['at', 'counter1', 'kitchen1'], ['at', 'broom1', 'broomcloset1'], ['at', 'bed1', 'bedroom1'], ['at', 'player1', 'broomcloset1'], ['at', 'pillow1', 'bedroom1'], ['at', 'sidetable1', 'livingroom1']], 'known_entities_ratio': 0.4827586206896552, 'visited_rooms': ['livingroom1', 'hallway1', 'broomcloset1', 'kitchen1', 'bedroom1'], 'visited_rooms_ratio': 0.8333333333333334, 'known_goal_entities': [['at', 'pottedplant1', 'kitchen1'], ['at', 'wardrobe1', 'bedroom1'], ['at', 'plate1', 'kitchen1'], ['at', 'counter1', 'kitchen1'], ['at', 'broom1', 'broomcloset1']], 'known_goal_entities_ratio': 0.8333333333333334}} 
	  } 
	   } 
	   } 
	 & & \\ 
 

    \theutterance \stepcounter{utterance}  

    & & & \multicolumn{2}{p{0.3\linewidth}}{\cellcolor[rgb]{0.95,0.95,0.95}{%
	\makecell[{{p{\linewidth}}}]{% 
	  \tt {\tiny [GM$|$GM]}  
	 {'goal_states_achieved': [['on', 'pottedplant1', 'counter1']], 'turn_goal_score': 0} 
	  } 
	   } 
	   } 
	 & & \\ 
 

    \theutterance \stepcounter{utterance}  

    & & \multicolumn{4}{p{0.6\linewidth}}{\cellcolor[rgb]{0.9,0.9,0.9}{%
	\makecell[{{p{\linewidth}}}]{% 
	  \tt {\tiny [A$\langle$GM]}  
	 You can't go to a living room from here.\\ \tt  
	  } 
	   } 
	   } 
	 & \\ 
 

    \theutterance \stepcounter{utterance}  

    & \multicolumn{4}{p{0.6\linewidth}}{\cellcolor[rgb]{0.8,1,0.9}{%
	\makecell[{{p{\linewidth}}}]{% 
	  \tt {\tiny [A$\rangle$GM]}  
	 > go to hallway 
	  } 
	   } 
	   } 
	 & & \\ 
 

    \theutterance \stepcounter{utterance}  

    & & & \multicolumn{2}{p{0.3\linewidth}}{\cellcolor[rgb]{0.95,0.95,0.95}{%
	\makecell[{{p{\linewidth}}}]{% 
	  \tt {\tiny [GM$|$GM]}  
	 Round: 36 
	  } 
	   } 
	   } 
	 & & \\ 
 

    \theutterance \stepcounter{utterance}  

    & & & \multicolumn{2}{p{0.3\linewidth}}{\cellcolor[rgb]{0.95,0.95,0.95}{%
	\makecell[{{p{\linewidth}}}]{% 
	  \tt {\tiny [GM$|$GM]}  
	 {'action_type': 'go', 'exploration_info': {'action_epistemic': True, 'action_pragmatic': True, 'effective_epistemic_gain_facts': [['at', 'player1', 'hallway1'], ['at', 'book1', 'hallway1'], ['at', 'broom1', 'hallway1']], 'effective_epistemic_gain_amount': 3, 'known_entities': [['at', 'pottedplant1', 'kitchen1'], ['at', 'cupboard1', 'kitchen1'], ['at', 'broom1', 'hallway1'], ['at', 'mop1', 'broomcloset1'], ['at', 'player1', 'hallway1'], ['at', 'wardrobe1', 'bedroom1'], ['at', 'chair1', 'livingroom1'], ['at', 'plate1', 'kitchen1'], ['at', 'couch1', 'livingroom1'], ['at', 'counter1', 'kitchen1'], ['at', 'book1', 'hallway1'], ['at', 'bed1', 'bedroom1'], ['at', 'pillow1', 'bedroom1'], ['at', 'sidetable1', 'livingroom1']], 'known_entities_ratio': 0.4827586206896552, 'visited_rooms': ['livingroom1', 'hallway1', 'broomcloset1', 'kitchen1', 'bedroom1'], 'visited_rooms_ratio': 0.8333333333333334, 'known_goal_entities': [['at', 'pottedplant1', 'kitchen1'], ['at', 'broom1', 'hallway1'], ['at', 'wardrobe1', 'bedroom1'], ['at', 'plate1', 'kitchen1'], ['at', 'counter1', 'kitchen1']], 'known_goal_entities_ratio': 0.8333333333333334}} 
	  } 
	   } 
	   } 
	 & & \\ 
 

    \theutterance \stepcounter{utterance}  

    & & & \multicolumn{2}{p{0.3\linewidth}}{\cellcolor[rgb]{0.95,0.95,0.95}{%
	\makecell[{{p{\linewidth}}}]{% 
	  \tt {\tiny [GM$|$GM]}  
	 {'goal_states_achieved': ['on(pottedplant1,counter1)'], 'turn_goal_score': 0} 
	  } 
	   } 
	   } 
	 & & \\ 
 

    \theutterance \stepcounter{utterance}  

    & & \multicolumn{4}{p{0.6\linewidth}}{\cellcolor[rgb]{0.9,0.9,0.9}{%
	\makecell[{{p{\linewidth}}}]{% 
	  \tt {\tiny [A$\langle$GM]}  
	 You are in a hallway now. There are passages to a broom closet and a living room here.\\ \tt  
	  } 
	   } 
	   } 
	 & \\ 
 

    \theutterance \stepcounter{utterance}  

    & \multicolumn{4}{p{0.6\linewidth}}{\cellcolor[rgb]{0.8,1,0.9}{%
	\makecell[{{p{\linewidth}}}]{% 
	  \tt {\tiny [A$\rangle$GM]}  
	 > go to living room 
	  } 
	   } 
	   } 
	 & & \\ 
 

    \theutterance \stepcounter{utterance}  

    & & & \multicolumn{2}{p{0.3\linewidth}}{\cellcolor[rgb]{0.95,0.95,0.95}{%
	\makecell[{{p{\linewidth}}}]{% 
	  \tt {\tiny [GM$|$GM]}  
	 Round: 37 
	  } 
	   } 
	   } 
	 & & \\ 
 

    \theutterance \stepcounter{utterance}  

    & & & \multicolumn{2}{p{0.3\linewidth}}{\cellcolor[rgb]{0.95,0.95,0.95}{%
	\makecell[{{p{\linewidth}}}]{% 
	  \tt {\tiny [GM$|$GM]}  
	 {'action_type': 'go', 'exploration_info': {'action_epistemic': True, 'action_pragmatic': True, 'effective_epistemic_gain_facts': [['at', 'player1', 'livingroom1'], ['at', 'book1', 'livingroom1'], ['at', 'broom1', 'livingroom1']], 'effective_epistemic_gain_amount': 3, 'known_entities': [['at', 'pottedplant1', 'kitchen1'], ['at', 'cupboard1', 'kitchen1'], ['at', 'mop1', 'broomcloset1'], ['at', 'wardrobe1', 'bedroom1'], ['at', 'book1', 'livingroom1'], ['at', 'chair1', 'livingroom1'], ['at', 'plate1', 'kitchen1'], ['at', 'couch1', 'livingroom1'], ['at', 'broom1', 'livingroom1'], ['at', 'counter1', 'kitchen1'], ['at', 'player1', 'livingroom1'], ['at', 'bed1', 'bedroom1'], ['at', 'pillow1', 'bedroom1'], ['at', 'sidetable1', 'livingroom1']], 'known_entities_ratio': 0.4827586206896552, 'visited_rooms': ['livingroom1', 'hallway1', 'broomcloset1', 'kitchen1', 'bedroom1'], 'visited_rooms_ratio': 0.8333333333333334, 'known_goal_entities': [['at', 'pottedplant1', 'kitchen1'], ['at', 'wardrobe1', 'bedroom1'], ['at', 'plate1', 'kitchen1'], ['at', 'counter1', 'kitchen1'], ['at', 'broom1', 'livingroom1']], 'known_goal_entities_ratio': 0.8333333333333334}} 
	  } 
	   } 
	   } 
	 & & \\ 
 

    \theutterance \stepcounter{utterance}  

    & & & \multicolumn{2}{p{0.3\linewidth}}{\cellcolor[rgb]{0.95,0.95,0.95}{%
	\makecell[{{p{\linewidth}}}]{% 
	  \tt {\tiny [GM$|$GM]}  
	 {'goal_states_achieved': ['on(pottedplant1,counter1)'], 'turn_goal_score': 0} 
	  } 
	   } 
	   } 
	 & & \\ 
 

    \theutterance \stepcounter{utterance}  

    & & \multicolumn{4}{p{0.6\linewidth}}{\cellcolor[rgb]{0.9,0.9,0.9}{%
	\makecell[{{p{\linewidth}}}]{% 
	  \tt {\tiny [A$\langle$GM]}  
	 You are in a living room now. There are a couch, a chair and a side table. There are passages to a bedroom, a hallway and a kitchen here.\\ \tt  
	  } 
	   } 
	   } 
	 & \\ 
 

    \theutterance \stepcounter{utterance}  

    & \multicolumn{4}{p{0.6\linewidth}}{\cellcolor[rgb]{0.8,1,0.9}{%
	\makecell[{{p{\linewidth}}}]{% 
	  \tt {\tiny [A$\rangle$GM]}  
	 > put broom on shelf 
	  } 
	   } 
	   } 
	 & & \\ 
 

    \theutterance \stepcounter{utterance}  

    & & & \multicolumn{2}{p{0.3\linewidth}}{\cellcolor[rgb]{0.95,0.95,0.95}{%
	\makecell[{{p{\linewidth}}}]{% 
	  \tt {\tiny [GM$|$GM]}  
	 Round: 38 
	  } 
	   } 
	   } 
	 & & \\ 
 

    \theutterance \stepcounter{utterance}  

    & & & \multicolumn{2}{p{0.3\linewidth}}{\cellcolor[rgb]{0.95,0.95,0.95}{%
	\makecell[{{p{\linewidth}}}]{% 
	  \tt {\tiny [GM$|$GM]}  
	 {'phase': 'resolution', 'fail_type': 'entity_not_accessible', 'arg': {'failed_action_type': 'put', 'failed_precon_predicate': {'predicate_tuple': ['at', 'shelf1', 'livingroom1'], 'fulfilled': False, 'precon_idx': 2}}, 'exploration_info': {'action_epistemic': False, 'action_pragmatic': True, 'effective_epistemic_gain_amount': 0, 'known_entities': [['at', 'pottedplant1', 'kitchen1'], ['at', 'cupboard1', 'kitchen1'], ['at', 'mop1', 'broomcloset1'], ['at', 'wardrobe1', 'bedroom1'], ['at', 'book1', 'livingroom1'], ['at', 'chair1', 'livingroom1'], ['at', 'plate1', 'kitchen1'], ['at', 'couch1', 'livingroom1'], ['at', 'broom1', 'livingroom1'], ['at', 'counter1', 'kitchen1'], ['at', 'player1', 'livingroom1'], ['at', 'bed1', 'bedroom1'], ['at', 'pillow1', 'bedroom1'], ['at', 'sidetable1', 'livingroom1']], 'known_entities_ratio': 0.4827586206896552, 'visited_rooms': ['livingroom1', 'hallway1', 'broomcloset1', 'kitchen1', 'bedroom1'], 'visited_rooms_ratio': 0.8333333333333334, 'known_goal_entities': [['at', 'pottedplant1', 'kitchen1'], ['at', 'wardrobe1', 'bedroom1'], ['at', 'plate1', 'kitchen1'], ['at', 'counter1', 'kitchen1'], ['at', 'broom1', 'livingroom1']], 'known_goal_entities_ratio': 0.8333333333333334}} 
	  } 
	   } 
	   } 
	 & & \\ 
 

    \theutterance \stepcounter{utterance}  

    & & & \multicolumn{2}{p{0.3\linewidth}}{\cellcolor[rgb]{0.95,0.95,0.95}{%
	\makecell[{{p{\linewidth}}}]{% 
	  \tt {\tiny [GM$|$GM]}  
	 {'goal_states_achieved': [['on', 'pottedplant1', 'counter1']], 'turn_goal_score': 0} 
	  } 
	   } 
	   } 
	 & & \\ 
 

    \theutterance \stepcounter{utterance}  

    & & \multicolumn{4}{p{0.6\linewidth}}{\cellcolor[rgb]{0.9,0.9,0.9}{%
	\makecell[{{p{\linewidth}}}]{% 
	  \tt {\tiny [A$\langle$GM]}  
	 You can't see a shelf here.\\ \tt  
	  } 
	   } 
	   } 
	 & \\ 
 

    \theutterance \stepcounter{utterance}  

    & \multicolumn{4}{p{0.6\linewidth}}{\cellcolor[rgb]{0.8,1,0.9}{%
	\makecell[{{p{\linewidth}}}]{% 
	  \tt {\tiny [A$\rangle$GM]}  
	 > go to bedroom 
	  } 
	   } 
	   } 
	 & & \\ 
 

    \theutterance \stepcounter{utterance}  

    & & & \multicolumn{2}{p{0.3\linewidth}}{\cellcolor[rgb]{0.95,0.95,0.95}{%
	\makecell[{{p{\linewidth}}}]{% 
	  \tt {\tiny [GM$|$GM]}  
	 Round: 39 
	  } 
	   } 
	   } 
	 & & \\ 
 

    \theutterance \stepcounter{utterance}  

    & & & \multicolumn{2}{p{0.3\linewidth}}{\cellcolor[rgb]{0.95,0.95,0.95}{%
	\makecell[{{p{\linewidth}}}]{% 
	  \tt {\tiny [GM$|$GM]}  
	 {'action_type': 'go', 'exploration_info': {'action_epistemic': True, 'action_pragmatic': True, 'effective_epistemic_gain_facts': [['at', 'player1', 'bedroom1'], ['at', 'book1', 'bedroom1'], ['at', 'broom1', 'bedroom1']], 'effective_epistemic_gain_amount': 3, 'known_entities': [['at', 'player1', 'bedroom1'], ['at', 'pottedplant1', 'kitchen1'], ['at', 'cupboard1', 'kitchen1'], ['at', 'mop1', 'broomcloset1'], ['at', 'broom1', 'bedroom1'], ['at', 'wardrobe1', 'bedroom1'], ['at', 'chair1', 'livingroom1'], ['at', 'plate1', 'kitchen1'], ['at', 'book1', 'bedroom1'], ['at', 'couch1', 'livingroom1'], ['at', 'counter1', 'kitchen1'], ['at', 'bed1', 'bedroom1'], ['at', 'pillow1', 'bedroom1'], ['at', 'sidetable1', 'livingroom1']], 'known_entities_ratio': 0.4827586206896552, 'visited_rooms': ['livingroom1', 'hallway1', 'broomcloset1', 'kitchen1', 'bedroom1'], 'visited_rooms_ratio': 0.8333333333333334, 'known_goal_entities': [['at', 'pottedplant1', 'kitchen1'], ['at', 'wardrobe1', 'bedroom1'], ['at', 'plate1', 'kitchen1'], ['at', 'counter1', 'kitchen1'], ['at', 'broom1', 'bedroom1']], 'known_goal_entities_ratio': 0.8333333333333334}} 
	  } 
	   } 
	   } 
	 & & \\ 
 

    \theutterance \stepcounter{utterance}  

    & & & \multicolumn{2}{p{0.3\linewidth}}{\cellcolor[rgb]{0.95,0.95,0.95}{%
	\makecell[{{p{\linewidth}}}]{% 
	  \tt {\tiny [GM$|$GM]}  
	 {'goal_states_achieved': ['on(pottedplant1,counter1)'], 'turn_goal_score': 0} 
	  } 
	   } 
	   } 
	 & & \\ 
 

    \theutterance \stepcounter{utterance}  

    & & \multicolumn{4}{p{0.6\linewidth}}{\cellcolor[rgb]{0.9,0.9,0.9}{%
	\makecell[{{p{\linewidth}}}]{% 
	  \tt {\tiny [A$\langle$GM]}  
	 You are in a bedroom now. There are a bed, a wardrobe and a pillow. The wardrobe is open. The pillow is in the wardrobe. There is a passage to a living room here.\\ \tt  
	  } 
	   } 
	   } 
	 & \\ 
 

    \theutterance \stepcounter{utterance}  

    & \multicolumn{4}{p{0.6\linewidth}}{\cellcolor[rgb]{0.8,1,0.9}{%
	\makecell[{{p{\linewidth}}}]{% 
	  \tt {\tiny [A$\rangle$GM]}  
	 > put broom on shelf 
	  } 
	   } 
	   } 
	 & & \\ 
 

    \theutterance \stepcounter{utterance}  

    & & & \multicolumn{2}{p{0.3\linewidth}}{\cellcolor[rgb]{0.95,0.95,0.95}{%
	\makecell[{{p{\linewidth}}}]{% 
	  \tt {\tiny [GM$|$GM]}  
	 Round: 40 
	  } 
	   } 
	   } 
	 & & \\ 
 

    \theutterance \stepcounter{utterance}  

    & & & \multicolumn{2}{p{0.3\linewidth}}{\cellcolor[rgb]{0.95,0.95,0.95}{%
	\makecell[{{p{\linewidth}}}]{% 
	  \tt {\tiny [GM$|$GM]}  
	 {'phase': 'resolution', 'fail_type': 'entity_not_accessible', 'arg': {'failed_action_type': 'put', 'failed_precon_predicate': {'predicate_tuple': ['at', 'shelf1', 'bedroom1'], 'fulfilled': False, 'precon_idx': 2}}, 'exploration_info': {'action_epistemic': False, 'action_pragmatic': True, 'effective_epistemic_gain_amount': 0, 'known_entities': [['at', 'player1', 'bedroom1'], ['at', 'pottedplant1', 'kitchen1'], ['at', 'cupboard1', 'kitchen1'], ['at', 'mop1', 'broomcloset1'], ['at', 'broom1', 'bedroom1'], ['at', 'wardrobe1', 'bedroom1'], ['at', 'chair1', 'livingroom1'], ['at', 'plate1', 'kitchen1'], ['at', 'book1', 'bedroom1'], ['at', 'couch1', 'livingroom1'], ['at', 'counter1', 'kitchen1'], ['at', 'bed1', 'bedroom1'], ['at', 'pillow1', 'bedroom1'], ['at', 'sidetable1', 'livingroom1']], 'known_entities_ratio': 0.4827586206896552, 'visited_rooms': ['livingroom1', 'hallway1', 'broomcloset1', 'kitchen1', 'bedroom1'], 'visited_rooms_ratio': 0.8333333333333334, 'known_goal_entities': [['at', 'pottedplant1', 'kitchen1'], ['at', 'wardrobe1', 'bedroom1'], ['at', 'plate1', 'kitchen1'], ['at', 'counter1', 'kitchen1'], ['at', 'broom1', 'bedroom1']], 'known_goal_entities_ratio': 0.8333333333333334}} 
	  } 
	   } 
	   } 
	 & & \\ 
 

    \theutterance \stepcounter{utterance}  

    & & & \multicolumn{2}{p{0.3\linewidth}}{\cellcolor[rgb]{0.95,0.95,0.95}{%
	\makecell[{{p{\linewidth}}}]{% 
	  \tt {\tiny [GM$|$GM]}  
	 {'goal_states_achieved': [['on', 'pottedplant1', 'counter1']], 'turn_goal_score': 0} 
	  } 
	   } 
	   } 
	 & & \\ 
 

    \theutterance \stepcounter{utterance}  

    & & \multicolumn{4}{p{0.6\linewidth}}{\cellcolor[rgb]{0.9,0.9,0.9}{%
	\makecell[{{p{\linewidth}}}]{% 
	  \tt {\tiny [A$\langle$GM]}  
	 You can't see a shelf here.\\ \tt  
	  } 
	   } 
	   } 
	 & \\ 
 

    \theutterance \stepcounter{utterance}  

    & \multicolumn{4}{p{0.6\linewidth}}{\cellcolor[rgb]{0.8,1,0.9}{%
	\makecell[{{p{\linewidth}}}]{% 
	  \tt {\tiny [A$\rangle$GM]}  
	 > put broom on shelf \\ \tt > put book on shelf 
	  } 
	   } 
	   } 
	 & & \\ 
 

    \theutterance \stepcounter{utterance}  

    & & & \multicolumn{2}{p{0.3\linewidth}}{\cellcolor[rgb]{0.95,0.95,0.95}{%
	\makecell[{{p{\linewidth}}}]{% 
	  \tt {\tiny [GM$|$GM]}  
	 Round: 41 
	  } 
	   } 
	   } 
	 & & \\ 
 

    \theutterance \stepcounter{utterance}  

    & & & \multicolumn{2}{p{0.3\linewidth}}{\cellcolor[rgb]{0.95,0.95,0.95}{%
	\makecell[{{p{\linewidth}}}]{% 
	  \tt {\tiny [GM$|$GM]}  
	 {'phase': 'resolution', 'fail_type': 'entity_not_accessible', 'arg': {'failed_action_type': 'put', 'failed_precon_predicate': {'predicate_tuple': ['at', 'shelf1', 'bedroom1'], 'fulfilled': False, 'precon_idx': 2}}, 'exploration_info': {'action_epistemic': False, 'action_pragmatic': True, 'effective_epistemic_gain_amount': 0, 'known_entities': [['at', 'player1', 'bedroom1'], ['at', 'pottedplant1', 'kitchen1'], ['at', 'cupboard1', 'kitchen1'], ['at', 'mop1', 'broomcloset1'], ['at', 'broom1', 'bedroom1'], ['at', 'wardrobe1', 'bedroom1'], ['at', 'chair1', 'livingroom1'], ['at', 'plate1', 'kitchen1'], ['at', 'book1', 'bedroom1'], ['at', 'couch1', 'livingroom1'], ['at', 'counter1', 'kitchen1'], ['at', 'bed1', 'bedroom1'], ['at', 'pillow1', 'bedroom1'], ['at', 'sidetable1', 'livingroom1']], 'known_entities_ratio': 0.4827586206896552, 'visited_rooms': ['livingroom1', 'hallway1', 'broomcloset1', 'kitchen1', 'bedroom1'], 'visited_rooms_ratio': 0.8333333333333334, 'known_goal_entities': [['at', 'pottedplant1', 'kitchen1'], ['at', 'wardrobe1', 'bedroom1'], ['at', 'plate1', 'kitchen1'], ['at', 'counter1', 'kitchen1'], ['at', 'broom1', 'bedroom1']], 'known_goal_entities_ratio': 0.8333333333333334}} 
	  } 
	   } 
	   } 
	 & & \\ 
 

    \theutterance \stepcounter{utterance}  

    & & & \multicolumn{2}{p{0.3\linewidth}}{\cellcolor[rgb]{0.95,0.95,0.95}{%
	\makecell[{{p{\linewidth}}}]{% 
	  \tt {\tiny [GM$|$GM]}  
	 {'goal_states_achieved': [['on', 'pottedplant1', 'counter1']], 'turn_goal_score': 0} 
	  } 
	   } 
	   } 
	 & & \\ 
 

    \theutterance \stepcounter{utterance}  

    & & \multicolumn{4}{p{0.6\linewidth}}{\cellcolor[rgb]{0.9,0.9,0.9}{%
	\makecell[{{p{\linewidth}}}]{% 
	  \tt {\tiny [A$\langle$GM]}  
	 You can't see a shelf here.\\ \tt  
	  } 
	   } 
	   } 
	 & \\ 
 

    \theutterance \stepcounter{utterance}  

    & \multicolumn{4}{p{0.6\linewidth}}{\cellcolor[rgb]{0.8,1,0.9}{%
	\makecell[{{p{\linewidth}}}]{% 
	  \tt {\tiny [A$\rangle$GM]}  
	 > go to living room 
	  } 
	   } 
	   } 
	 & & \\ 
 

    \theutterance \stepcounter{utterance}  

    & & & \multicolumn{2}{p{0.3\linewidth}}{\cellcolor[rgb]{0.95,0.95,0.95}{%
	\makecell[{{p{\linewidth}}}]{% 
	  \tt {\tiny [GM$|$GM]}  
	 Round: 42 
	  } 
	   } 
	   } 
	 & & \\ 
 

    \theutterance \stepcounter{utterance}  

    & & & \multicolumn{2}{p{0.3\linewidth}}{\cellcolor[rgb]{0.95,0.95,0.95}{%
	\makecell[{{p{\linewidth}}}]{% 
	  \tt {\tiny [GM$|$GM]}  
	 {'action_type': 'go', 'exploration_info': {'action_epistemic': True, 'action_pragmatic': True, 'effective_epistemic_gain_facts': [['at', 'player1', 'livingroom1'], ['at', 'book1', 'livingroom1'], ['at', 'broom1', 'livingroom1']], 'effective_epistemic_gain_amount': 3, 'known_entities': [['at', 'pottedplant1', 'kitchen1'], ['at', 'cupboard1', 'kitchen1'], ['at', 'mop1', 'broomcloset1'], ['at', 'wardrobe1', 'bedroom1'], ['at', 'book1', 'livingroom1'], ['at', 'chair1', 'livingroom1'], ['at', 'plate1', 'kitchen1'], ['at', 'couch1', 'livingroom1'], ['at', 'broom1', 'livingroom1'], ['at', 'counter1', 'kitchen1'], ['at', 'player1', 'livingroom1'], ['at', 'bed1', 'bedroom1'], ['at', 'pillow1', 'bedroom1'], ['at', 'sidetable1', 'livingroom1']], 'known_entities_ratio': 0.4827586206896552, 'visited_rooms': ['livingroom1', 'hallway1', 'broomcloset1', 'kitchen1', 'bedroom1'], 'visited_rooms_ratio': 0.8333333333333334, 'known_goal_entities': [['at', 'pottedplant1', 'kitchen1'], ['at', 'wardrobe1', 'bedroom1'], ['at', 'plate1', 'kitchen1'], ['at', 'counter1', 'kitchen1'], ['at', 'broom1', 'livingroom1']], 'known_goal_entities_ratio': 0.8333333333333334}} 
	  } 
	   } 
	   } 
	 & & \\ 
 

    \theutterance \stepcounter{utterance}  

    & & & \multicolumn{2}{p{0.3\linewidth}}{\cellcolor[rgb]{0.95,0.95,0.95}{%
	\makecell[{{p{\linewidth}}}]{% 
	  \tt {\tiny [GM$|$GM]}  
	 {'goal_states_achieved': ['on(pottedplant1,counter1)'], 'turn_goal_score': 0} 
	  } 
	   } 
	   } 
	 & & \\ 
 

    \theutterance \stepcounter{utterance}  

    & & \multicolumn{4}{p{0.6\linewidth}}{\cellcolor[rgb]{0.9,0.9,0.9}{%
	\makecell[{{p{\linewidth}}}]{% 
	  \tt {\tiny [A$\langle$GM]}  
	 You are in a living room now. There are a couch, a chair and a side table. There are passages to a bedroom, a hallway and a kitchen here.\\ \tt  
	  } 
	   } 
	   } 
	 & \\ 
 

    \theutterance \stepcounter{utterance}  

    & \multicolumn{4}{p{0.6\linewidth}}{\cellcolor[rgb]{0.8,1,0.9}{%
	\makecell[{{p{\linewidth}}}]{% 
	  \tt {\tiny [A$\rangle$GM]}  
	 > put broom on shelf 
	  } 
	   } 
	   } 
	 & & \\ 
 

    \theutterance \stepcounter{utterance}  

    & & & \multicolumn{2}{p{0.3\linewidth}}{\cellcolor[rgb]{0.95,0.95,0.95}{%
	\makecell[{{p{\linewidth}}}]{% 
	  \tt {\tiny [GM$|$GM]}  
	 Round: 43 
	  } 
	   } 
	   } 
	 & & \\ 
 

    \theutterance \stepcounter{utterance}  

    & & & \multicolumn{2}{p{0.3\linewidth}}{\cellcolor[rgb]{0.95,0.95,0.95}{%
	\makecell[{{p{\linewidth}}}]{% 
	  \tt {\tiny [GM$|$GM]}  
	 {'phase': 'resolution', 'fail_type': 'entity_not_accessible', 'arg': {'failed_action_type': 'put', 'failed_precon_predicate': {'predicate_tuple': ['at', 'shelf1', 'livingroom1'], 'fulfilled': False, 'precon_idx': 2}}, 'exploration_info': {'action_epistemic': False, 'action_pragmatic': True, 'effective_epistemic_gain_amount': 0, 'known_entities': [['at', 'pottedplant1', 'kitchen1'], ['at', 'cupboard1', 'kitchen1'], ['at', 'mop1', 'broomcloset1'], ['at', 'wardrobe1', 'bedroom1'], ['at', 'book1', 'livingroom1'], ['at', 'chair1', 'livingroom1'], ['at', 'plate1', 'kitchen1'], ['at', 'couch1', 'livingroom1'], ['at', 'broom1', 'livingroom1'], ['at', 'counter1', 'kitchen1'], ['at', 'player1', 'livingroom1'], ['at', 'bed1', 'bedroom1'], ['at', 'pillow1', 'bedroom1'], ['at', 'sidetable1', 'livingroom1']], 'known_entities_ratio': 0.4827586206896552, 'visited_rooms': ['livingroom1', 'hallway1', 'broomcloset1', 'kitchen1', 'bedroom1'], 'visited_rooms_ratio': 0.8333333333333334, 'known_goal_entities': [['at', 'pottedplant1', 'kitchen1'], ['at', 'wardrobe1', 'bedroom1'], ['at', 'plate1', 'kitchen1'], ['at', 'counter1', 'kitchen1'], ['at', 'broom1', 'livingroom1']], 'known_goal_entities_ratio': 0.8333333333333334}} 
	  } 
	   } 
	   } 
	 & & \\ 
 

    \theutterance \stepcounter{utterance}  

    & & & \multicolumn{2}{p{0.3\linewidth}}{\cellcolor[rgb]{0.95,0.95,0.95}{%
	\makecell[{{p{\linewidth}}}]{% 
	  \tt {\tiny [GM$|$GM]}  
	 {'goal_states_achieved': [['on', 'pottedplant1', 'counter1']], 'turn_goal_score': 0} 
	  } 
	   } 
	   } 
	 & & \\ 
 

    \theutterance \stepcounter{utterance}  

    & & \multicolumn{4}{p{0.6\linewidth}}{\cellcolor[rgb]{0.9,0.9,0.9}{%
	\makecell[{{p{\linewidth}}}]{% 
	  \tt {\tiny [A$\langle$GM]}  
	 You can't see a shelf here.\\ \tt  
	  } 
	   } 
	   } 
	 & \\ 
 

    \theutterance \stepcounter{utterance}  

    & \multicolumn{4}{p{0.6\linewidth}}{\cellcolor[rgb]{0.8,1,0.9}{%
	\makecell[{{p{\linewidth}}}]{% 
	  \tt {\tiny [A$\rangle$GM]}  
	 > go to hallway 
	  } 
	   } 
	   } 
	 & & \\ 
 

    \theutterance \stepcounter{utterance}  

    & & & \multicolumn{2}{p{0.3\linewidth}}{\cellcolor[rgb]{0.95,0.95,0.95}{%
	\makecell[{{p{\linewidth}}}]{% 
	  \tt {\tiny [GM$|$GM]}  
	 Round: 44 
	  } 
	   } 
	   } 
	 & & \\ 
 

    \theutterance \stepcounter{utterance}  

    & & & \multicolumn{2}{p{0.3\linewidth}}{\cellcolor[rgb]{0.95,0.95,0.95}{%
	\makecell[{{p{\linewidth}}}]{% 
	  \tt {\tiny [GM$|$GM]}  
	 {'action_type': 'go', 'exploration_info': {'action_epistemic': True, 'action_pragmatic': True, 'effective_epistemic_gain_facts': [['at', 'player1', 'hallway1'], ['at', 'book1', 'hallway1'], ['at', 'broom1', 'hallway1']], 'effective_epistemic_gain_amount': 3, 'known_entities': [['at', 'pottedplant1', 'kitchen1'], ['at', 'cupboard1', 'kitchen1'], ['at', 'broom1', 'hallway1'], ['at', 'mop1', 'broomcloset1'], ['at', 'player1', 'hallway1'], ['at', 'wardrobe1', 'bedroom1'], ['at', 'chair1', 'livingroom1'], ['at', 'plate1', 'kitchen1'], ['at', 'couch1', 'livingroom1'], ['at', 'counter1', 'kitchen1'], ['at', 'book1', 'hallway1'], ['at', 'bed1', 'bedroom1'], ['at', 'pillow1', 'bedroom1'], ['at', 'sidetable1', 'livingroom1']], 'known_entities_ratio': 0.4827586206896552, 'visited_rooms': ['livingroom1', 'hallway1', 'broomcloset1', 'kitchen1', 'bedroom1'], 'visited_rooms_ratio': 0.8333333333333334, 'known_goal_entities': [['at', 'pottedplant1', 'kitchen1'], ['at', 'broom1', 'hallway1'], ['at', 'wardrobe1', 'bedroom1'], ['at', 'plate1', 'kitchen1'], ['at', 'counter1', 'kitchen1']], 'known_goal_entities_ratio': 0.8333333333333334}} 
	  } 
	   } 
	   } 
	 & & \\ 
 

    \theutterance \stepcounter{utterance}  

    & & & \multicolumn{2}{p{0.3\linewidth}}{\cellcolor[rgb]{0.95,0.95,0.95}{%
	\makecell[{{p{\linewidth}}}]{% 
	  \tt {\tiny [GM$|$GM]}  
	 {'goal_states_achieved': ['on(pottedplant1,counter1)'], 'turn_goal_score': 0} 
	  } 
	   } 
	   } 
	 & & \\ 
 

    \theutterance \stepcounter{utterance}  

    & & \multicolumn{4}{p{0.6\linewidth}}{\cellcolor[rgb]{0.9,0.9,0.9}{%
	\makecell[{{p{\linewidth}}}]{% 
	  \tt {\tiny [A$\langle$GM]}  
	 You are in a hallway now. There are passages to a broom closet and a living room here.\\ \tt  
	  } 
	   } 
	   } 
	 & \\ 
 

    \theutterance \stepcounter{utterance}  

    & \multicolumn{4}{p{0.6\linewidth}}{\cellcolor[rgb]{0.8,1,0.9}{%
	\makecell[{{p{\linewidth}}}]{% 
	  \tt {\tiny [A$\rangle$GM]}  
	 > go to living room 
	  } 
	   } 
	   } 
	 & & \\ 
 

    \theutterance \stepcounter{utterance}  

    & & & \multicolumn{2}{p{0.3\linewidth}}{\cellcolor[rgb]{0.95,0.95,0.95}{%
	\makecell[{{p{\linewidth}}}]{% 
	  \tt {\tiny [GM$|$GM]}  
	 Round: 45 
	  } 
	   } 
	   } 
	 & & \\ 
 

    \theutterance \stepcounter{utterance}  

    & & & \multicolumn{2}{p{0.3\linewidth}}{\cellcolor[rgb]{0.95,0.95,0.95}{%
	\makecell[{{p{\linewidth}}}]{% 
	  \tt {\tiny [GM$|$GM]}  
	 {'action_type': 'go', 'exploration_info': {'action_epistemic': True, 'action_pragmatic': True, 'effective_epistemic_gain_facts': [['at', 'player1', 'livingroom1'], ['at', 'book1', 'livingroom1'], ['at', 'broom1', 'livingroom1']], 'effective_epistemic_gain_amount': 3, 'known_entities': [['at', 'pottedplant1', 'kitchen1'], ['at', 'cupboard1', 'kitchen1'], ['at', 'mop1', 'broomcloset1'], ['at', 'wardrobe1', 'bedroom1'], ['at', 'book1', 'livingroom1'], ['at', 'chair1', 'livingroom1'], ['at', 'plate1', 'kitchen1'], ['at', 'couch1', 'livingroom1'], ['at', 'broom1', 'livingroom1'], ['at', 'counter1', 'kitchen1'], ['at', 'player1', 'livingroom1'], ['at', 'bed1', 'bedroom1'], ['at', 'pillow1', 'bedroom1'], ['at', 'sidetable1', 'livingroom1']], 'known_entities_ratio': 0.4827586206896552, 'visited_rooms': ['livingroom1', 'hallway1', 'broomcloset1', 'kitchen1', 'bedroom1'], 'visited_rooms_ratio': 0.8333333333333334, 'known_goal_entities': [['at', 'pottedplant1', 'kitchen1'], ['at', 'wardrobe1', 'bedroom1'], ['at', 'plate1', 'kitchen1'], ['at', 'counter1', 'kitchen1'], ['at', 'broom1', 'livingroom1']], 'known_goal_entities_ratio': 0.8333333333333334}} 
	  } 
	   } 
	   } 
	 & & \\ 
 

    \theutterance \stepcounter{utterance}  

    & & & \multicolumn{2}{p{0.3\linewidth}}{\cellcolor[rgb]{0.95,0.95,0.95}{%
	\makecell[{{p{\linewidth}}}]{% 
	  \tt {\tiny [GM$|$GM]}  
	 {'goal_states_achieved': ['on(pottedplant1,counter1)'], 'turn_goal_score': 0} 
	  } 
	   } 
	   } 
	 & & \\ 
 

    \theutterance \stepcounter{utterance}  

    & & \multicolumn{4}{p{0.6\linewidth}}{\cellcolor[rgb]{0.9,0.9,0.9}{%
	\makecell[{{p{\linewidth}}}]{% 
	  \tt {\tiny [A$\langle$GM]}  
	 You are in a living room now. There are a couch, a chair and a side table. There are passages to a bedroom, a hallway and a kitchen here.\\ \tt  
	  } 
	   } 
	   } 
	 & \\ 
 

    \theutterance \stepcounter{utterance}  

    & \multicolumn{4}{p{0.6\linewidth}}{\cellcolor[rgb]{0.8,1,0.9}{%
	\makecell[{{p{\linewidth}}}]{% 
	  \tt {\tiny [A$\rangle$GM]}  
	 > put broom on shelf \\ \tt > put potted plant on counter 
	  } 
	   } 
	   } 
	 & & \\ 
 

    \theutterance \stepcounter{utterance}  

    & & & \multicolumn{2}{p{0.3\linewidth}}{\cellcolor[rgb]{0.95,0.95,0.95}{%
	\makecell[{{p{\linewidth}}}]{% 
	  \tt {\tiny [GM$|$GM]}  
	 Round: 46 
	  } 
	   } 
	   } 
	 & & \\ 
 

    \theutterance \stepcounter{utterance}  

    & & & \multicolumn{2}{p{0.3\linewidth}}{\cellcolor[rgb]{0.95,0.95,0.95}{%
	\makecell[{{p{\linewidth}}}]{% 
	  \tt {\tiny [GM$|$GM]}  
	 {'phase': 'resolution', 'fail_type': 'entity_not_accessible', 'arg': {'failed_action_type': 'put', 'failed_precon_predicate': {'predicate_tuple': ['at', 'shelf1', 'livingroom1'], 'fulfilled': False, 'precon_idx': 2}}, 'exploration_info': {'action_epistemic': False, 'action_pragmatic': True, 'effective_epistemic_gain_amount': 0, 'known_entities': [['at', 'pottedplant1', 'kitchen1'], ['at', 'cupboard1', 'kitchen1'], ['at', 'mop1', 'broomcloset1'], ['at', 'wardrobe1', 'bedroom1'], ['at', 'book1', 'livingroom1'], ['at', 'chair1', 'livingroom1'], ['at', 'plate1', 'kitchen1'], ['at', 'couch1', 'livingroom1'], ['at', 'broom1', 'livingroom1'], ['at', 'counter1', 'kitchen1'], ['at', 'player1', 'livingroom1'], ['at', 'bed1', 'bedroom1'], ['at', 'pillow1', 'bedroom1'], ['at', 'sidetable1', 'livingroom1']], 'known_entities_ratio': 0.4827586206896552, 'visited_rooms': ['livingroom1', 'hallway1', 'broomcloset1', 'kitchen1', 'bedroom1'], 'visited_rooms_ratio': 0.8333333333333334, 'known_goal_entities': [['at', 'pottedplant1', 'kitchen1'], ['at', 'wardrobe1', 'bedroom1'], ['at', 'plate1', 'kitchen1'], ['at', 'counter1', 'kitchen1'], ['at', 'broom1', 'livingroom1']], 'known_goal_entities_ratio': 0.8333333333333334}} 
	  } 
	   } 
	   } 
	 & & \\ 
 

    \theutterance \stepcounter{utterance}  

    & & & \multicolumn{2}{p{0.3\linewidth}}{\cellcolor[rgb]{0.95,0.95,0.95}{%
	\makecell[{{p{\linewidth}}}]{% 
	  \tt {\tiny [GM$|$GM]}  
	 {'goal_states_achieved': [['on', 'pottedplant1', 'counter1']], 'turn_goal_score': 0} 
	  } 
	   } 
	   } 
	 & & \\ 
 

    \theutterance \stepcounter{utterance}  

    & & \multicolumn{4}{p{0.6\linewidth}}{\cellcolor[rgb]{0.9,0.9,0.9}{%
	\makecell[{{p{\linewidth}}}]{% 
	  \tt {\tiny [A$\langle$GM]}  
	 You can't see a shelf here.\\ \tt  
	  } 
	   } 
	   } 
	 & \\ 
 

    \theutterance \stepcounter{utterance}  

    & \multicolumn{4}{p{0.6\linewidth}}{\cellcolor[rgb]{0.8,1,0.9}{%
	\makecell[{{p{\linewidth}}}]{% 
	  \tt {\tiny [A$\rangle$GM]}  
	 > go to kitchen 
	  } 
	   } 
	   } 
	 & & \\ 
 

    \theutterance \stepcounter{utterance}  

    & & & \multicolumn{2}{p{0.3\linewidth}}{\cellcolor[rgb]{0.95,0.95,0.95}{%
	\makecell[{{p{\linewidth}}}]{% 
	  \tt {\tiny [GM$|$GM]}  
	 Round: 47 
	  } 
	   } 
	   } 
	 & & \\ 
 

    \theutterance \stepcounter{utterance}  

    & & & \multicolumn{2}{p{0.3\linewidth}}{\cellcolor[rgb]{0.95,0.95,0.95}{%
	\makecell[{{p{\linewidth}}}]{% 
	  \tt {\tiny [GM$|$GM]}  
	 {'action_type': 'go', 'exploration_info': {'action_epistemic': True, 'action_pragmatic': True, 'effective_epistemic_gain_facts': [['at', 'broom1', 'kitchen1'], ['at', 'player1', 'kitchen1'], ['at', 'book1', 'kitchen1']], 'effective_epistemic_gain_amount': 3, 'known_entities': [['at', 'pottedplant1', 'kitchen1'], ['at', 'cupboard1', 'kitchen1'], ['at', 'mop1', 'broomcloset1'], ['at', 'wardrobe1', 'bedroom1'], ['at', 'broom1', 'kitchen1'], ['at', 'chair1', 'livingroom1'], ['at', 'player1', 'kitchen1'], ['at', 'plate1', 'kitchen1'], ['at', 'couch1', 'livingroom1'], ['at', 'counter1', 'kitchen1'], ['at', 'bed1', 'bedroom1'], ['at', 'pillow1', 'bedroom1'], ['at', 'sidetable1', 'livingroom1'], ['at', 'book1', 'kitchen1']], 'known_entities_ratio': 0.4827586206896552, 'visited_rooms': ['livingroom1', 'hallway1', 'broomcloset1', 'kitchen1', 'bedroom1'], 'visited_rooms_ratio': 0.8333333333333334, 'known_goal_entities': [['at', 'pottedplant1', 'kitchen1'], ['at', 'wardrobe1', 'bedroom1'], ['at', 'broom1', 'kitchen1'], ['at', 'plate1', 'kitchen1'], ['at', 'counter1', 'kitchen1']], 'known_goal_entities_ratio': 0.8333333333333334}} 
	  } 
	   } 
	   } 
	 & & \\ 
 

    \theutterance \stepcounter{utterance}  

    & & & \multicolumn{2}{p{0.3\linewidth}}{\cellcolor[rgb]{0.95,0.95,0.95}{%
	\makecell[{{p{\linewidth}}}]{% 
	  \tt {\tiny [GM$|$GM]}  
	 {'goal_states_achieved': ['on(pottedplant1,counter1)'], 'turn_goal_score': 0} 
	  } 
	   } 
	   } 
	 & & \\ 
 

    \theutterance \stepcounter{utterance}  

    & & \multicolumn{4}{p{0.6\linewidth}}{\cellcolor[rgb]{0.9,0.9,0.9}{%
	\makecell[{{p{\linewidth}}}]{% 
	  \tt {\tiny [A$\langle$GM]}  
	 You are in a kitchen now. There are a potted plant, a cupboard, a plate and a counter. The potted plant is on the counter. The cupboard is open. The plate is on the counter. There are passages to a living room and a pantry here.\\ \tt  
	  } 
	   } 
	   } 
	 & \\ 
 

    \theutterance \stepcounter{utterance}  

    & \multicolumn{4}{p{0.6\linewidth}}{\cellcolor[rgb]{0.8,1,0.9}{%
	\makecell[{{p{\linewidth}}}]{% 
	  \tt {\tiny [A$\rangle$GM]}  
	 > put broom on shelf \\ \tt > put plate in wardrobe 
	  } 
	   } 
	   } 
	 & & \\ 
 

    \theutterance \stepcounter{utterance}  

    & & & \multicolumn{2}{p{0.3\linewidth}}{\cellcolor[rgb]{0.95,0.95,0.95}{%
	\makecell[{{p{\linewidth}}}]{% 
	  \tt {\tiny [GM$|$GM]}  
	 Round: 48 
	  } 
	   } 
	   } 
	 & & \\ 
 

    \theutterance \stepcounter{utterance}  

    & & & \multicolumn{2}{p{0.3\linewidth}}{\cellcolor[rgb]{0.95,0.95,0.95}{%
	\makecell[{{p{\linewidth}}}]{% 
	  \tt {\tiny [GM$|$GM]}  
	 {'phase': 'resolution', 'fail_type': 'entity_not_accessible', 'arg': {'failed_action_type': 'put', 'failed_precon_predicate': {'predicate_tuple': ['at', 'shelf1', 'kitchen1'], 'fulfilled': False, 'precon_idx': 2}}, 'exploration_info': {'action_epistemic': False, 'action_pragmatic': True, 'effective_epistemic_gain_amount': 0, 'known_entities': [['at', 'pottedplant1', 'kitchen1'], ['at', 'cupboard1', 'kitchen1'], ['at', 'mop1', 'broomcloset1'], ['at', 'wardrobe1', 'bedroom1'], ['at', 'broom1', 'kitchen1'], ['at', 'chair1', 'livingroom1'], ['at', 'player1', 'kitchen1'], ['at', 'plate1', 'kitchen1'], ['at', 'couch1', 'livingroom1'], ['at', 'counter1', 'kitchen1'], ['at', 'bed1', 'bedroom1'], ['at', 'pillow1', 'bedroom1'], ['at', 'sidetable1', 'livingroom1'], ['at', 'book1', 'kitchen1']], 'known_entities_ratio': 0.4827586206896552, 'visited_rooms': ['livingroom1', 'hallway1', 'broomcloset1', 'kitchen1', 'bedroom1'], 'visited_rooms_ratio': 0.8333333333333334, 'known_goal_entities': [['at', 'pottedplant1', 'kitchen1'], ['at', 'wardrobe1', 'bedroom1'], ['at', 'broom1', 'kitchen1'], ['at', 'plate1', 'kitchen1'], ['at', 'counter1', 'kitchen1']], 'known_goal_entities_ratio': 0.8333333333333334}} 
	  } 
	   } 
	   } 
	 & & \\ 
 

    \theutterance \stepcounter{utterance}  

    & & & \multicolumn{2}{p{0.3\linewidth}}{\cellcolor[rgb]{0.95,0.95,0.95}{%
	\makecell[{{p{\linewidth}}}]{% 
	  \tt {\tiny [GM$|$GM]}  
	 {'goal_states_achieved': [['on', 'pottedplant1', 'counter1']], 'turn_goal_score': 0} 
	  } 
	   } 
	   } 
	 & & \\ 
 

    \theutterance \stepcounter{utterance}  

    & & \multicolumn{4}{p{0.6\linewidth}}{\cellcolor[rgb]{0.9,0.9,0.9}{%
	\makecell[{{p{\linewidth}}}]{% 
	  \tt {\tiny [A$\langle$GM]}  
	 You can't see a shelf here.\\ \tt  
	  } 
	   } 
	   } 
	 & \\ 
 

    \theutterance \stepcounter{utterance}  

    & \multicolumn{4}{p{0.6\linewidth}}{\cellcolor[rgb]{0.8,1,0.9}{%
	\makecell[{{p{\linewidth}}}]{% 
	  \tt {\tiny [A$\rangle$GM]}  
	 > go to living room 
	  } 
	   } 
	   } 
	 & & \\ 
 

    \theutterance \stepcounter{utterance}  

    & & & \multicolumn{2}{p{0.3\linewidth}}{\cellcolor[rgb]{0.95,0.95,0.95}{%
	\makecell[{{p{\linewidth}}}]{% 
	  \tt {\tiny [GM$|$GM]}  
	 Round: 49 
	  } 
	   } 
	   } 
	 & & \\ 
 

    \theutterance \stepcounter{utterance}  

    & & & \multicolumn{2}{p{0.3\linewidth}}{\cellcolor[rgb]{0.95,0.95,0.95}{%
	\makecell[{{p{\linewidth}}}]{% 
	  \tt {\tiny [GM$|$GM]}  
	 {'action_type': 'go', 'exploration_info': {'action_epistemic': True, 'action_pragmatic': True, 'effective_epistemic_gain_facts': [['at', 'player1', 'livingroom1'], ['at', 'book1', 'livingroom1'], ['at', 'broom1', 'livingroom1']], 'effective_epistemic_gain_amount': 3, 'known_entities': [['at', 'pottedplant1', 'kitchen1'], ['at', 'cupboard1', 'kitchen1'], ['at', 'mop1', 'broomcloset1'], ['at', 'wardrobe1', 'bedroom1'], ['at', 'book1', 'livingroom1'], ['at', 'chair1', 'livingroom1'], ['at', 'plate1', 'kitchen1'], ['at', 'couch1', 'livingroom1'], ['at', 'broom1', 'livingroom1'], ['at', 'counter1', 'kitchen1'], ['at', 'player1', 'livingroom1'], ['at', 'bed1', 'bedroom1'], ['at', 'pillow1', 'bedroom1'], ['at', 'sidetable1', 'livingroom1']], 'known_entities_ratio': 0.4827586206896552, 'visited_rooms': ['livingroom1', 'hallway1', 'broomcloset1', 'kitchen1', 'bedroom1'], 'visited_rooms_ratio': 0.8333333333333334, 'known_goal_entities': [['at', 'pottedplant1', 'kitchen1'], ['at', 'wardrobe1', 'bedroom1'], ['at', 'plate1', 'kitchen1'], ['at', 'counter1', 'kitchen1'], ['at', 'broom1', 'livingroom1']], 'known_goal_entities_ratio': 0.8333333333333334}} 
	  } 
	   } 
	   } 
	 & & \\ 
 

    \theutterance \stepcounter{utterance}  

    & & & \multicolumn{2}{p{0.3\linewidth}}{\cellcolor[rgb]{0.95,0.95,0.95}{%
	\makecell[{{p{\linewidth}}}]{% 
	  \tt {\tiny [GM$|$GM]}  
	 {'goal_states_achieved': ['on(pottedplant1,counter1)'], 'turn_goal_score': 0} 
	  } 
	   } 
	   } 
	 & & \\ 
 

    \theutterance \stepcounter{utterance}  

    & & & \multicolumn{2}{p{0.3\linewidth}}{\cellcolor[rgb]{0.95,0.95,0.95}{%
	\makecell[{{p{\linewidth}}}]{% 
	  \tt {\tiny [GM$|$GM]}  
	 Turn limit 50 reached, end episode. 
	  } 
	   } 
	   } 
	 & & \\ 
 

    \theutterance \stepcounter{utterance}  

    & & & \multicolumn{2}{p{0.3\linewidth}}{\cellcolor[rgb]{0.95,0.95,0.95}{%
	\makecell[{{p{\linewidth}}}]{% 
	  \tt {\tiny [GM$|$GM]}  
	 {'goal_states_achieved': ['on(pottedplant1,counter1)'], 'game_successfully_finished': False} 
	  } 
	   } 
	   } 
	 & & \\ 
 

\end{supertabular}
}

\end{document}

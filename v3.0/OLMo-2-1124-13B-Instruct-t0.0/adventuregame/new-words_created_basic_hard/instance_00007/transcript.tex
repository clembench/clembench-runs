
\documentclass{article}
\usepackage{colortbl}
\usepackage{makecell}
\usepackage{multirow}
\usepackage{supertabular}

\begin{document}

\newcounter{utterance}

\twocolumn

{ \footnotesize  \setcounter{utterance}{1}
\setlength{\tabcolsep}{0pt}
\begin{supertabular}{c@{$\;$}|p{.15\linewidth}@{}p{.15\linewidth}p{.15\linewidth}p{.15\linewidth}p{.15\linewidth}p{.15\linewidth}}

    \# & $\;$A & \multicolumn{4}{c}{Game Master} & $\;\:$B\\
    \hline 

    \theutterance \stepcounter{utterance}  

    & & \multicolumn{4}{p{0.6\linewidth}}{\cellcolor[rgb]{0.9,0.9,0.9}{%
	\makecell[{{p{\linewidth}}}]{% 
	  \tt {\tiny [A$\langle$GM]}  
	 You are playing a text adventure game. I will describe what you can perceive in the game. You write the single action you want to take in the game starting with >. Only reply with an action.\\ \tt For example:\\ \tt > open cupboard\\ \tt \\ \tt In addition to common actions, you can onser, mellum, clogery, ellous, ially and upbes.\\ \tt \\ \tt Your goal for this game is: Make the hetle anird, the ioness opige and the rotont reforian.\\ \tt \\ \tt Once you have achieved your goal, write "> done" to end the game.\\ \tt \\ \tt You are in a atival now. There are a rotont, a herper, a fervan, a ioness, a equadish and a thium. There is a passage to a routiced here. 
	  } 
	   } 
	   } 
	 & \\ 
 

    \theutterance \stepcounter{utterance}  

    & \multicolumn{4}{p{0.6\linewidth}}{\cellcolor[rgb]{0.8,1,0.9}{%
	\makecell[{{p{\linewidth}}}]{% 
	  \tt {\tiny [A$\rangle$GM]}  
	 > take rotont and fervan\\ \tt \\ \tt > use fervan to cook rotont\\ \tt \\ \tt > serve cooked rotont to ioness\\ \tt \\ \tt > interact with herper to make it anird\\ \tt \\ \tt > if herper becomes anird, examine equadish and thium for potential use in reforian\\ \tt \\ \tt > if any items can contribute to reforian, use them appropriately\\ \tt \\ \tt > once all conditions are met, write "> done" 
	  } 
	   } 
	   } 
	 & & \\ 
 

    \theutterance \stepcounter{utterance}  

    & & & \multicolumn{2}{p{0.3\linewidth}}{\cellcolor[rgb]{0.95,0.95,0.95}{%
	\makecell[{{p{\linewidth}}}]{% 
	  \tt {\tiny [GM$|$GM]}  
	 Round: 0 
	  } 
	   } 
	   } 
	 & & \\ 
 

    \theutterance \stepcounter{utterance}  

    & & & \multicolumn{2}{p{0.3\linewidth}}{\cellcolor[rgb]{0.95,0.95,0.95}{%
	\makecell[{{p{\linewidth}}}]{% 
	  \tt {\tiny [GM$|$GM]}  
	 {'phase': 'parsing', 'fail_type': 'undefined_action_verb', 'arg': 'take', 'exploration_info': {'action_epistemic': False, 'action_pragmatic': False, 'effective_epistemic_gain_amount': 0, 'known_entities': [['at', 'rotont1', 'atival1'], ['at', 'thium1', 'atival1'], ['at', 'herper1', 'atival1'], ['at', 'ioness1', 'atival1'], ['at', 'equadish1', 'atival1'], ['at', 'fervan1', 'atival1'], ['at', 'player1', 'atival1']], 'known_entities_ratio': 0.4666666666666667, 'visited_rooms': ['atival1'], 'visited_rooms_ratio': 0.25, 'known_goal_entities': [['at', 'rotont1', 'atival1'], ['at', 'ioness1', 'atival1']], 'known_goal_entities_ratio': 0.6666666666666666}} 
	  } 
	   } 
	   } 
	 & & \\ 
 

    \theutterance \stepcounter{utterance}  

    & & & \multicolumn{2}{p{0.3\linewidth}}{\cellcolor[rgb]{0.95,0.95,0.95}{%
	\makecell[{{p{\linewidth}}}]{% 
	  \tt {\tiny [GM$|$GM]}  
	 {'goal_states_achieved': [], 'turn_goal_score': 0} 
	  } 
	   } 
	   } 
	 & & \\ 
 

    \theutterance \stepcounter{utterance}  

    & & \multicolumn{4}{p{0.6\linewidth}}{\cellcolor[rgb]{0.9,0.9,0.9}{%
	\makecell[{{p{\linewidth}}}]{% 
	  \tt {\tiny [A$\langle$GM]}  
	 I don't know how to interpret this 'take' action.\\ \tt  
	  } 
	   } 
	   } 
	 & \\ 
 

    \theutterance \stepcounter{utterance}  

    & \multicolumn{4}{p{0.6\linewidth}}{\cellcolor[rgb]{0.8,1,0.9}{%
	\makecell[{{p{\linewidth}}}]{% 
	  \tt {\tiny [A$\rangle$GM]}  
	 > take rotont and fervan\\ \tt \\ \tt > use fervan to cook rotont\\ \tt \\ \tt > serve cooked rotont to ioness\\ \tt \\ \tt > try to calm the herper to make it anird\\ \tt \\ \tt > if successful, examine equadish and thium for their potential use in reforian\\ \tt \\ \tt > if any items can contribute to reforian, use them appropriately\\ \tt \\ \tt > once all conditions are met, write "> done" 
	  } 
	   } 
	   } 
	 & & \\ 
 

    \theutterance \stepcounter{utterance}  

    & & & \multicolumn{2}{p{0.3\linewidth}}{\cellcolor[rgb]{0.95,0.95,0.95}{%
	\makecell[{{p{\linewidth}}}]{% 
	  \tt {\tiny [GM$|$GM]}  
	 Round: 1 
	  } 
	   } 
	   } 
	 & & \\ 
 

    \theutterance \stepcounter{utterance}  

    & & & \multicolumn{2}{p{0.3\linewidth}}{\cellcolor[rgb]{0.95,0.95,0.95}{%
	\makecell[{{p{\linewidth}}}]{% 
	  \tt {\tiny [GM$|$GM]}  
	 {'phase': 'parsing', 'fail_type': 'undefined_action_verb', 'arg': 'take', 'exploration_info': {'action_epistemic': False, 'action_pragmatic': False, 'effective_epistemic_gain_amount': 0, 'known_entities': [['at', 'rotont1', 'atival1'], ['at', 'thium1', 'atival1'], ['at', 'herper1', 'atival1'], ['at', 'ioness1', 'atival1'], ['at', 'equadish1', 'atival1'], ['at', 'fervan1', 'atival1'], ['at', 'player1', 'atival1']], 'known_entities_ratio': 0.4666666666666667, 'visited_rooms': ['atival1'], 'visited_rooms_ratio': 0.25, 'known_goal_entities': [['at', 'rotont1', 'atival1'], ['at', 'ioness1', 'atival1']], 'known_goal_entities_ratio': 0.6666666666666666}} 
	  } 
	   } 
	   } 
	 & & \\ 
 

    \theutterance \stepcounter{utterance}  

    & & & \multicolumn{2}{p{0.3\linewidth}}{\cellcolor[rgb]{0.95,0.95,0.95}{%
	\makecell[{{p{\linewidth}}}]{% 
	  \tt {\tiny [GM$|$GM]}  
	 {'goal_states_achieved': [], 'turn_goal_score': 0} 
	  } 
	   } 
	   } 
	 & & \\ 
 

    \theutterance \stepcounter{utterance}  

    & & \multicolumn{4}{p{0.6\linewidth}}{\cellcolor[rgb]{0.9,0.9,0.9}{%
	\makecell[{{p{\linewidth}}}]{% 
	  \tt {\tiny [A$\langle$GM]}  
	 I don't know how to interpret this 'take' action.\\ \tt  
	  } 
	   } 
	   } 
	 & \\ 
 

    \theutterance \stepcounter{utterance}  

    & \multicolumn{4}{p{0.6\linewidth}}{\cellcolor[rgb]{0.8,1,0.9}{%
	\makecell[{{p{\linewidth}}}]{% 
	  \tt {\tiny [A$\rangle$GM]}  
	 Given the unique vocabulary in your text adventure game, it seems that "take" is a standard action to acquire items. Let's proceed with the actions based on the assumption that "take" allows you to pick up objects:\\ \tt \\ \tt > take rotont and fervan\\ \tt \\ \tt > use fervan to cook rotont\\ \tt \\ \tt > serve cooked rotont to ioness\\ \tt \\ \tt > try to soothe or calm the herper to make it anird\\ \tt \\ \tt > if the herper becomes anird, examine equadish and thium for their potential use in reforian\\ \tt \\ \tt > if any items can contribute to reforian, use them appropriately\\ \tt \\ \tt > once all conditions are met, write "> done" 
	  } 
	   } 
	   } 
	 & & \\ 
 

    \theutterance \stepcounter{utterance}  

    & & & \multicolumn{2}{p{0.3\linewidth}}{\cellcolor[rgb]{0.95,0.95,0.95}{%
	\makecell[{{p{\linewidth}}}]{% 
	  \tt {\tiny [GM$|$GM]}  
	 Round: 2 
	  } 
	   } 
	   } 
	 & & \\ 
 

    \theutterance \stepcounter{utterance}  

    & & & \multicolumn{2}{p{0.3\linewidth}}{\cellcolor[rgb]{0.95,0.95,0.95}{%
	\makecell[{{p{\linewidth}}}]{% 
	  \tt {\tiny [GM$|$GM]}  
	 Given the unique vocabulary in your text adventure game, it seems that "take" is a standard action to acquire items. Let's proceed with the actions based on the assumption that "take" allows you to pick up objects:\\ \tt \\ \tt > take rotont and fervan\\ \tt \\ \tt > use fervan to cook rotont\\ \tt \\ \tt > serve cooked rotont to ioness\\ \tt \\ \tt > try to soothe or calm the herper to make it anird\\ \tt \\ \tt > if the herper becomes anird, examine equadish and thium for their potential use in reforian\\ \tt \\ \tt > if any items can contribute to reforian, use them appropriately\\ \tt \\ \tt > once all conditions are met, write "> done" 
	  } 
	   } 
	   } 
	 & & \\ 
 

    \theutterance \stepcounter{utterance}  

    & & & \multicolumn{2}{p{0.3\linewidth}}{\cellcolor[rgb]{0.95,0.95,0.95}{%
	\makecell[{{p{\linewidth}}}]{% 
	  \tt {\tiny [GM$|$GM]}  
	 command_tag_missing 
	  } 
	   } 
	   } 
	 & & \\ 
 

    \theutterance \stepcounter{utterance}  

    & & & \multicolumn{2}{p{0.3\linewidth}}{\cellcolor[rgb]{0.95,0.95,0.95}{%
	\makecell[{{p{\linewidth}}}]{% 
	  \tt {\tiny [GM$|$GM]}  
	 {'goal_states_achieved': [], 'game_successfully_finished': False} 
	  } 
	   } 
	   } 
	 & & \\ 
 

\end{supertabular}
}

\end{document}


\documentclass{article}
\usepackage{colortbl}
\usepackage{makecell}
\usepackage{multirow}
\usepackage{supertabular}

\begin{document}

\newcounter{utterance}

\twocolumn

{ \footnotesize  \setcounter{utterance}{1}
\setlength{\tabcolsep}{0pt}
\begin{supertabular}{c@{$\;$}|p{.15\linewidth}@{}p{.15\linewidth}p{.15\linewidth}p{.15\linewidth}p{.15\linewidth}p{.15\linewidth}}

    \# & $\;$A & \multicolumn{4}{c}{Game Master} & $\;\:$B\\
    \hline 

    \theutterance \stepcounter{utterance}  

    & & \multicolumn{4}{p{0.6\linewidth}}{\cellcolor[rgb]{0.9,0.9,0.9}{%
	\makecell[{{p{\linewidth}}}]{% 
	  \tt {\tiny [A$\langle$GM]}  
	 Please help me with the following task. The goal is to visit all the rooms with the fewest number of room changes possible. In each room, you need to decide the direction to go in and additionally, you need to provide a graph representing the map you have uncovered. Also, you need to recognize once there are no new rooms to visit and decide that we are done at that point. Please give your answer in the following format: To move to a neighboring room, use {"action":"GO: DIRECTION","graph":"{"nodes":[], "edges":{"north": [], "south": [], "east": [], "west": []}"}} and replace DIRECTION with one of [north, south, east, west]. To stop the exploration, answer with "DONE" instead. Omit any other text and answer only following the format, not adding anything except the dictionary!\\ \tt Here is an example: \\ \tt You are in the Living Room. Currently available directions: south, west. What is your next instruction?\\ \tt {"action":"GO: west", "graph": {"nodes":["Living Room"],"edges":{"north":[],"south":[],"east":[],"west":[]}}}\\ \tt You have made a step and entered a Library. Currently available directions: east, north. What is your next instruction?\\ \tt {"action":"GO: north", "graph":{"nodes":["Living Room", "Library"],"edges":{"north":[],"south":[],"east":[],"west":[("Living Room", "Library")]}}}\\ \tt You have made a step and entered a Kitchen. Currently available directions: south, east. What is your next instruction?\\ \tt {"action":"GO: east", "graph":{"nodes": ["Living Room", "Library", "Kitchen"], "edges":{"north": [("Library", "Kitchen")],"south": [],"east": [],"west": [("Living Room", "Library")]}}}\\ \tt ...\\ \tt You have made a step and entered a Bedroom. Currently available directions: south, west. What is your next instruction?\\ \tt {"action": "DONE", "graph": {...}}\\ \tt Let us start.\\ \tt You are in the Sauna. Currently available directions: east. What is your next instruction? 
	  } 
	   } 
	   } 
	 & \\ 
 

    \theutterance \stepcounter{utterance}  

    & \multicolumn{4}{p{0.6\linewidth}}{\cellcolor[rgb]{0.8,1,0.9}{%
	\makecell[{{p{\linewidth}}}]{% 
	  \tt {\tiny [A$\rangle$GM]}  
	 <think>\\ \tt Okay, let's start. I'm in the Sauna, and the only direction available is east. So I need to go east first. The graph should have the Sauna as a node and no edges yet since I haven't moved. Wait, when I move east, the Sauna will connect to the next room. Let me make sure to update the graph each time.\\ \tt \\ \tt After moving east, I enter a new room. Let's say that's the first step. Now, the graph has Sauna and the new room. The edge between them would be east from Sauna to the new room, and west from the new room back to Sauna. But the example shows that edges are stored as tuples in the direction they were taken. So when I go east from Sauna, the edge is added to the east list as (Sauna, new_room). Then, when I go back west, the edge would be in the west list of the previous room. Wait, the example shows that when moving west from Living Room to Library, the west edge is ("Living Room", "Library"). So the direction is the direction taken from the current node. So each time I move, I add the edge in the direction I took from the previous room to the current one.\\ \tt \\ \tt So for the first move, going east from Sauna to the next room (let's call it Room A), the graph's east edges would have ("Sauna", "Room A"). Then 
	  } 
	   } 
	   } 
	 & & \\ 
 

    \theutterance \stepcounter{utterance}  

    & & & \multicolumn{2}{p{0.3\linewidth}}{\cellcolor[rgb]{0.95,0.95,0.95}{%
	\makecell[{{p{\linewidth}}}]{% 
	  \tt {\tiny [GM$|$GM]}  
	 abort game 
	  } 
	   } 
	   } 
	 & & \\ 
 

\end{supertabular}
}

\end{document}

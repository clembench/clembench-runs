
\documentclass{article}
\usepackage{colortbl}
\usepackage{makecell}
\usepackage{multirow}
\usepackage{supertabular}

\begin{document}

\newcounter{utterance}

\twocolumn

{ \footnotesize  \setcounter{utterance}{1}
\setlength{\tabcolsep}{0pt}
\begin{supertabular}{c@{$\;$}|p{.15\linewidth}@{}p{.15\linewidth}p{.15\linewidth}p{.15\linewidth}p{.15\linewidth}p{.15\linewidth}}

    \# & $\;$A & \multicolumn{4}{c}{Game Master} & $\;\:$B\\
    \hline 

    \theutterance \stepcounter{utterance}  

    & & \multicolumn{4}{p{0.6\linewidth}}{\cellcolor[rgb]{0.9,0.9,0.9}{%
	\makecell[{{p{\linewidth}}}]{% 
	  \tt {\tiny [A$\langle$GM]}  
	 Please help me with the following task. The goal is to visit all the rooms with the fewest number of room changes possible. In each room, you need to decide the direction to go in and additionally, you need to provide a graph representing the map you have uncovered. Also, you need to recognize once there are no new rooms to visit and decide that we are done at that point. Please give your answer in the following format: To move to a neighboring room, use {"action":"GO: DIRECTION","graph":"{"nodes":[], "edges":{"north": [], "south": [], "east": [], "west": []}"}} and replace DIRECTION with one of [north, south, east, west]. To stop the exploration, answer with "DONE" instead. Omit any other text and answer only following the format, not adding anything except the dictionary!\\ \tt Here is an example: \\ \tt You are in the Living Room. Currently available directions: south, west. What is your next instruction?\\ \tt {"action":"GO: west", "graph": {"nodes":["Living Room"],"edges":{"north":[],"south":[],"east":[],"west":[]}}}\\ \tt You have made a step and entered a Library. Currently available directions: east, north. What is your next instruction?\\ \tt {"action":"GO: north", "graph":{"nodes":["Living Room", "Library"],"edges":{"north":[],"south":[],"east":[],"west":[("Living Room", "Library")]}}}\\ \tt You have made a step and entered a Kitchen. Currently available directions: south, east. What is your next instruction?\\ \tt {"action":"GO: east", "graph":{"nodes": ["Living Room", "Library", "Kitchen"], "edges":{"north": [("Library", "Kitchen")],"south": [],"east": [],"west": [("Living Room", "Library")]}}}\\ \tt ...\\ \tt You have made a step and entered a Bedroom. Currently available directions: south, west. What is your next instruction?\\ \tt {"action": "DONE", "graph": {...}}\\ \tt Let us start.\\ \tt You are in the Living room. Currently available directions: west, east. What is your next instruction? 
	  } 
	   } 
	   } 
	 & \\ 
 

    \theutterance \stepcounter{utterance}  

    & \multicolumn{4}{p{0.6\linewidth}}{\cellcolor[rgb]{0.8,1,0.9}{%
	\makecell[{{p{\linewidth}}}]{% 
	  \tt {\tiny [A$\rangle$GM]}  
	 <think>\\ \tt Okay, let's see. I need to figure out the best way to visit all the rooms with the fewest changes. The user is in the Living Room and can go west or east. The example they gave uses a graph to track nodes and edges. So, first, I should start by moving in one direction, maybe west, and then keep track of the rooms and connections.\\ \tt \\ \tt Wait, the initial graph has the Living Room as a node. When moving west, the graph's west edge should have the connection from Living Room to the next room. But since I don't know the next room's name yet, maybe I just add the direction. Hmm, but in the example, when moving west from Living Room to Library, the west edge has ("Living Room", "Library"). So the graph is built incrementally as rooms are discovered.\\ \tt \\ \tt So the strategy here is to explore as much as possible without backtracking. Maybe use a depth-first search approach. Start by going west first. Then, from the next room, check available directions. If there are unvisited directions, go that way. If not, backtrack. But since the user wants the fewest room changes, maybe avoid backtracking as much as possible. Wait, but sometimes backtracking is necessary if there are branches.\\ \tt \\ \tt Wait, the problem says to decide the direction to go in each room and provide the graph. Also, recognize when all rooms are visited. So the key is to keep track of visited rooms 
	  } 
	   } 
	   } 
	 & & \\ 
 

    \theutterance \stepcounter{utterance}  

    & & & \multicolumn{2}{p{0.3\linewidth}}{\cellcolor[rgb]{0.95,0.95,0.95}{%
	\makecell[{{p{\linewidth}}}]{% 
	  \tt {\tiny [GM$|$GM]}  
	 abort game 
	  } 
	   } 
	   } 
	 & & \\ 
 

\end{supertabular}
}

\end{document}

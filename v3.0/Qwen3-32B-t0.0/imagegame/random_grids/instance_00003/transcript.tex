
\documentclass{article}
\usepackage{colortbl}
\usepackage{makecell}
\usepackage{multirow}
\usepackage{supertabular}

\begin{document}

\newcounter{utterance}

\twocolumn

{ \footnotesize  \setcounter{utterance}{1}
\setlength{\tabcolsep}{0pt}
\begin{supertabular}{c@{$\;$}|p{.15\linewidth}@{}p{.15\linewidth}p{.15\linewidth}p{.15\linewidth}p{.15\linewidth}p{.15\linewidth}}

    \# & $\;$A & \multicolumn{4}{c}{Game Master} & $\;\:$B\\
    \hline 

    \theutterance \stepcounter{utterance}  

    & & \multicolumn{4}{p{0.6\linewidth}}{\cellcolor[rgb]{0.9,0.9,0.9}{%
	\makecell[{{p{\linewidth}}}]{% 
	  \tt {\tiny [A$\langle$GM]}  
	 Let us do the following task. The goal is to fill an empty grid that looks like this:\\ \tt \\ \tt ▢ ▢ ▢ ▢ ▢\\ \tt ▢ ▢ ▢ ▢ ▢\\ \tt ▢ ▢ ▢ ▢ ▢\\ \tt ▢ ▢ ▢ ▢ ▢\\ \tt ▢ ▢ ▢ ▢ ▢\\ \tt \\ \tt A filled grid below is 5 by 5 and can look like this:\\ \tt \\ \tt ▢ ▢ ▢ ▢ ▢\\ \tt ▢ ▢ E ▢ ▢\\ \tt ▢ ▢ ▢ ▢ ▢\\ \tt ▢ ▢ ▢ ▢ ▢\\ \tt X X X X X\\ \tt \\ \tt I want you to describe this grid to me, step by step. You don't need to describe the empty squares, which are denoted with "▢". Only describe the location of letters in the grid. Then you wait for me to say "What is your next command?", and then you continue with the next step. Take the size of the grid into consideration while giving commands. You can give only one command at each turn but it can refer to single or multiple cells, rows, columns or even the whole grid at once. When you have described everything, you say "Command: DONE". Only give the command and omit everything else.\\ \tt \\ \tt For the filled grid above, here are the example steps.\\ \tt \\ \tt Command: Put an E in second row third column\\ \tt \\ \tt What is your next command?\\ \tt Command: Fill the last row with X\\ \tt \\ \tt What is your next command?\\ \tt Command: DONE\\ \tt \\ \tt Ok. Please do this for the following example, which is a 5 by 5 grid.\\ \tt ▢ ▢ Y Y ▢\\ \tt ▢ ▢ ▢ Y ▢\\ \tt ▢ ▢ Y ▢ ▢\\ \tt ▢ ▢ ▢ ▢ ▢\\ \tt ▢ ▢ ▢ ▢ Y\\ \tt What is your next command? 
	  } 
	   } 
	   } 
	 & \\ 
 

    \theutterance \stepcounter{utterance}  

    & \multicolumn{4}{p{0.6\linewidth}}{\cellcolor[rgb]{0.8,1,0.9}{%
	\makecell[{{p{\linewidth}}}]{% 
	  \tt {\tiny [A$\rangle$GM]}  
	 <think>\\ \tt Okay, let's see. The user wants me to describe the filled grid step by step, only mentioning the letters and their positions. The grid is 5x5. Let me look at the example they provided.\\ \tt \\ \tt First, the example grid has Ys and a Y in specific positions. Let me map out where each letter is.\\ \tt \\ \tt Looking at the first row: ▢ ▢ Y Y ▢. So positions (1,3) and (1,4) are Y.\\ \tt \\ \tt Second row: ▢ ▢ ▢ Y ▢. That's (2,4) as Y.\\ \tt \\ \tt Third row: ▢ ▢ Y ▢ ▢. So (3,3) is Y.\\ \tt \\ \tt Fourth row is all empty except maybe the fifth row. Fifth row: ▢ ▢ ▢ ▢ Y. So (5,5) is Y.\\ \tt \\ \tt So the letters are at (1,3), (1,4), (2,4), (3,3), and (5,5). \\ \tt \\ \tt Now, the user's example steps first mentioned putting an E in a specific cell, then filling a row. So I need to follow that pattern. Each command should be a single action. Let me start with the first Y in the first row, third column. \\ \tt \\ \tt First command: Put a Y in first row third column. Then next, the adjacent Y in first row fourth column. Then the Y in second row fourth column, then third 
	  } 
	   } 
	   } 
	 & & \\ 
 

    \theutterance \stepcounter{utterance}  

    & & & \multicolumn{2}{p{0.3\linewidth}}{\cellcolor[rgb]{0.95,0.95,0.95}{%
	\makecell[{{p{\linewidth}}}]{% 
	  \tt {\tiny [GM$|$GM]}  
	 Invalid instruction format 
	  } 
	   } 
	   } 
	 & & \\ 
 

\end{supertabular}
}

\end{document}

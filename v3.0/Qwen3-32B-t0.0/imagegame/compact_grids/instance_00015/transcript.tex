
\documentclass{article}
\usepackage{colortbl}
\usepackage{makecell}
\usepackage{multirow}
\usepackage{supertabular}

\begin{document}

\newcounter{utterance}

\twocolumn

{ \footnotesize  \setcounter{utterance}{1}
\setlength{\tabcolsep}{0pt}
\begin{supertabular}{c@{$\;$}|p{.15\linewidth}@{}p{.15\linewidth}p{.15\linewidth}p{.15\linewidth}p{.15\linewidth}p{.15\linewidth}}

    \# & $\;$A & \multicolumn{4}{c}{Game Master} & $\;\:$B\\
    \hline 

    \theutterance \stepcounter{utterance}  

    & & \multicolumn{4}{p{0.6\linewidth}}{\cellcolor[rgb]{0.9,0.9,0.9}{%
	\makecell[{{p{\linewidth}}}]{% 
	  \tt {\tiny [A$\langle$GM]}  
	 Let us do the following task. The goal is to fill an empty grid that looks like this:\\ \tt \\ \tt ▢ ▢ ▢ ▢ ▢\\ \tt ▢ ▢ ▢ ▢ ▢\\ \tt ▢ ▢ ▢ ▢ ▢\\ \tt ▢ ▢ ▢ ▢ ▢\\ \tt ▢ ▢ ▢ ▢ ▢\\ \tt \\ \tt A filled grid below is 5 by 5 and can look like this:\\ \tt \\ \tt ▢ ▢ ▢ ▢ ▢\\ \tt ▢ ▢ E ▢ ▢\\ \tt ▢ ▢ ▢ ▢ ▢\\ \tt ▢ ▢ ▢ ▢ ▢\\ \tt X X X X X\\ \tt \\ \tt I want you to describe this grid to me, step by step. You don't need to describe the empty squares, which are denoted with "▢". Only describe the location of letters in the grid. Then you wait for me to say "What is your next command?", and then you continue with the next step. Take the size of the grid into consideration while giving commands. You can give only one command at each turn but it can refer to single or multiple cells, rows, columns or even the whole grid at once. When you have described everything, you say "Command: DONE". Only give the command and omit everything else.\\ \tt \\ \tt For the filled grid above, here are the example steps.\\ \tt \\ \tt Command: Put an E in second row third column\\ \tt \\ \tt What is your next command?\\ \tt Command: Fill the last row with X\\ \tt \\ \tt What is your next command?\\ \tt Command: DONE\\ \tt \\ \tt Ok. Please do this for the following example, which is a 5 by 5 grid.\\ \tt ▢ ▢ ▢ ▢ W\\ \tt ▢ ▢ ▢ ▢ W\\ \tt ▢ ▢ ▢ ▢ W\\ \tt ▢ ▢ ▢ ▢ W\\ \tt W W W W W\\ \tt What is your next command? 
	  } 
	   } 
	   } 
	 & \\ 
 

    \theutterance \stepcounter{utterance}  

    & \multicolumn{4}{p{0.6\linewidth}}{\cellcolor[rgb]{0.8,1,0.9}{%
	\makecell[{{p{\linewidth}}}]{% 
	  \tt {\tiny [A$\rangle$GM]}  
	 <think>\\ \tt Okay, let's see. The user provided a 5x5 grid where most of the cells are empty (denoted by ▢), except for some W's and a possible other letter. Let me look at the example they gave.\\ \tt \\ \tt The grid is:\\ \tt \\ \tt Row 1: ▢ ▢ ▢ ▢ W\\ \tt Row 2: ▢ ▢ ▢ ▢ W\\ \tt Row 3: ▢ ▢ ▢ ▢ W\\ \tt Row 4: ▢ ▢ ▢ ▢ W\\ \tt Row 5: W W W W W\\ \tt \\ \tt So, the first thing I notice is that in each of the first four rows, the fifth column has a W. Then the fifth row is entirely filled with W's. \\ \tt \\ \tt The user wants me to describe the grid step by step, only mentioning the letters and their positions. Each command should be one step, and I can refer to cells, rows, columns, or the whole grid. The example they gave used commands like "Put an E in second row third column" and "Fill the last row with X". \\ \tt \\ \tt So, for this grid, the first step would be to place W's in the fifth column of the first four rows. But since each of those is a single cell, I could do each one individually. However, maybe there's a more efficient way. Let me check the rules again. The user said that a command can refer to multiple cells, rows, 
	  } 
	   } 
	   } 
	 & & \\ 
 

    \theutterance \stepcounter{utterance}  

    & & & \multicolumn{2}{p{0.3\linewidth}}{\cellcolor[rgb]{0.95,0.95,0.95}{%
	\makecell[{{p{\linewidth}}}]{% 
	  \tt {\tiny [GM$|$GM]}  
	 Invalid instruction format 
	  } 
	   } 
	   } 
	 & & \\ 
 

\end{supertabular}
}

\end{document}


\documentclass{article}
\usepackage{colortbl}
\usepackage{makecell}
\usepackage{multirow}
\usepackage{supertabular}

\begin{document}

\newcounter{utterance}

\twocolumn

{ \footnotesize  \setcounter{utterance}{1}
\setlength{\tabcolsep}{0pt}
\begin{supertabular}{c@{$\;$}|p{.15\linewidth}@{}p{.15\linewidth}p{.15\linewidth}p{.15\linewidth}p{.15\linewidth}p{.15\linewidth}}

    \# & $\;$A & \multicolumn{4}{c}{Game Master} & $\;\:$B\\
    \hline 

    \theutterance \stepcounter{utterance}  

    & & \multicolumn{4}{p{0.6\linewidth}}{\cellcolor[rgb]{0.9,0.9,0.9}{%
	\makecell[{{p{\linewidth}}}]{% 
	  \tt {\tiny [A$\langle$GM]}  
	 Let's play a guessing game. Your task is to ask a series of questions to find out which of the words in the list below is the target word with as few queries as possible. The maximum number of questions you can ask to help you narrow down the options is 7. Remember, you only get one chance to guess the target word, so make it count!\\ \tt \\ \tt The rules of the game are: \\ \tt \\ \tt 1. Only ask one question per turn. DO NOT ASK MORE THAN ONE QUESTION.\\ \tt \\ \tt 2. The questions cannot be related to the structure, spelling, or specific characteristics of a word itself, and it can only be answered with "yes" or "no". \\ \tt \\ \tt 3. Use the following format for asking questions and DO NOT ADD ANY TEXT OTHER THAN THIS: \\ \tt \\ \tt QUESTION: <your question>\\ \tt \\ \tt 4. You have only one opportunity to guess the target word. To make your guess, use the following format and DO NOT ADD ANY TEXT OTHER THAN THIS: \\ \tt \\ \tt GUESS: <target word>\\ \tt \\ \tt Remember you cannot ask for a hint. Make your question brief. Let’s start!\\ \tt \\ \tt List of candidate words: ['Mosquito', 'Bee', 'Disco', 'Blues', 'Fog', 'Hurricane', 'Cannabis', 'Pepper'] 
	  } 
	   } 
	   } 
	 & \\ 
 

    \theutterance \stepcounter{utterance}  

    & \multicolumn{4}{p{0.6\linewidth}}{\cellcolor[rgb]{0.8,1,0.9}{%
	\makecell[{{p{\linewidth}}}]{% 
	  \tt {\tiny [A$\rangle$GM]}  
	 <think>\\ \tt Okay, let's see. I need to figure out which word is the target by asking yes/no questions. The list is ['Mosquito', 'Bee', 'Disco', 'Blues', 'Fog', 'Hurricane', 'Cannabis', 'Pepper']. First, I should look for categories or common attributes that can split the list into halves.\\ \tt \\ \tt Looking at the words, some are insects (Mosquito, Bee), some are related to weather or natural phenomena (Fog, Hurricane), some are music genres (Disco, Blues), and others are plants or spices (Cannabis, Pepper). Maybe start by categorizing them. \\ \tt \\ \tt First question: Is the target a type of insect? If yes, then it's either Mosquito or Bee. If no, then check other categories. Let's ask that. If the answer is yes, next question could be about stinging. If no, maybe check if it's a weather-related term. If the answer is no to insect, then maybe ask if it's a music genre. If yes, then Disco or Blues. If not, check if it's a plant or spice. \\ \tt \\ \tt Alternatively, maybe start with whether it's a natural phenomenon. But the list has Fog and Hurricane. If the answer is yes, then those two. If no, proceed. But maybe the first approach is better. Let me outline possible splits:\\ \tt \\ \tt Insects: 2\\ \tt Weather: 2\\ \tt Music 
	  } 
	   } 
	   } 
	 & & \\ 
 

    \theutterance \stepcounter{utterance}  

    & & & \multicolumn{2}{p{0.3\linewidth}}{\cellcolor[rgb]{0.95,0.95,0.95}{%
	\makecell[{{p{\linewidth}}}]{% 
	  \tt {\tiny [GM$|$GM]}  
	 Invalid format. Guesser must use the form 'QUESTION: ' or 'GUESS: '. 
	  } 
	   } 
	   } 
	 & & \\ 
 

    \theutterance \stepcounter{utterance}  

    & & & \multicolumn{2}{p{0.3\linewidth}}{\cellcolor[rgb]{0.95,0.95,0.95}{%
	\makecell[{{p{\linewidth}}}]{% 
	  \tt {\tiny [GM$|$GM]}  
	 abort game 
	  } 
	   } 
	   } 
	 & & \\ 
 

\end{supertabular}
}

\end{document}

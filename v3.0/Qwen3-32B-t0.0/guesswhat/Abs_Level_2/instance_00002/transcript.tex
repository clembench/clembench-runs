
\documentclass{article}
\usepackage{colortbl}
\usepackage{makecell}
\usepackage{multirow}
\usepackage{supertabular}

\begin{document}

\newcounter{utterance}

\twocolumn

{ \footnotesize  \setcounter{utterance}{1}
\setlength{\tabcolsep}{0pt}
\begin{supertabular}{c@{$\;$}|p{.15\linewidth}@{}p{.15\linewidth}p{.15\linewidth}p{.15\linewidth}p{.15\linewidth}p{.15\linewidth}}

    \# & $\;$A & \multicolumn{4}{c}{Game Master} & $\;\:$B\\
    \hline 

    \theutterance \stepcounter{utterance}  

    & & \multicolumn{4}{p{0.6\linewidth}}{\cellcolor[rgb]{0.9,0.9,0.9}{%
	\makecell[{{p{\linewidth}}}]{% 
	  \tt {\tiny [A$\langle$GM]}  
	 Let's play a guessing game. Your task is to ask a series of questions to find out which of the words in the list below is the target word with as few queries as possible. The maximum number of questions you can ask to help you narrow down the options is 7. Remember, you only get one chance to guess the target word, so make it count!\\ \tt \\ \tt The rules of the game are: \\ \tt \\ \tt 1. Only ask one question per turn. DO NOT ASK MORE THAN ONE QUESTION.\\ \tt \\ \tt 2. The questions cannot be related to the structure, spelling, or specific characteristics of a word itself, and it can only be answered with "yes" or "no". \\ \tt \\ \tt 3. Use the following format for asking questions and DO NOT ADD ANY TEXT OTHER THAN THIS: \\ \tt \\ \tt QUESTION: <your question>\\ \tt \\ \tt 4. You have only one opportunity to guess the target word. To make your guess, use the following format and DO NOT ADD ANY TEXT OTHER THAN THIS: \\ \tt \\ \tt GUESS: <target word>\\ \tt \\ \tt Remember you cannot ask for a hint. Make your question brief. Let’s start!\\ \tt \\ \tt List of candidate words: ['french', 'spanish', 'technology', 'engineering', 'twenty-three', 'eleven', 'one', 'seven'] 
	  } 
	   } 
	   } 
	 & \\ 
 

    \theutterance \stepcounter{utterance}  

    & \multicolumn{4}{p{0.6\linewidth}}{\cellcolor[rgb]{0.8,1,0.9}{%
	\makecell[{{p{\linewidth}}}]{% 
	  \tt {\tiny [A$\rangle$GM]}  
	 <think>\\ \tt Okay, let's see. I need to figure out which word is the target by asking yes/no questions. The list has words like french, spanish, technology, engineering, twenty-three, eleven, one, seven. \\ \tt \\ \tt First, I should categorize them. Some are languages (french, spanish), some are fields (technology, engineering), and others are numbers (twenty-three, eleven, one, seven). Maybe start by asking if the word is a language. If yes, then between french and spanish. If not, check if it's a field. If not, then it's a number. \\ \tt \\ \tt But wait, the numbers are written out. Maybe the next step after ruling out languages and fields is to ask if it's a number. Then, if yes, narrow down the numbers. For example, ask if it's a two-digit number. Twenty-three and eleven are two-digit, while one and seven are single-digit. Then, if it's two-digit, ask if it's eleven. If not, then twenty-three. If it's single-digit, then between one and seven. Maybe ask if it's one. \\ \tt \\ \tt But I need to make sure each question splits the possibilities as much as possible. Let me outline possible questions:\\ \tt \\ \tt 1. Is the word a language? (french, spanish)\\ \tt    - Yes: next question between the two.\\ \tt    - No: proceed.\\ \tt \\ \tt 2. If not a language, is it a field? (technology, 
	  } 
	   } 
	   } 
	 & & \\ 
 

    \theutterance \stepcounter{utterance}  

    & & & \multicolumn{2}{p{0.3\linewidth}}{\cellcolor[rgb]{0.95,0.95,0.95}{%
	\makecell[{{p{\linewidth}}}]{% 
	  \tt {\tiny [GM$|$GM]}  
	 Invalid format. Guesser must use the form 'QUESTION: ' or 'GUESS: '. 
	  } 
	   } 
	   } 
	 & & \\ 
 

    \theutterance \stepcounter{utterance}  

    & & & \multicolumn{2}{p{0.3\linewidth}}{\cellcolor[rgb]{0.95,0.95,0.95}{%
	\makecell[{{p{\linewidth}}}]{% 
	  \tt {\tiny [GM$|$GM]}  
	 abort game 
	  } 
	   } 
	   } 
	 & & \\ 
 

\end{supertabular}
}

\end{document}

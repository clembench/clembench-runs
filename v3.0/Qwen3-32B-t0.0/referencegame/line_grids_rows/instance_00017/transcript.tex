
\documentclass{article}
\usepackage{colortbl}
\usepackage{makecell}
\usepackage{multirow}
\usepackage{supertabular}

\begin{document}

\newcounter{utterance}

\twocolumn

{ \footnotesize  \setcounter{utterance}{1}
\setlength{\tabcolsep}{0pt}
\begin{supertabular}{c@{$\;$}|p{.15\linewidth}@{}p{.15\linewidth}p{.15\linewidth}p{.15\linewidth}p{.15\linewidth}p{.15\linewidth}}

    \# & $\;$A & \multicolumn{4}{c}{Game Master} & $\;\:$B\\
    \hline 

    \theutterance \stepcounter{utterance}  

    & & \multicolumn{4}{p{0.6\linewidth}}{\cellcolor[rgb]{0.9,0.9,0.9}{%
	\makecell[{{p{\linewidth}}}]{% 
	  \tt {\tiny [A$\langle$GM]}  
	 You are given three grids, where each of them is 5 by 5 in size.\\ \tt Grids have empty cells marked with "▢" and filled cells marked with "X".\\ \tt Your task is to generate a referring expression that best describes the target grid while distinguishing it from the two other distractor grids.\\ \tt The first grid is the target grid, and the following two grids are the distractors.\\ \tt \\ \tt Target grid:\\ \tt \\ \tt □ X X □ X\\ \tt □ X X □ X\\ \tt □ X X □ X\\ \tt □ X X □ X\\ \tt □ X X □ X\\ \tt \\ \tt Distractor grid 1:\\ \tt \\ \tt □ □ X X X\\ \tt □ □ X X X\\ \tt □ □ X X X\\ \tt □ □ X X X\\ \tt □ □ X X X\\ \tt \\ \tt Distractor grid 2:\\ \tt \\ \tt X X X X X\\ \tt □ □ □ □ □\\ \tt □ □ □ □ □\\ \tt X X X X X\\ \tt □ □ □ □ □\\ \tt \\ \tt Instruction: Describe the target grid.\\ \tt Generate the referring expression starting with the tag "Expression: " for the given target grid. Omit any other text. 
	  } 
	   } 
	   } 
	 & \\ 
 

    \theutterance \stepcounter{utterance}  

    & \multicolumn{4}{p{0.6\linewidth}}{\cellcolor[rgb]{0.8,1,0.9}{%
	\makecell[{{p{\linewidth}}}]{% 
	  \tt {\tiny [A$\rangle$GM]}  
	 <think>\\ \tt Okay, let's see. I need to describe the target grid so that it's different from the two distractors. Let me look at each grid carefully.\\ \tt \\ \tt First, the target grid. It's a 5x5 grid. Looking at the rows: each row starts with a □, then two Xs, another □, and ends with an X. So the pattern is □ X X □ X repeated for each row. So every row has the same structure. The filled cells (Xs) are in columns 2, 3, and 5, with the first and fourth columns empty. So each row has three Xs in positions 2, 3, 5.\\ \tt \\ \tt Now, Distractor 1: Each row starts with two □s, then three Xs. So the pattern is □ □ X X X. So in this grid, the first two columns are empty, and columns 3,4,5 are filled. Each row has three Xs, but their positions are different from the target. So the key difference here is the placement of the Xs. Target has Xs in columns 2,3,5, while D1 has them in 3,4,5.\\ \tt \\ \tt Distractor 2 is different. The first row is all Xs (X X X X X), then two rows of all □s, then a row of Xs again, then � 
	  } 
	   } 
	   } 
	 & & \\ 
 

    \theutterance \stepcounter{utterance}  

    & & & \multicolumn{2}{p{0.3\linewidth}}{\cellcolor[rgb]{0.95,0.95,0.95}{%
	\makecell[{{p{\linewidth}}}]{% 
	  \tt {\tiny [GM$|$GM]}  
	 Invalid generated expression 
	  } 
	   } 
	   } 
	 & & \\ 
 

\end{supertabular}
}

\end{document}

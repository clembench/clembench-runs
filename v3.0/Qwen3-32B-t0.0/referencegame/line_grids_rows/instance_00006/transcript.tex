
\documentclass{article}
\usepackage{colortbl}
\usepackage{makecell}
\usepackage{multirow}
\usepackage{supertabular}

\begin{document}

\newcounter{utterance}

\twocolumn

{ \footnotesize  \setcounter{utterance}{1}
\setlength{\tabcolsep}{0pt}
\begin{supertabular}{c@{$\;$}|p{.15\linewidth}@{}p{.15\linewidth}p{.15\linewidth}p{.15\linewidth}p{.15\linewidth}p{.15\linewidth}}

    \# & $\;$A & \multicolumn{4}{c}{Game Master} & $\;\:$B\\
    \hline 

    \theutterance \stepcounter{utterance}  

    & & \multicolumn{4}{p{0.6\linewidth}}{\cellcolor[rgb]{0.9,0.9,0.9}{%
	\makecell[{{p{\linewidth}}}]{% 
	  \tt {\tiny [A$\langle$GM]}  
	 You are given three grids, where each of them is 5 by 5 in size.\\ \tt Grids have empty cells marked with "▢" and filled cells marked with "X".\\ \tt Your task is to generate a referring expression that best describes the target grid while distinguishing it from the two other distractor grids.\\ \tt The first grid is the target grid, and the following two grids are the distractors.\\ \tt \\ \tt Target grid:\\ \tt \\ \tt X □ □ X X\\ \tt X □ □ X X\\ \tt X □ □ X X\\ \tt X □ □ X X\\ \tt X □ □ X X\\ \tt \\ \tt Distractor grid 1:\\ \tt \\ \tt □ □ □ X X\\ \tt □ □ □ X X\\ \tt □ □ □ X X\\ \tt □ □ □ X X\\ \tt □ □ □ X X\\ \tt \\ \tt Distractor grid 2:\\ \tt \\ \tt X X □ X X\\ \tt X X □ X X\\ \tt X X □ X X\\ \tt X X □ X X\\ \tt X X □ X X\\ \tt \\ \tt Instruction: Describe the target grid.\\ \tt Generate the referring expression starting with the tag "Expression: " for the given target grid. Omit any other text. 
	  } 
	   } 
	   } 
	 & \\ 
 

    \theutterance \stepcounter{utterance}  

    & \multicolumn{4}{p{0.6\linewidth}}{\cellcolor[rgb]{0.8,1,0.9}{%
	\makecell[{{p{\linewidth}}}]{% 
	  \tt {\tiny [A$\rangle$GM]}  
	 <think>\\ \tt Okay, let's see. I need to create a referring expression that describes the target grid and distinguishes it from the two distractors. Let me start by looking at all three grids carefully.\\ \tt \\ \tt The target grid is a 5x5 grid. Each row starts with an X, then two empty cells (□), followed by two Xs. So the pattern is X □ □ X X repeated for each row. Let me check the rows one by one to confirm. Yes, every row in the target is the same: first column is X, columns 2 and 3 are empty, columns 4 and 5 are X. So it's consistent across all five rows.\\ \tt \\ \tt Now, Distractor 1 has the first column all empty (□ □ □ X X for each row). So instead of starting with X, it starts with three empty cells. The last two columns are Xs, same as the target. So the main difference here is the first column: target has Xs there, distractor 1 has empties.\\ \tt \\ \tt Distractor 2 has the first two columns filled with Xs (X X □ X X). The third column is empty, then Xs again. So the difference here is that in the target, the first column is X and the second and third are empty, whereas in distractor 2, the first two columns are X and the third is empty. \\ \tt \\ \tt So the key features of the 
	  } 
	   } 
	   } 
	 & & \\ 
 

    \theutterance \stepcounter{utterance}  

    & & & \multicolumn{2}{p{0.3\linewidth}}{\cellcolor[rgb]{0.95,0.95,0.95}{%
	\makecell[{{p{\linewidth}}}]{% 
	  \tt {\tiny [GM$|$GM]}  
	 Invalid generated expression 
	  } 
	   } 
	   } 
	 & & \\ 
 

\end{supertabular}
}

\end{document}

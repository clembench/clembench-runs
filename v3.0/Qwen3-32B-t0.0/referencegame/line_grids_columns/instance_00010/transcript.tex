
\documentclass{article}
\usepackage{colortbl}
\usepackage{makecell}
\usepackage{multirow}
\usepackage{supertabular}

\begin{document}

\newcounter{utterance}

\twocolumn

{ \footnotesize  \setcounter{utterance}{1}
\setlength{\tabcolsep}{0pt}
\begin{supertabular}{c@{$\;$}|p{.15\linewidth}@{}p{.15\linewidth}p{.15\linewidth}p{.15\linewidth}p{.15\linewidth}p{.15\linewidth}}

    \# & $\;$A & \multicolumn{4}{c}{Game Master} & $\;\:$B\\
    \hline 

    \theutterance \stepcounter{utterance}  

    & & \multicolumn{4}{p{0.6\linewidth}}{\cellcolor[rgb]{0.9,0.9,0.9}{%
	\makecell[{{p{\linewidth}}}]{% 
	  \tt {\tiny [A$\langle$GM]}  
	 You are given three grids, where each of them is 5 by 5 in size.\\ \tt Grids have empty cells marked with "▢" and filled cells marked with "X".\\ \tt Your task is to generate a referring expression that best describes the target grid while distinguishing it from the two other distractor grids.\\ \tt The first grid is the target grid, and the following two grids are the distractors.\\ \tt \\ \tt Target grid:\\ \tt \\ \tt X X X □ □\\ \tt X X X □ □\\ \tt X X X □ □\\ \tt X X X □ □\\ \tt X X X □ □\\ \tt \\ \tt Distractor grid 1:\\ \tt \\ \tt □ X X X □\\ \tt □ X X X □\\ \tt □ X X X □\\ \tt □ X X X □\\ \tt □ X X X □\\ \tt \\ \tt Distractor grid 2:\\ \tt \\ \tt □ X X □ X\\ \tt □ X X □ X\\ \tt □ X X □ X\\ \tt □ X X □ X\\ \tt □ X X □ X\\ \tt \\ \tt Instruction: Describe the target grid.\\ \tt Generate the referring expression starting with the tag "Expression: " for the given target grid. Omit any other text. 
	  } 
	   } 
	   } 
	 & \\ 
 

    \theutterance \stepcounter{utterance}  

    & \multicolumn{4}{p{0.6\linewidth}}{\cellcolor[rgb]{0.8,1,0.9}{%
	\makecell[{{p{\linewidth}}}]{% 
	  \tt {\tiny [A$\rangle$GM]}  
	 <think>\\ \tt Okay, let's see. I need to describe the target grid so that it's different from the two distractors. Let me look at each grid carefully.\\ \tt \\ \tt The target grid has the first three columns filled with X's and the last two columns empty. So each row starts with three Xs followed by two □s. \\ \tt \\ \tt Distractor 1 has a column of □s in the first column, then three Xs, then a □. So each row is □ X X X □. That's different because the first column is empty and the fourth is empty. The target's empty columns are the last two, so maybe mention the position of the empty columns.\\ \tt \\ \tt Distractor 2 has a pattern of □ X X □ X. So columns 1 and 4 are empty, and column 5 is filled. The target doesn't have that. \\ \tt \\ \tt So the key differences are the placement of the empty columns. Target has columns 4 and 5 empty. Distractor 1 has columns 1 and 4 empty. Distractor 2 has columns 1 and 4 empty but also column 5 filled. \\ \tt \\ \tt So I need to describe the target's empty columns as the last two. Maybe say "the first three columns are completely filled with Xs, and the last two columns are entirely empty." That should distinguish it from both distractors. Let me check:\\ \tt \\ \tt Distractor 1's first 
	  } 
	   } 
	   } 
	 & & \\ 
 

    \theutterance \stepcounter{utterance}  

    & & & \multicolumn{2}{p{0.3\linewidth}}{\cellcolor[rgb]{0.95,0.95,0.95}{%
	\makecell[{{p{\linewidth}}}]{% 
	  \tt {\tiny [GM$|$GM]}  
	 Invalid generated expression 
	  } 
	   } 
	   } 
	 & & \\ 
 

\end{supertabular}
}

\end{document}

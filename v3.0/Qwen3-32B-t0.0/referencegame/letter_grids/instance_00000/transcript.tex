
\documentclass{article}
\usepackage{colortbl}
\usepackage{makecell}
\usepackage{multirow}
\usepackage{supertabular}

\begin{document}

\newcounter{utterance}

\twocolumn

{ \footnotesize  \setcounter{utterance}{1}
\setlength{\tabcolsep}{0pt}
\begin{supertabular}{c@{$\;$}|p{.15\linewidth}@{}p{.15\linewidth}p{.15\linewidth}p{.15\linewidth}p{.15\linewidth}p{.15\linewidth}}

    \# & $\;$A & \multicolumn{4}{c}{Game Master} & $\;\:$B\\
    \hline 

    \theutterance \stepcounter{utterance}  

    & & \multicolumn{4}{p{0.6\linewidth}}{\cellcolor[rgb]{0.9,0.9,0.9}{%
	\makecell[{{p{\linewidth}}}]{% 
	  \tt {\tiny [A$\langle$GM]}  
	 You are given three grids, where each of them is 5 by 5 in size.\\ \tt Grids have empty cells marked with "▢" and filled cells marked with "X".\\ \tt Your task is to generate a referring expression that best describes the target grid while distinguishing it from the two other distractor grids.\\ \tt The first grid is the target grid, and the following two grids are the distractors.\\ \tt \\ \tt Target grid:\\ \tt \\ \tt X X X X X\\ \tt X □ X □ X\\ \tt X X X X X\\ \tt X □ X □ X\\ \tt X X X X X\\ \tt \\ \tt Distractor grid 1:\\ \tt \\ \tt X X X X X\\ \tt X □ □ □ □\\ \tt X X X X X\\ \tt X □ □ □ □\\ \tt X X X X X\\ \tt \\ \tt Distractor grid 2:\\ \tt \\ \tt X X X X X\\ \tt X □ □ □ □\\ \tt X □ □ □ □\\ \tt X □ □ □ □\\ \tt X X X X X\\ \tt \\ \tt Instruction: Describe the target grid.\\ \tt Generate the referring expression starting with the tag "Expression: " for the given target grid. Omit any other text. 
	  } 
	   } 
	   } 
	 & \\ 
 

    \theutterance \stepcounter{utterance}  

    & \multicolumn{4}{p{0.6\linewidth}}{\cellcolor[rgb]{0.8,1,0.9}{%
	\makecell[{{p{\linewidth}}}]{% 
	  \tt {\tiny [A$\rangle$GM]}  
	 <think>\\ \tt Okay, let's see. I need to describe the target grid so that it's different from the two distractors. Let me look at all three grids again.\\ \tt \\ \tt The target grid has X's all around the edges. In the middle rows (rows 2 and 4), there are alternating X and □. Like, in row 2, it's X, □, X, □, X. Same for row 4. The other rows (1,3,5) are all X's. \\ \tt \\ \tt Distractor 1: Rows 2 and 4 have four □ in a row. So row 2 is X followed by four □, and row 4 is similar. The rest are X's. So the middle rows have a block of four □, unlike the target's alternating pattern.\\ \tt \\ \tt Distractor 2: It has three middle rows (rows 2,3,4) each with four □. So rows 2-4 all have X followed by four □. The target only has rows 2 and 4 with alternating, so the number of middle rows with □ is different here.\\ \tt \\ \tt So the key differences are the pattern in the middle rows. The target alternates X and □ in rows 2 and 4, while the distractors have continuous □s. Also, Distractor 2 has three such rows instead of two. \\ \tt \\ \tt I need to mention that the 
	  } 
	   } 
	   } 
	 & & \\ 
 

    \theutterance \stepcounter{utterance}  

    & & & \multicolumn{2}{p{0.3\linewidth}}{\cellcolor[rgb]{0.95,0.95,0.95}{%
	\makecell[{{p{\linewidth}}}]{% 
	  \tt {\tiny [GM$|$GM]}  
	 Invalid generated expression 
	  } 
	   } 
	   } 
	 & & \\ 
 

\end{supertabular}
}

\end{document}

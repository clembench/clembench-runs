
\documentclass{article}
\usepackage{colortbl}
\usepackage{makecell}
\usepackage{multirow}
\usepackage{supertabular}

\begin{document}

\newcounter{utterance}

\twocolumn

{ \footnotesize  \setcounter{utterance}{1}
\setlength{\tabcolsep}{0pt}
\begin{supertabular}{c@{$\;$}|p{.15\linewidth}@{}p{.15\linewidth}p{.15\linewidth}p{.15\linewidth}p{.15\linewidth}p{.15\linewidth}}

    \# & $\;$A & \multicolumn{4}{c}{Game Master} & $\;\:$B\\
    \hline 

    \theutterance \stepcounter{utterance}  

    & & \multicolumn{4}{p{0.6\linewidth}}{\cellcolor[rgb]{0.9,0.9,0.9}{%
	\makecell[{{p{\linewidth}}}]{% 
	  \tt {\tiny [A$\langle$GM]}  
	 You are given three grids, where each of them is 5 by 5 in size.\\ \tt Grids have empty cells marked with "▢" and filled cells marked with "X".\\ \tt Your task is to generate a referring expression that best describes the target grid while distinguishing it from the two other distractor grids.\\ \tt The first grid is the target grid, and the following two grids are the distractors.\\ \tt \\ \tt Target grid:\\ \tt \\ \tt X □ □ □ X\\ \tt X □ □ □ X\\ \tt X X X X X\\ \tt X □ □ □ X\\ \tt X □ □ □ X\\ \tt \\ \tt Distractor grid 1:\\ \tt \\ \tt X X X X X\\ \tt X □ X □ X\\ \tt X X X X X\\ \tt X □ X □ X\\ \tt X X X X X\\ \tt \\ \tt Distractor grid 2:\\ \tt \\ \tt X X X X X\\ \tt X □ □ □ □\\ \tt X X X X X\\ \tt X □ □ □ □\\ \tt X X X X X\\ \tt \\ \tt Instruction: Describe the target grid.\\ \tt Generate the referring expression starting with the tag "Expression: " for the given target grid. Omit any other text. 
	  } 
	   } 
	   } 
	 & \\ 
 

    \theutterance \stepcounter{utterance}  

    & \multicolumn{4}{p{0.6\linewidth}}{\cellcolor[rgb]{0.8,1,0.9}{%
	\makecell[{{p{\linewidth}}}]{% 
	  \tt {\tiny [A$\rangle$GM]}  
	 <think>\\ \tt Okay, let's see. I need to describe the target grid so that it's clear and different from the two distractors. Let me start by looking at each grid carefully.\\ \tt \\ \tt The target grid is a 5x5 grid. Let me visualize it:\\ \tt \\ \tt Row 1: X □ □ □ X\\ \tt Row 2: X □ □ □ X\\ \tt Row 3: X X X X X\\ \tt Row 4: X □ □ □ X\\ \tt Row 5: X □ □ □ X\\ \tt \\ \tt So, the middle row (row 3) is all X's. The other rows have X's at the start and end, with empty cells in between. It looks like a cross shape, maybe? The vertical and horizontal lines meet in the middle.\\ \tt \\ \tt Now, the first distractor grid:\\ \tt \\ \tt Row 1: All X's\\ \tt Row 2: X □ X □ X\\ \tt Row 3: All X's\\ \tt Row 4: X □ X □ X\\ \tt Row 5: All X's\\ \tt \\ \tt This one has a sort of checkerboard pattern in the even rows, with X's in the odd rows. So it's more dense with X's except for some gaps in rows 2 and 4.\\ \tt \\ \tt Second distractor grid:\\ \tt \\ \tt Row 1: All X's\\ \tt Row 2: X □ □ □ □ (only first cell is X)\\ \tt Row 
	  } 
	   } 
	   } 
	 & & \\ 
 

    \theutterance \stepcounter{utterance}  

    & & & \multicolumn{2}{p{0.3\linewidth}}{\cellcolor[rgb]{0.95,0.95,0.95}{%
	\makecell[{{p{\linewidth}}}]{% 
	  \tt {\tiny [GM$|$GM]}  
	 Invalid generated expression 
	  } 
	   } 
	   } 
	 & & \\ 
 

\end{supertabular}
}

\end{document}

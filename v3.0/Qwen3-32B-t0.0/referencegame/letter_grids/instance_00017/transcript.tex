
\documentclass{article}
\usepackage{colortbl}
\usepackage{makecell}
\usepackage{multirow}
\usepackage{supertabular}

\begin{document}

\newcounter{utterance}

\twocolumn

{ \footnotesize  \setcounter{utterance}{1}
\setlength{\tabcolsep}{0pt}
\begin{supertabular}{c@{$\;$}|p{.15\linewidth}@{}p{.15\linewidth}p{.15\linewidth}p{.15\linewidth}p{.15\linewidth}p{.15\linewidth}}

    \# & $\;$A & \multicolumn{4}{c}{Game Master} & $\;\:$B\\
    \hline 

    \theutterance \stepcounter{utterance}  

    & & \multicolumn{4}{p{0.6\linewidth}}{\cellcolor[rgb]{0.9,0.9,0.9}{%
	\makecell[{{p{\linewidth}}}]{% 
	  \tt {\tiny [A$\langle$GM]}  
	 You are given three grids, where each of them is 5 by 5 in size.\\ \tt Grids have empty cells marked with "▢" and filled cells marked with "X".\\ \tt Your task is to generate a referring expression that best describes the target grid while distinguishing it from the two other distractor grids.\\ \tt The first grid is the target grid, and the following two grids are the distractors.\\ \tt \\ \tt Target grid:\\ \tt \\ \tt X X X X X\\ \tt □ □ □ X X\\ \tt □ □ X X □\\ \tt □ X X □ □\\ \tt X X X X X\\ \tt \\ \tt Distractor grid 1:\\ \tt \\ \tt X X X X X\\ \tt X □ X □ X\\ \tt X X X X X\\ \tt X □ X □ X\\ \tt X X X X X\\ \tt \\ \tt Distractor grid 2:\\ \tt \\ \tt X X X X X\\ \tt X □ □ □ □\\ \tt X □ □ □ □\\ \tt X □ □ □ □\\ \tt X X X X X\\ \tt \\ \tt Instruction: Describe the target grid.\\ \tt Generate the referring expression starting with the tag "Expression: " for the given target grid. Omit any other text. 
	  } 
	   } 
	   } 
	 & \\ 
 

    \theutterance \stepcounter{utterance}  

    & \multicolumn{4}{p{0.6\linewidth}}{\cellcolor[rgb]{0.8,1,0.9}{%
	\makecell[{{p{\linewidth}}}]{% 
	  \tt {\tiny [A$\rangle$GM]}  
	 <think>\\ \tt Okay, let's tackle this problem. So, I need to create a referring expression that describes the target grid and distinguishes it from the two distractors. Let me start by looking at all three grids carefully.\\ \tt \\ \tt First, the target grid. Let me visualize it:\\ \tt \\ \tt Row 1: All X's. So five Xs in the first row.\\ \tt Row 2: Two □, then two Xs. Wait, the second row is □ □ □ X X. So three □ followed by two Xs.\\ \tt Row 3: □ □ X X □. So two □, then two Xs, then one □.\\ \tt Row 4: □ X X □ □. One □, two Xs, then two □.\\ \tt Row 5: All Xs again.\\ \tt \\ \tt Now, the distractor grids. Let's look at Distractor 1:\\ \tt \\ \tt Row 1: All Xs.\\ \tt Row 2: X □ X □ X. Alternating X and □ starting with X.\\ \tt Row 3: All Xs.\\ \tt Row 4: X □ X □ X. Same as row 2.\\ \tt Row 5: All Xs.\\ \tt \\ \tt So Distractor 1 has a checkerboard pattern in rows 2 and 4. The target grid's second row is three □ followed by two Xs, which is different from Distractor 1's 
	  } 
	   } 
	   } 
	 & & \\ 
 

    \theutterance \stepcounter{utterance}  

    & & & \multicolumn{2}{p{0.3\linewidth}}{\cellcolor[rgb]{0.95,0.95,0.95}{%
	\makecell[{{p{\linewidth}}}]{% 
	  \tt {\tiny [GM$|$GM]}  
	 Invalid generated expression 
	  } 
	   } 
	   } 
	 & & \\ 
 

\end{supertabular}
}

\end{document}


\documentclass{article}
\usepackage{colortbl}
\usepackage{makecell}
\usepackage{multirow}
\usepackage{supertabular}

\begin{document}

\newcounter{utterance}

\twocolumn

{ \footnotesize  \setcounter{utterance}{1}
\setlength{\tabcolsep}{0pt}
\begin{supertabular}{c@{$\;$}|p{.15\linewidth}@{}p{.15\linewidth}p{.15\linewidth}p{.15\linewidth}p{.15\linewidth}p{.15\linewidth}}

    \# & $\;$A & \multicolumn{4}{c}{Game Master} & $\;\:$B\\
    \hline 

    \theutterance \stepcounter{utterance}  

    & & \multicolumn{4}{p{0.6\linewidth}}{\cellcolor[rgb]{0.9,0.9,0.9}{%
	\makecell[{{p{\linewidth}}}]{% 
	  \tt {\tiny [A$\langle$GM]}  
	 You are given three grids, where each of them is 5 by 5 in size.\\ \tt Grids have empty cells marked with "▢" and filled cells marked with "X".\\ \tt Your task is to generate a referring expression that best describes the target grid while distinguishing it from the two other distractor grids.\\ \tt The first grid is the target grid, and the following two grids are the distractors.\\ \tt \\ \tt Target grid:\\ \tt \\ \tt X X X X □\\ \tt X □ □ □ □\\ \tt X X X X □\\ \tt X □ □ □ X\\ \tt X X X X □\\ \tt \\ \tt Distractor grid 1:\\ \tt \\ \tt X X X X □\\ \tt X □ □ □ □\\ \tt X X X X □\\ \tt □ □ □ X □\\ \tt X X X X □\\ \tt \\ \tt Distractor grid 2:\\ \tt \\ \tt □ X X X X\\ \tt □ X □ □ X\\ \tt □ X X X X\\ \tt □ X □ □ X\\ \tt □ X X X X\\ \tt \\ \tt Instruction: Describe the target grid.\\ \tt Generate the referring expression starting with the tag "Expression: " for the given target grid. Omit any other text. 
	  } 
	   } 
	   } 
	 & \\ 
 

    \theutterance \stepcounter{utterance}  

    & \multicolumn{4}{p{0.6\linewidth}}{\cellcolor[rgb]{0.8,1,0.9}{%
	\makecell[{{p{\linewidth}}}]{% 
	  \tt {\tiny [A$\rangle$GM]}  
	 <think>\\ \tt Okay, let's tackle this problem. I need to create a referring expression that describes the target grid and distinguishes it from the two distractors. Let me start by analyzing each grid carefully.\\ \tt \\ \tt First, the target grid is a 5x5 grid. Let me write it out to visualize better:\\ \tt \\ \tt Row 1: X X X X □\\ \tt Row 2: X □ □ □ □\\ \tt Row 3: X X X X □\\ \tt Row 4: X □ □ □ X\\ \tt Row 5: X X X X □\\ \tt \\ \tt Now, the two distractors:\\ \tt \\ \tt Distractor 1:\\ \tt Row 1: X X X X □\\ \tt Row 2: X □ □ □ □\\ \tt Row 3: X X X X □\\ \tt Row 4: □ □ □ X □\\ \tt Row 5: X X X X □\\ \tt \\ \tt Distractor 2:\\ \tt Row 1: □ X X X X\\ \tt Row 2: □ X □ □ X\\ \tt Row 3: □ X X X X\\ \tt Row 4: □ X □ □ X\\ \tt Row 5: □ X X X X\\ \tt \\ \tt So, the task is to find features in the target grid that are different from both distractors. Let me compare them step by step.\\ \tt \\ \tt Looking at the target grid first. Let's note 
	  } 
	   } 
	   } 
	 & & \\ 
 

    \theutterance \stepcounter{utterance}  

    & & & \multicolumn{2}{p{0.3\linewidth}}{\cellcolor[rgb]{0.95,0.95,0.95}{%
	\makecell[{{p{\linewidth}}}]{% 
	  \tt {\tiny [GM$|$GM]}  
	 Invalid generated expression 
	  } 
	   } 
	   } 
	 & & \\ 
 

\end{supertabular}
}

\end{document}

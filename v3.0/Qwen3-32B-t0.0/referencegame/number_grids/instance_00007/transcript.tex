
\documentclass{article}
\usepackage{colortbl}
\usepackage{makecell}
\usepackage{multirow}
\usepackage{supertabular}

\begin{document}

\newcounter{utterance}

\twocolumn

{ \footnotesize  \setcounter{utterance}{1}
\setlength{\tabcolsep}{0pt}
\begin{supertabular}{c@{$\;$}|p{.15\linewidth}@{}p{.15\linewidth}p{.15\linewidth}p{.15\linewidth}p{.15\linewidth}p{.15\linewidth}}

    \# & $\;$A & \multicolumn{4}{c}{Game Master} & $\;\:$B\\
    \hline 

    \theutterance \stepcounter{utterance}  

    & & \multicolumn{4}{p{0.6\linewidth}}{\cellcolor[rgb]{0.9,0.9,0.9}{%
	\makecell[{{p{\linewidth}}}]{% 
	  \tt {\tiny [A$\langle$GM]}  
	 You are given three grids, where each of them is 5 by 5 in size.\\ \tt Grids have empty cells marked with "▢" and filled cells marked with "X".\\ \tt Your task is to generate a referring expression that best describes the target grid while distinguishing it from the two other distractor grids.\\ \tt The first grid is the target grid, and the following two grids are the distractors.\\ \tt \\ \tt Target grid:\\ \tt \\ \tt X X X X X\\ \tt □ □ □ □ X\\ \tt □ □ □ □ X\\ \tt □ □ □ □ X\\ \tt □ □ □ □ X\\ \tt \\ \tt Distractor grid 1:\\ \tt \\ \tt □ X X X X\\ \tt □ X □ □ X\\ \tt □ X □ □ X\\ \tt □ X □ □ X\\ \tt □ X X X X\\ \tt \\ \tt Distractor grid 2:\\ \tt \\ \tt □ X X X X\\ \tt □ X □ □ X\\ \tt □ X X X X\\ \tt □ X □ □ X\\ \tt □ X X X X\\ \tt \\ \tt Instruction: Describe the target grid.\\ \tt Generate the referring expression starting with the tag "Expression: " for the given target grid. Omit any other text. 
	  } 
	   } 
	   } 
	 & \\ 
 

    \theutterance \stepcounter{utterance}  

    & \multicolumn{4}{p{0.6\linewidth}}{\cellcolor[rgb]{0.8,1,0.9}{%
	\makecell[{{p{\linewidth}}}]{% 
	  \tt {\tiny [A$\rangle$GM]}  
	 <think>\\ \tt Okay, let's see. I need to describe the target grid so that it's clear and different from the two distractors. Let me start by looking at the target grid.\\ \tt \\ \tt The target grid has the first row all filled with X's. Then, the next four rows have four empty cells (▢) in the first four columns and an X in the fifth column. So, the first row is all Xs, and the rest of the rows have X only in the last column.\\ \tt \\ \tt Now, looking at Distractor 1. Its first row starts with a ▢ followed by four Xs. Then, in the second to fifth rows, there's an X in the second column, then some ▢s and an X in the fifth. The pattern here is a vertical line of Xs in column 2 from row 2 to 5, plus some other Xs in the first row and others. \\ \tt \\ \tt Distractor 2 is similar but in row 3, column 2 to 4 are Xs. So, in row 3, there's a horizontal line of Xs from column 2 to 4. The other rows have similar patterns to Distractor 1 but with that difference in row 3.\\ \tt \\ \tt So, the target grid's unique feature is the first row all Xs. Both distractors have ▢ in the first cell of the first row. Also, in the target, after the first row, only 
	  } 
	   } 
	   } 
	 & & \\ 
 

    \theutterance \stepcounter{utterance}  

    & & & \multicolumn{2}{p{0.3\linewidth}}{\cellcolor[rgb]{0.95,0.95,0.95}{%
	\makecell[{{p{\linewidth}}}]{% 
	  \tt {\tiny [GM$|$GM]}  
	 Invalid generated expression 
	  } 
	   } 
	   } 
	 & & \\ 
 

\end{supertabular}
}

\end{document}

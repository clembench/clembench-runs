
\documentclass{article}
\usepackage{colortbl}
\usepackage{makecell}
\usepackage{multirow}
\usepackage{supertabular}

\begin{document}

\newcounter{utterance}

\twocolumn

{ \footnotesize  \setcounter{utterance}{1}
\setlength{\tabcolsep}{0pt}
\begin{supertabular}{c@{$\;$}|p{.15\linewidth}@{}p{.15\linewidth}p{.15\linewidth}p{.15\linewidth}p{.15\linewidth}p{.15\linewidth}}

    \# & $\;$A & \multicolumn{4}{c}{Game Master} & $\;\:$B\\
    \hline 

    \theutterance \stepcounter{utterance}  

    & & \multicolumn{4}{p{0.6\linewidth}}{\cellcolor[rgb]{0.9,0.9,0.9}{%
	\makecell[{{p{\linewidth}}}]{% 
	  \tt {\tiny [A$\langle$GM]}  
	 You are a language wizard who likes to guess words by using the given rules.\\ \tt \\ \tt Welcome to Wordle! You have six attempts to guess the target word, a valid English word of five lowercase letters (a-z). Please use the tags "explanation:" and "guess:" to provide a concise explanation for each guess.\\ \tt \\ \tt To help you make an informed guess, you will receive a clue for the word, such as\\ \tt clue: snowy white.\\ \tt \\ \tt Here is an example guess based on the clue:\\ \tt explanation: In the fairy tail Snow White, the girl is killed because she eats a poisoned apple. And the word apple has 5 letters.\\ \tt guess: apple\\ \tt \\ \tt After each guess, your answer will be validated, and you will receive feedback indicating which letters are correct (green), which letters are correct but in the wrong position (yellow), and which letters are incorrect (red). This feedback can be useful in determining which letters to include or exclude in your next guess.\\ \tt \\ \tt For example, the feedback for "apple" might be:\\ \tt guess_feedback: a<yellow> p<yellow> p<green> l<yellow> e<red>\\ \tt \\ \tt The explanation should contain details about how both the guess_feedback and clue work together in tandem to formulate a new guess.\\ \tt \\ \tt Do not generate any other text.\\ \tt \\ \tt Let's begin with your first guess.\\ \tt \\ \tt clue: winter wear 
	  } 
	   } 
	   } 
	 & \\ 
 

    \theutterance \stepcounter{utterance}  

    & \multicolumn{4}{p{0.6\linewidth}}{\cellcolor[rgb]{0.8,1,0.9}{%
	\makecell[{{p{\linewidth}}}]{% 
	  \tt {\tiny [A$\rangle$GM]}  
	 <think>\\ \tt Okay, let's see. The clue is "winter wear," so I need to think of a five-letter word related to clothing worn in winter. Common winter clothes include items like coats, hats, scarves, gloves, boots, etc. Let me brainstorm some five-letter words.\\ \tt \\ \tt First, "jacket" is six letters, so that's out. "Coat" is four letters. Maybe "boots"? That's five letters. But is "boots" considered winter wear? Yes, especially snow boots. But let me check if there's a more specific term. "Gloves" is six letters. "Mittens" is seven. "Scarf" is five letters. That's a good candidate. "Scarf" is definitely a winter item. Another possibility is "hat," but that's three letters. "Hoodie" is six letters. "Sweater" is seven. Hmm. Maybe "boots" is a solid guess. Let me confirm the letters. B-O-O-T-S. Wait, the word needs to be all lowercase and five letters. "Boots" is correct. Alternatively, "scarf" is another option. Let me think which is more likely. The clue is "winter wear," so both are possible. Maybe start with "boots" as it's a common winter footwear. Let's go with that for the first guess.\\ \tt </think>\\ \tt \\ \tt explanation: Winter wear often includes items like 
	  } 
	   } 
	   } 
	 & & \\ 
 

    \theutterance \stepcounter{utterance}  

    & & & \multicolumn{2}{p{0.3\linewidth}}{\cellcolor[rgb]{0.95,0.95,0.95}{%
	\makecell[{{p{\linewidth}}}]{% 
	  \tt {\tiny [GM$|$GM]}  
	 Error: The response should always start with the keyword 'explanation:' 
	  } 
	   } 
	   } 
	 & & \\ 
 

    \theutterance \stepcounter{utterance}  

    & & & \multicolumn{2}{p{0.3\linewidth}}{\cellcolor[rgb]{0.95,0.95,0.95}{%
	\makecell[{{p{\linewidth}}}]{% 
	  \tt {\tiny [GM$|$GM]}  
	 game_result = ABORT 
	  } 
	   } 
	   } 
	 & & \\ 
 

\end{supertabular}
}

\end{document}


\documentclass{article}
\usepackage{colortbl}
\usepackage{makecell}
\usepackage{multirow}
\usepackage{supertabular}

\begin{document}

\newcounter{utterance}

\twocolumn

{ \footnotesize  \setcounter{utterance}{1}
\setlength{\tabcolsep}{0pt}
\begin{supertabular}{c@{$\;$}|p{.15\linewidth}@{}p{.15\linewidth}p{.15\linewidth}p{.15\linewidth}p{.15\linewidth}p{.15\linewidth}}

    \# & $\;$A & \multicolumn{4}{c}{Game Master} & $\;\:$B\\
    \hline 

    \theutterance \stepcounter{utterance}  

    & & \multicolumn{4}{p{0.6\linewidth}}{\cellcolor[rgb]{0.9,0.9,0.9}{%
	\makecell[{{p{\linewidth}}}]{% 
	  \tt {\tiny [A$\langle$GM]}  
	 You are playing a text adventure game. I will describe what you can perceive in the game. You write the single action you want to take in the game starting with >. Only reply with an action.\\ \tt For example:\\ \tt > open cupboard\\ \tt \\ \tt In addition to common actions, you can maten. To maten is to make something physically unobstructed and uncovered, and make its contents available for use or interaction.\\ \tt \\ \tt Your goal for this game is: Put the uness on the table, the potted plant on the cally and the peach on the cally.\\ \tt \\ \tt Once you have achieved your goal, write "> done" to end the game.\\ \tt \\ \tt You are in a bedroom now. There are a pillow, a wardrobe, a book and a bed. The pillow is on the bed. The wardrobe is closed. The book is on the floor. There is a passage to a hallway here. 
	  } 
	   } 
	   } 
	 & \\ 
 

    \theutterance \stepcounter{utterance}  

    & \multicolumn{4}{p{0.6\linewidth}}{\cellcolor[rgb]{0.8,1,0.9}{%
	\makecell[{{p{\linewidth}}}]{% 
	  \tt {\tiny [A$\rangle$GM]}  
	 <think>\\ \tt Okay, let's see. The user is playing a text adventure game and needs to put three items in specific places: the uness on the table, the potted plant on the cally, and the peach on the cally. First, I need to figure out where these items are and what actions are needed.\\ \tt \\ \tt The current location is a bedroom with a pillow on the bed, a closed wardrobe, a book on the floor, and a passage to a hallway. The goal items aren't listed here. The user might need to go to the hallway first to find the table, cally, and the items. But wait, the items to move (uness, potted plant, peach) aren't mentioned in the bedroom. So maybe they're in the hallway or other rooms connected through the passage. \\ \tt \\ \tt Since the wardrobe is closed, maybe opening it could reveal some items, but the goal items aren't mentioned there. The book is on the floor; perhaps reading it or interacting with it could help, but the goal doesn't mention the book. The user's first step should probably be to go through the passage to the hallway to find the required items and locations. \\ \tt \\ \tt So the first action would be to go to the hallway. Let me check if there's any other immediate action. The wardrobe is closed, but opening it might not help right now. The book is on the floor, but again, not part of the goal. The passage to the hallway is 
	  } 
	   } 
	   } 
	 & & \\ 
 

    \theutterance \stepcounter{utterance}  

    & & & \multicolumn{2}{p{0.3\linewidth}}{\cellcolor[rgb]{0.95,0.95,0.95}{%
	\makecell[{{p{\linewidth}}}]{% 
	  \tt {\tiny [GM$|$GM]}  
	 Round: 0 
	  } 
	   } 
	   } 
	 & & \\ 
 

    \theutterance \stepcounter{utterance}  

    & & & \multicolumn{2}{p{0.3\linewidth}}{\cellcolor[rgb]{0.95,0.95,0.95}{%
	\makecell[{{p{\linewidth}}}]{% 
	  \tt {\tiny [GM$|$GM]}  
	 command_tag_missing 
	  } 
	   } 
	   } 
	 & & \\ 
 

    \theutterance \stepcounter{utterance}  

    & & & \multicolumn{2}{p{0.3\linewidth}}{\cellcolor[rgb]{0.95,0.95,0.95}{%
	\makecell[{{p{\linewidth}}}]{% 
	  \tt {\tiny [GM$|$GM]}  
	 {'goal_states_achieved': [], 'game_successfully_finished': False} 
	  } 
	   } 
	   } 
	 & & \\ 
 

\end{supertabular}
}

\end{document}

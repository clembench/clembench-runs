
\documentclass{article}
\usepackage{colortbl}
\usepackage{makecell}
\usepackage{multirow}
\usepackage{supertabular}

\begin{document}

\newcounter{utterance}

\twocolumn

{ \footnotesize  \setcounter{utterance}{1}
\setlength{\tabcolsep}{0pt}
\begin{supertabular}{c@{$\;$}|p{.15\linewidth}@{}p{.15\linewidth}p{.15\linewidth}p{.15\linewidth}p{.15\linewidth}p{.15\linewidth}}

    \# & $\;$A & \multicolumn{4}{c}{Game Master} & $\;\:$B\\
    \hline 

    \theutterance \stepcounter{utterance}  

    & & \multicolumn{4}{p{0.6\linewidth}}{\cellcolor[rgb]{0.9,0.9,0.9}{%
	\makecell[{{p{\linewidth}}}]{% 
	  \tt {\tiny [A$\langle$GM]}  
	 You are playing a text adventure game. I will describe what you can perceive in the game. You write the single action you want to take in the game starting with >. Only reply with an action.\\ \tt For example:\\ \tt > open cupboard\\ \tt \\ \tt In addition to common actions, you can udial, aterous, ingnus, ivess, ceorid and unmastic.\\ \tt \\ \tt Your goal for this game is: Make the pholoiced connone, the aeval proggiced and the rheably exsem.\\ \tt \\ \tt Once you have achieved your goal, write "> done" to end the game.\\ \tt \\ \tt You are in a peroury now. There are a shromen, a oidess, a eical, a versch, a pholoiced and a aeval. There is a passage to a erove here. 
	  } 
	   } 
	   } 
	 & \\ 
 

    \theutterance \stepcounter{utterance}  

    & \multicolumn{4}{p{0.6\linewidth}}{\cellcolor[rgb]{0.8,1,0.9}{%
	\makecell[{{p{\linewidth}}}]{% 
	  \tt {\tiny [A$\rangle$GM]}  
	 <think>\\ \tt Okay, let's see. I need to make the pholoiced connone, the aeval proggiced, and the rheably exsem. Hmm. First, I should figure out what these terms mean. Maybe they're items or actions in the game. The current location is a peroury, and there are several objects here: shromen, oidess, eical, versch, pholoiced, and aeval. There's also a passage to an erove.\\ \tt \\ \tt The goal mentions making the pholoiced connone. Since there's already a pholoiced here, maybe I need to combine it with something else to make it connone. Similarly, the aeval proggiced—there's an aeval present, so perhaps I need to do something to it. The rheably exsem is the third part; not sure what that is yet.\\ \tt \\ \tt Possible actions include common ones like open, take, use, but also some unique verbs: udial, aterous, ingnus, ivess, ceorid, unmastic. Maybe these verbs are key to achieving the goal. Let me think. For example, if I need to make the pholoiced connone, maybe using one of these verbs on it. Let's check the objects:\\ \tt \\ \tt - pholoiced is here. Maybe I need to use a verb on it. Let's try "udial pholoiced" or something. But I don't know what 
	  } 
	   } 
	   } 
	 & & \\ 
 

    \theutterance \stepcounter{utterance}  

    & & & \multicolumn{2}{p{0.3\linewidth}}{\cellcolor[rgb]{0.95,0.95,0.95}{%
	\makecell[{{p{\linewidth}}}]{% 
	  \tt {\tiny [GM$|$GM]}  
	 Round: 0 
	  } 
	   } 
	   } 
	 & & \\ 
 

    \theutterance \stepcounter{utterance}  

    & & & \multicolumn{2}{p{0.3\linewidth}}{\cellcolor[rgb]{0.95,0.95,0.95}{%
	\makecell[{{p{\linewidth}}}]{% 
	  \tt {\tiny [GM$|$GM]}  
	 command_tag_missing 
	  } 
	   } 
	   } 
	 & & \\ 
 

    \theutterance \stepcounter{utterance}  

    & & & \multicolumn{2}{p{0.3\linewidth}}{\cellcolor[rgb]{0.95,0.95,0.95}{%
	\makecell[{{p{\linewidth}}}]{% 
	  \tt {\tiny [GM$|$GM]}  
	 {'goal_states_achieved': [], 'game_successfully_finished': False} 
	  } 
	   } 
	   } 
	 & & \\ 
 

\end{supertabular}
}

\end{document}

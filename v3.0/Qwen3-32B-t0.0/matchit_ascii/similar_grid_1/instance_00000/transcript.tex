
\documentclass{article}
\usepackage{colortbl}
\usepackage{makecell}
\usepackage{multirow}
\usepackage{supertabular}

\begin{document}

\newcounter{utterance}

\twocolumn

{ \footnotesize  \setcounter{utterance}{1}
\setlength{\tabcolsep}{0pt}
\begin{supertabular}{c@{$\;$}|p{.15\linewidth}@{}p{.15\linewidth}p{.15\linewidth}p{.15\linewidth}p{.15\linewidth}p{.15\linewidth}}

    \# & $\;$A & \multicolumn{4}{c}{Game Master} & $\;\:$B\\
    \hline 

    \theutterance \stepcounter{utterance}  

    & & \multicolumn{4}{p{0.6\linewidth}}{\cellcolor[rgb]{0.9,0.9,0.9}{%
	\makecell[{{p{\linewidth}}}]{% 
	  \tt {\tiny [A$\langle$GM]}  
	 You are participating in a collaborative guessing game.\\ \tt The goal is to find out whether this grid and another picture only I can see, are the same.\\ \tt Please describe your grid first. Then, I will provide my description and we can ask each other questions about the grids to figure out whether they are the same. Now start your short grid description with "DESCRIPTION:" followed by the description. Do not add anything else. This is your grid:\\ \tt X X X X ▢\\ \tt X X X ▢ X\\ \tt X X ▢ X X\\ \tt X ▢ X X X\\ \tt ▢ X X X X 
	  } 
	   } 
	   } 
	 & \\ 
 

    \theutterance \stepcounter{utterance}  

    & \multicolumn{4}{p{0.6\linewidth}}{\cellcolor[rgb]{0.8,1,0.9}{%
	\makecell[{{p{\linewidth}}}]{% 
	  \tt {\tiny [A$\rangle$GM]}  
	 <think>\\ \tt Okay, let's see. The user wants me to describe my grid first. The grid they provided is a 5x5 grid with X's and empty squares (▢). Let me visualize it again.\\ \tt \\ \tt Row 1: X X X X ▢\\ \tt Row 2: X X X ▢ X\\ \tt Row 3: X X ▢ X X\\ \tt Row 4: X ▢ X X X\\ \tt Row 5: ▢ X X X X\\ \tt \\ \tt Hmm, I need to describe this clearly. Maybe note the positions of the empty squares. Let's count the rows and columns from 1 to 5. The empty squares are at positions (1,5), (2,4), (3,3), (4,2), and (5,1). So each row has one empty square, and they form a diagonal from top-right to bottom-left. Wait, is that right? Let me check again.\\ \tt \\ \tt Row 1: The empty is at column 5 (last position). Row 2: column 4. Row 3: column 3. Row 4: column 2. Row 5: column 1. Yes, that's a diagonal from (1,5) to (5,1). So the empty squares are along the anti-diagonal. The rest are X's. So the description should mention that each row has one empty square, and their positions form a diagonal from top-right to bottom 
	  } 
	   } 
	   } 
	 & & \\ 
 

    \theutterance \stepcounter{utterance}  

    & & & \multicolumn{2}{p{0.3\linewidth}}{\cellcolor[rgb]{0.95,0.95,0.95}{%
	\makecell[{{p{\linewidth}}}]{% 
	  \tt {\tiny [GM$|$GM]}  
	 abort, first word: <think>, but expected DESCRIPTION: 
	  } 
	   } 
	   } 
	 & & \\ 
 

    \theutterance \stepcounter{utterance}  

    & & & \multicolumn{2}{p{0.3\linewidth}}{\cellcolor[rgb]{0.95,0.95,0.95}{%
	\makecell[{{p{\linewidth}}}]{% 
	  \tt {\tiny [GM$|$GM]}  
	 Aborted 
	  } 
	   } 
	   } 
	 & & \\ 
 

\end{supertabular}
}

\end{document}

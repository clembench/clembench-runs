
\documentclass{article}
\usepackage{colortbl}
\usepackage{makecell}
\usepackage{multirow}
\usepackage{supertabular}

\begin{document}

\newcounter{utterance}

\twocolumn

{ \footnotesize  \setcounter{utterance}{1}
\setlength{\tabcolsep}{0pt}
\begin{supertabular}{c@{$\;$}|p{.15\linewidth}@{}p{.15\linewidth}p{.15\linewidth}p{.15\linewidth}p{.15\linewidth}p{.15\linewidth}}

    \# & $\;$A & \multicolumn{4}{c}{Game Master} & $\;\:$B\\
    \hline 

    \theutterance \stepcounter{utterance}  

    & & & \multicolumn{2}{p{0.3\linewidth}}{\cellcolor[rgb]{0.95,0.95,0.95}{%
	\makecell[{{p{\linewidth}}}]{% 
	  \tt {\tiny [GM$|$GM]}  
	 {'hidden': {'team': ['shirtfront', 'astrobiology', 'underclothing', 'tinsmith', 'housewifery', 'tragicomedy', 'gingersnap', 'demobilization', 'townswoman'], 'innocent': ['girlishness', 'horsewoman', 'pleader', 'catheterization', 'dualism', 'unacceptability', 'insurgence'], 'opponent': ['motionlessness', 'scheduler', 'newsagent', 'sexology', 'toxicologist', 'dwarfism', 'ducktail', 'neuropsychologist'], 'assassin': ['sideway']}, 'revealed': {'team': {'team': [], 'innocent': [], 'opponent': [], 'assassin': []}, 'opponent': {'team': [], 'innocent': [], 'opponent': [], 'assassin': []}}} 
	  } 
	   } 
	   } 
	 & & \\ 
 

    \theutterance \stepcounter{utterance}  

    & & \multicolumn{4}{p{0.6\linewidth}}{\cellcolor[rgb]{0.9,0.9,0.9}{%
	\makecell[{{p{\linewidth}}}]{% 
	  \tt {\tiny [A$\langle$GM]}  
	 Provide one single-word clue that relates to some of your team's words from the list below. You can choose to how many and to which words your clue relates to.\\ \tt The goal is to have your partner guess all of your team's words based on the clues you give them, before the other team has uncovered all of theirs. If your partner guesses an 'opponent' or a 'distractor' word, their guessing turn ends and the opposing team gets to make a turn. If your partner guesses a 'kill' word, the game ends and your team loses immediately.\\ \tt The more words your clue relates to, the quicker you can win the game, but the harder the associations may be to guess for your partner, so choose your clue word wisely. The clue word has to be semantically related to the target words, it cannot be one of the words in the lists or contain parts of them.\\ \tt \\ \tt Always give your single-word clue and your comma-separated list of related target words in the following format and make your answers as short as possible, never include any other text than is required in this form:\\ \tt \\ \tt CLUE: <WORD>\\ \tt TARGETS: <TARGETS>\\ \tt \\ \tt Your team words are: shirtfront, astrobiology, underclothing, tinsmith, housewifery, tragicomedy, gingersnap, demobilization, townswoman.\\ \tt Your opponent's team words are: motionlessness, scheduler, newsagent, sexology, toxicologist, dwarfism, ducktail, neuropsychologist.\\ \tt Distractor words are: girlishness, horsewoman, pleader, catheterization, dualism, unacceptability, insurgence.\\ \tt The kill words are: sideway. 
	  } 
	   } 
	   } 
	 & \\ 
 

    \theutterance \stepcounter{utterance}  

    & \multicolumn{4}{p{0.6\linewidth}}{\cellcolor[rgb]{0.8,1,0.9}{%
	\makecell[{{p{\linewidth}}}]{% 
	  \tt {\tiny [A$\rangle$GM]}  
	 <think>\\ \tt Okay, let's see. I need to come up with a single-word clue that relates to some of my team's words. The team words are shirtfront, astrobiology, underclothing, tinsmith, housewifery, tragicomedy, gingersnap, demobilization, townswoman. The opponent's words and distractors are listed too, and there's a kill word.\\ \tt \\ \tt First, I should look for common themes or connections among the team's words. Let's go through each one:\\ \tt \\ \tt - Shirtfront: part of a shirt, maybe related to clothing or fashion.\\ \tt - Astrobiology: study of life in space, science-related.\\ \tt - Underclothing: again, clothing, like underwear.\\ \tt - Tinsmith: someone who works with tin, a trade or occupation.\\ \tt - Housewifery: related to being a housewife, domestic work.\\ \tt - Tragicomedy: a mix of tragedy and comedy, a genre in literature or theater.\\ \tt - Gingersnap: a type of cookie, maybe related to baking or spices.\\ \tt - Demobilization: the process of disbanding an army or returning to peacetime, military term.\\ \tt - Townswoman: a woman living in a town, related to community or urban life.\\ \tt \\ \tt Looking for overlaps or common themes. Clothing terms: shirtfront, underclothing. Occupations: tinsmith, housewifery, townswoman. Maybe "occupation" could be a 
	  } 
	   } 
	   } 
	 & & \\ 
 

    \theutterance \stepcounter{utterance}  

    & & & \multicolumn{2}{p{0.3\linewidth}}{\cellcolor[rgb]{0.95,0.95,0.95}{%
	\makecell[{{p{\linewidth}}}]{% 
	  \tt {\tiny [GM$|$GM]}  
	 {'player': 'Cluegiver', 'type': 'rambling error', 'utterance': '<think>\nOkay, let\'s see. I need to come up with a single-word clue that relates to some of my team\'s words. The team words are shirtfront, astrobiology, underclothing, tinsmith, housewifery, tragicomedy, gingersnap, demobilization, townswoman. The opponent\'s words and distractors are listed too, and there\'s a kill word.\n\nFirst, I should look for common themes or connections among the team\'s words. Let\'s go through each one:\n\n- Shirtfront: part of a shirt, maybe related to clothing or fashion.\n- Astrobiology: study of life in space, science-related.\n- Underclothing: again, clothing, like underwear.\n- Tinsmith: someone who works with tin, a trade or occupation.\n- Housewifery: related to being a housewife, domestic work.\n- Tragicomedy: a mix of tragedy and comedy, a genre in literature or theater.\n- Gingersnap: a type of cookie, maybe related to baking or spices.\n- Demobilization: the process of disbanding an army or returning to peacetime, military term.\n- Townswoman: a woman living in a town, related to community or urban life.\n\nLooking for overlaps or common themes. Clothing terms: shirtfront, underclothing. Occupations: tinsmith, housewifery, townswoman. Maybe "occupation" could be a', 'message': 'Your answer contained more than two lines, please only give one clue and your targets on two separate lines.'} 
	  } 
	   } 
	   } 
	 & & \\ 
 

\end{supertabular}
}

\end{document}

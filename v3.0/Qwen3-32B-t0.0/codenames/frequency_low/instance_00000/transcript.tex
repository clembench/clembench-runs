
\documentclass{article}
\usepackage{colortbl}
\usepackage{makecell}
\usepackage{multirow}
\usepackage{supertabular}

\begin{document}

\newcounter{utterance}

\twocolumn

{ \footnotesize  \setcounter{utterance}{1}
\setlength{\tabcolsep}{0pt}
\begin{supertabular}{c@{$\;$}|p{.15\linewidth}@{}p{.15\linewidth}p{.15\linewidth}p{.15\linewidth}p{.15\linewidth}p{.15\linewidth}}

    \# & $\;$A & \multicolumn{4}{c}{Game Master} & $\;\:$B\\
    \hline 

    \theutterance \stepcounter{utterance}  

    & & & \multicolumn{2}{p{0.3\linewidth}}{\cellcolor[rgb]{0.95,0.95,0.95}{%
	\makecell[{{p{\linewidth}}}]{% 
	  \tt {\tiny [GM$|$GM]}  
	 {'hidden': {'team': ['elliptic', 'bashfulness', 'darer', 'dwarfism', 'dabbler', 'candleholder', 'cancelation', 'balsawood', 'banquette'], 'innocent': ['ungodliness', 'astrobiology', 'gyration', 'duckweed', 'dangerousness', 'fossilization', 'shuffler'], 'opponent': ['emotionality', 'nextdoor', 'photoelectron', 'backwardness', 'demobilization', 'serialization', 'undernourishment', 'nearsightedness'], 'assassin': ['caretaking']}, 'revealed': {'team': {'team': [], 'innocent': [], 'opponent': [], 'assassin': []}, 'opponent': {'team': [], 'innocent': [], 'opponent': [], 'assassin': []}}} 
	  } 
	   } 
	   } 
	 & & \\ 
 

    \theutterance \stepcounter{utterance}  

    & & \multicolumn{4}{p{0.6\linewidth}}{\cellcolor[rgb]{0.9,0.9,0.9}{%
	\makecell[{{p{\linewidth}}}]{% 
	  \tt {\tiny [A$\langle$GM]}  
	 Provide one single-word clue that relates to some of your team's words from the list below. You can choose to how many and to which words your clue relates to.\\ \tt The goal is to have your partner guess all of your team's words based on the clues you give them, before the other team has uncovered all of theirs. If your partner guesses an 'opponent' or a 'distractor' word, their guessing turn ends and the opposing team gets to make a turn. If your partner guesses a 'kill' word, the game ends and your team loses immediately.\\ \tt The more words your clue relates to, the quicker you can win the game, but the harder the associations may be to guess for your partner, so choose your clue word wisely. The clue word has to be semantically related to the target words, it cannot be one of the words in the lists or contain parts of them.\\ \tt \\ \tt Always give your single-word clue and your comma-separated list of related target words in the following format and make your answers as short as possible, never include any other text than is required in this form:\\ \tt \\ \tt CLUE: <WORD>\\ \tt TARGETS: <TARGETS>\\ \tt \\ \tt Your team words are: elliptic, bashfulness, darer, dwarfism, dabbler, candleholder, cancelation, balsawood, banquette.\\ \tt Your opponent's team words are: emotionality, nextdoor, photoelectron, backwardness, demobilization, serialization, undernourishment, nearsightedness.\\ \tt Distractor words are: ungodliness, astrobiology, gyration, duckweed, dangerousness, fossilization, shuffler.\\ \tt The kill words are: caretaking. 
	  } 
	   } 
	   } 
	 & \\ 
 

    \theutterance \stepcounter{utterance}  

    & \multicolumn{4}{p{0.6\linewidth}}{\cellcolor[rgb]{0.8,1,0.9}{%
	\makecell[{{p{\linewidth}}}]{% 
	  \tt {\tiny [A$\rangle$GM]}  
	 <think>\\ \tt Okay, let's see. I need to come up with a single-word clue that relates to some of my team's words. The team words are elliptic, bashfulness, darer, dwarfism, dabbler, candleholder, cancelation, balsawood, banquette. The opponent's words and distractors are other things, and there's a kill word to avoid.\\ \tt \\ \tt First, I should look for common themes or connections among the team's words. Let's go through each one:\\ \tt \\ \tt Elliptic: relates to ellipses, maybe geometry or shapes.\\ \tt Bashfulness: shyness, being timid.\\ \tt Darer: someone who dares others, maybe related to courage or challenge.\\ \tt Dwarfism: a condition affecting growth, short stature.\\ \tt Dabbler: someone who does something casually, not an expert.\\ \tt Candleholder: holds candles, maybe related to lighting or home items.\\ \tt Cancelation: the act of canceling something.\\ \tt Balsawood: a type of lightweight wood, used in models or crafts.\\ \tt Banquette: a type of seating, like in a restaurant, U-shaped.\\ \tt \\ \tt Looking for possible connections. Maybe some words start with 'd' or 'b'? But that's too obvious and might be part of the words, which isn't allowed. The clue can't be part of the words.\\ \tt \\ \tt Another angle: maybe some of these words relate to being small or short? Dwarfism and balsawood (light 
	  } 
	   } 
	   } 
	 & & \\ 
 

    \theutterance \stepcounter{utterance}  

    & & & \multicolumn{2}{p{0.3\linewidth}}{\cellcolor[rgb]{0.95,0.95,0.95}{%
	\makecell[{{p{\linewidth}}}]{% 
	  \tt {\tiny [GM$|$GM]}  
	 {'player': 'Cluegiver', 'type': 'rambling error', 'utterance': "<think>\nOkay, let's see. I need to come up with a single-word clue that relates to some of my team's words. The team words are elliptic, bashfulness, darer, dwarfism, dabbler, candleholder, cancelation, balsawood, banquette. The opponent's words and distractors are other things, and there's a kill word to avoid.\n\nFirst, I should look for common themes or connections among the team's words. Let's go through each one:\n\nElliptic: relates to ellipses, maybe geometry or shapes.\nBashfulness: shyness, being timid.\nDarer: someone who dares others, maybe related to courage or challenge.\nDwarfism: a condition affecting growth, short stature.\nDabbler: someone who does something casually, not an expert.\nCandleholder: holds candles, maybe related to lighting or home items.\nCancelation: the act of canceling something.\nBalsawood: a type of lightweight wood, used in models or crafts.\nBanquette: a type of seating, like in a restaurant, U-shaped.\n\nLooking for possible connections. Maybe some words start with 'd' or 'b'? But that's too obvious and might be part of the words, which isn't allowed. The clue can't be part of the words.\n\nAnother angle: maybe some of these words relate to being small or short? Dwarfism and balsawood (light", 'message': 'Your answer contained more than two lines, please only give one clue and your targets on two separate lines.'} 
	  } 
	   } 
	   } 
	 & & \\ 
 

\end{supertabular}
}

\end{document}

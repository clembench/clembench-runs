
\documentclass{article}
\usepackage{colortbl}
\usepackage{makecell}
\usepackage{multirow}
\usepackage{supertabular}

\begin{document}

\newcounter{utterance}

\twocolumn

{ \footnotesize  \setcounter{utterance}{1}
\setlength{\tabcolsep}{0pt}
\begin{supertabular}{c@{$\;$}|p{.15\linewidth}@{}p{.15\linewidth}p{.15\linewidth}p{.15\linewidth}p{.15\linewidth}p{.15\linewidth}}

    \# & $\;$A & \multicolumn{4}{c}{Game Master} & $\;\:$B\\
    \hline 

    \theutterance \stepcounter{utterance}  

    & & & \multicolumn{2}{p{0.3\linewidth}}{\cellcolor[rgb]{0.95,0.95,0.95}{%
	\makecell[{{p{\linewidth}}}]{% 
	  \tt {\tiny [GM$|$GM]}  
	 {'hidden': {'team': ['eggroll', 'scrubwoman', 'plantain', 'banquette', 'muddiness', 'unchallenging', 'moviemaker', 'hypersensitivity', 'typewrite'], 'innocent': ['bandmaster', 'ladylove', 'customizer', 'defenseman', 'narrowness', 'hotelkeeper', 'neckband'], 'opponent': ['hurdler', 'assassinator', 'emotionality', 'liquidness', 'bagpiper', 'deinstitutionalization', 'housewifery', 'perpetration'], 'assassin': ['seacoast']}, 'revealed': {'team': {'team': [], 'innocent': [], 'opponent': [], 'assassin': []}, 'opponent': {'team': [], 'innocent': [], 'opponent': [], 'assassin': []}}} 
	  } 
	   } 
	   } 
	 & & \\ 
 

    \theutterance \stepcounter{utterance}  

    & & \multicolumn{4}{p{0.6\linewidth}}{\cellcolor[rgb]{0.9,0.9,0.9}{%
	\makecell[{{p{\linewidth}}}]{% 
	  \tt {\tiny [A$\langle$GM]}  
	 Provide one single-word clue that relates to some of your team's words from the list below. You can choose to how many and to which words your clue relates to.\\ \tt The goal is to have your partner guess all of your team's words based on the clues you give them, before the other team has uncovered all of theirs. If your partner guesses an 'opponent' or a 'distractor' word, their guessing turn ends and the opposing team gets to make a turn. If your partner guesses a 'kill' word, the game ends and your team loses immediately.\\ \tt The more words your clue relates to, the quicker you can win the game, but the harder the associations may be to guess for your partner, so choose your clue word wisely. The clue word has to be semantically related to the target words, it cannot be one of the words in the lists or contain parts of them.\\ \tt \\ \tt Always give your single-word clue and your comma-separated list of related target words in the following format and make your answers as short as possible, never include any other text than is required in this form:\\ \tt \\ \tt CLUE: <WORD>\\ \tt TARGETS: <TARGETS>\\ \tt \\ \tt Your team words are: eggroll, scrubwoman, plantain, banquette, muddiness, unchallenging, moviemaker, hypersensitivity, typewrite.\\ \tt Your opponent's team words are: hurdler, assassinator, emotionality, liquidness, bagpiper, deinstitutionalization, housewifery, perpetration.\\ \tt Distractor words are: bandmaster, ladylove, customizer, defenseman, narrowness, hotelkeeper, neckband.\\ \tt The kill words are: seacoast. 
	  } 
	   } 
	   } 
	 & \\ 
 

    \theutterance \stepcounter{utterance}  

    & \multicolumn{4}{p{0.6\linewidth}}{\cellcolor[rgb]{0.8,1,0.9}{%
	\makecell[{{p{\linewidth}}}]{% 
	  \tt {\tiny [A$\rangle$GM]}  
	 <think>\\ \tt Okay, let's see. I need to come up with a single-word clue that relates to some of my team's words. The team words are eggroll, scrubwoman, plantain, banquette, muddiness, unchallenging, moviemaker, hypersensitivity, typewrite. The opponent's words and distractors are listed too, and there's a kill word.\\ \tt \\ \tt First, I should look for connections among the team's words. Let me list them again to see if any patterns emerge. Eggroll is a food, maybe Asian. Scrubwoman could be related to cleaning. Plantain is another food, but different from banana. Banquette is a type of seating, like in a restaurant. Muddiness is about being unclear or dirty. Unchallenging means easy. Moviemaker is someone who makes movies. Hypersensitivity is a medical term. Typewrite is related to typing.\\ \tt \\ \tt Hmm. Maybe some of these are related to food? Eggroll and plantain are both foods. But scrubwoman and moviemaker don't fit. Maybe "food" isn't the right clue. Let's think of other categories. \\ \tt \\ \tt Scrubwoman and typewrite could relate to work or professions. But moviemaker is also a profession. Maybe "occupation"? But then banquette is a piece of furniture. Not sure. \\ \tt \\ \tt Muddiness and hypersensitivity both relate to senses or conditions. Maybe "condition"? But that 
	  } 
	   } 
	   } 
	 & & \\ 
 

    \theutterance \stepcounter{utterance}  

    & & & \multicolumn{2}{p{0.3\linewidth}}{\cellcolor[rgb]{0.95,0.95,0.95}{%
	\makecell[{{p{\linewidth}}}]{% 
	  \tt {\tiny [GM$|$GM]}  
	 {'player': 'Cluegiver', 'type': 'rambling error', 'utterance': '<think>\nOkay, let\'s see. I need to come up with a single-word clue that relates to some of my team\'s words. The team words are eggroll, scrubwoman, plantain, banquette, muddiness, unchallenging, moviemaker, hypersensitivity, typewrite. The opponent\'s words and distractors are listed too, and there\'s a kill word.\n\nFirst, I should look for connections among the team\'s words. Let me list them again to see if any patterns emerge. Eggroll is a food, maybe Asian. Scrubwoman could be related to cleaning. Plantain is another food, but different from banana. Banquette is a type of seating, like in a restaurant. Muddiness is about being unclear or dirty. Unchallenging means easy. Moviemaker is someone who makes movies. Hypersensitivity is a medical term. Typewrite is related to typing.\n\nHmm. Maybe some of these are related to food? Eggroll and plantain are both foods. But scrubwoman and moviemaker don\'t fit. Maybe "food" isn\'t the right clue. Let\'s think of other categories. \n\nScrubwoman and typewrite could relate to work or professions. But moviemaker is also a profession. Maybe "occupation"? But then banquette is a piece of furniture. Not sure. \n\nMuddiness and hypersensitivity both relate to senses or conditions. Maybe "condition"? But that', 'message': 'Your answer contained more than two lines, please only give one clue and your targets on two separate lines.'} 
	  } 
	   } 
	   } 
	 & & \\ 
 

\end{supertabular}
}

\end{document}


\documentclass{article}
\usepackage{colortbl}
\usepackage{makecell}
\usepackage{multirow}
\usepackage{supertabular}

\begin{document}

\newcounter{utterance}

\twocolumn

{ \footnotesize  \setcounter{utterance}{1}
\setlength{\tabcolsep}{0pt}
\begin{supertabular}{c@{$\;$}|p{.15\linewidth}@{}p{.15\linewidth}p{.15\linewidth}p{.15\linewidth}p{.15\linewidth}p{.15\linewidth}}

    \# & $\;$A & \multicolumn{4}{c}{Game Master} & $\;\:$B\\
    \hline 

    \theutterance \stepcounter{utterance}  

    & & & \multicolumn{2}{p{0.3\linewidth}}{\cellcolor[rgb]{0.95,0.95,0.95}{%
	\makecell[{{p{\linewidth}}}]{% 
	  \tt {\tiny [GM$|$GM]}  
	 {'hidden': {'team': ['impunity', 'sexism', 'refutation', 'faithfulness', 'goodness', 'reconsideration', 'vindication', 'selectivity', 'repentance'], 'innocent': ['insentient', 'resourcefulness', 'veracity', 'exclusivity', 'savagery', 'symbolism', 'adage'], 'opponent': ['communicability', 'likelihood', 'interconnection', 'ability', 'imprecision', 'discernment', 'coercion', 'modularity'], 'assassin': ['formality']}, 'revealed': {'team': {'team': [], 'innocent': [], 'opponent': [], 'assassin': []}, 'opponent': {'team': [], 'innocent': [], 'opponent': [], 'assassin': []}}} 
	  } 
	   } 
	   } 
	 & & \\ 
 

    \theutterance \stepcounter{utterance}  

    & & \multicolumn{4}{p{0.6\linewidth}}{\cellcolor[rgb]{0.9,0.9,0.9}{%
	\makecell[{{p{\linewidth}}}]{% 
	  \tt {\tiny [A$\langle$GM]}  
	 Provide one single-word clue that relates to some of your team's words from the list below. You can choose to how many and to which words your clue relates to.\\ \tt The goal is to have your partner guess all of your team's words based on the clues you give them, before the other team has uncovered all of theirs. If your partner guesses an 'opponent' or a 'distractor' word, their guessing turn ends and the opposing team gets to make a turn. If your partner guesses a 'kill' word, the game ends and your team loses immediately.\\ \tt The more words your clue relates to, the quicker you can win the game, but the harder the associations may be to guess for your partner, so choose your clue word wisely. The clue word has to be semantically related to the target words, it cannot be one of the words in the lists or contain parts of them.\\ \tt \\ \tt Always give your single-word clue and your comma-separated list of related target words in the following format and make your answers as short as possible, never include any other text than is required in this form:\\ \tt \\ \tt CLUE: <WORD>\\ \tt TARGETS: <TARGETS>\\ \tt \\ \tt Your team words are: impunity, sexism, refutation, faithfulness, goodness, reconsideration, vindication, selectivity, repentance.\\ \tt Your opponent's team words are: communicability, likelihood, interconnection, ability, imprecision, discernment, coercion, modularity.\\ \tt Distractor words are: insentient, resourcefulness, veracity, exclusivity, savagery, symbolism, adage.\\ \tt The kill words are: formality. 
	  } 
	   } 
	   } 
	 & \\ 
 

    \theutterance \stepcounter{utterance}  

    & \multicolumn{4}{p{0.6\linewidth}}{\cellcolor[rgb]{0.8,1,0.9}{%
	\makecell[{{p{\linewidth}}}]{% 
	  \tt {\tiny [A$\rangle$GM]}  
	 <think>\\ \tt Okay, let's see. I need to come up with a single-word clue that relates to some of my team's words. The team words are impunity, sexism, refutation, faithfulness, goodness, reconsideration, vindication, selectivity, and repentance. The opponent's words are different, and there are distractors and a kill word.\\ \tt \\ \tt First, I should look for a common theme or concept that connects several of my team's words. Let me list them again to see if any patterns emerge. Impunity, sexism, refutation, faithfulness, goodness, reconsideration, vindication, selectivity, repentance. Hmm. \\ \tt \\ \tt Impunity is about not being punished. Sexism is discrimination based on gender. Refutation is disproving something. Faithfulness is loyalty. Goodness is moral virtue. Reconsideration is thinking again. Vindication is clearing someone's name. Selectivity is being choosy. Repentance is feeling sorry for wrongdoing.\\ \tt \\ \tt Looking for a connecting theme. Maybe "justice" or "morality"? Let's see. Impunity relates to justice (lack of punishment). Sexism is a social justice issue. Refutation could be part of seeking justice by disproving false claims. Vindication is about justice. Goodness and repentance are moral concepts. Faithfulness and selectivity might tie into personal virtues. Reconsideration could be part of a process towards justice or moral improvement. \\ \tt \\ \tt So "justice" 
	  } 
	   } 
	   } 
	 & & \\ 
 

    \theutterance \stepcounter{utterance}  

    & & & \multicolumn{2}{p{0.3\linewidth}}{\cellcolor[rgb]{0.95,0.95,0.95}{%
	\makecell[{{p{\linewidth}}}]{% 
	  \tt {\tiny [GM$|$GM]}  
	 {'player': 'Cluegiver', 'type': 'rambling error', 'utterance': '<think>\nOkay, let\'s see. I need to come up with a single-word clue that relates to some of my team\'s words. The team words are impunity, sexism, refutation, faithfulness, goodness, reconsideration, vindication, selectivity, and repentance. The opponent\'s words are different, and there are distractors and a kill word.\n\nFirst, I should look for a common theme or concept that connects several of my team\'s words. Let me list them again to see if any patterns emerge. Impunity, sexism, refutation, faithfulness, goodness, reconsideration, vindication, selectivity, repentance. Hmm. \n\nImpunity is about not being punished. Sexism is discrimination based on gender. Refutation is disproving something. Faithfulness is loyalty. Goodness is moral virtue. Reconsideration is thinking again. Vindication is clearing someone\'s name. Selectivity is being choosy. Repentance is feeling sorry for wrongdoing.\n\nLooking for a connecting theme. Maybe "justice" or "morality"? Let\'s see. Impunity relates to justice (lack of punishment). Sexism is a social justice issue. Refutation could be part of seeking justice by disproving false claims. Vindication is about justice. Goodness and repentance are moral concepts. Faithfulness and selectivity might tie into personal virtues. Reconsideration could be part of a process towards justice or moral improvement. \n\nSo "justice"', 'message': 'Your answer contained more than two lines, please only give one clue and your targets on two separate lines.'} 
	  } 
	   } 
	   } 
	 & & \\ 
 

\end{supertabular}
}

\end{document}


\documentclass{article}
\usepackage{colortbl}
\usepackage{makecell}
\usepackage{multirow}
\usepackage{supertabular}

\begin{document}

\newcounter{utterance}

\twocolumn

{ \footnotesize  \setcounter{utterance}{1}
\setlength{\tabcolsep}{0pt}
\begin{supertabular}{c@{$\;$}|p{.15\linewidth}@{}p{.15\linewidth}p{.15\linewidth}p{.15\linewidth}p{.15\linewidth}p{.15\linewidth}}

    \# & $\;$A & \multicolumn{4}{c}{Game Master} & $\;\:$B\\
    \hline 

    \theutterance \stepcounter{utterance}  

    & & & \multicolumn{2}{p{0.3\linewidth}}{\cellcolor[rgb]{0.95,0.95,0.95}{%
	\makecell[{{p{\linewidth}}}]{% 
	  \tt {\tiny [GM$|$GM]}  
	 {'hidden': {'team': ['impetuousness', 'endearment', 'heinousness', 'randomization', 'damnation', 'value', 'suspicion', 'willpower', 'integration'], 'innocent': ['instigation', 'amuck', 'stigma', 'desirability', 'avoidance', 'takings', 'despondency'], 'opponent': ['mercy', 'rhetoric', 'viewpoint', 'flimflam', 'peculiarity', 'hilarity', 'shrewdness', 'afterworld'], 'assassin': ['proneness']}, 'revealed': {'team': {'team': [], 'innocent': [], 'opponent': [], 'assassin': []}, 'opponent': {'team': [], 'innocent': [], 'opponent': [], 'assassin': []}}} 
	  } 
	   } 
	   } 
	 & & \\ 
 

    \theutterance \stepcounter{utterance}  

    & & \multicolumn{4}{p{0.6\linewidth}}{\cellcolor[rgb]{0.9,0.9,0.9}{%
	\makecell[{{p{\linewidth}}}]{% 
	  \tt {\tiny [A$\langle$GM]}  
	 Provide one single-word clue that relates to some of your team's words from the list below. You can choose to how many and to which words your clue relates to.\\ \tt The goal is to have your partner guess all of your team's words based on the clues you give them, before the other team has uncovered all of theirs. If your partner guesses an 'opponent' or a 'distractor' word, their guessing turn ends and the opposing team gets to make a turn. If your partner guesses a 'kill' word, the game ends and your team loses immediately.\\ \tt The more words your clue relates to, the quicker you can win the game, but the harder the associations may be to guess for your partner, so choose your clue word wisely. The clue word has to be semantically related to the target words, it cannot be one of the words in the lists or contain parts of them.\\ \tt \\ \tt Always give your single-word clue and your comma-separated list of related target words in the following format and make your answers as short as possible, never include any other text than is required in this form:\\ \tt \\ \tt CLUE: <WORD>\\ \tt TARGETS: <TARGETS>\\ \tt \\ \tt Your team words are: impetuousness, endearment, heinousness, randomization, damnation, value, suspicion, willpower, integration.\\ \tt Your opponent's team words are: mercy, rhetoric, viewpoint, flimflam, peculiarity, hilarity, shrewdness, afterworld.\\ \tt Distractor words are: instigation, amuck, stigma, desirability, avoidance, takings, despondency.\\ \tt The kill words are: proneness. 
	  } 
	   } 
	   } 
	 & \\ 
 

    \theutterance \stepcounter{utterance}  

    & \multicolumn{4}{p{0.6\linewidth}}{\cellcolor[rgb]{0.8,1,0.9}{%
	\makecell[{{p{\linewidth}}}]{% 
	  \tt {\tiny [A$\rangle$GM]}  
	 <think>\\ \tt Okay, let's see. I need to come up with a single-word clue that relates to some of my team's words. The team words are impetuousness, endearment, heinousness, randomization, damnation, value, suspicion, willpower, integration. The opponent's words are mercy, rhetoric, viewpoint, flimflam, peculiarity, hilarity, shrewdness, afterworld. Distractors are instigation, amuck, stigma, desirability, avoidance, takings, despondency. The kill word is proneness.\\ \tt \\ \tt First, I should look for a common theme or a word that can connect multiple team words. Let me list the team words again and see if any patterns emerge. Impetuousness is acting without thinking. Endearment is something that shows affection. Heinousness is extreme wickedness. Randomization is making things random. Damnation is condemnation. Value is worth. Suspicion is doubt. Willpower is determination. Integration is combining parts into a whole.\\ \tt \\ \tt Hmm. Maybe "judgment"? But that might not connect all. Or "moral"? Heinousness and damnation are moral terms. Value and willpower could relate to moral aspects. But endearment is more about affection. Maybe "trait"? But that's too vague. Or "quality"? Still, not specific enough.\\ \tt \\ \tt Wait, "judgment" could link to damnation, value, suspicion. 
	  } 
	   } 
	   } 
	 & & \\ 
 

    \theutterance \stepcounter{utterance}  

    & & & \multicolumn{2}{p{0.3\linewidth}}{\cellcolor[rgb]{0.95,0.95,0.95}{%
	\makecell[{{p{\linewidth}}}]{% 
	  \tt {\tiny [GM$|$GM]}  
	 {'player': 'Cluegiver', 'type': 'rambling error', 'utterance': '<think>\nOkay, let\'s see. I need to come up with a single-word clue that relates to some of my team\'s words. The team words are impetuousness, endearment, heinousness, randomization, damnation, value, suspicion, willpower, integration. The opponent\'s words are mercy, rhetoric, viewpoint, flimflam, peculiarity, hilarity, shrewdness, afterworld. Distractors are instigation, amuck, stigma, desirability, avoidance, takings, despondency. The kill word is proneness.\n\nFirst, I should look for a common theme or a word that can connect multiple team words. Let me list the team words again and see if any patterns emerge. Impetuousness is acting without thinking. Endearment is something that shows affection. Heinousness is extreme wickedness. Randomization is making things random. Damnation is condemnation. Value is worth. Suspicion is doubt. Willpower is determination. Integration is combining parts into a whole.\n\nHmm. Maybe "judgment"? But that might not connect all. Or "moral"? Heinousness and damnation are moral terms. Value and willpower could relate to moral aspects. But endearment is more about affection. Maybe "trait"? But that\'s too vague. Or "quality"? Still, not specific enough.\n\nWait, "judgment" could link to damnation, value, suspicion.', 'message': 'Your answer contained more than two lines, please only give one clue and your targets on two separate lines.'} 
	  } 
	   } 
	   } 
	 & & \\ 
 

\end{supertabular}
}

\end{document}

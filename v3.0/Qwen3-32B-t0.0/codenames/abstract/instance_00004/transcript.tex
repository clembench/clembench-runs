
\documentclass{article}
\usepackage{colortbl}
\usepackage{makecell}
\usepackage{multirow}
\usepackage{supertabular}

\begin{document}

\newcounter{utterance}

\twocolumn

{ \footnotesize  \setcounter{utterance}{1}
\setlength{\tabcolsep}{0pt}
\begin{supertabular}{c@{$\;$}|p{.15\linewidth}@{}p{.15\linewidth}p{.15\linewidth}p{.15\linewidth}p{.15\linewidth}p{.15\linewidth}}

    \# & $\;$A & \multicolumn{4}{c}{Game Master} & $\;\:$B\\
    \hline 

    \theutterance \stepcounter{utterance}  

    & & & \multicolumn{2}{p{0.3\linewidth}}{\cellcolor[rgb]{0.95,0.95,0.95}{%
	\makecell[{{p{\linewidth}}}]{% 
	  \tt {\tiny [GM$|$GM]}  
	 {'hidden': {'team': ['ingratitude', 'bonkers', 'fruition', 'ambivalence', 'veracity', 'affability', 'adaptation', 'discernment', 'richness'], 'innocent': ['incompatibility', 'gladness', 'proliferation', 'nationalization', 'subjugation', 'impulsivity', 'socialization'], 'opponent': ['knowledge', 'glorification', 'dissolution', 'unreality', 'urge', 'endlessness', 'anomaly', 'pragmatism'], 'assassin': ['willingness']}, 'revealed': {'team': {'team': [], 'innocent': [], 'opponent': [], 'assassin': []}, 'opponent': {'team': [], 'innocent': [], 'opponent': [], 'assassin': []}}} 
	  } 
	   } 
	   } 
	 & & \\ 
 

    \theutterance \stepcounter{utterance}  

    & & \multicolumn{4}{p{0.6\linewidth}}{\cellcolor[rgb]{0.9,0.9,0.9}{%
	\makecell[{{p{\linewidth}}}]{% 
	  \tt {\tiny [A$\langle$GM]}  
	 Provide one single-word clue that relates to some of your team's words from the list below. You can choose to how many and to which words your clue relates to.\\ \tt The goal is to have your partner guess all of your team's words based on the clues you give them, before the other team has uncovered all of theirs. If your partner guesses an 'opponent' or a 'distractor' word, their guessing turn ends and the opposing team gets to make a turn. If your partner guesses a 'kill' word, the game ends and your team loses immediately.\\ \tt The more words your clue relates to, the quicker you can win the game, but the harder the associations may be to guess for your partner, so choose your clue word wisely. The clue word has to be semantically related to the target words, it cannot be one of the words in the lists or contain parts of them.\\ \tt \\ \tt Always give your single-word clue and your comma-separated list of related target words in the following format and make your answers as short as possible, never include any other text than is required in this form:\\ \tt \\ \tt CLUE: <WORD>\\ \tt TARGETS: <TARGETS>\\ \tt \\ \tt Your team words are: ingratitude, bonkers, fruition, ambivalence, veracity, affability, adaptation, discernment, richness.\\ \tt Your opponent's team words are: knowledge, glorification, dissolution, unreality, urge, endlessness, anomaly, pragmatism.\\ \tt Distractor words are: incompatibility, gladness, proliferation, nationalization, subjugation, impulsivity, socialization.\\ \tt The kill words are: willingness. 
	  } 
	   } 
	   } 
	 & \\ 
 

    \theutterance \stepcounter{utterance}  

    & \multicolumn{4}{p{0.6\linewidth}}{\cellcolor[rgb]{0.8,1,0.9}{%
	\makecell[{{p{\linewidth}}}]{% 
	  \tt {\tiny [A$\rangle$GM]}  
	 <think>\\ \tt Okay, let's see. I need to come up with a single-word clue that relates to some of my team's words. The team words are ingratitude, bonkers, fruition, ambivalence, veracity, affability, adaptation, discernment, richness. The opponent's words are knowledge, glorification, etc., and there are distractors and a kill word.\\ \tt \\ \tt First, I should look for a common theme or a word that can connect multiple team words. Let me list the team words again to see if any patterns emerge. Ingratitude is about not being grateful. Bonkers means crazy. Fruition is the result of something. Ambivalence is having mixed feelings. Veracity is truthfulness. Affability is friendliness. Adaptation is adjusting. Discernment is the ability to judge well. Richness is abundance.\\ \tt \\ \tt Hmm. Maybe "perception" or "judgment"? Wait, discernment is about judgment. Veracity is truth, so maybe "truth"? But does that connect to others? Affability is friendliness, which is a trait. Adaptation is change. Maybe "change" or "adjustment"? But how does that link to others?\\ \tt \\ \tt Wait, "richness" could relate to abundance, which might connect to "proliferation," but proliferation is a distractor. Hmm. Let me think again. Ambivalence is mixed feelings, which could be linked to "conflict" or "duality." But I 
	  } 
	   } 
	   } 
	 & & \\ 
 

    \theutterance \stepcounter{utterance}  

    & & & \multicolumn{2}{p{0.3\linewidth}}{\cellcolor[rgb]{0.95,0.95,0.95}{%
	\makecell[{{p{\linewidth}}}]{% 
	  \tt {\tiny [GM$|$GM]}  
	 {'player': 'Cluegiver', 'type': 'rambling error', 'utterance': '<think>\nOkay, let\'s see. I need to come up with a single-word clue that relates to some of my team\'s words. The team words are ingratitude, bonkers, fruition, ambivalence, veracity, affability, adaptation, discernment, richness. The opponent\'s words are knowledge, glorification, etc., and there are distractors and a kill word.\n\nFirst, I should look for a common theme or a word that can connect multiple team words. Let me list the team words again to see if any patterns emerge. Ingratitude is about not being grateful. Bonkers means crazy. Fruition is the result of something. Ambivalence is having mixed feelings. Veracity is truthfulness. Affability is friendliness. Adaptation is adjusting. Discernment is the ability to judge well. Richness is abundance.\n\nHmm. Maybe "perception" or "judgment"? Wait, discernment is about judgment. Veracity is truth, so maybe "truth"? But does that connect to others? Affability is friendliness, which is a trait. Adaptation is change. Maybe "change" or "adjustment"? But how does that link to others?\n\nWait, "richness" could relate to abundance, which might connect to "proliferation, " but proliferation is a distractor. Hmm. Let me think again. Ambivalence is mixed feelings, which could be linked to "conflict" or "duality." But I', 'message': 'Your answer contained more than two lines, please only give one clue and your targets on two separate lines.'} 
	  } 
	   } 
	   } 
	 & & \\ 
 

\end{supertabular}
}

\end{document}

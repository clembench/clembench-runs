
\documentclass{article}
\usepackage{colortbl}
\usepackage{makecell}
\usepackage{multirow}
\usepackage{supertabular}

\begin{document}

\newcounter{utterance}

\twocolumn

{ \footnotesize  \setcounter{utterance}{1}
\setlength{\tabcolsep}{0pt}
\begin{supertabular}{c@{$\;$}|p{.15\linewidth}@{}p{.15\linewidth}p{.15\linewidth}p{.15\linewidth}p{.15\linewidth}p{.15\linewidth}}

    \# & $\;$A & \multicolumn{4}{c}{Game Master} & $\;\:$B\\
    \hline 

    \theutterance \stepcounter{utterance}  

    & & & \multicolumn{2}{p{0.3\linewidth}}{\cellcolor[rgb]{0.95,0.95,0.95}{%
	\makecell[{{p{\linewidth}}}]{% 
	  \tt {\tiny [GM$|$GM]}  
	 {'hidden': {'team': ['ambivalence', 'dilution', 'politeness', 'existentialism', 'concordance', 'prudence', 'reinstatement', 'gracefulness', 'nationalization'], 'innocent': ['stigma', 'ridiculousness', 'optimism', 'inefficiency', 'exclusivity', 'unsettlement', 'adorability'], 'opponent': ['competency', 'socialization', 'pacifism', 'ambiguity', 'intermix', 'fruition', 'inequity', 'deviousness'], 'assassin': ['coercion']}, 'revealed': {'team': {'team': [], 'innocent': [], 'opponent': [], 'assassin': []}, 'opponent': {'team': [], 'innocent': [], 'opponent': [], 'assassin': []}}} 
	  } 
	   } 
	   } 
	 & & \\ 
 

    \theutterance \stepcounter{utterance}  

    & & \multicolumn{4}{p{0.6\linewidth}}{\cellcolor[rgb]{0.9,0.9,0.9}{%
	\makecell[{{p{\linewidth}}}]{% 
	  \tt {\tiny [A$\langle$GM]}  
	 Provide one single-word clue that relates to some of your team's words from the list below. You can choose to how many and to which words your clue relates to.\\ \tt The goal is to have your partner guess all of your team's words based on the clues you give them, before the other team has uncovered all of theirs. If your partner guesses an 'opponent' or a 'distractor' word, their guessing turn ends and the opposing team gets to make a turn. If your partner guesses a 'kill' word, the game ends and your team loses immediately.\\ \tt The more words your clue relates to, the quicker you can win the game, but the harder the associations may be to guess for your partner, so choose your clue word wisely. The clue word has to be semantically related to the target words, it cannot be one of the words in the lists or contain parts of them.\\ \tt \\ \tt Always give your single-word clue and your comma-separated list of related target words in the following format and make your answers as short as possible, never include any other text than is required in this form:\\ \tt \\ \tt CLUE: <WORD>\\ \tt TARGETS: <TARGETS>\\ \tt \\ \tt Your team words are: ambivalence, dilution, politeness, existentialism, concordance, prudence, reinstatement, gracefulness, nationalization.\\ \tt Your opponent's team words are: competency, socialization, pacifism, ambiguity, intermix, fruition, inequity, deviousness.\\ \tt Distractor words are: stigma, ridiculousness, optimism, inefficiency, exclusivity, unsettlement, adorability.\\ \tt The kill words are: coercion. 
	  } 
	   } 
	   } 
	 & \\ 
 

    \theutterance \stepcounter{utterance}  

    & \multicolumn{4}{p{0.6\linewidth}}{\cellcolor[rgb]{0.8,1,0.9}{%
	\makecell[{{p{\linewidth}}}]{% 
	  \tt {\tiny [A$\rangle$GM]}  
	 <think>\\ \tt Okay, let's see. I need to come up with a single-word clue that relates to some of my team's words. The team words are ambivalence, dilution, politeness, existentialism, concordance, prudence, reinstatement, gracefulness, nationalization. The opponent's words are competency, socialization, pacifism, ambiguity, intermix, fruition, inequity, deviousness. Distractors are stigma, ridiculousness, optimism, inefficiency, exclusivity, unsettlement, adorability. The kill word is coercion.\\ \tt \\ \tt First, I should look for a common theme or concept that connects several of my team's words. Let's list them out again:\\ \tt \\ \tt - Ambivalence: mixed feelings\\ \tt - Dilution: making something weaker\\ \tt - Politeness: being courteous\\ \tt - Existentialism: philosophical concept about existence\\ \tt - Concordance: agreement\\ \tt - Prudence: caution\\ \tt - Reinstatement: putting back into a position\\ \tt - Gracefulness: elegance\\ \tt - Nationalization: government taking control of industries\\ \tt \\ \tt Looking for a common thread. Maybe "balance" or "moderation"? Ambivalence is about having two conflicting feelings, which is a form of balance. Dilution is reducing strength, which could relate to moderation. Politeness and gracefulness might relate to balance in behavior. Prudence is about carefulness, which is a form of moderation. Concordance is agreement, which is a balance between parties. 
	  } 
	   } 
	   } 
	 & & \\ 
 

    \theutterance \stepcounter{utterance}  

    & & & \multicolumn{2}{p{0.3\linewidth}}{\cellcolor[rgb]{0.95,0.95,0.95}{%
	\makecell[{{p{\linewidth}}}]{% 
	  \tt {\tiny [GM$|$GM]}  
	 {'player': 'Cluegiver', 'type': 'rambling error', 'utterance': '<think>\nOkay, let\'s see. I need to come up with a single-word clue that relates to some of my team\'s words. The team words are ambivalence, dilution, politeness, existentialism, concordance, prudence, reinstatement, gracefulness, nationalization. The opponent\'s words are competency, socialization, pacifism, ambiguity, intermix, fruition, inequity, deviousness. Distractors are stigma, ridiculousness, optimism, inefficiency, exclusivity, unsettlement, adorability. The kill word is coercion.\n\nFirst, I should look for a common theme or concept that connects several of my team\'s words. Let\'s list them out again:\n\n- Ambivalence: mixed feelings\n- Dilution: making something weaker\n- Politeness: being courteous\n- Existentialism: philosophical concept about existence\n- Concordance: agreement\n- Prudence: caution\n- Reinstatement: putting back into a position\n- Gracefulness: elegance\n- Nationalization: government taking control of industries\n\nLooking for a common thread. Maybe "balance" or "moderation"? Ambivalence is about having two conflicting feelings, which is a form of balance. Dilution is reducing strength, which could relate to moderation. Politeness and gracefulness might relate to balance in behavior. Prudence is about carefulness, which is a form of moderation. Concordance is agreement, which is a balance between parties.', 'message': 'Your answer contained more than two lines, please only give one clue and your targets on two separate lines.'} 
	  } 
	   } 
	   } 
	 & & \\ 
 

\end{supertabular}
}

\end{document}

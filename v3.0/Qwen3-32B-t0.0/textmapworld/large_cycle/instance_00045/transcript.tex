
\documentclass{article}
\usepackage{colortbl}
\usepackage{makecell}
\usepackage{multirow}
\usepackage{supertabular}

\begin{document}

\newcounter{utterance}

\twocolumn

{ \footnotesize  \setcounter{utterance}{1}
\setlength{\tabcolsep}{0pt}
\begin{supertabular}{c@{$\;$}|p{.15\linewidth}@{}p{.15\linewidth}p{.15\linewidth}p{.15\linewidth}p{.15\linewidth}p{.15\linewidth}}

    \# & $\;$A & \multicolumn{4}{c}{Game Master} & $\;\:$B\\
    \hline 

    \theutterance \stepcounter{utterance}  

    & & \multicolumn{4}{p{0.6\linewidth}}{\cellcolor[rgb]{0.9,0.9,0.9}{%
	\makecell[{{p{\linewidth}}}]{% 
	  \tt {\tiny [A$\langle$GM]}  
	 Please help me with the following task. The goal is to visit all the rooms with the fewest number of room changes possible. In each room, you need to decide the direction to go in. Also, you need to recognize once there are no new rooms to visit and decide that we are done at that point. Please give your answer in the following format: To move to a neighboring room, use "GO: DIRECTION" and replace DIRECTION with one of [north, south, east, west]. To stop the exploration, answer with "DONE" instead. Omit any other text.\\ \tt Here is an example:\\ \tt You are in the Kitchen. Currently available directions: south, west. What is your next instruction?\\ \tt GO: west\\ \tt You have made a step and entered a Lobby. Currently available directions: east, north. What is your next instruction?\\ \tt GO: north\\ \tt ...\\ \tt You have made a step and entered a Bedroom. Currently available directions: south. What is your next instruction?\\ \tt DONE\\ \tt Let us start. You are in the Game room. Currently available directions: north, west, east. What is your next instruction? 
	  } 
	   } 
	   } 
	 & \\ 
 

    \theutterance \stepcounter{utterance}  

    & \multicolumn{4}{p{0.6\linewidth}}{\cellcolor[rgb]{0.8,1,0.9}{%
	\makecell[{{p{\linewidth}}}]{% 
	  \tt {\tiny [A$\rangle$GM]}  
	 <think>\\ \tt Okay, let's see. I need to figure out the best path to visit all rooms with the fewest changes. Starting in the Game room with directions north, west, east. Hmm. The example used a depth-first approach, going west first, then north, etc. Maybe I should do the same here. Let's pick one direction first. Let's say west. Then, from there, explore as far as possible before backtracking. Wait, but I need to make sure I don't miss any rooms. Let me think step by step.\\ \tt \\ \tt First, in the Game room, available directions are north, west, east. If I go west first, then maybe that leads to another room. Suppose I go west. Then, in that new room, I check the available directions. If there's a new direction, I take it. Otherwise, I backtrack. But since I don't know the actual layout, I have to assume that each direction leads to a new room unless it's a dead end. The goal is to minimize room changes, so maybe a depth-first search is better because it reduces the number of backtracks. Alternatively, maybe a breadth-first approach, but that might require more steps. Wait, the example used depth-first. Let me try that.\\ \tt \\ \tt So first step: GO west. Then, in the next room, check directions. Suppose that room has directions like north, east, etc. If I keep going in one direction until I can't 
	  } 
	   } 
	   } 
	 & & \\ 
 

    \theutterance \stepcounter{utterance}  

    & & & \multicolumn{2}{p{0.3\linewidth}}{\cellcolor[rgb]{0.95,0.95,0.95}{%
	\makecell[{{p{\linewidth}}}]{% 
	  \tt {\tiny [GM$|$GM]}  
	 abort game 
	  } 
	   } 
	   } 
	 & & \\ 
 

\end{supertabular}
}

\end{document}

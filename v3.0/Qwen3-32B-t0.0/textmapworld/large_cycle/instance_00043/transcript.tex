
\documentclass{article}
\usepackage{colortbl}
\usepackage{makecell}
\usepackage{multirow}
\usepackage{supertabular}

\begin{document}

\newcounter{utterance}

\twocolumn

{ \footnotesize  \setcounter{utterance}{1}
\setlength{\tabcolsep}{0pt}
\begin{supertabular}{c@{$\;$}|p{.15\linewidth}@{}p{.15\linewidth}p{.15\linewidth}p{.15\linewidth}p{.15\linewidth}p{.15\linewidth}}

    \# & $\;$A & \multicolumn{4}{c}{Game Master} & $\;\:$B\\
    \hline 

    \theutterance \stepcounter{utterance}  

    & & \multicolumn{4}{p{0.6\linewidth}}{\cellcolor[rgb]{0.9,0.9,0.9}{%
	\makecell[{{p{\linewidth}}}]{% 
	  \tt {\tiny [A$\langle$GM]}  
	 Please help me with the following task. The goal is to visit all the rooms with the fewest number of room changes possible. In each room, you need to decide the direction to go in. Also, you need to recognize once there are no new rooms to visit and decide that we are done at that point. Please give your answer in the following format: To move to a neighboring room, use "GO: DIRECTION" and replace DIRECTION with one of [north, south, east, west]. To stop the exploration, answer with "DONE" instead. Omit any other text.\\ \tt Here is an example:\\ \tt You are in the Kitchen. Currently available directions: south, west. What is your next instruction?\\ \tt GO: west\\ \tt You have made a step and entered a Lobby. Currently available directions: east, north. What is your next instruction?\\ \tt GO: north\\ \tt ...\\ \tt You have made a step and entered a Bedroom. Currently available directions: south. What is your next instruction?\\ \tt DONE\\ \tt Let us start. You are in the Dining room. Currently available directions: west. What is your next instruction? 
	  } 
	   } 
	   } 
	 & \\ 
 

    \theutterance \stepcounter{utterance}  

    & \multicolumn{4}{p{0.6\linewidth}}{\cellcolor[rgb]{0.8,1,0.9}{%
	\makecell[{{p{\linewidth}}}]{% 
	  \tt {\tiny [A$\rangle$GM]}  
	 <think>\\ \tt Okay, let's see. I need to figure out the best way to visit all the rooms with the fewest changes. The user is in the Dining room and can go west. The example shows that when there's only one direction, you go there. But here, the Dining room only has west. So first step is to go west.\\ \tt \\ \tt After moving west, I'll enter a new room. Let's say that room has multiple directions. I need to choose a direction that hasn't been visited yet. But since I don't know the layout, I have to assume that each move is necessary. The key is to avoid backtracking as much as possible. Maybe a depth-first search approach? Explore as far as possible before backtracking.\\ \tt \\ \tt Wait, but the user wants the fewest room changes. So maybe a breadth-first approach? But without knowing the structure, it's hard. The example ends when there's only one direction left, which might mean you've covered all rooms. Hmm. Let me think through the example again. In the example, the last room had only south, which was the way back, so they stopped. So when you can't go anywhere new, you stop.\\ \tt \\ \tt So the strategy is: always pick a direction that leads to a new room if possible. If all directions lead to rooms already visited, then stop. Since the initial step is west from Dining room, let's proceed. Suppose the next room has multiple directions. Let's say, 
	  } 
	   } 
	   } 
	 & & \\ 
 

    \theutterance \stepcounter{utterance}  

    & & & \multicolumn{2}{p{0.3\linewidth}}{\cellcolor[rgb]{0.95,0.95,0.95}{%
	\makecell[{{p{\linewidth}}}]{% 
	  \tt {\tiny [GM$|$GM]}  
	 abort game 
	  } 
	   } 
	   } 
	 & & \\ 
 

\end{supertabular}
}

\end{document}

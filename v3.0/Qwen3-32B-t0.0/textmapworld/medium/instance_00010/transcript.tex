
\documentclass{article}
\usepackage{colortbl}
\usepackage{makecell}
\usepackage{multirow}
\usepackage{supertabular}

\begin{document}

\newcounter{utterance}

\twocolumn

{ \footnotesize  \setcounter{utterance}{1}
\setlength{\tabcolsep}{0pt}
\begin{supertabular}{c@{$\;$}|p{.15\linewidth}@{}p{.15\linewidth}p{.15\linewidth}p{.15\linewidth}p{.15\linewidth}p{.15\linewidth}}

    \# & $\;$A & \multicolumn{4}{c}{Game Master} & $\;\:$B\\
    \hline 

    \theutterance \stepcounter{utterance}  

    & & \multicolumn{4}{p{0.6\linewidth}}{\cellcolor[rgb]{0.9,0.9,0.9}{%
	\makecell[{{p{\linewidth}}}]{% 
	  \tt {\tiny [A$\langle$GM]}  
	 Please help me with the following task. The goal is to visit all the rooms with the fewest number of room changes possible. In each room, you need to decide the direction to go in. Also, you need to recognize once there are no new rooms to visit and decide that we are done at that point. Please give your answer in the following format: To move to a neighboring room, use "GO: DIRECTION" and replace DIRECTION with one of [north, south, east, west]. To stop the exploration, answer with "DONE" instead. Omit any other text.\\ \tt Here is an example:\\ \tt You are in the Kitchen. Currently available directions: south, west. What is your next instruction?\\ \tt GO: west\\ \tt You have made a step and entered a Lobby. Currently available directions: east, north. What is your next instruction?\\ \tt GO: north\\ \tt ...\\ \tt You have made a step and entered a Bedroom. Currently available directions: south. What is your next instruction?\\ \tt DONE\\ \tt Let us start. You are in the Media room. Currently available directions: north, south. What is your next instruction? 
	  } 
	   } 
	   } 
	 & \\ 
 

    \theutterance \stepcounter{utterance}  

    & \multicolumn{4}{p{0.6\linewidth}}{\cellcolor[rgb]{0.8,1,0.9}{%
	\makecell[{{p{\linewidth}}}]{% 
	  \tt {\tiny [A$\rangle$GM]}  
	 <think>\\ \tt Okay, let's see. I need to figure out the best way to visit all the rooms with the fewest changes. The user is in the Media room and can go north or south. The goal is to minimize the number of times they change direction, right? So maybe I should try to go in one direction as much as possible before backtracking.\\ \tt \\ \tt First, I'll start by choosing one direction. Let's say north. If I go north, maybe that leads to more rooms that need to be explored. If I go south, maybe that's a dead end or a shorter path. But without knowing the layout, I have to make an educated guess. Since the example had a path that went west, then north, then done, maybe a similar approach applies here.\\ \tt \\ \tt Wait, the example had a path that went from Kitchen to Lobby to Bedroom and then done. So the key is to traverse through rooms in a way that covers all without unnecessary backtracking. But how do I know which direction leads to more rooms? Since the user can't see the map, I have to assume that each direction might lead to new rooms or not.\\ \tt \\ \tt In the Media room, available directions are north and south. Let's pick north first. If going north leads to a room with more directions, that's good. If it's a dead end, then we can backtrack. But since the goal is to minimize room changes, maybe a depth-first approach is better. So go as far 
	  } 
	   } 
	   } 
	 & & \\ 
 

    \theutterance \stepcounter{utterance}  

    & & & \multicolumn{2}{p{0.3\linewidth}}{\cellcolor[rgb]{0.95,0.95,0.95}{%
	\makecell[{{p{\linewidth}}}]{% 
	  \tt {\tiny [GM$|$GM]}  
	 abort game 
	  } 
	   } 
	   } 
	 & & \\ 
 

\end{supertabular}
}

\end{document}

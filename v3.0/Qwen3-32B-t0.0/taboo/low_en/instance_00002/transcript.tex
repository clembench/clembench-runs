
\documentclass{article}
\usepackage{colortbl}
\usepackage{makecell}
\usepackage{multirow}
\usepackage{supertabular}

\begin{document}

\newcounter{utterance}

\twocolumn

{ \footnotesize  \setcounter{utterance}{1}
\setlength{\tabcolsep}{0pt}
\begin{supertabular}{c@{$\;$}|p{.15\linewidth}@{}p{.15\linewidth}p{.15\linewidth}p{.15\linewidth}p{.15\linewidth}p{.15\linewidth}}

    \# & $\;$A & \multicolumn{4}{c}{Game Master} & $\;\:$B\\
    \hline 

    \theutterance \stepcounter{utterance}  

    & & \multicolumn{4}{p{0.6\linewidth}}{\cellcolor[rgb]{0.9,0.9,0.9}{%
	\makecell[{{p{\linewidth}}}]{% 
	  \tt {\tiny [A$\langle$GM]}  
	 You are playing a collaborative word guessing game in which you have to describe a target word for another player to guess.\\ \tt \\ \tt Rules:\\ \tt (a) You have to reply in the form: CLUE: <some text>. Guesses from the other player will start with GUESS.\\ \tt (b) You cannot use the target word itself, parts or morphological variants of it in your description.\\ \tt (c) In addition, the same rules apply for related words which are provided below.\\ \tt \\ \tt End conditions:\\ \tt (i) If you use the target word or a related word in your description, then you lose.\\ \tt (ii) If the other player can guess the target word in 3 tries, you both win.\\ \tt \\ \tt Let us start.\\ \tt \\ \tt This is the target word that you need to describe and that the other player needs to guess:\\ \tt \\ \tt competent\\ \tt \\ \tt Related words are:\\ \tt \\ \tt - efficient\\ \tt - skilled\\ \tt - capable\\ \tt \\ \tt Important: You are under time pressure, give short descriptions that are to the point! 
	  } 
	   } 
	   } 
	 & \\ 
 

    \theutterance \stepcounter{utterance}  

    & \multicolumn{4}{p{0.6\linewidth}}{\cellcolor[rgb]{0.8,1,0.9}{%
	\makecell[{{p{\linewidth}}}]{% 
	  \tt {\tiny [A$\rangle$GM]}  
	 <think>\\ \tt Okay, let's see. The target word is "competent," and I need to describe it without using that word or the related ones: efficient, skilled, or capable. Hmm. Let me think about the meaning of competent. It means having the ability to do something well, right? So maybe I can focus on the idea of being able to handle tasks effectively.\\ \tt \\ \tt But I can't use "capable," which is a synonym. Maybe I can say something like "able to handle tasks without issues." Wait, but "able" might be part of "capable," which is a related word. Oh, right, "capable" is on the list, so I can't use "able" either? Or is "able" considered a part of it? The rules say I can't use parts or morphological variants. "Able" is part of "capable," so maybe that's not allowed. Hmm, tricky.\\ \tt \\ \tt What about "having the necessary skills"? But "skills" is part of "skilled," which is a related word. So I can't use that. Maybe focus on the outcome. Like "someone who can do the job well without mistakes." But "do the job well" might be too vague. Or "someone who is trusted to perform tasks effectively." Wait, "effectively" is similar to "efficient," which is a related word. So I can't use that either.\\ \tt \\ \tt How about "having the knowledge and 
	  } 
	   } 
	   } 
	 & & \\ 
 

    \theutterance \stepcounter{utterance}  

    & & & \multicolumn{2}{p{0.3\linewidth}}{\cellcolor[rgb]{0.95,0.95,0.95}{%
	\makecell[{{p{\linewidth}}}]{% 
	  \tt {\tiny [GM$|$GM]}  
	 abort game 
	  } 
	   } 
	   } 
	 & & \\ 
 

\end{supertabular}
}

\end{document}

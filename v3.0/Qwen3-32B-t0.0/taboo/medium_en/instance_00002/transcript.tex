
\documentclass{article}
\usepackage{colortbl}
\usepackage{makecell}
\usepackage{multirow}
\usepackage{supertabular}

\begin{document}

\newcounter{utterance}

\twocolumn

{ \footnotesize  \setcounter{utterance}{1}
\setlength{\tabcolsep}{0pt}
\begin{supertabular}{c@{$\;$}|p{.15\linewidth}@{}p{.15\linewidth}p{.15\linewidth}p{.15\linewidth}p{.15\linewidth}p{.15\linewidth}}

    \# & $\;$A & \multicolumn{4}{c}{Game Master} & $\;\:$B\\
    \hline 

    \theutterance \stepcounter{utterance}  

    & & \multicolumn{4}{p{0.6\linewidth}}{\cellcolor[rgb]{0.9,0.9,0.9}{%
	\makecell[{{p{\linewidth}}}]{% 
	  \tt {\tiny [A$\langle$GM]}  
	 You are playing a collaborative word guessing game in which you have to describe a target word for another player to guess.\\ \tt \\ \tt Rules:\\ \tt (a) You have to reply in the form: CLUE: <some text>. Guesses from the other player will start with GUESS.\\ \tt (b) You cannot use the target word itself, parts or morphological variants of it in your description.\\ \tt (c) In addition, the same rules apply for related words which are provided below.\\ \tt \\ \tt End conditions:\\ \tt (i) If you use the target word or a related word in your description, then you lose.\\ \tt (ii) If the other player can guess the target word in 3 tries, you both win.\\ \tt \\ \tt Let us start.\\ \tt \\ \tt This is the target word that you need to describe and that the other player needs to guess:\\ \tt \\ \tt prospective\\ \tt \\ \tt Related words are:\\ \tt \\ \tt - foresight\\ \tt - potential\\ \tt - future\\ \tt \\ \tt Important: You are under time pressure, give short descriptions that are to the point! 
	  } 
	   } 
	   } 
	 & \\ 
 

    \theutterance \stepcounter{utterance}  

    & \multicolumn{4}{p{0.6\linewidth}}{\cellcolor[rgb]{0.8,1,0.9}{%
	\makecell[{{p{\linewidth}}}]{% 
	  \tt {\tiny [A$\rangle$GM]}  
	 <think>\\ \tt Okay, let's see. The target word is "prospective," and I need to describe it without using the word itself or the related words: foresight, potential, or future. Hmm. Let me think about the definition of prospective. It usually refers to something expected to happen in the future, like prospective students or a prospective plan. But I can't mention future. Maybe I can focus on the idea of possibility or likelihood. Wait, "potential" is a related word, so I can't use that either. \\ \tt \\ \tt What about synonyms? Maybe "possible" or "anticipated"? But I need to avoid any related words. Let me think of a scenario. If someone is looking to join a program, they might be called a prospective member. So maybe I can say "possible candidate for a position." But I need to make sure not to use any forbidden terms. Let me check: "possible" isn't a related word. Okay. Another angle: "prospective" can mean looking forward to something. But "foresight" is related, so I can't use that. Maybe "looking ahead" but that's similar to foresight. Hmm.\\ \tt \\ \tt Wait, the key is that "prospective" often refers to something that is possible but not yet certain. So maybe "possible but not yet confirmed." Let me test that. If I say "possible but not yet confirmed," does that lead to "prospective"? Maybe. Let 
	  } 
	   } 
	   } 
	 & & \\ 
 

    \theutterance \stepcounter{utterance}  

    & & & \multicolumn{2}{p{0.3\linewidth}}{\cellcolor[rgb]{0.95,0.95,0.95}{%
	\makecell[{{p{\linewidth}}}]{% 
	  \tt {\tiny [GM$|$GM]}  
	 abort game 
	  } 
	   } 
	   } 
	 & & \\ 
 

\end{supertabular}
}

\end{document}


\documentclass{article}
\usepackage{colortbl}
\usepackage{makecell}
\usepackage{multirow}
\usepackage{supertabular}

\begin{document}

\newcounter{utterance}

\twocolumn

{ \footnotesize  \setcounter{utterance}{1}
\setlength{\tabcolsep}{0pt}
\begin{supertabular}{c@{$\;$}|p{.15\linewidth}@{}p{.15\linewidth}p{.15\linewidth}p{.15\linewidth}p{.15\linewidth}p{.15\linewidth}}

    \# & $\;$A & \multicolumn{4}{c}{Game Master} & $\;\:$B\\
    \hline 

    \theutterance \stepcounter{utterance}  

    & & \multicolumn{4}{p{0.6\linewidth}}{\cellcolor[rgb]{0.9,0.9,0.9}{%
	\makecell[{{p{\linewidth}}}]{% 
	  \tt {\tiny [A$\langle$GM]}  
	 You are playing a text adventure game. I will describe what you can perceive in the game. You write the single action you want to take in the game starting with >. Only reply with an action.\\ \tt For example:\\ \tt > open cupboard\\ \tt \\ \tt In addition to common actions, you can vater. To vater is to make something physically unobstructed and uncovered, and make its contents available for use or interaction.\\ \tt \\ \tt Your goal for this game is: Put the nogia on the table, the potted plant in the aling and the peach in the aling.\\ \tt \\ \tt Once you have achieved your goal, write "> done" to end the game.\\ \tt \\ \tt You are in a bedroom now. There are a book, a bed, a wardrobe and a pillow. The book is on the floor. The wardrobe is closed. The pillow is on the bed. There is a passage to a hallway here. 
	  } 
	   } 
	   } 
	 & \\ 
 

    \theutterance \stepcounter{utterance}  

    & \multicolumn{4}{p{0.6\linewidth}}{\cellcolor[rgb]{0.8,1,0.9}{%
	\makecell[{{p{\linewidth}}}]{% 
	  \tt {\tiny [A$\rangle$GM]}  
	 <think>\\ \tt Okay, let's see. The user is in a bedroom with a book, bed, wardrobe, and pillow. The book is on the floor, wardrobe is closed, pillow on the bed. Passage to hallway.\\ \tt \\ \tt The goal is to put the nogia on the table, potted plant in the aling, and peach in the aling. But first, I need to figure out where these items are. The current description doesn't mention a table, aling, or the specific items like nogia, potted plant, or peach. So maybe I need to explore further.\\ \tt \\ \tt The user can vater, which means making something unobstructed. Maybe the wardrobe is closed, so vater it to open? Or check the book? But the book is on the floor. Wait, the user's actions are to take items and place them. But first, they need to find the items. Since the current room has a book, maybe the book has information? Or perhaps the wardrobe contains items. Let me check the wardrobe. So the first action could be to open the wardrobe. But the user can vater, which is to make something unobstructed. So maybe "vater wardrobe" to open it. Alternatively, "open wardrobe". But the user's example uses "open cupboard", so maybe the correct action is to open the wardrobe. But the user mentioned that vater is an action. So maybe "vater wardrobe" would be the correct 
	  } 
	   } 
	   } 
	 & & \\ 
 

    \theutterance \stepcounter{utterance}  

    & & & \multicolumn{2}{p{0.3\linewidth}}{\cellcolor[rgb]{0.95,0.95,0.95}{%
	\makecell[{{p{\linewidth}}}]{% 
	  \tt {\tiny [GM$|$GM]}  
	 Round: 0 
	  } 
	   } 
	   } 
	 & & \\ 
 

    \theutterance \stepcounter{utterance}  

    & & & \multicolumn{2}{p{0.3\linewidth}}{\cellcolor[rgb]{0.95,0.95,0.95}{%
	\makecell[{{p{\linewidth}}}]{% 
	  \tt {\tiny [GM$|$GM]}  
	 command_tag_missing 
	  } 
	   } 
	   } 
	 & & \\ 
 

    \theutterance \stepcounter{utterance}  

    & & & \multicolumn{2}{p{0.3\linewidth}}{\cellcolor[rgb]{0.95,0.95,0.95}{%
	\makecell[{{p{\linewidth}}}]{% 
	  \tt {\tiny [GM$|$GM]}  
	 {'goal_states_achieved': [], 'game_successfully_finished': False} 
	  } 
	   } 
	   } 
	 & & \\ 
 

\end{supertabular}
}

\end{document}


\documentclass{article}
\usepackage{colortbl}
\usepackage{makecell}
\usepackage{multirow}
\usepackage{supertabular}

\begin{document}

\newcounter{utterance}

\twocolumn

{ \footnotesize  \setcounter{utterance}{1}
\setlength{\tabcolsep}{0pt}
\begin{supertabular}{c@{$\;$}|p{.15\linewidth}@{}p{.15\linewidth}p{.15\linewidth}p{.15\linewidth}p{.15\linewidth}p{.15\linewidth}}

    \# & $\;$A & \multicolumn{4}{c}{Game Master} & $\;\:$B\\
    \hline 

    \theutterance \stepcounter{utterance}  

    & & \multicolumn{4}{p{0.6\linewidth}}{\cellcolor[rgb]{0.9,0.9,0.9}{%
	\makecell[{{p{\linewidth}}}]{% 
	  \tt {\tiny [A$\langle$GM]}  
	 You are a language wizard who likes to guess words by using the given rules.\\ \tt \\ \tt Welcome to Wordle! You have six attempts to guess the target word, a valid English word of five lowercase letters (a-z). Please use the tags "explanation:" and "guess:" to provide a concise explanation for each guess.\\ \tt \\ \tt To help you make an informed guess, you will receive a clue for the word, such as\\ \tt clue: "snowy white"\\ \tt \\ \tt Here is an example guess based on the clue:\\ \tt explanation: In the fairy tail Snow White, the girl is killed because she eats a poisoned apple. And the word apple has 5 letters.\\ \tt guess: apple\\ \tt \\ \tt I will then indicate whether I agree or disagree with your guess and provide rationale, but agreeing with a guess does not confirm its correctness. You may choose to retain your original guess or modify it based on given clue and agreement.\\ \tt \\ \tt After each guess, your answer will be validated, and you will receive feedback indicating which letters are correct (green), which letters are correct but in the wrong position (yellow), and which letters are incorrect (red). This feedback can be useful in determining which letters to include or exclude in your next guess.\\ \tt \\ \tt For example, the feedback for "apple" might be:\\ \tt guess_feedback: a<yellow> p<yellow> p<green> l<yellow> e<red>\\ \tt \\ \tt Here are some rules to keep in mind:\\ \tt 1. Please do not generate any guess feedback or clue on your own\\ \tt 2. Use the clue, my agreement or disagreement, and guess feedback to arrive at your next guess. This should be reflected in your explanation as well.\\ \tt \\ \tt Do not generate any other text.\\ \tt \\ \tt Let's begin with your first guess.\\ \tt \\ \tt clue: baby conveyance 
	  } 
	   } 
	   } 
	 & \\ 
 

    \theutterance \stepcounter{utterance}  

    & \multicolumn{4}{p{0.6\linewidth}}{\cellcolor[rgb]{0.8,1,0.9}{%
	\makecell[{{p{\linewidth}}}]{% 
	  \tt {\tiny [A$\rangle$GM]}  
	 <think>\\ \tt Okay, let's see. The clue is "baby conveyance." Hmm, so I need to think of a five-letter word that refers to something used to carry a baby. Let's break it down.\\ \tt \\ \tt First, "baby" is the key here. Common items related to babies... Maybe a stroller? But that's seven letters. How about "pram"? Yes, a pram is a baby carriage. Let me check the letters: P-R-A-M. Wait, that's four letters. Oh, maybe "prams"? But that's plural. The target word is five letters. Wait, maybe "stroller" is too long. What else?\\ \tt \\ \tt Another thought: "carriage" is too long. "Buggy"? That's five letters. B-U-G-G-Y. Yes, a baby buggy. But is "buggy" the correct term? Or maybe "carseat"? No, that's longer. Wait, "baby carrier" is two words. Maybe "sling"? But that's five letters. S-L-I-N-G. But I'm not sure if that's considered a conveyance. \\ \tt \\ \tt Alternatively, "toddler" is not a conveyance. Wait, "pushchair" is another term, but again longer. Let me think again. The clue says "conveyance," which means something that carries. So maybe "cart"? No, too short. "Cradle"? That's five letters. 
	  } 
	   } 
	   } 
	 & & \\ 
 

    \theutterance \stepcounter{utterance}  

    & & & \multicolumn{2}{p{0.3\linewidth}}{\cellcolor[rgb]{0.95,0.95,0.95}{%
	\makecell[{{p{\linewidth}}}]{% 
	  \tt {\tiny [GM$|$GM]}  
	 Error: The response should always start with the keyword 'explanation:' 
	  } 
	   } 
	   } 
	 & & \\ 
 

    \theutterance \stepcounter{utterance}  

    & & & \multicolumn{2}{p{0.3\linewidth}}{\cellcolor[rgb]{0.95,0.95,0.95}{%
	\makecell[{{p{\linewidth}}}]{% 
	  \tt {\tiny [GM$|$GM]}  
	 game_result = ABORT 
	  } 
	   } 
	   } 
	 & & \\ 
 

\end{supertabular}
}

\end{document}

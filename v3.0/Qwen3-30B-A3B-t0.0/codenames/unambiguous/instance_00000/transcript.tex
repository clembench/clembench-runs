
\documentclass{article}
\usepackage{colortbl}
\usepackage{makecell}
\usepackage{multirow}
\usepackage{supertabular}

\begin{document}

\newcounter{utterance}

\twocolumn

{ \footnotesize  \setcounter{utterance}{1}
\setlength{\tabcolsep}{0pt}
\begin{supertabular}{c@{$\;$}|p{.15\linewidth}@{}p{.15\linewidth}p{.15\linewidth}p{.15\linewidth}p{.15\linewidth}p{.15\linewidth}}

    \# & $\;$A & \multicolumn{4}{c}{Game Master} & $\;\:$B\\
    \hline 

    \theutterance \stepcounter{utterance}  

    & & & \multicolumn{2}{p{0.3\linewidth}}{\cellcolor[rgb]{0.95,0.95,0.95}{%
	\makecell[{{p{\linewidth}}}]{% 
	  \tt {\tiny [GM$|$GM]}  
	 {'hidden': {'team': ['lighthouse', 'piranha', 'blazer', 'epilepsy', 'councilman', 'merger', 'riddance', 'tanker', 'graveyard'], 'innocent': ['underwear', 'bathtub', 'spider', 'tornado', 'opponent', 'broker', 'greed'], 'opponent': ['storeroom', 'affliction', 'bathrobe', 'cactus', 'thug', 'allergy', 'buzzer', 'lobotomy'], 'assassin': ['adultery']}, 'revealed': {'team': {'team': [], 'innocent': [], 'opponent': [], 'assassin': []}, 'opponent': {'team': [], 'innocent': [], 'opponent': [], 'assassin': []}}} 
	  } 
	   } 
	   } 
	 & & \\ 
 

    \theutterance \stepcounter{utterance}  

    & & \multicolumn{4}{p{0.6\linewidth}}{\cellcolor[rgb]{0.9,0.9,0.9}{%
	\makecell[{{p{\linewidth}}}]{% 
	  \tt {\tiny [A$\langle$GM]}  
	 Provide one single-word clue that relates to some of your team's words from the list below. You can choose to how many and to which words your clue relates to.\\ \tt The goal is to have your partner guess all of your team's words based on the clues you give them, before the other team has uncovered all of theirs. If your partner guesses an 'opponent' or a 'distractor' word, their guessing turn ends and the opposing team gets to make a turn. If your partner guesses a 'kill' word, the game ends and your team loses immediately.\\ \tt The more words your clue relates to, the quicker you can win the game, but the harder the associations may be to guess for your partner, so choose your clue word wisely. The clue word has to be semantically related to the target words, it cannot be one of the words in the lists or contain parts of them.\\ \tt \\ \tt Always give your single-word clue and your comma-separated list of related target words in the following format and make your answers as short as possible, never include any other text than is required in this form:\\ \tt \\ \tt CLUE: <WORD>\\ \tt TARGETS: <TARGETS>\\ \tt \\ \tt Your team words are: lighthouse, piranha, blazer, epilepsy, councilman, merger, riddance, tanker, graveyard.\\ \tt Your opponent's team words are: storeroom, affliction, bathrobe, cactus, thug, allergy, buzzer, lobotomy.\\ \tt Distractor words are: underwear, bathtub, spider, tornado, opponent, broker, greed.\\ \tt The kill words are: adultery. 
	  } 
	   } 
	   } 
	 & \\ 
 

    \theutterance \stepcounter{utterance}  

    & \multicolumn{4}{p{0.6\linewidth}}{\cellcolor[rgb]{0.8,1,0.9}{%
	\makecell[{{p{\linewidth}}}]{% 
	  \tt {\tiny [A$\rangle$GM]}  
	 <think>\\ \tt Okay, let's see. I need to come up with a single-word clue that relates to some of my team's words. The team words are lighthouse, piranha, blazer, epilepsy, councilman, merger, riddance, tanker, graveyard. The opponent's words are different, and there are distractors and a kill word.\\ \tt \\ \tt First, I should look for common themes or associations among my team's words. Let's go through them one by one.\\ \tt \\ \tt Lighthouse: maybe related to light, guidance, structure, or a building.\\ \tt Piranha: a fish, maybe related to water, danger, or a specific type of fish.\\ \tt Blazer: a type of jacket, clothing, maybe related to school or business attire.\\ \tt Epilepsy: a medical condition, seizures, health issue.\\ \tt Councilman: a politician, government, decision-making.\\ \tt Merger: a business term, combining companies.\\ \tt Riddance: getting rid of something, removal.\\ \tt Tanker: a large ship or vehicle, transportation.\\ \tt Graveyard: a place where dead are buried, cemetery.\\ \tt \\ \tt Looking for possible overlaps. Let's think of words that could connect multiple. For example, "danger" might relate to piranha (dangerous fish) and maybe epilepsy (seizures could be dangerous?), but not sure. Or "removal" for riddance and maybe merger (removing companies?), but not sure.\\ \tt \\ \tt Another angle: think of synonyms or related concepts 
	  } 
	   } 
	   } 
	 & & \\ 
 

    \theutterance \stepcounter{utterance}  

    & & & \multicolumn{2}{p{0.3\linewidth}}{\cellcolor[rgb]{0.95,0.95,0.95}{%
	\makecell[{{p{\linewidth}}}]{% 
	  \tt {\tiny [GM$|$GM]}  
	 {'player': 'Cluegiver', 'type': 'rambling error', 'utterance': '<think>\nOkay, let\'s see. I need to come up with a single-word clue that relates to some of my team\'s words. The team words are lighthouse, piranha, blazer, epilepsy, councilman, merger, riddance, tanker, graveyard. The opponent\'s words are different, and there are distractors and a kill word.\n\nFirst, I should look for common themes or associations among my team\'s words. Let\'s go through them one by one.\n\nLighthouse: maybe related to light, guidance, structure, or a building.\nPiranha: a fish, maybe related to water, danger, or a specific type of fish.\nBlazer: a type of jacket, clothing, maybe related to school or business attire.\nEpilepsy: a medical condition, seizures, health issue.\nCouncilman: a politician, government, decision-making.\nMerger: a business term, combining companies.\nRiddance: getting rid of something, removal.\nTanker: a large ship or vehicle, transportation.\nGraveyard: a place where dead are buried, cemetery.\n\nLooking for possible overlaps. Let\'s think of words that could connect multiple. For example, "danger" might relate to piranha (dangerous fish) and maybe epilepsy (seizures could be dangerous?), but not sure. Or "removal" for riddance and maybe merger (removing companies?), but not sure.\n\nAnother angle: think of synonyms or related concepts', 'message': 'Your answer contained more than two lines, please only give one clue and your targets on two separate lines.'} 
	  } 
	   } 
	   } 
	 & & \\ 
 

\end{supertabular}
}

\end{document}


\documentclass{article}
\usepackage{colortbl}
\usepackage{makecell}
\usepackage{multirow}
\usepackage{supertabular}

\begin{document}

\newcounter{utterance}

\twocolumn

{ \footnotesize  \setcounter{utterance}{1}
\setlength{\tabcolsep}{0pt}
\begin{supertabular}{c@{$\;$}|p{.15\linewidth}@{}p{.15\linewidth}p{.15\linewidth}p{.15\linewidth}p{.15\linewidth}p{.15\linewidth}}

    \# & $\;$A & \multicolumn{4}{c}{Game Master} & $\;\:$B\\
    \hline 

    \theutterance \stepcounter{utterance}  

    & & & \multicolumn{2}{p{0.3\linewidth}}{\cellcolor[rgb]{0.95,0.95,0.95}{%
	\makecell[{{p{\linewidth}}}]{% 
	  \tt {\tiny [GM$|$GM]}  
	 {'hidden': {'team': ['eggroll', 'scrubwoman', 'plantain', 'banquette', 'muddiness', 'unchallenging', 'moviemaker', 'hypersensitivity', 'typewrite'], 'innocent': ['bandmaster', 'ladylove', 'customizer', 'defenseman', 'narrowness', 'hotelkeeper', 'neckband'], 'opponent': ['hurdler', 'assassinator', 'emotionality', 'liquidness', 'bagpiper', 'deinstitutionalization', 'housewifery', 'perpetration'], 'assassin': ['seacoast']}, 'revealed': {'team': {'team': [], 'innocent': [], 'opponent': [], 'assassin': []}, 'opponent': {'team': [], 'innocent': [], 'opponent': [], 'assassin': []}}} 
	  } 
	   } 
	   } 
	 & & \\ 
 

    \theutterance \stepcounter{utterance}  

    & & \multicolumn{4}{p{0.6\linewidth}}{\cellcolor[rgb]{0.9,0.9,0.9}{%
	\makecell[{{p{\linewidth}}}]{% 
	  \tt {\tiny [A$\langle$GM]}  
	 Provide one single-word clue that relates to some of your team's words from the list below. You can choose to how many and to which words your clue relates to.\\ \tt The goal is to have your partner guess all of your team's words based on the clues you give them, before the other team has uncovered all of theirs. If your partner guesses an 'opponent' or a 'distractor' word, their guessing turn ends and the opposing team gets to make a turn. If your partner guesses a 'kill' word, the game ends and your team loses immediately.\\ \tt The more words your clue relates to, the quicker you can win the game, but the harder the associations may be to guess for your partner, so choose your clue word wisely. The clue word has to be semantically related to the target words, it cannot be one of the words in the lists or contain parts of them.\\ \tt \\ \tt Always give your single-word clue and your comma-separated list of related target words in the following format and make your answers as short as possible, never include any other text than is required in this form:\\ \tt \\ \tt CLUE: <WORD>\\ \tt TARGETS: <TARGETS>\\ \tt \\ \tt Your team words are: eggroll, scrubwoman, plantain, banquette, muddiness, unchallenging, moviemaker, hypersensitivity, typewrite.\\ \tt Your opponent's team words are: hurdler, assassinator, emotionality, liquidness, bagpiper, deinstitutionalization, housewifery, perpetration.\\ \tt Distractor words are: bandmaster, ladylove, customizer, defenseman, narrowness, hotelkeeper, neckband.\\ \tt The kill words are: seacoast. 
	  } 
	   } 
	   } 
	 & \\ 
 

    \theutterance \stepcounter{utterance}  

    & \multicolumn{4}{p{0.6\linewidth}}{\cellcolor[rgb]{0.8,1,0.9}{%
	\makecell[{{p{\linewidth}}}]{% 
	  \tt {\tiny [A$\rangle$GM]}  
	 <think>\\ \tt Okay, let's see. I need to come up with a single-word clue that relates to some of my team's words. The team words are: eggroll, scrubwoman, plantain, banquette, muddiness, unchallenging, moviemaker, hypersensitivity, typewrite.\\ \tt \\ \tt First, I should look for common themes or connections among these words. Let me go through each one:\\ \tt \\ \tt - Eggroll: A food item, a type of spring roll.\\ \tt - Scrubwoman: A female scrubber, maybe someone who cleans?\\ \tt - Plantain: A type of banana, also a food.\\ \tt - Banquette: A bench, often in a restaurant.\\ \tt - Muddiness: The state of being muddy, unclear.\\ \tt - Unchallenging: Not difficult, easy.\\ \tt - Moviemaker: Someone who makes movies, director or producer.\\ \tt - Hypersensitivity: Excessive sensitivity, like an allergy.\\ \tt - Typewrite: To write with a typewriter, or the act of typing.\\ \tt \\ \tt Looking for connections. Maybe some of these words are related to food? Eggroll and plantain are foods. But scrubwoman, banquette, etc., don't fit. Maybe not food.\\ \tt \\ \tt Another angle: parts of speech? Not sure. Maybe verbs vs nouns. But the clue needs to be a single word related to multiple targets.\\ \tt \\ \tt Looking for synonyms or broader categories. For example, "sensitivity" could relate to hypersensitivity, but not sure 
	  } 
	   } 
	   } 
	 & & \\ 
 

    \theutterance \stepcounter{utterance}  

    & & & \multicolumn{2}{p{0.3\linewidth}}{\cellcolor[rgb]{0.95,0.95,0.95}{%
	\makecell[{{p{\linewidth}}}]{% 
	  \tt {\tiny [GM$|$GM]}  
	 {'player': 'Cluegiver', 'type': 'rambling error', 'utterance': '<think>\nOkay, let\'s see. I need to come up with a single-word clue that relates to some of my team\'s words. The team words are: eggroll, scrubwoman, plantain, banquette, muddiness, unchallenging, moviemaker, hypersensitivity, typewrite.\n\nFirst, I should look for common themes or connections among these words. Let me go through each one:\n\n- Eggroll: A food item, a type of spring roll.\n- Scrubwoman: A female scrubber, maybe someone who cleans?\n- Plantain: A type of banana, also a food.\n- Banquette: A bench, often in a restaurant.\n- Muddiness: The state of being muddy, unclear.\n- Unchallenging: Not difficult, easy.\n- Moviemaker: Someone who makes movies, director or producer.\n- Hypersensitivity: Excessive sensitivity, like an allergy.\n- Typewrite: To write with a typewriter, or the act of typing.\n\nLooking for connections. Maybe some of these words are related to food? Eggroll and plantain are foods. But scrubwoman, banquette, etc., don\'t fit. Maybe not food.\n\nAnother angle: parts of speech? Not sure. Maybe verbs vs nouns. But the clue needs to be a single word related to multiple targets.\n\nLooking for synonyms or broader categories. For example, "sensitivity" could relate to hypersensitivity, but not sure', 'message': 'Your answer contained more than two lines, please only give one clue and your targets on two separate lines.'} 
	  } 
	   } 
	   } 
	 & & \\ 
 

\end{supertabular}
}

\end{document}

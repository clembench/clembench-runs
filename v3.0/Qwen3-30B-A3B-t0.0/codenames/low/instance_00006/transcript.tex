
\documentclass{article}
\usepackage{colortbl}
\usepackage{makecell}
\usepackage{multirow}
\usepackage{supertabular}

\begin{document}

\newcounter{utterance}

\twocolumn

{ \footnotesize  \setcounter{utterance}{1}
\setlength{\tabcolsep}{0pt}
\begin{supertabular}{c@{$\;$}|p{.15\linewidth}@{}p{.15\linewidth}p{.15\linewidth}p{.15\linewidth}p{.15\linewidth}p{.15\linewidth}}

    \# & $\;$A & \multicolumn{4}{c}{Game Master} & $\;\:$B\\
    \hline 

    \theutterance \stepcounter{utterance}  

    & & & \multicolumn{2}{p{0.3\linewidth}}{\cellcolor[rgb]{0.95,0.95,0.95}{%
	\makecell[{{p{\linewidth}}}]{% 
	  \tt {\tiny [GM$|$GM]}  
	 {'hidden': {'team': ['chilliness', 'defacement', 'liquidness', 'glasshouse', 'eggroll', 'ungodliness', 'bandmaster', 'unbefitting', 'ninepin'], 'innocent': ['slouchy', 'musicologist', 'cattiness', 'duplicator', 'duckweed', 'bagpiper', 'toxicologist'], 'opponent': ['secession', 'gentlemanliness', 'candleholder', 'shedder', 'inexistence', 'plantain', 'hydroplane', 'hotelkeeper'], 'assassin': ['geophysics']}, 'revealed': {'team': {'team': [], 'innocent': [], 'opponent': [], 'assassin': []}, 'opponent': {'team': [], 'innocent': [], 'opponent': [], 'assassin': []}}} 
	  } 
	   } 
	   } 
	 & & \\ 
 

    \theutterance \stepcounter{utterance}  

    & & \multicolumn{4}{p{0.6\linewidth}}{\cellcolor[rgb]{0.9,0.9,0.9}{%
	\makecell[{{p{\linewidth}}}]{% 
	  \tt {\tiny [A$\langle$GM]}  
	 Provide one single-word clue that relates to some of your team's words from the list below. You can choose to how many and to which words your clue relates to.\\ \tt The goal is to have your partner guess all of your team's words based on the clues you give them, before the other team has uncovered all of theirs. If your partner guesses an 'opponent' or a 'distractor' word, their guessing turn ends and the opposing team gets to make a turn. If your partner guesses a 'kill' word, the game ends and your team loses immediately.\\ \tt The more words your clue relates to, the quicker you can win the game, but the harder the associations may be to guess for your partner, so choose your clue word wisely. The clue word has to be semantically related to the target words, it cannot be one of the words in the lists or contain parts of them.\\ \tt \\ \tt Always give your single-word clue and your comma-separated list of related target words in the following format and make your answers as short as possible, never include any other text than is required in this form:\\ \tt \\ \tt CLUE: <WORD>\\ \tt TARGETS: <TARGETS>\\ \tt \\ \tt Your team words are: chilliness, defacement, liquidness, glasshouse, eggroll, ungodliness, bandmaster, unbefitting, ninepin.\\ \tt Your opponent's team words are: secession, gentlemanliness, candleholder, shedder, inexistence, plantain, hydroplane, hotelkeeper.\\ \tt Distractor words are: slouchy, musicologist, cattiness, duplicator, duckweed, bagpiper, toxicologist.\\ \tt The kill words are: geophysics. 
	  } 
	   } 
	   } 
	 & \\ 
 

    \theutterance \stepcounter{utterance}  

    & \multicolumn{4}{p{0.6\linewidth}}{\cellcolor[rgb]{0.8,1,0.9}{%
	\makecell[{{p{\linewidth}}}]{% 
	  \tt {\tiny [A$\rangle$GM]}  
	 <think>\\ \tt Okay, let's see. I need to come up with a single-word clue that relates to some of my team's words. The team words are: chilliness, defacement, liquidness, glasshouse, eggroll, ungodliness, bandmaster, unbefitting, ninepin.\\ \tt \\ \tt First, I should look for common themes or categories among these words. Let me go through them one by one.\\ \tt \\ \tt Chilliness: relates to coldness.\\ \tt Defacement: damage to something's appearance.\\ \tt Liquidness: state of being liquid.\\ \tt Glasshouse: a greenhouse, a building with glass.\\ \tt Eggroll: a food item, like a fried roll filled with ingredients.\\ \tt Ungodliness: lack of godliness, sinfulness.\\ \tt Bandmaster: leader of a band.\\ \tt Unbefitting: not suitable or appropriate.\\ \tt Ninepin: a bowling pin, part of the game of nine-pin bowling.\\ \tt \\ \tt Hmm. Maybe looking for words that have to do with specific categories. Let's see if any of these can be grouped. For example, "glasshouse" and "ninepin" might not relate directly. "Bandmaster" and "eggroll" don't seem related. "Chilliness" and "liquidness" both relate to states or properties. "Defacement" and "ungodliness" are more abstract. "Unbefitting" is an adjective. \\ \tt \\ \tt Wait, maybe looking for words that can be associated with a common term. For example, " 
	  } 
	   } 
	   } 
	 & & \\ 
 

    \theutterance \stepcounter{utterance}  

    & & & \multicolumn{2}{p{0.3\linewidth}}{\cellcolor[rgb]{0.95,0.95,0.95}{%
	\makecell[{{p{\linewidth}}}]{% 
	  \tt {\tiny [GM$|$GM]}  
	 {'player': 'Cluegiver', 'type': 'rambling error', 'utterance': '<think>\nOkay, let\'s see. I need to come up with a single-word clue that relates to some of my team\'s words. The team words are: chilliness, defacement, liquidness, glasshouse, eggroll, ungodliness, bandmaster, unbefitting, ninepin.\n\nFirst, I should look for common themes or categories among these words. Let me go through them one by one.\n\nChilliness: relates to coldness.\nDefacement: damage to something\'s appearance.\nLiquidness: state of being liquid.\nGlasshouse: a greenhouse, a building with glass.\nEggroll: a food item, like a fried roll filled with ingredients.\nUngodliness: lack of godliness, sinfulness.\nBandmaster: leader of a band.\nUnbefitting: not suitable or appropriate.\nNinepin: a bowling pin, part of the game of nine-pin bowling.\n\nHmm. Maybe looking for words that have to do with specific categories. Let\'s see if any of these can be grouped. For example, "glasshouse" and "ninepin" might not relate directly. "Bandmaster" and "eggroll" don\'t seem related. "Chilliness" and "liquidness" both relate to states or properties. "Defacement" and "ungodliness" are more abstract. "Unbefitting" is an adjective. \n\nWait, maybe looking for words that can be associated with a common term. For example, "', 'message': 'Your answer contained more than two lines, please only give one clue and your targets on two separate lines.'} 
	  } 
	   } 
	   } 
	 & & \\ 
 

\end{supertabular}
}

\end{document}


\documentclass{article}
\usepackage{colortbl}
\usepackage{makecell}
\usepackage{multirow}
\usepackage{supertabular}

\begin{document}

\newcounter{utterance}

\twocolumn

{ \footnotesize  \setcounter{utterance}{1}
\setlength{\tabcolsep}{0pt}
\begin{supertabular}{c@{$\;$}|p{.15\linewidth}@{}p{.15\linewidth}p{.15\linewidth}p{.15\linewidth}p{.15\linewidth}p{.15\linewidth}}

    \# & $\;$A & \multicolumn{4}{c}{Game Master} & $\;\:$B\\
    \hline 

    \theutterance \stepcounter{utterance}  

    & & & \multicolumn{2}{p{0.3\linewidth}}{\cellcolor[rgb]{0.95,0.95,0.95}{%
	\makecell[{{p{\linewidth}}}]{% 
	  \tt {\tiny [GM$|$GM]}  
	 {'hidden': {'team': ['loftiness', 'narrowness', 'backbreaker', 'candlepower', 'demineralization', 'nearness', 'defacement', 'denaturalization', 'checkmark'], 'innocent': ['pleasantness', 'denseness', 'assassinator', 'tonsillitis', 'defenseman', 'fortunetelling', 'secession'], 'opponent': ['typeset', 'demobilization', 'emersion', 'darer', 'nightcaps', 'hotelkeeper', 'dualism', 'shortlist'], 'assassin': ['shirtfront']}, 'revealed': {'team': {'team': [], 'innocent': [], 'opponent': [], 'assassin': []}, 'opponent': {'team': [], 'innocent': [], 'opponent': [], 'assassin': []}}} 
	  } 
	   } 
	   } 
	 & & \\ 
 

    \theutterance \stepcounter{utterance}  

    & & \multicolumn{4}{p{0.6\linewidth}}{\cellcolor[rgb]{0.9,0.9,0.9}{%
	\makecell[{{p{\linewidth}}}]{% 
	  \tt {\tiny [A$\langle$GM]}  
	 Provide one single-word clue that relates to some of your team's words from the list below. You can choose to how many and to which words your clue relates to.\\ \tt The goal is to have your partner guess all of your team's words based on the clues you give them, before the other team has uncovered all of theirs. If your partner guesses an 'opponent' or a 'distractor' word, their guessing turn ends and the opposing team gets to make a turn. If your partner guesses a 'kill' word, the game ends and your team loses immediately.\\ \tt The more words your clue relates to, the quicker you can win the game, but the harder the associations may be to guess for your partner, so choose your clue word wisely. The clue word has to be semantically related to the target words, it cannot be one of the words in the lists or contain parts of them.\\ \tt \\ \tt Always give your single-word clue and your comma-separated list of related target words in the following format and make your answers as short as possible, never include any other text than is required in this form:\\ \tt \\ \tt CLUE: <WORD>\\ \tt TARGETS: <TARGETS>\\ \tt \\ \tt Your team words are: loftiness, narrowness, backbreaker, candlepower, demineralization, nearness, defacement, denaturalization, checkmark.\\ \tt Your opponent's team words are: typeset, demobilization, emersion, darer, nightcaps, hotelkeeper, dualism, shortlist.\\ \tt Distractor words are: pleasantness, denseness, assassinator, tonsillitis, defenseman, fortunetelling, secession.\\ \tt The kill words are: shirtfront. 
	  } 
	   } 
	   } 
	 & \\ 
 

    \theutterance \stepcounter{utterance}  

    & \multicolumn{4}{p{0.6\linewidth}}{\cellcolor[rgb]{0.8,1,0.9}{%
	\makecell[{{p{\linewidth}}}]{% 
	  \tt {\tiny [A$\rangle$GM]}  
	 <think>\\ \tt Okay, let's see. I need to come up with a single-word clue that relates to some of my team's words. The team words are: loftiness, narrowness, backbreaker, candlepower, demineralization, nearness, defacement, denaturalization, checkmark.\\ \tt \\ \tt First, I should look for common themes or patterns among these words. Let me go through each one:\\ \tt \\ \tt - Loftiness: related to height or highness.\\ \tt - Narrowness: being narrow.\\ \tt - Backbreaker: something that breaks the back, maybe a move in wrestling? Or a heavy burden?\\ \tt - Candlepower: measure of light intensity.\\ \tt - Demineralization: loss of minerals.\\ \tt - Nearness: closeness.\\ \tt - Defacement: damage to appearance.\\ \tt - Denaturalization: removal of natural qualities.\\ \tt - Checkmark: a mark indicating correctness.\\ \tt \\ \tt Hmm, some of these might relate to "loss" or "removal" – like demineralization, denaturalization, defacement. But not sure. Or maybe "state of being" – loftiness, narrowness, nearness. But that's three words. Alternatively, looking for words that have "ness" suffix: loftiness, narrowness, nearness. But the clue can't be "ness" because it's not in the list. Wait, the clue can't be part of the target words. So "ness" isn't a target word here. But the clue has to 
	  } 
	   } 
	   } 
	 & & \\ 
 

    \theutterance \stepcounter{utterance}  

    & & & \multicolumn{2}{p{0.3\linewidth}}{\cellcolor[rgb]{0.95,0.95,0.95}{%
	\makecell[{{p{\linewidth}}}]{% 
	  \tt {\tiny [GM$|$GM]}  
	 {'player': 'Cluegiver', 'type': 'rambling error', 'utterance': '<think>\nOkay, let\'s see. I need to come up with a single-word clue that relates to some of my team\'s words. The team words are: loftiness, narrowness, backbreaker, candlepower, demineralization, nearness, defacement, denaturalization, checkmark.\n\nFirst, I should look for common themes or patterns among these words. Let me go through each one:\n\n- Loftiness: related to height or highness.\n- Narrowness: being narrow.\n- Backbreaker: something that breaks the back, maybe a move in wrestling? Or a heavy burden?\n- Candlepower: measure of light intensity.\n- Demineralization: loss of minerals.\n- Nearness: closeness.\n- Defacement: damage to appearance.\n- Denaturalization: removal of natural qualities.\n- Checkmark: a mark indicating correctness.\n\nHmm, some of these might relate to "loss" or "removal" – like demineralization, denaturalization, defacement. But not sure. Or maybe "state of being" – loftiness, narrowness, nearness. But that\'s three words. Alternatively, looking for words that have "ness" suffix: loftiness, narrowness, nearness. But the clue can\'t be "ness" because it\'s not in the list. Wait, the clue can\'t be part of the target words. So "ness" isn\'t a target word here. But the clue has to', 'message': 'Your answer contained more than two lines, please only give one clue and your targets on two separate lines.'} 
	  } 
	   } 
	   } 
	 & & \\ 
 

\end{supertabular}
}

\end{document}

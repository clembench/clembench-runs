
\documentclass{article}
\usepackage{colortbl}
\usepackage{makecell}
\usepackage{multirow}
\usepackage{supertabular}

\begin{document}

\newcounter{utterance}

\twocolumn

{ \footnotesize  \setcounter{utterance}{1}
\setlength{\tabcolsep}{0pt}
\begin{supertabular}{c@{$\;$}|p{.15\linewidth}@{}p{.15\linewidth}p{.15\linewidth}p{.15\linewidth}p{.15\linewidth}p{.15\linewidth}}

    \# & $\;$A & \multicolumn{4}{c}{Game Master} & $\;\:$B\\
    \hline 

    \theutterance \stepcounter{utterance}  

    & & & \multicolumn{2}{p{0.3\linewidth}}{\cellcolor[rgb]{0.95,0.95,0.95}{%
	\makecell[{{p{\linewidth}}}]{% 
	  \tt {\tiny [GM$|$GM]}  
	 {'hidden': {'team': ['affirmation', 'lunacy', 'adventurousness', 'interconnection', 'oppressiveness', 'insignificance', 'perpetration', 'dissonance', 'anticommunism'], 'innocent': ['fantasy', 'endearment', 'ethic', 'solemnity', 'adversity', 'manner', 'loathsomeness'], 'opponent': ['responsibility', 'splendor', 'grandiosity', 'austerity', 'viciousness', 'cohesion', 'unification', 'lessening'], 'assassin': ['optimism']}, 'revealed': {'team': {'team': [], 'innocent': [], 'opponent': [], 'assassin': []}, 'opponent': {'team': [], 'innocent': [], 'opponent': [], 'assassin': []}}} 
	  } 
	   } 
	   } 
	 & & \\ 
 

    \theutterance \stepcounter{utterance}  

    & & \multicolumn{4}{p{0.6\linewidth}}{\cellcolor[rgb]{0.9,0.9,0.9}{%
	\makecell[{{p{\linewidth}}}]{% 
	  \tt {\tiny [A$\langle$GM]}  
	 Provide one single-word clue that relates to some of your team's words from the list below. You can choose to how many and to which words your clue relates to.\\ \tt The goal is to have your partner guess all of your team's words based on the clues you give them, before the other team has uncovered all of theirs. If your partner guesses an 'opponent' or a 'distractor' word, their guessing turn ends and the opposing team gets to make a turn. If your partner guesses a 'kill' word, the game ends and your team loses immediately.\\ \tt The more words your clue relates to, the quicker you can win the game, but the harder the associations may be to guess for your partner, so choose your clue word wisely. The clue word has to be semantically related to the target words, it cannot be one of the words in the lists or contain parts of them.\\ \tt \\ \tt Always give your single-word clue and your comma-separated list of related target words in the following format and make your answers as short as possible, never include any other text than is required in this form:\\ \tt \\ \tt CLUE: <WORD>\\ \tt TARGETS: <TARGETS>\\ \tt \\ \tt Your team words are: affirmation, lunacy, adventurousness, interconnection, oppressiveness, insignificance, perpetration, dissonance, anticommunism.\\ \tt Your opponent's team words are: responsibility, splendor, grandiosity, austerity, viciousness, cohesion, unification, lessening.\\ \tt Distractor words are: fantasy, endearment, ethic, solemnity, adversity, manner, loathsomeness.\\ \tt The kill words are: optimism. 
	  } 
	   } 
	   } 
	 & \\ 
 

    \theutterance \stepcounter{utterance}  

    & \multicolumn{4}{p{0.6\linewidth}}{\cellcolor[rgb]{0.8,1,0.9}{%
	\makecell[{{p{\linewidth}}}]{% 
	  \tt {\tiny [A$\rangle$GM]}  
	 <think>\\ \tt Okay, let's see. I need to come up with a single-word clue that relates to some of my team's words. The team words are: affirmation, lunacy, adventurousness, interconnection, oppressiveness, insignificance, perpetration, dissonance, anticommunism. The opponent's words are different, and there are distractors and kill words.\\ \tt \\ \tt First, I should look for a common theme or a semantic link among some of my team's words. Let's go through them one by one.\\ \tt \\ \tt Affirmation: This is a positive statement or confirmation. Lunacy: Madness or foolishness. Adventurousness: Willingness to take risks. Interconnection: Relationship between things. Oppressiveness: Oppression or harsh treatment. Insignificance: Lack of importance. Perpetration: The act of committing something, like a crime. Dissonance: A lack of harmony. Anticommunism: Opposition to communism.\\ \tt \\ \tt Hmm. Maybe looking for words that have to do with conflict or tension? Dissonance, oppressiveness, maybe lunacy? Or perhaps something related to ideas or concepts. Alternatively, looking for words that are abstract nouns.\\ \tt \\ \tt Wait, the clue has to be a single word that's semantically related to the target words. Let me think of a word that could connect a few of these. Let's see:\\ \tt \\ \tt "Conflict" could relate to dissonance, oppressiveness, maybe lunacy? Not sure. 
	  } 
	   } 
	   } 
	 & & \\ 
 

    \theutterance \stepcounter{utterance}  

    & & & \multicolumn{2}{p{0.3\linewidth}}{\cellcolor[rgb]{0.95,0.95,0.95}{%
	\makecell[{{p{\linewidth}}}]{% 
	  \tt {\tiny [GM$|$GM]}  
	 {'player': 'Cluegiver', 'type': 'rambling error', 'utterance': '<think>\nOkay, let\'s see. I need to come up with a single-word clue that relates to some of my team\'s words. The team words are: affirmation, lunacy, adventurousness, interconnection, oppressiveness, insignificance, perpetration, dissonance, anticommunism. The opponent\'s words are different, and there are distractors and kill words.\n\nFirst, I should look for a common theme or a semantic link among some of my team\'s words. Let\'s go through them one by one.\n\nAffirmation: This is a positive statement or confirmation. Lunacy: Madness or foolishness. Adventurousness: Willingness to take risks. Interconnection: Relationship between things. Oppressiveness: Oppression or harsh treatment. Insignificance: Lack of importance. Perpetration: The act of committing something, like a crime. Dissonance: A lack of harmony. Anticommunism: Opposition to communism.\n\nHmm. Maybe looking for words that have to do with conflict or tension? Dissonance, oppressiveness, maybe lunacy? Or perhaps something related to ideas or concepts. Alternatively, looking for words that are abstract nouns.\n\nWait, the clue has to be a single word that\'s semantically related to the target words. Let me think of a word that could connect a few of these. Let\'s see:\n\n"Conflict" could relate to dissonance, oppressiveness, maybe lunacy? Not sure.', 'message': 'Your answer contained more than two lines, please only give one clue and your targets on two separate lines.'} 
	  } 
	   } 
	   } 
	 & & \\ 
 

\end{supertabular}
}

\end{document}

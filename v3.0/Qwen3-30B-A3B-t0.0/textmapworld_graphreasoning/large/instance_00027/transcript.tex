
\documentclass{article}
\usepackage{colortbl}
\usepackage{makecell}
\usepackage{multirow}
\usepackage{supertabular}

\begin{document}

\newcounter{utterance}

\twocolumn

{ \footnotesize  \setcounter{utterance}{1}
\setlength{\tabcolsep}{0pt}
\begin{supertabular}{c@{$\;$}|p{.15\linewidth}@{}p{.15\linewidth}p{.15\linewidth}p{.15\linewidth}p{.15\linewidth}p{.15\linewidth}}

    \# & $\;$A & \multicolumn{4}{c}{Game Master} & $\;\:$B\\
    \hline 

    \theutterance \stepcounter{utterance}  

    & & \multicolumn{4}{p{0.6\linewidth}}{\cellcolor[rgb]{0.9,0.9,0.9}{%
	\makecell[{{p{\linewidth}}}]{% 
	  \tt {\tiny [A$\langle$GM]}  
	 Please help me with the following task. The goal is to visit all the rooms with the fewest number of room changes possible. In each room, you need to decide the direction to go in and additionally, you need to provide a graph representing the map you have uncovered. Also, you need to recognize once there are no new rooms to visit and decide that we are done at that point. Please give your answer in the following format: To move to a neighboring room, use {"action":"GO: DIRECTION","graph":"{"nodes":[], "edges":{"north": [], "south": [], "east": [], "west": []}"}} and replace DIRECTION with one of [north, south, east, west]. To stop the exploration, answer with "DONE" instead. Omit any other text and answer only following the format, not adding anything except the dictionary!\\ \tt Here is an example: \\ \tt You are in the Living Room. Currently available directions: south, west. What is your next instruction?\\ \tt {"action":"GO: west", "graph": {"nodes":["Living Room"],"edges":{"north":[],"south":[],"east":[],"west":[]}}}\\ \tt You have made a step and entered a Library. Currently available directions: east, north. What is your next instruction?\\ \tt {"action":"GO: north", "graph":{"nodes":["Living Room", "Library"],"edges":{"north":[],"south":[],"east":[],"west":[("Living Room", "Library")]}}}\\ \tt You have made a step and entered a Kitchen. Currently available directions: south, east. What is your next instruction?\\ \tt {"action":"GO: east", "graph":{"nodes": ["Living Room", "Library", "Kitchen"], "edges":{"north": [("Library", "Kitchen")],"south": [],"east": [],"west": [("Living Room", "Library")]}}}\\ \tt ...\\ \tt You have made a step and entered a Bedroom. Currently available directions: south, west. What is your next instruction?\\ \tt {"action": "DONE", "graph": {...}}\\ \tt Let us start.\\ \tt You are in the Living room. Currently available directions: west, east. What is your next instruction? 
	  } 
	   } 
	   } 
	 & \\ 
 

    \theutterance \stepcounter{utterance}  

    & \multicolumn{4}{p{0.6\linewidth}}{\cellcolor[rgb]{0.8,1,0.9}{%
	\makecell[{{p{\linewidth}}}]{% 
	  \tt {\tiny [A$\rangle$GM]}  
	 <think>\\ \tt Okay, let's see. The user wants to visit all rooms with the fewest room changes. So the goal is to explore efficiently, right? Starting in the Living Room, which has directions west and east. I need to decide which way to go first.\\ \tt \\ \tt Hmm, the example shows that when you go west, the graph updates with the new node and edges. But since I don't know the layout yet, I should pick a direction and then keep track of the rooms and connections. The key is to avoid backtracking as much as possible to minimize changes. But since I don't have a map, I have to explore step by step.\\ \tt \\ \tt Let me think. If I go west first, then maybe I can explore that path. But maybe I should check both directions? Wait, but the problem says to use the fewest number of room changes. So maybe a depth-first approach would be better, exploring as far as possible before backtracking. But since I don't have the map, I need to make decisions based on available directions each time.\\ \tt \\ \tt Wait, the initial step is to choose between west and east. Let's pick west. So the first action is "GO: west". Then the graph will have the Living Room and the next room, say, Library. Then the next step would be to check available directions there. But since I don't know what's next, I have to proceed step by step.\\ \tt \\ \tt But the user wants the answer in the specific format 
	  } 
	   } 
	   } 
	 & & \\ 
 

    \theutterance \stepcounter{utterance}  

    & & & \multicolumn{2}{p{0.3\linewidth}}{\cellcolor[rgb]{0.95,0.95,0.95}{%
	\makecell[{{p{\linewidth}}}]{% 
	  \tt {\tiny [GM$|$GM]}  
	 abort game 
	  } 
	   } 
	   } 
	 & & \\ 
 

\end{supertabular}
}

\end{document}


\documentclass{article}
\usepackage{colortbl}
\usepackage{makecell}
\usepackage{multirow}
\usepackage{supertabular}

\begin{document}

\newcounter{utterance}

\twocolumn

{ \footnotesize  \setcounter{utterance}{1}
\setlength{\tabcolsep}{0pt}
\begin{supertabular}{c@{$\;$}|p{.15\linewidth}@{}p{.15\linewidth}p{.15\linewidth}p{.15\linewidth}p{.15\linewidth}p{.15\linewidth}}

    \# & $\;$A & \multicolumn{4}{c}{Game Master} & $\;\:$B\\
    \hline 

    \theutterance \stepcounter{utterance}  

    & & \multicolumn{4}{p{0.6\linewidth}}{\cellcolor[rgb]{0.9,0.9,0.9}{%
	\makecell[{{p{\linewidth}}}]{% 
	  \tt {\tiny [A$\langle$GM]}  
	 Please help me with the following task. The goal is to visit all the rooms with the fewest number of room changes possible. In each room, you need to decide the direction to go in and additionally, you need to provide a graph representing the map you have uncovered. Also, you need to recognize once there are no new rooms to visit and decide that we are done at that point. Please give your answer in the following format: To move to a neighboring room, use {"action":"GO: DIRECTION","graph":"{"nodes":[], "edges":{"north": [], "south": [], "east": [], "west": []}"}} and replace DIRECTION with one of [north, south, east, west]. To stop the exploration, answer with "DONE" instead. Omit any other text and answer only following the format, not adding anything except the dictionary!\\ \tt Here is an example: \\ \tt You are in the Living Room. Currently available directions: south, west. What is your next instruction?\\ \tt {"action":"GO: west", "graph": {"nodes":["Living Room"],"edges":{"north":[],"south":[],"east":[],"west":[]}}}\\ \tt You have made a step and entered a Library. Currently available directions: east, north. What is your next instruction?\\ \tt {"action":"GO: north", "graph":{"nodes":["Living Room", "Library"],"edges":{"north":[],"south":[],"east":[],"west":[("Living Room", "Library")]}}}\\ \tt You have made a step and entered a Kitchen. Currently available directions: south, east. What is your next instruction?\\ \tt {"action":"GO: east", "graph":{"nodes": ["Living Room", "Library", "Kitchen"], "edges":{"north": [("Library", "Kitchen")],"south": [],"east": [],"west": [("Living Room", "Library")]}}}\\ \tt ...\\ \tt You have made a step and entered a Bedroom. Currently available directions: south, west. What is your next instruction?\\ \tt {"action": "DONE", "graph": {...}}\\ \tt Let us start.\\ \tt You are in the Utility room. Currently available directions: south, east. What is your next instruction? 
	  } 
	   } 
	   } 
	 & \\ 
 

    \theutterance \stepcounter{utterance}  

    & \multicolumn{4}{p{0.6\linewidth}}{\cellcolor[rgb]{0.8,1,0.9}{%
	\makecell[{{p{\linewidth}}}]{% 
	  \tt {\tiny [A$\rangle$GM]}  
	 <think>\\ \tt Okay, let's see. The user is in the Utility room with directions south and east. The goal is to visit all rooms with the fewest room changes. So I need to figure out the best direction to go first.\\ \tt \\ \tt First, I should consider that each move counts as a room change. So, the strategy is to explore as much as possible without backtracking. But since the map isn't known yet, I have to make decisions based on available directions.\\ \tt \\ \tt In the Utility room, the options are south or east. Let's pick one. Maybe start with south. But wait, maybe there's a way to determine which direction leads to more rooms? Since I don't have the graph yet, I can't know. So the standard approach would be to choose one direction, explore, and then backtrack if needed.\\ \tt \\ \tt But the problem says to minimize room changes. So perhaps a depth-first search approach would be better, exploring as far as possible before backtracking. But since the user is asking for the first move, I need to choose between south or east. Let's pick south first. So the action would be "GO: south". Then, the graph would be updated with the Utility room and the next room. But since the next room isn't known yet, the edges would be added as the exploration progresses.\\ \tt \\ \tt Wait, the example shows that when moving west from Living Room to Library, the edges are updated. So in the initial step, the graph starts with the 
	  } 
	   } 
	   } 
	 & & \\ 
 

    \theutterance \stepcounter{utterance}  

    & & & \multicolumn{2}{p{0.3\linewidth}}{\cellcolor[rgb]{0.95,0.95,0.95}{%
	\makecell[{{p{\linewidth}}}]{% 
	  \tt {\tiny [GM$|$GM]}  
	 abort game 
	  } 
	   } 
	   } 
	 & & \\ 
 

\end{supertabular}
}

\end{document}


\documentclass{article}
\usepackage{colortbl}
\usepackage{makecell}
\usepackage{multirow}
\usepackage{supertabular}

\begin{document}

\newcounter{utterance}

\twocolumn

{ \footnotesize  \setcounter{utterance}{1}
\setlength{\tabcolsep}{0pt}
\begin{supertabular}{c@{$\;$}|p{.15\linewidth}@{}p{.15\linewidth}p{.15\linewidth}p{.15\linewidth}p{.15\linewidth}p{.15\linewidth}}

    \# & $\;$A & \multicolumn{4}{c}{Game Master} & $\;\:$B\\
    \hline 

    \theutterance \stepcounter{utterance}  

    & & \multicolumn{4}{p{0.6\linewidth}}{\cellcolor[rgb]{0.9,0.9,0.9}{%
	\makecell[{{p{\linewidth}}}]{% 
	  \tt {\tiny [A$\langle$GM]}  
	 You are given three grids, where each of them is 5 by 5 in size.\\ \tt Grids have empty cells marked with "▢" and filled cells marked with "X".\\ \tt Your task is to generate a referring expression that best describes the target grid while distinguishing it from the two other distractor grids.\\ \tt The first grid is the target grid, and the following two grids are the distractors.\\ \tt \\ \tt Target grid:\\ \tt \\ \tt X X X X X\\ \tt □ □ □ □ □\\ \tt X X X X □\\ \tt X □ □ □ □\\ \tt X □ □ □ □\\ \tt \\ \tt Distractor grid 1:\\ \tt \\ \tt X X X X X\\ \tt X □ □ □ □\\ \tt X X X X X\\ \tt X □ □ □ □\\ \tt X X X X X\\ \tt \\ \tt Distractor grid 2:\\ \tt \\ \tt X X X X X\\ \tt X □ □ □ □\\ \tt X □ □ □ □\\ \tt X □ □ □ □\\ \tt X X X X X\\ \tt \\ \tt Instruction: Describe the target grid.\\ \tt Generate the referring expression starting with the tag "Expression: " for the given target grid. Omit any other text. 
	  } 
	   } 
	   } 
	 & \\ 
 

    \theutterance \stepcounter{utterance}  

    & \multicolumn{4}{p{0.6\linewidth}}{\cellcolor[rgb]{0.8,1,0.9}{%
	\makecell[{{p{\linewidth}}}]{% 
	  \tt {\tiny [A$\rangle$GM]}  
	 <think>\\ \tt Okay, let's see. I need to describe the target grid in a way that distinguishes it from the two distractors. Let me start by looking at each grid carefully.\\ \tt \\ \tt First, the target grid:\\ \tt \\ \tt Row 1: All X's (X X X X X)\\ \tt Row 2: All squares (□ □ □ □ □)\\ \tt Row 3: Four X's and a square (X X X X □)\\ \tt Row 4: One X followed by four squares (X □ □ □ □)\\ \tt Row 5: Same as row 4 (X □ □ □ □)\\ \tt \\ \tt Now, the distractor 1:\\ \tt \\ \tt Row 1: All X's\\ \tt Row 2: X followed by four squares (X □ □ □ □)\\ \tt Row 3: All X's again\\ \tt Row 4: Same as row 2\\ \tt Row 5: All X's\\ \tt \\ \tt Distractor 2:\\ \tt \\ \tt Row 1: All X's\\ \tt Row 2: X followed by four squares\\ \tt Row 3: Same as row 2 (X □ □ □ □)\\ \tt Row 4: Same as row 2\\ \tt Row 5: All X's\\ \tt \\ \tt So, the target grid has row 2 completely empty (all squares), while the distractors have row 2 with an X in the first column. Also, in the target grid, rows 4 
	  } 
	   } 
	   } 
	 & & \\ 
 

    \theutterance \stepcounter{utterance}  

    & & & \multicolumn{2}{p{0.3\linewidth}}{\cellcolor[rgb]{0.95,0.95,0.95}{%
	\makecell[{{p{\linewidth}}}]{% 
	  \tt {\tiny [GM$|$GM]}  
	 Invalid generated expression 
	  } 
	   } 
	   } 
	 & & \\ 
 

\end{supertabular}
}

\end{document}

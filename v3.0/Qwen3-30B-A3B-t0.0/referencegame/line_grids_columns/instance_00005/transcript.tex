
\documentclass{article}
\usepackage{colortbl}
\usepackage{makecell}
\usepackage{multirow}
\usepackage{supertabular}

\begin{document}

\newcounter{utterance}

\twocolumn

{ \footnotesize  \setcounter{utterance}{1}
\setlength{\tabcolsep}{0pt}
\begin{supertabular}{c@{$\;$}|p{.15\linewidth}@{}p{.15\linewidth}p{.15\linewidth}p{.15\linewidth}p{.15\linewidth}p{.15\linewidth}}

    \# & $\;$A & \multicolumn{4}{c}{Game Master} & $\;\:$B\\
    \hline 

    \theutterance \stepcounter{utterance}  

    & & \multicolumn{4}{p{0.6\linewidth}}{\cellcolor[rgb]{0.9,0.9,0.9}{%
	\makecell[{{p{\linewidth}}}]{% 
	  \tt {\tiny [A$\langle$GM]}  
	 You are given three grids, where each of them is 5 by 5 in size.\\ \tt Grids have empty cells marked with "▢" and filled cells marked with "X".\\ \tt Your task is to generate a referring expression that best describes the target grid while distinguishing it from the two other distractor grids.\\ \tt The first grid is the target grid, and the following two grids are the distractors.\\ \tt \\ \tt Target grid:\\ \tt \\ \tt X □ □ X X\\ \tt X □ □ X X\\ \tt X □ □ X X\\ \tt X □ □ X X\\ \tt X □ □ X X\\ \tt \\ \tt Distractor grid 1:\\ \tt \\ \tt □ □ X X X\\ \tt □ □ X X X\\ \tt □ □ X X X\\ \tt □ □ X X X\\ \tt □ □ X X X\\ \tt \\ \tt Distractor grid 2:\\ \tt \\ \tt X □ X X □\\ \tt X □ X X □\\ \tt X □ X X □\\ \tt X □ X X □\\ \tt X □ X X □\\ \tt \\ \tt Instruction: Describe the target grid.\\ \tt Generate the referring expression starting with the tag "Expression: " for the given target grid. Omit any other text. 
	  } 
	   } 
	   } 
	 & \\ 
 

    \theutterance \stepcounter{utterance}  

    & \multicolumn{4}{p{0.6\linewidth}}{\cellcolor[rgb]{0.8,1,0.9}{%
	\makecell[{{p{\linewidth}}}]{% 
	  \tt {\tiny [A$\rangle$GM]}  
	 <think>\\ \tt Okay, let's see. I need to describe the target grid in a way that distinguishes it from the two distractors. Let me start by looking at each grid carefully.\\ \tt \\ \tt First, the target grid. It's a 5x5 grid. The first column is all X's. Then the next three columns have a pattern. Let me write it out:\\ \tt \\ \tt Row 1: X □ □ X X\\ \tt Row 2: X □ □ X X\\ \tt Row 3: X □ □ X X\\ \tt Row 4: X □ □ X X\\ \tt Row 5: X □ □ X X\\ \tt \\ \tt So, the first column is all X's. The second and third columns are all empty (□), and the fourth and fifth columns are X's. So, the first column is all X, then two empty, then two X's. So the pattern is X, □, □, X, X in each row.\\ \tt \\ \tt Now, looking at Distractor 1:\\ \tt \\ \tt All rows are □ □ X X X. So the first two columns are empty, then three X's. So the first column is all □, which is different from the target's first column being all X. So that's a key difference.\\ \tt \\ \tt Distractor 2:\\ \tt \\ \tt Each row is X □ X X □. So first column X, second □, third X, fourth X, fifth 
	  } 
	   } 
	   } 
	 & & \\ 
 

    \theutterance \stepcounter{utterance}  

    & & & \multicolumn{2}{p{0.3\linewidth}}{\cellcolor[rgb]{0.95,0.95,0.95}{%
	\makecell[{{p{\linewidth}}}]{% 
	  \tt {\tiny [GM$|$GM]}  
	 Invalid generated expression 
	  } 
	   } 
	   } 
	 & & \\ 
 

\end{supertabular}
}

\end{document}

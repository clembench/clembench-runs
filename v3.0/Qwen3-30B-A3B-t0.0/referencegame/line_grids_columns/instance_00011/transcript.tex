
\documentclass{article}
\usepackage{colortbl}
\usepackage{makecell}
\usepackage{multirow}
\usepackage{supertabular}

\begin{document}

\newcounter{utterance}

\twocolumn

{ \footnotesize  \setcounter{utterance}{1}
\setlength{\tabcolsep}{0pt}
\begin{supertabular}{c@{$\;$}|p{.15\linewidth}@{}p{.15\linewidth}p{.15\linewidth}p{.15\linewidth}p{.15\linewidth}p{.15\linewidth}}

    \# & $\;$A & \multicolumn{4}{c}{Game Master} & $\;\:$B\\
    \hline 

    \theutterance \stepcounter{utterance}  

    & & \multicolumn{4}{p{0.6\linewidth}}{\cellcolor[rgb]{0.9,0.9,0.9}{%
	\makecell[{{p{\linewidth}}}]{% 
	  \tt {\tiny [A$\langle$GM]}  
	 You are given three grids, where each of them is 5 by 5 in size.\\ \tt Grids have empty cells marked with "▢" and filled cells marked with "X".\\ \tt Your task is to generate a referring expression that best describes the target grid while distinguishing it from the two other distractor grids.\\ \tt The first grid is the target grid, and the following two grids are the distractors.\\ \tt \\ \tt Target grid:\\ \tt \\ \tt X X X □ □\\ \tt X X X □ □\\ \tt X X X □ □\\ \tt X X X □ □\\ \tt X X X □ □\\ \tt \\ \tt Distractor grid 1:\\ \tt \\ \tt □ X X X □\\ \tt □ X X X □\\ \tt □ X X X □\\ \tt □ X X X □\\ \tt □ X X X □\\ \tt \\ \tt Distractor grid 2:\\ \tt \\ \tt □ X X □ X\\ \tt □ X X □ X\\ \tt □ X X □ X\\ \tt □ X X □ X\\ \tt □ X X □ X\\ \tt \\ \tt Instruction: Describe the target grid.\\ \tt Generate the referring expression starting with the tag "Expression: " for the given target grid. Omit any other text. 
	  } 
	   } 
	   } 
	 & \\ 
 

    \theutterance \stepcounter{utterance}  

    & \multicolumn{4}{p{0.6\linewidth}}{\cellcolor[rgb]{0.8,1,0.9}{%
	\makecell[{{p{\linewidth}}}]{% 
	  \tt {\tiny [A$\rangle$GM]}  
	 <think>\\ \tt Okay, let's see. I need to create a referring expression for the target grid that differentiates it from the two distractors. Let me start by analyzing each grid.\\ \tt \\ \tt The target grid has the first three columns filled with X's, and the last two columns as empty (□). So each row is X X X □ □. That's consistent across all five rows. \\ \tt \\ \tt Now, looking at Distractor 1: The first column is empty (□), then the next three are X's, and the last is empty. So each row is □ X X X □. So the pattern here is a single empty, then three X's, then empty. That's different from the target, which has three X's first.\\ \tt \\ \tt Distractor 2: The grid has □ X X □ X in each row. So the pattern is empty, then two X's, empty, then X. So the arrangement here is different again. \\ \tt \\ \tt So the key difference for the target is that it has three X's in the first three columns, followed by two empty cells. The distractors have different arrangements. \\ \tt \\ \tt So the referring expression needs to highlight that the first three columns are all X's, and the last two are empty. Maybe something like "The first three columns are entirely filled with X's, and the last two columns are empty." But I need to make sure it's concise and distinct. \\ \tt \\ \tt Alternatively, since all rows 
	  } 
	   } 
	   } 
	 & & \\ 
 

    \theutterance \stepcounter{utterance}  

    & & & \multicolumn{2}{p{0.3\linewidth}}{\cellcolor[rgb]{0.95,0.95,0.95}{%
	\makecell[{{p{\linewidth}}}]{% 
	  \tt {\tiny [GM$|$GM]}  
	 Invalid generated expression 
	  } 
	   } 
	   } 
	 & & \\ 
 

\end{supertabular}
}

\end{document}

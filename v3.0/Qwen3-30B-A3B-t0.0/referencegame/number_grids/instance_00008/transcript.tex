
\documentclass{article}
\usepackage{colortbl}
\usepackage{makecell}
\usepackage{multirow}
\usepackage{supertabular}

\begin{document}

\newcounter{utterance}

\twocolumn

{ \footnotesize  \setcounter{utterance}{1}
\setlength{\tabcolsep}{0pt}
\begin{supertabular}{c@{$\;$}|p{.15\linewidth}@{}p{.15\linewidth}p{.15\linewidth}p{.15\linewidth}p{.15\linewidth}p{.15\linewidth}}

    \# & $\;$A & \multicolumn{4}{c}{Game Master} & $\;\:$B\\
    \hline 

    \theutterance \stepcounter{utterance}  

    & & \multicolumn{4}{p{0.6\linewidth}}{\cellcolor[rgb]{0.9,0.9,0.9}{%
	\makecell[{{p{\linewidth}}}]{% 
	  \tt {\tiny [A$\langle$GM]}  
	 You are given three grids, where each of them is 5 by 5 in size.\\ \tt Grids have empty cells marked with "▢" and filled cells marked with "X".\\ \tt Your task is to generate a referring expression that best describes the target grid while distinguishing it from the two other distractor grids.\\ \tt The first grid is the target grid, and the following two grids are the distractors.\\ \tt \\ \tt Target grid:\\ \tt \\ \tt X X X X X\\ \tt □ □ □ □ X\\ \tt □ □ □ □ X\\ \tt □ □ □ □ X\\ \tt □ □ □ □ X\\ \tt \\ \tt Distractor grid 1:\\ \tt \\ \tt □ X X X X\\ \tt □ X □ □ X\\ \tt □ X □ □ X\\ \tt □ X □ □ X\\ \tt □ X X X X\\ \tt \\ \tt Distractor grid 2:\\ \tt \\ \tt □ X X X X\\ \tt □ X □ □ X\\ \tt □ X X X X\\ \tt □ X □ □ X\\ \tt □ X X X X\\ \tt \\ \tt Instruction: Describe the target grid.\\ \tt Generate the referring expression starting with the tag "Expression: " for the given target grid. Omit any other text. 
	  } 
	   } 
	   } 
	 & \\ 
 

    \theutterance \stepcounter{utterance}  

    & \multicolumn{4}{p{0.6\linewidth}}{\cellcolor[rgb]{0.8,1,0.9}{%
	\makecell[{{p{\linewidth}}}]{% 
	  \tt {\tiny [A$\rangle$GM]}  
	 <think>\\ \tt Okay, let's see. I need to describe the target grid in a way that distinguishes it from the two distractors. Let me start by looking at each grid carefully.\\ \tt \\ \tt First, the target grid. It's a 5x5 grid. The first row is all X's. Then the next four rows have four empty cells (□) and then an X at the end. So each of the rows 2-5 has four □ followed by an X. So the rightmost column is all X's except the first row, which is all X's. Wait, no, the first row is all X's. So the first row is X X X X X. Then rows 2-5 have □ □ □ □ X. So the fifth column (since it's 5 columns) is X in all rows. Wait, no, the first row's fifth column is X, and rows 2-5 have X in the fifth column. So the entire fifth column is X's. But the first row is all X's. So the target grid has a full row of X's in the first row, and the fifth column is all X's. But maybe that's not the key difference.\\ \tt \\ \tt Now looking at Distractor 1. The first grid: the first row is □ X X X X. Then the next rows: □ X □ □ X, and so on. So the first 
	  } 
	   } 
	   } 
	 & & \\ 
 

    \theutterance \stepcounter{utterance}  

    & & & \multicolumn{2}{p{0.3\linewidth}}{\cellcolor[rgb]{0.95,0.95,0.95}{%
	\makecell[{{p{\linewidth}}}]{% 
	  \tt {\tiny [GM$|$GM]}  
	 Invalid generated expression 
	  } 
	   } 
	   } 
	 & & \\ 
 

\end{supertabular}
}

\end{document}

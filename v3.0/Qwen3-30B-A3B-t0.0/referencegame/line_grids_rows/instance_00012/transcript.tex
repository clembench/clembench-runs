
\documentclass{article}
\usepackage{colortbl}
\usepackage{makecell}
\usepackage{multirow}
\usepackage{supertabular}

\begin{document}

\newcounter{utterance}

\twocolumn

{ \footnotesize  \setcounter{utterance}{1}
\setlength{\tabcolsep}{0pt}
\begin{supertabular}{c@{$\;$}|p{.15\linewidth}@{}p{.15\linewidth}p{.15\linewidth}p{.15\linewidth}p{.15\linewidth}p{.15\linewidth}}

    \# & $\;$A & \multicolumn{4}{c}{Game Master} & $\;\:$B\\
    \hline 

    \theutterance \stepcounter{utterance}  

    & & \multicolumn{4}{p{0.6\linewidth}}{\cellcolor[rgb]{0.9,0.9,0.9}{%
	\makecell[{{p{\linewidth}}}]{% 
	  \tt {\tiny [A$\langle$GM]}  
	 You are given three grids, where each of them is 5 by 5 in size.\\ \tt Grids have empty cells marked with "▢" and filled cells marked with "X".\\ \tt Your task is to generate a referring expression that best describes the target grid while distinguishing it from the two other distractor grids.\\ \tt The first grid is the target grid, and the following two grids are the distractors.\\ \tt \\ \tt Target grid:\\ \tt \\ \tt X X □ X X\\ \tt X X □ X X\\ \tt X X □ X X\\ \tt X X □ X X\\ \tt X X □ X X\\ \tt \\ \tt Distractor grid 1:\\ \tt \\ \tt X □ □ X X\\ \tt X □ □ X X\\ \tt X □ □ X X\\ \tt X □ □ X X\\ \tt X □ □ X X\\ \tt \\ \tt Distractor grid 2:\\ \tt \\ \tt □ □ □ X X\\ \tt □ □ □ X X\\ \tt □ □ □ X X\\ \tt □ □ □ X X\\ \tt □ □ □ X X\\ \tt \\ \tt Instruction: Describe the target grid.\\ \tt Generate the referring expression starting with the tag "Expression: " for the given target grid. Omit any other text. 
	  } 
	   } 
	   } 
	 & \\ 
 

    \theutterance \stepcounter{utterance}  

    & \multicolumn{4}{p{0.6\linewidth}}{\cellcolor[rgb]{0.8,1,0.9}{%
	\makecell[{{p{\linewidth}}}]{% 
	  \tt {\tiny [A$\rangle$GM]}  
	 <think>\\ \tt Okay, let's see. I need to describe the target grid in a way that distinguishes it from the two distractors. Let me start by looking at each grid carefully.\\ \tt \\ \tt The target grid is 5x5. Looking at the rows, each row is X X □ X X. So the first two cells are X, then a blank, then two Xs. This pattern repeats for all five rows. So the third column is all blanks, and the rest are Xs. Wait, no, each row has the third cell as □, but the first two and last two are X. So the third column is all □? Let me check. Each row's third cell is □, so yes, the entire third column is empty. The rest of the cells are X. So the target grid has a vertical line of empty cells in the middle, column 3.\\ \tt \\ \tt Now, Distractor 1: Each row is X □ □ X X. So first cell X, then two blanks, then two Xs. So the second and third columns are blanks. So the middle two columns are empty. So the third column here is □, but in the target, the third column is □. Wait, but in the target, the third column is all □, but in Distractor 1, the second and third columns are □. So the target has only column 3 as empty, while 
	  } 
	   } 
	   } 
	 & & \\ 
 

    \theutterance \stepcounter{utterance}  

    & & & \multicolumn{2}{p{0.3\linewidth}}{\cellcolor[rgb]{0.95,0.95,0.95}{%
	\makecell[{{p{\linewidth}}}]{% 
	  \tt {\tiny [GM$|$GM]}  
	 Invalid generated expression 
	  } 
	   } 
	   } 
	 & & \\ 
 

\end{supertabular}
}

\end{document}


\documentclass{article}
\usepackage{colortbl}
\usepackage{makecell}
\usepackage{multirow}
\usepackage{supertabular}

\begin{document}

\newcounter{utterance}

\twocolumn

{ \footnotesize  \setcounter{utterance}{1}
\setlength{\tabcolsep}{0pt}
\begin{supertabular}{c@{$\;$}|p{.15\linewidth}@{}p{.15\linewidth}p{.15\linewidth}p{.15\linewidth}p{.15\linewidth}p{.15\linewidth}}

    \# & $\;$A & \multicolumn{4}{c}{Game Master} & $\;\:$B\\
    \hline 

    \theutterance \stepcounter{utterance}  

    & & \multicolumn{4}{p{0.6\linewidth}}{\cellcolor[rgb]{0.9,0.9,0.9}{%
	\makecell[{{p{\linewidth}}}]{% 
	  \tt {\tiny [A$\langle$GM]}  
	 You are given three grids, where each of them is 5 by 5 in size.\\ \tt Grids have empty cells marked with "▢" and filled cells marked with "X".\\ \tt Your task is to generate a referring expression that best describes the target grid while distinguishing it from the two other distractor grids.\\ \tt The first grid is the target grid, and the following two grids are the distractors.\\ \tt \\ \tt Target grid:\\ \tt \\ \tt □ □ X X X\\ \tt □ □ X X X\\ \tt □ □ X X X\\ \tt □ □ X X X\\ \tt □ □ X X X\\ \tt \\ \tt Distractor grid 1:\\ \tt \\ \tt □ □ □ X X\\ \tt □ □ □ X X\\ \tt □ □ □ X X\\ \tt □ □ □ X X\\ \tt □ □ □ X X\\ \tt \\ \tt Distractor grid 2:\\ \tt \\ \tt X □ □ X X\\ \tt X □ □ X X\\ \tt X □ □ X X\\ \tt X □ □ X X\\ \tt X □ □ X X\\ \tt \\ \tt Instruction: Describe the target grid.\\ \tt Generate the referring expression starting with the tag "Expression: " for the given target grid. Omit any other text. 
	  } 
	   } 
	   } 
	 & \\ 
 

    \theutterance \stepcounter{utterance}  

    & \multicolumn{4}{p{0.6\linewidth}}{\cellcolor[rgb]{0.8,1,0.9}{%
	\makecell[{{p{\linewidth}}}]{% 
	  \tt {\tiny [A$\rangle$GM]}  
	 <think>\\ \tt Okay, let's see. I need to create a referring expression for the target grid that differentiates it from the two distractors. Let me start by analyzing each grid.\\ \tt \\ \tt The target grid has the first two columns as empty (□) and the last three as X. So, rows are like □ □ X X X. All rows are the same. Now, looking at the distractors.\\ \tt \\ \tt Distractor 1 has the first three columns as □ and then X X. So, □ □ □ X X. So, the first three columns are empty, which is different from the target's first two. Distractor 2 has the first column as X, then □ □, then X X. So, X □ □ X X. So, the first column is filled, which is different from the target where the first two are empty.\\ \tt \\ \tt So the key difference is the number of empty columns on the left. Target has two empty columns, then three Xs. Distractor 1 has three empty, then two Xs. Distractor 2 has one X, then two empty, then two Xs.\\ \tt \\ \tt So the referring expression should mention the first two columns being empty and the last three filled. But I need to phrase it in a way that distinguishes it from the others. Maybe something like "The first two columns are empty, followed by three filled columns." But maybe more precise.\\ \tt \\ \tt Alternatively 
	  } 
	   } 
	   } 
	 & & \\ 
 

    \theutterance \stepcounter{utterance}  

    & & & \multicolumn{2}{p{0.3\linewidth}}{\cellcolor[rgb]{0.95,0.95,0.95}{%
	\makecell[{{p{\linewidth}}}]{% 
	  \tt {\tiny [GM$|$GM]}  
	 Invalid generated expression 
	  } 
	   } 
	   } 
	 & & \\ 
 

\end{supertabular}
}

\end{document}

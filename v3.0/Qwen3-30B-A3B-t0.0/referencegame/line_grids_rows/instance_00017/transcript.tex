
\documentclass{article}
\usepackage{colortbl}
\usepackage{makecell}
\usepackage{multirow}
\usepackage{supertabular}

\begin{document}

\newcounter{utterance}

\twocolumn

{ \footnotesize  \setcounter{utterance}{1}
\setlength{\tabcolsep}{0pt}
\begin{supertabular}{c@{$\;$}|p{.15\linewidth}@{}p{.15\linewidth}p{.15\linewidth}p{.15\linewidth}p{.15\linewidth}p{.15\linewidth}}

    \# & $\;$A & \multicolumn{4}{c}{Game Master} & $\;\:$B\\
    \hline 

    \theutterance \stepcounter{utterance}  

    & & \multicolumn{4}{p{0.6\linewidth}}{\cellcolor[rgb]{0.9,0.9,0.9}{%
	\makecell[{{p{\linewidth}}}]{% 
	  \tt {\tiny [A$\langle$GM]}  
	 You are given three grids, where each of them is 5 by 5 in size.\\ \tt Grids have empty cells marked with "▢" and filled cells marked with "X".\\ \tt Your task is to generate a referring expression that best describes the target grid while distinguishing it from the two other distractor grids.\\ \tt The first grid is the target grid, and the following two grids are the distractors.\\ \tt \\ \tt Target grid:\\ \tt \\ \tt □ X X □ X\\ \tt □ X X □ X\\ \tt □ X X □ X\\ \tt □ X X □ X\\ \tt □ X X □ X\\ \tt \\ \tt Distractor grid 1:\\ \tt \\ \tt □ □ X X X\\ \tt □ □ X X X\\ \tt □ □ X X X\\ \tt □ □ X X X\\ \tt □ □ X X X\\ \tt \\ \tt Distractor grid 2:\\ \tt \\ \tt X X X X X\\ \tt □ □ □ □ □\\ \tt □ □ □ □ □\\ \tt X X X X X\\ \tt □ □ □ □ □\\ \tt \\ \tt Instruction: Describe the target grid.\\ \tt Generate the referring expression starting with the tag "Expression: " for the given target grid. Omit any other text. 
	  } 
	   } 
	   } 
	 & \\ 
 

    \theutterance \stepcounter{utterance}  

    & \multicolumn{4}{p{0.6\linewidth}}{\cellcolor[rgb]{0.8,1,0.9}{%
	\makecell[{{p{\linewidth}}}]{% 
	  \tt {\tiny [A$\rangle$GM]}  
	 <think>\\ \tt Okay, let's see. I need to describe the target grid in a way that distinguishes it from the two distractors. Let me start by looking at each grid carefully.\\ \tt \\ \tt First, the target grid. Each row is □ X X □ X. So, in each row, the first cell is empty, then two Xs, then an empty, then an X. So, positions 1, 4 are empty, and 2,3,5 are filled. Let me check all rows. Yes, all five rows are the same: □ X X □ X. So the target has a consistent pattern across all rows.\\ \tt \\ \tt Now, Distractor 1: Each row is □ □ X X X. So first two cells empty, then three Xs. All rows same. So the pattern here is different from the target. The target has two Xs in the middle, but here it's three Xs at the end.\\ \tt \\ \tt Distractor 2: The first row is all Xs. Then the next three rows are all empty. Then the fifth row is all Xs again. So it's a checkerboard of sorts but in blocks. The target doesn't have any rows that are all X or all empty. So that's different.\\ \tt \\ \tt So the target's key feature is that each row has the same pattern: empty, X, X, empty, X. So maybe I can describe it as having rows with 
	  } 
	   } 
	   } 
	 & & \\ 
 

    \theutterance \stepcounter{utterance}  

    & & & \multicolumn{2}{p{0.3\linewidth}}{\cellcolor[rgb]{0.95,0.95,0.95}{%
	\makecell[{{p{\linewidth}}}]{% 
	  \tt {\tiny [GM$|$GM]}  
	 Invalid generated expression 
	  } 
	   } 
	   } 
	 & & \\ 
 

\end{supertabular}
}

\end{document}

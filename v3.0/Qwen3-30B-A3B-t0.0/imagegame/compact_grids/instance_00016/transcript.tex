
\documentclass{article}
\usepackage{colortbl}
\usepackage{makecell}
\usepackage{multirow}
\usepackage{supertabular}

\begin{document}

\newcounter{utterance}

\twocolumn

{ \footnotesize  \setcounter{utterance}{1}
\setlength{\tabcolsep}{0pt}
\begin{supertabular}{c@{$\;$}|p{.15\linewidth}@{}p{.15\linewidth}p{.15\linewidth}p{.15\linewidth}p{.15\linewidth}p{.15\linewidth}}

    \# & $\;$A & \multicolumn{4}{c}{Game Master} & $\;\:$B\\
    \hline 

    \theutterance \stepcounter{utterance}  

    & & \multicolumn{4}{p{0.6\linewidth}}{\cellcolor[rgb]{0.9,0.9,0.9}{%
	\makecell[{{p{\linewidth}}}]{% 
	  \tt {\tiny [A$\langle$GM]}  
	 Let us do the following task. The goal is to fill an empty grid that looks like this:\\ \tt \\ \tt ▢ ▢ ▢ ▢ ▢\\ \tt ▢ ▢ ▢ ▢ ▢\\ \tt ▢ ▢ ▢ ▢ ▢\\ \tt ▢ ▢ ▢ ▢ ▢\\ \tt ▢ ▢ ▢ ▢ ▢\\ \tt \\ \tt A filled grid below is 5 by 5 and can look like this:\\ \tt \\ \tt ▢ ▢ ▢ ▢ ▢\\ \tt ▢ ▢ E ▢ ▢\\ \tt ▢ ▢ ▢ ▢ ▢\\ \tt ▢ ▢ ▢ ▢ ▢\\ \tt X X X X X\\ \tt \\ \tt I want you to describe this grid to me, step by step. You don't need to describe the empty squares, which are denoted with "▢". Only describe the location of letters in the grid. Then you wait for me to say "What is your next command?", and then you continue with the next step. Take the size of the grid into consideration while giving commands. You can give only one command at each turn but it can refer to single or multiple cells, rows, columns or even the whole grid at once. When you have described everything, you say "Command: DONE". Only give the command and omit everything else.\\ \tt \\ \tt For the filled grid above, here are the example steps.\\ \tt \\ \tt Command: Put an E in second row third column\\ \tt \\ \tt What is your next command?\\ \tt Command: Fill the last row with X\\ \tt \\ \tt What is your next command?\\ \tt Command: DONE\\ \tt \\ \tt Ok. Please do this for the following example, which is a 5 by 5 grid.\\ \tt N N N N N\\ \tt N ▢ ▢ ▢ N\\ \tt N ▢ N ▢ N\\ \tt N ▢ ▢ ▢ N\\ \tt N N N N N\\ \tt What is your next command? 
	  } 
	   } 
	   } 
	 & \\ 
 

    \theutterance \stepcounter{utterance}  

    & \multicolumn{4}{p{0.6\linewidth}}{\cellcolor[rgb]{0.8,1,0.9}{%
	\makecell[{{p{\linewidth}}}]{% 
	  \tt {\tiny [A$\rangle$GM]}  
	 <think>\\ \tt Okay, let's see. The user wants me to describe how to fill a 5x5 grid step by step, but only mention the non-empty cells. The example they gave had an E in the second row third column and the last row filled with Xs. Now the new grid they provided has N's on the borders and some N's inside. Let me look at the grid again:\\ \tt \\ \tt The grid is:\\ \tt \\ \tt Row 1: N N N N N\\ \tt Row 2: N ▢ ▢ ▢ N\\ \tt Row 3: N ▢ N ▢ N\\ \tt Row 4: N ▢ ▢ ▢ N\\ \tt Row 5: N N N N N\\ \tt \\ \tt So the task is to describe the locations of the letters (N's) in the grid. The user wants me to give commands step by step, each time only one command, and after all steps, say DONE.\\ \tt \\ \tt First, I need to figure out which cells are filled with N. The first row is all N's. The last row is all N's. Then, in the second row, the first and last cells are N. Similarly for rows 3 and 4. In row 3, the third cell is N. So maybe the pattern is that the borders are N's, and there's an N in the center of row 3.\\ \tt \\ \tt But how to translate this into commands. The user's example had putting an E in a specific cell and 
	  } 
	   } 
	   } 
	 & & \\ 
 

    \theutterance \stepcounter{utterance}  

    & & & \multicolumn{2}{p{0.3\linewidth}}{\cellcolor[rgb]{0.95,0.95,0.95}{%
	\makecell[{{p{\linewidth}}}]{% 
	  \tt {\tiny [GM$|$GM]}  
	 Invalid instruction format 
	  } 
	   } 
	   } 
	 & & \\ 
 

\end{supertabular}
}

\end{document}

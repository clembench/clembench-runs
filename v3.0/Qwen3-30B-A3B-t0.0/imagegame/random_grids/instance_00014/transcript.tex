
\documentclass{article}
\usepackage{colortbl}
\usepackage{makecell}
\usepackage{multirow}
\usepackage{supertabular}

\begin{document}

\newcounter{utterance}

\twocolumn

{ \footnotesize  \setcounter{utterance}{1}
\setlength{\tabcolsep}{0pt}
\begin{supertabular}{c@{$\;$}|p{.15\linewidth}@{}p{.15\linewidth}p{.15\linewidth}p{.15\linewidth}p{.15\linewidth}p{.15\linewidth}}

    \# & $\;$A & \multicolumn{4}{c}{Game Master} & $\;\:$B\\
    \hline 

    \theutterance \stepcounter{utterance}  

    & & \multicolumn{4}{p{0.6\linewidth}}{\cellcolor[rgb]{0.9,0.9,0.9}{%
	\makecell[{{p{\linewidth}}}]{% 
	  \tt {\tiny [A$\langle$GM]}  
	 Let us do the following task. The goal is to fill an empty grid that looks like this:\\ \tt \\ \tt ▢ ▢ ▢ ▢ ▢\\ \tt ▢ ▢ ▢ ▢ ▢\\ \tt ▢ ▢ ▢ ▢ ▢\\ \tt ▢ ▢ ▢ ▢ ▢\\ \tt ▢ ▢ ▢ ▢ ▢\\ \tt \\ \tt A filled grid below is 5 by 5 and can look like this:\\ \tt \\ \tt ▢ ▢ ▢ ▢ ▢\\ \tt ▢ ▢ E ▢ ▢\\ \tt ▢ ▢ ▢ ▢ ▢\\ \tt ▢ ▢ ▢ ▢ ▢\\ \tt X X X X X\\ \tt \\ \tt I want you to describe this grid to me, step by step. You don't need to describe the empty squares, which are denoted with "▢". Only describe the location of letters in the grid. Then you wait for me to say "What is your next command?", and then you continue with the next step. Take the size of the grid into consideration while giving commands. You can give only one command at each turn but it can refer to single or multiple cells, rows, columns or even the whole grid at once. When you have described everything, you say "Command: DONE". Only give the command and omit everything else.\\ \tt \\ \tt For the filled grid above, here are the example steps.\\ \tt \\ \tt Command: Put an E in second row third column\\ \tt \\ \tt What is your next command?\\ \tt Command: Fill the last row with X\\ \tt \\ \tt What is your next command?\\ \tt Command: DONE\\ \tt \\ \tt Ok. Please do this for the following example, which is a 5 by 5 grid.\\ \tt ▢ ▢ ▢ ▢ ▢\\ \tt ▢ ▢ B B B\\ \tt ▢ ▢ B ▢ ▢\\ \tt ▢ ▢ ▢ B ▢\\ \tt ▢ ▢ ▢ ▢ ▢\\ \tt What is your next command? 
	  } 
	   } 
	   } 
	 & \\ 
 

    \theutterance \stepcounter{utterance}  

    & \multicolumn{4}{p{0.6\linewidth}}{\cellcolor[rgb]{0.8,1,0.9}{%
	\makecell[{{p{\linewidth}}}]{% 
	  \tt {\tiny [A$\rangle$GM]}  
	 <think>\\ \tt Okay, let's see. The user wants me to describe the grid step by step, only mentioning the letters and their positions. The grid is 5x5. The example given has an E in the second row third column and the last row filled with Xs. So, for the new grid provided, I need to figure out the commands.\\ \tt \\ \tt Looking at the grid:\\ \tt \\ \tt Row 1: All empty.\\ \tt Row 2: Third, fourth, fifth columns have B's.\\ \tt Row 3: Third column has a B.\\ \tt Row 4: Fourth column has a B.\\ \tt Row 5: All empty.\\ \tt \\ \tt So, the letters are in positions (2,3), (2,4), (2,5), (3,3), (4,4). Wait, the user's example had "B B B" in the second row, which would be columns 3,4,5. Then row 3 has "B" in column 3, row 4 has "B" in column 4. So the commands should be to place those Bs.\\ \tt \\ \tt But the user's instruction says to describe the location of letters, step by step. So first, maybe start with the second row. The second row has three Bs starting from column 3. So the first command could be "Fill columns 3-5 of row 2 with B". But the example used "Put an E in second row third column" and then "Fill the last 
	  } 
	   } 
	   } 
	 & & \\ 
 

    \theutterance \stepcounter{utterance}  

    & & & \multicolumn{2}{p{0.3\linewidth}}{\cellcolor[rgb]{0.95,0.95,0.95}{%
	\makecell[{{p{\linewidth}}}]{% 
	  \tt {\tiny [GM$|$GM]}  
	 Invalid instruction format 
	  } 
	   } 
	   } 
	 & & \\ 
 

\end{supertabular}
}

\end{document}


\documentclass{article}
\usepackage{colortbl}
\usepackage{makecell}
\usepackage{multirow}
\usepackage{supertabular}

\begin{document}

\newcounter{utterance}

\twocolumn

{ \footnotesize  \setcounter{utterance}{1}
\setlength{\tabcolsep}{0pt}
\begin{supertabular}{c@{$\;$}|p{.15\linewidth}@{}p{.15\linewidth}p{.15\linewidth}p{.15\linewidth}p{.15\linewidth}p{.15\linewidth}}

    \# & $\;$A & \multicolumn{4}{c}{Game Master} & $\;\:$B\\
    \hline 

    \theutterance \stepcounter{utterance}  

    & & \multicolumn{4}{p{0.6\linewidth}}{\cellcolor[rgb]{0.9,0.9,0.9}{%
	\makecell[{{p{\linewidth}}}]{% 
	  \tt {\tiny [A$\langle$GM]}  
	 You are a language wizard who likes to guess words by using the given rules.\\ \tt \\ \tt Welcome to Wordle! You have six attempts to guess the target word, a valid English word of five lowercase letters (a-z). Please use the tags "explanation:" and "guess:" to provide a concise explanation for each guess.\\ \tt \\ \tt To help you make an informed guess, you will receive a clue for the word, such as\\ \tt clue: snowy white.\\ \tt \\ \tt Here is an example guess based on the clue:\\ \tt explanation: In the fairy tail Snow White, the girl is killed because she eats a poisoned apple. And the word apple has 5 letters.\\ \tt guess: apple\\ \tt \\ \tt After each guess, your answer will be validated, and you will receive feedback indicating which letters are correct (green), which letters are correct but in the wrong position (yellow), and which letters are incorrect (red). This feedback can be useful in determining which letters to include or exclude in your next guess.\\ \tt \\ \tt For example, the feedback for "apple" might be:\\ \tt guess_feedback: a<yellow> p<yellow> p<green> l<yellow> e<red>\\ \tt \\ \tt The explanation should contain details about how both the guess_feedback and clue work together in tandem to formulate a new guess.\\ \tt \\ \tt Do not generate any other text.\\ \tt \\ \tt Let's begin with your first guess.\\ \tt \\ \tt clue: plastics base 
	  } 
	   } 
	   } 
	 & \\ 
 

    \theutterance \stepcounter{utterance}  

    & \multicolumn{4}{p{0.6\linewidth}}{\cellcolor[rgb]{0.8,1,0.9}{%
	\makecell[{{p{\linewidth}}}]{% 
	  \tt {\tiny [A$\rangle$GM]}  
	 explanation: The clue "plastics base" suggests a fundamental chemical compound used in plastic production. A common base is "ethylene," but it's 8 letters. A 5-letter candidate is "ethane," but it's not the primary base. "Propene" is 7. The most likely 5-letter base is "propylene" shortened? No. Wait — "monomer" is 7. Perhaps "petro" as in petroleum? But "petro" is 5 and used in plastics. "Petro" is a root, but is it a word? Alternatively, "plastic" itself is 7. Let’s think of "polymer" — 7. Wait — "ethylene" is too long. The most common 5-letter hydrocarbon base in plastics is "propene" — too long. "Ethene" is 6. "Butane" is 6. "Methane" is 7. What about "alkyl"? Too vague. Perhaps "vinyl" — yes! Vinyl is a base unit in plastics like PVC. "Vinyl" is 5 letters and directly associated with plastics.  \\ \tt guess: vinyl 
	  } 
	   } 
	   } 
	 & & \\ 
 

    \theutterance \stepcounter{utterance}  

    & & & \multicolumn{2}{p{0.3\linewidth}}{\cellcolor[rgb]{0.95,0.95,0.95}{%
	\makecell[{{p{\linewidth}}}]{% 
	  \tt {\tiny [GM$|$GM]}  
	 attempts = 1\\ \tt target_word = resin\\ \tt target_word_clue = plastics base\\ \tt guess = vinyl\\ \tt guess_feedback = v<red> i<yellow> n<yellow> y<red> l<red> 
	  } 
	   } 
	   } 
	 & & \\ 
 

    \theutterance \stepcounter{utterance}  

    & & \multicolumn{4}{p{0.6\linewidth}}{\cellcolor[rgb]{0.9,0.9,0.9}{%
	\makecell[{{p{\linewidth}}}]{% 
	  \tt {\tiny [A$\langle$GM]}  
	 guess_feedback: v<red> i<yellow> n<yellow> y<red> l<red>\\ \tt \\ \tt Provide your response only in this format.\\ \tt explanation: details\\ \tt guess: word 
	  } 
	   } 
	   } 
	 & \\ 
 

    \theutterance \stepcounter{utterance}  

    & \multicolumn{4}{p{0.6\linewidth}}{\cellcolor[rgb]{0.8,1,0.9}{%
	\makecell[{{p{\linewidth}}}]{% 
	  \tt {\tiny [A$\rangle$GM]}  
	 explanation: The feedback shows 'i' and 'n' are correct but misplaced, while 'v', 'y', and 'l' are incorrect. This eliminates "vinyl" but confirms the word contains 'i' and 'n'. The clue "plastics base" still points to a chemical building block. Common 5-letter plastic bases include "propene" (too long), "ethylene" (too long), but "ethylene" shortened? "Ethyl" is 5 letters and a key group in plastics like polyethylene. 'E', 't', 'h', 'y', 'l' — but 'y' and 'l' were red in "vinyl", so "ethyl" would reuse red letters. Wait — perhaps "monomer" is too long. "Polymer" is 7. "Styrene" is 7. What about "butene"? 6. "Ethene" is 6. "Propyl" is 6. "Isoprop" is 7. "Acryl" — 5 letters, and acrylic plastics are common. 'a','c','r','y','l' — but 'y' and 'l' were red, so invalid. 'i' and 'n' must be in the word. Let’s think: "nitrile" — 7. "Pentene" — 7. "Butadiene" — 9. " 
	  } 
	   } 
	   } 
	 & & \\ 
 

    \theutterance \stepcounter{utterance}  

    & & & \multicolumn{2}{p{0.3\linewidth}}{\cellcolor[rgb]{0.95,0.95,0.95}{%
	\makecell[{{p{\linewidth}}}]{% 
	  \tt {\tiny [GM$|$GM]}  
	 Error: The response should contain the 'guess:' keyword exactly once. 
	  } 
	   } 
	   } 
	 & & \\ 
 

    \theutterance \stepcounter{utterance}  

    & & & \multicolumn{2}{p{0.3\linewidth}}{\cellcolor[rgb]{0.95,0.95,0.95}{%
	\makecell[{{p{\linewidth}}}]{% 
	  \tt {\tiny [GM$|$GM]}  
	 game_result = ABORT 
	  } 
	   } 
	   } 
	 & & \\ 
 

\end{supertabular}
}

\end{document}

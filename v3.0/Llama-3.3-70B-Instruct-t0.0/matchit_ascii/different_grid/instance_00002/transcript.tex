
\documentclass{article}
\usepackage{colortbl}
\usepackage{makecell}
\usepackage{multirow}
\usepackage{supertabular}

\begin{document}

\newcounter{utterance}

\twocolumn

{ \footnotesize  \setcounter{utterance}{1}
\setlength{\tabcolsep}{0pt}
\begin{supertabular}{c@{$\;$}|p{.15\linewidth}@{}p{.15\linewidth}p{.15\linewidth}p{.15\linewidth}p{.15\linewidth}p{.15\linewidth}}

    \# & $\;$A & \multicolumn{4}{c}{Game Master} & $\;\:$B\\
    \hline 

    \theutterance \stepcounter{utterance}  

    & & \multicolumn{4}{p{0.6\linewidth}}{\cellcolor[rgb]{0.9,0.9,0.9}{%
	\makecell[{{p{\linewidth}}}]{% 
	  \tt {\tiny [A$\langle$GM]}  
	 You are participating in a collaborative guessing game.\\ \tt The goal is to find out whether this grid and another picture only I can see, are the same.\\ \tt Please describe your grid first. Then, I will provide my description and we can ask each other questions about the grids to figure out whether they are the same. Now start your short grid description with "DESCRIPTION:" followed by the description. Do not add anything else. This is your grid:\\ \tt X ▢ ▢ ▢ X\\ \tt ▢ X ▢ ▢ X\\ \tt ▢ ▢ X X X\\ \tt ▢ X ▢ ▢ X\\ \tt X ▢ ▢ ▢ X 
	  } 
	   } 
	   } 
	 & \\ 
 

    \theutterance \stepcounter{utterance}  

    & \multicolumn{4}{p{0.6\linewidth}}{\cellcolor[rgb]{0.8,1,0.9}{%
	\makecell[{{p{\linewidth}}}]{% 
	  \tt {\tiny [A$\rangle$GM]}  
	 DESCRIPTION: A 5x5 grid with X's and blank squares, where X's are scattered throughout the grid in a seemingly random pattern, with some rows having multiple X's and others having only one or two, and the first and last rows having X's in the first and last positions. 
	  } 
	   } 
	   } 
	 & & \\ 
 

    \theutterance \stepcounter{utterance}  

    & & & \multicolumn{2}{p{0.3\linewidth}}{\cellcolor[rgb]{0.95,0.95,0.95}{%
	\makecell[{{p{\linewidth}}}]{% 
	  \tt {\tiny [GM$|$GM]}  
	 continue 
	  } 
	   } 
	   } 
	 & & \\ 
 

    \theutterance \stepcounter{utterance}  

    & & \multicolumn{4}{p{0.6\linewidth}}{\cellcolor[rgb]{0.9,0.9,0.9}{%
	\makecell[{{p{\linewidth}}}]{% 
	  \tt {\tiny [GM$\rangle$B]}  
	 You are participating in a collaborative guessing game.\\ \tt The goal is to find out whether this grid and another picture only I can see, are the same.\\ \tt Please describe your grid first. Then, I will provide my description and we can ask each other questions about the grids to figure out whether they are the same. Now start your short grid description with "DESCRIPTION:" followed by the description. Do not add anything else. This is your grid:\\ \tt X X X X ▢\\ \tt X X X ▢ X\\ \tt X X ▢ X X\\ \tt X ▢ X X X\\ \tt ▢ X X X X 
	  } 
	   } 
	   } 
	 & \\ 
 

    \theutterance \stepcounter{utterance}  

    & & & \multicolumn{4}{p{0.6\linewidth}}{\cellcolor[rgb]{1,0.85,0.72}{%
	\makecell[{{p{\linewidth}}}]{% 
	  \tt {\tiny [GM$\langle$B]}  
	 DESCRIPTION: A 5x5 grid with X's and ▢'s, where each row has 4 X's and 1 ▢, and the ▢ moves one position to the right in each subsequent row, starting from the top left and moving down. 
	  } 
	   } 
	   } 
	  \\ 
 

    \theutterance \stepcounter{utterance}  

    & & & \multicolumn{2}{p{0.3\linewidth}}{\cellcolor[rgb]{0.95,0.95,0.95}{%
	\makecell[{{p{\linewidth}}}]{% 
	  \tt {\tiny [GM$|$GM]}  
	 continue 
	  } 
	   } 
	   } 
	 & & \\ 
 

    \theutterance \stepcounter{utterance}  

    & & \multicolumn{4}{p{0.6\linewidth}}{\cellcolor[rgb]{0.9,0.9,0.9}{%
	\makecell[{{p{\linewidth}}}]{% 
	  \tt {\tiny [GM$\rangle$B]}  
	 This is my DESCRIPTION: A 5x5 grid with X's and blank squares, where X's are scattered throughout the grid in a seemingly random pattern, with some rows having multiple X's and others having only one or two, and the first and last rows having X's in the first and last positions.\\ \tt Now ask a question in order to find out new aspects of my grid that may be different to your grid. Start with "QUESTION:" and do not add anything else. 
	  } 
	   } 
	   } 
	 & \\ 
 

    \theutterance \stepcounter{utterance}  

    & & & \multicolumn{4}{p{0.6\linewidth}}{\cellcolor[rgb]{1,0.85,0.72}{%
	\makecell[{{p{\linewidth}}}]{% 
	  \tt {\tiny [GM$\langle$B]}  
	 QUESTION: Are the blank squares in your grid symmetrically arranged or is the pattern of X's and blank squares unique and not mirrored in any way? 
	  } 
	   } 
	   } 
	  \\ 
 

    \theutterance \stepcounter{utterance}  

    & & & \multicolumn{2}{p{0.3\linewidth}}{\cellcolor[rgb]{0.95,0.95,0.95}{%
	\makecell[{{p{\linewidth}}}]{% 
	  \tt {\tiny [GM$|$GM]}  
	 continue 
	  } 
	   } 
	   } 
	 & & \\ 
 

    \theutterance \stepcounter{utterance}  

    & & \multicolumn{4}{p{0.6\linewidth}}{\cellcolor[rgb]{0.9,0.9,0.9}{%
	\makecell[{{p{\linewidth}}}]{% 
	  \tt {\tiny [A$\langle$GM]}  
	 This is my DESCRIPTION: A 5x5 grid with X's and ▢'s, where each row has 4 X's and 1 ▢, and the ▢ moves one position to the right in each subsequent row, starting from the top left and moving down.\\ \tt QUESTION: Are the blank squares in your grid symmetrically arranged or is the pattern of X's and blank squares unique and not mirrored in any way? Start your answer with "ANSWER:" and do not add anything else. 
	  } 
	   } 
	   } 
	 & \\ 
 

    \theutterance \stepcounter{utterance}  

    & \multicolumn{4}{p{0.6\linewidth}}{\cellcolor[rgb]{0.8,1,0.9}{%
	\makecell[{{p{\linewidth}}}]{% 
	  \tt {\tiny [A$\rangle$GM]}  
	 ANSWER: The blank squares in my grid are symmetrically arranged, with the pattern of X's and blank squares being mirrored along the vertical axis, and also having a similar symmetry along the horizontal axis, particularly in the first and last rows. 
	  } 
	   } 
	   } 
	 & & \\ 
 

    \theutterance \stepcounter{utterance}  

    & & & \multicolumn{2}{p{0.3\linewidth}}{\cellcolor[rgb]{0.95,0.95,0.95}{%
	\makecell[{{p{\linewidth}}}]{% 
	  \tt {\tiny [GM$|$GM]}  
	 continue 
	  } 
	   } 
	   } 
	 & & \\ 
 

    \theutterance \stepcounter{utterance}  

    & & \multicolumn{4}{p{0.6\linewidth}}{\cellcolor[rgb]{0.9,0.9,0.9}{%
	\makecell[{{p{\linewidth}}}]{% 
	  \tt {\tiny [A$\langle$GM]}  
	 Now ask a question in order to find out new aspects of my grid that may be different to your grid. Start with "QUESTION:" and do not add anything else. 
	  } 
	   } 
	   } 
	 & \\ 
 

    \theutterance \stepcounter{utterance}  

    & \multicolumn{4}{p{0.6\linewidth}}{\cellcolor[rgb]{0.8,1,0.9}{%
	\makecell[{{p{\linewidth}}}]{% 
	  \tt {\tiny [A$\rangle$GM]}  
	 QUESTION: Does your grid have any rows where the X's are consecutive, forming a solid block of X's, or are the X's always separated by at least one blank square? 
	  } 
	   } 
	   } 
	 & & \\ 
 

    \theutterance \stepcounter{utterance}  

    & & & \multicolumn{2}{p{0.3\linewidth}}{\cellcolor[rgb]{0.95,0.95,0.95}{%
	\makecell[{{p{\linewidth}}}]{% 
	  \tt {\tiny [GM$|$GM]}  
	 continue 
	  } 
	   } 
	   } 
	 & & \\ 
 

    \theutterance \stepcounter{utterance}  

    & & \multicolumn{4}{p{0.6\linewidth}}{\cellcolor[rgb]{0.9,0.9,0.9}{%
	\makecell[{{p{\linewidth}}}]{% 
	  \tt {\tiny [GM$\rangle$B]}  
	 ANSWER: The blank squares in my grid are symmetrically arranged, with the pattern of X's and blank squares being mirrored along the vertical axis, and also having a similar symmetry along the horizontal axis, particularly in the first and last rows.\\ \tt QUESTION: Does your grid have any rows where the X's are consecutive, forming a solid block of X's, or are the X's always separated by at least one blank square? Start your answer with "ANSWER:" and do not add anything else. 
	  } 
	   } 
	   } 
	 & \\ 
 

    \theutterance \stepcounter{utterance}  

    & & & \multicolumn{4}{p{0.6\linewidth}}{\cellcolor[rgb]{1,0.85,0.72}{%
	\makecell[{{p{\linewidth}}}]{% 
	  \tt {\tiny [GM$\langle$B]}  
	 ANSWER: In my grid, there are rows where the X's are consecutive, forming a solid block of X's, and this block of X's gets progressively smaller as you move down the rows, with the blank square separating the block of X's on the left from the block of X's on the right, except in the last row where the block of X's is only separated from the edge by the blank square. 
	  } 
	   } 
	   } 
	  \\ 
 

    \theutterance \stepcounter{utterance}  

    & & & \multicolumn{2}{p{0.3\linewidth}}{\cellcolor[rgb]{0.95,0.95,0.95}{%
	\makecell[{{p{\linewidth}}}]{% 
	  \tt {\tiny [GM$|$GM]}  
	 continue 
	  } 
	   } 
	   } 
	 & & \\ 
 

    \theutterance \stepcounter{utterance}  

    & & \multicolumn{4}{p{0.6\linewidth}}{\cellcolor[rgb]{0.9,0.9,0.9}{%
	\makecell[{{p{\linewidth}}}]{% 
	  \tt {\tiny [GM$\rangle$B]}  
	 Now ask a question in order to find out new aspects of my grid that may be different to your grid. Start with "QUESTION:" and do not add anything else. 
	  } 
	   } 
	   } 
	 & \\ 
 

    \theutterance \stepcounter{utterance}  

    & & & \multicolumn{4}{p{0.6\linewidth}}{\cellcolor[rgb]{1,0.85,0.72}{%
	\makecell[{{p{\linewidth}}}]{% 
	  \tt {\tiny [GM$\langle$B]}  
	 QUESTION: Are there any rows in your grid where all the positions are filled with X's, or are there always at least some blank squares present in every row? 
	  } 
	   } 
	   } 
	  \\ 
 

    \theutterance \stepcounter{utterance}  

    & & & \multicolumn{2}{p{0.3\linewidth}}{\cellcolor[rgb]{0.95,0.95,0.95}{%
	\makecell[{{p{\linewidth}}}]{% 
	  \tt {\tiny [GM$|$GM]}  
	 continue 
	  } 
	   } 
	   } 
	 & & \\ 
 

    \theutterance \stepcounter{utterance}  

    & & \multicolumn{4}{p{0.6\linewidth}}{\cellcolor[rgb]{0.9,0.9,0.9}{%
	\makecell[{{p{\linewidth}}}]{% 
	  \tt {\tiny [A$\langle$GM]}  
	 ANSWER: In my grid, there are rows where the X's are consecutive, forming a solid block of X's, and this block of X's gets progressively smaller as you move down the rows, with the blank square separating the block of X's on the left from the block of X's on the right, except in the last row where the block of X's is only separated from the edge by the blank square.\\ \tt QUESTION: Are there any rows in your grid where all the positions are filled with X's, or are there always at least some blank squares present in every row? Start your answer with "ANSWER:" and do not add anything else. 
	  } 
	   } 
	   } 
	 & \\ 
 

    \theutterance \stepcounter{utterance}  

    & \multicolumn{4}{p{0.6\linewidth}}{\cellcolor[rgb]{0.8,1,0.9}{%
	\makecell[{{p{\linewidth}}}]{% 
	  \tt {\tiny [A$\rangle$GM]}  
	 ANSWER: There are no rows in my grid where all the positions are filled with X's, as every row contains at least one blank square, with some rows having only one X and others having multiple X's, but never filling the entire row with X's. 
	  } 
	   } 
	   } 
	 & & \\ 
 

    \theutterance \stepcounter{utterance}  

    & & & \multicolumn{2}{p{0.3\linewidth}}{\cellcolor[rgb]{0.95,0.95,0.95}{%
	\makecell[{{p{\linewidth}}}]{% 
	  \tt {\tiny [GM$|$GM]}  
	 continue 
	  } 
	   } 
	   } 
	 & & \\ 
 

    \theutterance \stepcounter{utterance}  

    & & \multicolumn{4}{p{0.6\linewidth}}{\cellcolor[rgb]{0.9,0.9,0.9}{%
	\makecell[{{p{\linewidth}}}]{% 
	  \tt {\tiny [A$\langle$GM]}  
	 Now ask a question in order to find out new aspects of my grid that may be different to your grid. Start with "QUESTION:" and do not add anything else. 
	  } 
	   } 
	   } 
	 & \\ 
 

    \theutterance \stepcounter{utterance}  

    & \multicolumn{4}{p{0.6\linewidth}}{\cellcolor[rgb]{0.8,1,0.9}{%
	\makecell[{{p{\linewidth}}}]{% 
	  \tt {\tiny [A$\rangle$GM]}  
	 QUESTION: Are the X's in your grid arranged in such a way that they form any distinctive shapes or patterns, such as a diagonal line, a cross, or a frame, or is the arrangement of X's more random and without any discernible geometric shape? 
	  } 
	   } 
	   } 
	 & & \\ 
 

    \theutterance \stepcounter{utterance}  

    & & & \multicolumn{2}{p{0.3\linewidth}}{\cellcolor[rgb]{0.95,0.95,0.95}{%
	\makecell[{{p{\linewidth}}}]{% 
	  \tt {\tiny [GM$|$GM]}  
	 continue 
	  } 
	   } 
	   } 
	 & & \\ 
 

    \theutterance \stepcounter{utterance}  

    & & \multicolumn{4}{p{0.6\linewidth}}{\cellcolor[rgb]{0.9,0.9,0.9}{%
	\makecell[{{p{\linewidth}}}]{% 
	  \tt {\tiny [GM$\rangle$B]}  
	 ANSWER: There are no rows in my grid where all the positions are filled with X's, as every row contains at least one blank square, with some rows having only one X and others having multiple X's, but never filling the entire row with X's.\\ \tt QUESTION: Are the X's in your grid arranged in such a way that they form any distinctive shapes or patterns, such as a diagonal line, a cross, or a frame, or is the arrangement of X's more random and without any discernible geometric shape? Start your answer with "ANSWER:" and do not add anything else. 
	  } 
	   } 
	   } 
	 & \\ 
 

    \theutterance \stepcounter{utterance}  

    & & & \multicolumn{4}{p{0.6\linewidth}}{\cellcolor[rgb]{1,0.85,0.72}{%
	\makecell[{{p{\linewidth}}}]{% 
	  \tt {\tiny [GM$\langle$B]}  
	 ANSWER: The X's in my grid are arranged in a way that forms a distinctive frame or border around the blank squares, with the X's in each row being positioned in a consistent pattern that creates a sense of symmetry and structure, and the blank squares are positioned in a way that creates a diagonal line or stair-step pattern from the top left to the bottom right of the grid. 
	  } 
	   } 
	   } 
	  \\ 
 

    \theutterance \stepcounter{utterance}  

    & & & \multicolumn{2}{p{0.3\linewidth}}{\cellcolor[rgb]{0.95,0.95,0.95}{%
	\makecell[{{p{\linewidth}}}]{% 
	  \tt {\tiny [GM$|$GM]}  
	 continue 
	  } 
	   } 
	   } 
	 & & \\ 
 

    \theutterance \stepcounter{utterance}  

    & & \multicolumn{4}{p{0.6\linewidth}}{\cellcolor[rgb]{0.9,0.9,0.9}{%
	\makecell[{{p{\linewidth}}}]{% 
	  \tt {\tiny [GM$\rangle$B]}  
	 Now ask a question in order to find out new aspects of my grid that may be different to your grid. Start with "QUESTION:" and do not add anything else. 
	  } 
	   } 
	   } 
	 & \\ 
 

    \theutterance \stepcounter{utterance}  

    & & & \multicolumn{4}{p{0.6\linewidth}}{\cellcolor[rgb]{1,0.85,0.72}{%
	\makecell[{{p{\linewidth}}}]{% 
	  \tt {\tiny [GM$\langle$B]}  
	 QUESTION: Does your grid have a single blank square in each row, similar to the description of my grid, or can a row in your grid have multiple blank squares, and if so, how are they typically arranged in relation to the X's? 
	  } 
	   } 
	   } 
	  \\ 
 

    \theutterance \stepcounter{utterance}  

    & & & \multicolumn{2}{p{0.3\linewidth}}{\cellcolor[rgb]{0.95,0.95,0.95}{%
	\makecell[{{p{\linewidth}}}]{% 
	  \tt {\tiny [GM$|$GM]}  
	 continue 
	  } 
	   } 
	   } 
	 & & \\ 
 

    \theutterance \stepcounter{utterance}  

    & & \multicolumn{4}{p{0.6\linewidth}}{\cellcolor[rgb]{0.9,0.9,0.9}{%
	\makecell[{{p{\linewidth}}}]{% 
	  \tt {\tiny [A$\langle$GM]}  
	 ANSWER: The X's in my grid are arranged in a way that forms a distinctive frame or border around the blank squares, with the X's in each row being positioned in a consistent pattern that creates a sense of symmetry and structure, and the blank squares are positioned in a way that creates a diagonal line or stair-step pattern from the top left to the bottom right of the grid.\\ \tt QUESTION: Does your grid have a single blank square in each row, similar to the description of my grid, or can a row in your grid have multiple blank squares, and if so, how are they typically arranged in relation to the X's? Start your answer with "ANSWER:" and do not add anything else. 
	  } 
	   } 
	   } 
	 & \\ 
 

    \theutterance \stepcounter{utterance}  

    & \multicolumn{4}{p{0.6\linewidth}}{\cellcolor[rgb]{0.8,1,0.9}{%
	\makecell[{{p{\linewidth}}}]{% 
	  \tt {\tiny [A$\rangle$GM]}  
	 ANSWER: My grid can have multiple blank squares in a row, and they are typically arranged in a way that creates a sense of balance and symmetry, with the blank squares often being positioned on either side of a central X or group of X's, although there are some rows where the blank squares are clustered together, and others where they are separated by X's, resulting in a more varied and complex pattern. 
	  } 
	   } 
	   } 
	 & & \\ 
 

    \theutterance \stepcounter{utterance}  

    & & & \multicolumn{2}{p{0.3\linewidth}}{\cellcolor[rgb]{0.95,0.95,0.95}{%
	\makecell[{{p{\linewidth}}}]{% 
	  \tt {\tiny [GM$|$GM]}  
	 continue 
	  } 
	   } 
	   } 
	 & & \\ 
 

    \theutterance \stepcounter{utterance}  

    & & \multicolumn{4}{p{0.6\linewidth}}{\cellcolor[rgb]{0.9,0.9,0.9}{%
	\makecell[{{p{\linewidth}}}]{% 
	  \tt {\tiny [A$\langle$GM]}  
	 Now ask a question in order to find out new aspects of my grid that may be different to your grid. Start with "QUESTION:" and do not add anything else. 
	  } 
	   } 
	   } 
	 & \\ 
 

    \theutterance \stepcounter{utterance}  

    & \multicolumn{4}{p{0.6\linewidth}}{\cellcolor[rgb]{0.8,1,0.9}{%
	\makecell[{{p{\linewidth}}}]{% 
	  \tt {\tiny [A$\rangle$GM]}  
	 QUESTION: You mentioned that the blank squares in your grid form a diagonal line or stair-step pattern from the top left to the bottom right, but does this pattern continue consistently throughout the entire grid, or are there any rows or positions where the pattern is broken or interrupted, and if so, how do these interruptions affect the overall structure and appearance of the grid? 
	  } 
	   } 
	   } 
	 & & \\ 
 

    \theutterance \stepcounter{utterance}  

    & & & \multicolumn{2}{p{0.3\linewidth}}{\cellcolor[rgb]{0.95,0.95,0.95}{%
	\makecell[{{p{\linewidth}}}]{% 
	  \tt {\tiny [GM$|$GM]}  
	 continue 
	  } 
	   } 
	   } 
	 & & \\ 
 

    \theutterance \stepcounter{utterance}  

    & & \multicolumn{4}{p{0.6\linewidth}}{\cellcolor[rgb]{0.9,0.9,0.9}{%
	\makecell[{{p{\linewidth}}}]{% 
	  \tt {\tiny [GM$\rangle$B]}  
	 ANSWER: My grid can have multiple blank squares in a row, and they are typically arranged in a way that creates a sense of balance and symmetry, with the blank squares often being positioned on either side of a central X or group of X's, although there are some rows where the blank squares are clustered together, and others where they are separated by X's, resulting in a more varied and complex pattern.\\ \tt QUESTION: You mentioned that the blank squares in your grid form a diagonal line or stair-step pattern from the top left to the bottom right, but does this pattern continue consistently throughout the entire grid, or are there any rows or positions where the pattern is broken or interrupted, and if so, how do these interruptions affect the overall structure and appearance of the grid? Start your answer with "ANSWER:" and do not add anything else. 
	  } 
	   } 
	   } 
	 & \\ 
 

    \theutterance \stepcounter{utterance}  

    & & & \multicolumn{4}{p{0.6\linewidth}}{\cellcolor[rgb]{1,0.85,0.72}{%
	\makecell[{{p{\linewidth}}}]{% 
	  \tt {\tiny [GM$\langle$B]}  
	 ANSWER: The diagonal line or stair-step pattern formed by the blank squares in my grid continues consistently throughout the entire grid, with each row having one blank square that is positioned one column to the right of the blank square in the row above it, creating a smooth and uninterrupted diagonal line from the top left to the bottom right of the grid, and this consistent pattern gives the grid a sense of simplicity, order, and predictability, with no interruptions or breaks in the pattern to disrupt the overall structure and appearance of the grid. 
	  } 
	   } 
	   } 
	  \\ 
 

    \theutterance \stepcounter{utterance}  

    & & & \multicolumn{2}{p{0.3\linewidth}}{\cellcolor[rgb]{0.95,0.95,0.95}{%
	\makecell[{{p{\linewidth}}}]{% 
	  \tt {\tiny [GM$|$GM]}  
	 continue 
	  } 
	   } 
	   } 
	 & & \\ 
 

    \theutterance \stepcounter{utterance}  

    & & \multicolumn{4}{p{0.6\linewidth}}{\cellcolor[rgb]{0.9,0.9,0.9}{%
	\makecell[{{p{\linewidth}}}]{% 
	  \tt {\tiny [GM$\rangle$B]}  
	 Now come to a decision. What do you think: are your grid and the other grid described the same?  Write "DECISION: same grid" if you think they are the same grid or "DECISION: different grids" if you think they are different grids. Do not add anything else.\\ \tt  
	  } 
	   } 
	   } 
	 & \\ 
 

    \theutterance \stepcounter{utterance}  

    & & & \multicolumn{4}{p{0.6\linewidth}}{\cellcolor[rgb]{1,0.85,0.72}{%
	\makecell[{{p{\linewidth}}}]{% 
	  \tt {\tiny [GM$\langle$B]}  
	 DECISION: different grids 
	  } 
	   } 
	   } 
	  \\ 
 

    \theutterance \stepcounter{utterance}  

    & & & \multicolumn{2}{p{0.3\linewidth}}{\cellcolor[rgb]{0.95,0.95,0.95}{%
	\makecell[{{p{\linewidth}}}]{% 
	  \tt {\tiny [GM$|$GM]}  
	 continue 
	  } 
	   } 
	   } 
	 & & \\ 
 

    \theutterance \stepcounter{utterance}  

    & & & \multicolumn{2}{p{0.3\linewidth}}{\cellcolor[rgb]{0.95,0.95,0.95}{%
	\makecell[{{p{\linewidth}}}]{% 
	  \tt {\tiny [GM$|$GM]}  
	 success 
	  } 
	   } 
	   } 
	 & & \\ 
 

    \theutterance \stepcounter{utterance}  

    & & \multicolumn{4}{p{0.6\linewidth}}{\cellcolor[rgb]{0.9,0.9,0.9}{%
	\makecell[{{p{\linewidth}}}]{% 
	  \tt {\tiny [A$\langle$GM]}  
	 ANSWER: The diagonal line or stair-step pattern formed by the blank squares in my grid continues consistently throughout the entire grid, with each row having one blank square that is positioned one column to the right of the blank square in the row above it, creating a smooth and uninterrupted diagonal line from the top left to the bottom right of the grid, and this consistent pattern gives the grid a sense of simplicity, order, and predictability, with no interruptions or breaks in the pattern to disrupt the overall structure and appearance of the grid.\\ \tt Now come to a decision. What do you think: are your grid and the other grid described the same?  Write "DECISION: same grid" if you think they are the same grid or "DECISION: different grids" if you think they are different grids. Do not add anything else.\\ \tt  
	  } 
	   } 
	   } 
	 & \\ 
 

    \theutterance \stepcounter{utterance}  

    & \multicolumn{4}{p{0.6\linewidth}}{\cellcolor[rgb]{0.8,1,0.9}{%
	\makecell[{{p{\linewidth}}}]{% 
	  \tt {\tiny [A$\rangle$GM]}  
	 DECISION: different grids 
	  } 
	   } 
	   } 
	 & & \\ 
 

    \theutterance \stepcounter{utterance}  

    & & & \multicolumn{2}{p{0.3\linewidth}}{\cellcolor[rgb]{0.95,0.95,0.95}{%
	\makecell[{{p{\linewidth}}}]{% 
	  \tt {\tiny [GM$|$GM]}  
	 continue 
	  } 
	   } 
	   } 
	 & & \\ 
 

    \theutterance \stepcounter{utterance}  

    & & & \multicolumn{2}{p{0.3\linewidth}}{\cellcolor[rgb]{0.95,0.95,0.95}{%
	\makecell[{{p{\linewidth}}}]{% 
	  \tt {\tiny [GM$|$GM]}  
	 success 
	  } 
	   } 
	   } 
	 & & \\ 
 

\end{supertabular}
}

\end{document}

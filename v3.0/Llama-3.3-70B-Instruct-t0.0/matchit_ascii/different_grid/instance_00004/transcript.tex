
\documentclass{article}
\usepackage{colortbl}
\usepackage{makecell}
\usepackage{multirow}
\usepackage{supertabular}

\begin{document}

\newcounter{utterance}

\twocolumn

{ \footnotesize  \setcounter{utterance}{1}
\setlength{\tabcolsep}{0pt}
\begin{supertabular}{c@{$\;$}|p{.15\linewidth}@{}p{.15\linewidth}p{.15\linewidth}p{.15\linewidth}p{.15\linewidth}p{.15\linewidth}}

    \# & $\;$A & \multicolumn{4}{c}{Game Master} & $\;\:$B\\
    \hline 

    \theutterance \stepcounter{utterance}  

    & & \multicolumn{4}{p{0.6\linewidth}}{\cellcolor[rgb]{0.9,0.9,0.9}{%
	\makecell[{{p{\linewidth}}}]{% 
	  \tt {\tiny [A$\langle$GM]}  
	 You are participating in a collaborative guessing game.\\ \tt The goal is to find out whether this grid and another picture only I can see, are the same.\\ \tt Please describe your grid first. Then, I will provide my description and we can ask each other questions about the grids to figure out whether they are the same. Now start your short grid description with "DESCRIPTION:" followed by the description. Do not add anything else. This is your grid:\\ \tt ▢ ▢ ▢ ▢ ▢\\ \tt ▢ X X X X\\ \tt ▢ ▢ ▢ X X\\ \tt ▢ X X X X\\ \tt ▢ X X X X 
	  } 
	   } 
	   } 
	 & \\ 
 

    \theutterance \stepcounter{utterance}  

    & \multicolumn{4}{p{0.6\linewidth}}{\cellcolor[rgb]{0.8,1,0.9}{%
	\makecell[{{p{\linewidth}}}]{% 
	  \tt {\tiny [A$\rangle$GM]}  
	 DESCRIPTION: A 5x5 grid with some filled squares (represented by X) and some empty squares (represented by ▢), where the first row is completely empty, the second row is completely filled except for the first square, the third row has two filled squares at the end, and the last two rows are completely filled except for the first square in each row. 
	  } 
	   } 
	   } 
	 & & \\ 
 

    \theutterance \stepcounter{utterance}  

    & & & \multicolumn{2}{p{0.3\linewidth}}{\cellcolor[rgb]{0.95,0.95,0.95}{%
	\makecell[{{p{\linewidth}}}]{% 
	  \tt {\tiny [GM$|$GM]}  
	 continue 
	  } 
	   } 
	   } 
	 & & \\ 
 

    \theutterance \stepcounter{utterance}  

    & & \multicolumn{4}{p{0.6\linewidth}}{\cellcolor[rgb]{0.9,0.9,0.9}{%
	\makecell[{{p{\linewidth}}}]{% 
	  \tt {\tiny [GM$\rangle$B]}  
	 You are participating in a collaborative guessing game.\\ \tt The goal is to find out whether this grid and another picture only I can see, are the same.\\ \tt Please describe your grid first. Then, I will provide my description and we can ask each other questions about the grids to figure out whether they are the same. Now start your short grid description with "DESCRIPTION:" followed by the description. Do not add anything else. This is your grid:\\ \tt ▢ ▢ ▢ ▢ X\\ \tt ▢ ▢ ▢ X X\\ \tt ▢ ▢ X X X\\ \tt ▢ X X X X\\ \tt X X X X X 
	  } 
	   } 
	   } 
	 & \\ 
 

    \theutterance \stepcounter{utterance}  

    & & & \multicolumn{4}{p{0.6\linewidth}}{\cellcolor[rgb]{1,0.85,0.72}{%
	\makecell[{{p{\linewidth}}}]{% 
	  \tt {\tiny [GM$\langle$B]}  
	 DESCRIPTION: A 5x5 grid with a pattern of X's that increases by one additional X from top to bottom, starting with one X in the top row and five X's in the bottom row, with empty squares represented by ▢. 
	  } 
	   } 
	   } 
	  \\ 
 

    \theutterance \stepcounter{utterance}  

    & & & \multicolumn{2}{p{0.3\linewidth}}{\cellcolor[rgb]{0.95,0.95,0.95}{%
	\makecell[{{p{\linewidth}}}]{% 
	  \tt {\tiny [GM$|$GM]}  
	 continue 
	  } 
	   } 
	   } 
	 & & \\ 
 

    \theutterance \stepcounter{utterance}  

    & & \multicolumn{4}{p{0.6\linewidth}}{\cellcolor[rgb]{0.9,0.9,0.9}{%
	\makecell[{{p{\linewidth}}}]{% 
	  \tt {\tiny [GM$\rangle$B]}  
	 This is my DESCRIPTION: A 5x5 grid with some filled squares (represented by X) and some empty squares (represented by ▢), where the first row is completely empty, the second row is completely filled except for the first square, the third row has two filled squares at the end, and the last two rows are completely filled except for the first square in each row.\\ \tt Now ask a question in order to find out new aspects of my grid that may be different to your grid. Start with "QUESTION:" and do not add anything else. 
	  } 
	   } 
	   } 
	 & \\ 
 

    \theutterance \stepcounter{utterance}  

    & & & \multicolumn{4}{p{0.6\linewidth}}{\cellcolor[rgb]{1,0.85,0.72}{%
	\makecell[{{p{\linewidth}}}]{% 
	  \tt {\tiny [GM$\langle$B]}  
	 QUESTION: Are the filled squares in your grid arranged in a specific pattern or sequence, or are they randomly distributed, and can you tell me how many filled squares are in the third row of your grid? 
	  } 
	   } 
	   } 
	  \\ 
 

    \theutterance \stepcounter{utterance}  

    & & & \multicolumn{2}{p{0.3\linewidth}}{\cellcolor[rgb]{0.95,0.95,0.95}{%
	\makecell[{{p{\linewidth}}}]{% 
	  \tt {\tiny [GM$|$GM]}  
	 continue 
	  } 
	   } 
	   } 
	 & & \\ 
 

    \theutterance \stepcounter{utterance}  

    & & \multicolumn{4}{p{0.6\linewidth}}{\cellcolor[rgb]{0.9,0.9,0.9}{%
	\makecell[{{p{\linewidth}}}]{% 
	  \tt {\tiny [A$\langle$GM]}  
	 This is my DESCRIPTION: A 5x5 grid with a pattern of X's that increases by one additional X from top to bottom, starting with one X in the top row and five X's in the bottom row, with empty squares represented by ▢.\\ \tt QUESTION: Are the filled squares in your grid arranged in a specific pattern or sequence, or are they randomly distributed, and can you tell me how many filled squares are in the third row of your grid? Start your answer with "ANSWER:" and do not add anything else. 
	  } 
	   } 
	   } 
	 & \\ 
 

    \theutterance \stepcounter{utterance}  

    & \multicolumn{4}{p{0.6\linewidth}}{\cellcolor[rgb]{0.8,1,0.9}{%
	\makecell[{{p{\linewidth}}}]{% 
	  \tt {\tiny [A$\rangle$GM]}  
	 ANSWER: The filled squares in my grid are arranged in a specific pattern, but it doesn't seem to be a simple increasing sequence, and there are 2 filled squares in the third row of my grid. 
	  } 
	   } 
	   } 
	 & & \\ 
 

    \theutterance \stepcounter{utterance}  

    & & & \multicolumn{2}{p{0.3\linewidth}}{\cellcolor[rgb]{0.95,0.95,0.95}{%
	\makecell[{{p{\linewidth}}}]{% 
	  \tt {\tiny [GM$|$GM]}  
	 continue 
	  } 
	   } 
	   } 
	 & & \\ 
 

    \theutterance \stepcounter{utterance}  

    & & \multicolumn{4}{p{0.6\linewidth}}{\cellcolor[rgb]{0.9,0.9,0.9}{%
	\makecell[{{p{\linewidth}}}]{% 
	  \tt {\tiny [A$\langle$GM]}  
	 Now ask a question in order to find out new aspects of my grid that may be different to your grid. Start with "QUESTION:" and do not add anything else. 
	  } 
	   } 
	   } 
	 & \\ 
 

    \theutterance \stepcounter{utterance}  

    & \multicolumn{4}{p{0.6\linewidth}}{\cellcolor[rgb]{0.8,1,0.9}{%
	\makecell[{{p{\linewidth}}}]{% 
	  \tt {\tiny [A$\rangle$GM]}  
	 QUESTION: Are the X's in your grid contiguous, meaning they are grouped together in a single sequence, or are they scattered throughout each row, and can you tell me if the first row of your grid has the single X at a specific position, such as the first or last column? 
	  } 
	   } 
	   } 
	 & & \\ 
 

    \theutterance \stepcounter{utterance}  

    & & & \multicolumn{2}{p{0.3\linewidth}}{\cellcolor[rgb]{0.95,0.95,0.95}{%
	\makecell[{{p{\linewidth}}}]{% 
	  \tt {\tiny [GM$|$GM]}  
	 continue 
	  } 
	   } 
	   } 
	 & & \\ 
 

    \theutterance \stepcounter{utterance}  

    & & \multicolumn{4}{p{0.6\linewidth}}{\cellcolor[rgb]{0.9,0.9,0.9}{%
	\makecell[{{p{\linewidth}}}]{% 
	  \tt {\tiny [GM$\rangle$B]}  
	 ANSWER: The filled squares in my grid are arranged in a specific pattern, but it doesn't seem to be a simple increasing sequence, and there are 2 filled squares in the third row of my grid.\\ \tt QUESTION: Are the X's in your grid contiguous, meaning they are grouped together in a single sequence, or are they scattered throughout each row, and can you tell me if the first row of your grid has the single X at a specific position, such as the first or last column? Start your answer with "ANSWER:" and do not add anything else. 
	  } 
	   } 
	   } 
	 & \\ 
 

    \theutterance \stepcounter{utterance}  

    & & & \multicolumn{4}{p{0.6\linewidth}}{\cellcolor[rgb]{1,0.85,0.72}{%
	\makecell[{{p{\linewidth}}}]{% 
	  \tt {\tiny [GM$\langle$B]}  
	 ANSWER: The X's in my grid are contiguous and grouped together, forming a stair-step pattern that increases by one X from top to bottom, and the single X in the first row of my grid is located at the last column, and regarding the contiguity, the X's in each row are always grouped together at the end of the row. QUESTION: Does the specific pattern of filled squares in your grid also form a stair-step or triangular shape, and are the empty squares in your grid only located at the beginning of each row? 
	  } 
	   } 
	   } 
	  \\ 
 

    \theutterance \stepcounter{utterance}  

    & & & \multicolumn{2}{p{0.3\linewidth}}{\cellcolor[rgb]{0.95,0.95,0.95}{%
	\makecell[{{p{\linewidth}}}]{% 
	  \tt {\tiny [GM$|$GM]}  
	 continue 
	  } 
	   } 
	   } 
	 & & \\ 
 

    \theutterance \stepcounter{utterance}  

    & & \multicolumn{4}{p{0.6\linewidth}}{\cellcolor[rgb]{0.9,0.9,0.9}{%
	\makecell[{{p{\linewidth}}}]{% 
	  \tt {\tiny [GM$\rangle$B]}  
	 Now ask a question in order to find out new aspects of my grid that may be different to your grid. Start with "QUESTION:" and do not add anything else. 
	  } 
	   } 
	   } 
	 & \\ 
 

    \theutterance \stepcounter{utterance}  

    & & & \multicolumn{4}{p{0.6\linewidth}}{\cellcolor[rgb]{1,0.85,0.72}{%
	\makecell[{{p{\linewidth}}}]{% 
	  \tt {\tiny [GM$\langle$B]}  
	 QUESTION: You mentioned that the second row of your grid is completely filled except for the first square, which implies that the last four squares of the second row are filled with X's, but you also mentioned that the third row has two filled squares at the end, which seems to break the pattern of increasing filled squares, can you confirm if the number of filled squares in each row of your grid increases or decreases at some point, and how does the fourth row look like in terms of filled and empty squares? 
	  } 
	   } 
	   } 
	  \\ 
 

    \theutterance \stepcounter{utterance}  

    & & & \multicolumn{2}{p{0.3\linewidth}}{\cellcolor[rgb]{0.95,0.95,0.95}{%
	\makecell[{{p{\linewidth}}}]{% 
	  \tt {\tiny [GM$|$GM]}  
	 continue 
	  } 
	   } 
	   } 
	 & & \\ 
 

    \theutterance \stepcounter{utterance}  

    & & \multicolumn{4}{p{0.6\linewidth}}{\cellcolor[rgb]{0.9,0.9,0.9}{%
	\makecell[{{p{\linewidth}}}]{% 
	  \tt {\tiny [A$\langle$GM]}  
	 ANSWER: The X's in my grid are contiguous and grouped together, forming a stair-step pattern that increases by one X from top to bottom, and the single X in the first row of my grid is located at the last column, and regarding the contiguity, the X's in each row are always grouped together at the end of the row. QUESTION: Does the specific pattern of filled squares in your grid also form a stair-step or triangular shape, and are the empty squares in your grid only located at the beginning of each row?\\ \tt QUESTION: You mentioned that the second row of your grid is completely filled except for the first square, which implies that the last four squares of the second row are filled with X's, but you also mentioned that the third row has two filled squares at the end, which seems to break the pattern of increasing filled squares, can you confirm if the number of filled squares in each row of your grid increases or decreases at some point, and how does the fourth row look like in terms of filled and empty squares? Start your answer with "ANSWER:" and do not add anything else. 
	  } 
	   } 
	   } 
	 & \\ 
 

    \theutterance \stepcounter{utterance}  

    & \multicolumn{4}{p{0.6\linewidth}}{\cellcolor[rgb]{0.8,1,0.9}{%
	\makecell[{{p{\linewidth}}}]{% 
	  \tt {\tiny [A$\rangle$GM]}  
	 ANSWER: The pattern of filled squares in my grid does not form a perfect stair-step or triangular shape, and the empty squares are not only located at the beginning of each row, but can also be found at other positions, and regarding the second and third rows, you are correct that it seems to break a simple increasing pattern, and to confirm, the number of filled squares in each row of my grid does not consistently increase, the fourth row has 4 filled squares, which are located at the end of the row, similar to the second row, and the first square is empty, and the pattern of filled squares in my grid seems to be more repetitive than increasing, with the second, fourth, and fifth rows having the same number of filled squares. QUESTION: Does the pattern of X's in your grid continue to increase by one X per row up to the fifth row, and are the X's in the fifth row contiguous and grouped together at the end of the row, and can you tell me if there are any empty squares in the fifth row of your grid? 
	  } 
	   } 
	   } 
	 & & \\ 
 

    \theutterance \stepcounter{utterance}  

    & & & \multicolumn{2}{p{0.3\linewidth}}{\cellcolor[rgb]{0.95,0.95,0.95}{%
	\makecell[{{p{\linewidth}}}]{% 
	  \tt {\tiny [GM$|$GM]}  
	 continue 
	  } 
	   } 
	   } 
	 & & \\ 
 

    \theutterance \stepcounter{utterance}  

    & & \multicolumn{4}{p{0.6\linewidth}}{\cellcolor[rgb]{0.9,0.9,0.9}{%
	\makecell[{{p{\linewidth}}}]{% 
	  \tt {\tiny [A$\langle$GM]}  
	 Now ask a question in order to find out new aspects of my grid that may be different to your grid. Start with "QUESTION:" and do not add anything else. 
	  } 
	   } 
	   } 
	 & \\ 
 

    \theutterance \stepcounter{utterance}  

    & \multicolumn{4}{p{0.6\linewidth}}{\cellcolor[rgb]{0.8,1,0.9}{%
	\makecell[{{p{\linewidth}}}]{% 
	  \tt {\tiny [A$\rangle$GM]}  
	 QUESTION: Are there any symmetries or reflective properties in your grid, such as rotational symmetry or mirror symmetry, and can you tell me if the pattern of X's in your grid would look the same if it were rotated 90 or 180 degrees, or if it were reflected horizontally or vertically? 
	  } 
	   } 
	   } 
	 & & \\ 
 

    \theutterance \stepcounter{utterance}  

    & & & \multicolumn{2}{p{0.3\linewidth}}{\cellcolor[rgb]{0.95,0.95,0.95}{%
	\makecell[{{p{\linewidth}}}]{% 
	  \tt {\tiny [GM$|$GM]}  
	 continue 
	  } 
	   } 
	   } 
	 & & \\ 
 

    \theutterance \stepcounter{utterance}  

    & & \multicolumn{4}{p{0.6\linewidth}}{\cellcolor[rgb]{0.9,0.9,0.9}{%
	\makecell[{{p{\linewidth}}}]{% 
	  \tt {\tiny [GM$\rangle$B]}  
	 ANSWER: The pattern of filled squares in my grid does not form a perfect stair-step or triangular shape, and the empty squares are not only located at the beginning of each row, but can also be found at other positions, and regarding the second and third rows, you are correct that it seems to break a simple increasing pattern, and to confirm, the number of filled squares in each row of my grid does not consistently increase, the fourth row has 4 filled squares, which are located at the end of the row, similar to the second row, and the first square is empty, and the pattern of filled squares in my grid seems to be more repetitive than increasing, with the second, fourth, and fifth rows having the same number of filled squares. QUESTION: Does the pattern of X's in your grid continue to increase by one X per row up to the fifth row, and are the X's in the fifth row contiguous and grouped together at the end of the row, and can you tell me if there are any empty squares in the fifth row of your grid?\\ \tt QUESTION: Are there any symmetries or reflective properties in your grid, such as rotational symmetry or mirror symmetry, and can you tell me if the pattern of X's in your grid would look the same if it were rotated 90 or 180 degrees, or if it were reflected horizontally or vertically? Start your answer with "ANSWER:" and do not add anything else. 
	  } 
	   } 
	   } 
	 & \\ 
 

    \theutterance \stepcounter{utterance}  

    & & & \multicolumn{4}{p{0.6\linewidth}}{\cellcolor[rgb]{1,0.85,0.72}{%
	\makecell[{{p{\linewidth}}}]{% 
	  \tt {\tiny [GM$\langle$B]}  
	 ANSWER: The pattern of X's in my grid does continue to increase by one X per row up to the fifth row, and the X's in the fifth row are contiguous and grouped together, but they actually fill the entire row, leaving no empty squares in the fifth row of my grid, and regarding the second question, my grid does have a reflective property, it has a mirror symmetry along the vertical axis that runs through the middle of the grid, but it does not have rotational symmetry, and if it were rotated 90 or 180 degrees, the pattern of X's would not look the same, however, if it were reflected horizontally, the pattern would remain unchanged, but if it were reflected vertically, the pattern would be reversed, and the X's would be on the opposite side of the grid, QUESTION: Does your grid have any symmetries or reflective properties, and can you tell me if the pattern of filled squares in your grid would look the same if it were rotated or reflected in any way? 
	  } 
	   } 
	   } 
	  \\ 
 

    \theutterance \stepcounter{utterance}  

    & & & \multicolumn{2}{p{0.3\linewidth}}{\cellcolor[rgb]{0.95,0.95,0.95}{%
	\makecell[{{p{\linewidth}}}]{% 
	  \tt {\tiny [GM$|$GM]}  
	 continue 
	  } 
	   } 
	   } 
	 & & \\ 
 

    \theutterance \stepcounter{utterance}  

    & & \multicolumn{4}{p{0.6\linewidth}}{\cellcolor[rgb]{0.9,0.9,0.9}{%
	\makecell[{{p{\linewidth}}}]{% 
	  \tt {\tiny [GM$\rangle$B]}  
	 Now ask a question in order to find out new aspects of my grid that may be different to your grid. Start with "QUESTION:" and do not add anything else. 
	  } 
	   } 
	   } 
	 & \\ 
 

    \theutterance \stepcounter{utterance}  

    & & & \multicolumn{4}{p{0.6\linewidth}}{\cellcolor[rgb]{1,0.85,0.72}{%
	\makecell[{{p{\linewidth}}}]{% 
	  \tt {\tiny [GM$\langle$B]}  
	 QUESTION: You mentioned that the second, fourth, and fifth rows of your grid have the same number of filled squares, which is 4, and that the first square of the fourth and fifth rows is empty, but you didn't specify the exact arrangement of filled and empty squares in the fifth row, can you confirm if the fifth row of your grid is completely filled with X's, except for the first square, or if there are any other empty squares in the fifth row, and how does the arrangement of filled and empty squares in the fifth row of your grid compare to the arrangement in the second and fourth rows? 
	  } 
	   } 
	   } 
	  \\ 
 

    \theutterance \stepcounter{utterance}  

    & & & \multicolumn{2}{p{0.3\linewidth}}{\cellcolor[rgb]{0.95,0.95,0.95}{%
	\makecell[{{p{\linewidth}}}]{% 
	  \tt {\tiny [GM$|$GM]}  
	 continue 
	  } 
	   } 
	   } 
	 & & \\ 
 

    \theutterance \stepcounter{utterance}  

    & & \multicolumn{4}{p{0.6\linewidth}}{\cellcolor[rgb]{0.9,0.9,0.9}{%
	\makecell[{{p{\linewidth}}}]{% 
	  \tt {\tiny [A$\langle$GM]}  
	 ANSWER: The pattern of X's in my grid does continue to increase by one X per row up to the fifth row, and the X's in the fifth row are contiguous and grouped together, but they actually fill the entire row, leaving no empty squares in the fifth row of my grid, and regarding the second question, my grid does have a reflective property, it has a mirror symmetry along the vertical axis that runs through the middle of the grid, but it does not have rotational symmetry, and if it were rotated 90 or 180 degrees, the pattern of X's would not look the same, however, if it were reflected horizontally, the pattern would remain unchanged, but if it were reflected vertically, the pattern would be reversed, and the X's would be on the opposite side of the grid, QUESTION: Does your grid have any symmetries or reflective properties, and can you tell me if the pattern of filled squares in your grid would look the same if it were rotated or reflected in any way?\\ \tt QUESTION: You mentioned that the second, fourth, and fifth rows of your grid have the same number of filled squares, which is 4, and that the first square of the fourth and fifth rows is empty, but you didn't specify the exact arrangement of filled and empty squares in the fifth row, can you confirm if the fifth row of your grid is completely filled with X's, except for the first square, or if there are any other empty squares in the fifth row, and how does the arrangement of filled and empty squares in the fifth row of your grid compare to the arrangement in the second and fourth rows? Start your answer with "ANSWER:" and do not add anything else. 
	  } 
	   } 
	   } 
	 & \\ 
 

    \theutterance \stepcounter{utterance}  

    & \multicolumn{4}{p{0.6\linewidth}}{\cellcolor[rgb]{0.8,1,0.9}{%
	\makecell[{{p{\linewidth}}}]{% 
	  \tt {\tiny [A$\rangle$GM]}  
	 ANSWER: My grid does not have any rotational symmetries, but it does have a reflective property, however, it's not a perfect mirror symmetry, and if it were reflected horizontally, the pattern would not remain unchanged, and regarding the fifth row, it is indeed completely filled with X's, except for the first square, which is empty, similar to the second and fourth rows, and the arrangement of filled and empty squares in the fifth row is identical to the arrangement in the second and fourth rows, with 4 consecutive X's at the end of the row, and the first square being empty, and this repetitive pattern is a distinctive feature of my grid, QUESTION: Does the fact that the fifth row of your grid is completely filled with X's, with no empty squares, represent a unique or distinctive feature of your grid, and can you tell me if there are any other rows in your grid that have a similar arrangement of filled and empty squares, or if the fifth row is an outlier in terms of its pattern? 
	  } 
	   } 
	   } 
	 & & \\ 
 

    \theutterance \stepcounter{utterance}  

    & & & \multicolumn{2}{p{0.3\linewidth}}{\cellcolor[rgb]{0.95,0.95,0.95}{%
	\makecell[{{p{\linewidth}}}]{% 
	  \tt {\tiny [GM$|$GM]}  
	 continue 
	  } 
	   } 
	   } 
	 & & \\ 
 

    \theutterance \stepcounter{utterance}  

    & & \multicolumn{4}{p{0.6\linewidth}}{\cellcolor[rgb]{0.9,0.9,0.9}{%
	\makecell[{{p{\linewidth}}}]{% 
	  \tt {\tiny [A$\langle$GM]}  
	 Now ask a question in order to find out new aspects of my grid that may be different to your grid. Start with "QUESTION:" and do not add anything else. 
	  } 
	   } 
	   } 
	 & \\ 
 

    \theutterance \stepcounter{utterance}  

    & \multicolumn{4}{p{0.6\linewidth}}{\cellcolor[rgb]{0.8,1,0.9}{%
	\makecell[{{p{\linewidth}}}]{% 
	  \tt {\tiny [A$\rangle$GM]}  
	 QUESTION: Are there any specific rules or constraints that govern the placement of X's in your grid, such as a requirement that each row must have a certain number of X's, or that the X's must be placed in a specific order, and can you tell me if the pattern of X's in your grid is deterministic, meaning that it is completely predictable and follows a clear set of rules, or if there is any element of randomness or unpredictability in the arrangement of X's? 
	  } 
	   } 
	   } 
	 & & \\ 
 

    \theutterance \stepcounter{utterance}  

    & & & \multicolumn{2}{p{0.3\linewidth}}{\cellcolor[rgb]{0.95,0.95,0.95}{%
	\makecell[{{p{\linewidth}}}]{% 
	  \tt {\tiny [GM$|$GM]}  
	 continue 
	  } 
	   } 
	   } 
	 & & \\ 
 

    \theutterance \stepcounter{utterance}  

    & & \multicolumn{4}{p{0.6\linewidth}}{\cellcolor[rgb]{0.9,0.9,0.9}{%
	\makecell[{{p{\linewidth}}}]{% 
	  \tt {\tiny [GM$\rangle$B]}  
	 ANSWER: My grid does not have any rotational symmetries, but it does have a reflective property, however, it's not a perfect mirror symmetry, and if it were reflected horizontally, the pattern would not remain unchanged, and regarding the fifth row, it is indeed completely filled with X's, except for the first square, which is empty, similar to the second and fourth rows, and the arrangement of filled and empty squares in the fifth row is identical to the arrangement in the second and fourth rows, with 4 consecutive X's at the end of the row, and the first square being empty, and this repetitive pattern is a distinctive feature of my grid, QUESTION: Does the fact that the fifth row of your grid is completely filled with X's, with no empty squares, represent a unique or distinctive feature of your grid, and can you tell me if there are any other rows in your grid that have a similar arrangement of filled and empty squares, or if the fifth row is an outlier in terms of its pattern?\\ \tt QUESTION: Are there any specific rules or constraints that govern the placement of X's in your grid, such as a requirement that each row must have a certain number of X's, or that the X's must be placed in a specific order, and can you tell me if the pattern of X's in your grid is deterministic, meaning that it is completely predictable and follows a clear set of rules, or if there is any element of randomness or unpredictability in the arrangement of X's? Start your answer with "ANSWER:" and do not add anything else. 
	  } 
	   } 
	   } 
	 & \\ 
 

    \theutterance \stepcounter{utterance}  

    & & & \multicolumn{4}{p{0.6\linewidth}}{\cellcolor[rgb]{1,0.85,0.72}{%
	\makecell[{{p{\linewidth}}}]{% 
	  \tt {\tiny [GM$\langle$B]}  
	 ANSWER: The fact that the fifth row of my grid is completely filled with X's, with no empty squares, does represent a unique feature of my grid, and it is the only row that has this arrangement, as the other rows have a mix of X's and empty squares, and the fifth row can be considered an outlier in terms of its pattern, and regarding the second question, there is a specific rule that governs the placement of X's in my grid, which is that each row must have one more X than the previous row, and the X's must be placed contiguously at the end of each row, and this rule is deterministic, meaning that the pattern of X's is completely predictable and follows a clear set of rules, with no element of randomness or unpredictability, QUESTION: Does the repetitive pattern in your grid, where the second, fourth, and fifth rows have the same arrangement of filled and empty squares, suggest that there may be a cyclical or periodic component to the pattern, and can you tell me if the rules or constraints that govern the placement of X's in your grid are based on a simple iterative process, such as adding or removing X's from one row to the next, or if there is a more complex set of rules at play? 
	  } 
	   } 
	   } 
	  \\ 
 

    \theutterance \stepcounter{utterance}  

    & & & \multicolumn{2}{p{0.3\linewidth}}{\cellcolor[rgb]{0.95,0.95,0.95}{%
	\makecell[{{p{\linewidth}}}]{% 
	  \tt {\tiny [GM$|$GM]}  
	 continue 
	  } 
	   } 
	   } 
	 & & \\ 
 

    \theutterance \stepcounter{utterance}  

    & & \multicolumn{4}{p{0.6\linewidth}}{\cellcolor[rgb]{0.9,0.9,0.9}{%
	\makecell[{{p{\linewidth}}}]{% 
	  \tt {\tiny [GM$\rangle$B]}  
	 Now come to a decision. What do you think: are your grid and the other grid described the same?  Write "DECISION: same grid" if you think they are the same grid or "DECISION: different grids" if you think they are different grids. Do not add anything else.\\ \tt  
	  } 
	   } 
	   } 
	 & \\ 
 

    \theutterance \stepcounter{utterance}  

    & & & \multicolumn{4}{p{0.6\linewidth}}{\cellcolor[rgb]{1,0.85,0.72}{%
	\makecell[{{p{\linewidth}}}]{% 
	  \tt {\tiny [GM$\langle$B]}  
	 DECISION: different grids 
	  } 
	   } 
	   } 
	  \\ 
 

    \theutterance \stepcounter{utterance}  

    & & & \multicolumn{2}{p{0.3\linewidth}}{\cellcolor[rgb]{0.95,0.95,0.95}{%
	\makecell[{{p{\linewidth}}}]{% 
	  \tt {\tiny [GM$|$GM]}  
	 continue 
	  } 
	   } 
	   } 
	 & & \\ 
 

    \theutterance \stepcounter{utterance}  

    & & & \multicolumn{2}{p{0.3\linewidth}}{\cellcolor[rgb]{0.95,0.95,0.95}{%
	\makecell[{{p{\linewidth}}}]{% 
	  \tt {\tiny [GM$|$GM]}  
	 success 
	  } 
	   } 
	   } 
	 & & \\ 
 

    \theutterance \stepcounter{utterance}  

    & & \multicolumn{4}{p{0.6\linewidth}}{\cellcolor[rgb]{0.9,0.9,0.9}{%
	\makecell[{{p{\linewidth}}}]{% 
	  \tt {\tiny [A$\langle$GM]}  
	 ANSWER: The fact that the fifth row of my grid is completely filled with X's, with no empty squares, does represent a unique feature of my grid, and it is the only row that has this arrangement, as the other rows have a mix of X's and empty squares, and the fifth row can be considered an outlier in terms of its pattern, and regarding the second question, there is a specific rule that governs the placement of X's in my grid, which is that each row must have one more X than the previous row, and the X's must be placed contiguously at the end of each row, and this rule is deterministic, meaning that the pattern of X's is completely predictable and follows a clear set of rules, with no element of randomness or unpredictability, QUESTION: Does the repetitive pattern in your grid, where the second, fourth, and fifth rows have the same arrangement of filled and empty squares, suggest that there may be a cyclical or periodic component to the pattern, and can you tell me if the rules or constraints that govern the placement of X's in your grid are based on a simple iterative process, such as adding or removing X's from one row to the next, or if there is a more complex set of rules at play?\\ \tt Now come to a decision. What do you think: are your grid and the other grid described the same?  Write "DECISION: same grid" if you think they are the same grid or "DECISION: different grids" if you think they are different grids. Do not add anything else.\\ \tt  
	  } 
	   } 
	   } 
	 & \\ 
 

    \theutterance \stepcounter{utterance}  

    & \multicolumn{4}{p{0.6\linewidth}}{\cellcolor[rgb]{0.8,1,0.9}{%
	\makecell[{{p{\linewidth}}}]{% 
	  \tt {\tiny [A$\rangle$GM]}  
	 DECISION: different grids 
	  } 
	   } 
	   } 
	 & & \\ 
 

    \theutterance \stepcounter{utterance}  

    & & & \multicolumn{2}{p{0.3\linewidth}}{\cellcolor[rgb]{0.95,0.95,0.95}{%
	\makecell[{{p{\linewidth}}}]{% 
	  \tt {\tiny [GM$|$GM]}  
	 continue 
	  } 
	   } 
	   } 
	 & & \\ 
 

    \theutterance \stepcounter{utterance}  

    & & & \multicolumn{2}{p{0.3\linewidth}}{\cellcolor[rgb]{0.95,0.95,0.95}{%
	\makecell[{{p{\linewidth}}}]{% 
	  \tt {\tiny [GM$|$GM]}  
	 success 
	  } 
	   } 
	   } 
	 & & \\ 
 

\end{supertabular}
}

\end{document}

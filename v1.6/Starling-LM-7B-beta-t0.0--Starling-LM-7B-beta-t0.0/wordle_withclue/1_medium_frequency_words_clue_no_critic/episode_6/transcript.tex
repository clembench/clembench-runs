
\documentclass{article}
\usepackage{colortbl}
\usepackage{makecell}
\usepackage{multirow}
\usepackage{supertabular}

\begin{document}

\newcounter{utterance}

\twocolumn

{ \footnotesize  \setcounter{utterance}{1}
\setlength{\tabcolsep}{0pt}
\begin{supertabular}{c@{$\;$}|p{.15\linewidth}@{}p{.15\linewidth}p{.15\linewidth}p{.15\linewidth}p{.15\linewidth}p{.15\linewidth}}

    \# & $\;$A & \multicolumn{4}{c}{Game Master} & $\;\:$B\\
    \hline 

    \theutterance \stepcounter{utterance}  

    & & \multicolumn{4}{p{0.6\linewidth}}{\cellcolor[rgb]{0.9,0.9,0.9}{%
	\makecell[{{p{\linewidth}}}]{% 
	  \tt {\tiny [A$\langle$GM]}  
	 You are a language wizard who likes to guess words by using the given rules.\\ \tt \\ \tt Welcome to Wordle! You have six attempts to guess the target word, a valid English word of five lowercase letters (a-z). Please use the tags "guess:" and "explanation:" to provide a concise explanation for each guess.\\ \tt \\ \tt  To help you make an informed guess, you will receive a clue for the word, such as\\ \tt clue: snowy white.\\ \tt \\ \tt Here is an example guess based on the clue:\\ \tt guess: apple\\ \tt explanation: In the fairy tail Snow White, the girl is killed because she eats a poisoned apple. And the word apple has 5 letters.\\ \tt \\ \tt After each guess, your answer will be validated, and you will receive feedback indicating which letters are correct (green), which letters are correct but in the wrong position (yellow), and which letters are incorrect (red). This feedback can be useful in determining which letters to include or exclude in your next guess.\\ \tt \\ \tt For example, the feedback for "apple" might be:\\ \tt guess_feedback: a<yellow> p<yellow> p<green> l<yellow> e<red>\\ \tt \\ \tt The explanation should contain details about how both the guess_feedback and clue work together in tandem to formulate a new guess.\\ \tt \\ \tt Let's begin with your first guess.\\ \tt clue:chin indentation\\ \tt  
	  } 
	   } 
	   } 
	 & \\ 
 

    \theutterance \stepcounter{utterance}  

    & \multicolumn{4}{p{0.6\linewidth}}{\cellcolor[rgb]{0.8,1,0.9}{%
	\makecell[{{p{\linewidth}}}]{% 
	  \tt {\tiny [A$\rangle$GM]}  
	 Guess: indent\\ \tt \\ \tt Explanation: The clue "chin indentation" suggests that the target word is related to a feature on the chin. The word "indent" is a five-letter word that describes a small depression or hollow, which could be a fitting description for a chin indentation. Additionally, the word "indent" has two common letters, "e" and "n," which are likely to appear in the target word.\\ \tt \\ \tt Guess_feedback: 
	  } 
	   } 
	   } 
	 & & \\ 
 

    \theutterance \stepcounter{utterance}  

    & & & \multicolumn{2}{p{0.3\linewidth}}{\cellcolor[rgb]{0.95,0.95,0.95}{%
	\makecell[{{p{\linewidth}}}]{% 
	  \tt {\tiny [GM$|$GM]}  
	 {'guess': 'INVALID_FORMAT', 'explanation': 'INVALID_FORMAT'} 
	  } 
	   } 
	   } 
	 & & \\ 
 

    \theutterance \stepcounter{utterance}  

    & & & \multicolumn{2}{p{0.3\linewidth}}{\cellcolor[rgb]{0.95,0.95,0.95}{%
	\makecell[{{p{\linewidth}}}]{% 
	  \tt {\tiny [GM$|$GM]}  
	 Guesser Error: INVALID_FORMAT while parsing Player 1's (model: Starling-LM-7B-beta) response 
	  } 
	   } 
	   } 
	 & & \\ 
 

    \theutterance \stepcounter{utterance}  

    & & \multicolumn{4}{p{0.6\linewidth}}{\cellcolor[rgb]{0.9,0.9,0.9}{%
	\makecell[{{p{\linewidth}}}]{% 
	  \tt {\tiny [A$\langle$GM]}  
	 Provide your response only in this format:\\ \tt guess:word\\ \tt explanation:details.\\ \tt Please try again 
	  } 
	   } 
	   } 
	 & \\ 
 

    \theutterance \stepcounter{utterance}  

    & \multicolumn{4}{p{0.6\linewidth}}{\cellcolor[rgb]{0.8,1,0.9}{%
	\makecell[{{p{\linewidth}}}]{% 
	  \tt {\tiny [A$\rangle$GM]}  
	 guess: chime\\ \tt explanation: The clue "chin indentation" suggests a word related to a physical feature or a specific part of the body. The word "chime" is a five-letter word that can be associated with the chin, as it is a musical sound that can create an impression or resonate with someone. Additionally, the word "chime" has no direct connection to the word "indentation," which helps avoid common words that might be guessed based on 
	  } 
	   } 
	   } 
	 & & \\ 
 

    \theutterance \stepcounter{utterance}  

    & & & \multicolumn{2}{p{0.3\linewidth}}{\cellcolor[rgb]{0.95,0.95,0.95}{%
	\makecell[{{p{\linewidth}}}]{% 
	  \tt {\tiny [GM$|$GM]}  
	 {'guess': 'chime', 'explanation': 'The clue "chin indentation" suggests a word related to a physical feature or a specific part of the body. The word "chime" is a five-letter word that can be associated with the chin, as it is a musical sound that can create an impression or resonate with someone. Additionally, the word "chime" has no direct connection to the word "indentation," which helps avoid common words that might be guessed based on'} 
	  } 
	   } 
	   } 
	 & & \\ 
 

    \theutterance \stepcounter{utterance}  

    & & & \multicolumn{2}{p{0.3\linewidth}}{\cellcolor[rgb]{0.95,0.95,0.95}{%
	\makecell[{{p{\linewidth}}}]{% 
	  \tt {\tiny [GM$|$GM]}  
	 attempts: 1\\ \tt target_word = cleft\\ \tt target_word_clue = chin indentation\\ \tt guess: chime\\ \tt guess_feedback: c<green> h<red> i<red> m<red> e<yellow> 
	  } 
	   } 
	   } 
	 & & \\ 
 

    \theutterance \stepcounter{utterance}  

    & & \multicolumn{4}{p{0.6\linewidth}}{\cellcolor[rgb]{0.9,0.9,0.9}{%
	\makecell[{{p{\linewidth}}}]{% 
	  \tt {\tiny [A$\langle$GM]}  
	 guess_feedback: c<green> h<red> i<red> m<red> e<yellow> 
	  } 
	   } 
	   } 
	 & \\ 
 

    \theutterance \stepcounter{utterance}  

    & \multicolumn{4}{p{0.6\linewidth}}{\cellcolor[rgb]{0.8,1,0.9}{%
	\makecell[{{p{\linewidth}}}]{% 
	  \tt {\tiny [A$\rangle$GM]}  
	 Based on the feedback, here's the explanation for the next guess:\\ \tt \\ \tt guess: chord\\ \tt explanation: The feedback indicates that the first letter 'c' is correct, but the other letters 'h', 'i', and 'm' are incorrect. The clue "chin indentation" suggests a word related to a physical feature or a specific part of the body. The word "chord" is a five-letter word that can be associated with harmony or 
	  } 
	   } 
	   } 
	 & & \\ 
 

    \theutterance \stepcounter{utterance}  

    & & & \multicolumn{2}{p{0.3\linewidth}}{\cellcolor[rgb]{0.95,0.95,0.95}{%
	\makecell[{{p{\linewidth}}}]{% 
	  \tt {\tiny [GM$|$GM]}  
	 {'guess': '', 'explanation': 'The feedback indicates that the first letter \'c\' is correct, but the other letters \'h\', \'i\', and \'m\' are incorrect. The clue "chin indentation" suggests a word related to a physical feature or a specific part of the body. The word "chord" is a five-letter word that can be associated with harmony or'} 
	  } 
	   } 
	   } 
	 & & \\ 
 

    \theutterance \stepcounter{utterance}  

    & & & \multicolumn{2}{p{0.3\linewidth}}{\cellcolor[rgb]{0.95,0.95,0.95}{%
	\makecell[{{p{\linewidth}}}]{% 
	  \tt {\tiny [GM$|$GM]}  
	 Guesser Error: INVALID_WORD while parsing Player 1's (model: Starling-LM-7B-beta) response 
	  } 
	   } 
	   } 
	 & & \\ 
 

    \theutterance \stepcounter{utterance}  

    & & \multicolumn{4}{p{0.6\linewidth}}{\cellcolor[rgb]{0.9,0.9,0.9}{%
	\makecell[{{p{\linewidth}}}]{% 
	  \tt {\tiny [A$\langle$GM]}  
	 the word should contain only alphabets. Please try again 
	  } 
	   } 
	   } 
	 & \\ 
 

    \theutterance \stepcounter{utterance}  

    & \multicolumn{4}{p{0.6\linewidth}}{\cellcolor[rgb]{0.8,1,0.9}{%
	\makecell[{{p{\linewidth}}}]{% 
	  \tt {\tiny [A$\rangle$GM]}  
	 guess:chord\\ \tt explanation:The clue "chin indentation" suggests a word related to a physical feature or a specific part of the body. The word "chord" is a five-letter word that can be associated with harmony or musical notes, which can create an impression or resonate with someone. The word "chord" has the correct first letter 'c' as indicated by the feedback, and the other letters 'h', 'o', 'r', and 
	  } 
	   } 
	   } 
	 & & \\ 
 

    \theutterance \stepcounter{utterance}  

    & & & \multicolumn{2}{p{0.3\linewidth}}{\cellcolor[rgb]{0.95,0.95,0.95}{%
	\makecell[{{p{\linewidth}}}]{% 
	  \tt {\tiny [GM$|$GM]}  
	 {'guess': 'chord', 'explanation': 'The clue "chin indentation" suggests a word related to a physical feature or a specific part of the body. The word "chord" is a five-letter word that can be associated with harmony or musical notes, which can create an impression or resonate with someone. The word "chord" has the correct first letter \'c\' as indicated by the feedback, and the other letters \'h\', \'o\', \'r\', and'} 
	  } 
	   } 
	   } 
	 & & \\ 
 

    \theutterance \stepcounter{utterance}  

    & & & \multicolumn{2}{p{0.3\linewidth}}{\cellcolor[rgb]{0.95,0.95,0.95}{%
	\makecell[{{p{\linewidth}}}]{% 
	  \tt {\tiny [GM$|$GM]}  
	 attempts: 2\\ \tt target_word = cleft\\ \tt target_word_clue = chin indentation\\ \tt guess: chord\\ \tt guess_feedback: c<green> h<red> o<red> r<red> d<red> 
	  } 
	   } 
	   } 
	 & & \\ 
 

    \theutterance \stepcounter{utterance}  

    & & \multicolumn{4}{p{0.6\linewidth}}{\cellcolor[rgb]{0.9,0.9,0.9}{%
	\makecell[{{p{\linewidth}}}]{% 
	  \tt {\tiny [A$\langle$GM]}  
	 guess_feedback: c<green> h<red> o<red> r<red> d<red> 
	  } 
	   } 
	   } 
	 & \\ 
 

    \theutterance \stepcounter{utterance}  

    & \multicolumn{4}{p{0.6\linewidth}}{\cellcolor[rgb]{0.8,1,0.9}{%
	\makecell[{{p{\linewidth}}}]{% 
	  \tt {\tiny [A$\rangle$GM]}  
	 guess:chord\\ \tt explanation:The feedback indicates that the first letter 'c' is correct, but the other letters 'h', 'o', 'r', and 'd' are incorrect. The clue "chin indentation" suggests a word related to a physical feature or a specific part of the body. The word "chin" is a five-letter word that directly relates to the chin, which is a physical feature. The first letter 'c' matches the 
	  } 
	   } 
	   } 
	 & & \\ 
 

    \theutterance \stepcounter{utterance}  

    & & & \multicolumn{2}{p{0.3\linewidth}}{\cellcolor[rgb]{0.95,0.95,0.95}{%
	\makecell[{{p{\linewidth}}}]{% 
	  \tt {\tiny [GM$|$GM]}  
	 {'guess': 'chord', 'explanation': 'The feedback indicates that the first letter \'c\' is correct, but the other letters \'h\', \'o\', \'r\', and \'d\' are incorrect. The clue "chin indentation" suggests a word related to a physical feature or a specific part of the body. The word "chin" is a five-letter word that directly relates to the chin, which is a physical feature. The first letter \'c\' matches the'} 
	  } 
	   } 
	   } 
	 & & \\ 
 

    \theutterance \stepcounter{utterance}  

    & & & \multicolumn{2}{p{0.3\linewidth}}{\cellcolor[rgb]{0.95,0.95,0.95}{%
	\makecell[{{p{\linewidth}}}]{% 
	  \tt {\tiny [GM$|$GM]}  
	 attempts: 3\\ \tt target_word = cleft\\ \tt target_word_clue = chin indentation\\ \tt guess: chord\\ \tt guess_feedback: c<green> h<red> o<red> r<red> d<red> 
	  } 
	   } 
	   } 
	 & & \\ 
 

    \theutterance \stepcounter{utterance}  

    & & \multicolumn{4}{p{0.6\linewidth}}{\cellcolor[rgb]{0.9,0.9,0.9}{%
	\makecell[{{p{\linewidth}}}]{% 
	  \tt {\tiny [A$\langle$GM]}  
	 guess_feedback: c<green> h<red> o<red> r<red> d<red> 
	  } 
	   } 
	   } 
	 & \\ 
 

    \theutterance \stepcounter{utterance}  

    & \multicolumn{4}{p{0.6\linewidth}}{\cellcolor[rgb]{0.8,1,0.9}{%
	\makecell[{{p{\linewidth}}}]{% 
	  \tt {\tiny [A$\rangle$GM]}  
	 guess:chin\\ \tt explanation:The feedback indicates that the first letter 'c' is correct, but the other letters 'h', 'o', 'r', and 'd' are incorrect. The clue "chin indentation" suggests a word related to a physical feature or a specific part of the body. The word "chin" is a five-letter word that directly relates to the chin, which is a physical feature. The first letter 'c' matches the 
	  } 
	   } 
	   } 
	 & & \\ 
 

    \theutterance \stepcounter{utterance}  

    & & & \multicolumn{2}{p{0.3\linewidth}}{\cellcolor[rgb]{0.95,0.95,0.95}{%
	\makecell[{{p{\linewidth}}}]{% 
	  \tt {\tiny [GM$|$GM]}  
	 {'guess': 'chin', 'explanation': 'The feedback indicates that the first letter \'c\' is correct, but the other letters \'h\', \'o\', \'r\', and \'d\' are incorrect. The clue "chin indentation" suggests a word related to a physical feature or a specific part of the body. The word "chin" is a five-letter word that directly relates to the chin, which is a physical feature. The first letter \'c\' matches the'} 
	  } 
	   } 
	   } 
	 & & \\ 
 

    \theutterance \stepcounter{utterance}  

    & & & \multicolumn{2}{p{0.3\linewidth}}{\cellcolor[rgb]{0.95,0.95,0.95}{%
	\makecell[{{p{\linewidth}}}]{% 
	  \tt {\tiny [GM$|$GM]}  
	 Guesser Error: INVALID_WORD_LENGTH while parsing Player 1's (model: Starling-LM-7B-beta) response 
	  } 
	   } 
	   } 
	 & & \\ 
 

    \theutterance \stepcounter{utterance}  

    & & \multicolumn{4}{p{0.6\linewidth}}{\cellcolor[rgb]{0.9,0.9,0.9}{%
	\makecell[{{p{\linewidth}}}]{% 
	  \tt {\tiny [A$\langle$GM]}  
	 the word should have exactly 5 letters. Please try again 
	  } 
	   } 
	   } 
	 & \\ 
 

    \theutterance \stepcounter{utterance}  

    & \multicolumn{4}{p{0.6\linewidth}}{\cellcolor[rgb]{0.8,1,0.9}{%
	\makecell[{{p{\linewidth}}}]{% 
	  \tt {\tiny [A$\rangle$GM]}  
	 guess:chin\\ \tt explanation:The clue "chin indentation" suggests a word related to a physical feature or a specific part of the body. The word "chin" is a five-letter word that directly relates to the chin, which is a physical feature. The word "chin" has the correct first letter 'c' as indicated by the feedback, and the other letters 'h', 'i', 'n', and 'n' are also correct, 
	  } 
	   } 
	   } 
	 & & \\ 
 

    \theutterance \stepcounter{utterance}  

    & & & \multicolumn{2}{p{0.3\linewidth}}{\cellcolor[rgb]{0.95,0.95,0.95}{%
	\makecell[{{p{\linewidth}}}]{% 
	  \tt {\tiny [GM$|$GM]}  
	 {'guess': 'chin', 'explanation': 'The clue "chin indentation" suggests a word related to a physical feature or a specific part of the body. The word "chin" is a five-letter word that directly relates to the chin, which is a physical feature. The word "chin" has the correct first letter \'c\' as indicated by the feedback, and the other letters \'h\', \'i\', \'n\', and \'n\' are also correct,'} 
	  } 
	   } 
	   } 
	 & & \\ 
 

    \theutterance \stepcounter{utterance}  

    & & & \multicolumn{2}{p{0.3\linewidth}}{\cellcolor[rgb]{0.95,0.95,0.95}{%
	\makecell[{{p{\linewidth}}}]{% 
	  \tt {\tiny [GM$|$GM]}  
	 Guesser Error: INVALID_WORD_LENGTH while parsing Player 1's (model: Starling-LM-7B-beta) response 
	  } 
	   } 
	   } 
	 & & \\ 
 

    \theutterance \stepcounter{utterance}  

    & & \multicolumn{4}{p{0.6\linewidth}}{\cellcolor[rgb]{0.9,0.9,0.9}{%
	\makecell[{{p{\linewidth}}}]{% 
	  \tt {\tiny [A$\langle$GM]}  
	 the word should have exactly 5 letters. Please try again 
	  } 
	   } 
	   } 
	 & \\ 
 

    \theutterance \stepcounter{utterance}  

    & \multicolumn{4}{p{0.6\linewidth}}{\cellcolor[rgb]{0.8,1,0.9}{%
	\makecell[{{p{\linewidth}}}]{% 
	  \tt {\tiny [A$\rangle$GM]}  
	 guess:chime\\ \tt explanation:The clue "chin indentation" suggests a word related to a physical feature or a specific part of the body. The word "chime" is a five-letter word that can be associated with the chin, as it is a musical sound that can create an impression or resonate with someone. Additionally, the word "chime" has no direct connection to the word "indentation," which helps avoid common words that might be guessed based on 
	  } 
	   } 
	   } 
	 & & \\ 
 

    \theutterance \stepcounter{utterance}  

    & & & \multicolumn{2}{p{0.3\linewidth}}{\cellcolor[rgb]{0.95,0.95,0.95}{%
	\makecell[{{p{\linewidth}}}]{% 
	  \tt {\tiny [GM$|$GM]}  
	 {'guess': 'chime', 'explanation': 'The clue "chin indentation" suggests a word related to a physical feature or a specific part of the body. The word "chime" is a five-letter word that can be associated with the chin, as it is a musical sound that can create an impression or resonate with someone. Additionally, the word "chime" has no direct connection to the word "indentation," which helps avoid common words that might be guessed based on'} 
	  } 
	   } 
	   } 
	 & & \\ 
 

    \theutterance \stepcounter{utterance}  

    & & & \multicolumn{2}{p{0.3\linewidth}}{\cellcolor[rgb]{0.95,0.95,0.95}{%
	\makecell[{{p{\linewidth}}}]{% 
	  \tt {\tiny [GM$|$GM]}  
	 attempts: 4\\ \tt target_word = cleft\\ \tt target_word_clue = chin indentation\\ \tt guess: chime\\ \tt guess_feedback: c<green> h<red> i<red> m<red> e<yellow> 
	  } 
	   } 
	   } 
	 & & \\ 
 

    \theutterance \stepcounter{utterance}  

    & & \multicolumn{4}{p{0.6\linewidth}}{\cellcolor[rgb]{0.9,0.9,0.9}{%
	\makecell[{{p{\linewidth}}}]{% 
	  \tt {\tiny [A$\langle$GM]}  
	 guess_feedback: c<green> h<red> i<red> m<red> e<yellow> 
	  } 
	   } 
	   } 
	 & \\ 
 

    \theutterance \stepcounter{utterance}  

    & \multicolumn{4}{p{0.6\linewidth}}{\cellcolor[rgb]{0.8,1,0.9}{%
	\makecell[{{p{\linewidth}}}]{% 
	  \tt {\tiny [A$\rangle$GM]}  
	 guess:chime\\ \tt explanation:The feedback indicates that the first letter 'c' is correct, but the other letters 'h', 'i', 'm', and 'e' are incorrect. The clue "chin indentation" suggests a word related to a physical feature or a specific part of the body. The word "chime" is a five-letter word that can be associated with the chin, as it is a musical sound that can create an impression or resonate 
	  } 
	   } 
	   } 
	 & & \\ 
 

    \theutterance \stepcounter{utterance}  

    & & & \multicolumn{2}{p{0.3\linewidth}}{\cellcolor[rgb]{0.95,0.95,0.95}{%
	\makecell[{{p{\linewidth}}}]{% 
	  \tt {\tiny [GM$|$GM]}  
	 {'guess': 'chime', 'explanation': 'The feedback indicates that the first letter \'c\' is correct, but the other letters \'h\', \'i\', \'m\', and \'e\' are incorrect. The clue "chin indentation" suggests a word related to a physical feature or a specific part of the body. The word "chime" is a five-letter word that can be associated with the chin, as it is a musical sound that can create an impression or resonate'} 
	  } 
	   } 
	   } 
	 & & \\ 
 

    \theutterance \stepcounter{utterance}  

    & & & \multicolumn{2}{p{0.3\linewidth}}{\cellcolor[rgb]{0.95,0.95,0.95}{%
	\makecell[{{p{\linewidth}}}]{% 
	  \tt {\tiny [GM$|$GM]}  
	 attempts: 5\\ \tt target_word = cleft\\ \tt target_word_clue = chin indentation\\ \tt guess: chime\\ \tt guess_feedback: c<green> h<red> i<red> m<red> e<yellow> 
	  } 
	   } 
	   } 
	 & & \\ 
 

    \theutterance \stepcounter{utterance}  

    & & \multicolumn{4}{p{0.6\linewidth}}{\cellcolor[rgb]{0.9,0.9,0.9}{%
	\makecell[{{p{\linewidth}}}]{% 
	  \tt {\tiny [A$\langle$GM]}  
	 guess_feedback: c<green> h<red> i<red> m<red> e<yellow> 
	  } 
	   } 
	   } 
	 & \\ 
 

    \theutterance \stepcounter{utterance}  

    & \multicolumn{4}{p{0.6\linewidth}}{\cellcolor[rgb]{0.8,1,0.9}{%
	\makecell[{{p{\linewidth}}}]{% 
	  \tt {\tiny [A$\rangle$GM]}  
	 guess:chime\\ \tt explanation:The feedback indicates that the first letter 'c' is correct, but the other letters 'h', 'i', 'm', and 'e' are incorrect. The clue "chin indentation" suggests a word related to a physical feature or a specific part of the body. The word "chime" is a five-letter word that can be associated with the chin, as it is a musical sound that can create an impression or resonate 
	  } 
	   } 
	   } 
	 & & \\ 
 

    \theutterance \stepcounter{utterance}  

    & & & \multicolumn{2}{p{0.3\linewidth}}{\cellcolor[rgb]{0.95,0.95,0.95}{%
	\makecell[{{p{\linewidth}}}]{% 
	  \tt {\tiny [GM$|$GM]}  
	 {'guess': 'chime', 'explanation': 'The feedback indicates that the first letter \'c\' is correct, but the other letters \'h\', \'i\', \'m\', and \'e\' are incorrect. The clue "chin indentation" suggests a word related to a physical feature or a specific part of the body. The word "chime" is a five-letter word that can be associated with the chin, as it is a musical sound that can create an impression or resonate'} 
	  } 
	   } 
	   } 
	 & & \\ 
 

    \theutterance \stepcounter{utterance}  

    & & & \multicolumn{2}{p{0.3\linewidth}}{\cellcolor[rgb]{0.95,0.95,0.95}{%
	\makecell[{{p{\linewidth}}}]{% 
	  \tt {\tiny [GM$|$GM]}  
	 attempts: 6\\ \tt target_word = cleft\\ \tt target_word_clue = chin indentation\\ \tt guess: chime\\ \tt guess_feedback: c<green> h<red> i<red> m<red> e<yellow> 
	  } 
	   } 
	   } 
	 & & \\ 
 

    \theutterance \stepcounter{utterance}  

    & & & \multicolumn{2}{p{0.3\linewidth}}{\cellcolor[rgb]{0.95,0.95,0.95}{%
	\makecell[{{p{\linewidth}}}]{% 
	  \tt {\tiny [GM$|$GM]}  
	 game_result = LOSS 
	  } 
	   } 
	   } 
	 & & \\ 
 

\end{supertabular}
}

\end{document}

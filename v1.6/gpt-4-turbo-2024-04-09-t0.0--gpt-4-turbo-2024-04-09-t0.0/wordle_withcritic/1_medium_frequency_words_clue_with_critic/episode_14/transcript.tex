
\documentclass{article}
\usepackage{colortbl}
\usepackage{makecell}
\usepackage{multirow}
\usepackage{supertabular}

\begin{document}

\newcounter{utterance}

\twocolumn

{ \footnotesize  \setcounter{utterance}{1}
\setlength{\tabcolsep}{0pt}
\begin{supertabular}{c@{$\;$}|p{.15\linewidth}@{}p{.15\linewidth}p{.15\linewidth}p{.15\linewidth}p{.15\linewidth}p{.15\linewidth}}

    \# & $\;$A & \multicolumn{4}{c}{Game Master} & $\;\:$B\\
    \hline 

    \theutterance \stepcounter{utterance}  

    & & \multicolumn{4}{p{0.6\linewidth}}{\cellcolor[rgb]{0.9,0.9,0.9}{%
	\makecell[{{p{\linewidth}}}]{% 
	  \tt {\tiny [A$\langle$GM]}  
	 You are a language wizard who likes to guess words by using the given rules.\\ \tt \\ \tt Welcome to Wordle! You have six attempts to guess the target word, a valid English word of five lowercase letters (a-z). Please use the tags "guess:" and "explanation:" to provide a concise explanation for each guess.\\ \tt \\ \tt To help you make an informed guess, you will receive a clue for the word, such as\\ \tt clue: "snowy white"\\ \tt \\ \tt Here is an example guess based on the clue:\\ \tt guess: apple\\ \tt explanation: In the fairy tail Snow White, the girl is killed because she eats a poisoned apple. And the word apple has 5 letters.\\ \tt \\ \tt I will then indicate whether I agree or disagree with your guess and provide rationale, but agreeing with a guess does not confirm its correctness. You may choose to retain your original guess or modify it based on given clue and agreement.\\ \tt \\ \tt After each guess, your answer will be validated, and you will receive feedback indicating which letters are correct (green), which letters are correct but in the wrong position (yellow), and which letters are incorrect (red). This feedback can be useful in determining which letters to include or exclude in your next guess.\\ \tt \\ \tt For example, the feedback for "apple" might be:\\ \tt guess_feedback: a<yellow> p<yellow> p<green> l<yellow> e<red>\\ \tt \\ \tt Here are some rules to keep in mind:\\ \tt 1. Please do not generate any guess feedback or clue on your own\\ \tt 2. Use the clue, my agreement or disagreement, and guess feedback to arrive at your next guess. This should be reflected in your explanation as well.\\ \tt \\ \tt Let's begin with your first guess.\\ \tt \\ \tt clue:bravado\\ \tt  
	  } 
	   } 
	   } 
	 & \\ 
 

    \theutterance \stepcounter{utterance}  

    & \multicolumn{4}{p{0.6\linewidth}}{\cellcolor[rgb]{0.8,1,0.9}{%
	\makecell[{{p{\linewidth}}}]{% 
	  \tt {\tiny [A$\rangle$GM]}  
	 guess: macho\\ \tt explanation: The word "macho" is often associated with a strong, exaggerated sense of masculinity, which aligns with the concept of bravado, defined as a bold manner or a show of boldness intended to impress or intimidate. Additionally, "macho" fits the requirement of being a five-letter word. 
	  } 
	   } 
	   } 
	 & & \\ 
 

    \theutterance \stepcounter{utterance}  

    & & & \multicolumn{2}{p{0.3\linewidth}}{\cellcolor[rgb]{0.95,0.95,0.95}{%
	\makecell[{{p{\linewidth}}}]{% 
	  \tt {\tiny [GM$|$GM]}  
	 {'guess': 'macho', 'explanation': 'The word "macho" is often associated with a strong, exaggerated sense of masculinity, which aligns with the concept of bravado, defined as a bold manner or a show of boldness intended to impress or intimidate. Additionally, "macho" fits the requirement of being a five-letter word.'} 
	  } 
	   } 
	   } 
	 & & \\ 
 

    \theutterance \stepcounter{utterance}  

    & & \multicolumn{4}{p{0.6\linewidth}}{\cellcolor[rgb]{0.9,0.9,0.9}{%
	\makecell[{{p{\linewidth}}}]{% 
	  \tt {\tiny [GM$\rangle$B]}  
	 I need your assistance with a word game in which we need to find a 5-letter word using a clue, a guess and an explanation for the guess.\\ \tt \\ \tt For example, if the clue is "snow white" and the guess is "apple", you will get the information like this:\\ \tt clue:snow white\\ \tt guess:apple\\ \tt explanation:the fairy tale Snow White involves the girl eating a poisoned apple, which has 5 letters.\\ \tt \\ \tt Your task is to either agree or disagree with my guess based on the given clue.\\ \tt \\ \tt If you disagree, I will make another guess, and if you agree, I will proceed in the game using the same guess.\\ \tt \\ \tt Initially, guess feedback is not available, but as the game proceeds, it becomes available. The guess feedback is not for current guess, but rather an indication of what letters can be present in the current guess. A letter highlighted in green indicates that it is a correct letter in the correct position, while yellow indicates that it is a correct letter in the wrong position, and red indicates a wrong letter. At this point, you should use all the provided information, including the clue, guess, explanation, and guess feedback, to decide whether the given guess matches the clue, and also stick to guess feedback or not.\\ \tt \\ \tt Please respond in lowercase letters and stick to this format:\\ \tt agreement:yes or no\\ \tt explanation:your reason for doing so\\ \tt \\ \tt Please note that you must not guess any word based on the riddle clue and stick to the given format while responding.\\ \tt \\ \tt Let's begin!\\ \tt \\ \tt clue:bravado\\ \tt guess:macho\\ \tt explanation:The word "macho" is often associated with a strong, exaggerated sense of masculinity, which aligns with the concept of bravado, defined as a bold manner or a show of boldness intended to impress or intimidate. Additionally, "macho" fits the requirement of being a five-letter word.\\ \tt  
	  } 
	   } 
	   } 
	 & \\ 
 

    \theutterance \stepcounter{utterance}  

    & & & \multicolumn{4}{p{0.6\linewidth}}{\cellcolor[rgb]{1,0.85,0.72}{%
	\makecell[{{p{\linewidth}}}]{% 
	  \tt {\tiny [GM$\langle$B]}  
	 agreement:yes\\ \tt explanation:the guess "macho" accurately reflects the concept of bravado, which involves displaying exaggerated masculinity or boldness, and it meets the requirement of being a five-letter word. 
	  } 
	   } 
	   } 
	  \\ 
 

    \theutterance \stepcounter{utterance}  

    & & & \multicolumn{2}{p{0.3\linewidth}}{\cellcolor[rgb]{0.95,0.95,0.95}{%
	\makecell[{{p{\linewidth}}}]{% 
	  \tt {\tiny [GM$|$GM]}  
	 {'agreement': 'yes', 'explanation': 'the guess "macho" accurately reflects the concept of bravado, which involves displaying exaggerated masculinity or boldness, and it meets the requirement of being a five-letter word.'} 
	  } 
	   } 
	   } 
	 & & \\ 
 

    \theutterance \stepcounter{utterance}  

    & & & \multicolumn{2}{p{0.3\linewidth}}{\cellcolor[rgb]{0.95,0.95,0.95}{%
	\makecell[{{p{\linewidth}}}]{% 
	  \tt {\tiny [GM$|$GM]}  
	 Critic agrees with the Guesser -- Sharing the critic's explanation with the guesser 
	  } 
	   } 
	   } 
	 & & \\ 
 

    \theutterance \stepcounter{utterance}  

    & & \multicolumn{4}{p{0.6\linewidth}}{\cellcolor[rgb]{0.9,0.9,0.9}{%
	\makecell[{{p{\linewidth}}}]{% 
	  \tt {\tiny [A$\langle$GM]}  
	 clue:bravado\\ \tt guess_agreement:yes\\ \tt agreement_explanation:the guess "macho" accurately reflects the concept of bravado, which involves displaying exaggerated masculinity or boldness, and it meets the requirement of being a five-letter word.\\ \tt  
	  } 
	   } 
	   } 
	 & \\ 
 

    \theutterance \stepcounter{utterance}  

    & \multicolumn{4}{p{0.6\linewidth}}{\cellcolor[rgb]{0.8,1,0.9}{%
	\makecell[{{p{\linewidth}}}]{% 
	  \tt {\tiny [A$\rangle$GM]}  
	 Thank you for the agreement. Please provide feedback on the guess "macho" so I can make an informed next guess based on the positions and correctness of the letters. 
	  } 
	   } 
	   } 
	 & & \\ 
 

    \theutterance \stepcounter{utterance}  

    & & & \multicolumn{2}{p{0.3\linewidth}}{\cellcolor[rgb]{0.95,0.95,0.95}{%
	\makecell[{{p{\linewidth}}}]{% 
	  \tt {\tiny [GM$|$GM]}  
	 {'guess': 'INVALID_FORMAT', 'explanation': 'INVALID_FORMAT'} 
	  } 
	   } 
	   } 
	 & & \\ 
 

    \theutterance \stepcounter{utterance}  

    & & & \multicolumn{2}{p{0.3\linewidth}}{\cellcolor[rgb]{0.95,0.95,0.95}{%
	\makecell[{{p{\linewidth}}}]{% 
	  \tt {\tiny [GM$|$GM]}  
	 Guesser Error: INVALID_FORMAT while parsing Player 1's (model: gpt-4-turbo-2024-04-09) response 
	  } 
	   } 
	   } 
	 & & \\ 
 

    \theutterance \stepcounter{utterance}  

    & & \multicolumn{4}{p{0.6\linewidth}}{\cellcolor[rgb]{0.9,0.9,0.9}{%
	\makecell[{{p{\linewidth}}}]{% 
	  \tt {\tiny [A$\langle$GM]}  
	 Provide your response only in this format:\\ \tt guess:word\\ \tt explanation:details.\\ \tt Please try again 
	  } 
	   } 
	   } 
	 & \\ 
 

    \theutterance \stepcounter{utterance}  

    & \multicolumn{4}{p{0.6\linewidth}}{\cellcolor[rgb]{0.8,1,0.9}{%
	\makecell[{{p{\linewidth}}}]{% 
	  \tt {\tiny [A$\rangle$GM]}  
	 guess: macho\\ \tt explanation: The word "macho" is often associated with a strong, exaggerated sense of masculinity, which aligns with the concept of bravado, defined as a bold manner or a show of boldness intended to impress or intimidate. Additionally, "macho" fits the requirement of being a five-letter word. 
	  } 
	   } 
	   } 
	 & & \\ 
 

    \theutterance \stepcounter{utterance}  

    & & & \multicolumn{2}{p{0.3\linewidth}}{\cellcolor[rgb]{0.95,0.95,0.95}{%
	\makecell[{{p{\linewidth}}}]{% 
	  \tt {\tiny [GM$|$GM]}  
	 {'guess': 'macho', 'explanation': 'The word "macho" is often associated with a strong, exaggerated sense of masculinity, which aligns with the concept of bravado, defined as a bold manner or a show of boldness intended to impress or intimidate. Additionally, "macho" fits the requirement of being a five-letter word.'} 
	  } 
	   } 
	   } 
	 & & \\ 
 

    \theutterance \stepcounter{utterance}  

    & & & \multicolumn{2}{p{0.3\linewidth}}{\cellcolor[rgb]{0.95,0.95,0.95}{%
	\makecell[{{p{\linewidth}}}]{% 
	  \tt {\tiny [GM$|$GM]}  
	 No change in player1's guess\\ \tt guess_before_critic_opinion: macho\\ \tt                                                 critic_agreement: yes\\ \tt guess_after_critic_opinion: macho\\ \tt                                                 Proceeding with guess validation 
	  } 
	   } 
	   } 
	 & & \\ 
 

    \theutterance \stepcounter{utterance}  

    & & & \multicolumn{2}{p{0.3\linewidth}}{\cellcolor[rgb]{0.95,0.95,0.95}{%
	\makecell[{{p{\linewidth}}}]{% 
	  \tt {\tiny [GM$|$GM]}  
	 attempts: 1\\ \tt target_word = swash\\ \tt target_word_clue = bravado\\ \tt guess: macho\\ \tt guess_feedback: m<red> a<yellow> c<red> h<yellow> o<red> 
	  } 
	   } 
	   } 
	 & & \\ 
 

    \theutterance \stepcounter{utterance}  

    & & \multicolumn{4}{p{0.6\linewidth}}{\cellcolor[rgb]{0.9,0.9,0.9}{%
	\makecell[{{p{\linewidth}}}]{% 
	  \tt {\tiny [A$\langle$GM]}  
	 guess_feedback: m<red> a<yellow> c<red> h<yellow> o<red> 
	  } 
	   } 
	   } 
	 & \\ 
 

    \theutterance \stepcounter{utterance}  

    & \multicolumn{4}{p{0.6\linewidth}}{\cellcolor[rgb]{0.8,1,0.9}{%
	\makecell[{{p{\linewidth}}}]{% 
	  \tt {\tiny [A$\rangle$GM]}  
	 guess: alpha\\ \tt explanation: Based on the feedback, the letters 'a' and 'h' are in the word but in the wrong positions, and 'm', 'c', and 'o' are not in the word. The word "alpha" includes 'a' and 'h' and fits the theme of bravado, as "alpha" can imply dominance or being the leader, which is a form of bravado. Additionally, it rearranges the positions of 'a' 
	  } 
	   } 
	   } 
	 & & \\ 
 

    \theutterance \stepcounter{utterance}  

    & & & \multicolumn{2}{p{0.3\linewidth}}{\cellcolor[rgb]{0.95,0.95,0.95}{%
	\makecell[{{p{\linewidth}}}]{% 
	  \tt {\tiny [GM$|$GM]}  
	 {'guess': 'alpha', 'explanation': 'Based on the feedback, the letters \'a\' and \'h\' are in the word but in the wrong positions, and \'m\', \'c\', and \'o\' are not in the word. The word "alpha" includes \'a\' and \'h\' and fits the theme of bravado, as "alpha" can imply dominance or being the leader, which is a form of bravado. Additionally, it rearranges the positions of \'a\''} 
	  } 
	   } 
	   } 
	 & & \\ 
 

    \theutterance \stepcounter{utterance}  

    & & \multicolumn{4}{p{0.6\linewidth}}{\cellcolor[rgb]{0.9,0.9,0.9}{%
	\makecell[{{p{\linewidth}}}]{% 
	  \tt {\tiny [GM$\rangle$B]}  
	 clue:bravado\\ \tt guess:alpha\\ \tt explanation:Based on the feedback, the letters 'a' and 'h' are in the word but in the wrong positions, and 'm', 'c', and 'o' are not in the word. The word "alpha" includes 'a' and 'h' and fits the theme of bravado, as "alpha" can imply dominance or being the leader, which is a form of bravado. Additionally, it rearranges the positions of 'a'\\ \tt guess_feedback: m<red> a<yellow> c<red> h<yellow> o<red> 
	  } 
	   } 
	   } 
	 & \\ 
 

    \theutterance \stepcounter{utterance}  

    & & & \multicolumn{4}{p{0.6\linewidth}}{\cellcolor[rgb]{1,0.85,0.72}{%
	\makecell[{{p{\linewidth}}}]{% 
	  \tt {\tiny [GM$\langle$B]}  
	 agreement:no\\ \tt explanation:Although "alpha" rearranges the letters 'a' and 'h' based on the feedback, the guess feedback indicates that both 'a' and 'h' are in the wrong positions in the word "macho". In the guess "alpha", the letter 'a' appears in the first position, which was not highlighted as correct or incorrect in the feedback for "macho", and the letter 'h' is still in the second position, which was 
	  } 
	   } 
	   } 
	  \\ 
 

    \theutterance \stepcounter{utterance}  

    & & & \multicolumn{2}{p{0.3\linewidth}}{\cellcolor[rgb]{0.95,0.95,0.95}{%
	\makecell[{{p{\linewidth}}}]{% 
	  \tt {\tiny [GM$|$GM]}  
	 {'agreement': 'no', 'explanation': 'Although "alpha" rearranges the letters \'a\' and \'h\' based on the feedback, the guess feedback indicates that both \'a\' and \'h\' are in the wrong positions in the word "macho". In the guess "alpha", the letter \'a\' appears in the first position, which was not highlighted as correct or incorrect in the feedback for "macho", and the letter \'h\' is still in the second position, which was'} 
	  } 
	   } 
	   } 
	 & & \\ 
 

    \theutterance \stepcounter{utterance}  

    & & & \multicolumn{2}{p{0.3\linewidth}}{\cellcolor[rgb]{0.95,0.95,0.95}{%
	\makecell[{{p{\linewidth}}}]{% 
	  \tt {\tiny [GM$|$GM]}  
	 Critic disagrees with the Guesser -- Sharing the critic's explanation with the guesser 
	  } 
	   } 
	   } 
	 & & \\ 
 

    \theutterance \stepcounter{utterance}  

    & & \multicolumn{4}{p{0.6\linewidth}}{\cellcolor[rgb]{0.9,0.9,0.9}{%
	\makecell[{{p{\linewidth}}}]{% 
	  \tt {\tiny [A$\langle$GM]}  
	 clue:bravado\\ \tt guess_agreement:no\\ \tt agreement_explanation:Although "alpha" rearranges the letters 'a' and 'h' based on the feedback, the guess feedback indicates that both 'a' and 'h' are in the wrong positions in the word "macho". In the guess "alpha", the letter 'a' appears in the first position, which was not highlighted as correct or incorrect in the feedback for "macho", and the letter 'h' is still in the second position, which was\\ \tt  
	  } 
	   } 
	   } 
	 & \\ 
 

    \theutterance \stepcounter{utterance}  

    & \multicolumn{4}{p{0.6\linewidth}}{\cellcolor[rgb]{0.8,1,0.9}{%
	\makecell[{{p{\linewidth}}}]{% 
	  \tt {\tiny [A$\rangle$GM]}  
	 guess:harsh\\ \tt explanation:Considering the feedback and the clue, the letters 'a' and 'h' need to be rearranged. "Harsh" includes both 'a' and 'h' in new positions, fitting the requirement of bravado by implying a severe or strict demeanor, which can be a form of boldness or assertiveness. This guess also tests new positions for the letters 'a' and 'h' while introducing 'r', 's', and another ' 
	  } 
	   } 
	   } 
	 & & \\ 
 

    \theutterance \stepcounter{utterance}  

    & & & \multicolumn{2}{p{0.3\linewidth}}{\cellcolor[rgb]{0.95,0.95,0.95}{%
	\makecell[{{p{\linewidth}}}]{% 
	  \tt {\tiny [GM$|$GM]}  
	 {'guess': 'harsh', 'explanation': 'Considering the feedback and the clue, the letters \'a\' and \'h\' need to be rearranged. "Harsh" includes both \'a\' and \'h\' in new positions, fitting the requirement of bravado by implying a severe or strict demeanor, which can be a form of boldness or assertiveness. This guess also tests new positions for the letters \'a\' and \'h\' while introducing \'r\', \'s\', and another \''} 
	  } 
	   } 
	   } 
	 & & \\ 
 

    \theutterance \stepcounter{utterance}  

    & & & \multicolumn{2}{p{0.3\linewidth}}{\cellcolor[rgb]{0.95,0.95,0.95}{%
	\makecell[{{p{\linewidth}}}]{% 
	  \tt {\tiny [GM$|$GM]}  
	 Change in player1's guess\\ \tt guess_before_critic_opinion: alpha\\ \tt                                                 critic_agreement: no\\ \tt guess_after_critic_opinion: harsh\\ \tt                                                 Proceeding with guess validation 
	  } 
	   } 
	   } 
	 & & \\ 
 

    \theutterance \stepcounter{utterance}  

    & & & \multicolumn{2}{p{0.3\linewidth}}{\cellcolor[rgb]{0.95,0.95,0.95}{%
	\makecell[{{p{\linewidth}}}]{% 
	  \tt {\tiny [GM$|$GM]}  
	 attempts: 2\\ \tt target_word = swash\\ \tt target_word_clue = bravado\\ \tt guess: harsh\\ \tt guess_feedback: h<red> a<yellow> r<red> s<yellow> h<green> 
	  } 
	   } 
	   } 
	 & & \\ 
 

    \theutterance \stepcounter{utterance}  

    & & \multicolumn{4}{p{0.6\linewidth}}{\cellcolor[rgb]{0.9,0.9,0.9}{%
	\makecell[{{p{\linewidth}}}]{% 
	  \tt {\tiny [A$\langle$GM]}  
	 guess_feedback: h<red> a<yellow> r<red> s<yellow> h<green> 
	  } 
	   } 
	   } 
	 & \\ 
 

    \theutterance \stepcounter{utterance}  

    & \multicolumn{4}{p{0.6\linewidth}}{\cellcolor[rgb]{0.8,1,0.9}{%
	\makecell[{{p{\linewidth}}}]{% 
	  \tt {\tiny [A$\rangle$GM]}  
	 guess: shade\\ \tt explanation: From the feedback, the letter 'h' is correct and in the right position, while 'a' and 's' are correct but in the wrong positions. The word "shade" rearranges 'a' and 's' and keeps 'h' in the correct position. The term "shade" can metaphorically relate to bravado, as in "throwing shade" or displaying a subtle form of confrontation or bravado. 
	  } 
	   } 
	   } 
	 & & \\ 
 

    \theutterance \stepcounter{utterance}  

    & & & \multicolumn{2}{p{0.3\linewidth}}{\cellcolor[rgb]{0.95,0.95,0.95}{%
	\makecell[{{p{\linewidth}}}]{% 
	  \tt {\tiny [GM$|$GM]}  
	 {'guess': 'shade', 'explanation': 'From the feedback, the letter \'h\' is correct and in the right position, while \'a\' and \'s\' are correct but in the wrong positions. The word "shade" rearranges \'a\' and \'s\' and keeps \'h\' in the correct position. The term "shade" can metaphorically relate to bravado, as in "throwing shade" or displaying a subtle form of confrontation or bravado.'} 
	  } 
	   } 
	   } 
	 & & \\ 
 

    \theutterance \stepcounter{utterance}  

    & & \multicolumn{4}{p{0.6\linewidth}}{\cellcolor[rgb]{0.9,0.9,0.9}{%
	\makecell[{{p{\linewidth}}}]{% 
	  \tt {\tiny [GM$\rangle$B]}  
	 clue:bravado\\ \tt guess:shade\\ \tt explanation:From the feedback, the letter 'h' is correct and in the right position, while 'a' and 's' are correct but in the wrong positions. The word "shade" rearranges 'a' and 's' and keeps 'h' in the correct position. The term "shade" can metaphorically relate to bravado, as in "throwing shade" or displaying a subtle form of confrontation or bravado.\\ \tt guess_feedback: h<red> a<yellow> r<red> s<yellow> h<green> 
	  } 
	   } 
	   } 
	 & \\ 
 

    \theutterance \stepcounter{utterance}  

    & & & \multicolumn{4}{p{0.6\linewidth}}{\cellcolor[rgb]{1,0.85,0.72}{%
	\makecell[{{p{\linewidth}}}]{% 
	  \tt {\tiny [GM$\langle$B]}  
	 agreement:no\\ \tt explanation:The feedback indicates that the letter 'h' is correct and in the right position, and 'a' and 's' are correct but in the wrong positions. However, the word "shade" places 'a' in the same position as previously indicated incorrect by the feedback (second position), which contradicts the provided guess feedback. Additionally, the metaphorical connection between "shade" and bravado is weaker compared to more direct interpretations of the clue. 
	  } 
	   } 
	   } 
	  \\ 
 

    \theutterance \stepcounter{utterance}  

    & & & \multicolumn{2}{p{0.3\linewidth}}{\cellcolor[rgb]{0.95,0.95,0.95}{%
	\makecell[{{p{\linewidth}}}]{% 
	  \tt {\tiny [GM$|$GM]}  
	 {'agreement': 'no', 'explanation': 'The feedback indicates that the letter \'h\' is correct and in the right position, and \'a\' and \'s\' are correct but in the wrong positions. However, the word "shade" places \'a\' in the same position as previously indicated incorrect by the feedback (second position), which contradicts the provided guess feedback. Additionally, the metaphorical connection between "shade" and bravado is weaker compared to more direct interpretations of the clue.'} 
	  } 
	   } 
	   } 
	 & & \\ 
 

    \theutterance \stepcounter{utterance}  

    & & & \multicolumn{2}{p{0.3\linewidth}}{\cellcolor[rgb]{0.95,0.95,0.95}{%
	\makecell[{{p{\linewidth}}}]{% 
	  \tt {\tiny [GM$|$GM]}  
	 Critic disagrees with the Guesser -- Sharing the critic's explanation with the guesser 
	  } 
	   } 
	   } 
	 & & \\ 
 

    \theutterance \stepcounter{utterance}  

    & & \multicolumn{4}{p{0.6\linewidth}}{\cellcolor[rgb]{0.9,0.9,0.9}{%
	\makecell[{{p{\linewidth}}}]{% 
	  \tt {\tiny [A$\langle$GM]}  
	 clue:bravado\\ \tt guess_agreement:no\\ \tt agreement_explanation:The feedback indicates that the letter 'h' is correct and in the right position, and 'a' and 's' are correct but in the wrong positions. However, the word "shade" places 'a' in the same position as previously indicated incorrect by the feedback (second position), which contradicts the provided guess feedback. Additionally, the metaphorical connection between "shade" and bravado is weaker compared to more direct interpretations of the clue.\\ \tt  
	  } 
	   } 
	   } 
	 & \\ 
 

    \theutterance \stepcounter{utterance}  

    & \multicolumn{4}{p{0.6\linewidth}}{\cellcolor[rgb]{0.8,1,0.9}{%
	\makecell[{{p{\linewidth}}}]{% 
	  \tt {\tiny [A$\rangle$GM]}  
	 guess:slash\\ \tt explanation:Based on the feedback, 'h' is correct and in the right position, while 'a' and 's' are correct but were in the wrong positions in previous guesses. The word "slash" places 's' and 'a' in new positions, potentially aligning with the feedback. The term "slash" can imply a bold or aggressive action, fitting the theme of bravado. 
	  } 
	   } 
	   } 
	 & & \\ 
 

    \theutterance \stepcounter{utterance}  

    & & & \multicolumn{2}{p{0.3\linewidth}}{\cellcolor[rgb]{0.95,0.95,0.95}{%
	\makecell[{{p{\linewidth}}}]{% 
	  \tt {\tiny [GM$|$GM]}  
	 {'guess': 'slash', 'explanation': 'Based on the feedback, \'h\' is correct and in the right position, while \'a\' and \'s\' are correct but were in the wrong positions in previous guesses. The word "slash" places \'s\' and \'a\' in new positions, potentially aligning with the feedback. The term "slash" can imply a bold or aggressive action, fitting the theme of bravado.'} 
	  } 
	   } 
	   } 
	 & & \\ 
 

    \theutterance \stepcounter{utterance}  

    & & & \multicolumn{2}{p{0.3\linewidth}}{\cellcolor[rgb]{0.95,0.95,0.95}{%
	\makecell[{{p{\linewidth}}}]{% 
	  \tt {\tiny [GM$|$GM]}  
	 Change in player1's guess\\ \tt guess_before_critic_opinion: shade\\ \tt                                                 critic_agreement: no\\ \tt guess_after_critic_opinion: slash\\ \tt                                                 Proceeding with guess validation 
	  } 
	   } 
	   } 
	 & & \\ 
 

    \theutterance \stepcounter{utterance}  

    & & & \multicolumn{2}{p{0.3\linewidth}}{\cellcolor[rgb]{0.95,0.95,0.95}{%
	\makecell[{{p{\linewidth}}}]{% 
	  \tt {\tiny [GM$|$GM]}  
	 attempts: 3\\ \tt target_word = swash\\ \tt target_word_clue = bravado\\ \tt guess: slash\\ \tt guess_feedback: s<green> l<red> a<green> s<green> h<green> 
	  } 
	   } 
	   } 
	 & & \\ 
 

    \theutterance \stepcounter{utterance}  

    & & \multicolumn{4}{p{0.6\linewidth}}{\cellcolor[rgb]{0.9,0.9,0.9}{%
	\makecell[{{p{\linewidth}}}]{% 
	  \tt {\tiny [A$\langle$GM]}  
	 guess_feedback: s<green> l<red> a<green> s<green> h<green> 
	  } 
	   } 
	   } 
	 & \\ 
 

    \theutterance \stepcounter{utterance}  

    & \multicolumn{4}{p{0.6\linewidth}}{\cellcolor[rgb]{0.8,1,0.9}{%
	\makecell[{{p{\linewidth}}}]{% 
	  \tt {\tiny [A$\rangle$GM]}  
	 guess:stash\\ \tt explanation:From the feedback, the letters 's', 'a', 's', and 'h' are correct and in the right positions, except for the second letter which needs to be changed. The word "stash" fits this pattern, replacing 'l' with 't'. The term "stash" can imply keeping something valuable hidden, which might be metaphorically linked to maintaining a facade of bravado. 
	  } 
	   } 
	   } 
	 & & \\ 
 

    \theutterance \stepcounter{utterance}  

    & & & \multicolumn{2}{p{0.3\linewidth}}{\cellcolor[rgb]{0.95,0.95,0.95}{%
	\makecell[{{p{\linewidth}}}]{% 
	  \tt {\tiny [GM$|$GM]}  
	 {'guess': 'stash', 'explanation': 'From the feedback, the letters \'s\', \'a\', \'s\', and \'h\' are correct and in the right positions, except for the second letter which needs to be changed. The word "stash" fits this pattern, replacing \'l\' with \'t\'. The term "stash" can imply keeping something valuable hidden, which might be metaphorically linked to maintaining a facade of bravado.'} 
	  } 
	   } 
	   } 
	 & & \\ 
 

    \theutterance \stepcounter{utterance}  

    & & \multicolumn{4}{p{0.6\linewidth}}{\cellcolor[rgb]{0.9,0.9,0.9}{%
	\makecell[{{p{\linewidth}}}]{% 
	  \tt {\tiny [GM$\rangle$B]}  
	 clue:bravado\\ \tt guess:stash\\ \tt explanation:From the feedback, the letters 's', 'a', 's', and 'h' are correct and in the right positions, except for the second letter which needs to be changed. The word "stash" fits this pattern, replacing 'l' with 't'. The term "stash" can imply keeping something valuable hidden, which might be metaphorically linked to maintaining a facade of bravado.\\ \tt guess_feedback: s<green> l<red> a<green> s<green> h<green> 
	  } 
	   } 
	   } 
	 & \\ 
 

    \theutterance \stepcounter{utterance}  

    & & & \multicolumn{4}{p{0.6\linewidth}}{\cellcolor[rgb]{1,0.85,0.72}{%
	\makecell[{{p{\linewidth}}}]{% 
	  \tt {\tiny [GM$\langle$B]}  
	 agreement:no\\ \tt explanation:The word "stash" does not align well with the concept of bravado. Bravado involves a display of boldness or confidence, often in a public or outward manner, whereas "stash" refers to hiding or storing something away, which is more secretive and does not convey the same meaning as bravado. Additionally, the explanation does not convincingly link "stash" to the clue of bravado. 
	  } 
	   } 
	   } 
	  \\ 
 

    \theutterance \stepcounter{utterance}  

    & & & \multicolumn{2}{p{0.3\linewidth}}{\cellcolor[rgb]{0.95,0.95,0.95}{%
	\makecell[{{p{\linewidth}}}]{% 
	  \tt {\tiny [GM$|$GM]}  
	 {'agreement': 'no', 'explanation': 'The word "stash" does not align well with the concept of bravado. Bravado involves a display of boldness or confidence, often in a public or outward manner, whereas "stash" refers to hiding or storing something away, which is more secretive and does not convey the same meaning as bravado. Additionally, the explanation does not convincingly link "stash" to the clue of bravado.'} 
	  } 
	   } 
	   } 
	 & & \\ 
 

    \theutterance \stepcounter{utterance}  

    & & & \multicolumn{2}{p{0.3\linewidth}}{\cellcolor[rgb]{0.95,0.95,0.95}{%
	\makecell[{{p{\linewidth}}}]{% 
	  \tt {\tiny [GM$|$GM]}  
	 Critic disagrees with the Guesser -- Sharing the critic's explanation with the guesser 
	  } 
	   } 
	   } 
	 & & \\ 
 

    \theutterance \stepcounter{utterance}  

    & & \multicolumn{4}{p{0.6\linewidth}}{\cellcolor[rgb]{0.9,0.9,0.9}{%
	\makecell[{{p{\linewidth}}}]{% 
	  \tt {\tiny [A$\langle$GM]}  
	 clue:bravado\\ \tt guess_agreement:no\\ \tt agreement_explanation:The word "stash" does not align well with the concept of bravado. Bravado involves a display of boldness or confidence, often in a public or outward manner, whereas "stash" refers to hiding or storing something away, which is more secretive and does not convey the same meaning as bravado. Additionally, the explanation does not convincingly link "stash" to the clue of bravado.\\ \tt  
	  } 
	   } 
	   } 
	 & \\ 
 

    \theutterance \stepcounter{utterance}  

    & \multicolumn{4}{p{0.6\linewidth}}{\cellcolor[rgb]{0.8,1,0.9}{%
	\makecell[{{p{\linewidth}}}]{% 
	  \tt {\tiny [A$\rangle$GM]}  
	 guess:slash\\ \tt explanation:Returning to the previous feedback where 's', 'a', 's', and 'h' were correct and in the right positions, and only the second letter needed adjustment, "slash" fits this pattern perfectly. The word "slash" can be associated with bravado as it suggests a bold or dramatic action, which aligns well with the concept of displaying boldness or confidence. 
	  } 
	   } 
	   } 
	 & & \\ 
 

    \theutterance \stepcounter{utterance}  

    & & & \multicolumn{2}{p{0.3\linewidth}}{\cellcolor[rgb]{0.95,0.95,0.95}{%
	\makecell[{{p{\linewidth}}}]{% 
	  \tt {\tiny [GM$|$GM]}  
	 {'guess': 'slash', 'explanation': 'Returning to the previous feedback where \'s\', \'a\', \'s\', and \'h\' were correct and in the right positions, and only the second letter needed adjustment, "slash" fits this pattern perfectly. The word "slash" can be associated with bravado as it suggests a bold or dramatic action, which aligns well with the concept of displaying boldness or confidence.'} 
	  } 
	   } 
	   } 
	 & & \\ 
 

    \theutterance \stepcounter{utterance}  

    & & & \multicolumn{2}{p{0.3\linewidth}}{\cellcolor[rgb]{0.95,0.95,0.95}{%
	\makecell[{{p{\linewidth}}}]{% 
	  \tt {\tiny [GM$|$GM]}  
	 Change in player1's guess\\ \tt guess_before_critic_opinion: stash\\ \tt                                                 critic_agreement: no\\ \tt guess_after_critic_opinion: slash\\ \tt                                                 Proceeding with guess validation 
	  } 
	   } 
	   } 
	 & & \\ 
 

    \theutterance \stepcounter{utterance}  

    & & & \multicolumn{2}{p{0.3\linewidth}}{\cellcolor[rgb]{0.95,0.95,0.95}{%
	\makecell[{{p{\linewidth}}}]{% 
	  \tt {\tiny [GM$|$GM]}  
	 attempts: 4\\ \tt target_word = swash\\ \tt target_word_clue = bravado\\ \tt guess: slash\\ \tt guess_feedback: s<green> l<red> a<green> s<green> h<green> 
	  } 
	   } 
	   } 
	 & & \\ 
 

    \theutterance \stepcounter{utterance}  

    & & \multicolumn{4}{p{0.6\linewidth}}{\cellcolor[rgb]{0.9,0.9,0.9}{%
	\makecell[{{p{\linewidth}}}]{% 
	  \tt {\tiny [A$\langle$GM]}  
	 guess_feedback: s<green> l<red> a<green> s<green> h<green> 
	  } 
	   } 
	   } 
	 & \\ 
 

    \theutterance \stepcounter{utterance}  

    & \multicolumn{4}{p{0.6\linewidth}}{\cellcolor[rgb]{0.8,1,0.9}{%
	\makecell[{{p{\linewidth}}}]{% 
	  \tt {\tiny [A$\rangle$GM]}  
	 guess:swash\\ \tt explanation:Based on the feedback, the letters 's', 'a', 's', and 'h' are correct and in the right positions, but the second letter needs to be different. The word "swash" fits this pattern, replacing 'l' with 'w'. "Swash" relates to bravado as it can refer to swashbuckling, which is a flamboyant, daring, and adventurous style, often associated with exaggerated hero 
	  } 
	   } 
	   } 
	 & & \\ 
 

    \theutterance \stepcounter{utterance}  

    & & & \multicolumn{2}{p{0.3\linewidth}}{\cellcolor[rgb]{0.95,0.95,0.95}{%
	\makecell[{{p{\linewidth}}}]{% 
	  \tt {\tiny [GM$|$GM]}  
	 {'guess': 'swash', 'explanation': 'Based on the feedback, the letters \'s\', \'a\', \'s\', and \'h\' are correct and in the right positions, but the second letter needs to be different. The word "swash" fits this pattern, replacing \'l\' with \'w\'. "Swash" relates to bravado as it can refer to swashbuckling, which is a flamboyant, daring, and adventurous style, often associated with exaggerated hero'} 
	  } 
	   } 
	   } 
	 & & \\ 
 

    \theutterance \stepcounter{utterance}  

    & & \multicolumn{4}{p{0.6\linewidth}}{\cellcolor[rgb]{0.9,0.9,0.9}{%
	\makecell[{{p{\linewidth}}}]{% 
	  \tt {\tiny [GM$\rangle$B]}  
	 clue:bravado\\ \tt guess:swash\\ \tt explanation:Based on the feedback, the letters 's', 'a', 's', and 'h' are correct and in the right positions, but the second letter needs to be different. The word "swash" fits this pattern, replacing 'l' with 'w'. "Swash" relates to bravado as it can refer to swashbuckling, which is a flamboyant, daring, and adventurous style, often associated with exaggerated hero\\ \tt guess_feedback: s<green> l<red> a<green> s<green> h<green> 
	  } 
	   } 
	   } 
	 & \\ 
 

    \theutterance \stepcounter{utterance}  

    & & & \multicolumn{4}{p{0.6\linewidth}}{\cellcolor[rgb]{1,0.85,0.72}{%
	\makecell[{{p{\linewidth}}}]{% 
	  \tt {\tiny [GM$\langle$B]}  
	 agreement:yes\\ \tt explanation:The guess "swash" correctly follows the feedback pattern with 's', 'a', 's', and 'h' in the correct positions and replaces the incorrect 'l' with 'w'. The term "swash" aligns well with the clue "bravado" as it evokes the sense of swashbuckling, which is typically characterized by bold, adventurous, and somewhat showy behavior. 
	  } 
	   } 
	   } 
	  \\ 
 

    \theutterance \stepcounter{utterance}  

    & & & \multicolumn{2}{p{0.3\linewidth}}{\cellcolor[rgb]{0.95,0.95,0.95}{%
	\makecell[{{p{\linewidth}}}]{% 
	  \tt {\tiny [GM$|$GM]}  
	 {'agreement': 'yes', 'explanation': 'The guess "swash" correctly follows the feedback pattern with \'s\', \'a\', \'s\', and \'h\' in the correct positions and replaces the incorrect \'l\' with \'w\'. The term "swash" aligns well with the clue "bravado" as it evokes the sense of swashbuckling, which is typically characterized by bold, adventurous, and somewhat showy behavior.'} 
	  } 
	   } 
	   } 
	 & & \\ 
 

    \theutterance \stepcounter{utterance}  

    & & & \multicolumn{2}{p{0.3\linewidth}}{\cellcolor[rgb]{0.95,0.95,0.95}{%
	\makecell[{{p{\linewidth}}}]{% 
	  \tt {\tiny [GM$|$GM]}  
	 Critic agrees with the Guesser -- Sharing the critic's explanation with the guesser 
	  } 
	   } 
	   } 
	 & & \\ 
 

    \theutterance \stepcounter{utterance}  

    & & \multicolumn{4}{p{0.6\linewidth}}{\cellcolor[rgb]{0.9,0.9,0.9}{%
	\makecell[{{p{\linewidth}}}]{% 
	  \tt {\tiny [A$\langle$GM]}  
	 clue:bravado\\ \tt guess_agreement:yes\\ \tt agreement_explanation:The guess "swash" correctly follows the feedback pattern with 's', 'a', 's', and 'h' in the correct positions and replaces the incorrect 'l' with 'w'. The term "swash" aligns well with the clue "bravado" as it evokes the sense of swashbuckling, which is typically characterized by bold, adventurous, and somewhat showy behavior.\\ \tt  
	  } 
	   } 
	   } 
	 & \\ 
 

    \theutterance \stepcounter{utterance}  

    & \multicolumn{4}{p{0.6\linewidth}}{\cellcolor[rgb]{0.8,1,0.9}{%
	\makecell[{{p{\linewidth}}}]{% 
	  \tt {\tiny [A$\rangle$GM]}  
	 Thank you for confirming the alignment and the explanation. If there's any further action or feedback needed, please let me know! 
	  } 
	   } 
	   } 
	 & & \\ 
 

    \theutterance \stepcounter{utterance}  

    & & & \multicolumn{2}{p{0.3\linewidth}}{\cellcolor[rgb]{0.95,0.95,0.95}{%
	\makecell[{{p{\linewidth}}}]{% 
	  \tt {\tiny [GM$|$GM]}  
	 {'guess': 'INVALID_FORMAT', 'explanation': 'INVALID_FORMAT'} 
	  } 
	   } 
	   } 
	 & & \\ 
 

    \theutterance \stepcounter{utterance}  

    & & & \multicolumn{2}{p{0.3\linewidth}}{\cellcolor[rgb]{0.95,0.95,0.95}{%
	\makecell[{{p{\linewidth}}}]{% 
	  \tt {\tiny [GM$|$GM]}  
	 Guesser Error: INVALID_FORMAT while parsing Player 1's (model: gpt-4-turbo-2024-04-09) response 
	  } 
	   } 
	   } 
	 & & \\ 
 

    \theutterance \stepcounter{utterance}  

    & & \multicolumn{4}{p{0.6\linewidth}}{\cellcolor[rgb]{0.9,0.9,0.9}{%
	\makecell[{{p{\linewidth}}}]{% 
	  \tt {\tiny [A$\langle$GM]}  
	 Provide your response only in this format:\\ \tt guess:word\\ \tt explanation:details.\\ \tt Please try again 
	  } 
	   } 
	   } 
	 & \\ 
 

    \theutterance \stepcounter{utterance}  

    & \multicolumn{4}{p{0.6\linewidth}}{\cellcolor[rgb]{0.8,1,0.9}{%
	\makecell[{{p{\linewidth}}}]{% 
	  \tt {\tiny [A$\rangle$GM]}  
	 guess:swash\\ \tt explanation:Based on the feedback, the letters 's', 'a', 's', and 'h' are correct and in the right positions, but the second letter needed to be different. The word "swash" fits this pattern, replacing 'l' with 'w'. "Swash" relates to bravado as it can refer to swashbuckling, which is a flamboyant, daring, and adventurous style, often associated with exaggerated hero 
	  } 
	   } 
	   } 
	 & & \\ 
 

    \theutterance \stepcounter{utterance}  

    & & & \multicolumn{2}{p{0.3\linewidth}}{\cellcolor[rgb]{0.95,0.95,0.95}{%
	\makecell[{{p{\linewidth}}}]{% 
	  \tt {\tiny [GM$|$GM]}  
	 {'guess': 'swash', 'explanation': 'Based on the feedback, the letters \'s\', \'a\', \'s\', and \'h\' are correct and in the right positions, but the second letter needed to be different. The word "swash" fits this pattern, replacing \'l\' with \'w\'. "Swash" relates to bravado as it can refer to swashbuckling, which is a flamboyant, daring, and adventurous style, often associated with exaggerated hero'} 
	  } 
	   } 
	   } 
	 & & \\ 
 

    \theutterance \stepcounter{utterance}  

    & & & \multicolumn{2}{p{0.3\linewidth}}{\cellcolor[rgb]{0.95,0.95,0.95}{%
	\makecell[{{p{\linewidth}}}]{% 
	  \tt {\tiny [GM$|$GM]}  
	 No change in player1's guess\\ \tt guess_before_critic_opinion: swash\\ \tt                                                 critic_agreement: yes\\ \tt guess_after_critic_opinion: swash\\ \tt                                                 Proceeding with guess validation 
	  } 
	   } 
	   } 
	 & & \\ 
 

    \theutterance \stepcounter{utterance}  

    & & & \multicolumn{2}{p{0.3\linewidth}}{\cellcolor[rgb]{0.95,0.95,0.95}{%
	\makecell[{{p{\linewidth}}}]{% 
	  \tt {\tiny [GM$|$GM]}  
	 attempts: 5\\ \tt target_word = swash\\ \tt target_word_clue = bravado\\ \tt guess: swash\\ \tt guess_feedback: s<green> w<green> a<green> s<green> h<green> 
	  } 
	   } 
	   } 
	 & & \\ 
 

    \theutterance \stepcounter{utterance}  

    & & & \multicolumn{2}{p{0.3\linewidth}}{\cellcolor[rgb]{0.95,0.95,0.95}{%
	\makecell[{{p{\linewidth}}}]{% 
	  \tt {\tiny [GM$|$GM]}  
	 game_result = WIN 
	  } 
	   } 
	   } 
	 & & \\ 
 

\end{supertabular}
}

\end{document}


\documentclass{article}
\usepackage{colortbl}
\usepackage{makecell}
\usepackage{multirow}
\usepackage{supertabular}

\begin{document}

\newcounter{utterance}

\twocolumn

{ \footnotesize  \setcounter{utterance}{1}
\setlength{\tabcolsep}{0pt}
\begin{supertabular}{c@{$\;$}|p{.15\linewidth}@{}p{.15\linewidth}p{.15\linewidth}p{.15\linewidth}p{.15\linewidth}p{.15\linewidth}}

    \# & $\;$A & \multicolumn{4}{c}{Game Master} & $\;\:$B\\
    \hline 

    \theutterance \stepcounter{utterance}  

    & & \multicolumn{4}{p{0.6\linewidth}}{\cellcolor[rgb]{0.9,0.9,0.9}{%
	\makecell[{{p{\linewidth}}}]{% 
	  \tt {\tiny [A$\langle$GM]}  
	 You are a language wizard who likes to guess words by using the given rules.\\ \tt \\ \tt Welcome to Wordle! You have six attempts to guess the target word, a valid English word of five lowercase letters (a-z). Please use the tags "guess:" and "explanation:" to provide a concise explanation for each guess.\\ \tt \\ \tt To help you make an informed guess, you will receive a clue for the word, such as\\ \tt clue: "snowy white"\\ \tt \\ \tt Here is an example guess based on the clue:\\ \tt guess: apple\\ \tt explanation: In the fairy tail Snow White, the girl is killed because she eats a poisoned apple. And the word apple has 5 letters.\\ \tt \\ \tt I will then indicate whether I agree or disagree with your guess and provide rationale, but agreeing with a guess does not confirm its correctness. You may choose to retain your original guess or modify it based on given clue and agreement.\\ \tt \\ \tt After each guess, your answer will be validated, and you will receive feedback indicating which letters are correct (green), which letters are correct but in the wrong position (yellow), and which letters are incorrect (red). This feedback can be useful in determining which letters to include or exclude in your next guess.\\ \tt \\ \tt For example, the feedback for "apple" might be:\\ \tt guess_feedback: a<yellow> p<yellow> p<green> l<yellow> e<red>\\ \tt \\ \tt Here are some rules to keep in mind:\\ \tt 1. Please do not generate any guess feedback or clue on your own\\ \tt 2. Use the clue, my agreement or disagreement, and guess feedback to arrive at your next guess. This should be reflected in your explanation as well.\\ \tt \\ \tt Let's begin with your first guess.\\ \tt \\ \tt clue:square\\ \tt  
	  } 
	   } 
	   } 
	 & \\ 
 

    \theutterance \stepcounter{utterance}  

    & \multicolumn{4}{p{0.6\linewidth}}{\cellcolor[rgb]{0.8,1,0.9}{%
	\makecell[{{p{\linewidth}}}]{% 
	  \tt {\tiny [A$\rangle$GM]}  
	 guess: block\\ \tt explanation: A block can refer to a square shape, such as in building blocks, which are often square. The word "block" has five letters, fitting the criteria. 
	  } 
	   } 
	   } 
	 & & \\ 
 

    \theutterance \stepcounter{utterance}  

    & & & \multicolumn{2}{p{0.3\linewidth}}{\cellcolor[rgb]{0.95,0.95,0.95}{%
	\makecell[{{p{\linewidth}}}]{% 
	  \tt {\tiny [GM$|$GM]}  
	 {'guess': 'block', 'explanation': 'A block can refer to a square shape, such as in building blocks, which are often square. The word "block" has five letters, fitting the criteria.'} 
	  } 
	   } 
	   } 
	 & & \\ 
 

    \theutterance \stepcounter{utterance}  

    & & \multicolumn{4}{p{0.6\linewidth}}{\cellcolor[rgb]{0.9,0.9,0.9}{%
	\makecell[{{p{\linewidth}}}]{% 
	  \tt {\tiny [GM$\rangle$B]}  
	 I need your assistance with a word game in which we need to find a 5-letter word using a clue, a guess and an explanation for the guess.\\ \tt \\ \tt For example, if the clue is "snow white" and the guess is "apple", you will get the information like this:\\ \tt clue:snow white\\ \tt guess:apple\\ \tt explanation:the fairy tale Snow White involves the girl eating a poisoned apple, which has 5 letters.\\ \tt \\ \tt Your task is to either agree or disagree with my guess based on the given clue.\\ \tt \\ \tt If you disagree, I will make another guess, and if you agree, I will proceed in the game using the same guess.\\ \tt \\ \tt Initially, guess feedback is not available, but as the game proceeds, it becomes available. The guess feedback is not for current guess, but rather an indication of what letters can be present in the current guess. A letter highlighted in green indicates that it is a correct letter in the correct position, while yellow indicates that it is a correct letter in the wrong position, and red indicates a wrong letter. At this point, you should use all the provided information, including the clue, guess, explanation, and guess feedback, to decide whether the given guess matches the clue, and also stick to guess feedback or not.\\ \tt \\ \tt Please respond in lowercase letters and stick to this format:\\ \tt agreement:yes or no\\ \tt explanation:your reason for doing so\\ \tt \\ \tt Please note that you must not guess any word based on the riddle clue and stick to the given format while responding.\\ \tt \\ \tt Let's begin!\\ \tt \\ \tt clue:square\\ \tt guess:block\\ \tt explanation:A block can refer to a square shape, such as in building blocks, which are often square. The word "block" has five letters, fitting the criteria.\\ \tt  
	  } 
	   } 
	   } 
	 & \\ 
 

    \theutterance \stepcounter{utterance}  

    & & & \multicolumn{4}{p{0.6\linewidth}}{\cellcolor[rgb]{1,0.85,0.72}{%
	\makecell[{{p{\linewidth}}}]{% 
	  \tt {\tiny [GM$\langle$B]}  
	 agreement:yes\\ \tt explanation:The explanation logically connects the clue "square" with the guess "block," as blocks can indeed be square-shaped, and the word "block" meets the requirement of being a 5-letter word. 
	  } 
	   } 
	   } 
	  \\ 
 

    \theutterance \stepcounter{utterance}  

    & & & \multicolumn{2}{p{0.3\linewidth}}{\cellcolor[rgb]{0.95,0.95,0.95}{%
	\makecell[{{p{\linewidth}}}]{% 
	  \tt {\tiny [GM$|$GM]}  
	 {'agreement': 'yes', 'explanation': 'The explanation logically connects the clue "square" with the guess "block," as blocks can indeed be square-shaped, and the word "block" meets the requirement of being a 5-letter word.'} 
	  } 
	   } 
	   } 
	 & & \\ 
 

    \theutterance \stepcounter{utterance}  

    & & & \multicolumn{2}{p{0.3\linewidth}}{\cellcolor[rgb]{0.95,0.95,0.95}{%
	\makecell[{{p{\linewidth}}}]{% 
	  \tt {\tiny [GM$|$GM]}  
	 Critic agrees with the Guesser -- Sharing the critic's explanation with the guesser 
	  } 
	   } 
	   } 
	 & & \\ 
 

    \theutterance \stepcounter{utterance}  

    & & \multicolumn{4}{p{0.6\linewidth}}{\cellcolor[rgb]{0.9,0.9,0.9}{%
	\makecell[{{p{\linewidth}}}]{% 
	  \tt {\tiny [A$\langle$GM]}  
	 clue:square\\ \tt guess_agreement:yes\\ \tt agreement_explanation:The explanation logically connects the clue "square" with the guess "block," as blocks can indeed be square-shaped, and the word "block" meets the requirement of being a 5-letter word.\\ \tt  
	  } 
	   } 
	   } 
	 & \\ 
 

    \theutterance \stepcounter{utterance}  

    & \multicolumn{4}{p{0.6\linewidth}}{\cellcolor[rgb]{0.8,1,0.9}{%
	\makecell[{{p{\linewidth}}}]{% 
	  \tt {\tiny [A$\rangle$GM]}  
	 guess: chess\\ \tt explanation: A chessboard is made up of squares, and the word "chess" has five letters, fitting the criteria. 
	  } 
	   } 
	   } 
	 & & \\ 
 

    \theutterance \stepcounter{utterance}  

    & & & \multicolumn{2}{p{0.3\linewidth}}{\cellcolor[rgb]{0.95,0.95,0.95}{%
	\makecell[{{p{\linewidth}}}]{% 
	  \tt {\tiny [GM$|$GM]}  
	 {'guess': 'chess', 'explanation': 'A chessboard is made up of squares, and the word "chess" has five letters, fitting the criteria.'} 
	  } 
	   } 
	   } 
	 & & \\ 
 

    \theutterance \stepcounter{utterance}  

    & & & \multicolumn{2}{p{0.3\linewidth}}{\cellcolor[rgb]{0.95,0.95,0.95}{%
	\makecell[{{p{\linewidth}}}]{% 
	  \tt {\tiny [GM$|$GM]}  
	 Change in player1's guess\\ \tt guess_before_critic_opinion: block\\ \tt                                                 critic_agreement: yes\\ \tt guess_after_critic_opinion: chess\\ \tt                                                 Proceeding with guess validation 
	  } 
	   } 
	   } 
	 & & \\ 
 

    \theutterance \stepcounter{utterance}  

    & & & \multicolumn{2}{p{0.3\linewidth}}{\cellcolor[rgb]{0.95,0.95,0.95}{%
	\makecell[{{p{\linewidth}}}]{% 
	  \tt {\tiny [GM$|$GM]}  
	 attempts: 1\\ \tt target_word = nerdy\\ \tt target_word_clue = square\\ \tt guess: chess\\ \tt guess_feedback: c<red> h<red> e<yellow> s<red> s<red> 
	  } 
	   } 
	   } 
	 & & \\ 
 

    \theutterance \stepcounter{utterance}  

    & & \multicolumn{4}{p{0.6\linewidth}}{\cellcolor[rgb]{0.9,0.9,0.9}{%
	\makecell[{{p{\linewidth}}}]{% 
	  \tt {\tiny [A$\langle$GM]}  
	 guess_feedback: c<red> h<red> e<yellow> s<red> s<red> 
	  } 
	   } 
	   } 
	 & \\ 
 

    \theutterance \stepcounter{utterance}  

    & \multicolumn{4}{p{0.6\linewidth}}{\cellcolor[rgb]{0.8,1,0.9}{%
	\makecell[{{p{\linewidth}}}]{% 
	  \tt {\tiny [A$\rangle$GM]}  
	 guess:bevel\\ \tt explanation:Given the feedback, we know that 'e' is a letter in the word but in the wrong position. "Bevel" is chosen because it relates to angles and edges, which can be associated with squares, especially in terms of geometry or carpentry. It also repositions 'e' and introduces entirely new letters to test, adhering to the five-letter requirement. 
	  } 
	   } 
	   } 
	 & & \\ 
 

    \theutterance \stepcounter{utterance}  

    & & & \multicolumn{2}{p{0.3\linewidth}}{\cellcolor[rgb]{0.95,0.95,0.95}{%
	\makecell[{{p{\linewidth}}}]{% 
	  \tt {\tiny [GM$|$GM]}  
	 {'guess': 'bevel', 'explanation': 'Given the feedback, we know that \'e\' is a letter in the word but in the wrong position. "Bevel" is chosen because it relates to angles and edges, which can be associated with squares, especially in terms of geometry or carpentry. It also repositions \'e\' and introduces entirely new letters to test, adhering to the five-letter requirement.'} 
	  } 
	   } 
	   } 
	 & & \\ 
 

    \theutterance \stepcounter{utterance}  

    & & \multicolumn{4}{p{0.6\linewidth}}{\cellcolor[rgb]{0.9,0.9,0.9}{%
	\makecell[{{p{\linewidth}}}]{% 
	  \tt {\tiny [GM$\rangle$B]}  
	 clue:square\\ \tt guess:bevel\\ \tt explanation:Given the feedback, we know that 'e' is a letter in the word but in the wrong position. "Bevel" is chosen because it relates to angles and edges, which can be associated with squares, especially in terms of geometry or carpentry. It also repositions 'e' and introduces entirely new letters to test, adhering to the five-letter requirement.\\ \tt guess_feedback: c<red> h<red> e<yellow> s<red> s<red> 
	  } 
	   } 
	   } 
	 & \\ 
 

    \theutterance \stepcounter{utterance}  

    & & & \multicolumn{4}{p{0.6\linewidth}}{\cellcolor[rgb]{1,0.85,0.72}{%
	\makecell[{{p{\linewidth}}}]{% 
	  \tt {\tiny [GM$\langle$B]}  
	 agreement:no\\ \tt explanation:Based on the feedback, 'e' is a correct letter but in the wrong position in the word related to "square". However, "bevel" places 'e' in the same position as the initial guess "block", which contradicts the feedback indicating 'e' should be in a different position. 
	  } 
	   } 
	   } 
	  \\ 
 

    \theutterance \stepcounter{utterance}  

    & & & \multicolumn{2}{p{0.3\linewidth}}{\cellcolor[rgb]{0.95,0.95,0.95}{%
	\makecell[{{p{\linewidth}}}]{% 
	  \tt {\tiny [GM$|$GM]}  
	 {'agreement': 'no', 'explanation': 'Based on the feedback, \'e\' is a correct letter but in the wrong position in the word related to "square". However, "bevel" places \'e\' in the same position as the initial guess "block", which contradicts the feedback indicating \'e\' should be in a different position.'} 
	  } 
	   } 
	   } 
	 & & \\ 
 

    \theutterance \stepcounter{utterance}  

    & & & \multicolumn{2}{p{0.3\linewidth}}{\cellcolor[rgb]{0.95,0.95,0.95}{%
	\makecell[{{p{\linewidth}}}]{% 
	  \tt {\tiny [GM$|$GM]}  
	 Critic disagrees with the Guesser -- Sharing the critic's explanation with the guesser 
	  } 
	   } 
	   } 
	 & & \\ 
 

    \theutterance \stepcounter{utterance}  

    & & \multicolumn{4}{p{0.6\linewidth}}{\cellcolor[rgb]{0.9,0.9,0.9}{%
	\makecell[{{p{\linewidth}}}]{% 
	  \tt {\tiny [A$\langle$GM]}  
	 clue:square\\ \tt guess_agreement:no\\ \tt agreement_explanation:Based on the feedback, 'e' is a correct letter but in the wrong position in the word related to "square". However, "bevel" places 'e' in the same position as the initial guess "block", which contradicts the feedback indicating 'e' should be in a different position.\\ \tt  
	  } 
	   } 
	   } 
	 & \\ 
 

    \theutterance \stepcounter{utterance}  

    & \multicolumn{4}{p{0.6\linewidth}}{\cellcolor[rgb]{0.8,1,0.9}{%
	\makecell[{{p{\linewidth}}}]{% 
	  \tt {\tiny [A$\rangle$GM]}  
	 guess:ledge\\ \tt explanation:Acknowledging the previous oversight, 'e' is known to be in the word but not in the second position. "Ledge" is chosen because it introduces a new position for 'e' at the end of the word and includes entirely new letters to test against the clue "square", potentially connecting to the idea of edges or perimeters which are components of a square. 
	  } 
	   } 
	   } 
	 & & \\ 
 

    \theutterance \stepcounter{utterance}  

    & & & \multicolumn{2}{p{0.3\linewidth}}{\cellcolor[rgb]{0.95,0.95,0.95}{%
	\makecell[{{p{\linewidth}}}]{% 
	  \tt {\tiny [GM$|$GM]}  
	 {'guess': 'ledge', 'explanation': 'Acknowledging the previous oversight, \'e\' is known to be in the word but not in the second position. "Ledge" is chosen because it introduces a new position for \'e\' at the end of the word and includes entirely new letters to test against the clue "square", potentially connecting to the idea of edges or perimeters which are components of a square.'} 
	  } 
	   } 
	   } 
	 & & \\ 
 

    \theutterance \stepcounter{utterance}  

    & & & \multicolumn{2}{p{0.3\linewidth}}{\cellcolor[rgb]{0.95,0.95,0.95}{%
	\makecell[{{p{\linewidth}}}]{% 
	  \tt {\tiny [GM$|$GM]}  
	 Change in player1's guess\\ \tt guess_before_critic_opinion: bevel\\ \tt                                                 critic_agreement: no\\ \tt guess_after_critic_opinion: ledge\\ \tt                                                 Proceeding with guess validation 
	  } 
	   } 
	   } 
	 & & \\ 
 

    \theutterance \stepcounter{utterance}  

    & & & \multicolumn{2}{p{0.3\linewidth}}{\cellcolor[rgb]{0.95,0.95,0.95}{%
	\makecell[{{p{\linewidth}}}]{% 
	  \tt {\tiny [GM$|$GM]}  
	 attempts: 2\\ \tt target_word = nerdy\\ \tt target_word_clue = square\\ \tt guess: ledge\\ \tt guess_feedback: l<red> e<green> d<yellow> g<red> e<red> 
	  } 
	   } 
	   } 
	 & & \\ 
 

    \theutterance \stepcounter{utterance}  

    & & \multicolumn{4}{p{0.6\linewidth}}{\cellcolor[rgb]{0.9,0.9,0.9}{%
	\makecell[{{p{\linewidth}}}]{% 
	  \tt {\tiny [A$\langle$GM]}  
	 guess_feedback: l<red> e<green> d<yellow> g<red> e<red> 
	  } 
	   } 
	   } 
	 & \\ 
 

    \theutterance \stepcounter{utterance}  

    & \multicolumn{4}{p{0.6\linewidth}}{\cellcolor[rgb]{0.8,1,0.9}{%
	\makecell[{{p{\linewidth}}}]{% 
	  \tt {\tiny [A$\rangle$GM]}  
	 guess:debut\\ \tt explanation:Given the feedback, we know 'e' is correctly placed in the second position, and 'd' is in the word but in the wrong position. "Debut" is chosen because it repositions 'd' and introduces new letters to test, while still fitting the five-letter requirement. This guess does not directly relate to "square" but aims to identify the correct letter positions and fulfill the puzzle's requirements. 
	  } 
	   } 
	   } 
	 & & \\ 
 

    \theutterance \stepcounter{utterance}  

    & & & \multicolumn{2}{p{0.3\linewidth}}{\cellcolor[rgb]{0.95,0.95,0.95}{%
	\makecell[{{p{\linewidth}}}]{% 
	  \tt {\tiny [GM$|$GM]}  
	 {'guess': 'debut', 'explanation': 'Given the feedback, we know \'e\' is correctly placed in the second position, and \'d\' is in the word but in the wrong position. "Debut" is chosen because it repositions \'d\' and introduces new letters to test, while still fitting the five-letter requirement. This guess does not directly relate to "square" but aims to identify the correct letter positions and fulfill the puzzle\'s requirements.'} 
	  } 
	   } 
	   } 
	 & & \\ 
 

    \theutterance \stepcounter{utterance}  

    & & \multicolumn{4}{p{0.6\linewidth}}{\cellcolor[rgb]{0.9,0.9,0.9}{%
	\makecell[{{p{\linewidth}}}]{% 
	  \tt {\tiny [GM$\rangle$B]}  
	 clue:square\\ \tt guess:debut\\ \tt explanation:Given the feedback, we know 'e' is correctly placed in the second position, and 'd' is in the word but in the wrong position. "Debut" is chosen because it repositions 'd' and introduces new letters to test, while still fitting the five-letter requirement. This guess does not directly relate to "square" but aims to identify the correct letter positions and fulfill the puzzle's requirements.\\ \tt guess_feedback: l<red> e<green> d<yellow> g<red> e<red> 
	  } 
	   } 
	   } 
	 & \\ 
 

    \theutterance \stepcounter{utterance}  

    & & & \multicolumn{4}{p{0.6\linewidth}}{\cellcolor[rgb]{1,0.85,0.72}{%
	\makecell[{{p{\linewidth}}}]{% 
	  \tt {\tiny [GM$\langle$B]}  
	 agreement:no\\ \tt explanation:the feedback indicates that 'e' is correctly placed in the second position and 'd' is in the word but in the wrong position. However, "debut" places 'd' at the beginning, which contradicts the feedback that 'd' should be in a different position other than the first. 
	  } 
	   } 
	   } 
	  \\ 
 

    \theutterance \stepcounter{utterance}  

    & & & \multicolumn{2}{p{0.3\linewidth}}{\cellcolor[rgb]{0.95,0.95,0.95}{%
	\makecell[{{p{\linewidth}}}]{% 
	  \tt {\tiny [GM$|$GM]}  
	 {'agreement': 'no', 'explanation': 'the feedback indicates that \'e\' is correctly placed in the second position and \'d\' is in the word but in the wrong position. However, "debut" places \'d\' at the beginning, which contradicts the feedback that \'d\' should be in a different position other than the first.'} 
	  } 
	   } 
	   } 
	 & & \\ 
 

    \theutterance \stepcounter{utterance}  

    & & & \multicolumn{2}{p{0.3\linewidth}}{\cellcolor[rgb]{0.95,0.95,0.95}{%
	\makecell[{{p{\linewidth}}}]{% 
	  \tt {\tiny [GM$|$GM]}  
	 Critic disagrees with the Guesser -- Sharing the critic's explanation with the guesser 
	  } 
	   } 
	   } 
	 & & \\ 
 

    \theutterance \stepcounter{utterance}  

    & & \multicolumn{4}{p{0.6\linewidth}}{\cellcolor[rgb]{0.9,0.9,0.9}{%
	\makecell[{{p{\linewidth}}}]{% 
	  \tt {\tiny [A$\langle$GM]}  
	 clue:square\\ \tt guess_agreement:no\\ \tt agreement_explanation:the feedback indicates that 'e' is correctly placed in the second position and 'd' is in the word but in the wrong position. However, "debut" places 'd' at the beginning, which contradicts the feedback that 'd' should be in a different position other than the first.\\ \tt  
	  } 
	   } 
	   } 
	 & \\ 
 

    \theutterance \stepcounter{utterance}  

    & \multicolumn{4}{p{0.6\linewidth}}{\cellcolor[rgb]{0.8,1,0.9}{%
	\makecell[{{p{\linewidth}}}]{% 
	  \tt {\tiny [A$\rangle$GM]}  
	 guess:cedar\\ \tt explanation:Correcting the previous strategy, we know 'e' is correctly placed in the second position, and 'd' is in the word but not in the correct position. "Cedar" is chosen to adhere to these constraints, moving 'd' to a new position while also testing new letters 'c', 'a', and 'r' for any potential matches with the clue "square", despite "cedar" not having a direct connection to the concept 
	  } 
	   } 
	   } 
	 & & \\ 
 

    \theutterance \stepcounter{utterance}  

    & & & \multicolumn{2}{p{0.3\linewidth}}{\cellcolor[rgb]{0.95,0.95,0.95}{%
	\makecell[{{p{\linewidth}}}]{% 
	  \tt {\tiny [GM$|$GM]}  
	 {'guess': 'cedar', 'explanation': 'Correcting the previous strategy, we know \'e\' is correctly placed in the second position, and \'d\' is in the word but not in the correct position. "Cedar" is chosen to adhere to these constraints, moving \'d\' to a new position while also testing new letters \'c\', \'a\', and \'r\' for any potential matches with the clue "square", despite "cedar" not having a direct connection to the concept'} 
	  } 
	   } 
	   } 
	 & & \\ 
 

    \theutterance \stepcounter{utterance}  

    & & & \multicolumn{2}{p{0.3\linewidth}}{\cellcolor[rgb]{0.95,0.95,0.95}{%
	\makecell[{{p{\linewidth}}}]{% 
	  \tt {\tiny [GM$|$GM]}  
	 Change in player1's guess\\ \tt guess_before_critic_opinion: debut\\ \tt                                                 critic_agreement: no\\ \tt guess_after_critic_opinion: cedar\\ \tt                                                 Proceeding with guess validation 
	  } 
	   } 
	   } 
	 & & \\ 
 

    \theutterance \stepcounter{utterance}  

    & & & \multicolumn{2}{p{0.3\linewidth}}{\cellcolor[rgb]{0.95,0.95,0.95}{%
	\makecell[{{p{\linewidth}}}]{% 
	  \tt {\tiny [GM$|$GM]}  
	 attempts: 3\\ \tt target_word = nerdy\\ \tt target_word_clue = square\\ \tt guess: cedar\\ \tt guess_feedback: c<red> e<green> d<yellow> a<red> r<yellow> 
	  } 
	   } 
	   } 
	 & & \\ 
 

    \theutterance \stepcounter{utterance}  

    & & \multicolumn{4}{p{0.6\linewidth}}{\cellcolor[rgb]{0.9,0.9,0.9}{%
	\makecell[{{p{\linewidth}}}]{% 
	  \tt {\tiny [A$\langle$GM]}  
	 guess_feedback: c<red> e<green> d<yellow> a<red> r<yellow> 
	  } 
	   } 
	   } 
	 & \\ 
 

    \theutterance \stepcounter{utterance}  

    & \multicolumn{4}{p{0.6\linewidth}}{\cellcolor[rgb]{0.8,1,0.9}{%
	\makecell[{{p{\linewidth}}}]{% 
	  \tt {\tiny [A$\rangle$GM]}  
	 guess:reedy\\ \tt explanation:Based on the feedback, 'e' is correctly placed in the second position, 'd' and 'r' are in the word but in the wrong positions. "Reedy" is chosen to fit 'e' in the correct spot, and repositions 'd' and 'r' to test new placements. This guess continues to explore the correct arrangement of the known letters while introducing 'y' as a new letter. 
	  } 
	   } 
	   } 
	 & & \\ 
 

    \theutterance \stepcounter{utterance}  

    & & & \multicolumn{2}{p{0.3\linewidth}}{\cellcolor[rgb]{0.95,0.95,0.95}{%
	\makecell[{{p{\linewidth}}}]{% 
	  \tt {\tiny [GM$|$GM]}  
	 {'guess': 'reedy', 'explanation': 'Based on the feedback, \'e\' is correctly placed in the second position, \'d\' and \'r\' are in the word but in the wrong positions. "Reedy" is chosen to fit \'e\' in the correct spot, and repositions \'d\' and \'r\' to test new placements. This guess continues to explore the correct arrangement of the known letters while introducing \'y\' as a new letter.'} 
	  } 
	   } 
	   } 
	 & & \\ 
 

    \theutterance \stepcounter{utterance}  

    & & \multicolumn{4}{p{0.6\linewidth}}{\cellcolor[rgb]{0.9,0.9,0.9}{%
	\makecell[{{p{\linewidth}}}]{% 
	  \tt {\tiny [GM$\rangle$B]}  
	 clue:square\\ \tt guess:reedy\\ \tt explanation:Based on the feedback, 'e' is correctly placed in the second position, 'd' and 'r' are in the word but in the wrong positions. "Reedy" is chosen to fit 'e' in the correct spot, and repositions 'd' and 'r' to test new placements. This guess continues to explore the correct arrangement of the known letters while introducing 'y' as a new letter.\\ \tt guess_feedback: c<red> e<green> d<yellow> a<red> r<yellow> 
	  } 
	   } 
	   } 
	 & \\ 
 

    \theutterance \stepcounter{utterance}  

    & & & \multicolumn{4}{p{0.6\linewidth}}{\cellcolor[rgb]{1,0.85,0.72}{%
	\makecell[{{p{\linewidth}}}]{% 
	  \tt {\tiny [GM$\langle$B]}  
	 agreement:no\\ \tt explanation:the feedback indicates that 'd' and 'r' are in the word but in the wrong positions, and 'e' is correctly placed in the second position. However, the guess "reedy" places 'd' and 'r' in positions that were previously indicated as incorrect by the feedback ('d' was tried in the third position and 'r' in the first position before). Therefore, this guess does not correctly follow the feedback provided. 
	  } 
	   } 
	   } 
	  \\ 
 

    \theutterance \stepcounter{utterance}  

    & & & \multicolumn{2}{p{0.3\linewidth}}{\cellcolor[rgb]{0.95,0.95,0.95}{%
	\makecell[{{p{\linewidth}}}]{% 
	  \tt {\tiny [GM$|$GM]}  
	 {'agreement': 'no', 'explanation': 'the feedback indicates that \'d\' and \'r\' are in the word but in the wrong positions, and \'e\' is correctly placed in the second position. However, the guess "reedy" places \'d\' and \'r\' in positions that were previously indicated as incorrect by the feedback (\'d\' was tried in the third position and \'r\' in the first position before). Therefore, this guess does not correctly follow the feedback provided.'} 
	  } 
	   } 
	   } 
	 & & \\ 
 

    \theutterance \stepcounter{utterance}  

    & & & \multicolumn{2}{p{0.3\linewidth}}{\cellcolor[rgb]{0.95,0.95,0.95}{%
	\makecell[{{p{\linewidth}}}]{% 
	  \tt {\tiny [GM$|$GM]}  
	 Critic disagrees with the Guesser -- Sharing the critic's explanation with the guesser 
	  } 
	   } 
	   } 
	 & & \\ 
 

    \theutterance \stepcounter{utterance}  

    & & \multicolumn{4}{p{0.6\linewidth}}{\cellcolor[rgb]{0.9,0.9,0.9}{%
	\makecell[{{p{\linewidth}}}]{% 
	  \tt {\tiny [A$\langle$GM]}  
	 clue:square\\ \tt guess_agreement:no\\ \tt agreement_explanation:the feedback indicates that 'd' and 'r' are in the word but in the wrong positions, and 'e' is correctly placed in the second position. However, the guess "reedy" places 'd' and 'r' in positions that were previously indicated as incorrect by the feedback ('d' was tried in the third position and 'r' in the first position before). Therefore, this guess does not correctly follow the feedback provided.\\ \tt  
	  } 
	   } 
	   } 
	 & \\ 
 

    \theutterance \stepcounter{utterance}  

    & \multicolumn{4}{p{0.6\linewidth}}{\cellcolor[rgb]{0.8,1,0.9}{%
	\makecell[{{p{\linewidth}}}]{% 
	  \tt {\tiny [A$\rangle$GM]}  
	 guess:elder\\ \tt explanation:Correcting the oversight in the previous guess, we know 'e' is correctly placed in the second position, 'd' is in the word but not in its previously guessed positions, and 'r' is also in the word but needs to be repositioned. "Elder" fits these criteria by placing 'e' in the correct spot, moving 'd' to a new position that hasn't been tried based on the feedback, and placing 'r 
	  } 
	   } 
	   } 
	 & & \\ 
 

    \theutterance \stepcounter{utterance}  

    & & & \multicolumn{2}{p{0.3\linewidth}}{\cellcolor[rgb]{0.95,0.95,0.95}{%
	\makecell[{{p{\linewidth}}}]{% 
	  \tt {\tiny [GM$|$GM]}  
	 {'guess': 'elder', 'explanation': 'Correcting the oversight in the previous guess, we know \'e\' is correctly placed in the second position, \'d\' is in the word but not in its previously guessed positions, and \'r\' is also in the word but needs to be repositioned. "Elder" fits these criteria by placing \'e\' in the correct spot, moving \'d\' to a new position that hasn\'t been tried based on the feedback, and placing \'r'} 
	  } 
	   } 
	   } 
	 & & \\ 
 

    \theutterance \stepcounter{utterance}  

    & & & \multicolumn{2}{p{0.3\linewidth}}{\cellcolor[rgb]{0.95,0.95,0.95}{%
	\makecell[{{p{\linewidth}}}]{% 
	  \tt {\tiny [GM$|$GM]}  
	 Change in player1's guess\\ \tt guess_before_critic_opinion: reedy\\ \tt                                                 critic_agreement: no\\ \tt guess_after_critic_opinion: elder\\ \tt                                                 Proceeding with guess validation 
	  } 
	   } 
	   } 
	 & & \\ 
 

    \theutterance \stepcounter{utterance}  

    & & & \multicolumn{2}{p{0.3\linewidth}}{\cellcolor[rgb]{0.95,0.95,0.95}{%
	\makecell[{{p{\linewidth}}}]{% 
	  \tt {\tiny [GM$|$GM]}  
	 attempts: 4\\ \tt target_word = nerdy\\ \tt target_word_clue = square\\ \tt guess: elder\\ \tt guess_feedback: e<yellow> l<red> d<yellow> e<red> r<yellow> 
	  } 
	   } 
	   } 
	 & & \\ 
 

    \theutterance \stepcounter{utterance}  

    & & \multicolumn{4}{p{0.6\linewidth}}{\cellcolor[rgb]{0.9,0.9,0.9}{%
	\makecell[{{p{\linewidth}}}]{% 
	  \tt {\tiny [A$\langle$GM]}  
	 guess_feedback: e<yellow> l<red> d<yellow> e<red> r<yellow> 
	  } 
	   } 
	   } 
	 & \\ 
 

    \theutterance \stepcounter{utterance}  

    & \multicolumn{4}{p{0.6\linewidth}}{\cellcolor[rgb]{0.8,1,0.9}{%
	\makecell[{{p{\linewidth}}}]{% 
	  \tt {\tiny [A$\rangle$GM]}  
	 guess:drake\\ \tt explanation:Given the feedback, 'e' is not in the correct position when placed second, but it is in the word. 'd', and 'r' are in the word but were in the wrong positions in "elder". "Drake" repositions 'd' and 'r' to the beginning, places 'e' at the end, which aligns with the feedback, and introduces 'a' and 'k' to potentially match the remaining letters 
	  } 
	   } 
	   } 
	 & & \\ 
 

    \theutterance \stepcounter{utterance}  

    & & & \multicolumn{2}{p{0.3\linewidth}}{\cellcolor[rgb]{0.95,0.95,0.95}{%
	\makecell[{{p{\linewidth}}}]{% 
	  \tt {\tiny [GM$|$GM]}  
	 {'guess': 'drake', 'explanation': 'Given the feedback, \'e\' is not in the correct position when placed second, but it is in the word. \'d\', and \'r\' are in the word but were in the wrong positions in "elder". "Drake" repositions \'d\' and \'r\' to the beginning, places \'e\' at the end, which aligns with the feedback, and introduces \'a\' and \'k\' to potentially match the remaining letters'} 
	  } 
	   } 
	   } 
	 & & \\ 
 

    \theutterance \stepcounter{utterance}  

    & & \multicolumn{4}{p{0.6\linewidth}}{\cellcolor[rgb]{0.9,0.9,0.9}{%
	\makecell[{{p{\linewidth}}}]{% 
	  \tt {\tiny [GM$\rangle$B]}  
	 clue:square\\ \tt guess:drake\\ \tt explanation:Given the feedback, 'e' is not in the correct position when placed second, but it is in the word. 'd', and 'r' are in the word but were in the wrong positions in "elder". "Drake" repositions 'd' and 'r' to the beginning, places 'e' at the end, which aligns with the feedback, and introduces 'a' and 'k' to potentially match the remaining letters\\ \tt guess_feedback: e<yellow> l<red> d<yellow> e<red> r<yellow> 
	  } 
	   } 
	   } 
	 & \\ 
 

    \theutterance \stepcounter{utterance}  

    & & & \multicolumn{4}{p{0.6\linewidth}}{\cellcolor[rgb]{1,0.85,0.72}{%
	\makecell[{{p{\linewidth}}}]{% 
	  \tt {\tiny [GM$\langle$B]}  
	 agreement:no\\ \tt explanation:the feedback indicates 'e' should be in the second position, which contradicts the placement in "drake". 
	  } 
	   } 
	   } 
	  \\ 
 

    \theutterance \stepcounter{utterance}  

    & & & \multicolumn{2}{p{0.3\linewidth}}{\cellcolor[rgb]{0.95,0.95,0.95}{%
	\makecell[{{p{\linewidth}}}]{% 
	  \tt {\tiny [GM$|$GM]}  
	 {'agreement': 'no', 'explanation': 'the feedback indicates \'e\' should be in the second position, which contradicts the placement in "drake".'} 
	  } 
	   } 
	   } 
	 & & \\ 
 

    \theutterance \stepcounter{utterance}  

    & & & \multicolumn{2}{p{0.3\linewidth}}{\cellcolor[rgb]{0.95,0.95,0.95}{%
	\makecell[{{p{\linewidth}}}]{% 
	  \tt {\tiny [GM$|$GM]}  
	 Critic disagrees with the Guesser -- Sharing the critic's explanation with the guesser 
	  } 
	   } 
	   } 
	 & & \\ 
 

    \theutterance \stepcounter{utterance}  

    & & \multicolumn{4}{p{0.6\linewidth}}{\cellcolor[rgb]{0.9,0.9,0.9}{%
	\makecell[{{p{\linewidth}}}]{% 
	  \tt {\tiny [A$\langle$GM]}  
	 clue:square\\ \tt guess_agreement:no\\ \tt agreement_explanation:the feedback indicates 'e' should be in the second position, which contradicts the placement in "drake".\\ \tt  
	  } 
	   } 
	   } 
	 & \\ 
 

    \theutterance \stepcounter{utterance}  

    & \multicolumn{4}{p{0.6\linewidth}}{\cellcolor[rgb]{0.8,1,0.9}{%
	\makecell[{{p{\linewidth}}}]{% 
	  \tt {\tiny [A$\rangle$GM]}  
	 guess:dream\\ \tt explanation:Correcting the previous error, we know 'e' should indeed be in the second position as per the feedback. 'd' and 'r' are confirmed to be in the word but their correct positions were not identified in the last guess. "Dream" places 'e' correctly in the second position, repositions 'd' and 'r' based on the feedback that they are in the word but were in the wrong positions, and introduces 'a' 
	  } 
	   } 
	   } 
	 & & \\ 
 

    \theutterance \stepcounter{utterance}  

    & & & \multicolumn{2}{p{0.3\linewidth}}{\cellcolor[rgb]{0.95,0.95,0.95}{%
	\makecell[{{p{\linewidth}}}]{% 
	  \tt {\tiny [GM$|$GM]}  
	 {'guess': 'dream', 'explanation': 'Correcting the previous error, we know \'e\' should indeed be in the second position as per the feedback. \'d\' and \'r\' are confirmed to be in the word but their correct positions were not identified in the last guess. "Dream" places \'e\' correctly in the second position, repositions \'d\' and \'r\' based on the feedback that they are in the word but were in the wrong positions, and introduces \'a\''} 
	  } 
	   } 
	   } 
	 & & \\ 
 

    \theutterance \stepcounter{utterance}  

    & & & \multicolumn{2}{p{0.3\linewidth}}{\cellcolor[rgb]{0.95,0.95,0.95}{%
	\makecell[{{p{\linewidth}}}]{% 
	  \tt {\tiny [GM$|$GM]}  
	 Change in player1's guess\\ \tt guess_before_critic_opinion: drake\\ \tt                                                 critic_agreement: no\\ \tt guess_after_critic_opinion: dream\\ \tt                                                 Proceeding with guess validation 
	  } 
	   } 
	   } 
	 & & \\ 
 

    \theutterance \stepcounter{utterance}  

    & & & \multicolumn{2}{p{0.3\linewidth}}{\cellcolor[rgb]{0.95,0.95,0.95}{%
	\makecell[{{p{\linewidth}}}]{% 
	  \tt {\tiny [GM$|$GM]}  
	 attempts: 5\\ \tt target_word = nerdy\\ \tt target_word_clue = square\\ \tt guess: dream\\ \tt guess_feedback: d<yellow> r<yellow> e<yellow> a<red> m<red> 
	  } 
	   } 
	   } 
	 & & \\ 
 

    \theutterance \stepcounter{utterance}  

    & & \multicolumn{4}{p{0.6\linewidth}}{\cellcolor[rgb]{0.9,0.9,0.9}{%
	\makecell[{{p{\linewidth}}}]{% 
	  \tt {\tiny [A$\langle$GM]}  
	 guess_feedback: d<yellow> r<yellow> e<yellow> a<red> m<red> 
	  } 
	   } 
	   } 
	 & \\ 
 

    \theutterance \stepcounter{utterance}  

    & \multicolumn{4}{p{0.6\linewidth}}{\cellcolor[rgb]{0.8,1,0.9}{%
	\makecell[{{p{\linewidth}}}]{% 
	  \tt {\tiny [A$\rangle$GM]}  
	 guess:reeds\\ \tt explanation:Given the feedback, 'd', 'r', and 'e' are in the word but not in the positions tried in "dream". Since 'a' and 'm' are incorrect, and we know 'e' fits well in the second position from previous feedback, "reeds" repositions 'd' and 'r' while keeping 'e' in a spot we are confident about. This guess introduces 's' as a new letter, 
	  } 
	   } 
	   } 
	 & & \\ 
 

    \theutterance \stepcounter{utterance}  

    & & & \multicolumn{2}{p{0.3\linewidth}}{\cellcolor[rgb]{0.95,0.95,0.95}{%
	\makecell[{{p{\linewidth}}}]{% 
	  \tt {\tiny [GM$|$GM]}  
	 {'guess': 'reeds', 'explanation': 'Given the feedback, \'d\', \'r\', and \'e\' are in the word but not in the positions tried in "dream". Since \'a\' and \'m\' are incorrect, and we know \'e\' fits well in the second position from previous feedback, "reeds" repositions \'d\' and \'r\' while keeping \'e\' in a spot we are confident about. This guess introduces \'s\' as a new letter,'} 
	  } 
	   } 
	   } 
	 & & \\ 
 

    \theutterance \stepcounter{utterance}  

    & & \multicolumn{4}{p{0.6\linewidth}}{\cellcolor[rgb]{0.9,0.9,0.9}{%
	\makecell[{{p{\linewidth}}}]{% 
	  \tt {\tiny [GM$\rangle$B]}  
	 clue:square\\ \tt guess:reeds\\ \tt explanation:Given the feedback, 'd', 'r', and 'e' are in the word but not in the positions tried in "dream". Since 'a' and 'm' are incorrect, and we know 'e' fits well in the second position from previous feedback, "reeds" repositions 'd' and 'r' while keeping 'e' in a spot we are confident about. This guess introduces 's' as a new letter,\\ \tt guess_feedback: d<yellow> r<yellow> e<yellow> a<red> m<red> 
	  } 
	   } 
	   } 
	 & \\ 
 

    \theutterance \stepcounter{utterance}  

    & & & \multicolumn{4}{p{0.6\linewidth}}{\cellcolor[rgb]{1,0.85,0.72}{%
	\makecell[{{p{\linewidth}}}]{% 
	  \tt {\tiny [GM$\langle$B]}  
	 agreement:no\\ \tt explanation:the feedback indicates that 'd', 'r', and 'e' are in the word but not in their correct positions in the guess "drake". However, the guess "reeds" incorrectly interprets the feedback by placing 'e' in a position previously confirmed as correct, which contradicts the new feedback where 'e' is marked yellow, indicating it is in the wrong position. Additionally, 's' was not previously tested against the feedback. 
	  } 
	   } 
	   } 
	  \\ 
 

    \theutterance \stepcounter{utterance}  

    & & & \multicolumn{2}{p{0.3\linewidth}}{\cellcolor[rgb]{0.95,0.95,0.95}{%
	\makecell[{{p{\linewidth}}}]{% 
	  \tt {\tiny [GM$|$GM]}  
	 {'agreement': 'no', 'explanation': 'the feedback indicates that \'d\', \'r\', and \'e\' are in the word but not in their correct positions in the guess "drake". However, the guess "reeds" incorrectly interprets the feedback by placing \'e\' in a position previously confirmed as correct, which contradicts the new feedback where \'e\' is marked yellow, indicating it is in the wrong position. Additionally, \'s\' was not previously tested against the feedback.'} 
	  } 
	   } 
	   } 
	 & & \\ 
 

    \theutterance \stepcounter{utterance}  

    & & & \multicolumn{2}{p{0.3\linewidth}}{\cellcolor[rgb]{0.95,0.95,0.95}{%
	\makecell[{{p{\linewidth}}}]{% 
	  \tt {\tiny [GM$|$GM]}  
	 Critic disagrees with the Guesser -- Sharing the critic's explanation with the guesser 
	  } 
	   } 
	   } 
	 & & \\ 
 

    \theutterance \stepcounter{utterance}  

    & & \multicolumn{4}{p{0.6\linewidth}}{\cellcolor[rgb]{0.9,0.9,0.9}{%
	\makecell[{{p{\linewidth}}}]{% 
	  \tt {\tiny [A$\langle$GM]}  
	 clue:square\\ \tt guess_agreement:no\\ \tt agreement_explanation:the feedback indicates that 'd', 'r', and 'e' are in the word but not in their correct positions in the guess "drake". However, the guess "reeds" incorrectly interprets the feedback by placing 'e' in a position previously confirmed as correct, which contradicts the new feedback where 'e' is marked yellow, indicating it is in the wrong position. Additionally, 's' was not previously tested against the feedback.\\ \tt  
	  } 
	   } 
	   } 
	 & \\ 
 

    \theutterance \stepcounter{utterance}  

    & \multicolumn{4}{p{0.6\linewidth}}{\cellcolor[rgb]{0.8,1,0.9}{%
	\makecell[{{p{\linewidth}}}]{% 
	  \tt {\tiny [A$\rangle$GM]}  
	 guess:treed\\ \tt explanation:Correcting the misunderstanding from the previous feedback, we now understand that 'd', 'r', and 'e' are in the word but not in the positions they were placed in "dream". Given the new insight that 'e' should not be in the second position as previously thought, "treed" rearranges these letters, placing 'r' at the beginning and 'd' at the end, while also adjusting the position of 'e's to 
	  } 
	   } 
	   } 
	 & & \\ 
 

    \theutterance \stepcounter{utterance}  

    & & & \multicolumn{2}{p{0.3\linewidth}}{\cellcolor[rgb]{0.95,0.95,0.95}{%
	\makecell[{{p{\linewidth}}}]{% 
	  \tt {\tiny [GM$|$GM]}  
	 {'guess': 'treed', 'explanation': 'Correcting the misunderstanding from the previous feedback, we now understand that \'d\', \'r\', and \'e\' are in the word but not in the positions they were placed in "dream". Given the new insight that \'e\' should not be in the second position as previously thought, "treed" rearranges these letters, placing \'r\' at the beginning and \'d\' at the end, while also adjusting the position of \'e\'s to'} 
	  } 
	   } 
	   } 
	 & & \\ 
 

    \theutterance \stepcounter{utterance}  

    & & & \multicolumn{2}{p{0.3\linewidth}}{\cellcolor[rgb]{0.95,0.95,0.95}{%
	\makecell[{{p{\linewidth}}}]{% 
	  \tt {\tiny [GM$|$GM]}  
	 Change in player1's guess\\ \tt guess_before_critic_opinion: reeds\\ \tt                                                 critic_agreement: no\\ \tt guess_after_critic_opinion: treed\\ \tt                                                 Proceeding with guess validation 
	  } 
	   } 
	   } 
	 & & \\ 
 

    \theutterance \stepcounter{utterance}  

    & & & \multicolumn{2}{p{0.3\linewidth}}{\cellcolor[rgb]{0.95,0.95,0.95}{%
	\makecell[{{p{\linewidth}}}]{% 
	  \tt {\tiny [GM$|$GM]}  
	 attempts: 6\\ \tt target_word = nerdy\\ \tt target_word_clue = square\\ \tt guess: treed\\ \tt guess_feedback: t<red> r<yellow> e<yellow> e<red> d<yellow> 
	  } 
	   } 
	   } 
	 & & \\ 
 

    \theutterance \stepcounter{utterance}  

    & & & \multicolumn{2}{p{0.3\linewidth}}{\cellcolor[rgb]{0.95,0.95,0.95}{%
	\makecell[{{p{\linewidth}}}]{% 
	  \tt {\tiny [GM$|$GM]}  
	 game_result = LOSS 
	  } 
	   } 
	   } 
	 & & \\ 
 

\end{supertabular}
}

\end{document}


\documentclass{article}
\usepackage{colortbl}
\usepackage{makecell}
\usepackage{multirow}
\usepackage{supertabular}

\begin{document}

\newcounter{utterance}

\twocolumn

{ \footnotesize  \setcounter{utterance}{1}
\setlength{\tabcolsep}{0pt}
\begin{supertabular}{c@{$\;$}|p{.15\linewidth}@{}p{.15\linewidth}p{.15\linewidth}p{.15\linewidth}p{.15\linewidth}p{.15\linewidth}}

    \# & $\;$A & \multicolumn{4}{c}{Game Master} & $\;\:$B\\
    \hline 

    \theutterance \stepcounter{utterance}  

    & & \multicolumn{4}{p{0.6\linewidth}}{\cellcolor[rgb]{0.9,0.9,0.9}{%
	\makecell[{{p{\linewidth}}}]{% 
	  \tt {\tiny [A$\langle$GM]}  
	 You are a language wizard who likes to guess words by using the given rules.\\ \tt \\ \tt Welcome to Wordle! You have six attempts to guess the target word, a valid English word of five lowercase letters (a-z). Please use the tags "guess:" and "explanation:" to provide a concise explanation for each guess.\\ \tt \\ \tt To help you make an informed guess, you will receive a clue for the word, such as\\ \tt clue: "snowy white"\\ \tt \\ \tt Here is an example guess based on the clue:\\ \tt guess: apple\\ \tt explanation: In the fairy tail Snow White, the girl is killed because she eats a poisoned apple. And the word apple has 5 letters.\\ \tt \\ \tt I will then indicate whether I agree or disagree with your guess and provide rationale, but agreeing with a guess does not confirm its correctness. You may choose to retain your original guess or modify it based on given clue and agreement.\\ \tt \\ \tt After each guess, your answer will be validated, and you will receive feedback indicating which letters are correct (green), which letters are correct but in the wrong position (yellow), and which letters are incorrect (red). This feedback can be useful in determining which letters to include or exclude in your next guess.\\ \tt \\ \tt For example, the feedback for "apple" might be:\\ \tt guess_feedback: a<yellow> p<yellow> p<green> l<yellow> e<red>\\ \tt \\ \tt Here are some rules to keep in mind:\\ \tt 1. Please do not generate any guess feedback or clue on your own\\ \tt 2. Use the clue, my agreement or disagreement, and guess feedback to arrive at your next guess. This should be reflected in your explanation as well.\\ \tt \\ \tt Let's begin with your first guess.\\ \tt \\ \tt clue:passionate\\ \tt  
	  } 
	   } 
	   } 
	 & \\ 
 

    \theutterance \stepcounter{utterance}  

    & \multicolumn{4}{p{0.6\linewidth}}{\cellcolor[rgb]{0.8,1,0.9}{%
	\makecell[{{p{\linewidth}}}]{% 
	  \tt {\tiny [A$\rangle$GM]}  
	 guess: fiery\\ \tt explanation: The word "fiery" is often used to describe something that is intensely passionate, whether it's a person's temperament, a passionate speech, or a vivid sunset. It fits the clue "passionate" and is exactly five letters long. 
	  } 
	   } 
	   } 
	 & & \\ 
 

    \theutterance \stepcounter{utterance}  

    & & & \multicolumn{2}{p{0.3\linewidth}}{\cellcolor[rgb]{0.95,0.95,0.95}{%
	\makecell[{{p{\linewidth}}}]{% 
	  \tt {\tiny [GM$|$GM]}  
	 {'guess': 'fiery', 'explanation': 'The word "fiery" is often used to describe something that is intensely passionate, whether it\'s a person\'s temperament, a passionate speech, or a vivid sunset. It fits the clue "passionate" and is exactly five letters long.'} 
	  } 
	   } 
	   } 
	 & & \\ 
 

    \theutterance \stepcounter{utterance}  

    & & \multicolumn{4}{p{0.6\linewidth}}{\cellcolor[rgb]{0.9,0.9,0.9}{%
	\makecell[{{p{\linewidth}}}]{% 
	  \tt {\tiny [GM$\rangle$B]}  
	 I need your assistance with a word game in which we need to find a 5-letter word using a clue, a guess and an explanation for the guess.\\ \tt \\ \tt For example, if the clue is "snow white" and the guess is "apple", you will get the information like this:\\ \tt clue:snow white\\ \tt guess:apple\\ \tt explanation:the fairy tale Snow White involves the girl eating a poisoned apple, which has 5 letters.\\ \tt \\ \tt Your task is to either agree or disagree with my guess based on the given clue.\\ \tt \\ \tt If you disagree, I will make another guess, and if you agree, I will proceed in the game using the same guess.\\ \tt \\ \tt Initially, guess feedback is not available, but as the game proceeds, it becomes available. The guess feedback is not for current guess, but rather an indication of what letters can be present in the current guess. A letter highlighted in green indicates that it is a correct letter in the correct position, while yellow indicates that it is a correct letter in the wrong position, and red indicates a wrong letter. At this point, you should use all the provided information, including the clue, guess, explanation, and guess feedback, to decide whether the given guess matches the clue, and also stick to guess feedback or not.\\ \tt \\ \tt Please respond in lowercase letters and stick to this format:\\ \tt agreement:yes or no\\ \tt explanation:your reason for doing so\\ \tt \\ \tt Please note that you must not guess any word based on the riddle clue and stick to the given format while responding.\\ \tt \\ \tt Let's begin!\\ \tt \\ \tt clue:passionate\\ \tt guess:fiery\\ \tt explanation:The word "fiery" is often used to describe something that is intensely passionate, whether it's a person's temperament, a passionate speech, or a vivid sunset. It fits the clue "passionate" and is exactly five letters long.\\ \tt  
	  } 
	   } 
	   } 
	 & \\ 
 

    \theutterance \stepcounter{utterance}  

    & & & \multicolumn{4}{p{0.6\linewidth}}{\cellcolor[rgb]{1,0.85,0.72}{%
	\makecell[{{p{\linewidth}}}]{% 
	  \tt {\tiny [GM$\langle$B]}  
	 agreement:yes\\ \tt explanation:The explanation provided aligns well with the clue "passionate," and "fiery" is indeed a 5-letter word that can describe something intensely passionate. 
	  } 
	   } 
	   } 
	  \\ 
 

    \theutterance \stepcounter{utterance}  

    & & & \multicolumn{2}{p{0.3\linewidth}}{\cellcolor[rgb]{0.95,0.95,0.95}{%
	\makecell[{{p{\linewidth}}}]{% 
	  \tt {\tiny [GM$|$GM]}  
	 {'agreement': 'yes', 'explanation': 'The explanation provided aligns well with the clue "passionate," and "fiery" is indeed a 5-letter word that can describe something intensely passionate.'} 
	  } 
	   } 
	   } 
	 & & \\ 
 

    \theutterance \stepcounter{utterance}  

    & & & \multicolumn{2}{p{0.3\linewidth}}{\cellcolor[rgb]{0.95,0.95,0.95}{%
	\makecell[{{p{\linewidth}}}]{% 
	  \tt {\tiny [GM$|$GM]}  
	 Critic agrees with the Guesser -- Sharing the critic's explanation with the guesser 
	  } 
	   } 
	   } 
	 & & \\ 
 

    \theutterance \stepcounter{utterance}  

    & & \multicolumn{4}{p{0.6\linewidth}}{\cellcolor[rgb]{0.9,0.9,0.9}{%
	\makecell[{{p{\linewidth}}}]{% 
	  \tt {\tiny [A$\langle$GM]}  
	 clue:passionate\\ \tt guess_agreement:yes\\ \tt agreement_explanation:The explanation provided aligns well with the clue "passionate," and "fiery" is indeed a 5-letter word that can describe something intensely passionate.\\ \tt  
	  } 
	   } 
	   } 
	 & \\ 
 

    \theutterance \stepcounter{utterance}  

    & \multicolumn{4}{p{0.6\linewidth}}{\cellcolor[rgb]{0.8,1,0.9}{%
	\makecell[{{p{\linewidth}}}]{% 
	  \tt {\tiny [A$\rangle$GM]}  
	 Since "fiery" was agreed upon as aligning well with the clue "passionate," but without specific feedback on the correctness of the letters, I will make another guess that also embodies passion but might fit a different aspect or use of the word "passionate."\\ \tt \\ \tt guess:ardor\\ \tt explanation:The word "ardor" signifies great enthusiasm or passion, directly aligning with the clue "passionate." It is another attempt to capture the essence of the clue within a five-letter word, 
	  } 
	   } 
	   } 
	 & & \\ 
 

    \theutterance \stepcounter{utterance}  

    & & & \multicolumn{2}{p{0.3\linewidth}}{\cellcolor[rgb]{0.95,0.95,0.95}{%
	\makecell[{{p{\linewidth}}}]{% 
	  \tt {\tiny [GM$|$GM]}  
	 {'guess': 'ardor', 'explanation': 'The word "ardor" signifies great enthusiasm or passion, directly aligning with the clue "passionate." It is another attempt to capture the essence of the clue within a five-letter word,'} 
	  } 
	   } 
	   } 
	 & & \\ 
 

    \theutterance \stepcounter{utterance}  

    & & & \multicolumn{2}{p{0.3\linewidth}}{\cellcolor[rgb]{0.95,0.95,0.95}{%
	\makecell[{{p{\linewidth}}}]{% 
	  \tt {\tiny [GM$|$GM]}  
	 Change in player1's guess\\ \tt guess_before_critic_opinion: fiery\\ \tt                                                 critic_agreement: yes\\ \tt guess_after_critic_opinion: ardor\\ \tt                                                 Proceeding with guess validation 
	  } 
	   } 
	   } 
	 & & \\ 
 

    \theutterance \stepcounter{utterance}  

    & & & \multicolumn{2}{p{0.3\linewidth}}{\cellcolor[rgb]{0.95,0.95,0.95}{%
	\makecell[{{p{\linewidth}}}]{% 
	  \tt {\tiny [GM$|$GM]}  
	 attempts: 1\\ \tt target_word = fiery\\ \tt target_word_clue = passionate\\ \tt guess: ardor\\ \tt guess_feedback: a<red> r<yellow> d<red> o<red> r<red> 
	  } 
	   } 
	   } 
	 & & \\ 
 

    \theutterance \stepcounter{utterance}  

    & & \multicolumn{4}{p{0.6\linewidth}}{\cellcolor[rgb]{0.9,0.9,0.9}{%
	\makecell[{{p{\linewidth}}}]{% 
	  \tt {\tiny [A$\langle$GM]}  
	 guess_feedback: a<red> r<yellow> d<red> o<red> r<red> 
	  } 
	   } 
	   } 
	 & \\ 
 

    \theutterance \stepcounter{utterance}  

    & \multicolumn{4}{p{0.6\linewidth}}{\cellcolor[rgb]{0.8,1,0.9}{%
	\makecell[{{p{\linewidth}}}]{% 
	  \tt {\tiny [A$\rangle$GM]}  
	 guess:brave\\ \tt explanation:Given the feedback, we know that 'r' is in the word but in the wrong position, and 'a', 'd', 'o', and the second 'r' are not in the word. "Brave" is chosen because it can be associated with passionate actions or feelings, fitting the clue "passionate." It also repositions 'r' and introduces new letters to test against the feedback. 
	  } 
	   } 
	   } 
	 & & \\ 
 

    \theutterance \stepcounter{utterance}  

    & & & \multicolumn{2}{p{0.3\linewidth}}{\cellcolor[rgb]{0.95,0.95,0.95}{%
	\makecell[{{p{\linewidth}}}]{% 
	  \tt {\tiny [GM$|$GM]}  
	 {'guess': 'brave', 'explanation': 'Given the feedback, we know that \'r\' is in the word but in the wrong position, and \'a\', \'d\', \'o\', and the second \'r\' are not in the word. "Brave" is chosen because it can be associated with passionate actions or feelings, fitting the clue "passionate." It also repositions \'r\' and introduces new letters to test against the feedback.'} 
	  } 
	   } 
	   } 
	 & & \\ 
 

    \theutterance \stepcounter{utterance}  

    & & \multicolumn{4}{p{0.6\linewidth}}{\cellcolor[rgb]{0.9,0.9,0.9}{%
	\makecell[{{p{\linewidth}}}]{% 
	  \tt {\tiny [GM$\rangle$B]}  
	 clue:passionate\\ \tt guess:brave\\ \tt explanation:Given the feedback, we know that 'r' is in the word but in the wrong position, and 'a', 'd', 'o', and the second 'r' are not in the word. "Brave" is chosen because it can be associated with passionate actions or feelings, fitting the clue "passionate." It also repositions 'r' and introduces new letters to test against the feedback.\\ \tt guess_feedback: a<red> r<yellow> d<red> o<red> r<red> 
	  } 
	   } 
	   } 
	 & \\ 
 

    \theutterance \stepcounter{utterance}  

    & & & \multicolumn{4}{p{0.6\linewidth}}{\cellcolor[rgb]{1,0.85,0.72}{%
	\makecell[{{p{\linewidth}}}]{% 
	  \tt {\tiny [GM$\langle$B]}  
	 agreement:no\\ \tt explanation:the feedback indicates that 'r' is in the word but in the wrong position, and 'a' is not in the word, making "brave" an incorrect guess since it contains 'a'. 
	  } 
	   } 
	   } 
	  \\ 
 

    \theutterance \stepcounter{utterance}  

    & & & \multicolumn{2}{p{0.3\linewidth}}{\cellcolor[rgb]{0.95,0.95,0.95}{%
	\makecell[{{p{\linewidth}}}]{% 
	  \tt {\tiny [GM$|$GM]}  
	 {'agreement': 'no', 'explanation': 'the feedback indicates that \'r\' is in the word but in the wrong position, and \'a\' is not in the word, making "brave" an incorrect guess since it contains \'a\'.'} 
	  } 
	   } 
	   } 
	 & & \\ 
 

    \theutterance \stepcounter{utterance}  

    & & & \multicolumn{2}{p{0.3\linewidth}}{\cellcolor[rgb]{0.95,0.95,0.95}{%
	\makecell[{{p{\linewidth}}}]{% 
	  \tt {\tiny [GM$|$GM]}  
	 Critic disagrees with the Guesser -- Sharing the critic's explanation with the guesser 
	  } 
	   } 
	   } 
	 & & \\ 
 

    \theutterance \stepcounter{utterance}  

    & & \multicolumn{4}{p{0.6\linewidth}}{\cellcolor[rgb]{0.9,0.9,0.9}{%
	\makecell[{{p{\linewidth}}}]{% 
	  \tt {\tiny [A$\langle$GM]}  
	 clue:passionate\\ \tt guess_agreement:no\\ \tt agreement_explanation:the feedback indicates that 'r' is in the word but in the wrong position, and 'a' is not in the word, making "brave" an incorrect guess since it contains 'a'.\\ \tt  
	  } 
	   } 
	   } 
	 & \\ 
 

    \theutterance \stepcounter{utterance}  

    & \multicolumn{4}{p{0.6\linewidth}}{\cellcolor[rgb]{0.8,1,0.9}{%
	\makecell[{{p{\linewidth}}}]{% 
	  \tt {\tiny [A$\rangle$GM]}  
	 guess:crush\\ \tt explanation:Acknowledging the mistake in my previous guess, "crush" is chosen because it fits the theme of passion, often used to describe intense feelings of affection or attraction, which aligns with the clue "passionate." It also places 'r' in a new position, adheres to the feedback by excluding 'a', 'd', 'o', and the second 'r', and introduces entirely new letters to further refine the search for the correct word. 
	  } 
	   } 
	   } 
	 & & \\ 
 

    \theutterance \stepcounter{utterance}  

    & & & \multicolumn{2}{p{0.3\linewidth}}{\cellcolor[rgb]{0.95,0.95,0.95}{%
	\makecell[{{p{\linewidth}}}]{% 
	  \tt {\tiny [GM$|$GM]}  
	 {'guess': 'crush', 'explanation': 'Acknowledging the mistake in my previous guess, "crush" is chosen because it fits the theme of passion, often used to describe intense feelings of affection or attraction, which aligns with the clue "passionate." It also places \'r\' in a new position, adheres to the feedback by excluding \'a\', \'d\', \'o\', and the second \'r\', and introduces entirely new letters to further refine the search for the correct word.'} 
	  } 
	   } 
	   } 
	 & & \\ 
 

    \theutterance \stepcounter{utterance}  

    & & & \multicolumn{2}{p{0.3\linewidth}}{\cellcolor[rgb]{0.95,0.95,0.95}{%
	\makecell[{{p{\linewidth}}}]{% 
	  \tt {\tiny [GM$|$GM]}  
	 Change in player1's guess\\ \tt guess_before_critic_opinion: brave\\ \tt                                                 critic_agreement: no\\ \tt guess_after_critic_opinion: crush\\ \tt                                                 Proceeding with guess validation 
	  } 
	   } 
	   } 
	 & & \\ 
 

    \theutterance \stepcounter{utterance}  

    & & & \multicolumn{2}{p{0.3\linewidth}}{\cellcolor[rgb]{0.95,0.95,0.95}{%
	\makecell[{{p{\linewidth}}}]{% 
	  \tt {\tiny [GM$|$GM]}  
	 attempts: 2\\ \tt target_word = fiery\\ \tt target_word_clue = passionate\\ \tt guess: crush\\ \tt guess_feedback: c<red> r<yellow> u<red> s<red> h<red> 
	  } 
	   } 
	   } 
	 & & \\ 
 

    \theutterance \stepcounter{utterance}  

    & & \multicolumn{4}{p{0.6\linewidth}}{\cellcolor[rgb]{0.9,0.9,0.9}{%
	\makecell[{{p{\linewidth}}}]{% 
	  \tt {\tiny [A$\langle$GM]}  
	 guess_feedback: c<red> r<yellow> u<red> s<red> h<red> 
	  } 
	   } 
	   } 
	 & \\ 
 

    \theutterance \stepcounter{utterance}  

    & \multicolumn{4}{p{0.6\linewidth}}{\cellcolor[rgb]{0.8,1,0.9}{%
	\makecell[{{p{\linewidth}}}]{% 
	  \tt {\tiny [A$\rangle$GM]}  
	 guess:truer\\ \tt explanation:Given the feedback, 'r' is confirmed to be in the word but not in the second or fourth positions as previously guessed. The word "truer" is chosen because it fits the new letter arrangement criteria, placing 'r' in different positions to test its correct placement. Although "truer" might not directly align with the typical use of "passionate," it is selected based on the letter feedback and the challenge of fitting the clue within these constraints 
	  } 
	   } 
	   } 
	 & & \\ 
 

    \theutterance \stepcounter{utterance}  

    & & & \multicolumn{2}{p{0.3\linewidth}}{\cellcolor[rgb]{0.95,0.95,0.95}{%
	\makecell[{{p{\linewidth}}}]{% 
	  \tt {\tiny [GM$|$GM]}  
	 {'guess': 'truer', 'explanation': 'Given the feedback, \'r\' is confirmed to be in the word but not in the second or fourth positions as previously guessed. The word "truer" is chosen because it fits the new letter arrangement criteria, placing \'r\' in different positions to test its correct placement. Although "truer" might not directly align with the typical use of "passionate," it is selected based on the letter feedback and the challenge of fitting the clue within these constraints'} 
	  } 
	   } 
	   } 
	 & & \\ 
 

    \theutterance \stepcounter{utterance}  

    & & \multicolumn{4}{p{0.6\linewidth}}{\cellcolor[rgb]{0.9,0.9,0.9}{%
	\makecell[{{p{\linewidth}}}]{% 
	  \tt {\tiny [GM$\rangle$B]}  
	 clue:passionate\\ \tt guess:truer\\ \tt explanation:Given the feedback, 'r' is confirmed to be in the word but not in the second or fourth positions as previously guessed. The word "truer" is chosen because it fits the new letter arrangement criteria, placing 'r' in different positions to test its correct placement. Although "truer" might not directly align with the typical use of "passionate," it is selected based on the letter feedback and the challenge of fitting the clue within these constraints\\ \tt guess_feedback: c<red> r<yellow> u<red> s<red> h<red> 
	  } 
	   } 
	   } 
	 & \\ 
 

    \theutterance \stepcounter{utterance}  

    & & & \multicolumn{4}{p{0.6\linewidth}}{\cellcolor[rgb]{1,0.85,0.72}{%
	\makecell[{{p{\linewidth}}}]{% 
	  \tt {\tiny [GM$\langle$B]}  
	 agreement:no\\ \tt explanation:the feedback indicates that 'r' is a correct letter but in the wrong position, and 'u' is not in the word, making "truer" an incorrect guess based on the provided feedback. 
	  } 
	   } 
	   } 
	  \\ 
 

    \theutterance \stepcounter{utterance}  

    & & & \multicolumn{2}{p{0.3\linewidth}}{\cellcolor[rgb]{0.95,0.95,0.95}{%
	\makecell[{{p{\linewidth}}}]{% 
	  \tt {\tiny [GM$|$GM]}  
	 {'agreement': 'no', 'explanation': 'the feedback indicates that \'r\' is a correct letter but in the wrong position, and \'u\' is not in the word, making "truer" an incorrect guess based on the provided feedback.'} 
	  } 
	   } 
	   } 
	 & & \\ 
 

    \theutterance \stepcounter{utterance}  

    & & & \multicolumn{2}{p{0.3\linewidth}}{\cellcolor[rgb]{0.95,0.95,0.95}{%
	\makecell[{{p{\linewidth}}}]{% 
	  \tt {\tiny [GM$|$GM]}  
	 Critic disagrees with the Guesser -- Sharing the critic's explanation with the guesser 
	  } 
	   } 
	   } 
	 & & \\ 
 

    \theutterance \stepcounter{utterance}  

    & & \multicolumn{4}{p{0.6\linewidth}}{\cellcolor[rgb]{0.9,0.9,0.9}{%
	\makecell[{{p{\linewidth}}}]{% 
	  \tt {\tiny [A$\langle$GM]}  
	 clue:passionate\\ \tt guess_agreement:no\\ \tt agreement_explanation:the feedback indicates that 'r' is a correct letter but in the wrong position, and 'u' is not in the word, making "truer" an incorrect guess based on the provided feedback.\\ \tt  
	  } 
	   } 
	   } 
	 & \\ 
 

    \theutterance \stepcounter{utterance}  

    & \multicolumn{4}{p{0.6\linewidth}}{\cellcolor[rgb]{0.8,1,0.9}{%
	\makecell[{{p{\linewidth}}}]{% 
	  \tt {\tiny [A$\rangle$GM]}  
	 guess:flare\\ \tt explanation:Considering the feedback and the error in my previous guess, "flare" is chosen because it aligns with the theme of being passionate, as in a flare of emotions or passion. It correctly excludes previously guessed incorrect letters ('c', 'u', 's', 'h', 'a', 'd', 'o') and places 'r' in a new position, while introducing 'f', 'l', 'a', and 'e' to find the correct 
	  } 
	   } 
	   } 
	 & & \\ 
 

    \theutterance \stepcounter{utterance}  

    & & & \multicolumn{2}{p{0.3\linewidth}}{\cellcolor[rgb]{0.95,0.95,0.95}{%
	\makecell[{{p{\linewidth}}}]{% 
	  \tt {\tiny [GM$|$GM]}  
	 {'guess': 'flare', 'explanation': 'Considering the feedback and the error in my previous guess, "flare" is chosen because it aligns with the theme of being passionate, as in a flare of emotions or passion. It correctly excludes previously guessed incorrect letters (\'c\', \'u\', \'s\', \'h\', \'a\', \'d\', \'o\') and places \'r\' in a new position, while introducing \'f\', \'l\', \'a\', and \'e\' to find the correct'} 
	  } 
	   } 
	   } 
	 & & \\ 
 

    \theutterance \stepcounter{utterance}  

    & & & \multicolumn{2}{p{0.3\linewidth}}{\cellcolor[rgb]{0.95,0.95,0.95}{%
	\makecell[{{p{\linewidth}}}]{% 
	  \tt {\tiny [GM$|$GM]}  
	 Change in player1's guess\\ \tt guess_before_critic_opinion: truer\\ \tt                                                 critic_agreement: no\\ \tt guess_after_critic_opinion: flare\\ \tt                                                 Proceeding with guess validation 
	  } 
	   } 
	   } 
	 & & \\ 
 

    \theutterance \stepcounter{utterance}  

    & & & \multicolumn{2}{p{0.3\linewidth}}{\cellcolor[rgb]{0.95,0.95,0.95}{%
	\makecell[{{p{\linewidth}}}]{% 
	  \tt {\tiny [GM$|$GM]}  
	 attempts: 3\\ \tt target_word = fiery\\ \tt target_word_clue = passionate\\ \tt guess: flare\\ \tt guess_feedback: f<green> l<red> a<red> r<green> e<yellow> 
	  } 
	   } 
	   } 
	 & & \\ 
 

    \theutterance \stepcounter{utterance}  

    & & \multicolumn{4}{p{0.6\linewidth}}{\cellcolor[rgb]{0.9,0.9,0.9}{%
	\makecell[{{p{\linewidth}}}]{% 
	  \tt {\tiny [A$\langle$GM]}  
	 guess_feedback: f<green> l<red> a<red> r<green> e<yellow> 
	  } 
	   } 
	   } 
	 & \\ 
 

    \theutterance \stepcounter{utterance}  

    & \multicolumn{4}{p{0.6\linewidth}}{\cellcolor[rgb]{0.8,1,0.9}{%
	\makecell[{{p{\linewidth}}}]{% 
	  \tt {\tiny [A$\rangle$GM]}  
	 guess:freer\\ \tt explanation:Based on the feedback, we know 'f' and 'r' are correctly placed in the first and third positions, respectively, and 'e' is in the word but in the wrong position. "Freer" is chosen because it maintains the correct positions of 'f' and 'r', repositions 'e' to test its correct placement, and introduces a new 'r' to fit the feedback. This guess also aligns with the theme of 
	  } 
	   } 
	   } 
	 & & \\ 
 

    \theutterance \stepcounter{utterance}  

    & & & \multicolumn{2}{p{0.3\linewidth}}{\cellcolor[rgb]{0.95,0.95,0.95}{%
	\makecell[{{p{\linewidth}}}]{% 
	  \tt {\tiny [GM$|$GM]}  
	 {'guess': 'freer', 'explanation': 'Based on the feedback, we know \'f\' and \'r\' are correctly placed in the first and third positions, respectively, and \'e\' is in the word but in the wrong position. "Freer" is chosen because it maintains the correct positions of \'f\' and \'r\', repositions \'e\' to test its correct placement, and introduces a new \'r\' to fit the feedback. This guess also aligns with the theme of'} 
	  } 
	   } 
	   } 
	 & & \\ 
 

    \theutterance \stepcounter{utterance}  

    & & \multicolumn{4}{p{0.6\linewidth}}{\cellcolor[rgb]{0.9,0.9,0.9}{%
	\makecell[{{p{\linewidth}}}]{% 
	  \tt {\tiny [GM$\rangle$B]}  
	 clue:passionate\\ \tt guess:freer\\ \tt explanation:Based on the feedback, we know 'f' and 'r' are correctly placed in the first and third positions, respectively, and 'e' is in the word but in the wrong position. "Freer" is chosen because it maintains the correct positions of 'f' and 'r', repositions 'e' to test its correct placement, and introduces a new 'r' to fit the feedback. This guess also aligns with the theme of\\ \tt guess_feedback: f<green> l<red> a<red> r<green> e<yellow> 
	  } 
	   } 
	   } 
	 & \\ 
 

    \theutterance \stepcounter{utterance}  

    & & & \multicolumn{4}{p{0.6\linewidth}}{\cellcolor[rgb]{1,0.85,0.72}{%
	\makecell[{{p{\linewidth}}}]{% 
	  \tt {\tiny [GM$\langle$B]}  
	 agreement:yes\\ \tt explanation:Given the feedback, "freer" correctly places 'f' and 'r' as indicated by the green markers and repositions 'e' according to the yellow marker, making it a valid attempt to match the clue "passionate" with the constraints provided. 
	  } 
	   } 
	   } 
	  \\ 
 

    \theutterance \stepcounter{utterance}  

    & & & \multicolumn{2}{p{0.3\linewidth}}{\cellcolor[rgb]{0.95,0.95,0.95}{%
	\makecell[{{p{\linewidth}}}]{% 
	  \tt {\tiny [GM$|$GM]}  
	 {'agreement': 'yes', 'explanation': 'Given the feedback, "freer" correctly places \'f\' and \'r\' as indicated by the green markers and repositions \'e\' according to the yellow marker, making it a valid attempt to match the clue "passionate" with the constraints provided.'} 
	  } 
	   } 
	   } 
	 & & \\ 
 

    \theutterance \stepcounter{utterance}  

    & & & \multicolumn{2}{p{0.3\linewidth}}{\cellcolor[rgb]{0.95,0.95,0.95}{%
	\makecell[{{p{\linewidth}}}]{% 
	  \tt {\tiny [GM$|$GM]}  
	 Critic agrees with the Guesser -- Sharing the critic's explanation with the guesser 
	  } 
	   } 
	   } 
	 & & \\ 
 

    \theutterance \stepcounter{utterance}  

    & & \multicolumn{4}{p{0.6\linewidth}}{\cellcolor[rgb]{0.9,0.9,0.9}{%
	\makecell[{{p{\linewidth}}}]{% 
	  \tt {\tiny [A$\langle$GM]}  
	 clue:passionate\\ \tt guess_agreement:yes\\ \tt agreement_explanation:Given the feedback, "freer" correctly places 'f' and 'r' as indicated by the green markers and repositions 'e' according to the yellow marker, making it a valid attempt to match the clue "passionate" with the constraints provided.\\ \tt  
	  } 
	   } 
	   } 
	 & \\ 
 

    \theutterance \stepcounter{utterance}  

    & \multicolumn{4}{p{0.6\linewidth}}{\cellcolor[rgb]{0.8,1,0.9}{%
	\makecell[{{p{\linewidth}}}]{% 
	  \tt {\tiny [A$\rangle$GM]}  
	 It appears there was a misunderstanding in my previous response, as I inadvertently continued without receiving new feedback for the guess "freer." Without additional feedback or clues, I cannot accurately proceed to the next guess. My role is to respond based on the information and feedback provided by you. Could you please provide feedback for the guess "freer"? 
	  } 
	   } 
	   } 
	 & & \\ 
 

    \theutterance \stepcounter{utterance}  

    & & & \multicolumn{2}{p{0.3\linewidth}}{\cellcolor[rgb]{0.95,0.95,0.95}{%
	\makecell[{{p{\linewidth}}}]{% 
	  \tt {\tiny [GM$|$GM]}  
	 {'guess': 'INVALID_FORMAT', 'explanation': 'INVALID_FORMAT'} 
	  } 
	   } 
	   } 
	 & & \\ 
 

    \theutterance \stepcounter{utterance}  

    & & & \multicolumn{2}{p{0.3\linewidth}}{\cellcolor[rgb]{0.95,0.95,0.95}{%
	\makecell[{{p{\linewidth}}}]{% 
	  \tt {\tiny [GM$|$GM]}  
	 Guesser Error: INVALID_FORMAT while parsing Player 1's (model: gpt-4-0125-preview) response, retrying 
	  } 
	   } 
	   } 
	 & & \\ 
 

    \theutterance \stepcounter{utterance}  

    & & \multicolumn{4}{p{0.6\linewidth}}{\cellcolor[rgb]{0.9,0.9,0.9}{%
	\makecell[{{p{\linewidth}}}]{% 
	  \tt {\tiny [A$\langle$GM]}  
	 Provide your response only in this format:\\ \tt guess:word\\ \tt explanation:details.\\ \tt Please try again 
	  } 
	   } 
	   } 
	 & \\ 
 

    \theutterance \stepcounter{utterance}  

    & \multicolumn{4}{p{0.6\linewidth}}{\cellcolor[rgb]{0.8,1,0.9}{%
	\makecell[{{p{\linewidth}}}]{% 
	  \tt {\tiny [A$\rangle$GM]}  
	 guess:fever\\ \tt explanation:Given the feedback, 'f' and 'r' are correctly placed in the first and third positions, and 'e' is in the word but was in the wrong position in the previous guess. "Fever" fits this criteria, placing 'e' in a new position and introducing 'v' as a new letter. The term "fever" can metaphorically relate to being passionate or intense in emotions or actions. 
	  } 
	   } 
	   } 
	 & & \\ 
 

    \theutterance \stepcounter{utterance}  

    & & & \multicolumn{2}{p{0.3\linewidth}}{\cellcolor[rgb]{0.95,0.95,0.95}{%
	\makecell[{{p{\linewidth}}}]{% 
	  \tt {\tiny [GM$|$GM]}  
	 {'guess': 'fever', 'explanation': 'Given the feedback, \'f\' and \'r\' are correctly placed in the first and third positions, and \'e\' is in the word but was in the wrong position in the previous guess. "Fever" fits this criteria, placing \'e\' in a new position and introducing \'v\' as a new letter. The term "fever" can metaphorically relate to being passionate or intense in emotions or actions.'} 
	  } 
	   } 
	   } 
	 & & \\ 
 

    \theutterance \stepcounter{utterance}  

    & & & \multicolumn{2}{p{0.3\linewidth}}{\cellcolor[rgb]{0.95,0.95,0.95}{%
	\makecell[{{p{\linewidth}}}]{% 
	  \tt {\tiny [GM$|$GM]}  
	 Change in player1's guess\\ \tt guess_before_critic_opinion: freer\\ \tt                                                 critic_agreement: yes\\ \tt guess_after_critic_opinion: fever\\ \tt                                                 Proceeding with guess validation 
	  } 
	   } 
	   } 
	 & & \\ 
 

    \theutterance \stepcounter{utterance}  

    & & & \multicolumn{2}{p{0.3\linewidth}}{\cellcolor[rgb]{0.95,0.95,0.95}{%
	\makecell[{{p{\linewidth}}}]{% 
	  \tt {\tiny [GM$|$GM]}  
	 attempts: 4\\ \tt target_word = fiery\\ \tt target_word_clue = passionate\\ \tt guess: fever\\ \tt guess_feedback: f<green> e<yellow> v<red> e<red> r<yellow> 
	  } 
	   } 
	   } 
	 & & \\ 
 

    \theutterance \stepcounter{utterance}  

    & & \multicolumn{4}{p{0.6\linewidth}}{\cellcolor[rgb]{0.9,0.9,0.9}{%
	\makecell[{{p{\linewidth}}}]{% 
	  \tt {\tiny [A$\langle$GM]}  
	 guess_feedback: f<green> e<yellow> v<red> e<red> r<yellow> 
	  } 
	   } 
	   } 
	 & \\ 
 

    \theutterance \stepcounter{utterance}  

    & \multicolumn{4}{p{0.6\linewidth}}{\cellcolor[rgb]{0.8,1,0.9}{%
	\makecell[{{p{\linewidth}}}]{% 
	  \tt {\tiny [A$\rangle$GM]}  
	 guess:fryer\\ \tt explanation:With the feedback indicating 'f' is correct in the first position, and both 'e' and 'r' are in the word but in the wrong positions previously, "fryer" is chosen. It correctly places 'f' at the start, tests a new position for 'e' and 'r', and introduces 'y' to fit the pattern suggested by the feedback, while still attempting to align with the clue "passionate." 
	  } 
	   } 
	   } 
	 & & \\ 
 

    \theutterance \stepcounter{utterance}  

    & & & \multicolumn{2}{p{0.3\linewidth}}{\cellcolor[rgb]{0.95,0.95,0.95}{%
	\makecell[{{p{\linewidth}}}]{% 
	  \tt {\tiny [GM$|$GM]}  
	 {'guess': 'fryer', 'explanation': 'With the feedback indicating \'f\' is correct in the first position, and both \'e\' and \'r\' are in the word but in the wrong positions previously, "fryer" is chosen. It correctly places \'f\' at the start, tests a new position for \'e\' and \'r\', and introduces \'y\' to fit the pattern suggested by the feedback, while still attempting to align with the clue "passionate."'} 
	  } 
	   } 
	   } 
	 & & \\ 
 

    \theutterance \stepcounter{utterance}  

    & & \multicolumn{4}{p{0.6\linewidth}}{\cellcolor[rgb]{0.9,0.9,0.9}{%
	\makecell[{{p{\linewidth}}}]{% 
	  \tt {\tiny [GM$\rangle$B]}  
	 clue:passionate\\ \tt guess:fryer\\ \tt explanation:With the feedback indicating 'f' is correct in the first position, and both 'e' and 'r' are in the word but in the wrong positions previously, "fryer" is chosen. It correctly places 'f' at the start, tests a new position for 'e' and 'r', and introduces 'y' to fit the pattern suggested by the feedback, while still attempting to align with the clue "passionate."\\ \tt guess_feedback: f<green> e<yellow> v<red> e<red> r<yellow> 
	  } 
	   } 
	   } 
	 & \\ 
 

    \theutterance \stepcounter{utterance}  

    & & & \multicolumn{4}{p{0.6\linewidth}}{\cellcolor[rgb]{1,0.85,0.72}{%
	\makecell[{{p{\linewidth}}}]{% 
	  \tt {\tiny [GM$\langle$B]}  
	 agreement:no\\ \tt explanation:the feedback indicates that 'e' and 'r' are in the word but not in the positions guessed in "fryer." Since 'e' and the second 'r' in "fryer" are marked red, they cannot be in those positions or are incorrectly included based on the feedback rules provided. 
	  } 
	   } 
	   } 
	  \\ 
 

    \theutterance \stepcounter{utterance}  

    & & & \multicolumn{2}{p{0.3\linewidth}}{\cellcolor[rgb]{0.95,0.95,0.95}{%
	\makecell[{{p{\linewidth}}}]{% 
	  \tt {\tiny [GM$|$GM]}  
	 {'agreement': 'no', 'explanation': 'the feedback indicates that \'e\' and \'r\' are in the word but not in the positions guessed in "fryer." Since \'e\' and the second \'r\' in "fryer" are marked red, they cannot be in those positions or are incorrectly included based on the feedback rules provided.'} 
	  } 
	   } 
	   } 
	 & & \\ 
 

    \theutterance \stepcounter{utterance}  

    & & & \multicolumn{2}{p{0.3\linewidth}}{\cellcolor[rgb]{0.95,0.95,0.95}{%
	\makecell[{{p{\linewidth}}}]{% 
	  \tt {\tiny [GM$|$GM]}  
	 Critic disagrees with the Guesser -- Sharing the critic's explanation with the guesser 
	  } 
	   } 
	   } 
	 & & \\ 
 

    \theutterance \stepcounter{utterance}  

    & & \multicolumn{4}{p{0.6\linewidth}}{\cellcolor[rgb]{0.9,0.9,0.9}{%
	\makecell[{{p{\linewidth}}}]{% 
	  \tt {\tiny [A$\langle$GM]}  
	 clue:passionate\\ \tt guess_agreement:no\\ \tt agreement_explanation:the feedback indicates that 'e' and 'r' are in the word but not in the positions guessed in "fryer." Since 'e' and the second 'r' in "fryer" are marked red, they cannot be in those positions or are incorrectly included based on the feedback rules provided.\\ \tt  
	  } 
	   } 
	   } 
	 & \\ 
 

    \theutterance \stepcounter{utterance}  

    & \multicolumn{4}{p{0.6\linewidth}}{\cellcolor[rgb]{0.8,1,0.9}{%
	\makecell[{{p{\linewidth}}}]{% 
	  \tt {\tiny [A$\rangle$GM]}  
	 guess:refer\\ \tt explanation:Considering the correct feedback and previous errors, "refer" is chosen to adjust the positions of 'e' and 'r' based on the feedback. 'f' remains in the correct position, and 'e' and 'r' are rearranged to fit the clue "passionate," interpreting passion in the sense of referring to something with strong feeling or emphasis. 
	  } 
	   } 
	   } 
	 & & \\ 
 

    \theutterance \stepcounter{utterance}  

    & & & \multicolumn{2}{p{0.3\linewidth}}{\cellcolor[rgb]{0.95,0.95,0.95}{%
	\makecell[{{p{\linewidth}}}]{% 
	  \tt {\tiny [GM$|$GM]}  
	 {'guess': 'refer', 'explanation': 'Considering the correct feedback and previous errors, "refer" is chosen to adjust the positions of \'e\' and \'r\' based on the feedback. \'f\' remains in the correct position, and \'e\' and \'r\' are rearranged to fit the clue "passionate," interpreting passion in the sense of referring to something with strong feeling or emphasis.'} 
	  } 
	   } 
	   } 
	 & & \\ 
 

    \theutterance \stepcounter{utterance}  

    & & & \multicolumn{2}{p{0.3\linewidth}}{\cellcolor[rgb]{0.95,0.95,0.95}{%
	\makecell[{{p{\linewidth}}}]{% 
	  \tt {\tiny [GM$|$GM]}  
	 Change in player1's guess\\ \tt guess_before_critic_opinion: fryer\\ \tt                                                 critic_agreement: no\\ \tt guess_after_critic_opinion: refer\\ \tt                                                 Proceeding with guess validation 
	  } 
	   } 
	   } 
	 & & \\ 
 

    \theutterance \stepcounter{utterance}  

    & & & \multicolumn{2}{p{0.3\linewidth}}{\cellcolor[rgb]{0.95,0.95,0.95}{%
	\makecell[{{p{\linewidth}}}]{% 
	  \tt {\tiny [GM$|$GM]}  
	 attempts: 5\\ \tt target_word = fiery\\ \tt target_word_clue = passionate\\ \tt guess: refer\\ \tt guess_feedback: r<yellow> e<yellow> f<yellow> e<red> r<red> 
	  } 
	   } 
	   } 
	 & & \\ 
 

    \theutterance \stepcounter{utterance}  

    & & \multicolumn{4}{p{0.6\linewidth}}{\cellcolor[rgb]{0.9,0.9,0.9}{%
	\makecell[{{p{\linewidth}}}]{% 
	  \tt {\tiny [A$\langle$GM]}  
	 guess_feedback: r<yellow> e<yellow> f<yellow> e<red> r<red> 
	  } 
	   } 
	   } 
	 & \\ 
 

    \theutterance \stepcounter{utterance}  

    & \multicolumn{4}{p{0.6\linewidth}}{\cellcolor[rgb]{0.8,1,0.9}{%
	\makecell[{{p{\linewidth}}}]{% 
	  \tt {\tiny [A$\rangle$GM]}  
	 guess:forge\\ \tt explanation:Based on the feedback, 'r', 'e', and 'f' are in the word but not in their previously guessed positions. "Forge" is selected because it rearranges these letters to new positions (with 'f' now not starting the word) and introduces 'o' and 'g' as new letters. The word "forge" can be associated with passion in terms of creating or shaping something with intense dedication or emotion. 
	  } 
	   } 
	   } 
	 & & \\ 
 

    \theutterance \stepcounter{utterance}  

    & & & \multicolumn{2}{p{0.3\linewidth}}{\cellcolor[rgb]{0.95,0.95,0.95}{%
	\makecell[{{p{\linewidth}}}]{% 
	  \tt {\tiny [GM$|$GM]}  
	 {'guess': 'forge', 'explanation': 'Based on the feedback, \'r\', \'e\', and \'f\' are in the word but not in their previously guessed positions. "Forge" is selected because it rearranges these letters to new positions (with \'f\' now not starting the word) and introduces \'o\' and \'g\' as new letters. The word "forge" can be associated with passion in terms of creating or shaping something with intense dedication or emotion.'} 
	  } 
	   } 
	   } 
	 & & \\ 
 

    \theutterance \stepcounter{utterance}  

    & & \multicolumn{4}{p{0.6\linewidth}}{\cellcolor[rgb]{0.9,0.9,0.9}{%
	\makecell[{{p{\linewidth}}}]{% 
	  \tt {\tiny [GM$\rangle$B]}  
	 clue:passionate\\ \tt guess:forge\\ \tt explanation:Based on the feedback, 'r', 'e', and 'f' are in the word but not in their previously guessed positions. "Forge" is selected because it rearranges these letters to new positions (with 'f' now not starting the word) and introduces 'o' and 'g' as new letters. The word "forge" can be associated with passion in terms of creating or shaping something with intense dedication or emotion.\\ \tt guess_feedback: r<yellow> e<yellow> f<yellow> e<red> r<red> 
	  } 
	   } 
	   } 
	 & \\ 
 

    \theutterance \stepcounter{utterance}  

    & & & \multicolumn{4}{p{0.6\linewidth}}{\cellcolor[rgb]{1,0.85,0.72}{%
	\makecell[{{p{\linewidth}}}]{% 
	  \tt {\tiny [GM$\langle$B]}  
	 agreement:no\\ \tt explanation:the feedback indicates that 'f' should be in the correct position at the start of the word, contradicting the placement in the guess "forge". 
	  } 
	   } 
	   } 
	  \\ 
 

    \theutterance \stepcounter{utterance}  

    & & & \multicolumn{2}{p{0.3\linewidth}}{\cellcolor[rgb]{0.95,0.95,0.95}{%
	\makecell[{{p{\linewidth}}}]{% 
	  \tt {\tiny [GM$|$GM]}  
	 {'agreement': 'no', 'explanation': 'the feedback indicates that \'f\' should be in the correct position at the start of the word, contradicting the placement in the guess "forge".'} 
	  } 
	   } 
	   } 
	 & & \\ 
 

    \theutterance \stepcounter{utterance}  

    & & & \multicolumn{2}{p{0.3\linewidth}}{\cellcolor[rgb]{0.95,0.95,0.95}{%
	\makecell[{{p{\linewidth}}}]{% 
	  \tt {\tiny [GM$|$GM]}  
	 Critic disagrees with the Guesser -- Sharing the critic's explanation with the guesser 
	  } 
	   } 
	   } 
	 & & \\ 
 

    \theutterance \stepcounter{utterance}  

    & & \multicolumn{4}{p{0.6\linewidth}}{\cellcolor[rgb]{0.9,0.9,0.9}{%
	\makecell[{{p{\linewidth}}}]{% 
	  \tt {\tiny [A$\langle$GM]}  
	 clue:passionate\\ \tt guess_agreement:no\\ \tt agreement_explanation:the feedback indicates that 'f' should be in the correct position at the start of the word, contradicting the placement in the guess "forge".\\ \tt  
	  } 
	   } 
	   } 
	 & \\ 
 

    \theutterance \stepcounter{utterance}  

    & \multicolumn{4}{p{0.6\linewidth}}{\cellcolor[rgb]{0.8,1,0.9}{%
	\makecell[{{p{\linewidth}}}]{% 
	  \tt {\tiny [A$\rangle$GM]}  
	 guess:fresh\\ \tt explanation:Correcting the oversight regarding 'f' needing to be at the start, "fresh" is chosen to comply with the feedback that 'f' is correctly placed at the beginning. Given the yellow indicators for 'r' and 'e', they are included but in new positions, with 'r' now following 'f' directly as indicated by previous feedback to be incorrect, and 'e' placed later in the word. 's' and 'h' are 
	  } 
	   } 
	   } 
	 & & \\ 
 

    \theutterance \stepcounter{utterance}  

    & & & \multicolumn{2}{p{0.3\linewidth}}{\cellcolor[rgb]{0.95,0.95,0.95}{%
	\makecell[{{p{\linewidth}}}]{% 
	  \tt {\tiny [GM$|$GM]}  
	 {'guess': 'fresh', 'explanation': 'Correcting the oversight regarding \'f\' needing to be at the start, "fresh" is chosen to comply with the feedback that \'f\' is correctly placed at the beginning. Given the yellow indicators for \'r\' and \'e\', they are included but in new positions, with \'r\' now following \'f\' directly as indicated by previous feedback to be incorrect, and \'e\' placed later in the word. \'s\' and \'h\' are'} 
	  } 
	   } 
	   } 
	 & & \\ 
 

    \theutterance \stepcounter{utterance}  

    & & & \multicolumn{2}{p{0.3\linewidth}}{\cellcolor[rgb]{0.95,0.95,0.95}{%
	\makecell[{{p{\linewidth}}}]{% 
	  \tt {\tiny [GM$|$GM]}  
	 Change in player1's guess\\ \tt guess_before_critic_opinion: forge\\ \tt                                                 critic_agreement: no\\ \tt guess_after_critic_opinion: fresh\\ \tt                                                 Proceeding with guess validation 
	  } 
	   } 
	   } 
	 & & \\ 
 

    \theutterance \stepcounter{utterance}  

    & & & \multicolumn{2}{p{0.3\linewidth}}{\cellcolor[rgb]{0.95,0.95,0.95}{%
	\makecell[{{p{\linewidth}}}]{% 
	  \tt {\tiny [GM$|$GM]}  
	 attempts: 6\\ \tt target_word = fiery\\ \tt target_word_clue = passionate\\ \tt guess: fresh\\ \tt guess_feedback: f<green> r<yellow> e<green> s<red> h<red> 
	  } 
	   } 
	   } 
	 & & \\ 
 

    \theutterance \stepcounter{utterance}  

    & & & \multicolumn{2}{p{0.3\linewidth}}{\cellcolor[rgb]{0.95,0.95,0.95}{%
	\makecell[{{p{\linewidth}}}]{% 
	  \tt {\tiny [GM$|$GM]}  
	 game_result = LOSS 
	  } 
	   } 
	   } 
	 & & \\ 
 

\end{supertabular}
}

\end{document}

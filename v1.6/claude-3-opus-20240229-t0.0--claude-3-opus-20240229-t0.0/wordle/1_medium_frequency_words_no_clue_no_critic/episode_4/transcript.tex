
\documentclass{article}
\usepackage{colortbl}
\usepackage{makecell}
\usepackage{multirow}
\usepackage{supertabular}

\begin{document}

\newcounter{utterance}

\twocolumn

{ \footnotesize  \setcounter{utterance}{1}
\setlength{\tabcolsep}{0pt}
\begin{supertabular}{c@{$\;$}|p{.15\linewidth}@{}p{.15\linewidth}p{.15\linewidth}p{.15\linewidth}p{.15\linewidth}p{.15\linewidth}}

    \# & $\;$A & \multicolumn{4}{c}{Game Master} & $\;\:$B\\
    \hline 

    \theutterance \stepcounter{utterance}  

    & & \multicolumn{4}{p{0.6\linewidth}}{\cellcolor[rgb]{0.9,0.9,0.9}{%
	\makecell[{{p{\linewidth}}}]{% 
	  \tt {\tiny [A$\langle$GM]}  
	 You are a language wizard who likes to guess words by using the given rules.\\ \tt \\ \tt Welcome to Wordle! You have six attempts to guess the target word, a valid English word of five lowercase letters (a-z). Please use the tags "guess:" and "explanation:" to provide a concise explanation for each guess.\\ \tt \\ \tt For instance, if your guess is "apple", your response should be\\ \tt guess: apple\\ \tt  explanation: this is a common five-letter English word, and I am starting my guess with this word.\\ \tt \\ \tt After each guess, your answer will be validated, and you will receive feedback indicating which letters are correct (green), which letters are correct but in the wrong position (yellow), and which letters are incorrect (red). This feedback can be useful in determining which letters to include or exclude in your next guess.\\ \tt \\ \tt For example, the feedback for "apple" might be:\\ \tt guess_feedback: a<yellow> p<yellow> p<green> l<yellow> e<red>\\ \tt \\ \tt The explanation should contain details about how the guess_feedback is used to arrive at a new guess.\\ \tt \\ \tt Let's begin with your first guess. 
	  } 
	   } 
	   } 
	 & \\ 
 

    \theutterance \stepcounter{utterance}  

    & \multicolumn{4}{p{0.6\linewidth}}{\cellcolor[rgb]{0.8,1,0.9}{%
	\makecell[{{p{\linewidth}}}]{% 
	  \tt {\tiny [A$\rangle$GM]}  
	 guess: stare\\ \tt explanation: I chose "stare" as my first guess because it contains common letters in different positions, which can help narrow down the possibilities based on the feedback. The letters "s", "t", "a", "r", and "e" are all frequently used in English words. 
	  } 
	   } 
	   } 
	 & & \\ 
 

    \theutterance \stepcounter{utterance}  

    & & & \multicolumn{2}{p{0.3\linewidth}}{\cellcolor[rgb]{0.95,0.95,0.95}{%
	\makecell[{{p{\linewidth}}}]{% 
	  \tt {\tiny [GM$|$GM]}  
	 {'guess': 'stare', 'explanation': 'I chose "stare" as my first guess because it contains common letters in different positions, which can help narrow down the possibilities based on the feedback. The letters "s", "t", "a", "r", and "e" are all frequently used in English words.'} 
	  } 
	   } 
	   } 
	 & & \\ 
 

    \theutterance \stepcounter{utterance}  

    & & & \multicolumn{2}{p{0.3\linewidth}}{\cellcolor[rgb]{0.95,0.95,0.95}{%
	\makecell[{{p{\linewidth}}}]{% 
	  \tt {\tiny [GM$|$GM]}  
	 attempts: 1\\ \tt target_word = nerdy\\ \tt guess: stare\\ \tt guess_feedback: s<red> t<red> a<red> r<yellow> e<yellow> 
	  } 
	   } 
	   } 
	 & & \\ 
 

    \theutterance \stepcounter{utterance}  

    & & \multicolumn{4}{p{0.6\linewidth}}{\cellcolor[rgb]{0.9,0.9,0.9}{%
	\makecell[{{p{\linewidth}}}]{% 
	  \tt {\tiny [A$\langle$GM]}  
	 guess_feedback: s<red> t<red> a<red> r<yellow> e<yellow> 
	  } 
	   } 
	   } 
	 & \\ 
 

    \theutterance \stepcounter{utterance}  

    & \multicolumn{4}{p{0.6\linewidth}}{\cellcolor[rgb]{0.8,1,0.9}{%
	\makecell[{{p{\linewidth}}}]{% 
	  \tt {\tiny [A$\rangle$GM]}  
	 guess: loner\\ \tt explanation: Based on the feedback, I know that "r" and "e" are in the word but in different positions, while "s", "t", and "a" are not in the word at all. I chose "loner" because it contains "r" and "e" in new positions and introduces common letters "l", "o", and "n" to help narrow down the possibilities further. 
	  } 
	   } 
	   } 
	 & & \\ 
 

    \theutterance \stepcounter{utterance}  

    & & & \multicolumn{2}{p{0.3\linewidth}}{\cellcolor[rgb]{0.95,0.95,0.95}{%
	\makecell[{{p{\linewidth}}}]{% 
	  \tt {\tiny [GM$|$GM]}  
	 {'guess': 'loner', 'explanation': 'Based on the feedback, I know that "r" and "e" are in the word but in different positions, while "s", "t", and "a" are not in the word at all. I chose "loner" because it contains "r" and "e" in new positions and introduces common letters "l", "o", and "n" to help narrow down the possibilities further.'} 
	  } 
	   } 
	   } 
	 & & \\ 
 

    \theutterance \stepcounter{utterance}  

    & & & \multicolumn{2}{p{0.3\linewidth}}{\cellcolor[rgb]{0.95,0.95,0.95}{%
	\makecell[{{p{\linewidth}}}]{% 
	  \tt {\tiny [GM$|$GM]}  
	 attempts: 2\\ \tt target_word = nerdy\\ \tt guess: loner\\ \tt guess_feedback: l<red> o<red> n<yellow> e<yellow> r<yellow> 
	  } 
	   } 
	   } 
	 & & \\ 
 

    \theutterance \stepcounter{utterance}  

    & & \multicolumn{4}{p{0.6\linewidth}}{\cellcolor[rgb]{0.9,0.9,0.9}{%
	\makecell[{{p{\linewidth}}}]{% 
	  \tt {\tiny [A$\langle$GM]}  
	 guess_feedback: l<red> o<red> n<yellow> e<yellow> r<yellow> 
	  } 
	   } 
	   } 
	 & \\ 
 

    \theutterance \stepcounter{utterance}  

    & \multicolumn{4}{p{0.6\linewidth}}{\cellcolor[rgb]{0.8,1,0.9}{%
	\makecell[{{p{\linewidth}}}]{% 
	  \tt {\tiny [A$\rangle$GM]}  
	 guess:inert\\ \tt explanation:The feedback shows that "n", "e", and "r" are in the word but not in the correct positions, while "l" and "o" are not in the word. I chose "inert" because it rearranges the positions of "n", "e", and "r" while introducing the common letter "i" and the less common letter "t" to help identify more of the correct letters and their positions.\\ \tt  
	  } 
	   } 
	   } 
	 & & \\ 
 

    \theutterance \stepcounter{utterance}  

    & & & \multicolumn{2}{p{0.3\linewidth}}{\cellcolor[rgb]{0.95,0.95,0.95}{%
	\makecell[{{p{\linewidth}}}]{% 
	  \tt {\tiny [GM$|$GM]}  
	 {'guess': 'inert', 'explanation': 'The feedback shows that "n", "e", and "r" are in the word but not in the correct positions, while "l" and "o" are not in the word. I chose "inert" because it rearranges the positions of "n", "e", and "r" while introducing the common letter "i" and the less common letter "t" to help identify more of the correct letters and their positions.'} 
	  } 
	   } 
	   } 
	 & & \\ 
 

    \theutterance \stepcounter{utterance}  

    & & & \multicolumn{2}{p{0.3\linewidth}}{\cellcolor[rgb]{0.95,0.95,0.95}{%
	\makecell[{{p{\linewidth}}}]{% 
	  \tt {\tiny [GM$|$GM]}  
	 attempts: 3\\ \tt target_word = nerdy\\ \tt guess: inert\\ \tt guess_feedback: i<red> n<yellow> e<yellow> r<yellow> t<red> 
	  } 
	   } 
	   } 
	 & & \\ 
 

    \theutterance \stepcounter{utterance}  

    & & \multicolumn{4}{p{0.6\linewidth}}{\cellcolor[rgb]{0.9,0.9,0.9}{%
	\makecell[{{p{\linewidth}}}]{% 
	  \tt {\tiny [A$\langle$GM]}  
	 guess_feedback: i<red> n<yellow> e<yellow> r<yellow> t<red> 
	  } 
	   } 
	   } 
	 & \\ 
 

    \theutterance \stepcounter{utterance}  

    & \multicolumn{4}{p{0.6\linewidth}}{\cellcolor[rgb]{0.8,1,0.9}{%
	\makecell[{{p{\linewidth}}}]{% 
	  \tt {\tiny [A$\rangle$GM]}  
	 guess:under\\ \tt explanation:The previous feedback confirms that "n", "e", and "r" are in the word but still not in the correct positions. "I" and "t" are not in the word. For this guess, I chose "under" because it keeps the confirmed letters "n", "e", and "r" while placing them in new positions. It also introduces the common letter "u" and the less common letter "d" to help identify the remaining 
	  } 
	   } 
	   } 
	 & & \\ 
 

    \theutterance \stepcounter{utterance}  

    & & & \multicolumn{2}{p{0.3\linewidth}}{\cellcolor[rgb]{0.95,0.95,0.95}{%
	\makecell[{{p{\linewidth}}}]{% 
	  \tt {\tiny [GM$|$GM]}  
	 {'guess': 'under', 'explanation': 'The previous feedback confirms that "n", "e", and "r" are in the word but still not in the correct positions. "I" and "t" are not in the word. For this guess, I chose "under" because it keeps the confirmed letters "n", "e", and "r" while placing them in new positions. It also introduces the common letter "u" and the less common letter "d" to help identify the remaining'} 
	  } 
	   } 
	   } 
	 & & \\ 
 

    \theutterance \stepcounter{utterance}  

    & & & \multicolumn{2}{p{0.3\linewidth}}{\cellcolor[rgb]{0.95,0.95,0.95}{%
	\makecell[{{p{\linewidth}}}]{% 
	  \tt {\tiny [GM$|$GM]}  
	 attempts: 4\\ \tt target_word = nerdy\\ \tt guess: under\\ \tt guess_feedback: u<red> n<yellow> d<yellow> e<yellow> r<yellow> 
	  } 
	   } 
	   } 
	 & & \\ 
 

    \theutterance \stepcounter{utterance}  

    & & \multicolumn{4}{p{0.6\linewidth}}{\cellcolor[rgb]{0.9,0.9,0.9}{%
	\makecell[{{p{\linewidth}}}]{% 
	  \tt {\tiny [A$\langle$GM]}  
	 guess_feedback: u<red> n<yellow> d<yellow> e<yellow> r<yellow> 
	  } 
	   } 
	   } 
	 & \\ 
 

    \theutterance \stepcounter{utterance}  

    & \multicolumn{4}{p{0.6\linewidth}}{\cellcolor[rgb]{0.8,1,0.9}{%
	\makecell[{{p{\linewidth}}}]{% 
	  \tt {\tiny [A$\rangle$GM]}  
	 guess:trend\\ \tt explanation:The feedback shows that "n", "d", "e", and "r" are in the word but not in the correct positions, while "u" is not in the word. I chose "trend" because it keeps the confirmed letters "n", "d", "e", and "r" while rearranging their positions once again. This guess also reintroduces the letter "t", which was previously eliminated, to help determine if 
	  } 
	   } 
	   } 
	 & & \\ 
 

    \theutterance \stepcounter{utterance}  

    & & & \multicolumn{2}{p{0.3\linewidth}}{\cellcolor[rgb]{0.95,0.95,0.95}{%
	\makecell[{{p{\linewidth}}}]{% 
	  \tt {\tiny [GM$|$GM]}  
	 {'guess': 'trend', 'explanation': 'The feedback shows that "n", "d", "e", and "r" are in the word but not in the correct positions, while "u" is not in the word. I chose "trend" because it keeps the confirmed letters "n", "d", "e", and "r" while rearranging their positions once again. This guess also reintroduces the letter "t", which was previously eliminated, to help determine if'} 
	  } 
	   } 
	   } 
	 & & \\ 
 

    \theutterance \stepcounter{utterance}  

    & & & \multicolumn{2}{p{0.3\linewidth}}{\cellcolor[rgb]{0.95,0.95,0.95}{%
	\makecell[{{p{\linewidth}}}]{% 
	  \tt {\tiny [GM$|$GM]}  
	 attempts: 5\\ \tt target_word = nerdy\\ \tt guess: trend\\ \tt guess_feedback: t<red> r<yellow> e<yellow> n<yellow> d<yellow> 
	  } 
	   } 
	   } 
	 & & \\ 
 

    \theutterance \stepcounter{utterance}  

    & & \multicolumn{4}{p{0.6\linewidth}}{\cellcolor[rgb]{0.9,0.9,0.9}{%
	\makecell[{{p{\linewidth}}}]{% 
	  \tt {\tiny [A$\langle$GM]}  
	 guess_feedback: t<red> r<yellow> e<yellow> n<yellow> d<yellow> 
	  } 
	   } 
	   } 
	 & \\ 
 

    \theutterance \stepcounter{utterance}  

    & \multicolumn{4}{p{0.6\linewidth}}{\cellcolor[rgb]{0.8,1,0.9}{%
	\makecell[{{p{\linewidth}}}]{% 
	  \tt {\tiny [A$\rangle$GM]}  
	 guess:ender\\ \tt explanation:The previous feedback confirms that "r", "e", "n", and "d" are in the word, but they are still not in the correct positions. "T" is not in the word. For this final guess, I chose "ender" because it rearranges the positions of the confirmed letters "e", "n", "d", and "r" while keeping them all together. This guess maximizes the chances of placing these 
	  } 
	   } 
	   } 
	 & & \\ 
 

    \theutterance \stepcounter{utterance}  

    & & & \multicolumn{2}{p{0.3\linewidth}}{\cellcolor[rgb]{0.95,0.95,0.95}{%
	\makecell[{{p{\linewidth}}}]{% 
	  \tt {\tiny [GM$|$GM]}  
	 {'guess': 'ender', 'explanation': 'The previous feedback confirms that "r", "e", "n", and "d" are in the word, but they are still not in the correct positions. "T" is not in the word. For this final guess, I chose "ender" because it rearranges the positions of the confirmed letters "e", "n", "d", and "r" while keeping them all together. This guess maximizes the chances of placing these'} 
	  } 
	   } 
	   } 
	 & & \\ 
 

    \theutterance \stepcounter{utterance}  

    & & & \multicolumn{2}{p{0.3\linewidth}}{\cellcolor[rgb]{0.95,0.95,0.95}{%
	\makecell[{{p{\linewidth}}}]{% 
	  \tt {\tiny [GM$|$GM]}  
	 attempts: 6\\ \tt target_word = nerdy\\ \tt guess: ender\\ \tt guess_feedback: e<yellow> n<yellow> d<yellow> e<red> r<yellow> 
	  } 
	   } 
	   } 
	 & & \\ 
 

    \theutterance \stepcounter{utterance}  

    & & & \multicolumn{2}{p{0.3\linewidth}}{\cellcolor[rgb]{0.95,0.95,0.95}{%
	\makecell[{{p{\linewidth}}}]{% 
	  \tt {\tiny [GM$|$GM]}  
	 game_result = LOSS 
	  } 
	   } 
	   } 
	 & & \\ 
 

\end{supertabular}
}

\end{document}

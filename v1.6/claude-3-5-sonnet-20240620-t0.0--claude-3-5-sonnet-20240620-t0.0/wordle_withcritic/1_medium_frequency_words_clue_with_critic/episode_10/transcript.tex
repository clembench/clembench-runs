
\documentclass{article}
\usepackage{colortbl}
\usepackage{makecell}
\usepackage{multirow}
\usepackage{supertabular}

\begin{document}

\newcounter{utterance}

\twocolumn

{ \footnotesize  \setcounter{utterance}{1}
\setlength{\tabcolsep}{0pt}
\begin{supertabular}{c@{$\;$}|p{.15\linewidth}@{}p{.15\linewidth}p{.15\linewidth}p{.15\linewidth}p{.15\linewidth}p{.15\linewidth}}

    \# & $\;$A & \multicolumn{4}{c}{Game Master} & $\;\:$B\\
    \hline 

    \theutterance \stepcounter{utterance}  

    & & \multicolumn{4}{p{0.6\linewidth}}{\cellcolor[rgb]{0.9,0.9,0.9}{%
	\makecell[{{p{\linewidth}}}]{% 
	  \tt {\tiny [A$\langle$GM]}  
	 You are a language wizard who likes to guess words by using the given rules.\\ \tt \\ \tt Welcome to Wordle! You have six attempts to guess the target word, a valid English word of five lowercase letters (a-z). Please use the tags "guess:" and "explanation:" to provide a concise explanation for each guess.\\ \tt \\ \tt To help you make an informed guess, you will receive a clue for the word, such as\\ \tt clue: "snowy white"\\ \tt \\ \tt Here is an example guess based on the clue:\\ \tt guess: apple\\ \tt explanation: In the fairy tail Snow White, the girl is killed because she eats a poisoned apple. And the word apple has 5 letters.\\ \tt \\ \tt I will then indicate whether I agree or disagree with your guess and provide rationale, but agreeing with a guess does not confirm its correctness. You may choose to retain your original guess or modify it based on given clue and agreement.\\ \tt \\ \tt After each guess, your answer will be validated, and you will receive feedback indicating which letters are correct (green), which letters are correct but in the wrong position (yellow), and which letters are incorrect (red). This feedback can be useful in determining which letters to include or exclude in your next guess.\\ \tt \\ \tt For example, the feedback for "apple" might be:\\ \tt guess_feedback: a<yellow> p<yellow> p<green> l<yellow> e<red>\\ \tt \\ \tt Here are some rules to keep in mind:\\ \tt 1. Please do not generate any guess feedback or clue on your own\\ \tt 2. Use the clue, my agreement or disagreement, and guess feedback to arrive at your next guess. This should be reflected in your explanation as well.\\ \tt \\ \tt Let's begin with your first guess.\\ \tt \\ \tt clue:sandwich decoration\\ \tt  
	  } 
	   } 
	   } 
	 & \\ 
 

    \theutterance \stepcounter{utterance}  

    & \multicolumn{4}{p{0.6\linewidth}}{\cellcolor[rgb]{0.8,1,0.9}{%
	\makecell[{{p{\linewidth}}}]{% 
	  \tt {\tiny [A$\rangle$GM]}  
	 Certainly! I'm ready to play Wordle based on the given clue. Let's start with my first guess.\\ \tt \\ \tt guess: olive\\ \tt explanation: An olive is a common decoration or garnish for sandwiches, especially in fancy presentations or on cocktail picks. It's a five-letter word that fits the clue of being a sandwich decoration. 
	  } 
	   } 
	   } 
	 & & \\ 
 

    \theutterance \stepcounter{utterance}  

    & & & \multicolumn{2}{p{0.3\linewidth}}{\cellcolor[rgb]{0.95,0.95,0.95}{%
	\makecell[{{p{\linewidth}}}]{% 
	  \tt {\tiny [GM$|$GM]}  
	 {'guess': 'olive', 'explanation': "An olive is a common decoration or garnish for sandwiches, especially in fancy presentations or on cocktail picks. It's a five-letter word that fits the clue of being a sandwich decoration."} 
	  } 
	   } 
	   } 
	 & & \\ 
 

    \theutterance \stepcounter{utterance}  

    & & \multicolumn{4}{p{0.6\linewidth}}{\cellcolor[rgb]{0.9,0.9,0.9}{%
	\makecell[{{p{\linewidth}}}]{% 
	  \tt {\tiny [GM$\rangle$B]}  
	 I need your assistance with a word game in which we need to find a 5-letter word using a clue, a guess and an explanation for the guess.\\ \tt \\ \tt For example, if the clue is "snow white" and the guess is "apple", you will get the information like this:\\ \tt clue:snow white\\ \tt guess:apple\\ \tt explanation:the fairy tale Snow White involves the girl eating a poisoned apple, which has 5 letters.\\ \tt \\ \tt Your task is to either agree or disagree with my guess based on the given clue.\\ \tt \\ \tt If you disagree, I will make another guess, and if you agree, I will proceed in the game using the same guess.\\ \tt \\ \tt Initially, guess feedback is not available, but as the game proceeds, it becomes available. The guess feedback is not for current guess, but rather an indication of what letters can be present in the current guess. A letter highlighted in green indicates that it is a correct letter in the correct position, while yellow indicates that it is a correct letter in the wrong position, and red indicates a wrong letter. At this point, you should use all the provided information, including the clue, guess, explanation, and guess feedback, to decide whether the given guess matches the clue, and also stick to guess feedback or not.\\ \tt \\ \tt Please respond in lowercase letters and stick to this format:\\ \tt agreement:yes or no\\ \tt explanation:your reason for doing so\\ \tt \\ \tt Please note that you must not guess any word based on the riddle clue and stick to the given format while responding.\\ \tt \\ \tt Let's begin!\\ \tt \\ \tt clue:sandwich decoration\\ \tt guess:olive\\ \tt explanation:An olive is a common decoration or garnish for sandwiches, especially in fancy presentations or on cocktail picks. It's a five-letter word that fits the clue of being a sandwich decoration.\\ \tt  
	  } 
	   } 
	   } 
	 & \\ 
 

    \theutterance \stepcounter{utterance}  

    & & & \multicolumn{4}{p{0.6\linewidth}}{\cellcolor[rgb]{1,0.85,0.72}{%
	\makecell[{{p{\linewidth}}}]{% 
	  \tt {\tiny [GM$\langle$B]}  
	 agreement:yes\\ \tt explanation:your guess of "olive" is reasonable for the clue "sandwich decoration." olives are indeed commonly used as a garnish or topping for various types of sandwiches, adding both flavor and visual appeal. the word "olive" is five letters long, which fits the game's requirements. while there could be other possible answers, this guess is logical and fits well with the given clue. 
	  } 
	   } 
	   } 
	  \\ 
 

    \theutterance \stepcounter{utterance}  

    & & & \multicolumn{2}{p{0.3\linewidth}}{\cellcolor[rgb]{0.95,0.95,0.95}{%
	\makecell[{{p{\linewidth}}}]{% 
	  \tt {\tiny [GM$|$GM]}  
	 {'agreement': 'yes', 'explanation': 'your guess of "olive" is reasonable for the clue "sandwich decoration." olives are indeed commonly used as a garnish or topping for various types of sandwiches, adding both flavor and visual appeal. the word "olive" is five letters long, which fits the game\'s requirements. while there could be other possible answers, this guess is logical and fits well with the given clue.'} 
	  } 
	   } 
	   } 
	 & & \\ 
 

    \theutterance \stepcounter{utterance}  

    & & & \multicolumn{2}{p{0.3\linewidth}}{\cellcolor[rgb]{0.95,0.95,0.95}{%
	\makecell[{{p{\linewidth}}}]{% 
	  \tt {\tiny [GM$|$GM]}  
	 Critic agrees with the Guesser -- Sharing the critic's explanation with the guesser 
	  } 
	   } 
	   } 
	 & & \\ 
 

    \theutterance \stepcounter{utterance}  

    & & \multicolumn{4}{p{0.6\linewidth}}{\cellcolor[rgb]{0.9,0.9,0.9}{%
	\makecell[{{p{\linewidth}}}]{% 
	  \tt {\tiny [A$\langle$GM]}  
	 clue:sandwich decoration\\ \tt guess_agreement:yes\\ \tt agreement_explanation:your guess of "olive" is reasonable for the clue "sandwich decoration." olives are indeed commonly used as a garnish or topping for various types of sandwiches, adding both flavor and visual appeal. the word "olive" is five letters long, which fits the game's requirements. while there could be other possible answers, this guess is logical and fits well with the given clue.\\ \tt  
	  } 
	   } 
	   } 
	 & \\ 
 

    \theutterance \stepcounter{utterance}  

    & \multicolumn{4}{p{0.6\linewidth}}{\cellcolor[rgb]{0.8,1,0.9}{%
	\makecell[{{p{\linewidth}}}]{% 
	  \tt {\tiny [A$\rangle$GM]}  
	 Thank you for agreeing with my guess. I'll maintain my choice of "olive" for this round.\\ \tt \\ \tt guess: olive\\ \tt explanation: As agreed, "olive" is a fitting guess for the clue "sandwich decoration." Olives are frequently used to garnish or top various sandwiches, enhancing both taste and presentation. The word meets the five-letter requirement for Wordle, and aligns well with the given clue. 
	  } 
	   } 
	   } 
	 & & \\ 
 

    \theutterance \stepcounter{utterance}  

    & & & \multicolumn{2}{p{0.3\linewidth}}{\cellcolor[rgb]{0.95,0.95,0.95}{%
	\makecell[{{p{\linewidth}}}]{% 
	  \tt {\tiny [GM$|$GM]}  
	 {'guess': 'olive', 'explanation': 'As agreed, "olive" is a fitting guess for the clue "sandwich decoration." Olives are frequently used to garnish or top various sandwiches, enhancing both taste and presentation. The word meets the five-letter requirement for Wordle, and aligns well with the given clue.'} 
	  } 
	   } 
	   } 
	 & & \\ 
 

    \theutterance \stepcounter{utterance}  

    & & & \multicolumn{2}{p{0.3\linewidth}}{\cellcolor[rgb]{0.95,0.95,0.95}{%
	\makecell[{{p{\linewidth}}}]{% 
	  \tt {\tiny [GM$|$GM]}  
	 No change in player1's guess\\ \tt guess_before_critic_opinion: olive\\ \tt                                                 critic_agreement: yes\\ \tt guess_after_critic_opinion: olive\\ \tt                                                 Proceeding with guess validation 
	  } 
	   } 
	   } 
	 & & \\ 
 

    \theutterance \stepcounter{utterance}  

    & & & \multicolumn{2}{p{0.3\linewidth}}{\cellcolor[rgb]{0.95,0.95,0.95}{%
	\makecell[{{p{\linewidth}}}]{% 
	  \tt {\tiny [GM$|$GM]}  
	 attempts: 1\\ \tt target_word = caper\\ \tt target_word_clue = sandwich decoration\\ \tt guess: olive\\ \tt guess_feedback: o<red> l<red> i<red> v<red> e<yellow> 
	  } 
	   } 
	   } 
	 & & \\ 
 

    \theutterance \stepcounter{utterance}  

    & & \multicolumn{4}{p{0.6\linewidth}}{\cellcolor[rgb]{0.9,0.9,0.9}{%
	\makecell[{{p{\linewidth}}}]{% 
	  \tt {\tiny [A$\langle$GM]}  
	 guess_feedback: o<red> l<red> i<red> v<red> e<yellow> 
	  } 
	   } 
	   } 
	 & \\ 
 

    \theutterance \stepcounter{utterance}  

    & \multicolumn{4}{p{0.6\linewidth}}{\cellcolor[rgb]{0.8,1,0.9}{%
	\makecell[{{p{\linewidth}}}]{% 
	  \tt {\tiny [A$\rangle$GM]}  
	 guess: slice\\ \tt explanation: Based on the previous feedback, we know that "e" is in the word but not in the last position, and all other letters from "olive" are incorrect. "Slice" is another common sandwich-related term that fits the clue "sandwich decoration." A slice of tomato, cucumber, or cheese can be considered a decoration for a sandwich. This word keeps the "e" but moves it to a different position and introduces new letters. 
	  } 
	   } 
	   } 
	 & & \\ 
 

    \theutterance \stepcounter{utterance}  

    & & & \multicolumn{2}{p{0.3\linewidth}}{\cellcolor[rgb]{0.95,0.95,0.95}{%
	\makecell[{{p{\linewidth}}}]{% 
	  \tt {\tiny [GM$|$GM]}  
	 {'guess': 'slice', 'explanation': 'Based on the previous feedback, we know that "e" is in the word but not in the last position, and all other letters from "olive" are incorrect. "Slice" is another common sandwich-related term that fits the clue "sandwich decoration." A slice of tomato, cucumber, or cheese can be considered a decoration for a sandwich. This word keeps the "e" but moves it to a different position and introduces new letters.'} 
	  } 
	   } 
	   } 
	 & & \\ 
 

    \theutterance \stepcounter{utterance}  

    & & \multicolumn{4}{p{0.6\linewidth}}{\cellcolor[rgb]{0.9,0.9,0.9}{%
	\makecell[{{p{\linewidth}}}]{% 
	  \tt {\tiny [GM$\rangle$B]}  
	 clue:sandwich decoration\\ \tt guess:slice\\ \tt explanation:Based on the previous feedback, we know that "e" is in the word but not in the last position, and all other letters from "olive" are incorrect. "Slice" is another common sandwich-related term that fits the clue "sandwich decoration." A slice of tomato, cucumber, or cheese can be considered a decoration for a sandwich. This word keeps the "e" but moves it to a different position and introduces new letters.\\ \tt guess_feedback: o<red> l<red> i<red> v<red> e<yellow> 
	  } 
	   } 
	   } 
	 & \\ 
 

    \theutterance \stepcounter{utterance}  

    & & & \multicolumn{4}{p{0.6\linewidth}}{\cellcolor[rgb]{1,0.85,0.72}{%
	\makecell[{{p{\linewidth}}}]{% 
	  \tt {\tiny [GM$\langle$B]}  
	 agreement: yes\\ \tt explanation: Your guess "slice" is a good fit for the clue "sandwich decoration" and aligns with the given feedback. It maintains the 'e' from the previous guess but moves it to a different position, as indicated by the yellow feedback. The word "slice" can indeed refer to a decorative element on a sandwich, such as a slice of tomato, cucumber, or cheese, which can enhance the sandwich's appearance. Additionally, all the other 
	  } 
	   } 
	   } 
	  \\ 
 

    \theutterance \stepcounter{utterance}  

    & & & \multicolumn{2}{p{0.3\linewidth}}{\cellcolor[rgb]{0.95,0.95,0.95}{%
	\makecell[{{p{\linewidth}}}]{% 
	  \tt {\tiny [GM$|$GM]}  
	 {'agreement': 'yes', 'explanation': 'Your guess "slice" is a good fit for the clue "sandwich decoration" and aligns with the given feedback. It maintains the \'e\' from the previous guess but moves it to a different position, as indicated by the yellow feedback. The word "slice" can indeed refer to a decorative element on a sandwich, such as a slice of tomato, cucumber, or cheese, which can enhance the sandwich\'s appearance. Additionally, all the other'} 
	  } 
	   } 
	   } 
	 & & \\ 
 

    \theutterance \stepcounter{utterance}  

    & & & \multicolumn{2}{p{0.3\linewidth}}{\cellcolor[rgb]{0.95,0.95,0.95}{%
	\makecell[{{p{\linewidth}}}]{% 
	  \tt {\tiny [GM$|$GM]}  
	 Critic agrees with the Guesser -- Sharing the critic's explanation with the guesser 
	  } 
	   } 
	   } 
	 & & \\ 
 

    \theutterance \stepcounter{utterance}  

    & & \multicolumn{4}{p{0.6\linewidth}}{\cellcolor[rgb]{0.9,0.9,0.9}{%
	\makecell[{{p{\linewidth}}}]{% 
	  \tt {\tiny [A$\langle$GM]}  
	 clue:sandwich decoration\\ \tt guess_agreement:yes\\ \tt agreement_explanation:Your guess "slice" is a good fit for the clue "sandwich decoration" and aligns with the given feedback. It maintains the 'e' from the previous guess but moves it to a different position, as indicated by the yellow feedback. The word "slice" can indeed refer to a decorative element on a sandwich, such as a slice of tomato, cucumber, or cheese, which can enhance the sandwich's appearance. Additionally, all the other\\ \tt  
	  } 
	   } 
	   } 
	 & \\ 
 

    \theutterance \stepcounter{utterance}  

    & \multicolumn{4}{p{0.6\linewidth}}{\cellcolor[rgb]{0.8,1,0.9}{%
	\makecell[{{p{\linewidth}}}]{% 
	  \tt {\tiny [A$\rangle$GM]}  
	 guess:slice\\ \tt explanation:Thank you for agreeing with the guess "slice." As we've established, "slice" fits well with the clue "sandwich decoration" and adheres to the feedback from the previous guess. It keeps the 'e' but repositions it, and introduces new letters that could potentially match the target word. A slice of various ingredients can indeed serve as a decorative element on a sandwich, enhancing both its visual appeal and taste. 
	  } 
	   } 
	   } 
	 & & \\ 
 

    \theutterance \stepcounter{utterance}  

    & & & \multicolumn{2}{p{0.3\linewidth}}{\cellcolor[rgb]{0.95,0.95,0.95}{%
	\makecell[{{p{\linewidth}}}]{% 
	  \tt {\tiny [GM$|$GM]}  
	 {'guess': 'slice', 'explanation': 'Thank you for agreeing with the guess "slice." As we\'ve established, "slice" fits well with the clue "sandwich decoration" and adheres to the feedback from the previous guess. It keeps the \'e\' but repositions it, and introduces new letters that could potentially match the target word. A slice of various ingredients can indeed serve as a decorative element on a sandwich, enhancing both its visual appeal and taste.'} 
	  } 
	   } 
	   } 
	 & & \\ 
 

    \theutterance \stepcounter{utterance}  

    & & & \multicolumn{2}{p{0.3\linewidth}}{\cellcolor[rgb]{0.95,0.95,0.95}{%
	\makecell[{{p{\linewidth}}}]{% 
	  \tt {\tiny [GM$|$GM]}  
	 No change in player1's guess\\ \tt guess_before_critic_opinion: slice\\ \tt                                                 critic_agreement: yes\\ \tt guess_after_critic_opinion: slice\\ \tt                                                 Proceeding with guess validation 
	  } 
	   } 
	   } 
	 & & \\ 
 

    \theutterance \stepcounter{utterance}  

    & & & \multicolumn{2}{p{0.3\linewidth}}{\cellcolor[rgb]{0.95,0.95,0.95}{%
	\makecell[{{p{\linewidth}}}]{% 
	  \tt {\tiny [GM$|$GM]}  
	 attempts: 2\\ \tt target_word = caper\\ \tt target_word_clue = sandwich decoration\\ \tt guess: slice\\ \tt guess_feedback: s<red> l<red> i<red> c<yellow> e<yellow> 
	  } 
	   } 
	   } 
	 & & \\ 
 

    \theutterance \stepcounter{utterance}  

    & & \multicolumn{4}{p{0.6\linewidth}}{\cellcolor[rgb]{0.9,0.9,0.9}{%
	\makecell[{{p{\linewidth}}}]{% 
	  \tt {\tiny [A$\langle$GM]}  
	 guess_feedback: s<red> l<red> i<red> c<yellow> e<yellow> 
	  } 
	   } 
	   } 
	 & \\ 
 

    \theutterance \stepcounter{utterance}  

    & \multicolumn{4}{p{0.6\linewidth}}{\cellcolor[rgb]{0.8,1,0.9}{%
	\makecell[{{p{\linewidth}}}]{% 
	  \tt {\tiny [A$\rangle$GM]}  
	 guess:chive\\ \tt explanation:Based on the previous feedback, we know that 'c' and 'e' are in the word but not in their current positions. "Chive" is another word that fits the clue "sandwich decoration" and incorporates these known correct letters. Chives are often used as a garnish or decoration on sandwiches, adding both flavor and visual appeal with their green color. This guess keeps 'c' and 'e 
	  } 
	   } 
	   } 
	 & & \\ 
 

    \theutterance \stepcounter{utterance}  

    & & & \multicolumn{2}{p{0.3\linewidth}}{\cellcolor[rgb]{0.95,0.95,0.95}{%
	\makecell[{{p{\linewidth}}}]{% 
	  \tt {\tiny [GM$|$GM]}  
	 {'guess': 'chive', 'explanation': 'Based on the previous feedback, we know that \'c\' and \'e\' are in the word but not in their current positions. "Chive" is another word that fits the clue "sandwich decoration" and incorporates these known correct letters. Chives are often used as a garnish or decoration on sandwiches, adding both flavor and visual appeal with their green color. This guess keeps \'c\' and \'e'} 
	  } 
	   } 
	   } 
	 & & \\ 
 

    \theutterance \stepcounter{utterance}  

    & & \multicolumn{4}{p{0.6\linewidth}}{\cellcolor[rgb]{0.9,0.9,0.9}{%
	\makecell[{{p{\linewidth}}}]{% 
	  \tt {\tiny [GM$\rangle$B]}  
	 clue:sandwich decoration\\ \tt guess:chive\\ \tt explanation:Based on the previous feedback, we know that 'c' and 'e' are in the word but not in their current positions. "Chive" is another word that fits the clue "sandwich decoration" and incorporates these known correct letters. Chives are often used as a garnish or decoration on sandwiches, adding both flavor and visual appeal with their green color. This guess keeps 'c' and 'e\\ \tt guess_feedback: s<red> l<red> i<red> c<yellow> e<yellow> 
	  } 
	   } 
	   } 
	 & \\ 
 

    \theutterance \stepcounter{utterance}  

    & & & \multicolumn{4}{p{0.6\linewidth}}{\cellcolor[rgb]{1,0.85,0.72}{%
	\makecell[{{p{\linewidth}}}]{% 
	  \tt {\tiny [GM$\langle$B]}  
	 agreement: yes\\ \tt explanation: Your guess "chive" is a good fit for the clue "sandwich decoration" and aligns with the available guess feedback. Chives are indeed often used as a garnish or decoration on sandwiches, adding both flavor and visual appeal. The word "chive" is five letters long and incorporates the yellow 'c' and 'e' from the previous feedback, placing them in different positions. This guess also introduces new letters that 
	  } 
	   } 
	   } 
	  \\ 
 

    \theutterance \stepcounter{utterance}  

    & & & \multicolumn{2}{p{0.3\linewidth}}{\cellcolor[rgb]{0.95,0.95,0.95}{%
	\makecell[{{p{\linewidth}}}]{% 
	  \tt {\tiny [GM$|$GM]}  
	 {'agreement': 'yes', 'explanation': 'Your guess "chive" is a good fit for the clue "sandwich decoration" and aligns with the available guess feedback. Chives are indeed often used as a garnish or decoration on sandwiches, adding both flavor and visual appeal. The word "chive" is five letters long and incorporates the yellow \'c\' and \'e\' from the previous feedback, placing them in different positions. This guess also introduces new letters that'} 
	  } 
	   } 
	   } 
	 & & \\ 
 

    \theutterance \stepcounter{utterance}  

    & & & \multicolumn{2}{p{0.3\linewidth}}{\cellcolor[rgb]{0.95,0.95,0.95}{%
	\makecell[{{p{\linewidth}}}]{% 
	  \tt {\tiny [GM$|$GM]}  
	 Critic agrees with the Guesser -- Sharing the critic's explanation with the guesser 
	  } 
	   } 
	   } 
	 & & \\ 
 

    \theutterance \stepcounter{utterance}  

    & & \multicolumn{4}{p{0.6\linewidth}}{\cellcolor[rgb]{0.9,0.9,0.9}{%
	\makecell[{{p{\linewidth}}}]{% 
	  \tt {\tiny [A$\langle$GM]}  
	 clue:sandwich decoration\\ \tt guess_agreement:yes\\ \tt agreement_explanation:Your guess "chive" is a good fit for the clue "sandwich decoration" and aligns with the available guess feedback. Chives are indeed often used as a garnish or decoration on sandwiches, adding both flavor and visual appeal. The word "chive" is five letters long and incorporates the yellow 'c' and 'e' from the previous feedback, placing them in different positions. This guess also introduces new letters that\\ \tt  
	  } 
	   } 
	   } 
	 & \\ 
 

    \theutterance \stepcounter{utterance}  

    & \multicolumn{4}{p{0.6\linewidth}}{\cellcolor[rgb]{0.8,1,0.9}{%
	\makecell[{{p{\linewidth}}}]{% 
	  \tt {\tiny [A$\rangle$GM]}  
	 guess:chive\\ \tt explanation:Thank you for agreeing with the guess "chive." As confirmed, "chive" fits well with the clue "sandwich decoration" and adheres to the feedback from the previous guesses. It maintains the 'c' and 'e' from the last feedback but repositions them, which is in line with their yellow status. Chives are commonly used as a garnish on sandwiches, adding both visual appeal and flavor, 
	  } 
	   } 
	   } 
	 & & \\ 
 

    \theutterance \stepcounter{utterance}  

    & & & \multicolumn{2}{p{0.3\linewidth}}{\cellcolor[rgb]{0.95,0.95,0.95}{%
	\makecell[{{p{\linewidth}}}]{% 
	  \tt {\tiny [GM$|$GM]}  
	 {'guess': 'chive', 'explanation': 'Thank you for agreeing with the guess "chive." As confirmed, "chive" fits well with the clue "sandwich decoration" and adheres to the feedback from the previous guesses. It maintains the \'c\' and \'e\' from the last feedback but repositions them, which is in line with their yellow status. Chives are commonly used as a garnish on sandwiches, adding both visual appeal and flavor,'} 
	  } 
	   } 
	   } 
	 & & \\ 
 

    \theutterance \stepcounter{utterance}  

    & & & \multicolumn{2}{p{0.3\linewidth}}{\cellcolor[rgb]{0.95,0.95,0.95}{%
	\makecell[{{p{\linewidth}}}]{% 
	  \tt {\tiny [GM$|$GM]}  
	 No change in player1's guess\\ \tt guess_before_critic_opinion: chive\\ \tt                                                 critic_agreement: yes\\ \tt guess_after_critic_opinion: chive\\ \tt                                                 Proceeding with guess validation 
	  } 
	   } 
	   } 
	 & & \\ 
 

    \theutterance \stepcounter{utterance}  

    & & & \multicolumn{2}{p{0.3\linewidth}}{\cellcolor[rgb]{0.95,0.95,0.95}{%
	\makecell[{{p{\linewidth}}}]{% 
	  \tt {\tiny [GM$|$GM]}  
	 attempts: 3\\ \tt target_word = caper\\ \tt target_word_clue = sandwich decoration\\ \tt guess: chive\\ \tt guess_feedback: c<green> h<red> i<red> v<red> e<yellow> 
	  } 
	   } 
	   } 
	 & & \\ 
 

    \theutterance \stepcounter{utterance}  

    & & \multicolumn{4}{p{0.6\linewidth}}{\cellcolor[rgb]{0.9,0.9,0.9}{%
	\makecell[{{p{\linewidth}}}]{% 
	  \tt {\tiny [A$\langle$GM]}  
	 guess_feedback: c<green> h<red> i<red> v<red> e<yellow> 
	  } 
	   } 
	   } 
	 & \\ 
 

    \theutterance \stepcounter{utterance}  

    & \multicolumn{4}{p{0.6\linewidth}}{\cellcolor[rgb]{0.8,1,0.9}{%
	\makecell[{{p{\linewidth}}}]{% 
	  \tt {\tiny [A$\rangle$GM]}  
	 guess:cress\\ \tt explanation:Based on the latest feedback, we know that 'c' is in the correct position at the start of the word, and 'e' is in the word but not in the last position. "Cress" fits these criteria and also aligns with the clue "sandwich decoration." Cress, particularly watercress, is often used as a garnish or topping on sandwiches, adding a fresh, peppery flavor 
	  } 
	   } 
	   } 
	 & & \\ 
 

    \theutterance \stepcounter{utterance}  

    & & & \multicolumn{2}{p{0.3\linewidth}}{\cellcolor[rgb]{0.95,0.95,0.95}{%
	\makecell[{{p{\linewidth}}}]{% 
	  \tt {\tiny [GM$|$GM]}  
	 {'guess': 'cress', 'explanation': 'Based on the latest feedback, we know that \'c\' is in the correct position at the start of the word, and \'e\' is in the word but not in the last position. "Cress" fits these criteria and also aligns with the clue "sandwich decoration." Cress, particularly watercress, is often used as a garnish or topping on sandwiches, adding a fresh, peppery flavor'} 
	  } 
	   } 
	   } 
	 & & \\ 
 

    \theutterance \stepcounter{utterance}  

    & & \multicolumn{4}{p{0.6\linewidth}}{\cellcolor[rgb]{0.9,0.9,0.9}{%
	\makecell[{{p{\linewidth}}}]{% 
	  \tt {\tiny [GM$\rangle$B]}  
	 clue:sandwich decoration\\ \tt guess:cress\\ \tt explanation:Based on the latest feedback, we know that 'c' is in the correct position at the start of the word, and 'e' is in the word but not in the last position. "Cress" fits these criteria and also aligns with the clue "sandwich decoration." Cress, particularly watercress, is often used as a garnish or topping on sandwiches, adding a fresh, peppery flavor\\ \tt guess_feedback: c<green> h<red> i<red> v<red> e<yellow> 
	  } 
	   } 
	   } 
	 & \\ 
 

    \theutterance \stepcounter{utterance}  

    & & & \multicolumn{4}{p{0.6\linewidth}}{\cellcolor[rgb]{1,0.85,0.72}{%
	\makecell[{{p{\linewidth}}}]{% 
	  \tt {\tiny [GM$\langle$B]}  
	 agreement: yes\\ \tt explanation: Your guess "cress" is a good fit for the clue "sandwich decoration" and aligns well with the given feedback. Cress is indeed a common sandwich topping or garnish, often used to add flavor and visual appeal. The word "cress" matches the feedback perfectly: it starts with 'c' (green), contains 'e' in a position other than the last (yellow), and doesn't include any of the previously 
	  } 
	   } 
	   } 
	  \\ 
 

    \theutterance \stepcounter{utterance}  

    & & & \multicolumn{2}{p{0.3\linewidth}}{\cellcolor[rgb]{0.95,0.95,0.95}{%
	\makecell[{{p{\linewidth}}}]{% 
	  \tt {\tiny [GM$|$GM]}  
	 {'agreement': 'yes', 'explanation': 'Your guess "cress" is a good fit for the clue "sandwich decoration" and aligns well with the given feedback. Cress is indeed a common sandwich topping or garnish, often used to add flavor and visual appeal. The word "cress" matches the feedback perfectly: it starts with \'c\' (green), contains \'e\' in a position other than the last (yellow), and doesn\'t include any of the previously'} 
	  } 
	   } 
	   } 
	 & & \\ 
 

    \theutterance \stepcounter{utterance}  

    & & & \multicolumn{2}{p{0.3\linewidth}}{\cellcolor[rgb]{0.95,0.95,0.95}{%
	\makecell[{{p{\linewidth}}}]{% 
	  \tt {\tiny [GM$|$GM]}  
	 Critic agrees with the Guesser -- Sharing the critic's explanation with the guesser 
	  } 
	   } 
	   } 
	 & & \\ 
 

    \theutterance \stepcounter{utterance}  

    & & \multicolumn{4}{p{0.6\linewidth}}{\cellcolor[rgb]{0.9,0.9,0.9}{%
	\makecell[{{p{\linewidth}}}]{% 
	  \tt {\tiny [A$\langle$GM]}  
	 clue:sandwich decoration\\ \tt guess_agreement:yes\\ \tt agreement_explanation:Your guess "cress" is a good fit for the clue "sandwich decoration" and aligns well with the given feedback. Cress is indeed a common sandwich topping or garnish, often used to add flavor and visual appeal. The word "cress" matches the feedback perfectly: it starts with 'c' (green), contains 'e' in a position other than the last (yellow), and doesn't include any of the previously\\ \tt  
	  } 
	   } 
	   } 
	 & \\ 
 

    \theutterance \stepcounter{utterance}  

    & \multicolumn{4}{p{0.6\linewidth}}{\cellcolor[rgb]{0.8,1,0.9}{%
	\makecell[{{p{\linewidth}}}]{% 
	  \tt {\tiny [A$\rangle$GM]}  
	 guess:cress\\ \tt explanation:Thank you for agreeing with the guess "cress." As confirmed, "cress" aligns well with both the clue "sandwich decoration" and the previous feedback. It maintains the correct 'c' at the beginning, includes 'e' in a new position as required, and introduces new letters that haven't been ruled out. Cress, particularly watercress, is a common sandwich topping that adds both flavor and visual appeal 
	  } 
	   } 
	   } 
	 & & \\ 
 

    \theutterance \stepcounter{utterance}  

    & & & \multicolumn{2}{p{0.3\linewidth}}{\cellcolor[rgb]{0.95,0.95,0.95}{%
	\makecell[{{p{\linewidth}}}]{% 
	  \tt {\tiny [GM$|$GM]}  
	 {'guess': 'cress', 'explanation': 'Thank you for agreeing with the guess "cress." As confirmed, "cress" aligns well with both the clue "sandwich decoration" and the previous feedback. It maintains the correct \'c\' at the beginning, includes \'e\' in a new position as required, and introduces new letters that haven\'t been ruled out. Cress, particularly watercress, is a common sandwich topping that adds both flavor and visual appeal'} 
	  } 
	   } 
	   } 
	 & & \\ 
 

    \theutterance \stepcounter{utterance}  

    & & & \multicolumn{2}{p{0.3\linewidth}}{\cellcolor[rgb]{0.95,0.95,0.95}{%
	\makecell[{{p{\linewidth}}}]{% 
	  \tt {\tiny [GM$|$GM]}  
	 No change in player1's guess\\ \tt guess_before_critic_opinion: cress\\ \tt                                                 critic_agreement: yes\\ \tt guess_after_critic_opinion: cress\\ \tt                                                 Proceeding with guess validation 
	  } 
	   } 
	   } 
	 & & \\ 
 

    \theutterance \stepcounter{utterance}  

    & & & \multicolumn{2}{p{0.3\linewidth}}{\cellcolor[rgb]{0.95,0.95,0.95}{%
	\makecell[{{p{\linewidth}}}]{% 
	  \tt {\tiny [GM$|$GM]}  
	 attempts: 4\\ \tt target_word = caper\\ \tt target_word_clue = sandwich decoration\\ \tt guess: cress\\ \tt guess_feedback: c<green> r<yellow> e<yellow> s<red> s<red> 
	  } 
	   } 
	   } 
	 & & \\ 
 

    \theutterance \stepcounter{utterance}  

    & & \multicolumn{4}{p{0.6\linewidth}}{\cellcolor[rgb]{0.9,0.9,0.9}{%
	\makecell[{{p{\linewidth}}}]{% 
	  \tt {\tiny [A$\langle$GM]}  
	 guess_feedback: c<green> r<yellow> e<yellow> s<red> s<red> 
	  } 
	   } 
	   } 
	 & \\ 
 

    \theutterance \stepcounter{utterance}  

    & \multicolumn{4}{p{0.6\linewidth}}{\cellcolor[rgb]{0.8,1,0.9}{%
	\makecell[{{p{\linewidth}}}]{% 
	  \tt {\tiny [A$\rangle$GM]}  
	 guess:cream\\ \tt explanation:Based on the latest feedback, we know that 'c' is in the correct position at the start of the word, and both 'r' and 'e' are in the word but not in their current positions. "Cream" fits these criteria and also aligns with the clue "sandwich decoration." Cream, such as cream cheese or whipped cream, can be used as a decorative element on certain sandwiches, especially in 
	  } 
	   } 
	   } 
	 & & \\ 
 

    \theutterance \stepcounter{utterance}  

    & & & \multicolumn{2}{p{0.3\linewidth}}{\cellcolor[rgb]{0.95,0.95,0.95}{%
	\makecell[{{p{\linewidth}}}]{% 
	  \tt {\tiny [GM$|$GM]}  
	 {'guess': 'cream', 'explanation': 'Based on the latest feedback, we know that \'c\' is in the correct position at the start of the word, and both \'r\' and \'e\' are in the word but not in their current positions. "Cream" fits these criteria and also aligns with the clue "sandwich decoration." Cream, such as cream cheese or whipped cream, can be used as a decorative element on certain sandwiches, especially in'} 
	  } 
	   } 
	   } 
	 & & \\ 
 

    \theutterance \stepcounter{utterance}  

    & & \multicolumn{4}{p{0.6\linewidth}}{\cellcolor[rgb]{0.9,0.9,0.9}{%
	\makecell[{{p{\linewidth}}}]{% 
	  \tt {\tiny [GM$\rangle$B]}  
	 clue:sandwich decoration\\ \tt guess:cream\\ \tt explanation:Based on the latest feedback, we know that 'c' is in the correct position at the start of the word, and both 'r' and 'e' are in the word but not in their current positions. "Cream" fits these criteria and also aligns with the clue "sandwich decoration." Cream, such as cream cheese or whipped cream, can be used as a decorative element on certain sandwiches, especially in\\ \tt guess_feedback: c<green> r<yellow> e<yellow> s<red> s<red> 
	  } 
	   } 
	   } 
	 & \\ 
 

    \theutterance \stepcounter{utterance}  

    & & & \multicolumn{4}{p{0.6\linewidth}}{\cellcolor[rgb]{1,0.85,0.72}{%
	\makecell[{{p{\linewidth}}}]{% 
	  \tt {\tiny [GM$\langle$B]}  
	 agreement: yes\\ \tt explanation: The guess "cream" aligns well with the clue "sandwich decoration" and follows the feedback from previous guesses. It starts with 'c' in the correct position, and includes 'r' and 'e' which are known to be in the word but in different positions. Cream, in various forms like cream cheese or whipped cream, can indeed be used as a decorative element on certain sandwiches, particularly in dessert 
	  } 
	   } 
	   } 
	  \\ 
 

    \theutterance \stepcounter{utterance}  

    & & & \multicolumn{2}{p{0.3\linewidth}}{\cellcolor[rgb]{0.95,0.95,0.95}{%
	\makecell[{{p{\linewidth}}}]{% 
	  \tt {\tiny [GM$|$GM]}  
	 {'agreement': 'yes', 'explanation': 'The guess "cream" aligns well with the clue "sandwich decoration" and follows the feedback from previous guesses. It starts with \'c\' in the correct position, and includes \'r\' and \'e\' which are known to be in the word but in different positions. Cream, in various forms like cream cheese or whipped cream, can indeed be used as a decorative element on certain sandwiches, particularly in dessert'} 
	  } 
	   } 
	   } 
	 & & \\ 
 

    \theutterance \stepcounter{utterance}  

    & & & \multicolumn{2}{p{0.3\linewidth}}{\cellcolor[rgb]{0.95,0.95,0.95}{%
	\makecell[{{p{\linewidth}}}]{% 
	  \tt {\tiny [GM$|$GM]}  
	 Critic agrees with the Guesser -- Sharing the critic's explanation with the guesser 
	  } 
	   } 
	   } 
	 & & \\ 
 

    \theutterance \stepcounter{utterance}  

    & & \multicolumn{4}{p{0.6\linewidth}}{\cellcolor[rgb]{0.9,0.9,0.9}{%
	\makecell[{{p{\linewidth}}}]{% 
	  \tt {\tiny [A$\langle$GM]}  
	 clue:sandwich decoration\\ \tt guess_agreement:yes\\ \tt agreement_explanation:The guess "cream" aligns well with the clue "sandwich decoration" and follows the feedback from previous guesses. It starts with 'c' in the correct position, and includes 'r' and 'e' which are known to be in the word but in different positions. Cream, in various forms like cream cheese or whipped cream, can indeed be used as a decorative element on certain sandwiches, particularly in dessert\\ \tt  
	  } 
	   } 
	   } 
	 & \\ 
 

    \theutterance \stepcounter{utterance}  

    & \multicolumn{4}{p{0.6\linewidth}}{\cellcolor[rgb]{0.8,1,0.9}{%
	\makecell[{{p{\linewidth}}}]{% 
	  \tt {\tiny [A$\rangle$GM]}  
	 guess:cream\\ \tt explanation:Thank you for agreeing with the guess "cream." As confirmed, "cream" fits well with both the clue "sandwich decoration" and the previous feedback. It maintains the correct 'c' at the beginning, includes 'r' and 'e' in new positions as required, and introduces a new letter 'a' that hasn't been ruled out. Cream, in forms such as cream cheese or whipped cream, can indeed serve 
	  } 
	   } 
	   } 
	 & & \\ 
 

    \theutterance \stepcounter{utterance}  

    & & & \multicolumn{2}{p{0.3\linewidth}}{\cellcolor[rgb]{0.95,0.95,0.95}{%
	\makecell[{{p{\linewidth}}}]{% 
	  \tt {\tiny [GM$|$GM]}  
	 {'guess': 'cream', 'explanation': 'Thank you for agreeing with the guess "cream." As confirmed, "cream" fits well with both the clue "sandwich decoration" and the previous feedback. It maintains the correct \'c\' at the beginning, includes \'r\' and \'e\' in new positions as required, and introduces a new letter \'a\' that hasn\'t been ruled out. Cream, in forms such as cream cheese or whipped cream, can indeed serve'} 
	  } 
	   } 
	   } 
	 & & \\ 
 

    \theutterance \stepcounter{utterance}  

    & & & \multicolumn{2}{p{0.3\linewidth}}{\cellcolor[rgb]{0.95,0.95,0.95}{%
	\makecell[{{p{\linewidth}}}]{% 
	  \tt {\tiny [GM$|$GM]}  
	 No change in player1's guess\\ \tt guess_before_critic_opinion: cream\\ \tt                                                 critic_agreement: yes\\ \tt guess_after_critic_opinion: cream\\ \tt                                                 Proceeding with guess validation 
	  } 
	   } 
	   } 
	 & & \\ 
 

    \theutterance \stepcounter{utterance}  

    & & & \multicolumn{2}{p{0.3\linewidth}}{\cellcolor[rgb]{0.95,0.95,0.95}{%
	\makecell[{{p{\linewidth}}}]{% 
	  \tt {\tiny [GM$|$GM]}  
	 attempts: 5\\ \tt target_word = caper\\ \tt target_word_clue = sandwich decoration\\ \tt guess: cream\\ \tt guess_feedback: c<green> r<yellow> e<yellow> a<yellow> m<red> 
	  } 
	   } 
	   } 
	 & & \\ 
 

    \theutterance \stepcounter{utterance}  

    & & \multicolumn{4}{p{0.6\linewidth}}{\cellcolor[rgb]{0.9,0.9,0.9}{%
	\makecell[{{p{\linewidth}}}]{% 
	  \tt {\tiny [A$\langle$GM]}  
	 guess_feedback: c<green> r<yellow> e<yellow> a<yellow> m<red> 
	  } 
	   } 
	   } 
	 & \\ 
 

    \theutterance \stepcounter{utterance}  

    & \multicolumn{4}{p{0.6\linewidth}}{\cellcolor[rgb]{0.8,1,0.9}{%
	\makecell[{{p{\linewidth}}}]{% 
	  \tt {\tiny [A$\rangle$GM]}  
	 guess:caper\\ \tt explanation:Based on the latest feedback, we know that 'c' is in the correct position at the start of the word, and 'r', 'e', and 'a' are in the word but not in their current positions. "Caper" fits these criteria and also aligns with the clue "sandwich decoration." Capers are often used as a garnish or topping on sandwiches, adding a distinctive tangy 
	  } 
	   } 
	   } 
	 & & \\ 
 

    \theutterance \stepcounter{utterance}  

    & & & \multicolumn{2}{p{0.3\linewidth}}{\cellcolor[rgb]{0.95,0.95,0.95}{%
	\makecell[{{p{\linewidth}}}]{% 
	  \tt {\tiny [GM$|$GM]}  
	 {'guess': 'caper', 'explanation': 'Based on the latest feedback, we know that \'c\' is in the correct position at the start of the word, and \'r\', \'e\', and \'a\' are in the word but not in their current positions. "Caper" fits these criteria and also aligns with the clue "sandwich decoration." Capers are often used as a garnish or topping on sandwiches, adding a distinctive tangy'} 
	  } 
	   } 
	   } 
	 & & \\ 
 

    \theutterance \stepcounter{utterance}  

    & & \multicolumn{4}{p{0.6\linewidth}}{\cellcolor[rgb]{0.9,0.9,0.9}{%
	\makecell[{{p{\linewidth}}}]{% 
	  \tt {\tiny [GM$\rangle$B]}  
	 clue:sandwich decoration\\ \tt guess:caper\\ \tt explanation:Based on the latest feedback, we know that 'c' is in the correct position at the start of the word, and 'r', 'e', and 'a' are in the word but not in their current positions. "Caper" fits these criteria and also aligns with the clue "sandwich decoration." Capers are often used as a garnish or topping on sandwiches, adding a distinctive tangy\\ \tt guess_feedback: c<green> r<yellow> e<yellow> a<yellow> m<red> 
	  } 
	   } 
	   } 
	 & \\ 
 

    \theutterance \stepcounter{utterance}  

    & & & \multicolumn{4}{p{0.6\linewidth}}{\cellcolor[rgb]{1,0.85,0.72}{%
	\makecell[{{p{\linewidth}}}]{% 
	  \tt {\tiny [GM$\langle$B]}  
	 agreement: yes\\ \tt explanation: Your guess "caper" aligns perfectly with both the clue and the feedback provided. Capers are indeed used as a sandwich decoration, adding a tangy flavor and visual appeal. The word matches the clue "sandwich decoration" and satisfies all the conditions from the guess feedback: 'c' is in the correct first position, and 'a', 'e', and 'r' are all present in the word but in different 
	  } 
	   } 
	   } 
	  \\ 
 

    \theutterance \stepcounter{utterance}  

    & & & \multicolumn{2}{p{0.3\linewidth}}{\cellcolor[rgb]{0.95,0.95,0.95}{%
	\makecell[{{p{\linewidth}}}]{% 
	  \tt {\tiny [GM$|$GM]}  
	 {'agreement': 'yes', 'explanation': 'Your guess "caper" aligns perfectly with both the clue and the feedback provided. Capers are indeed used as a sandwich decoration, adding a tangy flavor and visual appeal. The word matches the clue "sandwich decoration" and satisfies all the conditions from the guess feedback: \'c\' is in the correct first position, and \'a\', \'e\', and \'r\' are all present in the word but in different'} 
	  } 
	   } 
	   } 
	 & & \\ 
 

    \theutterance \stepcounter{utterance}  

    & & & \multicolumn{2}{p{0.3\linewidth}}{\cellcolor[rgb]{0.95,0.95,0.95}{%
	\makecell[{{p{\linewidth}}}]{% 
	  \tt {\tiny [GM$|$GM]}  
	 Critic agrees with the Guesser -- Sharing the critic's explanation with the guesser 
	  } 
	   } 
	   } 
	 & & \\ 
 

    \theutterance \stepcounter{utterance}  

    & & \multicolumn{4}{p{0.6\linewidth}}{\cellcolor[rgb]{0.9,0.9,0.9}{%
	\makecell[{{p{\linewidth}}}]{% 
	  \tt {\tiny [A$\langle$GM]}  
	 clue:sandwich decoration\\ \tt guess_agreement:yes\\ \tt agreement_explanation:Your guess "caper" aligns perfectly with both the clue and the feedback provided. Capers are indeed used as a sandwich decoration, adding a tangy flavor and visual appeal. The word matches the clue "sandwich decoration" and satisfies all the conditions from the guess feedback: 'c' is in the correct first position, and 'a', 'e', and 'r' are all present in the word but in different\\ \tt  
	  } 
	   } 
	   } 
	 & \\ 
 

    \theutterance \stepcounter{utterance}  

    & \multicolumn{4}{p{0.6\linewidth}}{\cellcolor[rgb]{0.8,1,0.9}{%
	\makecell[{{p{\linewidth}}}]{% 
	  \tt {\tiny [A$\rangle$GM]}  
	 guess:caper\\ \tt explanation:Thank you for agreeing with the guess "caper." As confirmed, "caper" aligns excellently with both the clue "sandwich decoration" and the previous feedback. It maintains the correct 'c' at the beginning and includes 'a', 'e', and 'r' in new positions as required by the yellow feedback. Capers are indeed a common sandwich decoration, adding a distinctive tangy flavor and visual interest to 
	  } 
	   } 
	   } 
	 & & \\ 
 

    \theutterance \stepcounter{utterance}  

    & & & \multicolumn{2}{p{0.3\linewidth}}{\cellcolor[rgb]{0.95,0.95,0.95}{%
	\makecell[{{p{\linewidth}}}]{% 
	  \tt {\tiny [GM$|$GM]}  
	 {'guess': 'caper', 'explanation': 'Thank you for agreeing with the guess "caper." As confirmed, "caper" aligns excellently with both the clue "sandwich decoration" and the previous feedback. It maintains the correct \'c\' at the beginning and includes \'a\', \'e\', and \'r\' in new positions as required by the yellow feedback. Capers are indeed a common sandwich decoration, adding a distinctive tangy flavor and visual interest to'} 
	  } 
	   } 
	   } 
	 & & \\ 
 

    \theutterance \stepcounter{utterance}  

    & & & \multicolumn{2}{p{0.3\linewidth}}{\cellcolor[rgb]{0.95,0.95,0.95}{%
	\makecell[{{p{\linewidth}}}]{% 
	  \tt {\tiny [GM$|$GM]}  
	 No change in player1's guess\\ \tt guess_before_critic_opinion: caper\\ \tt                                                 critic_agreement: yes\\ \tt guess_after_critic_opinion: caper\\ \tt                                                 Proceeding with guess validation 
	  } 
	   } 
	   } 
	 & & \\ 
 

    \theutterance \stepcounter{utterance}  

    & & & \multicolumn{2}{p{0.3\linewidth}}{\cellcolor[rgb]{0.95,0.95,0.95}{%
	\makecell[{{p{\linewidth}}}]{% 
	  \tt {\tiny [GM$|$GM]}  
	 attempts: 6\\ \tt target_word = caper\\ \tt target_word_clue = sandwich decoration\\ \tt guess: caper\\ \tt guess_feedback: c<green> a<green> p<green> e<green> r<green> 
	  } 
	   } 
	   } 
	 & & \\ 
 

    \theutterance \stepcounter{utterance}  

    & & & \multicolumn{2}{p{0.3\linewidth}}{\cellcolor[rgb]{0.95,0.95,0.95}{%
	\makecell[{{p{\linewidth}}}]{% 
	  \tt {\tiny [GM$|$GM]}  
	 game_result = WIN 
	  } 
	   } 
	   } 
	 & & \\ 
 

\end{supertabular}
}

\end{document}

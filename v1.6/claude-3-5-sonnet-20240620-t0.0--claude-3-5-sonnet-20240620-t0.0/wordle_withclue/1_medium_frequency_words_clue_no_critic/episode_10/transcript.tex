
\documentclass{article}
\usepackage{colortbl}
\usepackage{makecell}
\usepackage{multirow}
\usepackage{supertabular}

\begin{document}

\newcounter{utterance}

\twocolumn

{ \footnotesize  \setcounter{utterance}{1}
\setlength{\tabcolsep}{0pt}
\begin{supertabular}{c@{$\;$}|p{.15\linewidth}@{}p{.15\linewidth}p{.15\linewidth}p{.15\linewidth}p{.15\linewidth}p{.15\linewidth}}

    \# & $\;$A & \multicolumn{4}{c}{Game Master} & $\;\:$B\\
    \hline 

    \theutterance \stepcounter{utterance}  

    & & \multicolumn{4}{p{0.6\linewidth}}{\cellcolor[rgb]{0.9,0.9,0.9}{%
	\makecell[{{p{\linewidth}}}]{% 
	  \tt {\tiny [A$\langle$GM]}  
	 You are a language wizard who likes to guess words by using the given rules.\\ \tt \\ \tt Welcome to Wordle! You have six attempts to guess the target word, a valid English word of five lowercase letters (a-z). Please use the tags "guess:" and "explanation:" to provide a concise explanation for each guess.\\ \tt \\ \tt  To help you make an informed guess, you will receive a clue for the word, such as\\ \tt clue: snowy white.\\ \tt \\ \tt Here is an example guess based on the clue:\\ \tt guess: apple\\ \tt explanation: In the fairy tail Snow White, the girl is killed because she eats a poisoned apple. And the word apple has 5 letters.\\ \tt \\ \tt After each guess, your answer will be validated, and you will receive feedback indicating which letters are correct (green), which letters are correct but in the wrong position (yellow), and which letters are incorrect (red). This feedback can be useful in determining which letters to include or exclude in your next guess.\\ \tt \\ \tt For example, the feedback for "apple" might be:\\ \tt guess_feedback: a<yellow> p<yellow> p<green> l<yellow> e<red>\\ \tt \\ \tt The explanation should contain details about how both the guess_feedback and clue work together in tandem to formulate a new guess.\\ \tt \\ \tt Let's begin with your first guess.\\ \tt clue:sandwich decoration\\ \tt  
	  } 
	   } 
	   } 
	 & \\ 
 

    \theutterance \stepcounter{utterance}  

    & \multicolumn{4}{p{0.6\linewidth}}{\cellcolor[rgb]{0.8,1,0.9}{%
	\makecell[{{p{\linewidth}}}]{% 
	  \tt {\tiny [A$\rangle$GM]}  
	 Understood. I'll start with my first guess based on the clue provided.\\ \tt \\ \tt guess: olive\\ \tt explanation: An olive is a common decoration or topping for sandwiches, especially in deli-style establishments. It's often used as a garnish or added for flavor. The word "olive" also fits the five-letter requirement for Wordle. 
	  } 
	   } 
	   } 
	 & & \\ 
 

    \theutterance \stepcounter{utterance}  

    & & & \multicolumn{2}{p{0.3\linewidth}}{\cellcolor[rgb]{0.95,0.95,0.95}{%
	\makecell[{{p{\linewidth}}}]{% 
	  \tt {\tiny [GM$|$GM]}  
	 {'guess': 'olive', 'explanation': 'An olive is a common decoration or topping for sandwiches, especially in deli-style establishments. It\'s often used as a garnish or added for flavor. The word "olive" also fits the five-letter requirement for Wordle.'} 
	  } 
	   } 
	   } 
	 & & \\ 
 

    \theutterance \stepcounter{utterance}  

    & & & \multicolumn{2}{p{0.3\linewidth}}{\cellcolor[rgb]{0.95,0.95,0.95}{%
	\makecell[{{p{\linewidth}}}]{% 
	  \tt {\tiny [GM$|$GM]}  
	 attempts: 1\\ \tt target_word = caper\\ \tt target_word_clue = sandwich decoration\\ \tt guess: olive\\ \tt guess_feedback: o<red> l<red> i<red> v<red> e<yellow> 
	  } 
	   } 
	   } 
	 & & \\ 
 

    \theutterance \stepcounter{utterance}  

    & & \multicolumn{4}{p{0.6\linewidth}}{\cellcolor[rgb]{0.9,0.9,0.9}{%
	\makecell[{{p{\linewidth}}}]{% 
	  \tt {\tiny [A$\langle$GM]}  
	 guess_feedback: o<red> l<red> i<red> v<red> e<yellow> 
	  } 
	   } 
	   } 
	 & \\ 
 

    \theutterance \stepcounter{utterance}  

    & \multicolumn{4}{p{0.6\linewidth}}{\cellcolor[rgb]{0.8,1,0.9}{%
	\makecell[{{p{\linewidth}}}]{% 
	  \tt {\tiny [A$\rangle$GM]}  
	 Thank you for the feedback. Let's analyze this information and make a new guess.\\ \tt \\ \tt guess: slice\\ \tt explanation: Based on the previous feedback, we know that the letter 'e' is in the word but not in the last position. The word "slice" incorporates this information by placing 'e' in a different position. Additionally, "slice" relates to the clue "sandwich decoration" as sandwiches are often sliced diagonally or into smaller portions for 
	  } 
	   } 
	   } 
	 & & \\ 
 

    \theutterance \stepcounter{utterance}  

    & & & \multicolumn{2}{p{0.3\linewidth}}{\cellcolor[rgb]{0.95,0.95,0.95}{%
	\makecell[{{p{\linewidth}}}]{% 
	  \tt {\tiny [GM$|$GM]}  
	 {'guess': 'slice', 'explanation': 'Based on the previous feedback, we know that the letter \'e\' is in the word but not in the last position. The word "slice" incorporates this information by placing \'e\' in a different position. Additionally, "slice" relates to the clue "sandwich decoration" as sandwiches are often sliced diagonally or into smaller portions for'} 
	  } 
	   } 
	   } 
	 & & \\ 
 

    \theutterance \stepcounter{utterance}  

    & & & \multicolumn{2}{p{0.3\linewidth}}{\cellcolor[rgb]{0.95,0.95,0.95}{%
	\makecell[{{p{\linewidth}}}]{% 
	  \tt {\tiny [GM$|$GM]}  
	 attempts: 2\\ \tt target_word = caper\\ \tt target_word_clue = sandwich decoration\\ \tt guess: slice\\ \tt guess_feedback: s<red> l<red> i<red> c<yellow> e<yellow> 
	  } 
	   } 
	   } 
	 & & \\ 
 

    \theutterance \stepcounter{utterance}  

    & & \multicolumn{4}{p{0.6\linewidth}}{\cellcolor[rgb]{0.9,0.9,0.9}{%
	\makecell[{{p{\linewidth}}}]{% 
	  \tt {\tiny [A$\langle$GM]}  
	 guess_feedback: s<red> l<red> i<red> c<yellow> e<yellow> 
	  } 
	   } 
	   } 
	 & \\ 
 

    \theutterance \stepcounter{utterance}  

    & \multicolumn{4}{p{0.6\linewidth}}{\cellcolor[rgb]{0.8,1,0.9}{%
	\makecell[{{p{\linewidth}}}]{% 
	  \tt {\tiny [A$\rangle$GM]}  
	 guess:cheer\\ \tt explanation: From the previous feedback, we know that 'c' and 'e' are in the word but not in their current positions. "Cheer" incorporates both these letters in new positions. While it doesn't directly relate to "sandwich decoration," it could be interpreted as something that might accompany a sandwich (e.g., "cheers" during a picnic or celebration where sandwiches are served). The word maintains the five- 
	  } 
	   } 
	   } 
	 & & \\ 
 

    \theutterance \stepcounter{utterance}  

    & & & \multicolumn{2}{p{0.3\linewidth}}{\cellcolor[rgb]{0.95,0.95,0.95}{%
	\makecell[{{p{\linewidth}}}]{% 
	  \tt {\tiny [GM$|$GM]}  
	 {'guess': 'cheer', 'explanation': 'From the previous feedback, we know that \'c\' and \'e\' are in the word but not in their current positions. "Cheer" incorporates both these letters in new positions. While it doesn\'t directly relate to "sandwich decoration," it could be interpreted as something that might accompany a sandwich (e.g., "cheers" during a picnic or celebration where sandwiches are served). The word maintains the five-'} 
	  } 
	   } 
	   } 
	 & & \\ 
 

    \theutterance \stepcounter{utterance}  

    & & & \multicolumn{2}{p{0.3\linewidth}}{\cellcolor[rgb]{0.95,0.95,0.95}{%
	\makecell[{{p{\linewidth}}}]{% 
	  \tt {\tiny [GM$|$GM]}  
	 attempts: 3\\ \tt target_word = caper\\ \tt target_word_clue = sandwich decoration\\ \tt guess: cheer\\ \tt guess_feedback: c<green> h<red> e<red> e<green> r<green> 
	  } 
	   } 
	   } 
	 & & \\ 
 

    \theutterance \stepcounter{utterance}  

    & & \multicolumn{4}{p{0.6\linewidth}}{\cellcolor[rgb]{0.9,0.9,0.9}{%
	\makecell[{{p{\linewidth}}}]{% 
	  \tt {\tiny [A$\langle$GM]}  
	 guess_feedback: c<green> h<red> e<red> e<green> r<green> 
	  } 
	   } 
	   } 
	 & \\ 
 

    \theutterance \stepcounter{utterance}  

    & \multicolumn{4}{p{0.6\linewidth}}{\cellcolor[rgb]{0.8,1,0.9}{%
	\makecell[{{p{\linewidth}}}]{% 
	  \tt {\tiny [A$\rangle$GM]}  
	 guess:caper\\ \tt explanation:Based on the feedback, we now know that the word starts with 'c' and ends with 'er'. The letter 'e' is not in the second or third position, so it must be in the fourth position. "Caper" fits this pattern perfectly. Moreover, capers are often used as a sandwich decoration or topping, especially in gourmet or deli-style sandwiches. They add a tangy, 
	  } 
	   } 
	   } 
	 & & \\ 
 

    \theutterance \stepcounter{utterance}  

    & & & \multicolumn{2}{p{0.3\linewidth}}{\cellcolor[rgb]{0.95,0.95,0.95}{%
	\makecell[{{p{\linewidth}}}]{% 
	  \tt {\tiny [GM$|$GM]}  
	 {'guess': 'caper', 'explanation': 'Based on the feedback, we now know that the word starts with \'c\' and ends with \'er\'. The letter \'e\' is not in the second or third position, so it must be in the fourth position. "Caper" fits this pattern perfectly. Moreover, capers are often used as a sandwich decoration or topping, especially in gourmet or deli-style sandwiches. They add a tangy,'} 
	  } 
	   } 
	   } 
	 & & \\ 
 

    \theutterance \stepcounter{utterance}  

    & & & \multicolumn{2}{p{0.3\linewidth}}{\cellcolor[rgb]{0.95,0.95,0.95}{%
	\makecell[{{p{\linewidth}}}]{% 
	  \tt {\tiny [GM$|$GM]}  
	 attempts: 4\\ \tt target_word = caper\\ \tt target_word_clue = sandwich decoration\\ \tt guess: caper\\ \tt guess_feedback: c<green> a<green> p<green> e<green> r<green> 
	  } 
	   } 
	   } 
	 & & \\ 
 

    \theutterance \stepcounter{utterance}  

    & & & \multicolumn{2}{p{0.3\linewidth}}{\cellcolor[rgb]{0.95,0.95,0.95}{%
	\makecell[{{p{\linewidth}}}]{% 
	  \tt {\tiny [GM$|$GM]}  
	 game_result = WIN 
	  } 
	   } 
	   } 
	 & & \\ 
 

\end{supertabular}
}

\end{document}

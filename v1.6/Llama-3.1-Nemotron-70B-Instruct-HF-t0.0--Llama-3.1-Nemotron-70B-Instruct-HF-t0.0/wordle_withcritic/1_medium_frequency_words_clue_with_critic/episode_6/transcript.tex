
\documentclass{article}
\usepackage{colortbl}
\usepackage{makecell}
\usepackage{multirow}
\usepackage{supertabular}

\begin{document}

\newcounter{utterance}

\twocolumn

{ \footnotesize  \setcounter{utterance}{1}
\setlength{\tabcolsep}{0pt}
\begin{supertabular}{c@{$\;$}|p{.15\linewidth}@{}p{.15\linewidth}p{.15\linewidth}p{.15\linewidth}p{.15\linewidth}p{.15\linewidth}}

    \# & $\;$A & \multicolumn{4}{c}{Game Master} & $\;\:$B\\
    \hline 

    \theutterance \stepcounter{utterance}  

    & & \multicolumn{4}{p{0.6\linewidth}}{\cellcolor[rgb]{0.9,0.9,0.9}{%
	\makecell[{{p{\linewidth}}}]{% 
	  \tt {\tiny [A$\langle$GM]}  
	 You are a language wizard who likes to guess words by using the given rules.\\ \tt \\ \tt Welcome to Wordle! You have six attempts to guess the target word, a valid English word of five lowercase letters (a-z). Please use the tags "guess:" and "explanation:" to provide a concise explanation for each guess.\\ \tt \\ \tt To help you make an informed guess, you will receive a clue for the word, such as\\ \tt clue: "snowy white"\\ \tt \\ \tt Here is an example guess based on the clue:\\ \tt guess: apple\\ \tt explanation: In the fairy tail Snow White, the girl is killed because she eats a poisoned apple. And the word apple has 5 letters.\\ \tt \\ \tt I will then indicate whether I agree or disagree with your guess and provide rationale, but agreeing with a guess does not confirm its correctness. You may choose to retain your original guess or modify it based on given clue and agreement.\\ \tt \\ \tt After each guess, your answer will be validated, and you will receive feedback indicating which letters are correct (green), which letters are correct but in the wrong position (yellow), and which letters are incorrect (red). This feedback can be useful in determining which letters to include or exclude in your next guess.\\ \tt \\ \tt For example, the feedback for "apple" might be:\\ \tt guess_feedback: a<yellow> p<yellow> p<green> l<yellow> e<red>\\ \tt \\ \tt Here are some rules to keep in mind:\\ \tt 1. Please do not generate any guess feedback or clue on your own\\ \tt 2. Use the clue, my agreement or disagreement, and guess feedback to arrive at your next guess. This should be reflected in your explanation as well.\\ \tt \\ \tt Let's begin with your first guess.\\ \tt \\ \tt clue:chin indentation\\ \tt  
	  } 
	   } 
	   } 
	 & \\ 
 

    \theutterance \stepcounter{utterance}  

    & \multicolumn{4}{p{0.6\linewidth}}{\cellcolor[rgb]{0.8,1,0.9}{%
	\makecell[{{p{\linewidth}}}]{% 
	  \tt {\tiny [A$\rangle$GM]}  
	 Sounds like a fascinating challenge! Here's my first guess:\\ \tt \\ \tt **guess:** dimple\\ \tt **explanation:** The clue "chin indentation" strongly suggests a physical feature commonly associated with the chin area. A dimple is a small natural indentation in the skin, often found on the chin or cheeks. The word "dimple" fits the required five-letter length, making it a plausible initial guess based on the provided clue. 
	  } 
	   } 
	   } 
	 & & \\ 
 

    \theutterance \stepcounter{utterance}  

    & & & \multicolumn{2}{p{0.3\linewidth}}{\cellcolor[rgb]{0.95,0.95,0.95}{%
	\makecell[{{p{\linewidth}}}]{% 
	  \tt {\tiny [GM$|$GM]}  
	 {'guess': '', 'explanation': '** The clue "chin indentation" strongly suggests a physical feature commonly associated with the chin area. A dimple is a small natural indentation in the skin, often found on the chin or cheeks. The word "dimple" fits the required five-letter length, making it a plausible initial guess based on the provided clue.'} 
	  } 
	   } 
	   } 
	 & & \\ 
 

    \theutterance \stepcounter{utterance}  

    & & & \multicolumn{2}{p{0.3\linewidth}}{\cellcolor[rgb]{0.95,0.95,0.95}{%
	\makecell[{{p{\linewidth}}}]{% 
	  \tt {\tiny [GM$|$GM]}  
	 Guesser Error: INVALID_WORD while parsing Player 1's (model: Llama-3.1-Nemotron-70B-Instruct-HF) response 
	  } 
	   } 
	   } 
	 & & \\ 
 

    \theutterance \stepcounter{utterance}  

    & & \multicolumn{4}{p{0.6\linewidth}}{\cellcolor[rgb]{0.9,0.9,0.9}{%
	\makecell[{{p{\linewidth}}}]{% 
	  \tt {\tiny [A$\langle$GM]}  
	 the word should contain only alphabets. Please try again 
	  } 
	   } 
	   } 
	 & \\ 
 

    \theutterance \stepcounter{utterance}  

    & \multicolumn{4}{p{0.6\linewidth}}{\cellcolor[rgb]{0.8,1,0.9}{%
	\makecell[{{p{\linewidth}}}]{% 
	  \tt {\tiny [A$\rangle$GM]}  
	 Here's another attempt:\\ \tt \\ \tt clue: chin indentation\\ \tt guess: cleft\\ \tt explanation: The term "cleft" is often used to describe a natural division or indentation in the body, and a "cleft chin" is a common idiomatic expression referring to a chin with a noticeable indentation or split. The word "cleft" meets the five-letter criteria, aligning with the game's rules. 
	  } 
	   } 
	   } 
	 & & \\ 
 

    \theutterance \stepcounter{utterance}  

    & & & \multicolumn{2}{p{0.3\linewidth}}{\cellcolor[rgb]{0.95,0.95,0.95}{%
	\makecell[{{p{\linewidth}}}]{% 
	  \tt {\tiny [GM$|$GM]}  
	 {'guess': 'cleft', 'explanation': 'The term "cleft" is often used to describe a natural division or indentation in the body, and a "cleft chin" is a common idiomatic expression referring to a chin with a noticeable indentation or split. The word "cleft" meets the five-letter criteria, aligning with the game\'s rules.'} 
	  } 
	   } 
	   } 
	 & & \\ 
 

    \theutterance \stepcounter{utterance}  

    & & \multicolumn{4}{p{0.6\linewidth}}{\cellcolor[rgb]{0.9,0.9,0.9}{%
	\makecell[{{p{\linewidth}}}]{% 
	  \tt {\tiny [GM$\rangle$B]}  
	 I need your assistance with a word game in which we need to find a 5-letter word using a clue, a guess and an explanation for the guess.\\ \tt \\ \tt For example, if the clue is "snow white" and the guess is "apple", you will get the information like this:\\ \tt clue:snow white\\ \tt guess:apple\\ \tt explanation:the fairy tale Snow White involves the girl eating a poisoned apple, which has 5 letters.\\ \tt \\ \tt Your task is to either agree or disagree with my guess based on the given clue.\\ \tt \\ \tt If you disagree, I will make another guess, and if you agree, I will proceed in the game using the same guess.\\ \tt \\ \tt Initially, guess feedback is not available, but as the game proceeds, it becomes available. The guess feedback is not for current guess, but rather an indication of what letters can be present in the current guess. A letter highlighted in green indicates that it is a correct letter in the correct position, while yellow indicates that it is a correct letter in the wrong position, and red indicates a wrong letter. At this point, you should use all the provided information, including the clue, guess, explanation, and guess feedback, to decide whether the given guess matches the clue, and also stick to guess feedback or not.\\ \tt \\ \tt Please respond in lowercase letters and stick to this format:\\ \tt agreement:yes or no\\ \tt explanation:your reason for doing so\\ \tt \\ \tt Please note that you must not guess any word based on the riddle clue and stick to the given format while responding.\\ \tt \\ \tt Let's begin!\\ \tt \\ \tt clue:chin indentation\\ \tt guess:cleft\\ \tt explanation:The term "cleft" is often used to describe a natural division or indentation in the body, and a "cleft chin" is a common idiomatic expression referring to a chin with a noticeable indentation or split. The word "cleft" meets the five-letter criteria, aligning with the game's rules.\\ \tt  
	  } 
	   } 
	   } 
	 & \\ 
 

    \theutterance \stepcounter{utterance}  

    & & & \multicolumn{4}{p{0.6\linewidth}}{\cellcolor[rgb]{1,0.85,0.72}{%
	\makecell[{{p{\linewidth}}}]{% 
	  \tt {\tiny [GM$\langle$B]}  
	 agreement: **yes**\\ \tt explanation: the provided explanation convincingly links the clue "chin indentation" to the guess "cleft" by referencing the common idiomatic expression "cleft chin", which directly relates to a chin with a noticeable indentation, thereby strongly supporting the relevance of the guess to the clue, and also adhering to the 5-letter word requirement. 
	  } 
	   } 
	   } 
	  \\ 
 

    \theutterance \stepcounter{utterance}  

    & & & \multicolumn{2}{p{0.3\linewidth}}{\cellcolor[rgb]{0.95,0.95,0.95}{%
	\makecell[{{p{\linewidth}}}]{% 
	  \tt {\tiny [GM$|$GM]}  
	 {'agreement': 'INVALID_FORMAT', 'explanation': 'the provided explanation convincingly links the clue "chin indentation" to the guess "cleft" by referencing the common idiomatic expression "cleft chin", which directly relates to a chin with a noticeable indentation, thereby strongly supporting the relevance of the guess to the clue, and also adhering to the 5-letter word requirement.'} 
	  } 
	   } 
	   } 
	 & & \\ 
 

    \theutterance \stepcounter{utterance}  

    & & \multicolumn{4}{p{0.6\linewidth}}{\cellcolor[rgb]{0.9,0.9,0.9}{%
	\makecell[{{p{\linewidth}}}]{% 
	  \tt {\tiny [GM$\rangle$B]}  
	 Provide your response only in this format:\\ \tt agreement:yes or no\\ \tt explanation:details\\ \tt Please try again 
	  } 
	   } 
	   } 
	 & \\ 
 

    \theutterance \stepcounter{utterance}  

    & & & \multicolumn{4}{p{0.6\linewidth}}{\cellcolor[rgb]{1,0.85,0.72}{%
	\makecell[{{p{\linewidth}}}]{% 
	  \tt {\tiny [GM$\langle$B]}  
	 Here is the revised response in the requested format:\\ \tt \\ \tt agreement:yes\\ \tt explanation:the provided explanation convincingly links the clue "chin indentation" to the guess "cleft" by referencing the common idiomatic expression "cleft chin", which directly relates to a chin with a noticeable indentation, thereby strongly supporting the relevance of the guess to the clue, and also adhering to the 5-letter word requirement. 
	  } 
	   } 
	   } 
	  \\ 
 

    \theutterance \stepcounter{utterance}  

    & & & \multicolumn{2}{p{0.3\linewidth}}{\cellcolor[rgb]{0.95,0.95,0.95}{%
	\makecell[{{p{\linewidth}}}]{% 
	  \tt {\tiny [GM$|$GM]}  
	 {'agreement': 'yes', 'explanation': 'the provided explanation convincingly links the clue "chin indentation" to the guess "cleft" by referencing the common idiomatic expression "cleft chin", which directly relates to a chin with a noticeable indentation, thereby strongly supporting the relevance of the guess to the clue, and also adhering to the 5-letter word requirement.'} 
	  } 
	   } 
	   } 
	 & & \\ 
 

    \theutterance \stepcounter{utterance}  

    & & & \multicolumn{2}{p{0.3\linewidth}}{\cellcolor[rgb]{0.95,0.95,0.95}{%
	\makecell[{{p{\linewidth}}}]{% 
	  \tt {\tiny [GM$|$GM]}  
	 Critic agrees with the Guesser -- Sharing the critic's explanation with the guesser 
	  } 
	   } 
	   } 
	 & & \\ 
 

    \theutterance \stepcounter{utterance}  

    & & \multicolumn{4}{p{0.6\linewidth}}{\cellcolor[rgb]{0.9,0.9,0.9}{%
	\makecell[{{p{\linewidth}}}]{% 
	  \tt {\tiny [A$\langle$GM]}  
	 clue:chin indentation\\ \tt guess_agreement:yes\\ \tt agreement_explanation:the provided explanation convincingly links the clue "chin indentation" to the guess "cleft" by referencing the common idiomatic expression "cleft chin", which directly relates to a chin with a noticeable indentation, thereby strongly supporting the relevance of the guess to the clue, and also adhering to the 5-letter word requirement.\\ \tt  
	  } 
	   } 
	   } 
	 & \\ 
 

    \theutterance \stepcounter{utterance}  

    & \multicolumn{4}{p{0.6\linewidth}}{\cellcolor[rgb]{0.8,1,0.9}{%
	\makecell[{{p{\linewidth}}}]{% 
	  \tt {\tiny [A$\rangle$GM]}  
	 Thank you for the agreement and detailed explanation. Since my initial guess was deemed relevant, I'll use this as a foundation to refine my next attempt, considering the possibility that "cleft" might not be the exact target word. Here's my next step:\\ \tt \\ \tt guess:chins\\ \tt explanation: **Building upon the agreement with "cleft" and its strong connection to "cleft chin", my next guess, "chins", directly incorporates the word from the clue ("chin") and meets 
	  } 
	   } 
	   } 
	 & & \\ 
 

    \theutterance \stepcounter{utterance}  

    & & & \multicolumn{2}{p{0.3\linewidth}}{\cellcolor[rgb]{0.95,0.95,0.95}{%
	\makecell[{{p{\linewidth}}}]{% 
	  \tt {\tiny [GM$|$GM]}  
	 {'guess': 'chins', 'explanation': '**Building upon the agreement with "cleft" and its strong connection to "cleft chin", my next guess, "chins", directly incorporates the word from the clue ("chin") and meets'} 
	  } 
	   } 
	   } 
	 & & \\ 
 

    \theutterance \stepcounter{utterance}  

    & & & \multicolumn{2}{p{0.3\linewidth}}{\cellcolor[rgb]{0.95,0.95,0.95}{%
	\makecell[{{p{\linewidth}}}]{% 
	  \tt {\tiny [GM$|$GM]}  
	 Change in player1's guess\\ \tt guess_before_critic_opinion: cleft\\ \tt                                                 critic_agreement: yes\\ \tt guess_after_critic_opinion: chins\\ \tt                                                 Proceeding with guess validation 
	  } 
	   } 
	   } 
	 & & \\ 
 

    \theutterance \stepcounter{utterance}  

    & & & \multicolumn{2}{p{0.3\linewidth}}{\cellcolor[rgb]{0.95,0.95,0.95}{%
	\makecell[{{p{\linewidth}}}]{% 
	  \tt {\tiny [GM$|$GM]}  
	 attempts: 1\\ \tt target_word = cleft\\ \tt target_word_clue = chin indentation\\ \tt guess: chins\\ \tt guess_feedback: c<green> h<red> i<red> n<red> s<red> 
	  } 
	   } 
	   } 
	 & & \\ 
 

    \theutterance \stepcounter{utterance}  

    & & \multicolumn{4}{p{0.6\linewidth}}{\cellcolor[rgb]{0.9,0.9,0.9}{%
	\makecell[{{p{\linewidth}}}]{% 
	  \tt {\tiny [A$\langle$GM]}  
	 guess_feedback: c<green> h<red> i<red> n<red> s<red> 
	  } 
	   } 
	   } 
	 & \\ 
 

    \theutterance \stepcounter{utterance}  

    & \multicolumn{4}{p{0.6\linewidth}}{\cellcolor[rgb]{0.8,1,0.9}{%
	\makecell[{{p{\linewidth}}}]{% 
	  \tt {\tiny [A$\rangle$GM]}  
	 **Processing Feedback**\\ \tt \\ \tt guess_feedback: c<green> h<red> i<red> n<red> s<red>\\ \tt \\ \tt **Next Guess**\\ \tt \\ \tt guess:clerk\\ \tt explanation:\\ \tt * The feedback indicates that only the letter "c" is correct (green) and in the right position, so I'll retain it as the starting letter.\\ \tt * The letters "h", "i", "n", and "s" are all incorrect (red), which rules out my previous connection to 
	  } 
	   } 
	   } 
	 & & \\ 
 

    \theutterance \stepcounter{utterance}  

    & & & \multicolumn{2}{p{0.3\linewidth}}{\cellcolor[rgb]{0.95,0.95,0.95}{%
	\makecell[{{p{\linewidth}}}]{% 
	  \tt {\tiny [GM$|$GM]}  
	 {'guess': 'clerk', 'explanation': ''} 
	  } 
	   } 
	   } 
	 & & \\ 
 

    \theutterance \stepcounter{utterance}  

    & & \multicolumn{4}{p{0.6\linewidth}}{\cellcolor[rgb]{0.9,0.9,0.9}{%
	\makecell[{{p{\linewidth}}}]{% 
	  \tt {\tiny [GM$\rangle$B]}  
	 clue:chin indentation\\ \tt guess:clerk\\ \tt explanation:\\ \tt guess_feedback: c<green> h<red> i<red> n<red> s<red> 
	  } 
	   } 
	   } 
	 & \\ 
 

    \theutterance \stepcounter{utterance}  

    & & & \multicolumn{4}{p{0.6\linewidth}}{\cellcolor[rgb]{1,0.85,0.72}{%
	\makecell[{{p{\linewidth}}}]{% 
	  \tt {\tiny [GM$\langle$B]}  
	 agreement:no\\ \tt explanation:the provided guess "clerk" does not semantically match the clue "chin indentation", as "clerk" refers to an office worker or a salesperson, showing no relation to physical body features or anatomical indentations, unlike the clue's clear reference to a facial characteristic. 
	  } 
	   } 
	   } 
	  \\ 
 

    \theutterance \stepcounter{utterance}  

    & & & \multicolumn{2}{p{0.3\linewidth}}{\cellcolor[rgb]{0.95,0.95,0.95}{%
	\makecell[{{p{\linewidth}}}]{% 
	  \tt {\tiny [GM$|$GM]}  
	 {'agreement': 'no', 'explanation': 'the provided guess "clerk" does not semantically match the clue "chin indentation", as "clerk" refers to an office worker or a salesperson, showing no relation to physical body features or anatomical indentations, unlike the clue\'s clear reference to a facial characteristic.'} 
	  } 
	   } 
	   } 
	 & & \\ 
 

    \theutterance \stepcounter{utterance}  

    & & & \multicolumn{2}{p{0.3\linewidth}}{\cellcolor[rgb]{0.95,0.95,0.95}{%
	\makecell[{{p{\linewidth}}}]{% 
	  \tt {\tiny [GM$|$GM]}  
	 Critic disagrees with the Guesser -- Sharing the critic's explanation with the guesser 
	  } 
	   } 
	   } 
	 & & \\ 
 

    \theutterance \stepcounter{utterance}  

    & & \multicolumn{4}{p{0.6\linewidth}}{\cellcolor[rgb]{0.9,0.9,0.9}{%
	\makecell[{{p{\linewidth}}}]{% 
	  \tt {\tiny [A$\langle$GM]}  
	 clue:chin indentation\\ \tt guess_agreement:no\\ \tt agreement_explanation:the provided guess "clerk" does not semantically match the clue "chin indentation", as "clerk" refers to an office worker or a salesperson, showing no relation to physical body features or anatomical indentations, unlike the clue's clear reference to a facial characteristic.\\ \tt  
	  } 
	   } 
	   } 
	 & \\ 
 

    \theutterance \stepcounter{utterance}  

    & \multicolumn{4}{p{0.6\linewidth}}{\cellcolor[rgb]{0.8,1,0.9}{%
	\makecell[{{p{\linewidth}}}]{% 
	  \tt {\tiny [A$\rangle$GM]}  
	 Thank you for the feedback. Here's my next attempt:\\ \tt \\ \tt guess:crepe\\ \tt explanation:\\ \tt * The guess "cleft" was initially agreed upon, and the feedback for "chins" confirmed that the letter "c" is correct (<green>).\\ \tt * Considering the clue "chin indentation" again, I'm exploring words related to skin or facial features.\\ \tt * "Crepe" relates to skin, as "crepey skin" is a term used to describe thin, wrinkled 
	  } 
	   } 
	   } 
	 & & \\ 
 

    \theutterance \stepcounter{utterance}  

    & & & \multicolumn{2}{p{0.3\linewidth}}{\cellcolor[rgb]{0.95,0.95,0.95}{%
	\makecell[{{p{\linewidth}}}]{% 
	  \tt {\tiny [GM$|$GM]}  
	 {'guess': 'crepe', 'explanation': ''} 
	  } 
	   } 
	   } 
	 & & \\ 
 

    \theutterance \stepcounter{utterance}  

    & & & \multicolumn{2}{p{0.3\linewidth}}{\cellcolor[rgb]{0.95,0.95,0.95}{%
	\makecell[{{p{\linewidth}}}]{% 
	  \tt {\tiny [GM$|$GM]}  
	 Change in player1's guess\\ \tt guess_before_critic_opinion: clerk\\ \tt                                                 critic_agreement: no\\ \tt guess_after_critic_opinion: crepe\\ \tt                                                 Proceeding with guess validation 
	  } 
	   } 
	   } 
	 & & \\ 
 

    \theutterance \stepcounter{utterance}  

    & & & \multicolumn{2}{p{0.3\linewidth}}{\cellcolor[rgb]{0.95,0.95,0.95}{%
	\makecell[{{p{\linewidth}}}]{% 
	  \tt {\tiny [GM$|$GM]}  
	 attempts: 2\\ \tt target_word = cleft\\ \tt target_word_clue = chin indentation\\ \tt guess: crepe\\ \tt guess_feedback: c<green> r<red> e<green> p<red> e<red> 
	  } 
	   } 
	   } 
	 & & \\ 
 

    \theutterance \stepcounter{utterance}  

    & & \multicolumn{4}{p{0.6\linewidth}}{\cellcolor[rgb]{0.9,0.9,0.9}{%
	\makecell[{{p{\linewidth}}}]{% 
	  \tt {\tiny [A$\langle$GM]}  
	 guess_feedback: c<green> r<red> e<green> p<red> e<red> 
	  } 
	   } 
	   } 
	 & \\ 
 

    \theutterance \stepcounter{utterance}  

    & \multicolumn{4}{p{0.6\linewidth}}{\cellcolor[rgb]{0.8,1,0.9}{%
	\makecell[{{p{\linewidth}}}]{% 
	  \tt {\tiny [A$\rangle$GM]}  
	 guess:crevice\\ \tt explanation:\\ \tt * **Retaining confirmed letters**: The guess_feedback indicated that letters "c" and "e" are correct (green), so I've kept them in the new guess.\\ \tt * **Addressing the clue**: The word "crevice" means a small, narrow opening or indentation in a surface, which aligns with the concept of a "chin indentation" (a small, defined space on the chin).\\ \tt * **Adapting to feedback**: Since " 
	  } 
	   } 
	   } 
	 & & \\ 
 

    \theutterance \stepcounter{utterance}  

    & & & \multicolumn{2}{p{0.3\linewidth}}{\cellcolor[rgb]{0.95,0.95,0.95}{%
	\makecell[{{p{\linewidth}}}]{% 
	  \tt {\tiny [GM$|$GM]}  
	 {'guess': 'crevice', 'explanation': ''} 
	  } 
	   } 
	   } 
	 & & \\ 
 

    \theutterance \stepcounter{utterance}  

    & & & \multicolumn{2}{p{0.3\linewidth}}{\cellcolor[rgb]{0.95,0.95,0.95}{%
	\makecell[{{p{\linewidth}}}]{% 
	  \tt {\tiny [GM$|$GM]}  
	 Guesser Error: INVALID_WORD_LENGTH while parsing Player 1's (model: Llama-3.1-Nemotron-70B-Instruct-HF) response 
	  } 
	   } 
	   } 
	 & & \\ 
 

    \theutterance \stepcounter{utterance}  

    & & \multicolumn{4}{p{0.6\linewidth}}{\cellcolor[rgb]{0.9,0.9,0.9}{%
	\makecell[{{p{\linewidth}}}]{% 
	  \tt {\tiny [A$\langle$GM]}  
	 the word should have exactly 5 letters. Please try again 
	  } 
	   } 
	   } 
	 & \\ 
 

    \theutterance \stepcounter{utterance}  

    & \multicolumn{4}{p{0.6\linewidth}}{\cellcolor[rgb]{0.8,1,0.9}{%
	\makecell[{{p{\linewidth}}}]{% 
	  \tt {\tiny [A$\rangle$GM]}  
	 guess:creep\\ \tt explanation:\\ \tt * The guess retains the letters "c" and "e" which were marked as <green> in the previous feedback, indicating they are in the correct position.\\ \tt * The clue "chin indentation" might imply a subtle or slight imperfection, and while "creep" doesn't directly relate to anatomy, it meets the 5-letter criteria and leverages the confirmed correct letters.\\ \tt * Alternatively, a more fitting interpretation could be that the guess needs to " 
	  } 
	   } 
	   } 
	 & & \\ 
 

    \theutterance \stepcounter{utterance}  

    & & & \multicolumn{2}{p{0.3\linewidth}}{\cellcolor[rgb]{0.95,0.95,0.95}{%
	\makecell[{{p{\linewidth}}}]{% 
	  \tt {\tiny [GM$|$GM]}  
	 {'guess': 'creep', 'explanation': ''} 
	  } 
	   } 
	   } 
	 & & \\ 
 

    \theutterance \stepcounter{utterance}  

    & & \multicolumn{4}{p{0.6\linewidth}}{\cellcolor[rgb]{0.9,0.9,0.9}{%
	\makecell[{{p{\linewidth}}}]{% 
	  \tt {\tiny [GM$\rangle$B]}  
	 clue:chin indentation\\ \tt guess:creep\\ \tt explanation:\\ \tt guess_feedback: c<green> r<red> e<green> p<red> e<red> 
	  } 
	   } 
	   } 
	 & \\ 
 

    \theutterance \stepcounter{utterance}  

    & & & \multicolumn{4}{p{0.6\linewidth}}{\cellcolor[rgb]{1,0.85,0.72}{%
	\makecell[{{p{\linewidth}}}]{% 
	  \tt {\tiny [GM$\langle$B]}  
	 Here are the responses in the requested format:\\ \tt \\ \tt **Response 1**\\ \tt agreement:no\\ \tt explanation:the guess "clerk" doesn't have a clear connection to the concept of "chin indentation", unlike terms that directly imply a split or division, and the provided guess_feedback doesn't support a strong link either.\\ \tt \\ \tt **Response 2**\\ \tt agreement:no\\ \tt explanation:while the guess "creep" has some letter alignment indicated by guess_feedback (c and e in correct positions), the 
	  } 
	   } 
	   } 
	  \\ 
 

    \theutterance \stepcounter{utterance}  

    & & & \multicolumn{2}{p{0.3\linewidth}}{\cellcolor[rgb]{0.95,0.95,0.95}{%
	\makecell[{{p{\linewidth}}}]{% 
	  \tt {\tiny [GM$|$GM]}  
	 {'agreement': 'no', 'explanation': 'the guess "clerk" doesn\'t have a clear connection to the concept of "chin indentation", unlike terms that directly imply a split or division, and the provided guess_feedback doesn\'t support a strong link either.'} 
	  } 
	   } 
	   } 
	 & & \\ 
 

    \theutterance \stepcounter{utterance}  

    & & & \multicolumn{2}{p{0.3\linewidth}}{\cellcolor[rgb]{0.95,0.95,0.95}{%
	\makecell[{{p{\linewidth}}}]{% 
	  \tt {\tiny [GM$|$GM]}  
	 Critic disagrees with the Guesser -- Sharing the critic's explanation with the guesser 
	  } 
	   } 
	   } 
	 & & \\ 
 

    \theutterance \stepcounter{utterance}  

    & & \multicolumn{4}{p{0.6\linewidth}}{\cellcolor[rgb]{0.9,0.9,0.9}{%
	\makecell[{{p{\linewidth}}}]{% 
	  \tt {\tiny [A$\langle$GM]}  
	 clue:chin indentation\\ \tt guess_agreement:no\\ \tt agreement_explanation:the guess "clerk" doesn't have a clear connection to the concept of "chin indentation", unlike terms that directly imply a split or division, and the provided guess_feedback doesn't support a strong link either.\\ \tt  
	  } 
	   } 
	   } 
	 & \\ 
 

    \theutterance \stepcounter{utterance}  

    & \multicolumn{4}{p{0.6\linewidth}}{\cellcolor[rgb]{0.8,1,0.9}{%
	\makecell[{{p{\linewidth}}}]{% 
	  \tt {\tiny [A$\rangle$GM]}  
	 It seems there's a mix-up in the response. You provided an agreement explanation for "clerk", but the guess in question was "creep". Assuming the disagreement is indeed for "creep", here's my next attempt:\\ \tt \\ \tt guess:cease\\ \tt explanation: **Revisiting the guess_feedback for "crepe" (c<green> e<green>...), I'm trying to utilize the confirmed green letters "c" and "e" in a new context. Although 
	  } 
	   } 
	   } 
	 & & \\ 
 

    \theutterance \stepcounter{utterance}  

    & & & \multicolumn{2}{p{0.3\linewidth}}{\cellcolor[rgb]{0.95,0.95,0.95}{%
	\makecell[{{p{\linewidth}}}]{% 
	  \tt {\tiny [GM$|$GM]}  
	 {'guess': 'cease', 'explanation': '**Revisiting the guess_feedback for "crepe" (c<green> e<green>...), I\'m trying to utilize the confirmed green letters "c" and "e" in a new context. Although'} 
	  } 
	   } 
	   } 
	 & & \\ 
 

    \theutterance \stepcounter{utterance}  

    & & & \multicolumn{2}{p{0.3\linewidth}}{\cellcolor[rgb]{0.95,0.95,0.95}{%
	\makecell[{{p{\linewidth}}}]{% 
	  \tt {\tiny [GM$|$GM]}  
	 Change in player1's guess\\ \tt guess_before_critic_opinion: creep\\ \tt                                                 critic_agreement: no\\ \tt guess_after_critic_opinion: cease\\ \tt                                                 Proceeding with guess validation 
	  } 
	   } 
	   } 
	 & & \\ 
 

    \theutterance \stepcounter{utterance}  

    & & & \multicolumn{2}{p{0.3\linewidth}}{\cellcolor[rgb]{0.95,0.95,0.95}{%
	\makecell[{{p{\linewidth}}}]{% 
	  \tt {\tiny [GM$|$GM]}  
	 attempts: 3\\ \tt target_word = cleft\\ \tt target_word_clue = chin indentation\\ \tt guess: cease\\ \tt guess_feedback: c<green> e<yellow> a<red> s<red> e<red> 
	  } 
	   } 
	   } 
	 & & \\ 
 

    \theutterance \stepcounter{utterance}  

    & & \multicolumn{4}{p{0.6\linewidth}}{\cellcolor[rgb]{0.9,0.9,0.9}{%
	\makecell[{{p{\linewidth}}}]{% 
	  \tt {\tiny [A$\langle$GM]}  
	 guess_feedback: c<green> e<yellow> a<red> s<red> e<red> 
	  } 
	   } 
	   } 
	 & \\ 
 

    \theutterance \stepcounter{utterance}  

    & \multicolumn{4}{p{0.6\linewidth}}{\cellcolor[rgb]{0.8,1,0.9}{%
	\makecell[{{p{\linewidth}}}]{% 
	  \tt {\tiny [A$\rangle$GM]}  
	 guess:celts\\ \tt explanation:**\\ \tt * **Utilizing confirmed green "c"**: Maintaining the correct letter "c" from previous feedback.\\ \tt * **Adjusting yellow "e"**: Trying to find the right position for the letter "e", which was marked yellow (indicating correct letter, wrong position) in the new guess.\\ \tt * **Incorporating a plausible link to "chin indentation"**: The word "celts" might be a stretch, but some historical or 
	  } 
	   } 
	   } 
	 & & \\ 
 

    \theutterance \stepcounter{utterance}  

    & & & \multicolumn{2}{p{0.3\linewidth}}{\cellcolor[rgb]{0.95,0.95,0.95}{%
	\makecell[{{p{\linewidth}}}]{% 
	  \tt {\tiny [GM$|$GM]}  
	 {'guess': 'celts', 'explanation': '**'} 
	  } 
	   } 
	   } 
	 & & \\ 
 

    \theutterance \stepcounter{utterance}  

    & & \multicolumn{4}{p{0.6\linewidth}}{\cellcolor[rgb]{0.9,0.9,0.9}{%
	\makecell[{{p{\linewidth}}}]{% 
	  \tt {\tiny [GM$\rangle$B]}  
	 clue:chin indentation\\ \tt guess:celts\\ \tt explanation:**\\ \tt guess_feedback: c<green> e<yellow> a<red> s<red> e<red> 
	  } 
	   } 
	   } 
	 & \\ 
 

    \theutterance \stepcounter{utterance}  

    & & & \multicolumn{4}{p{0.6\linewidth}}{\cellcolor[rgb]{1,0.85,0.72}{%
	\makecell[{{p{\linewidth}}}]{% 
	  \tt {\tiny [GM$\langle$B]}  
	 Here are the responses in the requested format:\\ \tt \\ \tt **1.**\\ \tt agreement:no\\ \tt explanation:the guess "clerk" doesn't have a clear connection to the concept of "chin indentation", unlike terms that directly imply a split or division, making it an unlikely match.\\ \tt \\ \tt **2.**\\ \tt agreement:no\\ \tt explanation:despite some correct letter placements (c, e), the guess "creep" lacks a logical or semantic link to the idea of "chin indentation", rendering it 
	  } 
	   } 
	   } 
	  \\ 
 

    \theutterance \stepcounter{utterance}  

    & & & \multicolumn{2}{p{0.3\linewidth}}{\cellcolor[rgb]{0.95,0.95,0.95}{%
	\makecell[{{p{\linewidth}}}]{% 
	  \tt {\tiny [GM$|$GM]}  
	 {'agreement': 'no', 'explanation': 'the guess "clerk" doesn\'t have a clear connection to the concept of "chin indentation", unlike terms that directly imply a split or division, making it an unlikely match.'} 
	  } 
	   } 
	   } 
	 & & \\ 
 

    \theutterance \stepcounter{utterance}  

    & & & \multicolumn{2}{p{0.3\linewidth}}{\cellcolor[rgb]{0.95,0.95,0.95}{%
	\makecell[{{p{\linewidth}}}]{% 
	  \tt {\tiny [GM$|$GM]}  
	 Critic disagrees with the Guesser -- Sharing the critic's explanation with the guesser 
	  } 
	   } 
	   } 
	 & & \\ 
 

    \theutterance \stepcounter{utterance}  

    & & \multicolumn{4}{p{0.6\linewidth}}{\cellcolor[rgb]{0.9,0.9,0.9}{%
	\makecell[{{p{\linewidth}}}]{% 
	  \tt {\tiny [A$\langle$GM]}  
	 clue:chin indentation\\ \tt guess_agreement:no\\ \tt agreement_explanation:the guess "clerk" doesn't have a clear connection to the concept of "chin indentation", unlike terms that directly imply a split or division, making it an unlikely match.\\ \tt  
	  } 
	   } 
	   } 
	 & \\ 
 

    \theutterance \stepcounter{utterance}  

    & \multicolumn{4}{p{0.6\linewidth}}{\cellcolor[rgb]{0.8,1,0.9}{%
	\makecell[{{p{\linewidth}}}]{% 
	  \tt {\tiny [A$\rangle$GM]}  
	 **Note:** The agreement explanation seems to reference "clerk", which was a previous guess, not "celts". I'll proceed with the understanding that "celts" also lacks a clear connection to "chin indentation".\\ \tt \\ \tt guess:cleft\\ \tt explanation:\\ \tt * Building upon the consistent presence of the green "c" in previous feedback (indicating its correct position)\\ \tt * Revisiting the initial clue "chin indentation" and the success of "cleft" (guess 1) in 
	  } 
	   } 
	   } 
	 & & \\ 
 

    \theutterance \stepcounter{utterance}  

    & & & \multicolumn{2}{p{0.3\linewidth}}{\cellcolor[rgb]{0.95,0.95,0.95}{%
	\makecell[{{p{\linewidth}}}]{% 
	  \tt {\tiny [GM$|$GM]}  
	 {'guess': 'cleft', 'explanation': ''} 
	  } 
	   } 
	   } 
	 & & \\ 
 

    \theutterance \stepcounter{utterance}  

    & & & \multicolumn{2}{p{0.3\linewidth}}{\cellcolor[rgb]{0.95,0.95,0.95}{%
	\makecell[{{p{\linewidth}}}]{% 
	  \tt {\tiny [GM$|$GM]}  
	 Change in player1's guess\\ \tt guess_before_critic_opinion: celts\\ \tt                                                 critic_agreement: no\\ \tt guess_after_critic_opinion: cleft\\ \tt                                                 Proceeding with guess validation 
	  } 
	   } 
	   } 
	 & & \\ 
 

    \theutterance \stepcounter{utterance}  

    & & & \multicolumn{2}{p{0.3\linewidth}}{\cellcolor[rgb]{0.95,0.95,0.95}{%
	\makecell[{{p{\linewidth}}}]{% 
	  \tt {\tiny [GM$|$GM]}  
	 attempts: 4\\ \tt target_word = cleft\\ \tt target_word_clue = chin indentation\\ \tt guess: cleft\\ \tt guess_feedback: c<green> l<green> e<green> f<green> t<green> 
	  } 
	   } 
	   } 
	 & & \\ 
 

    \theutterance \stepcounter{utterance}  

    & & & \multicolumn{2}{p{0.3\linewidth}}{\cellcolor[rgb]{0.95,0.95,0.95}{%
	\makecell[{{p{\linewidth}}}]{% 
	  \tt {\tiny [GM$|$GM]}  
	 game_result = WIN 
	  } 
	   } 
	   } 
	 & & \\ 
 

\end{supertabular}
}

\end{document}


\documentclass{article}
\usepackage{colortbl}
\usepackage{makecell}
\usepackage{multirow}
\usepackage{supertabular}

\begin{document}

\newcounter{utterance}

\twocolumn

{ \footnotesize  \setcounter{utterance}{1}
\setlength{\tabcolsep}{0pt}
\begin{supertabular}{c@{$\;$}|p{.15\linewidth}@{}p{.15\linewidth}p{.15\linewidth}p{.15\linewidth}p{.15\linewidth}p{.15\linewidth}}

    \# & $\;$A & \multicolumn{4}{c}{Game Master} & $\;\:$B\\
    \hline 

    \theutterance \stepcounter{utterance}  

    & & \multicolumn{4}{p{0.6\linewidth}}{\cellcolor[rgb]{0.9,0.9,0.9}{%
	\makecell[{{p{\linewidth}}}]{% 
	  \tt {\tiny [A$\langle$GM]}  
	 You are a language wizard who likes to guess words by using the given rules.\\ \tt \\ \tt Welcome to Wordle! You have six attempts to guess the target word, a valid English word of five lowercase letters (a-z). Please use the tags "guess:" and "explanation:" to provide a concise explanation for each guess.\\ \tt \\ \tt  To help you make an informed guess, you will receive a clue for the word, such as\\ \tt clue: snowy white.\\ \tt \\ \tt Here is an example guess based on the clue:\\ \tt guess: apple\\ \tt explanation: In the fairy tail Snow White, the girl is killed because she eats a poisoned apple. And the word apple has 5 letters.\\ \tt \\ \tt After each guess, your answer will be validated, and you will receive feedback indicating which letters are correct (green), which letters are correct but in the wrong position (yellow), and which letters are incorrect (red). This feedback can be useful in determining which letters to include or exclude in your next guess.\\ \tt \\ \tt For example, the feedback for "apple" might be:\\ \tt guess_feedback: a<yellow> p<yellow> p<green> l<yellow> e<red>\\ \tt \\ \tt The explanation should contain details about how both the guess_feedback and clue work together in tandem to formulate a new guess.\\ \tt \\ \tt Let's begin with your first guess.\\ \tt clue:hungry\\ \tt  
	  } 
	   } 
	   } 
	 & \\ 
 

    \theutterance \stepcounter{utterance}  

    & \multicolumn{4}{p{0.6\linewidth}}{\cellcolor[rgb]{0.8,1,0.9}{%
	\makecell[{{p{\linewidth}}}]{% 
	  \tt {\tiny [A$\rangle$GM]}  
	 guess: croissant\\ \tt explanation: The word "croissant" is often associated with hunger as it is a type of breakfast pastry. It has five letters and includes the letter 'c', which could potentially replace the 'h' in the clue since 'crongry' doesn't form a valid word. The other letters in "croissant" can help cover various possibilities for the remaining letters in the target word. 
	  } 
	   } 
	   } 
	 & & \\ 
 

    \theutterance \stepcounter{utterance}  

    & & & \multicolumn{2}{p{0.3\linewidth}}{\cellcolor[rgb]{0.95,0.95,0.95}{%
	\makecell[{{p{\linewidth}}}]{% 
	  \tt {\tiny [GM$|$GM]}  
	 {'guess': 'croissant', 'explanation': 'The word "croissant" is often associated with hunger as it is a type of breakfast pastry. It has five letters and includes the letter \'c\', which could potentially replace the \'h\' in the clue since \'crongry\' doesn\'t form a valid word. The other letters in "croissant" can help cover various possibilities for the remaining letters in the target word.'} 
	  } 
	   } 
	   } 
	 & & \\ 
 

    \theutterance \stepcounter{utterance}  

    & & & \multicolumn{2}{p{0.3\linewidth}}{\cellcolor[rgb]{0.95,0.95,0.95}{%
	\makecell[{{p{\linewidth}}}]{% 
	  \tt {\tiny [GM$|$GM]}  
	 Guesser Error: INVALID_WORD_LENGTH while parsing Player 1's (model: Qwen1.5-32B-Chat) response, retrying 
	  } 
	   } 
	   } 
	 & & \\ 
 

    \theutterance \stepcounter{utterance}  

    & & \multicolumn{4}{p{0.6\linewidth}}{\cellcolor[rgb]{0.9,0.9,0.9}{%
	\makecell[{{p{\linewidth}}}]{% 
	  \tt {\tiny [A$\langle$GM]}  
	 the word should have exactly 5 letters. Please try again 
	  } 
	   } 
	   } 
	 & \\ 
 

    \theutterance \stepcounter{utterance}  

    & \multicolumn{4}{p{0.6\linewidth}}{\cellcolor[rgb]{0.8,1,0.9}{%
	\makecell[{{p{\linewidth}}}]{% 
	  \tt {\tiny [A$\rangle$GM]}  
	 guess:snack\\ \tt explanation: "Snack" is a five-letter word that directly relates to the concept of being hungry. It's a common term for a small meal or food consumed between meals when one feels hungry. This choice also allows for potential letter overlap with the target word based on the given clue. 
	  } 
	   } 
	   } 
	 & & \\ 
 

    \theutterance \stepcounter{utterance}  

    & & & \multicolumn{2}{p{0.3\linewidth}}{\cellcolor[rgb]{0.95,0.95,0.95}{%
	\makecell[{{p{\linewidth}}}]{% 
	  \tt {\tiny [GM$|$GM]}  
	 {'guess': 'snack', 'explanation': '"Snack" is a five-letter word that directly relates to the concept of being hungry. It\'s a common term for a small meal or food consumed between meals when one feels hungry. This choice also allows for potential letter overlap with the target word based on the given clue.'} 
	  } 
	   } 
	   } 
	 & & \\ 
 

    \theutterance \stepcounter{utterance}  

    & & & \multicolumn{2}{p{0.3\linewidth}}{\cellcolor[rgb]{0.95,0.95,0.95}{%
	\makecell[{{p{\linewidth}}}]{% 
	  \tt {\tiny [GM$|$GM]}  
	 attempts: 1\\ \tt target_word = unfed\\ \tt target_word_clue = hungry\\ \tt guess: snack\\ \tt guess_feedback: s<red> n<green> a<red> c<red> k<red> 
	  } 
	   } 
	   } 
	 & & \\ 
 

    \theutterance \stepcounter{utterance}  

    & & \multicolumn{4}{p{0.6\linewidth}}{\cellcolor[rgb]{0.9,0.9,0.9}{%
	\makecell[{{p{\linewidth}}}]{% 
	  \tt {\tiny [A$\langle$GM]}  
	 guess_feedback: s<red> n<green> a<red> c<red> k<red> 
	  } 
	   } 
	   } 
	 & \\ 
 

    \theutterance \stepcounter{utterance}  

    & \multicolumn{4}{p{0.6\linewidth}}{\cellcolor[rgb]{0.8,1,0.9}{%
	\makecell[{{p{\linewidth}}}]{% 
	  \tt {\tiny [A$\rangle$GM]}  
	 guess:graze\\ \tt explanation:Given the feedback, we know that 'n' is the only correct letter in its correct position. The clue "hungry" suggests a word related to eating or animals searching for food. "Graze" fits this context as it means animals eating grass, and it has a similar structure to "snack" with an 'a' replaced by an 'e'. This guess will help to determine if 'g', 'r', 'z', or 
	  } 
	   } 
	   } 
	 & & \\ 
 

    \theutterance \stepcounter{utterance}  

    & & & \multicolumn{2}{p{0.3\linewidth}}{\cellcolor[rgb]{0.95,0.95,0.95}{%
	\makecell[{{p{\linewidth}}}]{% 
	  \tt {\tiny [GM$|$GM]}  
	 {'guess': 'graze', 'explanation': 'Given the feedback, we know that \'n\' is the only correct letter in its correct position. The clue "hungry" suggests a word related to eating or animals searching for food. "Graze" fits this context as it means animals eating grass, and it has a similar structure to "snack" with an \'a\' replaced by an \'e\'. This guess will help to determine if \'g\', \'r\', \'z\', or'} 
	  } 
	   } 
	   } 
	 & & \\ 
 

    \theutterance \stepcounter{utterance}  

    & & & \multicolumn{2}{p{0.3\linewidth}}{\cellcolor[rgb]{0.95,0.95,0.95}{%
	\makecell[{{p{\linewidth}}}]{% 
	  \tt {\tiny [GM$|$GM]}  
	 attempts: 2\\ \tt target_word = unfed\\ \tt target_word_clue = hungry\\ \tt guess: graze\\ \tt guess_feedback: g<red> r<red> a<red> z<red> e<yellow> 
	  } 
	   } 
	   } 
	 & & \\ 
 

    \theutterance \stepcounter{utterance}  

    & & \multicolumn{4}{p{0.6\linewidth}}{\cellcolor[rgb]{0.9,0.9,0.9}{%
	\makecell[{{p{\linewidth}}}]{% 
	  \tt {\tiny [A$\langle$GM]}  
	 guess_feedback: g<red> r<red> a<red> z<red> e<yellow> 
	  } 
	   } 
	   } 
	 & \\ 
 

    \theutterance \stepcounter{utterance}  

    & \multicolumn{4}{p{0.6\linewidth}}{\cellcolor[rgb]{0.8,1,0.9}{%
	\makecell[{{p{\linewidth}}}]{% 
	  \tt {\tiny [A$\rangle$GM]}  
	 guess:bane\\ \tt explanation:Based on the new feedback, we know that 'e' is in the correct position but not at the end. The clue "hungry" might also relate to something that causes hunger or is associated with famine. "Bane" can mean a source of trouble or a cause of suffering, sometimes associated with hunger. It's a five-letter word with 'e' in the third position, allowing us to test the other positions with this guess. 
	  } 
	   } 
	   } 
	 & & \\ 
 

    \theutterance \stepcounter{utterance}  

    & & & \multicolumn{2}{p{0.3\linewidth}}{\cellcolor[rgb]{0.95,0.95,0.95}{%
	\makecell[{{p{\linewidth}}}]{% 
	  \tt {\tiny [GM$|$GM]}  
	 {'guess': 'bane', 'explanation': 'Based on the new feedback, we know that \'e\' is in the correct position but not at the end. The clue "hungry" might also relate to something that causes hunger or is associated with famine. "Bane" can mean a source of trouble or a cause of suffering, sometimes associated with hunger. It\'s a five-letter word with \'e\' in the third position, allowing us to test the other positions with this guess.'} 
	  } 
	   } 
	   } 
	 & & \\ 
 

    \theutterance \stepcounter{utterance}  

    & & & \multicolumn{2}{p{0.3\linewidth}}{\cellcolor[rgb]{0.95,0.95,0.95}{%
	\makecell[{{p{\linewidth}}}]{% 
	  \tt {\tiny [GM$|$GM]}  
	 Guesser Error: INVALID_WORD_LENGTH while parsing Player 1's (model: Qwen1.5-32B-Chat) response, retrying 
	  } 
	   } 
	   } 
	 & & \\ 
 

    \theutterance \stepcounter{utterance}  

    & & \multicolumn{4}{p{0.6\linewidth}}{\cellcolor[rgb]{0.9,0.9,0.9}{%
	\makecell[{{p{\linewidth}}}]{% 
	  \tt {\tiny [A$\langle$GM]}  
	 the word should have exactly 5 letters. Please try again 
	  } 
	   } 
	   } 
	 & \\ 
 

    \theutterance \stepcounter{utterance}  

    & \multicolumn{4}{p{0.6\linewidth}}{\cellcolor[rgb]{0.8,1,0.9}{%
	\makecell[{{p{\linewidth}}}]{% 
	  \tt {\tiny [A$\rangle$GM]}  
	 guess:peach\\ \tt explanation:Since 'e' is in the correct position and we're looking for a five-letter word related to the concept of hunger, "peach" could be a suitable guess. Peaches are a type of fruit that people might crave or feel hungry for. This word also allows us to test the other letters against the target word, providing more information with the next feedback.\\ \tt  
	  } 
	   } 
	   } 
	 & & \\ 
 

    \theutterance \stepcounter{utterance}  

    & & & \multicolumn{2}{p{0.3\linewidth}}{\cellcolor[rgb]{0.95,0.95,0.95}{%
	\makecell[{{p{\linewidth}}}]{% 
	  \tt {\tiny [GM$|$GM]}  
	 {'guess': 'peach', 'explanation': 'Since \'e\' is in the correct position and we\'re looking for a five-letter word related to the concept of hunger, "peach" could be a suitable guess. Peaches are a type of fruit that people might crave or feel hungry for. This word also allows us to test the other letters against the target word, providing more information with the next feedback.'} 
	  } 
	   } 
	   } 
	 & & \\ 
 

    \theutterance \stepcounter{utterance}  

    & & & \multicolumn{2}{p{0.3\linewidth}}{\cellcolor[rgb]{0.95,0.95,0.95}{%
	\makecell[{{p{\linewidth}}}]{% 
	  \tt {\tiny [GM$|$GM]}  
	 attempts: 3\\ \tt target_word = unfed\\ \tt target_word_clue = hungry\\ \tt guess: peach\\ \tt guess_feedback: p<red> e<yellow> a<red> c<red> h<red> 
	  } 
	   } 
	   } 
	 & & \\ 
 

    \theutterance \stepcounter{utterance}  

    & & \multicolumn{4}{p{0.6\linewidth}}{\cellcolor[rgb]{0.9,0.9,0.9}{%
	\makecell[{{p{\linewidth}}}]{% 
	  \tt {\tiny [A$\langle$GM]}  
	 guess_feedback: p<red> e<yellow> a<red> c<red> h<red> 
	  } 
	   } 
	   } 
	 & \\ 
 

    \theutterance \stepcounter{utterance}  

    & \multicolumn{4}{p{0.6\linewidth}}{\cellcolor[rgb]{0.8,1,0.9}{%
	\makecell[{{p{\linewidth}}}]{% 
	  \tt {\tiny [A$\rangle$GM]}  
	 guess:hunger\\ \tt explanation:Given the feedback, we now know that 'e' is in the correct position and 'u' is another correct letter, likely in the second position. Since the clue is "hungry," it makes sense to try the word "hunger" itself. This will confirm if our understanding of the correct letters and their positions is accurate, and it directly relates to the given clue. 
	  } 
	   } 
	   } 
	 & & \\ 
 

    \theutterance \stepcounter{utterance}  

    & & & \multicolumn{2}{p{0.3\linewidth}}{\cellcolor[rgb]{0.95,0.95,0.95}{%
	\makecell[{{p{\linewidth}}}]{% 
	  \tt {\tiny [GM$|$GM]}  
	 {'guess': 'hunger', 'explanation': 'Given the feedback, we now know that \'e\' is in the correct position and \'u\' is another correct letter, likely in the second position. Since the clue is "hungry," it makes sense to try the word "hunger" itself. This will confirm if our understanding of the correct letters and their positions is accurate, and it directly relates to the given clue.'} 
	  } 
	   } 
	   } 
	 & & \\ 
 

    \theutterance \stepcounter{utterance}  

    & & & \multicolumn{2}{p{0.3\linewidth}}{\cellcolor[rgb]{0.95,0.95,0.95}{%
	\makecell[{{p{\linewidth}}}]{% 
	  \tt {\tiny [GM$|$GM]}  
	 Guesser Error: INVALID_WORD_LENGTH while parsing Player 1's (model: Qwen1.5-32B-Chat) response, retrying 
	  } 
	   } 
	   } 
	 & & \\ 
 

    \theutterance \stepcounter{utterance}  

    & & \multicolumn{4}{p{0.6\linewidth}}{\cellcolor[rgb]{0.9,0.9,0.9}{%
	\makecell[{{p{\linewidth}}}]{% 
	  \tt {\tiny [A$\langle$GM]}  
	 the word should have exactly 5 letters. Please try again 
	  } 
	   } 
	   } 
	 & \\ 
 

    \theutterance \stepcounter{utterance}  

    & \multicolumn{4}{p{0.6\linewidth}}{\cellcolor[rgb]{0.8,1,0.9}{%
	\makecell[{{p{\linewidth}}}]{% 
	  \tt {\tiny [A$\rangle$GM]}  
	 guess:beast\\ \tt explanation:Considering the feedback and the clue "hungry," the word "beast" could be a good guess. It implies an animal that might be hungry, and it has five letters. This choice allows us to test the positions of 'b', 'e', 'a', 's', and 't' in relation to the target word, providing valuable information for future guesses. 
	  } 
	   } 
	   } 
	 & & \\ 
 

    \theutterance \stepcounter{utterance}  

    & & & \multicolumn{2}{p{0.3\linewidth}}{\cellcolor[rgb]{0.95,0.95,0.95}{%
	\makecell[{{p{\linewidth}}}]{% 
	  \tt {\tiny [GM$|$GM]}  
	 {'guess': 'beast', 'explanation': 'Considering the feedback and the clue "hungry," the word "beast" could be a good guess. It implies an animal that might be hungry, and it has five letters. This choice allows us to test the positions of \'b\', \'e\', \'a\', \'s\', and \'t\' in relation to the target word, providing valuable information for future guesses.'} 
	  } 
	   } 
	   } 
	 & & \\ 
 

    \theutterance \stepcounter{utterance}  

    & & & \multicolumn{2}{p{0.3\linewidth}}{\cellcolor[rgb]{0.95,0.95,0.95}{%
	\makecell[{{p{\linewidth}}}]{% 
	  \tt {\tiny [GM$|$GM]}  
	 attempts: 4\\ \tt target_word = unfed\\ \tt target_word_clue = hungry\\ \tt guess: beast\\ \tt guess_feedback: b<red> e<yellow> a<red> s<red> t<red> 
	  } 
	   } 
	   } 
	 & & \\ 
 

    \theutterance \stepcounter{utterance}  

    & & \multicolumn{4}{p{0.6\linewidth}}{\cellcolor[rgb]{0.9,0.9,0.9}{%
	\makecell[{{p{\linewidth}}}]{% 
	  \tt {\tiny [A$\langle$GM]}  
	 guess_feedback: b<red> e<yellow> a<red> s<red> t<red> 
	  } 
	   } 
	   } 
	 & \\ 
 

    \theutterance \stepcounter{utterance}  

    & \multicolumn{4}{p{0.6\linewidth}}{\cellcolor[rgb]{0.8,1,0.9}{%
	\makecell[{{p{\linewidth}}}]{% 
	  \tt {\tiny [A$\rangle$GM]}  
	 guess:feast\\ \tt explanation:From the latest feedback, we know that 'e' is in the correct position. The clue "hungry" suggests a connection to food or a meal. "Feast" is a five-letter word that refers to a large meal, often associated with satisfying hunger. This guess will help us determine the correctness of 'f', 'e', 'a', 's', and 't' in relation to the target word.\\ \tt <|im_start|>\\ \tt <|im_start|>\\ \tt <|im_start|><|endoftext|> 
	  } 
	   } 
	   } 
	 & & \\ 
 

    \theutterance \stepcounter{utterance}  

    & & & \multicolumn{2}{p{0.3\linewidth}}{\cellcolor[rgb]{0.95,0.95,0.95}{%
	\makecell[{{p{\linewidth}}}]{% 
	  \tt {\tiny [GM$|$GM]}  
	 {'guess': 'feast', 'explanation': 'From the latest feedback, we know that \'e\' is in the correct position. The clue "hungry" suggests a connection to food or a meal. "Feast" is a five-letter word that refers to a large meal, often associated with satisfying hunger. This guess will help us determine the correctness of \'f\', \'e\', \'a\', \'s\', and \'t\' in relation to the target word.'} 
	  } 
	   } 
	   } 
	 & & \\ 
 

    \theutterance \stepcounter{utterance}  

    & & & \multicolumn{2}{p{0.3\linewidth}}{\cellcolor[rgb]{0.95,0.95,0.95}{%
	\makecell[{{p{\linewidth}}}]{% 
	  \tt {\tiny [GM$|$GM]}  
	 attempts: 5\\ \tt target_word = unfed\\ \tt target_word_clue = hungry\\ \tt guess: feast\\ \tt guess_feedback: f<yellow> e<yellow> a<red> s<red> t<red> 
	  } 
	   } 
	   } 
	 & & \\ 
 

    \theutterance \stepcounter{utterance}  

    & & \multicolumn{4}{p{0.6\linewidth}}{\cellcolor[rgb]{0.9,0.9,0.9}{%
	\makecell[{{p{\linewidth}}}]{% 
	  \tt {\tiny [A$\langle$GM]}  
	 guess_feedback: f<yellow> e<yellow> a<red> s<red> t<red> 
	  } 
	   } 
	   } 
	 & \\ 
 

    \theutterance \stepcounter{utterance}  

    & \multicolumn{4}{p{0.6\linewidth}}{\cellcolor[rgb]{0.8,1,0.9}{%
	\makecell[{{p{\linewidth}}}]{% 
	  \tt {\tiny [A$\rangle$GM]}  
	 guess:tease\\ \tt explanation:With the updated feedback, we now know that 'f' is in the correct position, and 'e' is still in the correct position. The clue "hungry" might be interpreted metaphorically as someone being teased with the promise of food. "Tease" is a five-letter word that can imply this situation. This guess will help us identify the correct positions for 't', 'e', 'a', 's', and potentially other letters if 
	  } 
	   } 
	   } 
	 & & \\ 
 

    \theutterance \stepcounter{utterance}  

    & & & \multicolumn{2}{p{0.3\linewidth}}{\cellcolor[rgb]{0.95,0.95,0.95}{%
	\makecell[{{p{\linewidth}}}]{% 
	  \tt {\tiny [GM$|$GM]}  
	 {'guess': 'tease', 'explanation': 'With the updated feedback, we now know that \'f\' is in the correct position, and \'e\' is still in the correct position. The clue "hungry" might be interpreted metaphorically as someone being teased with the promise of food. "Tease" is a five-letter word that can imply this situation. This guess will help us identify the correct positions for \'t\', \'e\', \'a\', \'s\', and potentially other letters if'} 
	  } 
	   } 
	   } 
	 & & \\ 
 

    \theutterance \stepcounter{utterance}  

    & & & \multicolumn{2}{p{0.3\linewidth}}{\cellcolor[rgb]{0.95,0.95,0.95}{%
	\makecell[{{p{\linewidth}}}]{% 
	  \tt {\tiny [GM$|$GM]}  
	 attempts: 6\\ \tt target_word = unfed\\ \tt target_word_clue = hungry\\ \tt guess: tease\\ \tt guess_feedback: t<red> e<yellow> a<red> s<red> e<red> 
	  } 
	   } 
	   } 
	 & & \\ 
 

    \theutterance \stepcounter{utterance}  

    & & & \multicolumn{2}{p{0.3\linewidth}}{\cellcolor[rgb]{0.95,0.95,0.95}{%
	\makecell[{{p{\linewidth}}}]{% 
	  \tt {\tiny [GM$|$GM]}  
	 game_result = LOSS 
	  } 
	   } 
	   } 
	 & & \\ 
 

\end{supertabular}
}

\end{document}

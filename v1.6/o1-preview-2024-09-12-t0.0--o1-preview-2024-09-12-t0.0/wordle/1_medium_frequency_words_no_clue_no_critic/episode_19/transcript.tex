
\documentclass{article}
\usepackage{colortbl}
\usepackage{makecell}
\usepackage{multirow}
\usepackage{supertabular}

\begin{document}

\newcounter{utterance}

\twocolumn

{ \footnotesize  \setcounter{utterance}{1}
\setlength{\tabcolsep}{0pt}
\begin{supertabular}{c@{$\;$}|p{.15\linewidth}@{}p{.15\linewidth}p{.15\linewidth}p{.15\linewidth}p{.15\linewidth}p{.15\linewidth}}

    \# & $\;$A & \multicolumn{4}{c}{Game Master} & $\;\:$B\\
    \hline 

    \theutterance \stepcounter{utterance}  

    & & \multicolumn{4}{p{0.6\linewidth}}{\cellcolor[rgb]{0.9,0.9,0.9}{%
	\makecell[{{p{\linewidth}}}]{% 
	  \tt {\tiny [A$\langle$GM]}  
	 You are a language wizard who likes to guess words by using the given rules.\\ \tt \\ \tt Welcome to Wordle! You have six attempts to guess the target word, a valid English word of five lowercase letters (a-z). Please use the tags "guess:" and "explanation:" to provide a concise explanation for each guess.\\ \tt \\ \tt For instance, if your guess is "apple", your response should be\\ \tt guess: apple\\ \tt  explanation: this is a common five-letter English word, and I am starting my guess with this word.\\ \tt \\ \tt After each guess, your answer will be validated, and you will receive feedback indicating which letters are correct (green), which letters are correct but in the wrong position (yellow), and which letters are incorrect (red). This feedback can be useful in determining which letters to include or exclude in your next guess.\\ \tt \\ \tt For example, the feedback for "apple" might be:\\ \tt guess_feedback: a<yellow> p<yellow> p<green> l<yellow> e<red>\\ \tt \\ \tt The explanation should contain details about how the guess_feedback is used to arrive at a new guess.\\ \tt \\ \tt Let's begin with your first guess. 
	  } 
	   } 
	   } 
	 & \\ 
 

    \theutterance \stepcounter{utterance}  

    & \multicolumn{4}{p{0.6\linewidth}}{\cellcolor[rgb]{0.8,1,0.9}{%
	\makecell[{{p{\linewidth}}}]{% 
	  \tt {\tiny [A$\rangle$GM]}  
	 guess: arose\\ \tt explanation: I chose "arose" because it contains common letters and vowels, which should help identify letters in the target word. 
	  } 
	   } 
	   } 
	 & & \\ 
 

    \theutterance \stepcounter{utterance}  

    & & & \multicolumn{2}{p{0.3\linewidth}}{\cellcolor[rgb]{0.95,0.95,0.95}{%
	\makecell[{{p{\linewidth}}}]{% 
	  \tt {\tiny [GM$|$GM]}  
	 {'guess': 'arose', 'explanation': 'I chose "arose" because it contains common letters and vowels, which should help identify letters in the target word.'} 
	  } 
	   } 
	   } 
	 & & \\ 
 

    \theutterance \stepcounter{utterance}  

    & & & \multicolumn{2}{p{0.3\linewidth}}{\cellcolor[rgb]{0.95,0.95,0.95}{%
	\makecell[{{p{\linewidth}}}]{% 
	  \tt {\tiny [GM$|$GM]}  
	 attempts: 1\\ \tt target_word = cacao\\ \tt guess: arose\\ \tt guess_feedback: a<yellow> r<red> o<yellow> s<red> e<red> 
	  } 
	   } 
	   } 
	 & & \\ 
 

    \theutterance \stepcounter{utterance}  

    & & \multicolumn{4}{p{0.6\linewidth}}{\cellcolor[rgb]{0.9,0.9,0.9}{%
	\makecell[{{p{\linewidth}}}]{% 
	  \tt {\tiny [A$\langle$GM]}  
	 guess_feedback: a<yellow> r<red> o<yellow> s<red> e<red> 
	  } 
	   } 
	   } 
	 & \\ 
 

    \theutterance \stepcounter{utterance}  

    & \multicolumn{4}{p{0.6\linewidth}}{\cellcolor[rgb]{0.8,1,0.9}{%
	\makecell[{{p{\linewidth}}}]{% 
	  \tt {\tiny [A$\rangle$GM]}  
	 guess: vocal\\ \tt explanation: Based on the feedback, the letters 'a' and 'o' are in the word but not in the positions we previously guessed ('a' not in position 1, 'o' not in position 3). The letters 'r', 's', and 'e' are not in the word. I chose "vocal" because it places 'a' and 'o' in new positions, avoids the letters we know are incorrect, and introduces new consonants 'v', 'c', and 'l' to gather more information about the target word. 
	  } 
	   } 
	   } 
	 & & \\ 
 

    \theutterance \stepcounter{utterance}  

    & & & \multicolumn{2}{p{0.3\linewidth}}{\cellcolor[rgb]{0.95,0.95,0.95}{%
	\makecell[{{p{\linewidth}}}]{% 
	  \tt {\tiny [GM$|$GM]}  
	 {'guess': 'vocal', 'explanation': 'Based on the feedback, the letters \'a\' and \'o\' are in the word but not in the positions we previously guessed (\'a\' not in position 1, \'o\' not in position 3). The letters \'r\', \'s\', and \'e\' are not in the word. I chose "vocal" because it places \'a\' and \'o\' in new positions, avoids the letters we know are incorrect, and introduces new consonants \'v\', \'c\', and \'l\' to gather more information about the target word.'} 
	  } 
	   } 
	   } 
	 & & \\ 
 

    \theutterance \stepcounter{utterance}  

    & & & \multicolumn{2}{p{0.3\linewidth}}{\cellcolor[rgb]{0.95,0.95,0.95}{%
	\makecell[{{p{\linewidth}}}]{% 
	  \tt {\tiny [GM$|$GM]}  
	 attempts: 2\\ \tt target_word = cacao\\ \tt guess: vocal\\ \tt guess_feedback: v<red> o<yellow> c<yellow> a<yellow> l<red> 
	  } 
	   } 
	   } 
	 & & \\ 
 

    \theutterance \stepcounter{utterance}  

    & & \multicolumn{4}{p{0.6\linewidth}}{\cellcolor[rgb]{0.9,0.9,0.9}{%
	\makecell[{{p{\linewidth}}}]{% 
	  \tt {\tiny [A$\langle$GM]}  
	 guess_feedback: v<red> o<yellow> c<yellow> a<yellow> l<red> 
	  } 
	   } 
	   } 
	 & \\ 
 

    \theutterance \stepcounter{utterance}  

    & \multicolumn{4}{p{0.6\linewidth}}{\cellcolor[rgb]{0.8,1,0.9}{%
	\makecell[{{p{\linewidth}}}]{% 
	  \tt {\tiny [A$\rangle$GM]}  
	 guess:canto\\ \tt explanation:Based on the feedback, we know that 'a' is in the word but not in positions 1 or 4, so I placed it in position 2. The letter 'o' is in the word but not in positions 2 or 3, so I placed it in position 5. The letter 'c' is in the word but not in position 3, so I placed it in position 1. I chose "canto" because it fits these constraints and introduces new letters 'n' and 't', which haven't been tried yet, to gather more information about the target word. 
	  } 
	   } 
	   } 
	 & & \\ 
 

    \theutterance \stepcounter{utterance}  

    & & & \multicolumn{2}{p{0.3\linewidth}}{\cellcolor[rgb]{0.95,0.95,0.95}{%
	\makecell[{{p{\linewidth}}}]{% 
	  \tt {\tiny [GM$|$GM]}  
	 {'guess': 'canto', 'explanation': 'Based on the feedback, we know that \'a\' is in the word but not in positions 1 or 4, so I placed it in position 2. The letter \'o\' is in the word but not in positions 2 or 3, so I placed it in position 5. The letter \'c\' is in the word but not in position 3, so I placed it in position 1. I chose "canto" because it fits these constraints and introduces new letters \'n\' and \'t\', which haven\'t been tried yet, to gather more information about the target word.'} 
	  } 
	   } 
	   } 
	 & & \\ 
 

    \theutterance \stepcounter{utterance}  

    & & & \multicolumn{2}{p{0.3\linewidth}}{\cellcolor[rgb]{0.95,0.95,0.95}{%
	\makecell[{{p{\linewidth}}}]{% 
	  \tt {\tiny [GM$|$GM]}  
	 attempts: 3\\ \tt target_word = cacao\\ \tt guess: canto\\ \tt guess_feedback: c<green> a<green> n<red> t<red> o<green> 
	  } 
	   } 
	   } 
	 & & \\ 
 

    \theutterance \stepcounter{utterance}  

    & & \multicolumn{4}{p{0.6\linewidth}}{\cellcolor[rgb]{0.9,0.9,0.9}{%
	\makecell[{{p{\linewidth}}}]{% 
	  \tt {\tiny [A$\langle$GM]}  
	 guess_feedback: c<green> a<green> n<red> t<red> o<green> 
	  } 
	   } 
	   } 
	 & \\ 
 

    \theutterance \stepcounter{utterance}  

    & \multicolumn{4}{p{0.6\linewidth}}{\cellcolor[rgb]{0.8,1,0.9}{%
	\makecell[{{p{\linewidth}}}]{% 
	  \tt {\tiny [A$\rangle$GM]}  
	 guess:campo\\ \tt explanation:Based on the feedback so far, the target word has 'c' in position 1, 'a' in position 2, and 'o' in position 5, as indicated by the green letters from the last guess "canto". The letters 'n' and 't' are not in the word, as they were marked red in the previous guess.\\ \tt \\ \tt From the second guess "vocal", we know that 'c' is not in position 3 (since 'c' was yellow when placed in position 3), so we need a different letter in position 3. Additionally, the letters 'r', 's', 'e', 'v', 'l', 'n', and 't' are not in the word based on previous feedback.\\ \tt \\ \tt I chose "campo" because:\\ \tt \\ \tt - It places 'c', 'a', and 'o' in the confirmed green positions (1, 2, and 5 respectively).\\ \tt - It introduces new letters 'm' and 'p' in positions 3 and 4, which haven't been tried yet.\\ \tt - It meets all the known constraints from the feedback.\\ \tt \\ \tt This guess allows us to test new letters while aligning with the information we've gathered so far. 
	  } 
	   } 
	   } 
	 & & \\ 
 

    \theutterance \stepcounter{utterance}  

    & & & \multicolumn{2}{p{0.3\linewidth}}{\cellcolor[rgb]{0.95,0.95,0.95}{%
	\makecell[{{p{\linewidth}}}]{% 
	  \tt {\tiny [GM$|$GM]}  
	 {'guess': 'campo', 'explanation': 'Based on the feedback so far, the target word has \'c\' in position 1, \'a\' in position 2, and \'o\' in position 5, as indicated by the green letters from the last guess "canto". The letters \'n\' and \'t\' are not in the word, as they were marked red in the previous guess.'} 
	  } 
	   } 
	   } 
	 & & \\ 
 

    \theutterance \stepcounter{utterance}  

    & & & \multicolumn{2}{p{0.3\linewidth}}{\cellcolor[rgb]{0.95,0.95,0.95}{%
	\makecell[{{p{\linewidth}}}]{% 
	  \tt {\tiny [GM$|$GM]}  
	 attempts: 4\\ \tt target_word = cacao\\ \tt guess: campo\\ \tt guess_feedback: c<green> a<green> m<red> p<red> o<green> 
	  } 
	   } 
	   } 
	 & & \\ 
 

    \theutterance \stepcounter{utterance}  

    & & \multicolumn{4}{p{0.6\linewidth}}{\cellcolor[rgb]{0.9,0.9,0.9}{%
	\makecell[{{p{\linewidth}}}]{% 
	  \tt {\tiny [A$\langle$GM]}  
	 guess_feedback: c<green> a<green> m<red> p<red> o<green> 
	  } 
	   } 
	   } 
	 & \\ 
 

    \theutterance \stepcounter{utterance}  

    & \multicolumn{4}{p{0.6\linewidth}}{\cellcolor[rgb]{0.8,1,0.9}{%
	\makecell[{{p{\linewidth}}}]{% 
	  \tt {\tiny [A$\rangle$GM]}  
	 guess: cacao\\ \tt explanation: Based on the feedback, we have confirmed that 'c' is in position 1, 'a' is in position 2, and 'o' is in position 5 (all green letters). The letters not in the word are 'r', 's', 'e', 'v', 'l', 'n', 't', 'm', and 'p'. \\ \tt \\ \tt We are left to determine the letters in positions 3 and 4. The letters we've not yet tried and are still viable are 'd', 'f', 'g', 'h', 'i', 'j', 'k', 'q', 'u', 'w', 'x', 'y', and 'z'.\\ \tt \\ \tt Considering the word structure **c a _ _ o**, and the constraints that 'a' is not in positions 1 or 4 (from previous yellow feedback), the word **"cacao"** fits. Although it repeats the letter 'a' in position 4, which we earlier thought was excluded, it's possible that the initial feedback was influenced by duplicate letters. Therefore, I will guess "cacao" as it fits the known letter positions and uses remaining viable letters. 
	  } 
	   } 
	   } 
	 & & \\ 
 

    \theutterance \stepcounter{utterance}  

    & & & \multicolumn{2}{p{0.3\linewidth}}{\cellcolor[rgb]{0.95,0.95,0.95}{%
	\makecell[{{p{\linewidth}}}]{% 
	  \tt {\tiny [GM$|$GM]}  
	 {'guess': 'cacao', 'explanation': "Based on the feedback, we have confirmed that 'c' is in position 1, 'a' is in position 2, and 'o' is in position 5 (all green letters). The letters not in the word are 'r', 's', 'e', 'v', 'l', 'n', 't', 'm', and 'p'."} 
	  } 
	   } 
	   } 
	 & & \\ 
 

    \theutterance \stepcounter{utterance}  

    & & & \multicolumn{2}{p{0.3\linewidth}}{\cellcolor[rgb]{0.95,0.95,0.95}{%
	\makecell[{{p{\linewidth}}}]{% 
	  \tt {\tiny [GM$|$GM]}  
	 attempts: 5\\ \tt target_word = cacao\\ \tt guess: cacao\\ \tt guess_feedback: c<green> a<green> c<green> a<green> o<green> 
	  } 
	   } 
	   } 
	 & & \\ 
 

    \theutterance \stepcounter{utterance}  

    & & & \multicolumn{2}{p{0.3\linewidth}}{\cellcolor[rgb]{0.95,0.95,0.95}{%
	\makecell[{{p{\linewidth}}}]{% 
	  \tt {\tiny [GM$|$GM]}  
	 game_result = WIN 
	  } 
	   } 
	   } 
	 & & \\ 
 

\end{supertabular}
}

\end{document}

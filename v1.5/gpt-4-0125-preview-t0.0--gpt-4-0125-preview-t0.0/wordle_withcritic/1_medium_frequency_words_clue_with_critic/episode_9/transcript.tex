
\documentclass{article}
\usepackage{colortbl}
\usepackage{makecell}
\usepackage{multirow}
\usepackage{supertabular}

\begin{document}

\newcounter{utterance}

\twocolumn

{ \footnotesize  \setcounter{utterance}{1}
\setlength{\tabcolsep}{0pt}
\begin{supertabular}{c@{$\;$}|p{.15\linewidth}@{}p{.15\linewidth}p{.15\linewidth}p{.15\linewidth}p{.15\linewidth}p{.15\linewidth}}

    \# & $\;$A & \multicolumn{4}{c}{Game Master} & $\;\:$B\\
    \hline 

    \theutterance \stepcounter{utterance}  

    & & \multicolumn{4}{p{0.6\linewidth}}{\cellcolor[rgb]{0.9,0.9,0.9}{%
	\makecell[{{p{\linewidth}}}]{% 
	  \tt {\tiny [A$\langle$GM]}  
	 You are a language wizard who likes to guess words by using the given rules.\\ \tt \\ \tt Welcome to Wordle! You have six attempts to guess the target word, a valid English word of five lowercase letters (a-z). Please use the tags "guess:" and "explanation:" to provide a concise explanation for each guess.\\ \tt \\ \tt To help you make an informed guess, you will receive a clue for the word, such as\\ \tt clue: "snowy white"\\ \tt \\ \tt Here is an example guess based on the clue:\\ \tt guess: apple\\ \tt explanation: In the fairy tail Snow White, the girl is killed because she eats a poisoned apple. And the word apple has 5 letters.\\ \tt \\ \tt I will then indicate whether I agree or disagree with your guess and provide rationale, but agreeing with a guess does not confirm its correctness. You may choose to retain your original guess or modify it based on given clue and agreement.\\ \tt \\ \tt After each guess, your answer will be validated, and you will receive feedback indicating which letters are correct (green), which letters are correct but in the wrong position (yellow), and which letters are incorrect (red). This feedback can be useful in determining which letters to include or exclude in your next guess.\\ \tt \\ \tt For example, the feedback for "apple" might be:\\ \tt guess_feedback: a<yellow> p<yellow> p<green> l<yellow> e<red>\\ \tt \\ \tt Here are some rules to keep in mind:\\ \tt 1. Please do not generate any guess feedback or clue on your own\\ \tt 2. Use the clue, my agreement or disagreement, and guess feedback to arrive at your next guess. This should be reflected in your explanation as well.\\ \tt \\ \tt Let's begin with your first guess.\\ \tt \\ \tt clue:rule\\ \tt  
	  } 
	   } 
	   } 
	 & \\ 
 

    \theutterance \stepcounter{utterance}  

    & \multicolumn{4}{p{0.6\linewidth}}{\cellcolor[rgb]{0.8,1,0.9}{%
	\makecell[{{p{\linewidth}}}]{% 
	  \tt {\tiny [A$\rangle$GM]}  
	 guess: order\\ \tt explanation: The word "order" is associated with the concept of rules, as rules are often put in place to maintain order. Additionally, "order" fits the requirement of being a five-letter English word. 
	  } 
	   } 
	   } 
	 & & \\ 
 

    \theutterance \stepcounter{utterance}  

    & & & \multicolumn{2}{p{0.3\linewidth}}{\cellcolor[rgb]{0.95,0.95,0.95}{%
	\makecell[{{p{\linewidth}}}]{% 
	  \tt {\tiny [GM$|$GM]}  
	 {'guess': 'order', 'explanation': 'The word "order" is associated with the concept of rules, as rules are often put in place to maintain order. Additionally, "order" fits the requirement of being a five-letter English word.'} 
	  } 
	   } 
	   } 
	 & & \\ 
 

    \theutterance \stepcounter{utterance}  

    & & \multicolumn{4}{p{0.6\linewidth}}{\cellcolor[rgb]{0.9,0.9,0.9}{%
	\makecell[{{p{\linewidth}}}]{% 
	  \tt {\tiny [GM$\rangle$B]}  
	 I need your assistance with a word game in which we need to find a 5-letter word using a clue, a guess and an explanation for the guess.\\ \tt \\ \tt For example, if the clue is "snow white" and the guess is "apple", you will get the information like this:\\ \tt clue:snow white\\ \tt guess:apple\\ \tt explanation:the fairy tale Snow White involves the girl eating a poisoned apple, which has 5 letters.\\ \tt \\ \tt Your task is to either agree or disagree with my guess based on the given clue.\\ \tt \\ \tt If you disagree, I will make another guess, and if you agree, I will proceed in the game using the same guess.\\ \tt \\ \tt Initially, guess feedback is not available, but as the game proceeds, it becomes available. The guess feedback is not for current guess, but rather an indication of what letters can be present in the current guess. A letter highlighted in green indicates that it is a correct letter in the correct position, while yellow indicates that it is a correct letter in the wrong position, and red indicates a wrong letter. At this point, you should use all the provided information, including the clue, guess, explanation, and guess feedback, to decide whether the given guess matches the clue, and also stick to guess feedback or not.\\ \tt \\ \tt Please respond in lowercase letters and stick to this format:\\ \tt agreement:yes or no\\ \tt explanation:your reason for doing so\\ \tt \\ \tt Please note that you must not guess any word based on the riddle clue and stick to the given format while responding.\\ \tt \\ \tt Let's begin!\\ \tt \\ \tt clue:rule\\ \tt guess:order\\ \tt explanation:The word "order" is associated with the concept of rules, as rules are often put in place to maintain order. Additionally, "order" fits the requirement of being a five-letter English word.\\ \tt  
	  } 
	   } 
	   } 
	 & \\ 
 

    \theutterance \stepcounter{utterance}  

    & & & \multicolumn{4}{p{0.6\linewidth}}{\cellcolor[rgb]{1,0.85,0.72}{%
	\makecell[{{p{\linewidth}}}]{% 
	  \tt {\tiny [GM$\langle$B]}  
	 agreement:yes\\ \tt explanation:The guess "order" is relevant to the clue "rule" since rules are indeed established to maintain order, and it meets the criteria of being a five-letter word. 
	  } 
	   } 
	   } 
	  \\ 
 

    \theutterance \stepcounter{utterance}  

    & & & \multicolumn{2}{p{0.3\linewidth}}{\cellcolor[rgb]{0.95,0.95,0.95}{%
	\makecell[{{p{\linewidth}}}]{% 
	  \tt {\tiny [GM$|$GM]}  
	 {'agreement': 'yes', 'explanation': 'The guess "order" is relevant to the clue "rule" since rules are indeed established to maintain order, and it meets the criteria of being a five-letter word.'} 
	  } 
	   } 
	   } 
	 & & \\ 
 

    \theutterance \stepcounter{utterance}  

    & & & \multicolumn{2}{p{0.3\linewidth}}{\cellcolor[rgb]{0.95,0.95,0.95}{%
	\makecell[{{p{\linewidth}}}]{% 
	  \tt {\tiny [GM$|$GM]}  
	 Critic agrees with the Guesser -- Sharing the critic's explanation with the guesser 
	  } 
	   } 
	   } 
	 & & \\ 
 

    \theutterance \stepcounter{utterance}  

    & & \multicolumn{4}{p{0.6\linewidth}}{\cellcolor[rgb]{0.9,0.9,0.9}{%
	\makecell[{{p{\linewidth}}}]{% 
	  \tt {\tiny [A$\langle$GM]}  
	 clue:rule\\ \tt guess_agreement:yes\\ \tt agreement_explanation:The guess "order" is relevant to the clue "rule" since rules are indeed established to maintain order, and it meets the criteria of being a five-letter word.\\ \tt  
	  } 
	   } 
	   } 
	 & \\ 
 

    \theutterance \stepcounter{utterance}  

    & \multicolumn{4}{p{0.6\linewidth}}{\cellcolor[rgb]{0.8,1,0.9}{%
	\makecell[{{p{\linewidth}}}]{% 
	  \tt {\tiny [A$\rangle$GM]}  
	 guess: guide\\ \tt explanation: Given the agreement with the previous guess and the clue "rule," another word that fits within this context is "guide." Rules can serve as a guide for behavior or actions, and "guide" also meets the five-letter word requirement. 
	  } 
	   } 
	   } 
	 & & \\ 
 

    \theutterance \stepcounter{utterance}  

    & & & \multicolumn{2}{p{0.3\linewidth}}{\cellcolor[rgb]{0.95,0.95,0.95}{%
	\makecell[{{p{\linewidth}}}]{% 
	  \tt {\tiny [GM$|$GM]}  
	 {'guess': 'guide', 'explanation': 'Given the agreement with the previous guess and the clue "rule," another word that fits within this context is "guide." Rules can serve as a guide for behavior or actions, and "guide" also meets the five-letter word requirement.'} 
	  } 
	   } 
	   } 
	 & & \\ 
 

    \theutterance \stepcounter{utterance}  

    & & & \multicolumn{2}{p{0.3\linewidth}}{\cellcolor[rgb]{0.95,0.95,0.95}{%
	\makecell[{{p{\linewidth}}}]{% 
	  \tt {\tiny [GM$|$GM]}  
	 Change in player1's guess\\ \tt guess_before_critic_opinion: order\\ \tt                                                 critic_agreement: yes\\ \tt guess_after_critic_opinion: guide\\ \tt                                                 Proceeding with guess validation 
	  } 
	   } 
	   } 
	 & & \\ 
 

    \theutterance \stepcounter{utterance}  

    & & & \multicolumn{2}{p{0.3\linewidth}}{\cellcolor[rgb]{0.95,0.95,0.95}{%
	\makecell[{{p{\linewidth}}}]{% 
	  \tt {\tiny [GM$|$GM]}  
	 attempts: 1\\ \tt target_word = reign\\ \tt target_word_clue = rule\\ \tt guess: guide\\ \tt guess_feedback: g<yellow> u<red> i<green> d<red> e<yellow> 
	  } 
	   } 
	   } 
	 & & \\ 
 

    \theutterance \stepcounter{utterance}  

    & & \multicolumn{4}{p{0.6\linewidth}}{\cellcolor[rgb]{0.9,0.9,0.9}{%
	\makecell[{{p{\linewidth}}}]{% 
	  \tt {\tiny [A$\langle$GM]}  
	 guess_feedback: g<yellow> u<red> i<green> d<red> e<yellow> 
	  } 
	   } 
	   } 
	 & \\ 
 

    \theutterance \stepcounter{utterance}  

    & \multicolumn{4}{p{0.6\linewidth}}{\cellcolor[rgb]{0.8,1,0.9}{%
	\makecell[{{p{\linewidth}}}]{% 
	  \tt {\tiny [A$\rangle$GM]}  
	 guess: begin\\ \tt explanation: Based on the feedback, the letter 'i' is correctly placed in the second position, and the letters 'g' and 'e' are correct but in the wrong positions. The letters 'u' and 'd' are incorrect. "Begin" incorporates the 'g' and 'e' in new positions and introduces new letters to replace the incorrect ones, while also fitting the theme of initiating or setting rules. 
	  } 
	   } 
	   } 
	 & & \\ 
 

    \theutterance \stepcounter{utterance}  

    & & & \multicolumn{2}{p{0.3\linewidth}}{\cellcolor[rgb]{0.95,0.95,0.95}{%
	\makecell[{{p{\linewidth}}}]{% 
	  \tt {\tiny [GM$|$GM]}  
	 {'guess': 'begin', 'explanation': 'Based on the feedback, the letter \'i\' is correctly placed in the second position, and the letters \'g\' and \'e\' are correct but in the wrong positions. The letters \'u\' and \'d\' are incorrect. "Begin" incorporates the \'g\' and \'e\' in new positions and introduces new letters to replace the incorrect ones, while also fitting the theme of initiating or setting rules.'} 
	  } 
	   } 
	   } 
	 & & \\ 
 

    \theutterance \stepcounter{utterance}  

    & & \multicolumn{4}{p{0.6\linewidth}}{\cellcolor[rgb]{0.9,0.9,0.9}{%
	\makecell[{{p{\linewidth}}}]{% 
	  \tt {\tiny [GM$\rangle$B]}  
	 clue:rule\\ \tt guess:begin\\ \tt explanation:Based on the feedback, the letter 'i' is correctly placed in the second position, and the letters 'g' and 'e' are correct but in the wrong positions. The letters 'u' and 'd' are incorrect. "Begin" incorporates the 'g' and 'e' in new positions and introduces new letters to replace the incorrect ones, while also fitting the theme of initiating or setting rules.\\ \tt guess_feedback: g<yellow> u<red> i<green> d<red> e<yellow> 
	  } 
	   } 
	   } 
	 & \\ 
 

    \theutterance \stepcounter{utterance}  

    & & & \multicolumn{4}{p{0.6\linewidth}}{\cellcolor[rgb]{1,0.85,0.72}{%
	\makecell[{{p{\linewidth}}}]{% 
	  \tt {\tiny [GM$\langle$B]}  
	 agreement:no\\ \tt explanation:The feedback indicates that 'i' is correctly placed in the second position, and 'g' and 'e' are correct but in the wrong positions. However, "begin" places 'e' in the second position, which contradicts the feedback that 'i' is correct in the second position. Additionally, 'g' is not placed in a new position compared to the feedback. 
	  } 
	   } 
	   } 
	  \\ 
 

    \theutterance \stepcounter{utterance}  

    & & & \multicolumn{2}{p{0.3\linewidth}}{\cellcolor[rgb]{0.95,0.95,0.95}{%
	\makecell[{{p{\linewidth}}}]{% 
	  \tt {\tiny [GM$|$GM]}  
	 {'agreement': 'no', 'explanation': 'The feedback indicates that \'i\' is correctly placed in the second position, and \'g\' and \'e\' are correct but in the wrong positions. However, "begin" places \'e\' in the second position, which contradicts the feedback that \'i\' is correct in the second position. Additionally, \'g\' is not placed in a new position compared to the feedback.'} 
	  } 
	   } 
	   } 
	 & & \\ 
 

    \theutterance \stepcounter{utterance}  

    & & & \multicolumn{2}{p{0.3\linewidth}}{\cellcolor[rgb]{0.95,0.95,0.95}{%
	\makecell[{{p{\linewidth}}}]{% 
	  \tt {\tiny [GM$|$GM]}  
	 Critic disagrees with the Guesser -- Sharing the critic's explanation with the guesser 
	  } 
	   } 
	   } 
	 & & \\ 
 

    \theutterance \stepcounter{utterance}  

    & & \multicolumn{4}{p{0.6\linewidth}}{\cellcolor[rgb]{0.9,0.9,0.9}{%
	\makecell[{{p{\linewidth}}}]{% 
	  \tt {\tiny [A$\langle$GM]}  
	 clue:rule\\ \tt guess_agreement:no\\ \tt agreement_explanation:The feedback indicates that 'i' is correctly placed in the second position, and 'g' and 'e' are correct but in the wrong positions. However, "begin" places 'e' in the second position, which contradicts the feedback that 'i' is correct in the second position. Additionally, 'g' is not placed in a new position compared to the feedback.\\ \tt  
	  } 
	   } 
	   } 
	 & \\ 
 

    \theutterance \stepcounter{utterance}  

    & \multicolumn{4}{p{0.6\linewidth}}{\cellcolor[rgb]{0.8,1,0.9}{%
	\makecell[{{p{\linewidth}}}]{% 
	  \tt {\tiny [A$\rangle$GM]}  
	 guess:given\\ \tt explanation:Acknowledging the oversight in my previous guess, "given" correctly places 'i' in the second position as indicated by the feedback. It also places 'g' and 'e' in new positions, with 'g' now at the start and 'e' at the end, adhering to the feedback received. This word choice also aligns with the theme of "rule" in the sense that rules are "given" or established. 
	  } 
	   } 
	   } 
	 & & \\ 
 

    \theutterance \stepcounter{utterance}  

    & & & \multicolumn{2}{p{0.3\linewidth}}{\cellcolor[rgb]{0.95,0.95,0.95}{%
	\makecell[{{p{\linewidth}}}]{% 
	  \tt {\tiny [GM$|$GM]}  
	 {'guess': 'given', 'explanation': 'Acknowledging the oversight in my previous guess, "given" correctly places \'i\' in the second position as indicated by the feedback. It also places \'g\' and \'e\' in new positions, with \'g\' now at the start and \'e\' at the end, adhering to the feedback received. This word choice also aligns with the theme of "rule" in the sense that rules are "given" or established.'} 
	  } 
	   } 
	   } 
	 & & \\ 
 

    \theutterance \stepcounter{utterance}  

    & & & \multicolumn{2}{p{0.3\linewidth}}{\cellcolor[rgb]{0.95,0.95,0.95}{%
	\makecell[{{p{\linewidth}}}]{% 
	  \tt {\tiny [GM$|$GM]}  
	 Change in player1's guess\\ \tt guess_before_critic_opinion: begin\\ \tt                                                 critic_agreement: no\\ \tt guess_after_critic_opinion: given\\ \tt                                                 Proceeding with guess validation 
	  } 
	   } 
	   } 
	 & & \\ 
 

    \theutterance \stepcounter{utterance}  

    & & & \multicolumn{2}{p{0.3\linewidth}}{\cellcolor[rgb]{0.95,0.95,0.95}{%
	\makecell[{{p{\linewidth}}}]{% 
	  \tt {\tiny [GM$|$GM]}  
	 attempts: 2\\ \tt target_word = reign\\ \tt target_word_clue = rule\\ \tt guess: given\\ \tt guess_feedback: g<yellow> i<yellow> v<red> e<yellow> n<green> 
	  } 
	   } 
	   } 
	 & & \\ 
 

    \theutterance \stepcounter{utterance}  

    & & \multicolumn{4}{p{0.6\linewidth}}{\cellcolor[rgb]{0.9,0.9,0.9}{%
	\makecell[{{p{\linewidth}}}]{% 
	  \tt {\tiny [A$\langle$GM]}  
	 guess_feedback: g<yellow> i<yellow> v<red> e<yellow> n<green> 
	  } 
	   } 
	   } 
	 & \\ 
 

    \theutterance \stepcounter{utterance}  

    & \multicolumn{4}{p{0.6\linewidth}}{\cellcolor[rgb]{0.8,1,0.9}{%
	\makecell[{{p{\linewidth}}}]{% 
	  \tt {\tiny [A$\rangle$GM]}  
	 guess:reign\\ \tt explanation:With the feedback indicating that 'i', 'g', and 'e' are correct but in the wrong positions, and 'n' is correct and in the right position, "reign" fits this pattern. It places 'i' and 'e' in new positions, retains 'n' at the end, and moves 'g' to a different spot. The word "reign" is associated with rule, as it refers to the period during 
	  } 
	   } 
	   } 
	 & & \\ 
 

    \theutterance \stepcounter{utterance}  

    & & & \multicolumn{2}{p{0.3\linewidth}}{\cellcolor[rgb]{0.95,0.95,0.95}{%
	\makecell[{{p{\linewidth}}}]{% 
	  \tt {\tiny [GM$|$GM]}  
	 {'guess': 'reign', 'explanation': 'With the feedback indicating that \'i\', \'g\', and \'e\' are correct but in the wrong positions, and \'n\' is correct and in the right position, "reign" fits this pattern. It places \'i\' and \'e\' in new positions, retains \'n\' at the end, and moves \'g\' to a different spot. The word "reign" is associated with rule, as it refers to the period during'} 
	  } 
	   } 
	   } 
	 & & \\ 
 

    \theutterance \stepcounter{utterance}  

    & & \multicolumn{4}{p{0.6\linewidth}}{\cellcolor[rgb]{0.9,0.9,0.9}{%
	\makecell[{{p{\linewidth}}}]{% 
	  \tt {\tiny [GM$\rangle$B]}  
	 clue:rule\\ \tt guess:reign\\ \tt explanation:With the feedback indicating that 'i', 'g', and 'e' are correct but in the wrong positions, and 'n' is correct and in the right position, "reign" fits this pattern. It places 'i' and 'e' in new positions, retains 'n' at the end, and moves 'g' to a different spot. The word "reign" is associated with rule, as it refers to the period during\\ \tt guess_feedback: g<yellow> i<yellow> v<red> e<yellow> n<green> 
	  } 
	   } 
	   } 
	 & \\ 
 

    \theutterance \stepcounter{utterance}  

    & & & \multicolumn{4}{p{0.6\linewidth}}{\cellcolor[rgb]{1,0.85,0.72}{%
	\makecell[{{p{\linewidth}}}]{% 
	  \tt {\tiny [GM$\langle$B]}  
	 agreement:yes\\ \tt explanation:Based on the guess feedback, the placement of 'n' in the correct position at the end and the letters 'i', 'g', and 'e' being correct but previously in the wrong positions aligns with the new guess "reign". The word "reign" appropriately relates to the clue "rule", fitting the criteria of a 5-letter word associated with the concept of ruling or governance. 
	  } 
	   } 
	   } 
	  \\ 
 

    \theutterance \stepcounter{utterance}  

    & & & \multicolumn{2}{p{0.3\linewidth}}{\cellcolor[rgb]{0.95,0.95,0.95}{%
	\makecell[{{p{\linewidth}}}]{% 
	  \tt {\tiny [GM$|$GM]}  
	 {'agreement': 'yes', 'explanation': 'Based on the guess feedback, the placement of \'n\' in the correct position at the end and the letters \'i\', \'g\', and \'e\' being correct but previously in the wrong positions aligns with the new guess "reign". The word "reign" appropriately relates to the clue "rule", fitting the criteria of a 5-letter word associated with the concept of ruling or governance.'} 
	  } 
	   } 
	   } 
	 & & \\ 
 

    \theutterance \stepcounter{utterance}  

    & & & \multicolumn{2}{p{0.3\linewidth}}{\cellcolor[rgb]{0.95,0.95,0.95}{%
	\makecell[{{p{\linewidth}}}]{% 
	  \tt {\tiny [GM$|$GM]}  
	 Critic agrees with the Guesser -- Sharing the critic's explanation with the guesser 
	  } 
	   } 
	   } 
	 & & \\ 
 

    \theutterance \stepcounter{utterance}  

    & & \multicolumn{4}{p{0.6\linewidth}}{\cellcolor[rgb]{0.9,0.9,0.9}{%
	\makecell[{{p{\linewidth}}}]{% 
	  \tt {\tiny [A$\langle$GM]}  
	 clue:rule\\ \tt guess_agreement:yes\\ \tt agreement_explanation:Based on the guess feedback, the placement of 'n' in the correct position at the end and the letters 'i', 'g', and 'e' being correct but previously in the wrong positions aligns with the new guess "reign". The word "reign" appropriately relates to the clue "rule", fitting the criteria of a 5-letter word associated with the concept of ruling or governance.\\ \tt  
	  } 
	   } 
	   } 
	 & \\ 
 

    \theutterance \stepcounter{utterance}  

    & \multicolumn{4}{p{0.6\linewidth}}{\cellcolor[rgb]{0.8,1,0.9}{%
	\makecell[{{p{\linewidth}}}]{% 
	  \tt {\tiny [A$\rangle$GM]}  
	 Since the agreement confirms the alignment of the guess "reign" with the feedback and clue, and no further feedback on the correctness of this guess has been provided, I will maintain "reign" as my final guess based on the given information and the rules of the game. \\ \tt \\ \tt This decision is based on the understanding that "reign" accurately incorporates the letters 'i', 'g', 'e', and 'n' in positions that were indicated by the feedback, and it semantically align 
	  } 
	   } 
	   } 
	 & & \\ 
 

    \theutterance \stepcounter{utterance}  

    & & & \multicolumn{2}{p{0.3\linewidth}}{\cellcolor[rgb]{0.95,0.95,0.95}{%
	\makecell[{{p{\linewidth}}}]{% 
	  \tt {\tiny [GM$|$GM]}  
	 {'guess': 'INVALID_FORMAT', 'explanation': 'INVALID_FORMAT'} 
	  } 
	   } 
	   } 
	 & & \\ 
 

    \theutterance \stepcounter{utterance}  

    & & & \multicolumn{2}{p{0.3\linewidth}}{\cellcolor[rgb]{0.95,0.95,0.95}{%
	\makecell[{{p{\linewidth}}}]{% 
	  \tt {\tiny [GM$|$GM]}  
	 Guesser Error: INVALID_FORMAT while parsing Player 1's (model: gpt-4-0125-preview) response, retrying 
	  } 
	   } 
	   } 
	 & & \\ 
 

    \theutterance \stepcounter{utterance}  

    & & \multicolumn{4}{p{0.6\linewidth}}{\cellcolor[rgb]{0.9,0.9,0.9}{%
	\makecell[{{p{\linewidth}}}]{% 
	  \tt {\tiny [A$\langle$GM]}  
	 Provide your response only in this format:\\ \tt guess:word\\ \tt explanation:details.\\ \tt Please try again 
	  } 
	   } 
	   } 
	 & \\ 
 

    \theutterance \stepcounter{utterance}  

    & \multicolumn{4}{p{0.6\linewidth}}{\cellcolor[rgb]{0.8,1,0.9}{%
	\makecell[{{p{\linewidth}}}]{% 
	  \tt {\tiny [A$\rangle$GM]}  
	 guess:ingle\\ \tt explanation:Considering the feedback that 'i', 'g', 'e', and 'n' are correct letters with 'n' correctly placed at the end and 'i', 'g', 'e' needing new positions, "ingle" fits this criteria. It places 'i' at the beginning, 'g' and 'e' in new positions, and keeps 'n' at the end. While "ingle" might not directly relate to the concept of "rule 
	  } 
	   } 
	   } 
	 & & \\ 
 

    \theutterance \stepcounter{utterance}  

    & & & \multicolumn{2}{p{0.3\linewidth}}{\cellcolor[rgb]{0.95,0.95,0.95}{%
	\makecell[{{p{\linewidth}}}]{% 
	  \tt {\tiny [GM$|$GM]}  
	 {'guess': 'ingle', 'explanation': 'Considering the feedback that \'i\', \'g\', \'e\', and \'n\' are correct letters with \'n\' correctly placed at the end and \'i\', \'g\', \'e\' needing new positions, "ingle" fits this criteria. It places \'i\' at the beginning, \'g\' and \'e\' in new positions, and keeps \'n\' at the end. While "ingle" might not directly relate to the concept of "rule'} 
	  } 
	   } 
	   } 
	 & & \\ 
 

    \theutterance \stepcounter{utterance}  

    & & & \multicolumn{2}{p{0.3\linewidth}}{\cellcolor[rgb]{0.95,0.95,0.95}{%
	\makecell[{{p{\linewidth}}}]{% 
	  \tt {\tiny [GM$|$GM]}  
	 Change in player1's guess\\ \tt guess_before_critic_opinion: reign\\ \tt                                                 critic_agreement: yes\\ \tt guess_after_critic_opinion: ingle\\ \tt                                                 Proceeding with guess validation 
	  } 
	   } 
	   } 
	 & & \\ 
 

    \theutterance \stepcounter{utterance}  

    & & & \multicolumn{2}{p{0.3\linewidth}}{\cellcolor[rgb]{0.95,0.95,0.95}{%
	\makecell[{{p{\linewidth}}}]{% 
	  \tt {\tiny [GM$|$GM]}  
	 attempts: 3\\ \tt target_word = reign\\ \tt target_word_clue = rule\\ \tt guess: ingle\\ \tt guess_feedback: i<yellow> n<yellow> g<yellow> l<red> e<yellow> 
	  } 
	   } 
	   } 
	 & & \\ 
 

    \theutterance \stepcounter{utterance}  

    & & \multicolumn{4}{p{0.6\linewidth}}{\cellcolor[rgb]{0.9,0.9,0.9}{%
	\makecell[{{p{\linewidth}}}]{% 
	  \tt {\tiny [A$\langle$GM]}  
	 guess_feedback: i<yellow> n<yellow> g<yellow> l<red> e<yellow> 
	  } 
	   } 
	   } 
	 & \\ 
 

    \theutterance \stepcounter{utterance}  

    & \multicolumn{4}{p{0.6\linewidth}}{\cellcolor[rgb]{0.8,1,0.9}{%
	\makecell[{{p{\linewidth}}}]{% 
	  \tt {\tiny [A$\rangle$GM]}  
	 guess:genie\\ \tt explanation:Based on the feedback, all letters 'i', 'n', 'g', and 'e' are correct but were in the wrong positions in the word "ingle". The word "genie" rearranges these letters into a new configuration that fits the criteria, placing 'e' and 'i' in different positions, and introduces 'n' and 'g' also in new positions, potentially matching the feedback pattern while adhering to the clue "rule 
	  } 
	   } 
	   } 
	 & & \\ 
 

    \theutterance \stepcounter{utterance}  

    & & & \multicolumn{2}{p{0.3\linewidth}}{\cellcolor[rgb]{0.95,0.95,0.95}{%
	\makecell[{{p{\linewidth}}}]{% 
	  \tt {\tiny [GM$|$GM]}  
	 {'guess': 'genie', 'explanation': 'Based on the feedback, all letters \'i\', \'n\', \'g\', and \'e\' are correct but were in the wrong positions in the word "ingle". The word "genie" rearranges these letters into a new configuration that fits the criteria, placing \'e\' and \'i\' in different positions, and introduces \'n\' and \'g\' also in new positions, potentially matching the feedback pattern while adhering to the clue "rule'} 
	  } 
	   } 
	   } 
	 & & \\ 
 

    \theutterance \stepcounter{utterance}  

    & & \multicolumn{4}{p{0.6\linewidth}}{\cellcolor[rgb]{0.9,0.9,0.9}{%
	\makecell[{{p{\linewidth}}}]{% 
	  \tt {\tiny [GM$\rangle$B]}  
	 clue:rule\\ \tt guess:genie\\ \tt explanation:Based on the feedback, all letters 'i', 'n', 'g', and 'e' are correct but were in the wrong positions in the word "ingle". The word "genie" rearranges these letters into a new configuration that fits the criteria, placing 'e' and 'i' in different positions, and introduces 'n' and 'g' also in new positions, potentially matching the feedback pattern while adhering to the clue "rule\\ \tt guess_feedback: i<yellow> n<yellow> g<yellow> l<red> e<yellow> 
	  } 
	   } 
	   } 
	 & \\ 
 

    \theutterance \stepcounter{utterance}  

    & & & \multicolumn{4}{p{0.6\linewidth}}{\cellcolor[rgb]{1,0.85,0.72}{%
	\makecell[{{p{\linewidth}}}]{% 
	  \tt {\tiny [GM$\langle$B]}  
	 agreement:no\\ \tt explanation:the feedback indicates that 'i', 'n', 'g', and 'e' are correct letters but were in the wrong positions in the previous guess. However, the guess "genie" does not include a letter that could be in the correct position based on the feedback, as it places 'e' and 'i' in positions that were previously indicated as incorrect for them. Additionally, the clue "rule" does not directly relate to "genie" 
	  } 
	   } 
	   } 
	  \\ 
 

    \theutterance \stepcounter{utterance}  

    & & & \multicolumn{2}{p{0.3\linewidth}}{\cellcolor[rgb]{0.95,0.95,0.95}{%
	\makecell[{{p{\linewidth}}}]{% 
	  \tt {\tiny [GM$|$GM]}  
	 {'agreement': 'no', 'explanation': 'the feedback indicates that \'i\', \'n\', \'g\', and \'e\' are correct letters but were in the wrong positions in the previous guess. However, the guess "genie" does not include a letter that could be in the correct position based on the feedback, as it places \'e\' and \'i\' in positions that were previously indicated as incorrect for them. Additionally, the clue "rule" does not directly relate to "genie"'} 
	  } 
	   } 
	   } 
	 & & \\ 
 

    \theutterance \stepcounter{utterance}  

    & & & \multicolumn{2}{p{0.3\linewidth}}{\cellcolor[rgb]{0.95,0.95,0.95}{%
	\makecell[{{p{\linewidth}}}]{% 
	  \tt {\tiny [GM$|$GM]}  
	 Critic disagrees with the Guesser -- Sharing the critic's explanation with the guesser 
	  } 
	   } 
	   } 
	 & & \\ 
 

    \theutterance \stepcounter{utterance}  

    & & \multicolumn{4}{p{0.6\linewidth}}{\cellcolor[rgb]{0.9,0.9,0.9}{%
	\makecell[{{p{\linewidth}}}]{% 
	  \tt {\tiny [A$\langle$GM]}  
	 clue:rule\\ \tt guess_agreement:no\\ \tt agreement_explanation:the feedback indicates that 'i', 'n', 'g', and 'e' are correct letters but were in the wrong positions in the previous guess. However, the guess "genie" does not include a letter that could be in the correct position based on the feedback, as it places 'e' and 'i' in positions that were previously indicated as incorrect for them. Additionally, the clue "rule" does not directly relate to "genie"\\ \tt  
	  } 
	   } 
	   } 
	 & \\ 
 

    \theutterance \stepcounter{utterance}  

    & \multicolumn{4}{p{0.6\linewidth}}{\cellcolor[rgb]{0.8,1,0.9}{%
	\makecell[{{p{\linewidth}}}]{% 
	  \tt {\tiny [A$\rangle$GM]}  
	 guess:being\\ \tt explanation:Correcting the oversight in the previous guess and aligning with the feedback, "being" places 'i', 'n', 'g', and 'e' in new positions that have not been tried based on the previous feedback. This configuration adheres to the clue "rule" in a more abstract sense, as "being" can relate to the essence or existence that rules or principles may aim to address or govern. 
	  } 
	   } 
	   } 
	 & & \\ 
 

    \theutterance \stepcounter{utterance}  

    & & & \multicolumn{2}{p{0.3\linewidth}}{\cellcolor[rgb]{0.95,0.95,0.95}{%
	\makecell[{{p{\linewidth}}}]{% 
	  \tt {\tiny [GM$|$GM]}  
	 {'guess': 'being', 'explanation': 'Correcting the oversight in the previous guess and aligning with the feedback, "being" places \'i\', \'n\', \'g\', and \'e\' in new positions that have not been tried based on the previous feedback. This configuration adheres to the clue "rule" in a more abstract sense, as "being" can relate to the essence or existence that rules or principles may aim to address or govern.'} 
	  } 
	   } 
	   } 
	 & & \\ 
 

    \theutterance \stepcounter{utterance}  

    & & & \multicolumn{2}{p{0.3\linewidth}}{\cellcolor[rgb]{0.95,0.95,0.95}{%
	\makecell[{{p{\linewidth}}}]{% 
	  \tt {\tiny [GM$|$GM]}  
	 Change in player1's guess\\ \tt guess_before_critic_opinion: genie\\ \tt                                                 critic_agreement: no\\ \tt guess_after_critic_opinion: being\\ \tt                                                 Proceeding with guess validation 
	  } 
	   } 
	   } 
	 & & \\ 
 

    \theutterance \stepcounter{utterance}  

    & & & \multicolumn{2}{p{0.3\linewidth}}{\cellcolor[rgb]{0.95,0.95,0.95}{%
	\makecell[{{p{\linewidth}}}]{% 
	  \tt {\tiny [GM$|$GM]}  
	 attempts: 4\\ \tt target_word = reign\\ \tt target_word_clue = rule\\ \tt guess: being\\ \tt guess_feedback: b<red> e<green> i<green> n<yellow> g<yellow> 
	  } 
	   } 
	   } 
	 & & \\ 
 

    \theutterance \stepcounter{utterance}  

    & & \multicolumn{4}{p{0.6\linewidth}}{\cellcolor[rgb]{0.9,0.9,0.9}{%
	\makecell[{{p{\linewidth}}}]{% 
	  \tt {\tiny [A$\langle$GM]}  
	 guess_feedback: b<red> e<green> i<green> n<yellow> g<yellow> 
	  } 
	   } 
	   } 
	 & \\ 
 

    \theutterance \stepcounter{utterance}  

    & \multicolumn{4}{p{0.6\linewidth}}{\cellcolor[rgb]{0.8,1,0.9}{%
	\makecell[{{p{\linewidth}}}]{% 
	  \tt {\tiny [A$\rangle$GM]}  
	 guess:genii\\ \tt explanation:Given the feedback, 'e' and 'i' are correctly placed in the second and third positions, respectively. 'n' and 'g' are correct but in the wrong positions. "genii" rearranges 'n' and 'g' while keeping 'e' and 'i' in their correct spots, fitting the pattern suggested by the feedback and maintaining the five-letter requirement. 
	  } 
	   } 
	   } 
	 & & \\ 
 

    \theutterance \stepcounter{utterance}  

    & & & \multicolumn{2}{p{0.3\linewidth}}{\cellcolor[rgb]{0.95,0.95,0.95}{%
	\makecell[{{p{\linewidth}}}]{% 
	  \tt {\tiny [GM$|$GM]}  
	 {'guess': 'genii', 'explanation': 'Given the feedback, \'e\' and \'i\' are correctly placed in the second and third positions, respectively. \'n\' and \'g\' are correct but in the wrong positions. "genii" rearranges \'n\' and \'g\' while keeping \'e\' and \'i\' in their correct spots, fitting the pattern suggested by the feedback and maintaining the five-letter requirement.'} 
	  } 
	   } 
	   } 
	 & & \\ 
 

    \theutterance \stepcounter{utterance}  

    & & \multicolumn{4}{p{0.6\linewidth}}{\cellcolor[rgb]{0.9,0.9,0.9}{%
	\makecell[{{p{\linewidth}}}]{% 
	  \tt {\tiny [GM$\rangle$B]}  
	 clue:rule\\ \tt guess:genii\\ \tt explanation:Given the feedback, 'e' and 'i' are correctly placed in the second and third positions, respectively. 'n' and 'g' are correct but in the wrong positions. "genii" rearranges 'n' and 'g' while keeping 'e' and 'i' in their correct spots, fitting the pattern suggested by the feedback and maintaining the five-letter requirement.\\ \tt guess_feedback: b<red> e<green> i<green> n<yellow> g<yellow> 
	  } 
	   } 
	   } 
	 & \\ 
 

    \theutterance \stepcounter{utterance}  

    & & & \multicolumn{4}{p{0.6\linewidth}}{\cellcolor[rgb]{1,0.85,0.72}{%
	\makecell[{{p{\linewidth}}}]{% 
	  \tt {\tiny [GM$\langle$B]}  
	 agreement:yes\\ \tt explanation:The feedback and the explanation align perfectly, with 'e' and 'i' correctly placed in the second and third positions, and 'n' and 'g' being correct but previously in the wrong positions. "genii" correctly rearranges these letters according to the feedback, making it a valid guess based on the provided information. 
	  } 
	   } 
	   } 
	  \\ 
 

    \theutterance \stepcounter{utterance}  

    & & & \multicolumn{2}{p{0.3\linewidth}}{\cellcolor[rgb]{0.95,0.95,0.95}{%
	\makecell[{{p{\linewidth}}}]{% 
	  \tt {\tiny [GM$|$GM]}  
	 {'agreement': 'yes', 'explanation': 'The feedback and the explanation align perfectly, with \'e\' and \'i\' correctly placed in the second and third positions, and \'n\' and \'g\' being correct but previously in the wrong positions. "genii" correctly rearranges these letters according to the feedback, making it a valid guess based on the provided information.'} 
	  } 
	   } 
	   } 
	 & & \\ 
 

    \theutterance \stepcounter{utterance}  

    & & & \multicolumn{2}{p{0.3\linewidth}}{\cellcolor[rgb]{0.95,0.95,0.95}{%
	\makecell[{{p{\linewidth}}}]{% 
	  \tt {\tiny [GM$|$GM]}  
	 Critic agrees with the Guesser -- Sharing the critic's explanation with the guesser 
	  } 
	   } 
	   } 
	 & & \\ 
 

    \theutterance \stepcounter{utterance}  

    & & \multicolumn{4}{p{0.6\linewidth}}{\cellcolor[rgb]{0.9,0.9,0.9}{%
	\makecell[{{p{\linewidth}}}]{% 
	  \tt {\tiny [A$\langle$GM]}  
	 clue:rule\\ \tt guess_agreement:yes\\ \tt agreement_explanation:The feedback and the explanation align perfectly, with 'e' and 'i' correctly placed in the second and third positions, and 'n' and 'g' being correct but previously in the wrong positions. "genii" correctly rearranges these letters according to the feedback, making it a valid guess based on the provided information.\\ \tt  
	  } 
	   } 
	   } 
	 & \\ 
 

    \theutterance \stepcounter{utterance}  

    & \multicolumn{4}{p{0.6\linewidth}}{\cellcolor[rgb]{0.8,1,0.9}{%
	\makecell[{{p{\linewidth}}}]{% 
	  \tt {\tiny [A$\rangle$GM]}  
	 Since my role is to guess and not to confirm correctness beyond the guesses provided, I'll continue with the format as instructed:\\ \tt \\ \tt guess:inept\\ \tt explanation:Given the previous agreement and feedback, it seems there was a misunderstanding in my last guess. Considering 'e' and 'i' are correctly placed in the second and third positions and needing to correct the positions of 'n' and 'g', I mistakenly did not follow the clue and feedback accurately. "inept" does not align 
	  } 
	   } 
	   } 
	 & & \\ 
 

    \theutterance \stepcounter{utterance}  

    & & & \multicolumn{2}{p{0.3\linewidth}}{\cellcolor[rgb]{0.95,0.95,0.95}{%
	\makecell[{{p{\linewidth}}}]{% 
	  \tt {\tiny [GM$|$GM]}  
	 {'guess': 'inept', 'explanation': 'Given the previous agreement and feedback, it seems there was a misunderstanding in my last guess. Considering \'e\' and \'i\' are correctly placed in the second and third positions and needing to correct the positions of \'n\' and \'g\', I mistakenly did not follow the clue and feedback accurately. "inept" does not align'} 
	  } 
	   } 
	   } 
	 & & \\ 
 

    \theutterance \stepcounter{utterance}  

    & & & \multicolumn{2}{p{0.3\linewidth}}{\cellcolor[rgb]{0.95,0.95,0.95}{%
	\makecell[{{p{\linewidth}}}]{% 
	  \tt {\tiny [GM$|$GM]}  
	 Change in player1's guess\\ \tt guess_before_critic_opinion: genii\\ \tt                                                 critic_agreement: yes\\ \tt guess_after_critic_opinion: inept\\ \tt                                                 Proceeding with guess validation 
	  } 
	   } 
	   } 
	 & & \\ 
 

    \theutterance \stepcounter{utterance}  

    & & & \multicolumn{2}{p{0.3\linewidth}}{\cellcolor[rgb]{0.95,0.95,0.95}{%
	\makecell[{{p{\linewidth}}}]{% 
	  \tt {\tiny [GM$|$GM]}  
	 attempts: 5\\ \tt target_word = reign\\ \tt target_word_clue = rule\\ \tt guess: inept\\ \tt guess_feedback: i<yellow> n<yellow> e<yellow> p<red> t<red> 
	  } 
	   } 
	   } 
	 & & \\ 
 

    \theutterance \stepcounter{utterance}  

    & & \multicolumn{4}{p{0.6\linewidth}}{\cellcolor[rgb]{0.9,0.9,0.9}{%
	\makecell[{{p{\linewidth}}}]{% 
	  \tt {\tiny [A$\langle$GM]}  
	 guess_feedback: i<yellow> n<yellow> e<yellow> p<red> t<red> 
	  } 
	   } 
	   } 
	 & \\ 
 

    \theutterance \stepcounter{utterance}  

    & \multicolumn{4}{p{0.6\linewidth}}{\cellcolor[rgb]{0.8,1,0.9}{%
	\makecell[{{p{\linewidth}}}]{% 
	  \tt {\tiny [A$\rangle$GM]}  
	 guess:inept was an invalid guess based on the previous feedback and agreement, as it did not correctly follow the clues and feedback provided. Given the correct feedback and the correct positions of 'e' and 'i', and knowing 'n' and 'g' are correct but were in the wrong positions, a valid guess cannot be provided based on the incorrect path taken. My apologies for the confusion. Without accurate feedback to proceed from the last valid guess, further accurate guessing is not possible. 
	  } 
	   } 
	   } 
	 & & \\ 
 

    \theutterance \stepcounter{utterance}  

    & & & \multicolumn{2}{p{0.3\linewidth}}{\cellcolor[rgb]{0.95,0.95,0.95}{%
	\makecell[{{p{\linewidth}}}]{% 
	  \tt {\tiny [GM$|$GM]}  
	 {'guess': "inept was an invalid guess based on the previous feedback and agreement, as it did not correctly follow the clues and feedback provided. Given the correct feedback and the correct positions of 'e' and 'i', and knowing 'n' and 'g' are correct but were in the wrong positions, a valid guess cannot be provided based on the incorrect path taken. My apologies for the confusion. Without accurate feedback to proceed from the last valid guess, further accurate guessing is not possible.", 'explanation': 'INVALID_FORMAT'} 
	  } 
	   } 
	   } 
	 & & \\ 
 

    \theutterance \stepcounter{utterance}  

    & & & \multicolumn{2}{p{0.3\linewidth}}{\cellcolor[rgb]{0.95,0.95,0.95}{%
	\makecell[{{p{\linewidth}}}]{% 
	  \tt {\tiny [GM$|$GM]}  
	 Guesser Error: INVALID_WORD while parsing Player 1's (model: gpt-4-0125-preview) response, retrying 
	  } 
	   } 
	   } 
	 & & \\ 
 

    \theutterance \stepcounter{utterance}  

    & & \multicolumn{4}{p{0.6\linewidth}}{\cellcolor[rgb]{0.9,0.9,0.9}{%
	\makecell[{{p{\linewidth}}}]{% 
	  \tt {\tiny [A$\langle$GM]}  
	 the word should contain only alphabets. Please try again 
	  } 
	   } 
	   } 
	 & \\ 
 

    \theutterance \stepcounter{utterance}  

    & \multicolumn{4}{p{0.6\linewidth}}{\cellcolor[rgb]{0.8,1,0.9}{%
	\makecell[{{p{\linewidth}}}]{% 
	  \tt {\tiny [A$\rangle$GM]}  
	 guess:ingle\\ \tt explanation:Revisiting the feedback where 'i', 'n', 'g', and 'e' are correct but were in the wrong positions except for 'n' which was not correctly placed in "genii", and considering the correct feedback mechanism, it appears there was a misunderstanding in my processing of the feedback. "ingle" was previously guessed and did not align correctly with the feedback. However, based on the constraints and previous guesses, a correct guess cannot be made 
	  } 
	   } 
	   } 
	 & & \\ 
 

    \theutterance \stepcounter{utterance}  

    & & & \multicolumn{2}{p{0.3\linewidth}}{\cellcolor[rgb]{0.95,0.95,0.95}{%
	\makecell[{{p{\linewidth}}}]{% 
	  \tt {\tiny [GM$|$GM]}  
	 {'guess': 'ingle', 'explanation': 'Revisiting the feedback where \'i\', \'n\', \'g\', and \'e\' are correct but were in the wrong positions except for \'n\' which was not correctly placed in "genii", and considering the correct feedback mechanism, it appears there was a misunderstanding in my processing of the feedback. "ingle" was previously guessed and did not align correctly with the feedback. However, based on the constraints and previous guesses, a correct guess cannot be made'} 
	  } 
	   } 
	   } 
	 & & \\ 
 

    \theutterance \stepcounter{utterance}  

    & & \multicolumn{4}{p{0.6\linewidth}}{\cellcolor[rgb]{0.9,0.9,0.9}{%
	\makecell[{{p{\linewidth}}}]{% 
	  \tt {\tiny [GM$\rangle$B]}  
	 clue:rule\\ \tt guess:ingle\\ \tt explanation:Revisiting the feedback where 'i', 'n', 'g', and 'e' are correct but were in the wrong positions except for 'n' which was not correctly placed in "genii", and considering the correct feedback mechanism, it appears there was a misunderstanding in my processing of the feedback. "ingle" was previously guessed and did not align correctly with the feedback. However, based on the constraints and previous guesses, a correct guess cannot be made\\ \tt guess_feedback: i<yellow> n<yellow> e<yellow> p<red> t<red> 
	  } 
	   } 
	   } 
	 & \\ 
 

    \theutterance \stepcounter{utterance}  

    & & & \multicolumn{4}{p{0.6\linewidth}}{\cellcolor[rgb]{1,0.85,0.72}{%
	\makecell[{{p{\linewidth}}}]{% 
	  \tt {\tiny [GM$\langle$B]}  
	 agreement:no\\ \tt explanation:the feedback provided does not match the guess "ingle" as it was not previously discussed, and the explanation does not accurately follow from the feedback given. Additionally, the explanation incorrectly references a guess and feedback that were not part of the provided sequence. 
	  } 
	   } 
	   } 
	  \\ 
 

    \theutterance \stepcounter{utterance}  

    & & & \multicolumn{2}{p{0.3\linewidth}}{\cellcolor[rgb]{0.95,0.95,0.95}{%
	\makecell[{{p{\linewidth}}}]{% 
	  \tt {\tiny [GM$|$GM]}  
	 {'agreement': 'no', 'explanation': 'the feedback provided does not match the guess "ingle" as it was not previously discussed, and the explanation does not accurately follow from the feedback given. Additionally, the explanation incorrectly references a guess and feedback that were not part of the provided sequence.'} 
	  } 
	   } 
	   } 
	 & & \\ 
 

    \theutterance \stepcounter{utterance}  

    & & & \multicolumn{2}{p{0.3\linewidth}}{\cellcolor[rgb]{0.95,0.95,0.95}{%
	\makecell[{{p{\linewidth}}}]{% 
	  \tt {\tiny [GM$|$GM]}  
	 Critic disagrees with the Guesser -- Sharing the critic's explanation with the guesser 
	  } 
	   } 
	   } 
	 & & \\ 
 

    \theutterance \stepcounter{utterance}  

    & & \multicolumn{4}{p{0.6\linewidth}}{\cellcolor[rgb]{0.9,0.9,0.9}{%
	\makecell[{{p{\linewidth}}}]{% 
	  \tt {\tiny [A$\langle$GM]}  
	 clue:rule\\ \tt guess_agreement:no\\ \tt agreement_explanation:the feedback provided does not match the guess "ingle" as it was not previously discussed, and the explanation does not accurately follow from the feedback given. Additionally, the explanation incorrectly references a guess and feedback that were not part of the provided sequence.\\ \tt  
	  } 
	   } 
	   } 
	 & \\ 
 

    \theutterance \stepcounter{utterance}  

    & \multicolumn{4}{p{0.6\linewidth}}{\cellcolor[rgb]{0.8,1,0.9}{%
	\makecell[{{p{\linewidth}}}]{% 
	  \tt {\tiny [A$\rangle$GM]}  
	 I acknowledge the confusion and the error in my previous responses. Given the correct feedback and the constraints provided, without creating further confusion, I must correct my approach based on the valid feedback given:\\ \tt \\ \tt - 'e' and 'i' are in the correct positions in the word "being" (second and third letters, respectively).\\ \tt - 'n' and 'g' were indicated to be correct letters but in the wrong positions.\\ \tt \\ \tt Given this, and understanding that my previous guess "genii" was aligned 
	  } 
	   } 
	   } 
	 & & \\ 
 

    \theutterance \stepcounter{utterance}  

    & & & \multicolumn{2}{p{0.3\linewidth}}{\cellcolor[rgb]{0.95,0.95,0.95}{%
	\makecell[{{p{\linewidth}}}]{% 
	  \tt {\tiny [GM$|$GM]}  
	 {'guess': 'INVALID_FORMAT', 'explanation': 'INVALID_FORMAT'} 
	  } 
	   } 
	   } 
	 & & \\ 
 

    \theutterance \stepcounter{utterance}  

    & & & \multicolumn{2}{p{0.3\linewidth}}{\cellcolor[rgb]{0.95,0.95,0.95}{%
	\makecell[{{p{\linewidth}}}]{% 
	  \tt {\tiny [GM$|$GM]}  
	 Guesser Error: INVALID_FORMAT while parsing Player 1's (model: gpt-4-0125-preview) response, retrying 
	  } 
	   } 
	   } 
	 & & \\ 
 

    \theutterance \stepcounter{utterance}  

    & & \multicolumn{4}{p{0.6\linewidth}}{\cellcolor[rgb]{0.9,0.9,0.9}{%
	\makecell[{{p{\linewidth}}}]{% 
	  \tt {\tiny [A$\langle$GM]}  
	 Provide your response only in this format:\\ \tt guess:word\\ \tt explanation:details.\\ \tt Please try again 
	  } 
	   } 
	   } 
	 & \\ 
 

    \theutterance \stepcounter{utterance}  

    & \multicolumn{4}{p{0.6\linewidth}}{\cellcolor[rgb]{0.8,1,0.9}{%
	\makecell[{{p{\linewidth}}}]{% 
	  \tt {\tiny [A$\rangle$GM]}  
	 I apologize for the confusion in my previous responses. Given the correct feedback and the constraints provided, I will attempt a new guess based on the valid information:\\ \tt \\ \tt guess:ingle\\ \tt explanation:Given the feedback that 'i', 'n', 'g', and 'e' are correct but were in the wrong positions, and considering the clue "rule" alongside the requirement for a new guess that adheres to these constraints, I mistakenly repeated "ingle" without properly addressing the feedback. This was an 
	  } 
	   } 
	   } 
	 & & \\ 
 

    \theutterance \stepcounter{utterance}  

    & & & \multicolumn{2}{p{0.3\linewidth}}{\cellcolor[rgb]{0.95,0.95,0.95}{%
	\makecell[{{p{\linewidth}}}]{% 
	  \tt {\tiny [GM$|$GM]}  
	 {'guess': 'ingle', 'explanation': 'Given the feedback that \'i\', \'n\', \'g\', and \'e\' are correct but were in the wrong positions, and considering the clue "rule" alongside the requirement for a new guess that adheres to these constraints, I mistakenly repeated "ingle" without properly addressing the feedback. This was an'} 
	  } 
	   } 
	   } 
	 & & \\ 
 

    \theutterance \stepcounter{utterance}  

    & & & \multicolumn{2}{p{0.3\linewidth}}{\cellcolor[rgb]{0.95,0.95,0.95}{%
	\makecell[{{p{\linewidth}}}]{% 
	  \tt {\tiny [GM$|$GM]}  
	 No change in player1's guess\\ \tt guess_before_critic_opinion: ingle\\ \tt                                                 critic_agreement: no\\ \tt guess_after_critic_opinion: ingle\\ \tt                                                 Proceeding with guess validation 
	  } 
	   } 
	   } 
	 & & \\ 
 

    \theutterance \stepcounter{utterance}  

    & & & \multicolumn{2}{p{0.3\linewidth}}{\cellcolor[rgb]{0.95,0.95,0.95}{%
	\makecell[{{p{\linewidth}}}]{% 
	  \tt {\tiny [GM$|$GM]}  
	 attempts: 6\\ \tt target_word = reign\\ \tt target_word_clue = rule\\ \tt guess: ingle\\ \tt guess_feedback: i<yellow> n<yellow> g<yellow> l<red> e<yellow> 
	  } 
	   } 
	   } 
	 & & \\ 
 

    \theutterance \stepcounter{utterance}  

    & & & \multicolumn{2}{p{0.3\linewidth}}{\cellcolor[rgb]{0.95,0.95,0.95}{%
	\makecell[{{p{\linewidth}}}]{% 
	  \tt {\tiny [GM$|$GM]}  
	 game_result = LOSS 
	  } 
	   } 
	   } 
	 & & \\ 
 

\end{supertabular}
}

\end{document}

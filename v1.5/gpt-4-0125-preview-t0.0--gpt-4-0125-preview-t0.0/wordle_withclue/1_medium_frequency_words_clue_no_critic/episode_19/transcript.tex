
\documentclass{article}
\usepackage{colortbl}
\usepackage{makecell}
\usepackage{multirow}
\usepackage{supertabular}

\begin{document}

\newcounter{utterance}

\twocolumn

{ \footnotesize  \setcounter{utterance}{1}
\setlength{\tabcolsep}{0pt}
\begin{supertabular}{c@{$\;$}|p{.15\linewidth}@{}p{.15\linewidth}p{.15\linewidth}p{.15\linewidth}p{.15\linewidth}p{.15\linewidth}}

    \# & $\;$A & \multicolumn{4}{c}{Game Master} & $\;\:$B\\
    \hline 

    \theutterance \stepcounter{utterance}  

    & & \multicolumn{4}{p{0.6\linewidth}}{\cellcolor[rgb]{0.9,0.9,0.9}{%
	\makecell[{{p{\linewidth}}}]{% 
	  \tt {\tiny [A$\langle$GM]}  
	 You are a language wizard who likes to guess words by using the given rules.\\ \tt \\ \tt Welcome to Wordle! You have six attempts to guess the target word, a valid English word of five lowercase letters (a-z). Please use the tags "guess:" and "explanation:" to provide a concise explanation for each guess.\\ \tt \\ \tt  To help you make an informed guess, you will receive a clue for the word, such as\\ \tt clue: snowy white.\\ \tt \\ \tt Here is an example guess based on the clue:\\ \tt guess: apple\\ \tt explanation: In the fairy tail Snow White, the girl is killed because she eats a poisoned apple. And the word apple has 5 letters.\\ \tt \\ \tt After each guess, your answer will be validated, and you will receive feedback indicating which letters are correct (green), which letters are correct but in the wrong position (yellow), and which letters are incorrect (red). This feedback can be useful in determining which letters to include or exclude in your next guess.\\ \tt \\ \tt For example, the feedback for "apple" might be:\\ \tt guess_feedback: a<yellow> p<yellow> p<green> l<yellow> e<red>\\ \tt \\ \tt The explanation should contain details about how both the guess_feedback and clue work together in tandem to formulate a new guess.\\ \tt \\ \tt Let's begin with your first guess.\\ \tt clue:dessert bean\\ \tt  
	  } 
	   } 
	   } 
	 & \\ 
 

    \theutterance \stepcounter{utterance}  

    & \multicolumn{4}{p{0.6\linewidth}}{\cellcolor[rgb]{0.8,1,0.9}{%
	\makecell[{{p{\linewidth}}}]{% 
	  \tt {\tiny [A$\rangle$GM]}  
	 guess: cocoa\\ \tt explanation: Cocoa beans are often associated with desserts, especially chocolate, and "cocoa" is a five-letter word that fits the clue "dessert bean." 
	  } 
	   } 
	   } 
	 & & \\ 
 

    \theutterance \stepcounter{utterance}  

    & & & \multicolumn{2}{p{0.3\linewidth}}{\cellcolor[rgb]{0.95,0.95,0.95}{%
	\makecell[{{p{\linewidth}}}]{% 
	  \tt {\tiny [GM$|$GM]}  
	 {'guess': 'cocoa', 'explanation': 'Cocoa beans are often associated with desserts, especially chocolate, and "cocoa" is a five-letter word that fits the clue "dessert bean."'} 
	  } 
	   } 
	   } 
	 & & \\ 
 

    \theutterance \stepcounter{utterance}  

    & & & \multicolumn{2}{p{0.3\linewidth}}{\cellcolor[rgb]{0.95,0.95,0.95}{%
	\makecell[{{p{\linewidth}}}]{% 
	  \tt {\tiny [GM$|$GM]}  
	 attempts: 1\\ \tt target_word = cacao\\ \tt target_word_clue = dessert bean\\ \tt guess: cocoa\\ \tt guess_feedback: c<green> o<yellow> c<green> o<red> a<yellow> 
	  } 
	   } 
	   } 
	 & & \\ 
 

    \theutterance \stepcounter{utterance}  

    & & \multicolumn{4}{p{0.6\linewidth}}{\cellcolor[rgb]{0.9,0.9,0.9}{%
	\makecell[{{p{\linewidth}}}]{% 
	  \tt {\tiny [A$\langle$GM]}  
	 guess_feedback: c<green> o<yellow> c<green> o<red> a<yellow> 
	  } 
	   } 
	   } 
	 & \\ 
 

    \theutterance \stepcounter{utterance}  

    & \multicolumn{4}{p{0.6\linewidth}}{\cellcolor[rgb]{0.8,1,0.9}{%
	\makecell[{{p{\linewidth}}}]{% 
	  \tt {\tiny [A$\rangle$GM]}  
	 guess: cache\\ \tt explanation: Given the feedback, we know that 'c' is in the correct position at the start, and 'o' and 'a' are in the word but in the wrong positions. Since 'o' cannot be in the second or fourth position and 'a' cannot be in the fifth position, I'm guessing "cache" to test the placement of 'a' and to introduce new letters while adhering to the clue's theme, albeit loosely, as " 
	  } 
	   } 
	   } 
	 & & \\ 
 

    \theutterance \stepcounter{utterance}  

    & & & \multicolumn{2}{p{0.3\linewidth}}{\cellcolor[rgb]{0.95,0.95,0.95}{%
	\makecell[{{p{\linewidth}}}]{% 
	  \tt {\tiny [GM$|$GM]}  
	 {'guess': 'cache', 'explanation': 'Given the feedback, we know that \'c\' is in the correct position at the start, and \'o\' and \'a\' are in the word but in the wrong positions. Since \'o\' cannot be in the second or fourth position and \'a\' cannot be in the fifth position, I\'m guessing "cache" to test the placement of \'a\' and to introduce new letters while adhering to the clue\'s theme, albeit loosely, as "'} 
	  } 
	   } 
	   } 
	 & & \\ 
 

    \theutterance \stepcounter{utterance}  

    & & & \multicolumn{2}{p{0.3\linewidth}}{\cellcolor[rgb]{0.95,0.95,0.95}{%
	\makecell[{{p{\linewidth}}}]{% 
	  \tt {\tiny [GM$|$GM]}  
	 attempts: 2\\ \tt target_word = cacao\\ \tt target_word_clue = dessert bean\\ \tt guess: cache\\ \tt guess_feedback: c<green> a<green> c<green> h<red> e<red> 
	  } 
	   } 
	   } 
	 & & \\ 
 

    \theutterance \stepcounter{utterance}  

    & & \multicolumn{4}{p{0.6\linewidth}}{\cellcolor[rgb]{0.9,0.9,0.9}{%
	\makecell[{{p{\linewidth}}}]{% 
	  \tt {\tiny [A$\langle$GM]}  
	 guess_feedback: c<green> a<green> c<green> h<red> e<red> 
	  } 
	   } 
	   } 
	 & \\ 
 

    \theutterance \stepcounter{utterance}  

    & \multicolumn{4}{p{0.6\linewidth}}{\cellcolor[rgb]{0.8,1,0.9}{%
	\makecell[{{p{\linewidth}}}]{% 
	  \tt {\tiny [A$\rangle$GM]}  
	 guess:carob\\ \tt explanation:With the feedback showing that 'c', 'a', and 'c' are in the correct positions, we need to replace 'h' and 'e' with letters that could fit the "dessert bean" clue. "Carob" is a type of bean often used as a chocolate substitute in desserts, fitting the clue perfectly while also placing 'o' and 'b' in the new positions to see if they match. 
	  } 
	   } 
	   } 
	 & & \\ 
 

    \theutterance \stepcounter{utterance}  

    & & & \multicolumn{2}{p{0.3\linewidth}}{\cellcolor[rgb]{0.95,0.95,0.95}{%
	\makecell[{{p{\linewidth}}}]{% 
	  \tt {\tiny [GM$|$GM]}  
	 {'guess': 'carob', 'explanation': 'With the feedback showing that \'c\', \'a\', and \'c\' are in the correct positions, we need to replace \'h\' and \'e\' with letters that could fit the "dessert bean" clue. "Carob" is a type of bean often used as a chocolate substitute in desserts, fitting the clue perfectly while also placing \'o\' and \'b\' in the new positions to see if they match.'} 
	  } 
	   } 
	   } 
	 & & \\ 
 

    \theutterance \stepcounter{utterance}  

    & & & \multicolumn{2}{p{0.3\linewidth}}{\cellcolor[rgb]{0.95,0.95,0.95}{%
	\makecell[{{p{\linewidth}}}]{% 
	  \tt {\tiny [GM$|$GM]}  
	 attempts: 3\\ \tt target_word = cacao\\ \tt target_word_clue = dessert bean\\ \tt guess: carob\\ \tt guess_feedback: c<green> a<green> r<red> o<yellow> b<red> 
	  } 
	   } 
	   } 
	 & & \\ 
 

    \theutterance \stepcounter{utterance}  

    & & \multicolumn{4}{p{0.6\linewidth}}{\cellcolor[rgb]{0.9,0.9,0.9}{%
	\makecell[{{p{\linewidth}}}]{% 
	  \tt {\tiny [A$\langle$GM]}  
	 guess_feedback: c<green> a<green> r<red> o<yellow> b<red> 
	  } 
	   } 
	   } 
	 & \\ 
 

    \theutterance \stepcounter{utterance}  

    & \multicolumn{4}{p{0.6\linewidth}}{\cellcolor[rgb]{0.8,1,0.9}{%
	\makecell[{{p{\linewidth}}}]{% 
	  \tt {\tiny [A$\rangle$GM]}  
	 guess: cocoa\\ \tt explanation: Given the feedback, we know 'c' and 'a' are correctly placed at the beginning and end, respectively. The letter 'o' is in the word but in the wrong position, and it cannot be the fourth letter. Since 'r' and 'b' are incorrect, and we previously guessed 'o' in incorrect positions, placing 'o' in the second and third positions might fit, revisiting the initial guess with the correct placement based on 
	  } 
	   } 
	   } 
	 & & \\ 
 

    \theutterance \stepcounter{utterance}  

    & & & \multicolumn{2}{p{0.3\linewidth}}{\cellcolor[rgb]{0.95,0.95,0.95}{%
	\makecell[{{p{\linewidth}}}]{% 
	  \tt {\tiny [GM$|$GM]}  
	 {'guess': 'cocoa', 'explanation': "Given the feedback, we know 'c' and 'a' are correctly placed at the beginning and end, respectively. The letter 'o' is in the word but in the wrong position, and it cannot be the fourth letter. Since 'r' and 'b' are incorrect, and we previously guessed 'o' in incorrect positions, placing 'o' in the second and third positions might fit, revisiting the initial guess with the correct placement based on"} 
	  } 
	   } 
	   } 
	 & & \\ 
 

    \theutterance \stepcounter{utterance}  

    & & & \multicolumn{2}{p{0.3\linewidth}}{\cellcolor[rgb]{0.95,0.95,0.95}{%
	\makecell[{{p{\linewidth}}}]{% 
	  \tt {\tiny [GM$|$GM]}  
	 attempts: 4\\ \tt target_word = cacao\\ \tt target_word_clue = dessert bean\\ \tt guess: cocoa\\ \tt guess_feedback: c<green> o<yellow> c<green> o<red> a<yellow> 
	  } 
	   } 
	   } 
	 & & \\ 
 

    \theutterance \stepcounter{utterance}  

    & & \multicolumn{4}{p{0.6\linewidth}}{\cellcolor[rgb]{0.9,0.9,0.9}{%
	\makecell[{{p{\linewidth}}}]{% 
	  \tt {\tiny [A$\langle$GM]}  
	 guess_feedback: c<green> o<yellow> c<green> o<red> a<yellow> 
	  } 
	   } 
	   } 
	 & \\ 
 

    \theutterance \stepcounter{utterance}  

    & \multicolumn{4}{p{0.6\linewidth}}{\cellcolor[rgb]{0.8,1,0.9}{%
	\makecell[{{p{\linewidth}}}]{% 
	  \tt {\tiny [A$\rangle$GM]}  
	 guess: cocoa\\ \tt explanation: Given the corrected understanding of the feedback and the realization of an oversight in interpreting it, it's clear that the initial guess of "cocoa" actually aligns perfectly with the feedback and the clue "dessert bean." The repetition of the guess "cocoa" is to acknowledge the correct interpretation of the feedback, where 'c' and 'c' are correctly placed, and 'o' and 'a' are indeed part of the word but were initially placed 
	  } 
	   } 
	   } 
	 & & \\ 
 

    \theutterance \stepcounter{utterance}  

    & & & \multicolumn{2}{p{0.3\linewidth}}{\cellcolor[rgb]{0.95,0.95,0.95}{%
	\makecell[{{p{\linewidth}}}]{% 
	  \tt {\tiny [GM$|$GM]}  
	 {'guess': 'cocoa', 'explanation': 'Given the corrected understanding of the feedback and the realization of an oversight in interpreting it, it\'s clear that the initial guess of "cocoa" actually aligns perfectly with the feedback and the clue "dessert bean." The repetition of the guess "cocoa" is to acknowledge the correct interpretation of the feedback, where \'c\' and \'c\' are correctly placed, and \'o\' and \'a\' are indeed part of the word but were initially placed'} 
	  } 
	   } 
	   } 
	 & & \\ 
 

    \theutterance \stepcounter{utterance}  

    & & & \multicolumn{2}{p{0.3\linewidth}}{\cellcolor[rgb]{0.95,0.95,0.95}{%
	\makecell[{{p{\linewidth}}}]{% 
	  \tt {\tiny [GM$|$GM]}  
	 attempts: 5\\ \tt target_word = cacao\\ \tt target_word_clue = dessert bean\\ \tt guess: cocoa\\ \tt guess_feedback: c<green> o<yellow> c<green> o<red> a<yellow> 
	  } 
	   } 
	   } 
	 & & \\ 
 

    \theutterance \stepcounter{utterance}  

    & & \multicolumn{4}{p{0.6\linewidth}}{\cellcolor[rgb]{0.9,0.9,0.9}{%
	\makecell[{{p{\linewidth}}}]{% 
	  \tt {\tiny [A$\langle$GM]}  
	 guess_feedback: c<green> o<yellow> c<green> o<red> a<yellow> 
	  } 
	   } 
	   } 
	 & \\ 
 

    \theutterance \stepcounter{utterance}  

    & \multicolumn{4}{p{0.6\linewidth}}{\cellcolor[rgb]{0.8,1,0.9}{%
	\makecell[{{p{\linewidth}}}]{% 
	  \tt {\tiny [A$\rangle$GM]}  
	 I apologize for the confusion in my previous responses. Given the feedback and the rules, it seems there was a misunderstanding in my last explanation. However, based on the feedback provided:\\ \tt \\ \tt - 'c' is correct and in the right position at the beginning and again in the third position.\\ \tt - 'o' and 'a' are in the word but not in the second or fifth positions.\\ \tt \\ \tt Given this feedback, it appears there was an error in my process, as repeating "cocoa" would not 
	  } 
	   } 
	   } 
	 & & \\ 
 

    \theutterance \stepcounter{utterance}  

    & & & \multicolumn{2}{p{0.3\linewidth}}{\cellcolor[rgb]{0.95,0.95,0.95}{%
	\makecell[{{p{\linewidth}}}]{% 
	  \tt {\tiny [GM$|$GM]}  
	 {'guess': 'INVALID_FORMAT', 'explanation': 'INVALID_FORMAT'} 
	  } 
	   } 
	   } 
	 & & \\ 
 

    \theutterance \stepcounter{utterance}  

    & & & \multicolumn{2}{p{0.3\linewidth}}{\cellcolor[rgb]{0.95,0.95,0.95}{%
	\makecell[{{p{\linewidth}}}]{% 
	  \tt {\tiny [GM$|$GM]}  
	 Guesser Error: INVALID_FORMAT while parsing Player 1's (model: gpt-4-0125-preview) response, retrying 
	  } 
	   } 
	   } 
	 & & \\ 
 

    \theutterance \stepcounter{utterance}  

    & & \multicolumn{4}{p{0.6\linewidth}}{\cellcolor[rgb]{0.9,0.9,0.9}{%
	\makecell[{{p{\linewidth}}}]{% 
	  \tt {\tiny [A$\langle$GM]}  
	 Provide your response only in this format:\\ \tt guess:word\\ \tt explanation:details.\\ \tt Please try again 
	  } 
	   } 
	   } 
	 & \\ 
 

    \theutterance \stepcounter{utterance}  

    & \multicolumn{4}{p{0.6\linewidth}}{\cellcolor[rgb]{0.8,1,0.9}{%
	\makecell[{{p{\linewidth}}}]{% 
	  \tt {\tiny [A$\rangle$GM]}  
	 guess:cacon\\ \tt explanation:Given the feedback, 'c' is correct in the first and third positions, 'a' and 'o' are in the word but incorrectly placed. Since 'o' cannot be in the second or fourth position as per the feedback, and knowing 'a' also needs to be repositioned, "cacon" is an attempt to rearrange these letters into a new pattern, despite "cacon" not fitting the clue or being a valid English 
	  } 
	   } 
	   } 
	 & & \\ 
 

    \theutterance \stepcounter{utterance}  

    & & & \multicolumn{2}{p{0.3\linewidth}}{\cellcolor[rgb]{0.95,0.95,0.95}{%
	\makecell[{{p{\linewidth}}}]{% 
	  \tt {\tiny [GM$|$GM]}  
	 {'guess': 'cacon', 'explanation': 'Given the feedback, \'c\' is correct in the first and third positions, \'a\' and \'o\' are in the word but incorrectly placed. Since \'o\' cannot be in the second or fourth position as per the feedback, and knowing \'a\' also needs to be repositioned, "cacon" is an attempt to rearrange these letters into a new pattern, despite "cacon" not fitting the clue or being a valid English'} 
	  } 
	   } 
	   } 
	 & & \\ 
 

    \theutterance \stepcounter{utterance}  

    & & & \multicolumn{2}{p{0.3\linewidth}}{\cellcolor[rgb]{0.95,0.95,0.95}{%
	\makecell[{{p{\linewidth}}}]{% 
	  \tt {\tiny [GM$|$GM]}  
	 Guesser Error: NOT_VALID_ENGLISH_WORD while parsing Player 1's (model: gpt-4-0125-preview) response, retrying 
	  } 
	   } 
	   } 
	 & & \\ 
 

    \theutterance \stepcounter{utterance}  

    & & \multicolumn{4}{p{0.6\linewidth}}{\cellcolor[rgb]{0.9,0.9,0.9}{%
	\makecell[{{p{\linewidth}}}]{% 
	  \tt {\tiny [A$\langle$GM]}  
	 your guess is not a valid word for this game. Please try again 
	  } 
	   } 
	   } 
	 & \\ 
 

    \theutterance \stepcounter{utterance}  

    & \multicolumn{4}{p{0.6\linewidth}}{\cellcolor[rgb]{0.8,1,0.9}{%
	\makecell[{{p{\linewidth}}}]{% 
	  \tt {\tiny [A$\rangle$GM]}  
	 guess:cacao\\ \tt explanation:Considering the correct placements of 'c' and the presence of 'a' and 'o' but in the wrong positions as indicated by the feedback, the word "cacao" fits perfectly. 'Cacao' is the bean from which chocolate is made, directly aligning with the clue "dessert bean." This arrangement corrects the placement of 'a' and 'o' according to the feedback provided. 
	  } 
	   } 
	   } 
	 & & \\ 
 

    \theutterance \stepcounter{utterance}  

    & & & \multicolumn{2}{p{0.3\linewidth}}{\cellcolor[rgb]{0.95,0.95,0.95}{%
	\makecell[{{p{\linewidth}}}]{% 
	  \tt {\tiny [GM$|$GM]}  
	 {'guess': 'cacao', 'explanation': 'Considering the correct placements of \'c\' and the presence of \'a\' and \'o\' but in the wrong positions as indicated by the feedback, the word "cacao" fits perfectly. \'Cacao\' is the bean from which chocolate is made, directly aligning with the clue "dessert bean." This arrangement corrects the placement of \'a\' and \'o\' according to the feedback provided.'} 
	  } 
	   } 
	   } 
	 & & \\ 
 

    \theutterance \stepcounter{utterance}  

    & & & \multicolumn{2}{p{0.3\linewidth}}{\cellcolor[rgb]{0.95,0.95,0.95}{%
	\makecell[{{p{\linewidth}}}]{% 
	  \tt {\tiny [GM$|$GM]}  
	 attempts: 6\\ \tt target_word = cacao\\ \tt target_word_clue = dessert bean\\ \tt guess: cacao\\ \tt guess_feedback: c<green> a<green> c<green> a<green> o<green> 
	  } 
	   } 
	   } 
	 & & \\ 
 

    \theutterance \stepcounter{utterance}  

    & & & \multicolumn{2}{p{0.3\linewidth}}{\cellcolor[rgb]{0.95,0.95,0.95}{%
	\makecell[{{p{\linewidth}}}]{% 
	  \tt {\tiny [GM$|$GM]}  
	 game_result = WIN 
	  } 
	   } 
	   } 
	 & & \\ 
 

\end{supertabular}
}

\end{document}

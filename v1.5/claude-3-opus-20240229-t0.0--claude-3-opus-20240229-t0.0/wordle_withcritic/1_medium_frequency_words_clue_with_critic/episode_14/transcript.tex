
\documentclass{article}
\usepackage{colortbl}
\usepackage{makecell}
\usepackage{multirow}
\usepackage{supertabular}

\begin{document}

\newcounter{utterance}

\twocolumn

{ \footnotesize  \setcounter{utterance}{1}
\setlength{\tabcolsep}{0pt}
\begin{supertabular}{c@{$\;$}|p{.15\linewidth}@{}p{.15\linewidth}p{.15\linewidth}p{.15\linewidth}p{.15\linewidth}p{.15\linewidth}}

    \# & $\;$A & \multicolumn{4}{c}{Game Master} & $\;\:$B\\
    \hline 

    \theutterance \stepcounter{utterance}  

    & & \multicolumn{4}{p{0.6\linewidth}}{\cellcolor[rgb]{0.9,0.9,0.9}{%
	\makecell[{{p{\linewidth}}}]{% 
	  \tt {\tiny [A$\langle$GM]}  
	 You are a language wizard who likes to guess words by using the given rules.\\ \tt \\ \tt Welcome to Wordle! You have six attempts to guess the target word, a valid English word of five lowercase letters (a-z). Please use the tags "guess:" and "explanation:" to provide a concise explanation for each guess.\\ \tt \\ \tt To help you make an informed guess, you will receive a clue for the word, such as\\ \tt clue: "snowy white"\\ \tt \\ \tt Here is an example guess based on the clue:\\ \tt guess: apple\\ \tt explanation: In the fairy tail Snow White, the girl is killed because she eats a poisoned apple. And the word apple has 5 letters.\\ \tt \\ \tt I will then indicate whether I agree or disagree with your guess and provide rationale, but agreeing with a guess does not confirm its correctness. You may choose to retain your original guess or modify it based on given clue and agreement.\\ \tt \\ \tt After each guess, your answer will be validated, and you will receive feedback indicating which letters are correct (green), which letters are correct but in the wrong position (yellow), and which letters are incorrect (red). This feedback can be useful in determining which letters to include or exclude in your next guess.\\ \tt \\ \tt For example, the feedback for "apple" might be:\\ \tt guess_feedback: a<yellow> p<yellow> p<green> l<yellow> e<red>\\ \tt \\ \tt Here are some rules to keep in mind:\\ \tt 1. Please do not generate any guess feedback or clue on your own\\ \tt 2. Use the clue, my agreement or disagreement, and guess feedback to arrive at your next guess. This should be reflected in your explanation as well.\\ \tt \\ \tt Let's begin with your first guess.\\ \tt \\ \tt clue:bravado\\ \tt  
	  } 
	   } 
	   } 
	 & \\ 
 

    \theutterance \stepcounter{utterance}  

    & \multicolumn{4}{p{0.6\linewidth}}{\cellcolor[rgb]{0.8,1,0.9}{%
	\makecell[{{p{\linewidth}}}]{% 
	  \tt {\tiny [A$\rangle$GM]}  
	 guess: boast\\ \tt explanation: The clue "bravado" suggests a word related to showing off or displaying confidence. "Boast" fits this meaning and has five letters, making it a good initial guess based on the limited information provided. 
	  } 
	   } 
	   } 
	 & & \\ 
 

    \theutterance \stepcounter{utterance}  

    & & & \multicolumn{2}{p{0.3\linewidth}}{\cellcolor[rgb]{0.95,0.95,0.95}{%
	\makecell[{{p{\linewidth}}}]{% 
	  \tt {\tiny [GM$|$GM]}  
	 {'guess': 'boast', 'explanation': 'The clue "bravado" suggests a word related to showing off or displaying confidence. "Boast" fits this meaning and has five letters, making it a good initial guess based on the limited information provided.'} 
	  } 
	   } 
	   } 
	 & & \\ 
 

    \theutterance \stepcounter{utterance}  

    & & \multicolumn{4}{p{0.6\linewidth}}{\cellcolor[rgb]{0.9,0.9,0.9}{%
	\makecell[{{p{\linewidth}}}]{% 
	  \tt {\tiny [GM$\rangle$B]}  
	 I need your assistance with a word game in which we need to find a 5-letter word using a clue, a guess and an explanation for the guess.\\ \tt \\ \tt For example, if the clue is "snow white" and the guess is "apple", you will get the information like this:\\ \tt clue:snow white\\ \tt guess:apple\\ \tt explanation:the fairy tale Snow White involves the girl eating a poisoned apple, which has 5 letters.\\ \tt \\ \tt Your task is to either agree or disagree with my guess based on the given clue.\\ \tt \\ \tt If you disagree, I will make another guess, and if you agree, I will proceed in the game using the same guess.\\ \tt \\ \tt Initially, guess feedback is not available, but as the game proceeds, it becomes available. The guess feedback is not for current guess, but rather an indication of what letters can be present in the current guess. A letter highlighted in green indicates that it is a correct letter in the correct position, while yellow indicates that it is a correct letter in the wrong position, and red indicates a wrong letter. At this point, you should use all the provided information, including the clue, guess, explanation, and guess feedback, to decide whether the given guess matches the clue, and also stick to guess feedback or not.\\ \tt \\ \tt Please respond in lowercase letters and stick to this format:\\ \tt agreement:yes or no\\ \tt explanation:your reason for doing so\\ \tt \\ \tt Please note that you must not guess any word based on the riddle clue and stick to the given format while responding.\\ \tt \\ \tt Let's begin!\\ \tt \\ \tt clue:bravado\\ \tt guess:boast\\ \tt explanation:The clue "bravado" suggests a word related to showing off or displaying confidence. "Boast" fits this meaning and has five letters, making it a good initial guess based on the limited information provided.\\ \tt  
	  } 
	   } 
	   } 
	 & \\ 
 

    \theutterance \stepcounter{utterance}  

    & & & \multicolumn{4}{p{0.6\linewidth}}{\cellcolor[rgb]{1,0.85,0.72}{%
	\makecell[{{p{\linewidth}}}]{% 
	  \tt {\tiny [GM$\langle$B]}  
	 agreement:yes\\ \tt explanation:the guess "boast" aligns well with the clue "bravado," as both words relate to displaying confidence or showing off. with the limited information provided, "boast" is a reasonable initial guess that fits the clue and meets the 5-letter requirement. 
	  } 
	   } 
	   } 
	  \\ 
 

    \theutterance \stepcounter{utterance}  

    & & & \multicolumn{2}{p{0.3\linewidth}}{\cellcolor[rgb]{0.95,0.95,0.95}{%
	\makecell[{{p{\linewidth}}}]{% 
	  \tt {\tiny [GM$|$GM]}  
	 {'agreement': 'yes', 'explanation': 'the guess "boast" aligns well with the clue "bravado," as both words relate to displaying confidence or showing off. with the limited information provided, "boast" is a reasonable initial guess that fits the clue and meets the 5-letter requirement.'} 
	  } 
	   } 
	   } 
	 & & \\ 
 

    \theutterance \stepcounter{utterance}  

    & & & \multicolumn{2}{p{0.3\linewidth}}{\cellcolor[rgb]{0.95,0.95,0.95}{%
	\makecell[{{p{\linewidth}}}]{% 
	  \tt {\tiny [GM$|$GM]}  
	 Critic agrees with the Guesser -- Sharing the critic's explanation with the guesser 
	  } 
	   } 
	   } 
	 & & \\ 
 

    \theutterance \stepcounter{utterance}  

    & & \multicolumn{4}{p{0.6\linewidth}}{\cellcolor[rgb]{0.9,0.9,0.9}{%
	\makecell[{{p{\linewidth}}}]{% 
	  \tt {\tiny [A$\langle$GM]}  
	 clue:bravado\\ \tt guess_agreement:yes\\ \tt agreement_explanation:the guess "boast" aligns well with the clue "bravado," as both words relate to displaying confidence or showing off. with the limited information provided, "boast" is a reasonable initial guess that fits the clue and meets the 5-letter requirement.\\ \tt  
	  } 
	   } 
	   } 
	 & \\ 
 

    \theutterance \stepcounter{utterance}  

    & \multicolumn{4}{p{0.6\linewidth}}{\cellcolor[rgb]{0.8,1,0.9}{%
	\makecell[{{p{\linewidth}}}]{% 
	  \tt {\tiny [A$\rangle$GM]}  
	 guess: brag\\ \tt explanation: Since the previous guess "boast" aligns well with the clue "bravado," I'll continue exploring synonyms. "Brag" is another word that means to speak boastfully or show off, making it a suitable guess that fits the clue and has five letters. 
	  } 
	   } 
	   } 
	 & & \\ 
 

    \theutterance \stepcounter{utterance}  

    & & & \multicolumn{2}{p{0.3\linewidth}}{\cellcolor[rgb]{0.95,0.95,0.95}{%
	\makecell[{{p{\linewidth}}}]{% 
	  \tt {\tiny [GM$|$GM]}  
	 {'guess': 'brag', 'explanation': 'Since the previous guess "boast" aligns well with the clue "bravado," I\'ll continue exploring synonyms. "Brag" is another word that means to speak boastfully or show off, making it a suitable guess that fits the clue and has five letters.'} 
	  } 
	   } 
	   } 
	 & & \\ 
 

    \theutterance \stepcounter{utterance}  

    & & & \multicolumn{2}{p{0.3\linewidth}}{\cellcolor[rgb]{0.95,0.95,0.95}{%
	\makecell[{{p{\linewidth}}}]{% 
	  \tt {\tiny [GM$|$GM]}  
	 Guesser Error: INVALID_WORD_LENGTH while parsing Player 1's (model: claude-3-opus-20240229) response, retrying 
	  } 
	   } 
	   } 
	 & & \\ 
 

    \theutterance \stepcounter{utterance}  

    & & \multicolumn{4}{p{0.6\linewidth}}{\cellcolor[rgb]{0.9,0.9,0.9}{%
	\makecell[{{p{\linewidth}}}]{% 
	  \tt {\tiny [A$\langle$GM]}  
	 the word should have exactly 5 letters. Please try again 
	  } 
	   } 
	   } 
	 & \\ 
 

    \theutterance \stepcounter{utterance}  

    & \multicolumn{4}{p{0.6\linewidth}}{\cellcolor[rgb]{0.8,1,0.9}{%
	\makecell[{{p{\linewidth}}}]{% 
	  \tt {\tiny [A$\rangle$GM]}  
	 guess:vaunt\\ \tt explanation:Apologies for the oversight. Continuing with the theme of words related to boasting or showing off, "vaunt" is another synonym that fits the clue "bravado." It means to speak boastfully or proudly about oneself, and importantly, it has exactly five letters, making it a valid guess. 
	  } 
	   } 
	   } 
	 & & \\ 
 

    \theutterance \stepcounter{utterance}  

    & & & \multicolumn{2}{p{0.3\linewidth}}{\cellcolor[rgb]{0.95,0.95,0.95}{%
	\makecell[{{p{\linewidth}}}]{% 
	  \tt {\tiny [GM$|$GM]}  
	 {'guess': 'vaunt', 'explanation': 'Apologies for the oversight. Continuing with the theme of words related to boasting or showing off, "vaunt" is another synonym that fits the clue "bravado." It means to speak boastfully or proudly about oneself, and importantly, it has exactly five letters, making it a valid guess.'} 
	  } 
	   } 
	   } 
	 & & \\ 
 

    \theutterance \stepcounter{utterance}  

    & & & \multicolumn{2}{p{0.3\linewidth}}{\cellcolor[rgb]{0.95,0.95,0.95}{%
	\makecell[{{p{\linewidth}}}]{% 
	  \tt {\tiny [GM$|$GM]}  
	 Change in player1's guess\\ \tt guess_before_critic_opinion: boast\\ \tt                                                 critic_agreement: yes\\ \tt guess_after_critic_opinion: vaunt\\ \tt                                                 Proceeding with guess validation 
	  } 
	   } 
	   } 
	 & & \\ 
 

    \theutterance \stepcounter{utterance}  

    & & & \multicolumn{2}{p{0.3\linewidth}}{\cellcolor[rgb]{0.95,0.95,0.95}{%
	\makecell[{{p{\linewidth}}}]{% 
	  \tt {\tiny [GM$|$GM]}  
	 attempts: 1\\ \tt target_word = swash\\ \tt target_word_clue = bravado\\ \tt guess: vaunt\\ \tt guess_feedback: v<red> a<yellow> u<red> n<red> t<red> 
	  } 
	   } 
	   } 
	 & & \\ 
 

    \theutterance \stepcounter{utterance}  

    & & \multicolumn{4}{p{0.6\linewidth}}{\cellcolor[rgb]{0.9,0.9,0.9}{%
	\makecell[{{p{\linewidth}}}]{% 
	  \tt {\tiny [A$\langle$GM]}  
	 guess_feedback: v<red> a<yellow> u<red> n<red> t<red> 
	  } 
	   } 
	   } 
	 & \\ 
 

    \theutterance \stepcounter{utterance}  

    & \multicolumn{4}{p{0.6\linewidth}}{\cellcolor[rgb]{0.8,1,0.9}{%
	\makecell[{{p{\linewidth}}}]{% 
	  \tt {\tiny [A$\rangle$GM]}  
	 guess: swank\\ \tt explanation: Based on the feedback, 'a' is the only correct letter from the previous guess "vaunt," but it's in the wrong position. I'll focus on words with 'a' in a different position, particularly the second or third letter. "Swank" means to show off or display confidently, aligning with the clue "bravado." It has the letter 'a' in the third position and introduces new letters 
	  } 
	   } 
	   } 
	 & & \\ 
 

    \theutterance \stepcounter{utterance}  

    & & & \multicolumn{2}{p{0.3\linewidth}}{\cellcolor[rgb]{0.95,0.95,0.95}{%
	\makecell[{{p{\linewidth}}}]{% 
	  \tt {\tiny [GM$|$GM]}  
	 {'guess': 'swank', 'explanation': 'Based on the feedback, \'a\' is the only correct letter from the previous guess "vaunt," but it\'s in the wrong position. I\'ll focus on words with \'a\' in a different position, particularly the second or third letter. "Swank" means to show off or display confidently, aligning with the clue "bravado." It has the letter \'a\' in the third position and introduces new letters'} 
	  } 
	   } 
	   } 
	 & & \\ 
 

    \theutterance \stepcounter{utterance}  

    & & \multicolumn{4}{p{0.6\linewidth}}{\cellcolor[rgb]{0.9,0.9,0.9}{%
	\makecell[{{p{\linewidth}}}]{% 
	  \tt {\tiny [GM$\rangle$B]}  
	 clue:bravado\\ \tt guess:swank\\ \tt explanation:Based on the feedback, 'a' is the only correct letter from the previous guess "vaunt," but it's in the wrong position. I'll focus on words with 'a' in a different position, particularly the second or third letter. "Swank" means to show off or display confidently, aligning with the clue "bravado." It has the letter 'a' in the third position and introduces new letters\\ \tt guess_feedback: v<red> a<yellow> u<red> n<red> t<red> 
	  } 
	   } 
	   } 
	 & \\ 
 

    \theutterance \stepcounter{utterance}  

    & & & \multicolumn{4}{p{0.6\linewidth}}{\cellcolor[rgb]{1,0.85,0.72}{%
	\makecell[{{p{\linewidth}}}]{% 
	  \tt {\tiny [GM$\langle$B]}  
	 agreement:no\\ \tt explanation:while "swank" fits the meaning of "bravado" and has the 'a' in a different position as indicated by the previous guess feedback, it does not include the 'a' in the correct position according to the new guess feedback for "vaunt." the feedback shows that 'a' should be the second letter, but "swank" has 'a' as the third letter. therefore, "swank" cannot be 
	  } 
	   } 
	   } 
	  \\ 
 

    \theutterance \stepcounter{utterance}  

    & & & \multicolumn{2}{p{0.3\linewidth}}{\cellcolor[rgb]{0.95,0.95,0.95}{%
	\makecell[{{p{\linewidth}}}]{% 
	  \tt {\tiny [GM$|$GM]}  
	 {'agreement': 'no', 'explanation': 'while "swank" fits the meaning of "bravado" and has the \'a\' in a different position as indicated by the previous guess feedback, it does not include the \'a\' in the correct position according to the new guess feedback for "vaunt." the feedback shows that \'a\' should be the second letter, but "swank" has \'a\' as the third letter. therefore, "swank" cannot be'} 
	  } 
	   } 
	   } 
	 & & \\ 
 

    \theutterance \stepcounter{utterance}  

    & & & \multicolumn{2}{p{0.3\linewidth}}{\cellcolor[rgb]{0.95,0.95,0.95}{%
	\makecell[{{p{\linewidth}}}]{% 
	  \tt {\tiny [GM$|$GM]}  
	 Critic disagrees with the Guesser -- Sharing the critic's explanation with the guesser 
	  } 
	   } 
	   } 
	 & & \\ 
 

    \theutterance \stepcounter{utterance}  

    & & \multicolumn{4}{p{0.6\linewidth}}{\cellcolor[rgb]{0.9,0.9,0.9}{%
	\makecell[{{p{\linewidth}}}]{% 
	  \tt {\tiny [A$\langle$GM]}  
	 clue:bravado\\ \tt guess_agreement:no\\ \tt agreement_explanation:while "swank" fits the meaning of "bravado" and has the 'a' in a different position as indicated by the previous guess feedback, it does not include the 'a' in the correct position according to the new guess feedback for "vaunt." the feedback shows that 'a' should be the second letter, but "swank" has 'a' as the third letter. therefore, "swank" cannot be\\ \tt  
	  } 
	   } 
	   } 
	 & \\ 
 

    \theutterance \stepcounter{utterance}  

    & \multicolumn{4}{p{0.6\linewidth}}{\cellcolor[rgb]{0.8,1,0.9}{%
	\makecell[{{p{\linewidth}}}]{% 
	  \tt {\tiny [A$\rangle$GM]}  
	 guess:brash\\ \tt explanation:Thank you for the clarification. Based on the feedback for "vaunt," the letter 'a' should be in the second position. "Brash" means bold and self-assertive, often in an impudent way, which aligns with the clue "bravado." It has the letter 'a' in the correct second position and introduces new letters that haven't been tested yet, making it a strong guess considering the 
	  } 
	   } 
	   } 
	 & & \\ 
 

    \theutterance \stepcounter{utterance}  

    & & & \multicolumn{2}{p{0.3\linewidth}}{\cellcolor[rgb]{0.95,0.95,0.95}{%
	\makecell[{{p{\linewidth}}}]{% 
	  \tt {\tiny [GM$|$GM]}  
	 {'guess': 'brash', 'explanation': 'Thank you for the clarification. Based on the feedback for "vaunt," the letter \'a\' should be in the second position. "Brash" means bold and self-assertive, often in an impudent way, which aligns with the clue "bravado." It has the letter \'a\' in the correct second position and introduces new letters that haven\'t been tested yet, making it a strong guess considering the'} 
	  } 
	   } 
	   } 
	 & & \\ 
 

    \theutterance \stepcounter{utterance}  

    & & & \multicolumn{2}{p{0.3\linewidth}}{\cellcolor[rgb]{0.95,0.95,0.95}{%
	\makecell[{{p{\linewidth}}}]{% 
	  \tt {\tiny [GM$|$GM]}  
	 Change in player1's guess\\ \tt guess_before_critic_opinion: swank\\ \tt                                                 critic_agreement: no\\ \tt guess_after_critic_opinion: brash\\ \tt                                                 Proceeding with guess validation 
	  } 
	   } 
	   } 
	 & & \\ 
 

    \theutterance \stepcounter{utterance}  

    & & & \multicolumn{2}{p{0.3\linewidth}}{\cellcolor[rgb]{0.95,0.95,0.95}{%
	\makecell[{{p{\linewidth}}}]{% 
	  \tt {\tiny [GM$|$GM]}  
	 attempts: 2\\ \tt target_word = swash\\ \tt target_word_clue = bravado\\ \tt guess: brash\\ \tt guess_feedback: b<red> r<red> a<green> s<yellow> h<green> 
	  } 
	   } 
	   } 
	 & & \\ 
 

    \theutterance \stepcounter{utterance}  

    & & \multicolumn{4}{p{0.6\linewidth}}{\cellcolor[rgb]{0.9,0.9,0.9}{%
	\makecell[{{p{\linewidth}}}]{% 
	  \tt {\tiny [A$\langle$GM]}  
	 guess_feedback: b<red> r<red> a<green> s<yellow> h<green> 
	  } 
	   } 
	   } 
	 & \\ 
 

    \theutterance \stepcounter{utterance}  

    & \multicolumn{4}{p{0.6\linewidth}}{\cellcolor[rgb]{0.8,1,0.9}{%
	\makecell[{{p{\linewidth}}}]{% 
	  \tt {\tiny [A$\rangle$GM]}  
	 guess:flash\\ \tt explanation:The feedback for "brash" confirms that 'a' is in the correct second position, 's' is in the word but in the wrong position, and 'h' is in the correct fifth position. Building on this information, "flash" fits the clue "bravado" as it can mean a sudden and often ostentatious display, which aligns with showing off. It keeps the 'a' in the second position 
	  } 
	   } 
	   } 
	 & & \\ 
 

    \theutterance \stepcounter{utterance}  

    & & & \multicolumn{2}{p{0.3\linewidth}}{\cellcolor[rgb]{0.95,0.95,0.95}{%
	\makecell[{{p{\linewidth}}}]{% 
	  \tt {\tiny [GM$|$GM]}  
	 {'guess': 'flash', 'explanation': 'The feedback for "brash" confirms that \'a\' is in the correct second position, \'s\' is in the word but in the wrong position, and \'h\' is in the correct fifth position. Building on this information, "flash" fits the clue "bravado" as it can mean a sudden and often ostentatious display, which aligns with showing off. It keeps the \'a\' in the second position'} 
	  } 
	   } 
	   } 
	 & & \\ 
 

    \theutterance \stepcounter{utterance}  

    & & \multicolumn{4}{p{0.6\linewidth}}{\cellcolor[rgb]{0.9,0.9,0.9}{%
	\makecell[{{p{\linewidth}}}]{% 
	  \tt {\tiny [GM$\rangle$B]}  
	 clue:bravado\\ \tt guess:flash\\ \tt explanation:The feedback for "brash" confirms that 'a' is in the correct second position, 's' is in the word but in the wrong position, and 'h' is in the correct fifth position. Building on this information, "flash" fits the clue "bravado" as it can mean a sudden and often ostentatious display, which aligns with showing off. It keeps the 'a' in the second position\\ \tt guess_feedback: b<red> r<red> a<green> s<yellow> h<green> 
	  } 
	   } 
	   } 
	 & \\ 
 

    \theutterance \stepcounter{utterance}  

    & & & \multicolumn{4}{p{0.6\linewidth}}{\cellcolor[rgb]{1,0.85,0.72}{%
	\makecell[{{p{\linewidth}}}]{% 
	  \tt {\tiny [GM$\langle$B]}  
	 agreement:no\\ \tt explanation:while "flash" fits the clue "bravado" and keeps the 'a' in the second position and 'h' in the fifth position as indicated by the previous feedback, it does not include the 's' which was marked as a correct letter in the wrong position. based on the feedback, the correct word should have 'a' in the second position, 's' somewhere other than the fourth position, and 'h' 
	  } 
	   } 
	   } 
	  \\ 
 

    \theutterance \stepcounter{utterance}  

    & & & \multicolumn{2}{p{0.3\linewidth}}{\cellcolor[rgb]{0.95,0.95,0.95}{%
	\makecell[{{p{\linewidth}}}]{% 
	  \tt {\tiny [GM$|$GM]}  
	 {'agreement': 'no', 'explanation': 'while "flash" fits the clue "bravado" and keeps the \'a\' in the second position and \'h\' in the fifth position as indicated by the previous feedback, it does not include the \'s\' which was marked as a correct letter in the wrong position. based on the feedback, the correct word should have \'a\' in the second position, \'s\' somewhere other than the fourth position, and \'h\''} 
	  } 
	   } 
	   } 
	 & & \\ 
 

    \theutterance \stepcounter{utterance}  

    & & & \multicolumn{2}{p{0.3\linewidth}}{\cellcolor[rgb]{0.95,0.95,0.95}{%
	\makecell[{{p{\linewidth}}}]{% 
	  \tt {\tiny [GM$|$GM]}  
	 Critic disagrees with the Guesser -- Sharing the critic's explanation with the guesser 
	  } 
	   } 
	   } 
	 & & \\ 
 

    \theutterance \stepcounter{utterance}  

    & & \multicolumn{4}{p{0.6\linewidth}}{\cellcolor[rgb]{0.9,0.9,0.9}{%
	\makecell[{{p{\linewidth}}}]{% 
	  \tt {\tiny [A$\langle$GM]}  
	 clue:bravado\\ \tt guess_agreement:no\\ \tt agreement_explanation:while "flash" fits the clue "bravado" and keeps the 'a' in the second position and 'h' in the fifth position as indicated by the previous feedback, it does not include the 's' which was marked as a correct letter in the wrong position. based on the feedback, the correct word should have 'a' in the second position, 's' somewhere other than the fourth position, and 'h'\\ \tt  
	  } 
	   } 
	   } 
	 & \\ 
 

    \theutterance \stepcounter{utterance}  

    & \multicolumn{4}{p{0.6\linewidth}}{\cellcolor[rgb]{0.8,1,0.9}{%
	\makecell[{{p{\linewidth}}}]{% 
	  \tt {\tiny [A$\rangle$GM]}  
	 guess:slash\\ \tt explanation:Apologies for missing the 's' in the previous guess. Taking into account the feedback for "brash," the correct word should have 'a' in the second position, 's' somewhere other than the fourth position, and 'h' in the fifth position. "Slash" means to make a sweeping or violent cut, which could be seen as a bold or brazen action, loosely fitting the clue "bravado 
	  } 
	   } 
	   } 
	 & & \\ 
 

    \theutterance \stepcounter{utterance}  

    & & & \multicolumn{2}{p{0.3\linewidth}}{\cellcolor[rgb]{0.95,0.95,0.95}{%
	\makecell[{{p{\linewidth}}}]{% 
	  \tt {\tiny [GM$|$GM]}  
	 {'guess': 'slash', 'explanation': 'Apologies for missing the \'s\' in the previous guess. Taking into account the feedback for "brash," the correct word should have \'a\' in the second position, \'s\' somewhere other than the fourth position, and \'h\' in the fifth position. "Slash" means to make a sweeping or violent cut, which could be seen as a bold or brazen action, loosely fitting the clue "bravado'} 
	  } 
	   } 
	   } 
	 & & \\ 
 

    \theutterance \stepcounter{utterance}  

    & & & \multicolumn{2}{p{0.3\linewidth}}{\cellcolor[rgb]{0.95,0.95,0.95}{%
	\makecell[{{p{\linewidth}}}]{% 
	  \tt {\tiny [GM$|$GM]}  
	 Change in player1's guess\\ \tt guess_before_critic_opinion: flash\\ \tt                                                 critic_agreement: no\\ \tt guess_after_critic_opinion: slash\\ \tt                                                 Proceeding with guess validation 
	  } 
	   } 
	   } 
	 & & \\ 
 

    \theutterance \stepcounter{utterance}  

    & & & \multicolumn{2}{p{0.3\linewidth}}{\cellcolor[rgb]{0.95,0.95,0.95}{%
	\makecell[{{p{\linewidth}}}]{% 
	  \tt {\tiny [GM$|$GM]}  
	 attempts: 3\\ \tt target_word = swash\\ \tt target_word_clue = bravado\\ \tt guess: slash\\ \tt guess_feedback: s<green> l<red> a<green> s<green> h<green> 
	  } 
	   } 
	   } 
	 & & \\ 
 

    \theutterance \stepcounter{utterance}  

    & & \multicolumn{4}{p{0.6\linewidth}}{\cellcolor[rgb]{0.9,0.9,0.9}{%
	\makecell[{{p{\linewidth}}}]{% 
	  \tt {\tiny [A$\langle$GM]}  
	 guess_feedback: s<green> l<red> a<green> s<green> h<green> 
	  } 
	   } 
	   } 
	 & \\ 
 

    \theutterance \stepcounter{utterance}  

    & \multicolumn{4}{p{0.6\linewidth}}{\cellcolor[rgb]{0.8,1,0.9}{%
	\makecell[{{p{\linewidth}}}]{% 
	  \tt {\tiny [A$\rangle$GM]}  
	 Congratulations! The feedback for "slash" shows that all letters are in the correct positions, indicating that "slash" is the correct answer. The word "slash" fits the clue "bravado" as it can be associated with bold, brazen, or aggressive actions, which align with the concept of showing off or displaying confidence.\\ \tt \\ \tt Throughout the guessing process, we built upon the information provided by the clue and the feedback for each guess. We started with words closely 
	  } 
	   } 
	   } 
	 & & \\ 
 

    \theutterance \stepcounter{utterance}  

    & & & \multicolumn{2}{p{0.3\linewidth}}{\cellcolor[rgb]{0.95,0.95,0.95}{%
	\makecell[{{p{\linewidth}}}]{% 
	  \tt {\tiny [GM$|$GM]}  
	 {'guess': 'INVALID_FORMAT', 'explanation': 'INVALID_FORMAT'} 
	  } 
	   } 
	   } 
	 & & \\ 
 

    \theutterance \stepcounter{utterance}  

    & & & \multicolumn{2}{p{0.3\linewidth}}{\cellcolor[rgb]{0.95,0.95,0.95}{%
	\makecell[{{p{\linewidth}}}]{% 
	  \tt {\tiny [GM$|$GM]}  
	 Guesser Error: INVALID_FORMAT while parsing Player 1's (model: claude-3-opus-20240229) response, retrying 
	  } 
	   } 
	   } 
	 & & \\ 
 

    \theutterance \stepcounter{utterance}  

    & & \multicolumn{4}{p{0.6\linewidth}}{\cellcolor[rgb]{0.9,0.9,0.9}{%
	\makecell[{{p{\linewidth}}}]{% 
	  \tt {\tiny [A$\langle$GM]}  
	 Provide your response only in this format:\\ \tt guess:word\\ \tt explanation:details.\\ \tt Please try again 
	  } 
	   } 
	   } 
	 & \\ 
 

    \theutterance \stepcounter{utterance}  

    & \multicolumn{4}{p{0.6\linewidth}}{\cellcolor[rgb]{0.8,1,0.9}{%
	\makecell[{{p{\linewidth}}}]{% 
	  \tt {\tiny [A$\rangle$GM]}  
	 guess:slash\\ \tt explanation:The feedback for "slash" shows that all letters are in the correct positions, forming the word "slash." While "slash" may not be the most direct synonym for "bravado," it can be associated with bold, brazen, or aggressive actions, which aligns with the clue. As all letters are marked green, indicating they are correct and in the right positions, "slash" is likely the target word.\\ \tt  
	  } 
	   } 
	   } 
	 & & \\ 
 

    \theutterance \stepcounter{utterance}  

    & & & \multicolumn{2}{p{0.3\linewidth}}{\cellcolor[rgb]{0.95,0.95,0.95}{%
	\makecell[{{p{\linewidth}}}]{% 
	  \tt {\tiny [GM$|$GM]}  
	 {'guess': 'slash', 'explanation': 'The feedback for "slash" shows that all letters are in the correct positions, forming the word "slash." While "slash" may not be the most direct synonym for "bravado," it can be associated with bold, brazen, or aggressive actions, which aligns with the clue. As all letters are marked green, indicating they are correct and in the right positions, "slash" is likely the target word.'} 
	  } 
	   } 
	   } 
	 & & \\ 
 

    \theutterance \stepcounter{utterance}  

    & & \multicolumn{4}{p{0.6\linewidth}}{\cellcolor[rgb]{0.9,0.9,0.9}{%
	\makecell[{{p{\linewidth}}}]{% 
	  \tt {\tiny [GM$\rangle$B]}  
	 clue:bravado\\ \tt guess:slash\\ \tt explanation:The feedback for "slash" shows that all letters are in the correct positions, forming the word "slash." While "slash" may not be the most direct synonym for "bravado," it can be associated with bold, brazen, or aggressive actions, which aligns with the clue. As all letters are marked green, indicating they are correct and in the right positions, "slash" is likely the target word.\\ \tt guess_feedback: s<green> l<red> a<green> s<green> h<green> 
	  } 
	   } 
	   } 
	 & \\ 
 

    \theutterance \stepcounter{utterance}  

    & & & \multicolumn{4}{p{0.6\linewidth}}{\cellcolor[rgb]{1,0.85,0.72}{%
	\makecell[{{p{\linewidth}}}]{% 
	  \tt {\tiny [GM$\langle$B]}  
	 agreement:no\\ \tt explanation:while "slash" fits the pattern indicated by the guess feedback, it doesn't align well with the meaning of "bravado." bravado refers to a bold, confident, and often exaggerated display of courage or self-importance, whereas "slash" is more associated with cutting actions or aggressive reduction. i suggest considering other words that match the letter pattern and more closely relate to the clue's meaning of showing off or displaying confidence. 
	  } 
	   } 
	   } 
	  \\ 
 

    \theutterance \stepcounter{utterance}  

    & & & \multicolumn{2}{p{0.3\linewidth}}{\cellcolor[rgb]{0.95,0.95,0.95}{%
	\makecell[{{p{\linewidth}}}]{% 
	  \tt {\tiny [GM$|$GM]}  
	 {'agreement': 'no', 'explanation': 'while "slash" fits the pattern indicated by the guess feedback, it doesn\'t align well with the meaning of "bravado." bravado refers to a bold, confident, and often exaggerated display of courage or self-importance, whereas "slash" is more associated with cutting actions or aggressive reduction. i suggest considering other words that match the letter pattern and more closely relate to the clue\'s meaning of showing off or displaying confidence.'} 
	  } 
	   } 
	   } 
	 & & \\ 
 

    \theutterance \stepcounter{utterance}  

    & & & \multicolumn{2}{p{0.3\linewidth}}{\cellcolor[rgb]{0.95,0.95,0.95}{%
	\makecell[{{p{\linewidth}}}]{% 
	  \tt {\tiny [GM$|$GM]}  
	 Critic disagrees with the Guesser -- Sharing the critic's explanation with the guesser 
	  } 
	   } 
	   } 
	 & & \\ 
 

    \theutterance \stepcounter{utterance}  

    & & \multicolumn{4}{p{0.6\linewidth}}{\cellcolor[rgb]{0.9,0.9,0.9}{%
	\makecell[{{p{\linewidth}}}]{% 
	  \tt {\tiny [A$\langle$GM]}  
	 clue:bravado\\ \tt guess_agreement:no\\ \tt agreement_explanation:while "slash" fits the pattern indicated by the guess feedback, it doesn't align well with the meaning of "bravado." bravado refers to a bold, confident, and often exaggerated display of courage or self-importance, whereas "slash" is more associated with cutting actions or aggressive reduction. i suggest considering other words that match the letter pattern and more closely relate to the clue's meaning of showing off or displaying confidence.\\ \tt  
	  } 
	   } 
	   } 
	 & \\ 
 

    \theutterance \stepcounter{utterance}  

    & \multicolumn{4}{p{0.6\linewidth}}{\cellcolor[rgb]{0.8,1,0.9}{%
	\makecell[{{p{\linewidth}}}]{% 
	  \tt {\tiny [A$\rangle$GM]}  
	 guess:stash\\ \tt explanation:Thank you for the guidance. Reconsidering the clue "bravado" and the letter pattern, "stash" could be a better fit. While it doesn't directly mean showing off or displaying confidence, it can be related to the idea of boldly or brazenly hiding something valuable, which might be done with a sense of bravado. "Stash" maintains the correct positions of 's,' 'a,' and  
	  } 
	   } 
	   } 
	 & & \\ 
 

    \theutterance \stepcounter{utterance}  

    & & & \multicolumn{2}{p{0.3\linewidth}}{\cellcolor[rgb]{0.95,0.95,0.95}{%
	\makecell[{{p{\linewidth}}}]{% 
	  \tt {\tiny [GM$|$GM]}  
	 {'guess': 'stash', 'explanation': 'Thank you for the guidance. Reconsidering the clue "bravado" and the letter pattern, "stash" could be a better fit. While it doesn\'t directly mean showing off or displaying confidence, it can be related to the idea of boldly or brazenly hiding something valuable, which might be done with a sense of bravado. "Stash" maintains the correct positions of \'s,\' \'a,\' and'} 
	  } 
	   } 
	   } 
	 & & \\ 
 

    \theutterance \stepcounter{utterance}  

    & & & \multicolumn{2}{p{0.3\linewidth}}{\cellcolor[rgb]{0.95,0.95,0.95}{%
	\makecell[{{p{\linewidth}}}]{% 
	  \tt {\tiny [GM$|$GM]}  
	 Change in player1's guess\\ \tt guess_before_critic_opinion: slash\\ \tt                                                 critic_agreement: no\\ \tt guess_after_critic_opinion: stash\\ \tt                                                 Proceeding with guess validation 
	  } 
	   } 
	   } 
	 & & \\ 
 

    \theutterance \stepcounter{utterance}  

    & & & \multicolumn{2}{p{0.3\linewidth}}{\cellcolor[rgb]{0.95,0.95,0.95}{%
	\makecell[{{p{\linewidth}}}]{% 
	  \tt {\tiny [GM$|$GM]}  
	 attempts: 4\\ \tt target_word = swash\\ \tt target_word_clue = bravado\\ \tt guess: stash\\ \tt guess_feedback: s<green> t<red> a<green> s<green> h<green> 
	  } 
	   } 
	   } 
	 & & \\ 
 

    \theutterance \stepcounter{utterance}  

    & & \multicolumn{4}{p{0.6\linewidth}}{\cellcolor[rgb]{0.9,0.9,0.9}{%
	\makecell[{{p{\linewidth}}}]{% 
	  \tt {\tiny [A$\langle$GM]}  
	 guess_feedback: s<green> t<red> a<green> s<green> h<green> 
	  } 
	   } 
	   } 
	 & \\ 
 

    \theutterance \stepcounter{utterance}  

    & \multicolumn{4}{p{0.6\linewidth}}{\cellcolor[rgb]{0.8,1,0.9}{%
	\makecell[{{p{\linewidth}}}]{% 
	  \tt {\tiny [A$\rangle$GM]}  
	 guess:smash\\ \tt explanation:Based on the latest feedback, 't' is the only incorrect letter in "stash." Considering the clue "bravado" and the confirmed letters, "smash" fits well. To "smash" something can be seen as a bold, confident, or even brazen action, which aligns with the meaning of "bravado." "Smash" keeps the correct letters 's,' 'a,' 's,' 
	  } 
	   } 
	   } 
	 & & \\ 
 

    \theutterance \stepcounter{utterance}  

    & & & \multicolumn{2}{p{0.3\linewidth}}{\cellcolor[rgb]{0.95,0.95,0.95}{%
	\makecell[{{p{\linewidth}}}]{% 
	  \tt {\tiny [GM$|$GM]}  
	 {'guess': 'smash', 'explanation': 'Based on the latest feedback, \'t\' is the only incorrect letter in "stash." Considering the clue "bravado" and the confirmed letters, "smash" fits well. To "smash" something can be seen as a bold, confident, or even brazen action, which aligns with the meaning of "bravado." "Smash" keeps the correct letters \'s,\' \'a,\' \'s,\''} 
	  } 
	   } 
	   } 
	 & & \\ 
 

    \theutterance \stepcounter{utterance}  

    & & \multicolumn{4}{p{0.6\linewidth}}{\cellcolor[rgb]{0.9,0.9,0.9}{%
	\makecell[{{p{\linewidth}}}]{% 
	  \tt {\tiny [GM$\rangle$B]}  
	 clue:bravado\\ \tt guess:smash\\ \tt explanation:Based on the latest feedback, 't' is the only incorrect letter in "stash." Considering the clue "bravado" and the confirmed letters, "smash" fits well. To "smash" something can be seen as a bold, confident, or even brazen action, which aligns with the meaning of "bravado." "Smash" keeps the correct letters 's,' 'a,' 's,'\\ \tt guess_feedback: s<green> t<red> a<green> s<green> h<green> 
	  } 
	   } 
	   } 
	 & \\ 
 

    \theutterance \stepcounter{utterance}  

    & & & \multicolumn{4}{p{0.6\linewidth}}{\cellcolor[rgb]{1,0.85,0.72}{%
	\makecell[{{p{\linewidth}}}]{% 
	  \tt {\tiny [GM$\langle$B]}  
	 agreement:yes\\ \tt explanation:the guess "smash" fits the clue "bravado" perfectly. it maintains the correct letters 's', 'a', 's', and 'h' in their respective positions based on the previous feedback. additionally, "smash" conveys a sense of bold, forceful action, which aligns with the meaning of "bravado." since all the letters are in the correct positions, forming the word "smash," it 
	  } 
	   } 
	   } 
	  \\ 
 

    \theutterance \stepcounter{utterance}  

    & & & \multicolumn{2}{p{0.3\linewidth}}{\cellcolor[rgb]{0.95,0.95,0.95}{%
	\makecell[{{p{\linewidth}}}]{% 
	  \tt {\tiny [GM$|$GM]}  
	 {'agreement': 'yes', 'explanation': 'the guess "smash" fits the clue "bravado" perfectly. it maintains the correct letters \'s\', \'a\', \'s\', and \'h\' in their respective positions based on the previous feedback. additionally, "smash" conveys a sense of bold, forceful action, which aligns with the meaning of "bravado." since all the letters are in the correct positions, forming the word "smash," it'} 
	  } 
	   } 
	   } 
	 & & \\ 
 

    \theutterance \stepcounter{utterance}  

    & & & \multicolumn{2}{p{0.3\linewidth}}{\cellcolor[rgb]{0.95,0.95,0.95}{%
	\makecell[{{p{\linewidth}}}]{% 
	  \tt {\tiny [GM$|$GM]}  
	 Critic agrees with the Guesser -- Sharing the critic's explanation with the guesser 
	  } 
	   } 
	   } 
	 & & \\ 
 

    \theutterance \stepcounter{utterance}  

    & & \multicolumn{4}{p{0.6\linewidth}}{\cellcolor[rgb]{0.9,0.9,0.9}{%
	\makecell[{{p{\linewidth}}}]{% 
	  \tt {\tiny [A$\langle$GM]}  
	 clue:bravado\\ \tt guess_agreement:yes\\ \tt agreement_explanation:the guess "smash" fits the clue "bravado" perfectly. it maintains the correct letters 's', 'a', 's', and 'h' in their respective positions based on the previous feedback. additionally, "smash" conveys a sense of bold, forceful action, which aligns with the meaning of "bravado." since all the letters are in the correct positions, forming the word "smash," it\\ \tt  
	  } 
	   } 
	   } 
	 & \\ 
 

    \theutterance \stepcounter{utterance}  

    & \multicolumn{4}{p{0.6\linewidth}}{\cellcolor[rgb]{0.8,1,0.9}{%
	\makecell[{{p{\linewidth}}}]{% 
	  \tt {\tiny [A$\rangle$GM]}  
	 guess:smash\\ \tt explanation:Thank you for confirming that "smash" is an excellent fit for the clue "bravado." It maintains the correct positions of the letters 's,' 'a,' 's,' and 'h' as indicated by the previous feedback. Moreover, the meaning of "smash" aligns perfectly with the idea of a bold, confident, and forceful action, which captures the essence of "bravado." Since all the letters 
	  } 
	   } 
	   } 
	 & & \\ 
 

    \theutterance \stepcounter{utterance}  

    & & & \multicolumn{2}{p{0.3\linewidth}}{\cellcolor[rgb]{0.95,0.95,0.95}{%
	\makecell[{{p{\linewidth}}}]{% 
	  \tt {\tiny [GM$|$GM]}  
	 {'guess': 'smash', 'explanation': 'Thank you for confirming that "smash" is an excellent fit for the clue "bravado." It maintains the correct positions of the letters \'s,\' \'a,\' \'s,\' and \'h\' as indicated by the previous feedback. Moreover, the meaning of "smash" aligns perfectly with the idea of a bold, confident, and forceful action, which captures the essence of "bravado." Since all the letters'} 
	  } 
	   } 
	   } 
	 & & \\ 
 

    \theutterance \stepcounter{utterance}  

    & & & \multicolumn{2}{p{0.3\linewidth}}{\cellcolor[rgb]{0.95,0.95,0.95}{%
	\makecell[{{p{\linewidth}}}]{% 
	  \tt {\tiny [GM$|$GM]}  
	 No change in player1's guess\\ \tt guess_before_critic_opinion: smash\\ \tt                                                 critic_agreement: yes\\ \tt guess_after_critic_opinion: smash\\ \tt                                                 Proceeding with guess validation 
	  } 
	   } 
	   } 
	 & & \\ 
 

    \theutterance \stepcounter{utterance}  

    & & & \multicolumn{2}{p{0.3\linewidth}}{\cellcolor[rgb]{0.95,0.95,0.95}{%
	\makecell[{{p{\linewidth}}}]{% 
	  \tt {\tiny [GM$|$GM]}  
	 attempts: 5\\ \tt target_word = swash\\ \tt target_word_clue = bravado\\ \tt guess: smash\\ \tt guess_feedback: s<green> m<red> a<green> s<green> h<green> 
	  } 
	   } 
	   } 
	 & & \\ 
 

    \theutterance \stepcounter{utterance}  

    & & \multicolumn{4}{p{0.6\linewidth}}{\cellcolor[rgb]{0.9,0.9,0.9}{%
	\makecell[{{p{\linewidth}}}]{% 
	  \tt {\tiny [A$\langle$GM]}  
	 guess_feedback: s<green> m<red> a<green> s<green> h<green> 
	  } 
	   } 
	   } 
	 & \\ 
 

    \theutterance \stepcounter{utterance}  

    & \multicolumn{4}{p{0.6\linewidth}}{\cellcolor[rgb]{0.8,1,0.9}{%
	\makecell[{{p{\linewidth}}}]{% 
	  \tt {\tiny [A$\rangle$GM]}  
	 guess:stash\\ \tt explanation:Apologies for the confusion in my previous explanation. Based on the latest feedback, 'm' is the only incorrect letter in "smash." Reconsidering the clue "bravado" and the confirmed letters, I believe "stash" is the correct answer. "Stash" maintains the correct positions of 's,' 'a,' 's,' and 'h' as indicated by the feedback. While it may not 
	  } 
	   } 
	   } 
	 & & \\ 
 

    \theutterance \stepcounter{utterance}  

    & & & \multicolumn{2}{p{0.3\linewidth}}{\cellcolor[rgb]{0.95,0.95,0.95}{%
	\makecell[{{p{\linewidth}}}]{% 
	  \tt {\tiny [GM$|$GM]}  
	 {'guess': 'stash', 'explanation': 'Apologies for the confusion in my previous explanation. Based on the latest feedback, \'m\' is the only incorrect letter in "smash." Reconsidering the clue "bravado" and the confirmed letters, I believe "stash" is the correct answer. "Stash" maintains the correct positions of \'s,\' \'a,\' \'s,\' and \'h\' as indicated by the feedback. While it may not'} 
	  } 
	   } 
	   } 
	 & & \\ 
 

    \theutterance \stepcounter{utterance}  

    & & \multicolumn{4}{p{0.6\linewidth}}{\cellcolor[rgb]{0.9,0.9,0.9}{%
	\makecell[{{p{\linewidth}}}]{% 
	  \tt {\tiny [GM$\rangle$B]}  
	 clue:bravado\\ \tt guess:stash\\ \tt explanation:Apologies for the confusion in my previous explanation. Based on the latest feedback, 'm' is the only incorrect letter in "smash." Reconsidering the clue "bravado" and the confirmed letters, I believe "stash" is the correct answer. "Stash" maintains the correct positions of 's,' 'a,' 's,' and 'h' as indicated by the feedback. While it may not\\ \tt guess_feedback: s<green> m<red> a<green> s<green> h<green> 
	  } 
	   } 
	   } 
	 & \\ 
 

    \theutterance \stepcounter{utterance}  

    & & & \multicolumn{4}{p{0.6\linewidth}}{\cellcolor[rgb]{1,0.85,0.72}{%
	\makecell[{{p{\linewidth}}}]{% 
	  \tt {\tiny [GM$\langle$B]}  
	 agreement:no\\ \tt explanation:while "stash" maintains the correct positions of the letters 's', 'a', 's', and 'h' based on the previous feedback, its meaning does not align closely with the clue "bravado." to "stash" something means to hide or store it away, which does not directly relate to the boldness, confidence, or brazenness implied by "bravado." therefore, despite fitting the letter positions, " 
	  } 
	   } 
	   } 
	  \\ 
 

    \theutterance \stepcounter{utterance}  

    & & & \multicolumn{2}{p{0.3\linewidth}}{\cellcolor[rgb]{0.95,0.95,0.95}{%
	\makecell[{{p{\linewidth}}}]{% 
	  \tt {\tiny [GM$|$GM]}  
	 {'agreement': 'no', 'explanation': 'while "stash" maintains the correct positions of the letters \'s\', \'a\', \'s\', and \'h\' based on the previous feedback, its meaning does not align closely with the clue "bravado." to "stash" something means to hide or store it away, which does not directly relate to the boldness, confidence, or brazenness implied by "bravado." therefore, despite fitting the letter positions, "'} 
	  } 
	   } 
	   } 
	 & & \\ 
 

    \theutterance \stepcounter{utterance}  

    & & & \multicolumn{2}{p{0.3\linewidth}}{\cellcolor[rgb]{0.95,0.95,0.95}{%
	\makecell[{{p{\linewidth}}}]{% 
	  \tt {\tiny [GM$|$GM]}  
	 Critic disagrees with the Guesser -- Sharing the critic's explanation with the guesser 
	  } 
	   } 
	   } 
	 & & \\ 
 

    \theutterance \stepcounter{utterance}  

    & & \multicolumn{4}{p{0.6\linewidth}}{\cellcolor[rgb]{0.9,0.9,0.9}{%
	\makecell[{{p{\linewidth}}}]{% 
	  \tt {\tiny [A$\langle$GM]}  
	 clue:bravado\\ \tt guess_agreement:no\\ \tt agreement_explanation:while "stash" maintains the correct positions of the letters 's', 'a', 's', and 'h' based on the previous feedback, its meaning does not align closely with the clue "bravado." to "stash" something means to hide or store it away, which does not directly relate to the boldness, confidence, or brazenness implied by "bravado." therefore, despite fitting the letter positions, "\\ \tt  
	  } 
	   } 
	   } 
	 & \\ 
 

    \theutterance \stepcounter{utterance}  

    & \multicolumn{4}{p{0.6\linewidth}}{\cellcolor[rgb]{0.8,1,0.9}{%
	\makecell[{{p{\linewidth}}}]{% 
	  \tt {\tiny [A$\rangle$GM]}  
	 guess:slash\\ \tt explanation:Thank you for the clarification. Upon further reflection, "slash" is the most fitting guess for the clue "bravado." It maintains the correct positions of 's,' 'a,' 's,' and 'h' as indicated by the previous feedback. While "slash" may not be the most direct synonym for "bravado," it can be associated with bold, decisive, and aggressive actions, which capture the essence of the clue 
	  } 
	   } 
	   } 
	 & & \\ 
 

    \theutterance \stepcounter{utterance}  

    & & & \multicolumn{2}{p{0.3\linewidth}}{\cellcolor[rgb]{0.95,0.95,0.95}{%
	\makecell[{{p{\linewidth}}}]{% 
	  \tt {\tiny [GM$|$GM]}  
	 {'guess': 'slash', 'explanation': 'Thank you for the clarification. Upon further reflection, "slash" is the most fitting guess for the clue "bravado." It maintains the correct positions of \'s,\' \'a,\' \'s,\' and \'h\' as indicated by the previous feedback. While "slash" may not be the most direct synonym for "bravado," it can be associated with bold, decisive, and aggressive actions, which capture the essence of the clue'} 
	  } 
	   } 
	   } 
	 & & \\ 
 

    \theutterance \stepcounter{utterance}  

    & & & \multicolumn{2}{p{0.3\linewidth}}{\cellcolor[rgb]{0.95,0.95,0.95}{%
	\makecell[{{p{\linewidth}}}]{% 
	  \tt {\tiny [GM$|$GM]}  
	 Change in player1's guess\\ \tt guess_before_critic_opinion: stash\\ \tt                                                 critic_agreement: no\\ \tt guess_after_critic_opinion: slash\\ \tt                                                 Proceeding with guess validation 
	  } 
	   } 
	   } 
	 & & \\ 
 

    \theutterance \stepcounter{utterance}  

    & & & \multicolumn{2}{p{0.3\linewidth}}{\cellcolor[rgb]{0.95,0.95,0.95}{%
	\makecell[{{p{\linewidth}}}]{% 
	  \tt {\tiny [GM$|$GM]}  
	 attempts: 6\\ \tt target_word = swash\\ \tt target_word_clue = bravado\\ \tt guess: slash\\ \tt guess_feedback: s<green> l<red> a<green> s<green> h<green> 
	  } 
	   } 
	   } 
	 & & \\ 
 

    \theutterance \stepcounter{utterance}  

    & & & \multicolumn{2}{p{0.3\linewidth}}{\cellcolor[rgb]{0.95,0.95,0.95}{%
	\makecell[{{p{\linewidth}}}]{% 
	  \tt {\tiny [GM$|$GM]}  
	 game_result = LOSS 
	  } 
	   } 
	   } 
	 & & \\ 
 

\end{supertabular}
}

\end{document}


\documentclass{article}
\usepackage{colortbl}
\usepackage{makecell}
\usepackage{multirow}
\usepackage{supertabular}

\begin{document}

\newcounter{utterance}

\twocolumn

{ \footnotesize  \setcounter{utterance}{1}
\setlength{\tabcolsep}{0pt}
\begin{supertabular}{c@{$\;$}|p{.15\linewidth}@{}p{.15\linewidth}p{.15\linewidth}p{.15\linewidth}p{.15\linewidth}p{.15\linewidth}}

    \# & $\;$A & \multicolumn{4}{c}{Game Master} & $\;\:$B\\
    \hline 

    \theutterance \stepcounter{utterance}  

    & & \multicolumn{4}{p{0.6\linewidth}}{\cellcolor[rgb]{0.9,0.9,0.9}{%
	\makecell[{{p{\linewidth}}}]{% 
	  \tt {\tiny [A$\langle$GM]}  
	 You are a language wizard who likes to guess words by using the given rules.\\ \tt \\ \tt Welcome to Wordle! You have six attempts to guess the target word, a valid English word of five lowercase letters (a-z). Please use the tags "guess:" and "explanation:" to provide a concise explanation for each guess.\\ \tt \\ \tt To help you make an informed guess, you will receive a clue for the word, such as\\ \tt clue: "snowy white"\\ \tt \\ \tt Here is an example guess based on the clue:\\ \tt guess: apple\\ \tt explanation: In the fairy tail Snow White, the girl is killed because she eats a poisoned apple. And the word apple has 5 letters.\\ \tt \\ \tt I will then indicate whether I agree or disagree with your guess and provide rationale, but agreeing with a guess does not confirm its correctness. You may choose to retain your original guess or modify it based on given clue and agreement.\\ \tt \\ \tt After each guess, your answer will be validated, and you will receive feedback indicating which letters are correct (green), which letters are correct but in the wrong position (yellow), and which letters are incorrect (red). This feedback can be useful in determining which letters to include or exclude in your next guess.\\ \tt \\ \tt For example, the feedback for "apple" might be:\\ \tt guess_feedback: a<yellow> p<yellow> p<green> l<yellow> e<red>\\ \tt \\ \tt Here are some rules to keep in mind:\\ \tt 1. Please do not generate any guess feedback or clue on your own\\ \tt 2. Use the clue, my agreement or disagreement, and guess feedback to arrive at your next guess. This should be reflected in your explanation as well.\\ \tt \\ \tt Let's begin with your first guess.\\ \tt \\ \tt clue:hungry\\ \tt  
	  } 
	   } 
	   } 
	 & \\ 
 

    \theutterance \stepcounter{utterance}  

    & \multicolumn{4}{p{0.6\linewidth}}{\cellcolor[rgb]{0.8,1,0.9}{%
	\makecell[{{p{\linewidth}}}]{% 
	  \tt {\tiny [A$\rangle$GM]}  
	 guess: feast\\ \tt explanation: The word "feast" is associated with a large meal, which is often consumed when someone is very hungry. It also has five letters, matching the criteria for a valid guess. 
	  } 
	   } 
	   } 
	 & & \\ 
 

    \theutterance \stepcounter{utterance}  

    & & & \multicolumn{2}{p{0.3\linewidth}}{\cellcolor[rgb]{0.95,0.95,0.95}{%
	\makecell[{{p{\linewidth}}}]{% 
	  \tt {\tiny [GM$|$GM]}  
	 {'guess': 'feast', 'explanation': 'The word "feast" is associated with a large meal, which is often consumed when someone is very hungry. It also has five letters, matching the criteria for a valid guess.'} 
	  } 
	   } 
	   } 
	 & & \\ 
 

    \theutterance \stepcounter{utterance}  

    & & \multicolumn{4}{p{0.6\linewidth}}{\cellcolor[rgb]{0.9,0.9,0.9}{%
	\makecell[{{p{\linewidth}}}]{% 
	  \tt {\tiny [GM$\rangle$B]}  
	 I need your assistance with a word game in which we need to find a 5-letter word using a clue, a guess and an explanation for the guess.\\ \tt \\ \tt For example, if the clue is "snow white" and the guess is "apple", you will get the information like this:\\ \tt clue:snow white\\ \tt guess:apple\\ \tt explanation:the fairy tale Snow White involves the girl eating a poisoned apple, which has 5 letters.\\ \tt \\ \tt Your task is to either agree or disagree with my guess based on the given clue.\\ \tt \\ \tt If you disagree, I will make another guess, and if you agree, I will proceed in the game using the same guess.\\ \tt \\ \tt Initially, guess feedback is not available, but as the game proceeds, it becomes available. The guess feedback is not for current guess, but rather an indication of what letters can be present in the current guess. A letter highlighted in green indicates that it is a correct letter in the correct position, while yellow indicates that it is a correct letter in the wrong position, and red indicates a wrong letter. At this point, you should use all the provided information, including the clue, guess, explanation, and guess feedback, to decide whether the given guess matches the clue, and also stick to guess feedback or not.\\ \tt \\ \tt Please respond in lowercase letters and stick to this format:\\ \tt agreement:yes or no\\ \tt explanation:your reason for doing so\\ \tt \\ \tt Please note that you must not guess any word based on the riddle clue and stick to the given format while responding.\\ \tt \\ \tt Let's begin!\\ \tt \\ \tt clue:hungry\\ \tt guess:feast\\ \tt explanation:The word "feast" is associated with a large meal, which is often consumed when someone is very hungry. It also has five letters, matching the criteria for a valid guess.\\ \tt  
	  } 
	   } 
	   } 
	 & \\ 
 

    \theutterance \stepcounter{utterance}  

    & & & \multicolumn{4}{p{0.6\linewidth}}{\cellcolor[rgb]{1,0.85,0.72}{%
	\makecell[{{p{\linewidth}}}]{% 
	  \tt {\tiny [GM$\langle$B]}  
	 agreement:no\\ \tt explanation:while a feast is associated with eating a large meal, it doesn't directly relate to the state of being hungry. a feast is more about the abundance and variety of food rather than satisfying hunger specifically. there could be better guesses that more precisely capture the feeling or state of hunger itself. 
	  } 
	   } 
	   } 
	  \\ 
 

    \theutterance \stepcounter{utterance}  

    & & & \multicolumn{2}{p{0.3\linewidth}}{\cellcolor[rgb]{0.95,0.95,0.95}{%
	\makecell[{{p{\linewidth}}}]{% 
	  \tt {\tiny [GM$|$GM]}  
	 {'agreement': 'no', 'explanation': "while a feast is associated with eating a large meal, it doesn't directly relate to the state of being hungry. a feast is more about the abundance and variety of food rather than satisfying hunger specifically. there could be better guesses that more precisely capture the feeling or state of hunger itself."} 
	  } 
	   } 
	   } 
	 & & \\ 
 

    \theutterance \stepcounter{utterance}  

    & & & \multicolumn{2}{p{0.3\linewidth}}{\cellcolor[rgb]{0.95,0.95,0.95}{%
	\makecell[{{p{\linewidth}}}]{% 
	  \tt {\tiny [GM$|$GM]}  
	 Critic disagrees with the Guesser -- Sharing the critic's explanation with the guesser 
	  } 
	   } 
	   } 
	 & & \\ 
 

    \theutterance \stepcounter{utterance}  

    & & \multicolumn{4}{p{0.6\linewidth}}{\cellcolor[rgb]{0.9,0.9,0.9}{%
	\makecell[{{p{\linewidth}}}]{% 
	  \tt {\tiny [A$\langle$GM]}  
	 clue:hungry\\ \tt guess_agreement:no\\ \tt agreement_explanation:while a feast is associated with eating a large meal, it doesn't directly relate to the state of being hungry. a feast is more about the abundance and variety of food rather than satisfying hunger specifically. there could be better guesses that more precisely capture the feeling or state of hunger itself.\\ \tt  
	  } 
	   } 
	   } 
	 & \\ 
 

    \theutterance \stepcounter{utterance}  

    & \multicolumn{4}{p{0.6\linewidth}}{\cellcolor[rgb]{0.8,1,0.9}{%
	\makecell[{{p{\linewidth}}}]{% 
	  \tt {\tiny [A$\rangle$GM]}  
	 guess: starve\\ \tt explanation: The word "starve" is more directly related to the state of being hungry. When someone is starving, they are suffering from extreme hunger due to a lack of food. This guess better captures the feeling and state of hunger itself, as opposed to focusing on the act of eating a large meal. 
	  } 
	   } 
	   } 
	 & & \\ 
 

    \theutterance \stepcounter{utterance}  

    & & & \multicolumn{2}{p{0.3\linewidth}}{\cellcolor[rgb]{0.95,0.95,0.95}{%
	\makecell[{{p{\linewidth}}}]{% 
	  \tt {\tiny [GM$|$GM]}  
	 {'guess': 'starve', 'explanation': 'The word "starve" is more directly related to the state of being hungry. When someone is starving, they are suffering from extreme hunger due to a lack of food. This guess better captures the feeling and state of hunger itself, as opposed to focusing on the act of eating a large meal.'} 
	  } 
	   } 
	   } 
	 & & \\ 
 

    \theutterance \stepcounter{utterance}  

    & & & \multicolumn{2}{p{0.3\linewidth}}{\cellcolor[rgb]{0.95,0.95,0.95}{%
	\makecell[{{p{\linewidth}}}]{% 
	  \tt {\tiny [GM$|$GM]}  
	 Guesser Error: INVALID_WORD_LENGTH while parsing Player 1's (model: claude-3-opus-20240229) response, retrying 
	  } 
	   } 
	   } 
	 & & \\ 
 

    \theutterance \stepcounter{utterance}  

    & & \multicolumn{4}{p{0.6\linewidth}}{\cellcolor[rgb]{0.9,0.9,0.9}{%
	\makecell[{{p{\linewidth}}}]{% 
	  \tt {\tiny [A$\langle$GM]}  
	 the word should have exactly 5 letters. Please try again 
	  } 
	   } 
	   } 
	 & \\ 
 

    \theutterance \stepcounter{utterance}  

    & \multicolumn{4}{p{0.6\linewidth}}{\cellcolor[rgb]{0.8,1,0.9}{%
	\makecell[{{p{\linewidth}}}]{% 
	  \tt {\tiny [A$\rangle$GM]}  
	 clue:hungry\\ \tt guess:crave\\ \tt explanation:The word "crave" means to have an intense desire for something, often used in the context of food when someone is hungry. For example, "I'm so hungry, I'm craving a big burger right now." The word "crave" also has exactly five letters, making it a valid guess according to the rules of the game.\\ \tt  
	  } 
	   } 
	   } 
	 & & \\ 
 

    \theutterance \stepcounter{utterance}  

    & & & \multicolumn{2}{p{0.3\linewidth}}{\cellcolor[rgb]{0.95,0.95,0.95}{%
	\makecell[{{p{\linewidth}}}]{% 
	  \tt {\tiny [GM$|$GM]}  
	 {'guess': 'crave', 'explanation': 'The word "crave" means to have an intense desire for something, often used in the context of food when someone is hungry. For example, "I\'m so hungry, I\'m craving a big burger right now." The word "crave" also has exactly five letters, making it a valid guess according to the rules of the game.'} 
	  } 
	   } 
	   } 
	 & & \\ 
 

    \theutterance \stepcounter{utterance}  

    & & & \multicolumn{2}{p{0.3\linewidth}}{\cellcolor[rgb]{0.95,0.95,0.95}{%
	\makecell[{{p{\linewidth}}}]{% 
	  \tt {\tiny [GM$|$GM]}  
	 Change in player1's guess\\ \tt guess_before_critic_opinion: feast\\ \tt                                                 critic_agreement: no\\ \tt guess_after_critic_opinion: crave\\ \tt                                                 Proceeding with guess validation 
	  } 
	   } 
	   } 
	 & & \\ 
 

    \theutterance \stepcounter{utterance}  

    & & & \multicolumn{2}{p{0.3\linewidth}}{\cellcolor[rgb]{0.95,0.95,0.95}{%
	\makecell[{{p{\linewidth}}}]{% 
	  \tt {\tiny [GM$|$GM]}  
	 attempts: 1\\ \tt target_word = unfed\\ \tt target_word_clue = hungry\\ \tt guess: crave\\ \tt guess_feedback: c<red> r<red> a<red> v<red> e<yellow> 
	  } 
	   } 
	   } 
	 & & \\ 
 

    \theutterance \stepcounter{utterance}  

    & & \multicolumn{4}{p{0.6\linewidth}}{\cellcolor[rgb]{0.9,0.9,0.9}{%
	\makecell[{{p{\linewidth}}}]{% 
	  \tt {\tiny [A$\langle$GM]}  
	 guess_feedback: c<red> r<red> a<red> v<red> e<yellow> 
	  } 
	   } 
	   } 
	 & \\ 
 

    \theutterance \stepcounter{utterance}  

    & \multicolumn{4}{p{0.6\linewidth}}{\cellcolor[rgb]{0.8,1,0.9}{%
	\makecell[{{p{\linewidth}}}]{% 
	  \tt {\tiny [A$\rangle$GM]}  
	 guess:empty\\ \tt explanation:Based on the feedback, the letter "e" is correct but in the wrong position, while the other letters in "crave" are incorrect. "Empty" is a good guess because it relates to the feeling of having an empty stomach when one is hungry. It also contains the letter "e" and has exactly five letters. The word "empty" captures the physical sensation of hunger.\\ \tt  
	  } 
	   } 
	   } 
	 & & \\ 
 

    \theutterance \stepcounter{utterance}  

    & & & \multicolumn{2}{p{0.3\linewidth}}{\cellcolor[rgb]{0.95,0.95,0.95}{%
	\makecell[{{p{\linewidth}}}]{% 
	  \tt {\tiny [GM$|$GM]}  
	 {'guess': 'empty', 'explanation': 'Based on the feedback, the letter "e" is correct but in the wrong position, while the other letters in "crave" are incorrect. "Empty" is a good guess because it relates to the feeling of having an empty stomach when one is hungry. It also contains the letter "e" and has exactly five letters. The word "empty" captures the physical sensation of hunger.'} 
	  } 
	   } 
	   } 
	 & & \\ 
 

    \theutterance \stepcounter{utterance}  

    & & \multicolumn{4}{p{0.6\linewidth}}{\cellcolor[rgb]{0.9,0.9,0.9}{%
	\makecell[{{p{\linewidth}}}]{% 
	  \tt {\tiny [GM$\rangle$B]}  
	 clue:hungry\\ \tt guess:empty\\ \tt explanation:Based on the feedback, the letter "e" is correct but in the wrong position, while the other letters in "crave" are incorrect. "Empty" is a good guess because it relates to the feeling of having an empty stomach when one is hungry. It also contains the letter "e" and has exactly five letters. The word "empty" captures the physical sensation of hunger.\\ \tt guess_feedback: c<red> r<red> a<red> v<red> e<yellow> 
	  } 
	   } 
	   } 
	 & \\ 
 

    \theutterance \stepcounter{utterance}  

    & & & \multicolumn{4}{p{0.6\linewidth}}{\cellcolor[rgb]{1,0.85,0.72}{%
	\makecell[{{p{\linewidth}}}]{% 
	  \tt {\tiny [GM$\langle$B]}  
	 agreement:yes\\ \tt explanation:the guess "empty" fits well with the clue "hungry" as an empty stomach is a common sensation when feeling hungry. it also correctly incorporates the feedback that "e" is a correct letter but in the wrong position, while avoiding the incorrect letters from the previous guess. the explanation provides a logical connection between the guess and the clue. 
	  } 
	   } 
	   } 
	  \\ 
 

    \theutterance \stepcounter{utterance}  

    & & & \multicolumn{2}{p{0.3\linewidth}}{\cellcolor[rgb]{0.95,0.95,0.95}{%
	\makecell[{{p{\linewidth}}}]{% 
	  \tt {\tiny [GM$|$GM]}  
	 {'agreement': 'yes', 'explanation': 'the guess "empty" fits well with the clue "hungry" as an empty stomach is a common sensation when feeling hungry. it also correctly incorporates the feedback that "e" is a correct letter but in the wrong position, while avoiding the incorrect letters from the previous guess. the explanation provides a logical connection between the guess and the clue.'} 
	  } 
	   } 
	   } 
	 & & \\ 
 

    \theutterance \stepcounter{utterance}  

    & & & \multicolumn{2}{p{0.3\linewidth}}{\cellcolor[rgb]{0.95,0.95,0.95}{%
	\makecell[{{p{\linewidth}}}]{% 
	  \tt {\tiny [GM$|$GM]}  
	 Critic agrees with the Guesser -- Sharing the critic's explanation with the guesser 
	  } 
	   } 
	   } 
	 & & \\ 
 

    \theutterance \stepcounter{utterance}  

    & & \multicolumn{4}{p{0.6\linewidth}}{\cellcolor[rgb]{0.9,0.9,0.9}{%
	\makecell[{{p{\linewidth}}}]{% 
	  \tt {\tiny [A$\langle$GM]}  
	 clue:hungry\\ \tt guess_agreement:yes\\ \tt agreement_explanation:the guess "empty" fits well with the clue "hungry" as an empty stomach is a common sensation when feeling hungry. it also correctly incorporates the feedback that "e" is a correct letter but in the wrong position, while avoiding the incorrect letters from the previous guess. the explanation provides a logical connection between the guess and the clue.\\ \tt  
	  } 
	   } 
	   } 
	 & \\ 
 

    \theutterance \stepcounter{utterance}  

    & \multicolumn{4}{p{0.6\linewidth}}{\cellcolor[rgb]{0.8,1,0.9}{%
	\makecell[{{p{\linewidth}}}]{% 
	  \tt {\tiny [A$\rangle$GM]}  
	 guess:empty\\ \tt explanation:Thank you for the feedback and agreement. Since "empty" seems to be a good guess that fits the clue "hungry" and incorporates the previous feedback, I will stick with this guess for now. An empty stomach is indeed a common physical sensation associated with feeling hungry, and the word has the correct number of letters. The explanation connects the guess to the clue logically, so I feel confident about this choice.\\ \tt  
	  } 
	   } 
	   } 
	 & & \\ 
 

    \theutterance \stepcounter{utterance}  

    & & & \multicolumn{2}{p{0.3\linewidth}}{\cellcolor[rgb]{0.95,0.95,0.95}{%
	\makecell[{{p{\linewidth}}}]{% 
	  \tt {\tiny [GM$|$GM]}  
	 {'guess': 'empty', 'explanation': 'Thank you for the feedback and agreement. Since "empty" seems to be a good guess that fits the clue "hungry" and incorporates the previous feedback, I will stick with this guess for now. An empty stomach is indeed a common physical sensation associated with feeling hungry, and the word has the correct number of letters. The explanation connects the guess to the clue logically, so I feel confident about this choice.'} 
	  } 
	   } 
	   } 
	 & & \\ 
 

    \theutterance \stepcounter{utterance}  

    & & & \multicolumn{2}{p{0.3\linewidth}}{\cellcolor[rgb]{0.95,0.95,0.95}{%
	\makecell[{{p{\linewidth}}}]{% 
	  \tt {\tiny [GM$|$GM]}  
	 No change in player1's guess\\ \tt guess_before_critic_opinion: empty\\ \tt                                                 critic_agreement: yes\\ \tt guess_after_critic_opinion: empty\\ \tt                                                 Proceeding with guess validation 
	  } 
	   } 
	   } 
	 & & \\ 
 

    \theutterance \stepcounter{utterance}  

    & & & \multicolumn{2}{p{0.3\linewidth}}{\cellcolor[rgb]{0.95,0.95,0.95}{%
	\makecell[{{p{\linewidth}}}]{% 
	  \tt {\tiny [GM$|$GM]}  
	 attempts: 2\\ \tt target_word = unfed\\ \tt target_word_clue = hungry\\ \tt guess: empty\\ \tt guess_feedback: e<yellow> m<red> p<red> t<red> y<red> 
	  } 
	   } 
	   } 
	 & & \\ 
 

    \theutterance \stepcounter{utterance}  

    & & \multicolumn{4}{p{0.6\linewidth}}{\cellcolor[rgb]{0.9,0.9,0.9}{%
	\makecell[{{p{\linewidth}}}]{% 
	  \tt {\tiny [A$\langle$GM]}  
	 guess_feedback: e<yellow> m<red> p<red> t<red> y<red> 
	  } 
	   } 
	   } 
	 & \\ 
 

    \theutterance \stepcounter{utterance}  

    & \multicolumn{4}{p{0.6\linewidth}}{\cellcolor[rgb]{0.8,1,0.9}{%
	\makecell[{{p{\linewidth}}}]{% 
	  \tt {\tiny [A$\rangle$GM]}  
	 guess:eager\\ \tt explanation:Based on the new feedback, the letter "e" is correct but still in the wrong position, while the other letters in "empty" are incorrect. "Eager" is a good guess because it relates to the strong desire or enthusiasm one might feel when hungry, such as being eager to eat. The word "eager" contains the letter "e" twice, increasing the chances of placing it in the correct position, and it has exactly five letters.\\ \tt  
	  } 
	   } 
	   } 
	 & & \\ 
 

    \theutterance \stepcounter{utterance}  

    & & & \multicolumn{2}{p{0.3\linewidth}}{\cellcolor[rgb]{0.95,0.95,0.95}{%
	\makecell[{{p{\linewidth}}}]{% 
	  \tt {\tiny [GM$|$GM]}  
	 {'guess': 'eager', 'explanation': 'Based on the new feedback, the letter "e" is correct but still in the wrong position, while the other letters in "empty" are incorrect. "Eager" is a good guess because it relates to the strong desire or enthusiasm one might feel when hungry, such as being eager to eat. The word "eager" contains the letter "e" twice, increasing the chances of placing it in the correct position, and it has exactly five letters.'} 
	  } 
	   } 
	   } 
	 & & \\ 
 

    \theutterance \stepcounter{utterance}  

    & & \multicolumn{4}{p{0.6\linewidth}}{\cellcolor[rgb]{0.9,0.9,0.9}{%
	\makecell[{{p{\linewidth}}}]{% 
	  \tt {\tiny [GM$\rangle$B]}  
	 clue:hungry\\ \tt guess:eager\\ \tt explanation:Based on the new feedback, the letter "e" is correct but still in the wrong position, while the other letters in "empty" are incorrect. "Eager" is a good guess because it relates to the strong desire or enthusiasm one might feel when hungry, such as being eager to eat. The word "eager" contains the letter "e" twice, increasing the chances of placing it in the correct position, and it has exactly five letters.\\ \tt guess_feedback: e<yellow> m<red> p<red> t<red> y<red> 
	  } 
	   } 
	   } 
	 & \\ 
 

    \theutterance \stepcounter{utterance}  

    & & & \multicolumn{4}{p{0.6\linewidth}}{\cellcolor[rgb]{1,0.85,0.72}{%
	\makecell[{{p{\linewidth}}}]{% 
	  \tt {\tiny [GM$\langle$B]}  
	 agreement:no\\ \tt explanation:while "eager" is a good guess related to the feeling of hunger, it does not match the guess feedback. the feedback indicates that "e" should not be the first letter, but "eager" has "e" as the first letter. a guess that fits the clue and feedback would have "e" in a different position, such as "greed" or "crave". 
	  } 
	   } 
	   } 
	  \\ 
 

    \theutterance \stepcounter{utterance}  

    & & & \multicolumn{2}{p{0.3\linewidth}}{\cellcolor[rgb]{0.95,0.95,0.95}{%
	\makecell[{{p{\linewidth}}}]{% 
	  \tt {\tiny [GM$|$GM]}  
	 {'agreement': 'no', 'explanation': 'while "eager" is a good guess related to the feeling of hunger, it does not match the guess feedback. the feedback indicates that "e" should not be the first letter, but "eager" has "e" as the first letter. a guess that fits the clue and feedback would have "e" in a different position, such as "greed" or "crave".'} 
	  } 
	   } 
	   } 
	 & & \\ 
 

    \theutterance \stepcounter{utterance}  

    & & & \multicolumn{2}{p{0.3\linewidth}}{\cellcolor[rgb]{0.95,0.95,0.95}{%
	\makecell[{{p{\linewidth}}}]{% 
	  \tt {\tiny [GM$|$GM]}  
	 Critic disagrees with the Guesser -- Sharing the critic's explanation with the guesser 
	  } 
	   } 
	   } 
	 & & \\ 
 

    \theutterance \stepcounter{utterance}  

    & & \multicolumn{4}{p{0.6\linewidth}}{\cellcolor[rgb]{0.9,0.9,0.9}{%
	\makecell[{{p{\linewidth}}}]{% 
	  \tt {\tiny [A$\langle$GM]}  
	 clue:hungry\\ \tt guess_agreement:no\\ \tt agreement_explanation:while "eager" is a good guess related to the feeling of hunger, it does not match the guess feedback. the feedback indicates that "e" should not be the first letter, but "eager" has "e" as the first letter. a guess that fits the clue and feedback would have "e" in a different position, such as "greed" or "crave".\\ \tt  
	  } 
	   } 
	   } 
	 & \\ 
 

    \theutterance \stepcounter{utterance}  

    & \multicolumn{4}{p{0.6\linewidth}}{\cellcolor[rgb]{0.8,1,0.9}{%
	\makecell[{{p{\linewidth}}}]{% 
	  \tt {\tiny [A$\rangle$GM]}  
	 guess:greed\\ \tt explanation:Thank you for pointing out that "eager" does not match the feedback, as "e" should not be the first letter. "Greed" is a better guess because it relates to an intense desire or longing, which can be associated with the feeling of hunger. The word "greed" has the letter "e" in the fourth position, which is different from the previous guesses and aligns with the feedback. Additionally, "greed 
	  } 
	   } 
	   } 
	 & & \\ 
 

    \theutterance \stepcounter{utterance}  

    & & & \multicolumn{2}{p{0.3\linewidth}}{\cellcolor[rgb]{0.95,0.95,0.95}{%
	\makecell[{{p{\linewidth}}}]{% 
	  \tt {\tiny [GM$|$GM]}  
	 {'guess': 'greed', 'explanation': 'Thank you for pointing out that "eager" does not match the feedback, as "e" should not be the first letter. "Greed" is a better guess because it relates to an intense desire or longing, which can be associated with the feeling of hunger. The word "greed" has the letter "e" in the fourth position, which is different from the previous guesses and aligns with the feedback. Additionally, "greed'} 
	  } 
	   } 
	   } 
	 & & \\ 
 

    \theutterance \stepcounter{utterance}  

    & & & \multicolumn{2}{p{0.3\linewidth}}{\cellcolor[rgb]{0.95,0.95,0.95}{%
	\makecell[{{p{\linewidth}}}]{% 
	  \tt {\tiny [GM$|$GM]}  
	 Change in player1's guess\\ \tt guess_before_critic_opinion: eager\\ \tt                                                 critic_agreement: no\\ \tt guess_after_critic_opinion: greed\\ \tt                                                 Proceeding with guess validation 
	  } 
	   } 
	   } 
	 & & \\ 
 

    \theutterance \stepcounter{utterance}  

    & & & \multicolumn{2}{p{0.3\linewidth}}{\cellcolor[rgb]{0.95,0.95,0.95}{%
	\makecell[{{p{\linewidth}}}]{% 
	  \tt {\tiny [GM$|$GM]}  
	 attempts: 3\\ \tt target_word = unfed\\ \tt target_word_clue = hungry\\ \tt guess: greed\\ \tt guess_feedback: g<red> r<red> e<red> e<green> d<green> 
	  } 
	   } 
	   } 
	 & & \\ 
 

    \theutterance \stepcounter{utterance}  

    & & \multicolumn{4}{p{0.6\linewidth}}{\cellcolor[rgb]{0.9,0.9,0.9}{%
	\makecell[{{p{\linewidth}}}]{% 
	  \tt {\tiny [A$\langle$GM]}  
	 guess_feedback: g<red> r<red> e<red> e<green> d<green> 
	  } 
	   } 
	   } 
	 & \\ 
 

    \theutterance \stepcounter{utterance}  

    & \multicolumn{4}{p{0.6\linewidth}}{\cellcolor[rgb]{0.8,1,0.9}{%
	\makecell[{{p{\linewidth}}}]{% 
	  \tt {\tiny [A$\rangle$GM]}  
	 guess:speed\\ \tt explanation:Based on the latest feedback, the letters "e" and "d" are correct and in the right positions (green), while "g," "r," and the second "e" are incorrect (red). "Speed" is a good guess because it fits the pattern of having "e" as the fourth letter and "d" as the fifth letter. Although not directly related to hunger, "speed" could be associated with the idea of eating quickly when one 
	  } 
	   } 
	   } 
	 & & \\ 
 

    \theutterance \stepcounter{utterance}  

    & & & \multicolumn{2}{p{0.3\linewidth}}{\cellcolor[rgb]{0.95,0.95,0.95}{%
	\makecell[{{p{\linewidth}}}]{% 
	  \tt {\tiny [GM$|$GM]}  
	 {'guess': 'speed', 'explanation': 'Based on the latest feedback, the letters "e" and "d" are correct and in the right positions (green), while "g," "r," and the second "e" are incorrect (red). "Speed" is a good guess because it fits the pattern of having "e" as the fourth letter and "d" as the fifth letter. Although not directly related to hunger, "speed" could be associated with the idea of eating quickly when one'} 
	  } 
	   } 
	   } 
	 & & \\ 
 

    \theutterance \stepcounter{utterance}  

    & & \multicolumn{4}{p{0.6\linewidth}}{\cellcolor[rgb]{0.9,0.9,0.9}{%
	\makecell[{{p{\linewidth}}}]{% 
	  \tt {\tiny [GM$\rangle$B]}  
	 clue:hungry\\ \tt guess:speed\\ \tt explanation:Based on the latest feedback, the letters "e" and "d" are correct and in the right positions (green), while "g," "r," and the second "e" are incorrect (red). "Speed" is a good guess because it fits the pattern of having "e" as the fourth letter and "d" as the fifth letter. Although not directly related to hunger, "speed" could be associated with the idea of eating quickly when one\\ \tt guess_feedback: g<red> r<red> e<red> e<green> d<green> 
	  } 
	   } 
	   } 
	 & \\ 
 

    \theutterance \stepcounter{utterance}  

    & & & \multicolumn{4}{p{0.6\linewidth}}{\cellcolor[rgb]{1,0.85,0.72}{%
	\makecell[{{p{\linewidth}}}]{% 
	  \tt {\tiny [GM$\langle$B]}  
	 agreement:no\\ \tt explanation:while "speed" fits the pattern of having "e" and "d" in the correct positions based on the previous feedback, it does not relate directly to the concept of being hungry. the clue "hungry" suggests a word that is more closely associated with the feeling or state of hunger, rather than the action of eating quickly. therefore, a word that better captures the sensation or desire related to hunger would be more appropriate, even if it matches the letter positions 
	  } 
	   } 
	   } 
	  \\ 
 

    \theutterance \stepcounter{utterance}  

    & & & \multicolumn{2}{p{0.3\linewidth}}{\cellcolor[rgb]{0.95,0.95,0.95}{%
	\makecell[{{p{\linewidth}}}]{% 
	  \tt {\tiny [GM$|$GM]}  
	 {'agreement': 'no', 'explanation': 'while "speed" fits the pattern of having "e" and "d" in the correct positions based on the previous feedback, it does not relate directly to the concept of being hungry. the clue "hungry" suggests a word that is more closely associated with the feeling or state of hunger, rather than the action of eating quickly. therefore, a word that better captures the sensation or desire related to hunger would be more appropriate, even if it matches the letter positions'} 
	  } 
	   } 
	   } 
	 & & \\ 
 

    \theutterance \stepcounter{utterance}  

    & & & \multicolumn{2}{p{0.3\linewidth}}{\cellcolor[rgb]{0.95,0.95,0.95}{%
	\makecell[{{p{\linewidth}}}]{% 
	  \tt {\tiny [GM$|$GM]}  
	 Critic disagrees with the Guesser -- Sharing the critic's explanation with the guesser 
	  } 
	   } 
	   } 
	 & & \\ 
 

    \theutterance \stepcounter{utterance}  

    & & \multicolumn{4}{p{0.6\linewidth}}{\cellcolor[rgb]{0.9,0.9,0.9}{%
	\makecell[{{p{\linewidth}}}]{% 
	  \tt {\tiny [A$\langle$GM]}  
	 clue:hungry\\ \tt guess_agreement:no\\ \tt agreement_explanation:while "speed" fits the pattern of having "e" and "d" in the correct positions based on the previous feedback, it does not relate directly to the concept of being hungry. the clue "hungry" suggests a word that is more closely associated with the feeling or state of hunger, rather than the action of eating quickly. therefore, a word that better captures the sensation or desire related to hunger would be more appropriate, even if it matches the letter positions\\ \tt  
	  } 
	   } 
	   } 
	 & \\ 
 

    \theutterance \stepcounter{utterance}  

    & \multicolumn{4}{p{0.6\linewidth}}{\cellcolor[rgb]{0.8,1,0.9}{%
	\makecell[{{p{\linewidth}}}]{% 
	  \tt {\tiny [A$\rangle$GM]}  
	 guess:greed\\ \tt explanation:You're right that "speed" doesn't directly relate to the concept of being hungry, and I should focus on words that better capture the sensation or desire associated with hunger. Sticking with my previous guess, "greed" is a good fit because it refers to an intense longing or desire, which can be likened to the strong craving one feels when hungry. The word "greed" also matches the feedback, with "e" and 
	  } 
	   } 
	   } 
	 & & \\ 
 

    \theutterance \stepcounter{utterance}  

    & & & \multicolumn{2}{p{0.3\linewidth}}{\cellcolor[rgb]{0.95,0.95,0.95}{%
	\makecell[{{p{\linewidth}}}]{% 
	  \tt {\tiny [GM$|$GM]}  
	 {'guess': 'greed', 'explanation': 'You\'re right that "speed" doesn\'t directly relate to the concept of being hungry, and I should focus on words that better capture the sensation or desire associated with hunger. Sticking with my previous guess, "greed" is a good fit because it refers to an intense longing or desire, which can be likened to the strong craving one feels when hungry. The word "greed" also matches the feedback, with "e" and'} 
	  } 
	   } 
	   } 
	 & & \\ 
 

    \theutterance \stepcounter{utterance}  

    & & & \multicolumn{2}{p{0.3\linewidth}}{\cellcolor[rgb]{0.95,0.95,0.95}{%
	\makecell[{{p{\linewidth}}}]{% 
	  \tt {\tiny [GM$|$GM]}  
	 Change in player1's guess\\ \tt guess_before_critic_opinion: speed\\ \tt                                                 critic_agreement: no\\ \tt guess_after_critic_opinion: greed\\ \tt                                                 Proceeding with guess validation 
	  } 
	   } 
	   } 
	 & & \\ 
 

    \theutterance \stepcounter{utterance}  

    & & & \multicolumn{2}{p{0.3\linewidth}}{\cellcolor[rgb]{0.95,0.95,0.95}{%
	\makecell[{{p{\linewidth}}}]{% 
	  \tt {\tiny [GM$|$GM]}  
	 attempts: 4\\ \tt target_word = unfed\\ \tt target_word_clue = hungry\\ \tt guess: greed\\ \tt guess_feedback: g<red> r<red> e<red> e<green> d<green> 
	  } 
	   } 
	   } 
	 & & \\ 
 

    \theutterance \stepcounter{utterance}  

    & & \multicolumn{4}{p{0.6\linewidth}}{\cellcolor[rgb]{0.9,0.9,0.9}{%
	\makecell[{{p{\linewidth}}}]{% 
	  \tt {\tiny [A$\langle$GM]}  
	 guess_feedback: g<red> r<red> e<red> e<green> d<green> 
	  } 
	   } 
	   } 
	 & \\ 
 

    \theutterance \stepcounter{utterance}  

    & \multicolumn{4}{p{0.6\linewidth}}{\cellcolor[rgb]{0.8,1,0.9}{%
	\makecell[{{p{\linewidth}}}]{% 
	  \tt {\tiny [A$\rangle$GM]}  
	 guess:needy\\ \tt explanation:Based on the feedback, "e" and "d" are in the correct positions, while "g" and "r" are incorrect. "Needy" is a good guess because it relates to the state of being in need or wanting something strongly, which can be associated with the feeling of hunger. When someone is hungry, they may feel a strong need for food. The word "needy" fits the pattern of having "e" in the 
	  } 
	   } 
	   } 
	 & & \\ 
 

    \theutterance \stepcounter{utterance}  

    & & & \multicolumn{2}{p{0.3\linewidth}}{\cellcolor[rgb]{0.95,0.95,0.95}{%
	\makecell[{{p{\linewidth}}}]{% 
	  \tt {\tiny [GM$|$GM]}  
	 {'guess': 'needy', 'explanation': 'Based on the feedback, "e" and "d" are in the correct positions, while "g" and "r" are incorrect. "Needy" is a good guess because it relates to the state of being in need or wanting something strongly, which can be associated with the feeling of hunger. When someone is hungry, they may feel a strong need for food. The word "needy" fits the pattern of having "e" in the'} 
	  } 
	   } 
	   } 
	 & & \\ 
 

    \theutterance \stepcounter{utterance}  

    & & \multicolumn{4}{p{0.6\linewidth}}{\cellcolor[rgb]{0.9,0.9,0.9}{%
	\makecell[{{p{\linewidth}}}]{% 
	  \tt {\tiny [GM$\rangle$B]}  
	 clue:hungry\\ \tt guess:needy\\ \tt explanation:Based on the feedback, "e" and "d" are in the correct positions, while "g" and "r" are incorrect. "Needy" is a good guess because it relates to the state of being in need or wanting something strongly, which can be associated with the feeling of hunger. When someone is hungry, they may feel a strong need for food. The word "needy" fits the pattern of having "e" in the\\ \tt guess_feedback: g<red> r<red> e<red> e<green> d<green> 
	  } 
	   } 
	   } 
	 & \\ 
 

    \theutterance \stepcounter{utterance}  

    & & & \multicolumn{4}{p{0.6\linewidth}}{\cellcolor[rgb]{1,0.85,0.72}{%
	\makecell[{{p{\linewidth}}}]{% 
	  \tt {\tiny [GM$\langle$B]}  
	 agreement:yes\\ \tt explanation:the guess "needy" fits well with the clue "hungry" because when someone is hungry, they have a strong need for food. the word also matches the guess feedback, with "e" and "d" in the correct positions (green) and the other letters not appearing in the target word (red). "needy" captures both the meaning related to hunger and the letter positions indicated by the previous guesses. 
	  } 
	   } 
	   } 
	  \\ 
 

    \theutterance \stepcounter{utterance}  

    & & & \multicolumn{2}{p{0.3\linewidth}}{\cellcolor[rgb]{0.95,0.95,0.95}{%
	\makecell[{{p{\linewidth}}}]{% 
	  \tt {\tiny [GM$|$GM]}  
	 {'agreement': 'yes', 'explanation': 'the guess "needy" fits well with the clue "hungry" because when someone is hungry, they have a strong need for food. the word also matches the guess feedback, with "e" and "d" in the correct positions (green) and the other letters not appearing in the target word (red). "needy" captures both the meaning related to hunger and the letter positions indicated by the previous guesses.'} 
	  } 
	   } 
	   } 
	 & & \\ 
 

    \theutterance \stepcounter{utterance}  

    & & & \multicolumn{2}{p{0.3\linewidth}}{\cellcolor[rgb]{0.95,0.95,0.95}{%
	\makecell[{{p{\linewidth}}}]{% 
	  \tt {\tiny [GM$|$GM]}  
	 Critic agrees with the Guesser -- Sharing the critic's explanation with the guesser 
	  } 
	   } 
	   } 
	 & & \\ 
 

    \theutterance \stepcounter{utterance}  

    & & \multicolumn{4}{p{0.6\linewidth}}{\cellcolor[rgb]{0.9,0.9,0.9}{%
	\makecell[{{p{\linewidth}}}]{% 
	  \tt {\tiny [A$\langle$GM]}  
	 clue:hungry\\ \tt guess_agreement:yes\\ \tt agreement_explanation:the guess "needy" fits well with the clue "hungry" because when someone is hungry, they have a strong need for food. the word also matches the guess feedback, with "e" and "d" in the correct positions (green) and the other letters not appearing in the target word (red). "needy" captures both the meaning related to hunger and the letter positions indicated by the previous guesses.\\ \tt  
	  } 
	   } 
	   } 
	 & \\ 
 

    \theutterance \stepcounter{utterance}  

    & \multicolumn{4}{p{0.6\linewidth}}{\cellcolor[rgb]{0.8,1,0.9}{%
	\makecell[{{p{\linewidth}}}]{% 
	  \tt {\tiny [A$\rangle$GM]}  
	 guess:needy\\ \tt explanation:Thank you for the agreement and explanation. "Needy" is a fitting guess as it captures the strong desire and need for food that one experiences when hungry. It also correctly incorporates the feedback from the previous guesses, with "e" and "d" in the right positions (green) and avoiding the incorrect letters (red). The word "needy" effectively combines the meaning related to hunger with the letter positions indicated by the earlier feedback. I 
	  } 
	   } 
	   } 
	 & & \\ 
 

    \theutterance \stepcounter{utterance}  

    & & & \multicolumn{2}{p{0.3\linewidth}}{\cellcolor[rgb]{0.95,0.95,0.95}{%
	\makecell[{{p{\linewidth}}}]{% 
	  \tt {\tiny [GM$|$GM]}  
	 {'guess': 'needy', 'explanation': 'Thank you for the agreement and explanation. "Needy" is a fitting guess as it captures the strong desire and need for food that one experiences when hungry. It also correctly incorporates the feedback from the previous guesses, with "e" and "d" in the right positions (green) and avoiding the incorrect letters (red). The word "needy" effectively combines the meaning related to hunger with the letter positions indicated by the earlier feedback. I'} 
	  } 
	   } 
	   } 
	 & & \\ 
 

    \theutterance \stepcounter{utterance}  

    & & & \multicolumn{2}{p{0.3\linewidth}}{\cellcolor[rgb]{0.95,0.95,0.95}{%
	\makecell[{{p{\linewidth}}}]{% 
	  \tt {\tiny [GM$|$GM]}  
	 No change in player1's guess\\ \tt guess_before_critic_opinion: needy\\ \tt                                                 critic_agreement: yes\\ \tt guess_after_critic_opinion: needy\\ \tt                                                 Proceeding with guess validation 
	  } 
	   } 
	   } 
	 & & \\ 
 

    \theutterance \stepcounter{utterance}  

    & & & \multicolumn{2}{p{0.3\linewidth}}{\cellcolor[rgb]{0.95,0.95,0.95}{%
	\makecell[{{p{\linewidth}}}]{% 
	  \tt {\tiny [GM$|$GM]}  
	 attempts: 5\\ \tt target_word = unfed\\ \tt target_word_clue = hungry\\ \tt guess: needy\\ \tt guess_feedback: n<yellow> e<yellow> e<red> d<yellow> y<red> 
	  } 
	   } 
	   } 
	 & & \\ 
 

    \theutterance \stepcounter{utterance}  

    & & \multicolumn{4}{p{0.6\linewidth}}{\cellcolor[rgb]{0.9,0.9,0.9}{%
	\makecell[{{p{\linewidth}}}]{% 
	  \tt {\tiny [A$\langle$GM]}  
	 guess_feedback: n<yellow> e<yellow> e<red> d<yellow> y<red> 
	  } 
	   } 
	   } 
	 & \\ 
 

    \theutterance \stepcounter{utterance}  

    & \multicolumn{4}{p{0.6\linewidth}}{\cellcolor[rgb]{0.8,1,0.9}{%
	\makecell[{{p{\linewidth}}}]{% 
	  \tt {\tiny [A$\rangle$GM]}  
	 guess:dined\\ \tt explanation:Based on the latest feedback, "n," "e," and "d" are correct letters but in the wrong positions (yellow), while the second "e" and "y" are incorrect (red). "Dined" is a good guess because it relates to the act of eating a meal, which satisfies hunger. The word "dined" contains the correct letters "n," "e," and "d" in different positions than the previous 
	  } 
	   } 
	   } 
	 & & \\ 
 

    \theutterance \stepcounter{utterance}  

    & & & \multicolumn{2}{p{0.3\linewidth}}{\cellcolor[rgb]{0.95,0.95,0.95}{%
	\makecell[{{p{\linewidth}}}]{% 
	  \tt {\tiny [GM$|$GM]}  
	 {'guess': 'dined', 'explanation': 'Based on the latest feedback, "n," "e," and "d" are correct letters but in the wrong positions (yellow), while the second "e" and "y" are incorrect (red). "Dined" is a good guess because it relates to the act of eating a meal, which satisfies hunger. The word "dined" contains the correct letters "n," "e," and "d" in different positions than the previous'} 
	  } 
	   } 
	   } 
	 & & \\ 
 

    \theutterance \stepcounter{utterance}  

    & & \multicolumn{4}{p{0.6\linewidth}}{\cellcolor[rgb]{0.9,0.9,0.9}{%
	\makecell[{{p{\linewidth}}}]{% 
	  \tt {\tiny [GM$\rangle$B]}  
	 clue:hungry\\ \tt guess:dined\\ \tt explanation:Based on the latest feedback, "n," "e," and "d" are correct letters but in the wrong positions (yellow), while the second "e" and "y" are incorrect (red). "Dined" is a good guess because it relates to the act of eating a meal, which satisfies hunger. The word "dined" contains the correct letters "n," "e," and "d" in different positions than the previous\\ \tt guess_feedback: n<yellow> e<yellow> e<red> d<yellow> y<red> 
	  } 
	   } 
	   } 
	 & \\ 
 

    \theutterance \stepcounter{utterance}  

    & & & \multicolumn{4}{p{0.6\linewidth}}{\cellcolor[rgb]{1,0.85,0.72}{%
	\makecell[{{p{\linewidth}}}]{% 
	  \tt {\tiny [GM$\langle$B]}  
	 agreement:no\\ \tt explanation:while "dined" is related to eating and satisfying hunger, it does not match the guess feedback. the feedback indicates that "e" and "d" should be in the fourth and fifth positions, respectively, but "dined" has them in different positions. additionally, the letter "i" in "dined" has not been confirmed as a correct letter based on the previous feedback. therefore, "dined" is not a suitable guess given the 
	  } 
	   } 
	   } 
	  \\ 
 

    \theutterance \stepcounter{utterance}  

    & & & \multicolumn{2}{p{0.3\linewidth}}{\cellcolor[rgb]{0.95,0.95,0.95}{%
	\makecell[{{p{\linewidth}}}]{% 
	  \tt {\tiny [GM$|$GM]}  
	 {'agreement': 'no', 'explanation': 'while "dined" is related to eating and satisfying hunger, it does not match the guess feedback. the feedback indicates that "e" and "d" should be in the fourth and fifth positions, respectively, but "dined" has them in different positions. additionally, the letter "i" in "dined" has not been confirmed as a correct letter based on the previous feedback. therefore, "dined" is not a suitable guess given the'} 
	  } 
	   } 
	   } 
	 & & \\ 
 

    \theutterance \stepcounter{utterance}  

    & & & \multicolumn{2}{p{0.3\linewidth}}{\cellcolor[rgb]{0.95,0.95,0.95}{%
	\makecell[{{p{\linewidth}}}]{% 
	  \tt {\tiny [GM$|$GM]}  
	 Critic disagrees with the Guesser -- Sharing the critic's explanation with the guesser 
	  } 
	   } 
	   } 
	 & & \\ 
 

    \theutterance \stepcounter{utterance}  

    & & \multicolumn{4}{p{0.6\linewidth}}{\cellcolor[rgb]{0.9,0.9,0.9}{%
	\makecell[{{p{\linewidth}}}]{% 
	  \tt {\tiny [A$\langle$GM]}  
	 clue:hungry\\ \tt guess_agreement:no\\ \tt agreement_explanation:while "dined" is related to eating and satisfying hunger, it does not match the guess feedback. the feedback indicates that "e" and "d" should be in the fourth and fifth positions, respectively, but "dined" has them in different positions. additionally, the letter "i" in "dined" has not been confirmed as a correct letter based on the previous feedback. therefore, "dined" is not a suitable guess given the\\ \tt  
	  } 
	   } 
	   } 
	 & \\ 
 

    \theutterance \stepcounter{utterance}  

    & \multicolumn{4}{p{0.6\linewidth}}{\cellcolor[rgb]{0.8,1,0.9}{%
	\makecell[{{p{\linewidth}}}]{% 
	  \tt {\tiny [A$\rangle$GM]}  
	 guess:mend\\ \tt explanation:You're right that "dined" doesn't match the feedback, as "e" and "d" should be in the fourth and fifth positions, respectively. "Mend" is a better guess because it contains the correct letters "n," "e," and "d" in different positions than the previous guess, with "e" and "d" in the fourth and fifth positions as indicated by the feedback. Although "mend" is not 
	  } 
	   } 
	   } 
	 & & \\ 
 

    \theutterance \stepcounter{utterance}  

    & & & \multicolumn{2}{p{0.3\linewidth}}{\cellcolor[rgb]{0.95,0.95,0.95}{%
	\makecell[{{p{\linewidth}}}]{% 
	  \tt {\tiny [GM$|$GM]}  
	 {'guess': 'mend', 'explanation': 'You\'re right that "dined" doesn\'t match the feedback, as "e" and "d" should be in the fourth and fifth positions, respectively. "Mend" is a better guess because it contains the correct letters "n," "e," and "d" in different positions than the previous guess, with "e" and "d" in the fourth and fifth positions as indicated by the feedback. Although "mend" is not'} 
	  } 
	   } 
	   } 
	 & & \\ 
 

    \theutterance \stepcounter{utterance}  

    & & & \multicolumn{2}{p{0.3\linewidth}}{\cellcolor[rgb]{0.95,0.95,0.95}{%
	\makecell[{{p{\linewidth}}}]{% 
	  \tt {\tiny [GM$|$GM]}  
	 Guesser Error: INVALID_WORD_LENGTH while parsing Player 1's (model: claude-3-opus-20240229) response, retrying 
	  } 
	   } 
	   } 
	 & & \\ 
 

    \theutterance \stepcounter{utterance}  

    & & \multicolumn{4}{p{0.6\linewidth}}{\cellcolor[rgb]{0.9,0.9,0.9}{%
	\makecell[{{p{\linewidth}}}]{% 
	  \tt {\tiny [A$\langle$GM]}  
	 the word should have exactly 5 letters. Please try again 
	  } 
	   } 
	   } 
	 & \\ 
 

    \theutterance \stepcounter{utterance}  

    & \multicolumn{4}{p{0.6\linewidth}}{\cellcolor[rgb]{0.8,1,0.9}{%
	\makecell[{{p{\linewidth}}}]{% 
	  \tt {\tiny [A$\rangle$GM]}  
	 guess:ended\\ \tt explanation:Thank you for pointing out that the word should have exactly five letters. "Ended" is a better guess because it contains the correct letters "n," "e," and "d" in different positions than the previous guesses, with "e" and "d" in the fourth and fifth positions as indicated by the feedback. Although "ended" is not directly related to hunger, it incorporates the information provided by the previous guesses and feedback. The word 
	  } 
	   } 
	   } 
	 & & \\ 
 

    \theutterance \stepcounter{utterance}  

    & & & \multicolumn{2}{p{0.3\linewidth}}{\cellcolor[rgb]{0.95,0.95,0.95}{%
	\makecell[{{p{\linewidth}}}]{% 
	  \tt {\tiny [GM$|$GM]}  
	 {'guess': 'ended', 'explanation': 'Thank you for pointing out that the word should have exactly five letters. "Ended" is a better guess because it contains the correct letters "n," "e," and "d" in different positions than the previous guesses, with "e" and "d" in the fourth and fifth positions as indicated by the feedback. Although "ended" is not directly related to hunger, it incorporates the information provided by the previous guesses and feedback. The word'} 
	  } 
	   } 
	   } 
	 & & \\ 
 

    \theutterance \stepcounter{utterance}  

    & & & \multicolumn{2}{p{0.3\linewidth}}{\cellcolor[rgb]{0.95,0.95,0.95}{%
	\makecell[{{p{\linewidth}}}]{% 
	  \tt {\tiny [GM$|$GM]}  
	 Change in player1's guess\\ \tt guess_before_critic_opinion: dined\\ \tt                                                 critic_agreement: no\\ \tt guess_after_critic_opinion: ended\\ \tt                                                 Proceeding with guess validation 
	  } 
	   } 
	   } 
	 & & \\ 
 

    \theutterance \stepcounter{utterance}  

    & & & \multicolumn{2}{p{0.3\linewidth}}{\cellcolor[rgb]{0.95,0.95,0.95}{%
	\makecell[{{p{\linewidth}}}]{% 
	  \tt {\tiny [GM$|$GM]}  
	 attempts: 6\\ \tt target_word = unfed\\ \tt target_word_clue = hungry\\ \tt guess: ended\\ \tt guess_feedback: e<red> n<green> d<red> e<green> d<green> 
	  } 
	   } 
	   } 
	 & & \\ 
 

    \theutterance \stepcounter{utterance}  

    & & & \multicolumn{2}{p{0.3\linewidth}}{\cellcolor[rgb]{0.95,0.95,0.95}{%
	\makecell[{{p{\linewidth}}}]{% 
	  \tt {\tiny [GM$|$GM]}  
	 game_result = LOSS 
	  } 
	   } 
	   } 
	 & & \\ 
 

\end{supertabular}
}

\end{document}


\documentclass{article}
\usepackage{colortbl}
\usepackage{makecell}
\usepackage{multirow}
\usepackage{supertabular}

\begin{document}

\newcounter{utterance}

\twocolumn

{ \footnotesize  \setcounter{utterance}{1}
\setlength{\tabcolsep}{0pt}
\begin{supertabular}{c@{$\;$}|p{.15\linewidth}@{}p{.15\linewidth}p{.15\linewidth}p{.15\linewidth}p{.15\linewidth}p{.15\linewidth}}

    \# & $\;$A & \multicolumn{4}{c}{Game Master} & $\;\:$B\\
    \hline 

    \theutterance \stepcounter{utterance}  

    & & \multicolumn{4}{p{0.6\linewidth}}{\cellcolor[rgb]{0.9,0.9,0.9}{%
	\makecell[{{p{\linewidth}}}]{% 
	  \tt {\tiny [A$\langle$GM]}  
	 You are a language wizard who likes to guess words by using the given rules.\\ \tt \\ \tt Welcome to Wordle! You have six attempts to guess the target word, a valid English word of five lowercase letters (a-z). Please use the tags "guess:" and "explanation:" to provide a concise explanation for each guess.\\ \tt \\ \tt  To help you make an informed guess, you will receive a clue for the word, such as\\ \tt clue: snowy white.\\ \tt \\ \tt Here is an example guess based on the clue:\\ \tt guess: apple\\ \tt explanation: In the fairy tail Snow White, the girl is killed because she eats a poisoned apple. And the word apple has 5 letters.\\ \tt \\ \tt After each guess, your answer will be validated, and you will receive feedback indicating which letters are correct (green), which letters are correct but in the wrong position (yellow), and which letters are incorrect (red). This feedback can be useful in determining which letters to include or exclude in your next guess.\\ \tt \\ \tt For example, the feedback for "apple" might be:\\ \tt guess_feedback: a<yellow> p<yellow> p<green> l<yellow> e<red>\\ \tt \\ \tt The explanation should contain details about how both the guess_feedback and clue work together in tandem to formulate a new guess.\\ \tt \\ \tt Let's begin with your first guess.\\ \tt clue:treacly\\ \tt  
	  } 
	   } 
	   } 
	 & \\ 
 

    \theutterance \stepcounter{utterance}  

    & \multicolumn{4}{p{0.6\linewidth}}{\cellcolor[rgb]{0.8,1,0.9}{%
	\makecell[{{p{\linewidth}}}]{% 
	  \tt {\tiny [A$\rangle$GM]}  
	 guess: syrup\\ \tt explanation: The clue "treacly" suggests a word related to sweetness, and "syrup" is a sweet liquid that can be used as a topping or in cooking, similar to treacle. Both words have five letters, making "syrup" a suitable guess. 
	  } 
	   } 
	   } 
	 & & \\ 
 

    \theutterance \stepcounter{utterance}  

    & & & \multicolumn{2}{p{0.3\linewidth}}{\cellcolor[rgb]{0.95,0.95,0.95}{%
	\makecell[{{p{\linewidth}}}]{% 
	  \tt {\tiny [GM$|$GM]}  
	 {'guess:': 'syrup', 'explanation:': 'The clue "treacly" suggests a word related to sweetness, and "syrup" is a sweet liquid that can be used as a topping or in cooking, similar to treacle. Both words have five letters, making "syrup" a suitable guess.'} 
	  } 
	   } 
	   } 
	 & & \\ 
 

    \theutterance \stepcounter{utterance}  

    & & & \multicolumn{2}{p{0.3\linewidth}}{\cellcolor[rgb]{0.95,0.95,0.95}{%
	\makecell[{{p{\linewidth}}}]{% 
	  \tt {\tiny [GM$|$GM]}  
	 attempts: 1\\ \tt target_word = sweet\\ \tt target_word_clue = treacly\\ \tt guess: syrup\\ \tt guess_feedback: s<green> y<red> r<red> u<red> p<red> 
	  } 
	   } 
	   } 
	 & & \\ 
 

    \theutterance \stepcounter{utterance}  

    & & \multicolumn{4}{p{0.6\linewidth}}{\cellcolor[rgb]{0.9,0.9,0.9}{%
	\makecell[{{p{\linewidth}}}]{% 
	  \tt {\tiny [A$\langle$GM]}  
	 guess_feedback: s<green> y<red> r<red> u<red> p<red> 
	  } 
	   } 
	   } 
	 & \\ 
 

    \theutterance \stepcounter{utterance}  

    & \multicolumn{4}{p{0.6\linewidth}}{\cellcolor[rgb]{0.8,1,0.9}{%
	\makecell[{{p{\linewidth}}}]{% 
	  \tt {\tiny [A$\rangle$GM]}  
	 Based on the feedback, the letter "s" is correct and in the right position, while the letters "y", "r", "u", and "p" are incorrect and in the wrong position. Using this information, we can eliminate "syrup" as a possibility and consider words that start with "s" and have the correct letter in the second position. 
	  } 
	   } 
	   } 
	 & & \\ 
 

    \theutterance \stepcounter{utterance}  

    & & & \multicolumn{2}{p{0.3\linewidth}}{\cellcolor[rgb]{0.95,0.95,0.95}{%
	\makecell[{{p{\linewidth}}}]{% 
	  \tt {\tiny [GM$|$GM]}  
	 {'guess:': 'INVALID_FORMAT', 'explanation:': 'INVALID_FORMAT'} 
	  } 
	   } 
	   } 
	 & & \\ 
 

    \theutterance \stepcounter{utterance}  

    & & & \multicolumn{2}{p{0.3\linewidth}}{\cellcolor[rgb]{0.95,0.95,0.95}{%
	\makecell[{{p{\linewidth}}}]{% 
	  \tt {\tiny [GM$|$GM]}  
	 Guesser Error: INVALID_FORMAT while parsing Player 1's (model: openchat-3.5-0106) response, retrying 
	  } 
	   } 
	   } 
	 & & \\ 
 

    \theutterance \stepcounter{utterance}  

    & & \multicolumn{4}{p{0.6\linewidth}}{\cellcolor[rgb]{0.9,0.9,0.9}{%
	\makecell[{{p{\linewidth}}}]{% 
	  \tt {\tiny [A$\langle$GM]}  
	 Provide your response only in this format:\\ \tt guess:word\\ \tt explanation:details.\\ \tt Please try again 
	  } 
	   } 
	   } 
	 & \\ 
 

    \theutterance \stepcounter{utterance}  

    & \multicolumn{4}{p{0.6\linewidth}}{\cellcolor[rgb]{0.8,1,0.9}{%
	\makecell[{{p{\linewidth}}}]{% 
	  \tt {\tiny [A$\rangle$GM]}  
	 guess:sugar\\ \tt explanation: The clue "treacly" suggests a word related to sweetness, and "sugar" is a common sweet substance that can be used in various foods and beverages. Both words have five letters, making "sugar" a suitable guess. 
	  } 
	   } 
	   } 
	 & & \\ 
 

    \theutterance \stepcounter{utterance}  

    & & & \multicolumn{2}{p{0.3\linewidth}}{\cellcolor[rgb]{0.95,0.95,0.95}{%
	\makecell[{{p{\linewidth}}}]{% 
	  \tt {\tiny [GM$|$GM]}  
	 {'guess:': 'sugar', 'explanation:': 'The clue "treacly" suggests a word related to sweetness, and "sugar" is a common sweet substance that can be used in various foods and beverages. Both words have five letters, making "sugar" a suitable guess.'} 
	  } 
	   } 
	   } 
	 & & \\ 
 

    \theutterance \stepcounter{utterance}  

    & & & \multicolumn{2}{p{0.3\linewidth}}{\cellcolor[rgb]{0.95,0.95,0.95}{%
	\makecell[{{p{\linewidth}}}]{% 
	  \tt {\tiny [GM$|$GM]}  
	 attempts: 2\\ \tt target_word = sweet\\ \tt target_word_clue = treacly\\ \tt guess: sugar\\ \tt guess_feedback: s<green> u<red> g<red> a<red> r<red> 
	  } 
	   } 
	   } 
	 & & \\ 
 

    \theutterance \stepcounter{utterance}  

    & & \multicolumn{4}{p{0.6\linewidth}}{\cellcolor[rgb]{0.9,0.9,0.9}{%
	\makecell[{{p{\linewidth}}}]{% 
	  \tt {\tiny [A$\langle$GM]}  
	 guess_feedback: s<green> u<red> g<red> a<red> r<red> 
	  } 
	   } 
	   } 
	 & \\ 
 

    \theutterance \stepcounter{utterance}  

    & \multicolumn{4}{p{0.6\linewidth}}{\cellcolor[rgb]{0.8,1,0.9}{%
	\makecell[{{p{\linewidth}}}]{% 
	  \tt {\tiny [A$\rangle$GM]}  
	 guess:sweet\\ \tt explanation:The clue "treacly" suggests a word related to sweetness, and "sweet" is an adjective used to describe something with a pleasant taste, often associated with sugar or other sweet substances. Both words have five letters, making "sweet" a suitable guess. 
	  } 
	   } 
	   } 
	 & & \\ 
 

    \theutterance \stepcounter{utterance}  

    & & & \multicolumn{2}{p{0.3\linewidth}}{\cellcolor[rgb]{0.95,0.95,0.95}{%
	\makecell[{{p{\linewidth}}}]{% 
	  \tt {\tiny [GM$|$GM]}  
	 {'guess:': 'sweet', 'explanation:': 'The clue "treacly" suggests a word related to sweetness, and "sweet" is an adjective used to describe something with a pleasant taste, often associated with sugar or other sweet substances. Both words have five letters, making "sweet" a suitable guess.'} 
	  } 
	   } 
	   } 
	 & & \\ 
 

    \theutterance \stepcounter{utterance}  

    & & & \multicolumn{2}{p{0.3\linewidth}}{\cellcolor[rgb]{0.95,0.95,0.95}{%
	\makecell[{{p{\linewidth}}}]{% 
	  \tt {\tiny [GM$|$GM]}  
	 attempts: 3\\ \tt target_word = sweet\\ \tt target_word_clue = treacly\\ \tt guess: sweet\\ \tt guess_feedback: s<green> w<green> e<green> e<green> t<green> 
	  } 
	   } 
	   } 
	 & & \\ 
 

    \theutterance \stepcounter{utterance}  

    & & & \multicolumn{2}{p{0.3\linewidth}}{\cellcolor[rgb]{0.95,0.95,0.95}{%
	\makecell[{{p{\linewidth}}}]{% 
	  \tt {\tiny [GM$|$GM]}  
	 game_result = WIN 
	  } 
	   } 
	   } 
	 & & \\ 
 

\end{supertabular}
}

\end{document}
